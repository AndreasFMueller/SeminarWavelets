%
% macros.tex -- some common macro definitions
%
% (c) 2019 Prof Dr Andreas Müller, Hochschule Rapperswil
%
\hypersetup{
    linktoc=all,
    linkcolor=blue
}
\newcounter{beispiel}
\newenvironment{beispiele}{
\bgroup\smallskip\parindent0pt\bf Beispiele\egroup

\begin{list}{\arabic{beispiel}.}
  {\usecounter{beispiel}
  \setlength{\labelsep}{5mm}
  \setlength{\rightmargin}{0pt}
}}{\end{list}}
\newcounter{uebungsaufgabezaehler}
% environment fuer uebungsaufgaben
\newenvironment{uebungsaufgaben}{
\begin{list}{\arabic{uebungsaufgabezaehler}.}
  {\usecounter{uebungsaufgabezaehler}
  \setlength{\labelwidth}{2cm}
  \setlength{\leftmargin}{0pt}
  \setlength{\labelsep}{5mm}
  \setlength{\rightmargin}{0pt}
  \setlength{\itemindent}{0pt}
}}{\end{list}\vfill\pagebreak}
\newenvironment{teilaufgaben}{
\begin{enumerate}
\renewcommand{\labelenumi}{\alph{enumi})}
}{\end{enumerate}}
% Aufgabe
\newcounter{problemcounter}[chapter]
\def\aufgabenpath{chapters/uebungsaufgaben/}
\def\ainput#1{\input\aufgabenpath/#1}
\def\verbatimainput#1{\expandafter\verbatiminput{\aufgabenpath/#1}}
\def\aufgabetoplevel#1{%
\expandafter\def\expandafter\inputpath{#1}%
\let\aufgabepath=\inputpath
}
\def\includeagraphics[#1]#2{\expandafter\includegraphics[#1]{\aufgabepath#2}}
% \aufgabe
\newcommand{\uebungsaufgabe}[1]{%
\refstepcounter{problemcounter}%
\label{#1}%
\bigskip{\parindent0pt\strut}\hbox{\bf\arabic{problemcounter}. }%
\expandafter\def\csname aufgabenpath\endcsname{\inputpath/}%
\expandafter\input{chapters/uebungsaufgaben/#1.tex}
}
\renewcommand\theproblemcounter{\thechapter.\arabic{problemcounter}}
% linsys
\newcolumntype{\linsysR}{>{$}r<{$}}
\newcolumntype{\linsysL}{>{$}l<{$}}
\newcolumntype{\linsysC}{>{$}c<{$}}
\newenvironment{linsys}[1]{%
\begin{tabular}{*{#1}{\linsysR@{\;}\linsysC}@{\;}\linsysR}}%
{\end{tabular}}

% Loesung
\def\swallow#1{
%nothing
}
\NewEnviron{loesung}[1][Lösung]{%
\begin{proof}[#1]%
\renewcommand{\qedsymbol}{$\bigcirc$}
\BODY
\end{proof}
}
\NewEnviron{bewertung}{%
\begin{proof}[Bewertung]%
\renewcommand{\qedsymbol}{}
\BODY
\end{proof}
}
\NewEnviron{diskussion}{%
\begin{proof}[Diskussion]%
\renewcommand{\qedsymbol}{}
\BODY
\end{proof}
}
\NewEnviron{hinweis}{%
\begin{proof}[Hinweis]%
\renewcommand{\qedsymbol}{}
\BODY
\end{proof}
}
\def\keineloesungen{%
\RenewEnviron{loesung}{\relax}
\RenewEnviron{bewertung}{\relax}
\RenewEnviron{diskussion}{\relax}
}
\newenvironment{beispiel}{%
\begin{proof}[Beispiel]%
\renewcommand{\qedsymbol}{$\bigcirc$}
}{\end{proof}}

\allowdisplaybreaks

\lhead{Inhaltsverzeichnis}
\rhead{}
\tableofcontents
\newtheorem{satz}{Satz}[chapter]
\newtheorem{hilfssatz}[satz]{Hilfssatz}
\newtheorem{korollar}[satz]{Korollar}
\newtheorem{lemma}[satz]{Lemma}
\newtheorem{definition}[satz]{Definition}
\newtheorem{annahme}[satz]{Annahme}
\newtheorem{problem}[satz]{Problem}
\newtheorem{aufgabe}[satz]{Aufgabe}
\newtheorem*{problem*}{Problem}
\newtheorem{forderung}{Forderung}[chapter]
\newtheorem{konsequenz}[satz]{Konsequenz}
\newtheorem{algorithmus}[satz]{Algorithmus}
\renewcommand{\floatpagefraction}{0.7}

\def\sinc{\operatorname{sinc}}
\def\si{\operatorname{si}}
\definecolor{darkgreen}{rgb}{0,0.6,0}
\definecolor{orange}{rgb}{1,0.6,0}

% from https://tex.stackexchange.com/questions/156862/displaying-author-for-each-chapter-in-book

%\newcommand\chapterauthor[1]{\authortoc{#1}\printchapterauthor{#1}}
%\WithSuffix\newcommand\chapterauthor*[1]{\printchapterauthor{#1}}
%
%\makeatletter
%\newcommand{\printchapterauthor}[1]{%
%  {\parindent0pt\vspace*{-25pt}%
%  \linespread{1.1}\large\scshape#1%
%  \par\nobreak\vspace*{35pt}}
%  \@afterheading%
%}
%\newcommand{\authortoc}[1]{%
%  \addtocontents{toc}{\vskip-10pt}%
%  \addtocontents{toc}{%
%    \protect\contentsline{chapter}%
%    {\hskip1.3em\mdseries\scshape\protect\scriptsize#1}{}{}}
%  \addtocontents{toc}{\vskip5pt}%
%}
%\makeatother
