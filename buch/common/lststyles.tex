%
% lststyles.tex -- styles for the listings package
%
% (c) 2019 Prof Dr Andreas Müller, Hochschule Rapperswil
%

%
% lststyle for Matlab
%
\usepackage{listings}
\usepackage{color} %red, green, blue, yellow, cyan, magenta, black, white
\definecolor{mygreen}{RGB}{28,172,0} % color values Red, Green, Blue
\definecolor{mylilas}{RGB}{170,55,241}

\lstset{language=Matlab,%
    %basicstyle=\color{red},
    breaklines=true,%
    morekeywords={matlab2tikz},
    keywordstyle=\color{blue},%
    morekeywords=[2]{1}, keywordstyle=[2]{\color{black}},
    identifierstyle=\color{black},%
    stringstyle=\color{mylilas},
    commentstyle=\color{mygreen},%
    showstringspaces=false,%without this there will be a symbol in the places where there is a space
    numbers=left,%
    %numberstyle={\tiny \color{black}},% size of the numbers
    numbersep=9pt, % this defines how far the numbers are from the text
    emph=[1]{break},emphstyle=[1]\color{red}, %some words to emphasise
    %emph=[2]{word1,word2}, emphstyle=[2]{style},    
}
\lstdefinestyle{Matlab}{
  numbers=left,
  belowcaptionskip=1\baselineskip,
  breaklines=true,
  frame=L,
  xleftmargin=\parindent,
  language=Matlab,
  showstringspaces=false,
  basicstyle=\footnotesize\ttfamily,
  keywordstyle=\bfseries\color{green!40!black},
  commentstyle=\itshape\color{purple!40!black},
  identifierstyle=\color{blue},
  stringstyle=\color{orange},
  numberstyle=\ttfamily\tiny,
  morestring=*[d]{"},
  numbersep=9pt
}
\lstdefinelanguage{Matlab}{
  keywords={function,global,zeros,switch,case,otherwise,end,sin,cos,cot,floor,ode45,hold,polarplot,},
  sensitive=true
}

%
% lst style for Maxima
%
\lstdefinelanguage{Maxima}{
  keywords={addrow,addcol,zeromatrix,ident,augcoefmatrix,ratsubst,sum,diff,ev,tex,%
    with_stdout,nouns,express,depends,load,length,submatrix,div,grad,curl,matrix,%
    invert,lambda,facsum,expand,false,then,if,else,subst,batchload,%
    rootscontract,solve,part,assume,sqrt,integrate,abs,inf,exp,sin,cos,sinh,cosh,taylor,ratsimp},
  sensitive=true,
  comment=[n][\itshape]{/*}{*/}
}
\lstdefinestyle{Maxima}{
  numbers=left,
  belowcaptionskip=1\baselineskip,
  breaklines=true,
  frame=L,
  xleftmargin=\parindent,
  language=Maxima,
  showstringspaces=false,
  basicstyle=\footnotesize\ttfamily,
  keywordstyle=\bfseries\color{green!40!black},
  commentstyle=\itshape\color{purple!40!black},
  identifierstyle=\color{blue},
  stringstyle=\color{orange},
  numberstyle=\ttfamily\tiny
}

%
% lst style for Octave
%
\lstset{language=Octave,%
    %basicstyle=\color{red},
    breaklines=true,%
    morekeywords={function,global,size,zeros,switch,case,otherwise,end,sin,cos,cot,floor,ode45,hold,polarplot,endfunction,size,endswitch,cat,printf,for,endfor,if,return,endif,abs,while,endwhile},
    keywordstyle=\color{blue},%
    morekeywords=[2]{1}, keywordstyle=[2]{\color{black}},
    identifierstyle=\color{black},%
    stringstyle=\color{mylilas},
    commentstyle=\color{mygreen},%
    showstringspaces=false,%without this there will be a symbol in the places where there is a space
    numbers=left,%
    %numberstyle={\tiny \color{black}},% size of the numbers
    numbersep=9pt, % this defines how far the numbers are from the text
    emph=[1]{break},emphstyle=[1]\color{red}, %some words to emphasise
    %emph=[2]{word1,word2}, emphstyle=[2]{style},    
}
\lstdefinestyle{Octave}{
  numbers=left,
  belowcaptionskip=1\baselineskip,
  breaklines=true,
  frame=L,
  xleftmargin=\parindent,
  language=Octave,
  showstringspaces=false,
  basicstyle=\footnotesize\ttfamily,
  keywordstyle=\bfseries\color{green!40!black},
  commentstyle=\itshape\color{purple!40!black},
  identifierstyle=\color{blue},
  stringstyle=\color{orange},
  numberstyle=\ttfamily\tiny
}

%
% lst style for C
%
\lstdefinestyle{C}{
  numbers=left,
  belowcaptionskip=1\baselineskip,
  breaklines=true,
  frame=L,
  xleftmargin=\parindent,
  language=C,
  showstringspaces=false,
  basicstyle=\footnotesize\ttfamily,
  keywordstyle=\bfseries\color{green!40!black},
  commentstyle=\itshape\color{purple!40!black},
  identifierstyle=\color{blue},
  stringstyle=\color{orange},
  numberstyle=\ttfamily\tiny
}

%
% lst style for Python
%
\lstdefinestyle{Python}{
  numbers=left,
  belowcaptionskip=1\baselineskip,
  breaklines=true,
  frame=l,
  framerule=0pt,
  framesep=-1pt,
  xleftmargin=1em,
  language=Python,
  showstringspaces=false,
  basicstyle=\footnotesize\ttfamily,
  keywordstyle=\bfseries\color{green!40!black},
  commentstyle=\itshape\color{purple!40!black},
  identifierstyle=\color{blue},
  stringstyle=\color{orange},
  numberstyle=\ttfamily\tiny
}
