%
% dual.tex -- duale unterteilung
%
% (c) 2019 Prof Dr Andreas Müller, Hochschule Rapperswil
%
\section{Dualzahlen-Unterteilung%
\label{section:dualzahlen}}
\rhead{Dualzahlen}
Wir legen die Untertteilung $x_k = k$ mit $k\in\mathbb Z$ zu Grunde.
Wir schreiben dafür
\[
U_0
=
\{ k\,|\, k\in\mathbb Z\}
=
\mathbb Z.
\]
Abtastung erolgt also nur an ganzzahligen $x$-Werten und
Translationen um ganze Zahlen sind uneingeschränkt möglich.

Um mehr Detail zu erhalten, kann das Abtastinterval halbiert werden.
Die Unterteilung bekommt dann die Form
\[
U_1
=
\{ k2^{-1}\,|\, k\in\mathbb Z\}
=
2^{-1}\mathbb Z.
\]
Offenbar können damit doppelt so hohe Frequenzen abgetastet werden.
Beim Übergang von $U_0$ zu $U_1$ können die bereits bekannten Samples
an den Stellen $U_0$ weiterverwendet werden.
In $U_1$ sind Translationen um ganzzahlige Vielfache von $\frac12$ 
möglich.

Der Verfeinerungsprozess kann weitergeführt werden, so dass beliebig
feine Unterteilung
\[
U_j = \{ k2^{-j}\,|\,k\in\mathbb Z\}
\]
entstehen.
In jedem Schritt wird die Menge an Informationen der gesampelten Funktion
verdoppelt, ohne dass früher gefundene Samples nutzlos werden.
In $U_j$ sind Verschiebungen um ganzzahlige Vielfache von $2^{-j}$ 
möglich.
Es wird daher in $U_j$ nicht nur möglich, Signale für genügend grosses $j$
beliebig genau zu approximieren, sondern auch beliebig fein aufgelöste
Translationen abzubilden.
Wir bezeichnen den Vektorraum der stückweise konstanten Funktionen
bezüglich der Unterteilung $U_j$ mit $V_j$.
Es gilt also $V_j \subset V_{j+1}$.

Man kann auch versuchen, eine Basis zu konstruieren.
Im Vektorraum $V_j$ können die Funktionen mit Samplewerten
\[
e^{(l)}_k = \frac{1}{2^{-j/2}}\delta_{kl}
\]
als Basisvektoren verwendet werden.
Wegen
\[
\langle e^{(l)},e^{(r)}\rangle
=
2^{-j}
\biggl(
\frac{1}{2^{-j/2}}
\biggr)^2\delta_{lr}
=
\delta_{lr},
\]
Jeder einzelne der Vektorräume $V_j$ hat also eine einfach anzgebende
Basis.

Die Konstruktion ist aber trotzdem nicht befriedigend.
Zum einen sind die Funktionen $e^{(k)}$ der verschiedenen Vektorräume
nicht orthogonal.
Andererseits lösen sie das bereits angesprochen Problem der Redundanz
der Information nicht.
Das Skalarprodukt eines Signals mit $e^{(l)}$ ist im Wesentlichen der
Abstastwert in der Nähe von $l2^{-j}$.
Für langsam veränderliche Signale sind nahe beeinander liegende 
Abtastwerte nahe beeinander.
Mit den Vektorräumen $V_j$ ist es also möglich, ein Signal beliebig
genau abzutasten, aber die naheliegende Basis führt nicht zu einer
effizienten Charakterisierung des Signals.



