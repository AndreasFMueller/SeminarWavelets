%
% chapter.tex -- chapter on haar wavelet
%
% (c) 2019 Prof Dr Andreas Müller, Hochschule Rapperswil
%
\chapter{Das Haar-Wavelet
\label{chapter:haar-wavelet}}
Alfred Haar hat schon 1910 die Analyse stetiger Funktionen mit Hilfe
von Funktionen demonstriert, die wir heute als Wavelets bezeichnen
würden.
Dieses Kapitel ist einer ausführlichen Darstellung des Haar-Wavelets
und all seiner Eigenschaften im Lichte der in den ersten beiden Kapiteln
entwickelten allgemeinen Theorie gewidmet.
Damit soll einerseits die Tragfähigkeit der abstrakten Theorie
demonstriert werden, andererseits soll auch der Blick für die
wesentlichen Eigenschaften geschärft werden, nach denen wir bei
anspruchsvolleren Wavelet-Entwicklungen ausschau halten sollten.


