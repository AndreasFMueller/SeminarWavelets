%
% stueckweise.tex
%
% (c) 2019 Prof Dr Andreas Müller, Hochschule Rapperswil
%
\section{Stückweise konstante Funktionen%
\label{section:stueckweise}}
\rhead{Stückweise konstante Funktionen}
Eine Funktion $f\colon[a,b]\to\mathbb R$ heisst {\em stückweise konstant},
wenn es eine Unterteilung des Intervals $[a,b]$ durch Punkte
$x_k\in\mathbb R$ mit $0\le k\le n$ gibt, also
\[
a=x_0 < x_1 < \dots < x_k < \dots < x_{n-1} < x_n = b,
\]
so dass die Funktion $f$ in jedem Teilinterval $[x_k,x_{k+1})$
konstant ist.
Es gibt also eine Menge von Zahlen $f_k$ derart, dass
\begin{equation}
f(x)
=
f_k
\qquad\text{falls}\qquad
x_k \le x < x_{k+1}.
\label{buch:stückweisef}
\end{equation}
Stückweise konstante Funktionen sind natürlich nicht überall stetig sein,
vielmehr ist jeder der Punkte $x_k$ ein potentielle Unstetigkeitsstelle.
Es gilt nämlich
\begin{align*}
\lim_{x\to x_k-} f(x)
&=
f_{k-1}&\forall k\text{ mit }&1\le k \le n
\\
\lim_{x\to x_k+} f(x)
&=
f_k&\forall k\text{ mit }&0\le k \le n-1
\end{align*}
Wenn also $f_k\ne f_{k-1}$, dann ist $x_k$ eine Unstetigkeitsstelle.
$f(x)$ ist immer noch rechtsseitig stetig, aber der linksseitige
Grenzwert ist verschieden, die Funktion hat einen Sprung an der
Stelle $x_k$.

Mit Hilfe der charakteristischen Funktion eines Intervals lässt sich
eine stückweise konstante Funktion etwas eleganter schreiben.
Wir definieren  die {\em charakteristische Funktion} oder
{\em Indikatorfunktion} des Intervals $I$ als
\[
\chi_{I}(x) = \begin{cases}
1&\qquad x\in I\\
0&\qquad x\not\in I.
\end{cases}
\]
Damit kann man die stückweise konsante Funktion $f(x)$ aus
\eqref{buch:stückweisef}
als Linearkombination
\[
f(x)
=
\sum_{k=0}^{n-1} f_k\chi_{[x_k,x_{k+1})}(x)
\]
von Indikatorfunktionen schreiben.

Stückweise konstante Funktionen sind zwar nicht stetig, aber sie können
stetige Funktionen beliebig genau approximieren.
Dazu ist aber auch nötig, die Punkte $x_k$ genügen nahe beeinander zu
haben.
Wir nennen die grösste Länge eines Teilintervals einer Unterteilung
\[
\delta(U)
=
\delta(\{x_0,\dots,x_n\})
=
\max_{0\le k < n} (x_{k+1}-x_{k})
\]
das {\em Korn} der Unterteilung.
Es wird also behauptet, dass eine stetige Funktion durch eine stückweise
konstante Funktion beliebig genau approximiert werden kann, wenn das
Korn der Unterteilung klein genug gemacht wird.
Genauer, sei $\varepsilon>0$ geben.
Da die Funktion $f(x)$ stetig ist, gibt es eine Zahl $\delta>0$ derart,
dass
\[
|f(x) - f(y)| < \varepsilon\qquad\forall x,y\in [a,b]\quad\text{mit}\quad
|x-y|<\delta.
\]
Wählt man nun eine Unterteilung des Intervals mit Korn
$\delta(\{x_0,\dots,x_n\}) < \delta$ und setzt
\[
g_k = f(x_k),\quad 0\le k < n.
\]
Die Zahlen $g_k$ definieren eine stückweise konstante Funktion, die 
die Funktion $f$ mit maximalem Fehler $\varepsilon$ approximiert:
\[
|f(x)-g(x)|
=
|f(x) - f(x_k) + f(x_k) - g(x_k)|
\le
|f(x) - f(x_k)| + |f(x_k) - g(x_k)|
<
\varepsilon + 0
\]
für $x\in [x_k,x_{k+1})$.

Sei $f$ ein stetige Funktion, und $\{x_0,\dots,x_n\}$ eine Unterteilung
des Intervals $[a,b]$.
Dann kann man sofort mehrere stückweise approximationen Approximationen 
konstruieren:
\begin{align*}
\overline{f}(x)
&= 
\sup_{x\in[x_k,x_{k+1})} f(x)
&\text{falls }&x\in[x_k,x_{k+1}]
\\
\underline{f}(x)
&= 
\inf_{x\in[x_k,x_{k+1})} f(x)
&\text{falls }&x\in[x_k,x_{k+1}]
\end{align*}
Die Funktion $\overline{f}$ ist also immer mindestens so gross wie $f$
und $\underline{f}$ ist immer höchstens so gross wie $f$.
Es gilt also
\[
\underline{f}(x) \le f(x) \le \overline{f}(x)
\]
für $x\in[a,b]$.
Ausserdem gilt für die weiter oben konstruierte Approximation $g(x)$ der
Funktion $f(x)$ die Ungleichung
\[
\underline{f}(x) \le g(x) \le \overline{f}(x)
\]
für $x\in[a,b]$.

Die stetigen Funktionen auf einem Interval bilden einen Vektorraum: eine
Summe von stetigen Funktionen ist wieder eine stetige Funktionen.
Intuitiv ist auch klar, dass ein Summe stückweise konstanter Funktionen
wieder stückweise konstant ist.
Es ist aber auch klar, dass die Summe zweier Funktionen, die verschiedene
Unterteilungen verwenden, im allgemeinen mit keiner der Unterteilungen
stückweise konstant ist.
Seien $U_f=\{x_0^{(f)},\dots,x_n^{(f)}\}$ und
$U_g=\{x_0^{(g)},\dots,x_m^{(g)}\}$ die Unterteilung, bezüglich der 
die Funktionen $f$ und $g$ stückweise konstant sind.
Dann ist die Vereinigungsmenge
\[
U= \{x_0,\dots,x_N\} = U_f\cup U_g
\]
eine Unterteilung von $[a,b]$.
Da alle Teilpunkte von $U_f$ in $U$ sind, ist $f$ auch eine stückweise
konstante Funktion bezüglich der Unterteilung $U$. 
Ebenso ist $g$ eine stückweise konstante Funktion bezüglich $U$.
Da $U$ eine gemeinsame Unterteilung für $f$ und $g$ ist, kann man
die Summe $h=f+g$ als stückweise konstante Funktion mit
\[
h_k
=
(f+g)(x_k)
= 
f(x_k) + f(x_k)
\qquad\text{mit}\quad
0\le k\le N
\]
betrachten.
Die Vereinigung der Unterteilungen stellt also sicher, dass sich
zu zwei beliebigen stückweise konstanten Funktionen immer eine
gemeinsame Unterteilung finden lässt, die klar macht, dass $f+g$
wieder eine stückweise konstante Funktion ist.
Die Vektorraumstruktur verlangt also etwas mehr sorgfalt.

Das Integral einer stückweise konstanten Funktion ist besonders einfach 
zu berechnen, nämlich
\[
\int_{\mathbb R} f(x)\,dx = \sum_{k=0}^{n-1} f_i\cdot (x_{k+1}-x_k).
\]
Dies ist im wesentlichen die Definition des Riemann Integrals.
Genauer wird das Riemann-Integral einer stetigen Funktion jeweils
konstruiert mit Hilfe von genügend feinen Unterteilungen und den
Approximationen $\overline{f}$ und $\underline{f}$.
Dazu bildet man die Integrale
\begin{align*}
\int_a^b \underline{f}(x)\,dx
=
\sum_{k=0}^{n-1} \underline{f}_{k}\,(x_{k+1}-x_k)
\le
\sum_{k=0}^{n-1} \overline{f}_{k}\,(x_{k+1}-x_k)
=
\int_a^b \overline{f}(x)\,dx
\end{align*}
Verfeinerung des Korns der verwendeten Unterteilung bringt für eine
stetige Funktion die beiden Schranken näher zusammen.
Das Integral der stetigen Funktion $f(x)$ ist dann der Grenzwert
dieser beiden Schranken:
\[
\lim_{\delta\to 0}
\int_a^b \underline{f}(x)\,dx
=
\int_a^b f(x)\,dx
=
\lim_{\delta\to 0}
\int_a^b \overline{f}(x)\,dx,
\]
wobei für den Grenzwert Unterteilungen verwendet werden sollen, deren
Korn $\delta$ gegen $0$ geht.
Das Riemann-Integral verträgt sich daher gut mit den arithmetischen
Operationen: es ist eine lineare Abbildung.
Es verträgt sich aber auch mit dem Begriff der Approximation von
stetigen Funktionen: nahe beeinanderliegende Funktionen haben nahe
beeinanderliegende Integrale.

Die stückweise konstanten Funktionen bilden aber auch einen Vektorraum
mit Skalarprodukt.
Das Skalarprodukt ist wieder definiert durch
\[
\langle f,g\rangle
=
\int_a^b
f(x) g(x)\,dx.
\]
Ist $U$ eine gemeinsame Unterteilung für die stückweise konstanten
Funktionen $f$ und $g$, dann ist das Skalarprodukt
\[
\langle f,g\rangle
=
\sum_{k=0}^{n-1} f(x_k) g(x_k).
\]
Das Skalarprodukt hat also genau die gleiche Form wie das Skalarprodukt
in einem endlichdimensionalen Vektorraum.
Daraus kann zum Beispiel geschlossen werden, dass für dieses Skalarprodukt
die Cauchy-Schwarz-Ungleichung gilt und dass es positiv definit ist.

Der Vektorraum der stückweise konstanten Funktionen kann nicht vollständig
sein.
Jede stetige Funktion ist ein Grenzwert von stückweise konstanten
Funktionen.
Selbstverständlich kann man analoge Art auch einen komplexen
Vektorraum mit hermitschem Skalarprodukt konstruieren.
Weil er nicht vollständig ist, kann er kein Hilbertraum sein.
Um ihn besser zu verstehen wäre es schön, wenn man eine Hilbert-Basis
angeben könnte, also eine Folge von orthonormierten Funktionen, mit
denen man jede Funktion approximieren kann.
Die vielen möglichen Teilpunkte des Intervals sind dafür aber eher 
hinderlich.

