%
% sampling.tex
%
% (c) 2019 Prof Dr Andreas Müller, Hochschule Rapperswil
%
\section{Sampling%
\label{section:sampling}}
\rhead{Sampling}
Bei der Digitalisierung eines Signals wird eine sample hold Schaltung am
Eingang des Analog-/Digital-Wandlers verwendet, welche das Signal für die
Zeit der Wandlung in einen digitalen Wert konstant hält.
Die Abtastung eines Signals produziert also auf natürliche Weise
stückweise konstante Funktionen.
Digitalisierte Signal bilden also einen Vektorraum mit Skalarprodukt,
erfüllen also alle Voraussetzungen für eine Theorie ähnlich der Fourier-Theorie
bauen kann.
Im Abschnitt~\ref{section:stueckweise} wurde auch gezeigt, dass die freie
Unterteilbarkeit des Intervals die Konstruktion einer orthonormierten Basis
erschwert hat, dieses Problem soll in diesem Abschnitt in Angriff genommen
werden.

\subsection{Konstante Sampling-Rate}
Das Problem kann gelöst werden, indem man die Unterteilung ein für alle
Male festlegt, zum Beispiel indem man das Interval in $n$ gleichgrosse
Teil unterteilt:
\begin{equation}
x_k = a + k\cdot\frac{b-a}{n} = a + kh
\qquad
\text{mit $\displaystyle h=\frac{b-a}n$}
\label{sampling:unterteilung}
\end{equation}
Das Korn dieser Unterteilung ist $(b-a)/n$, durch Vergrösserung von $n$
kann man die Approximation verbessern.

%Diese spezielle Unterteilung ist von grosser praktischer Bedeutung,
%da die digitale Abtastung eines Signals $f(x)$ genau so eine Unterteilung
%der Zeitachse verwendet und die Funktion $f(x)$ durch die stückweise
%konstante Funktion
%\[
%\tilde{f} (x) = f(x_k)\quad\text{für $x\in[x_k,x_{k+1})$}.
%\]
% XXX Graphische Darstellung der gesampelten Funktion
Die Abtastung einer Funktion $f(x)$ an den Stellen $x_k$ ersetzt
die Funktion durch eine Folge von Zahlen $f_k = f(x_k)$.
Durch die Verwendung einer festen Unterteilung werden die arithmetischen
Operationen viel einfacher.
Um die Summe zu bilden ist es nicht mehr nötig, eine gemeinsame
Unterteilung zu finden, weil alle Funktionen bereits die gemeinsame
Unterteilung~\eqref{sampling:unterteilung} verwenden.
Die Vektorraum-Operationen sind ebenfalls einfach zu bilden,
da sie direkt auf den Sample-Vektoren wirken:
\begin{align*}
(f+g)_k &= f_k + g_k \\
(\lambda f)_k &= \lambda f_k.
\end{align*}

\subsection{Skalarprodukt
\label{haar:subsection:skalarprodukt}}
Auch das Skalarprodukt 
\[
\langle f,g\rangle
=
\frac{b-a}{n}
\sum_{k=0}^{n-1} f_k g_k
=
h
\sum_{k=0}^{n-1} f_k g_k
\]
kann allein mit Operationen auf den Samplevektoren berechnet werden.

\subsection{Basis
\label{haar:subsection:basis}}
Es ist auch einfach, eine Basis zu finden.
Dazu kann man die Indikatorfunktionen der $\chi_{[lh,(l+1)h)}(x)$
verwenden.
Zwecks Normierung verwenden wir statt der Indikatorfunktion die
Funktionen
\[
e^{(l)}(x) = \sqrt{\frac{n}{b-a}}\chi_{[lh,(l+1)h)}(x)j
\]
Sie sind auch definiert durch die Abtastwerte
\[
e_k^{(l)} = \sqrt{\frac{n}{b-a}}\delta_{kl}.
\]
Die Skalarprodukte dieser Funktionen sind
\[
\langle e^{(l)}, e^{(r)}\rangle
=
\frac{n}{b-a}
\sum_{k=0}^{n-1} e^{(l)}_k e^{(r)}_k
=
\frac{n}{b-a}
\biggl(\sqrt{\frac{b-a}{n}}\biggr)^2 \delta_{lr}
=
\delta_{lr},
\]
die Funktionen $e^{(l)}$ sind daher orthonormiert.

\subsection{Erweiterung auf $\mathbb R$}
Diese Konstruktion kann auf die ganze reelle Achse erweitert werden.
Der Sample-Abstand $h$ definiert eine Unterteilung
\[
\dots < x_{k} = kh < x_{k+1} < \dots
\]
der reellen Achse mit Korn $h$.
Zu einer stetigen Funktion $f(x)$ gehört dann ein Sample-Vektor
$f_k=f(x_k)=f(kh)$ mit $k\in\mathbb Z$, für den auch die
arithmetischen Operationen definiert sind.
Das Skalarprodukt wird durch die unendliche Summe
\[
\langle f, g\rangle
=
h
\sum_{k\in\mathbb Z} f_k g_k
\]
definiert.
Es ist natürlich nur definiert, wenn die Summe konvergiert.
Dazu müssen die Folgen $f_k$ und $g_k$ quadratsummierbar sein.
Diese Bedingung ist für die praktischen Anwendungen keine Einschränkung,
denn praktische Signale sind nur für endliche Zeit von $0$ verschieden,
die Summe hat also von vornherein nur endliche viele Terme.
Die Funktionen
\[
e^{(r)}_k = \delta_{rk}
\]
können als orthonormierte Basis verwendet werden.

\subsection{Translation
\label{haar:subsection:translation}}
Der erweiterte Definitionsbereich der $\{kh\,|\,k\in\mathbb Z\}$ ermöglicht,
Signale zu vergleichen, die nur zeitlich verschoben sind.
Sind zwei Funktionen $f(x)$ und $g(x)$ um ein Vielfaches von $h$ verschoben,
dann gibt es eine ganze Zahl so dass $f(x)=g(x+lh)$ gilt und daher auch
$f_k = g_{k+l}$.
Zeitliche Translationen der Funktion $f(x)$ können also abgebildet
werden auf ganzzahlige Translationen der Folgen $f_k$.
Diese Translationen sind bezüglich des Skalarprodukts Isometrien.

Allerdings haben wir uns jetzt ein weiteres Problem eingehandelt.
Weil das Korn der Unterteilung fest ist, können wir nicht mehr jede
beliebige Funktion beliebig genau approximieren.
%In der Tat hat das Sampling-Theorem gezeigt, dass nur bandbreitenbegrenzte
%Funktionen exakt rekonstruiert werden können.
Man könnte natürlich von Vornherein ein so feines Korn wählen, dass alle
möglicherweise interessierenden Details durch den Sampling-Prozess
erfasst werden.
Bei einem langsam veränderlichen Signal werden aufeinanderfolgende Samples
sehr nahe beeinander liegen.
Es wird daher eine grosse Datenmenge produziert, die aber in hohem
Masse redundant ist, da der grösste Teil der Information bereits bei
viel kleinerem Sample-Abstand erkennbar wird.

Gesucht ist also eine Hierarchie von Unterteilungen, so dass wir einerseits
die nützliche Trans\-la\-tions-Invarianz retten können, aber andererseits auch
beliebig gute Approximation eines Signals finden können.



