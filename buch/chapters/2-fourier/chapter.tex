%
% chapter.tex
%
% (c) 2019 Prof Dr Andreas Müller, Hochschule Rapperswil
%
\chapter{Fouriertheorie und die $L^2$-Hilberträume
\label{chapter:fourier}}
\lhead{Fouriertheorie}
Die Approximation von Funktionen auf einem endlichen oder unendlichen
Interval soll gemäss der Ideen in Kapitel~\ref{chapter:geometrie}
mit Hilfe der Ideen der Vektorgeometrie und des Skalarproduktes
durchgeführt werden.
Dazu muss die Menge der Funktionen zunächst in einen Hilbertraum
verwandelt werden, dies geschieht in Abschnitt~\ref{section:l2}.
Die anschliessend zusammengefasste Fourier-Theorie zeigt, dass sich
die geometrisch motivierte Analyse und Synthese von Funktionen
tatsächlich durchführen lässt.

%
% l2.tex -- L2 Räume 
%
% (c) 2019 Prof Dr Andreas Müller, Hochschule Rapperswil
%
\section{Der Hilbertraum $L^2$
\label{section:l2}}
\rhead{Der Hilbertraum $L^2$}
Bereits im Beispiel auf Seite~\pageref{geometrie:l2-beispiel} wurde
angedeutet, wie man aus einem Funktionenraum einen Hilbertraum
machen kann, indem man ein geeignetes Skalarprodukt definiert.
In diesem Abschnitt soll diese Konstruktion etwas detaillierter 
durchgeführt und verallgemeinert werden.

\subsection{Funktionenräume}
Wir bezeichnen mit $\mathbb C^X$ die Menge der Funktionen
$f\colon X\to \mathbb C$.
Sie wird auf natürliche Art zu einem Vektorraum, indem man die
Addition von Funktionen und die Multiplikation mit Skalaren
punktweise definiert.
Seien $f,g$ Funktionen auf $X$, dann definieren wir die Summe $f+g$ und
$\lambda f$ als
\begin{align*}
f+g&\colon X\to\mathbb C: x \mapsto f(x) + g(x)
\\
\lambda f &\colon X \to \mathbb C: x \mapsto \lambda f(x).
\end{align*}
Die Signale, die im Folgenden analysiert werden sollen, sind jedoch
nicht beliebige Funktionen.
Sie haben zusätzliche Eigenschaften, zum Beispiel sind sie oft stetig,
oder beschränkt.
Die bekannten Rechenregeln für stetige Funktionen stellen sicher, dass
diese Eigenschaften in Summen von Funktionen und skalaren Vielfachen
erhalten bleiben.

Wenn die Funktion $f$ durch andere Funktionen approximiert werden soll,
dann wird dazu ein Abstandsbegriff benötigt.

\begin{definition}
Eine reellwertige Funktion $\|\mathstrut\cdot\mathstrut\|$ heisst
eine Norm auf dem Vektorraum $V$, wenn sie folgende Eigenschaften hat:
\index{Norm}
\begin{enumerate}
\item $\|v\|=0\;\Leftrightarrow\; v = 0$
\item $\| \lambda u \| = |\lambda| \,\|u\|$
\item $\|u + v\| \le \|u\| + \|v\|$ (Dreiecksungleichung)
\end{enumerate}
\end{definition}

Die Norm, die in Abschnitt~\ref{section:hilbertraum} aus dem Skalarprodukt
gewonnen wurde, hat tatsächlich diese Eigenschaften.
Dabei ist nur die Dreiecksungleichung nicht unmittelbar klar.
Doch aus der Cauchy-Schwarz-Ungleichung folgt
\begin{align*}
\| u + v \|^2
&=
\langle u+v,u+v\rangle
=
\| u \|^2 + 2\operatorname{Re} \langle u,v\rangle + \| v\|^2
\\
&\le
\| u \|^2 + 2|\operatorname{Re} \langle u,v\rangle| + \| v\|^2
\\
&\le
\| u \|^2 + 2|\langle u,v\rangle| + \| v\|^2
\\
&\le
\| u \|^2 + 2\| u \| \cdot \|v\| + \| v\|^2
=
(\|u\| + \| v \|)^2
\\
\|u+v\|
&\le
\|u\| + \|v\|
\end{align*}

\begin{definition}
Der Vektorraum der stetigen Funktionen auf $X\subset R^n$ ist die Menge
\[
C_{\mathbb C}(X)
=
C(X)
=
\{ f\in \mathbb C^X\,|\, \text{$f$ ist stetig}\}
\]
mit der punktweisen Addition und Multiplikation mit Skalaren und der
Norm
\[
\| f \| = \sup_{x\in X} |f(x)|.
\]
$\|f\|$ heisst auch die Supremum-Norm.
\end{definition}

Man beachte, dass diese Norm nicht von einem Skalarprodukt herkommt.
Man kann aber zeigen, dass Cauchy-Folgen in dieser Norm gegen eine
stetige Funktion konvergieren.
Diese Norm stellt also sicher, dass die Grenzfunktion einer
Approximationsfolge aus stetigen Funktion wieder stetig ist.
Umgekehrt können wir bei einer anderen Norm wie der im nächsten
Abschnitt definierten $L^2$-Norm nicht mehr garantieren, dass 
Grenzfunktionen stetig sind.
Dies verursacht zwar ein paar mathematische Unannehmlichkeiten,
kommt aber den Anwendungen entgegen, da in der Praxis durchaus
nicht stetige Signale vorkommen.

\subsection{Definition des Skalarproduktes}
In diesem Abschnitt betrachten wir ausschliesslich Funktionen, die auf
einem endlichen oder unendlichen Interval $I$ definiert sind.

\begin{definition}
Das Skalarprodukt zweier Funktionen $f,g\colon I\to\mathbb C$ ist definiert
als
\[
\langle f,g\rangle
=
\int_I f(t) \bar{g}(t)\,dt
\]
\end{definition}

Die bekannten Rechenregeln für Integrale stellen sicher, dass dies
tatsächlich ein sesquilineares Produkt ist, wie die folgende Rechnung
zeigt:
\begin{align*}
\langle \lambda_1 f_1+\lambda_2 f_2,g\rangle
&=
\int_I (\lambda_1 f_1(t) + \lambda_2 f_2(t))\bar{g}(t)\,dt
\\
&=
\lambda_1 \int_I f_1(t) \bar{g}(t)\,dt + \lambda_2 \int_I f_2(t)\bar{g}(t)\,dt
= \lambda_1 \langle f_1,g\rangle + \lambda_2 \langle f_2,g\rangle
\\
\langle f,\mu_1 g_1 + \mu_2 g_2\rangle
&=
\int_I f(t) \overline{(\mu_1 g_1(t) + \mu_2 g_2(t))}\,dt
=
\bar{\mu}_1 \int_I f(t) \bar{g}_1(t)\,dt
+
\bar{\mu}_2 \int_I f(t) \bar{g}_2(t)\,dt
\\
&=
\bar{\mu}_1 \langle f,g_1\rangle
+
\bar{\mu}_2 \langle f,g_2\rangle
\\
\overline{
\langle f,g\rangle
}
&=
\overline{ \int_I f(t)\bar{g}(t)\,dt}
=
\int_i \bar{f}(t) g(t)\,dt
=
\int_i g(t) \bar{f}(t)\,dt
=
\langle g,f\rangle.
\end{align*}
Weniger klar ist jedoch, ob dieses Produkt auch tatsächlich definit ist.
Die aus dem Skalarprodukt abgeleitete Norm ist
\[
\|f\|^2 = \int_I |f(t)|^2\,dt \ge 0.
\]
Für eine stetige Funktion folgt tatsächlich, dass die Norm nur
verschwinden kann, wenn die Funktion überall verschwindet.
Eine Funktion, die nur an endlich vielen Stellen nicht verschwindet,
ist immer noch integrierbar und ihr Integral ist $0$.
Eine solche Funktion hätte also Norm $0$ ohne selbst die Nullfunktion zu
sein.
Dies deutet an, dass wir bei der Auswahl der Funktionenmenge, mit der
wir arbeiten wollen, sehr viel sorgfältiger sein müssen.

Die Definition ist natürlich nur sinnvoll für Funktionen, für die diese
Integrale tatsächlich existieren.
\begin{definition}
Sei $p\in \{1,2\}$ 
\begin{align*}
\mathcal{L}^p(I)
=
\left\{ f \in \mathbb C^I \, \left|
\text{
$f$ ist integrierbar und $\int_I |f(t)|^p\,dt<\infty$
}
\right.\right\}
\end{align*}
mit zugehöriger Norm
\[
\|f\|_p = \biggl(\int_I |f(t)|^p \,dt\biggr)^{\frac1p}.
\]
\end{definition}

Für $p=2$ ist die Norm $\|f\|_2$ bereits bekannt, es ist die Norm, die
vom Skalarprodukt herkommt.

Die Beziehung der Vektorräume zueinander ist nicht ganz offensichtlich.

\begin{lemma}
\label{fourier:l1l2}
Ist $I$ ein beschränktes Interval, dann ist $\mathcal{L}^2(I)$ eine
echte Teilmenge von $\mathcal{L}^1(I)$.
Es gibt also eine integrierbare Funktion auf $I$, die nicht
quadratintegrierbar ist.
\end{lemma}

\begin{proof}[Beweis]
\begin{figure}
\centering
\includegraphics{chapters/2-fourier/images/intfun.pdf}
\caption{Funktionen der Form $f(x)=x^{-\alpha}$ auf dem Interval
$[0,\frac{10}{3}]$ sind nicht integrierbar für $\alpha > 1$ (grün).
Für $\alpha <\frac12$ ist $f(x)$ quadratintegrierbar und auch integrierbar.
Dazwischen, für $\frac12 <\alpha < 1$, ist $f(x)$ zwar integrierbar, aber
nicht quadratintegrierbar.
\label{fourier:intfun}}
\end{figure}
Sei $f\in\mathcal{L}^2(I)$, es gilt also
\[
\int_I |f(x)|^2\,dx < \infty.
\]
Wir versuchen jetzt, dass Integral von $|f(x)|$ zu berechnen.
Dazu verwenden wir, dass wir $|f(x)|$ als das Produkt
$|f(x)|\cdot 1$ mit der Konstanten Funktion $1$ schreiben
und die Cauchy-Schwarz-Ungleichung auf das Produkt anwenden können:
\begin{align*}
%\|f\|_1
%&
=
\int_I |f(x)|\,dx
&=
\int_I |f(x)| \cdot 1 \,dx
=
\langle |f|, 1\rangle
\le
\|f\|_2 \cdot \| 1 \|_2
\\
&=
\int_I |f(x)|^2\,dx \cdot \int_I 1\,dx
\end{align*}
Da das Interval $I$ beschränkt ist, ist das rechte Integral gerade die
Länge des Intervals.
Das linke Integral ist die Norm $\|f\|_2$, so dass wir ablesen können,
dass $\|f\|_1$ beschränkt ist.
Eine quadratintegrierbare Funktion ist also automatisch integrierbar,
$\mathcal{L}^2(I)\subset \mathcal{L}^1(I)$.

Um zu zeigen, dass $\mathcal{L}^2(I)$ eine echte Teilmenge von 
$\mathcal{L}^1(I)$ ist, 
müssen wir jetzt eine integrierbare Funktion
finden, die nicht quadratintegrierbar ist.
Wir nehmen dazu an, dass $I=[a,b]$ und versuchen eine Funktion der Form
\begin{align*}
f(x)
&=
\begin{cases}
0&\qquad x=a\\
(x-a)^\alpha&\qquad x>a.
\end{cases}
\\
\end{align*}
Die Normen für $p=1$ ist
\begin{align*}
\|f\|_1
&=
\int_a^b |f(x)|\,dx
=
\int_a^b (x-a)^\alpha\,dx 
=
\int_0^{b-a} x^\alpha\,dx 
=
\biggl[
\frac{1}{\alpha+1}x^{\alpha+1}
\biggr]_0^{b-a}.
\intertext{Für $p=2$ kann man die gleiche Formel mit $2\alpha$ an Stelle
von $\alpha$ vewenden:}
\|f\|_2^2
&=
\biggl[
\frac{1}{2\alpha+1}x^{2\alpha+1}
\biggr]_0^{b-a}.
\end{align*}
Das Integral existiert also genau dann, wenn der Exponent von $x$
positiv ist.
Wenn $\alpha$ so gewählt wird, dass $\alpha+1>0$ und $2\alpha+1 <0$ ist,
dann haben wir ein Beispiel der gesuchten Art gefunden.
Die beiden Ungleichungen bedeuten $\alpha >-1$ und $\alpha < -\frac12$.
Die Funktionen
\[
f_\alpha(x)
=
\begin{cases}
\displaystyle\frac{1}{(x-a)^\alpha}&\qquad x > a\\
0&\qquad x=a
\end{cases}
\qquad
\text{für}
\quad
\frac12 <\alpha < 1
\]
sind integrierbar, aber nicht quadratintegrierbar.
Die Funktionen $f_\alpha(x)$ sind in Abbildung~\ref{fourier:intfun}
dargestellt.
\end{proof}

Das nachfolgende Beispiel sollen zeigen, dass die Situation für eine
unbeschränktes Interval viel komplizierter ist.
Insbesondere gibt es quadratintegrierbare Funktionen auf $\mathbb R$,
die nicht integrierbar sind, und integrierbare Funktionen, die nicht
quadratintegrierbar sind.

\begin{beispiel}
Den Funktion $f_\alpha$, die in Beweis von Lemma~\ref{fourier:l1l2}
verwendet wurden, haben ausgenutzt, dass sie für $x\to 0$ verschieden 
schnell divergiert haben.
Das Verhalten für $x\to \frac{10}3$ war nicht relevant, weil dort keine
Divergenz auftrat.

Auf dem unbeschränkten Interval $I=[0,\infty)$ kann eine Funktion 
für $x\to\infty$ zwar gegen Null gehen, aber das Integral kann trotzdenm
divergieren.
Diese Eigenschaft können wir ausnützen, um zu zeigen, dass weder
$\mathcal{L}^1(I)\subset\mathcal{L}^2(I)$
noch
$\mathcal{L}^2(I)\subset\mathcal{L}^1(I)$.

Wir betrachten zunächst die Funktionen
\[
f_\alpha(x) = \begin{cases}
\displaystyle\frac1{x^\alpha}&\qquad0<x\le 1\\
0&\qquad\text{sonst.}
\end{cases}
\]
Gemäss der früheren Rechnungen ist
$f_\alpha\in\mathcal{L}^1(I)\setminus\mathcal{L}^2(I)$,
wenn $\frac12<\alpha<1$.

Andererseits können wir auch die Funktionen
\[
g_\beta(x)
=
\begin{cases}
\displaystyle\frac1{x^\beta}&\qquad x>1\\
0&\qquad\text{sonst}
\end{cases}
\]
betrachten.
Wir berechnen wieder die Normen
\begin{align*}
\| g_\beta\|_1
&=
\int_I|g_\beta(x)|\,dx
=
\int_1^\infty x^{-\beta}\,dx
=
\biggl[
\frac1{-\beta+1}x^{-\beta+1}
\biggr]_1^\infty
\\
\|g_\beta\|_2^2
&=
\biggl[
\frac1{-2\beta+1}x^{-2\beta+1}
\biggr]_1^\infty
\end{align*}
Die Normen existieren genau dann, wenn die Exponenten negativ sind.
Die Norm für $p=1$ ist also beschränkt, wenn $-\beta+1<0$ oder $\beta > 1$
ist.
Die Norm für $p=2$ ist beschränkt, wenn $-2\beta+1<0$ oder $\beta > \frac12$
ist.
Für $\beta$ zwischen $\frac12$ und $1$ ist $g_\beta$ also quadratintegrierbar
aber nicht integrierbar.
Somit ist $g_\beta\in\mathcal{L}^2(I)\setminus\mathcal{L}^1(I)$.
Die beiden Beispiele zeigen, dass weder
$\mathcal{L}^1(I)\subset\mathcal{L}^2(I)$
noch
$\mathcal{L}^2(I)\subset\mathcal{L}^1(I)$.
\end{beispiel}


\subsection{Lebesgue-Integral und Vollständigkeit}
Das Riemann-Integral, das man im Analysis-Unterricht kennen lernt, 
ist leider nicht geeignet für das vorliegende Approximationsproblem.
Eine Funktion soll durch quadratintegrierbare Funktionen approximiert
werden im Sinne der Norm $\|\mathstrut\cdot\mathstrut\|_2$.
Doch kann man leicht Folgen konstruieren, die im Sinne dieser Norm
Cauchy-Folgen sind, aber die Grenzfunktion ist nicht mehr integrierbar.

\begin{beispiel}
Die Menge $[0,1]\cap \mathbb Q$ der rationalen Zahlen im Interval
$[0,1]$ ist abzählbar, es gibt  also eine Folge $q_k$, die alle rationalen
Zahlen durchläuft.
Daraus kann man jetzt eine Funktionen-Folge $f_n$ wie folgt konstruieren:
\[
f_n(t) = \begin{cases} 
1&\qquad \text{$t=t_k$ für ein $k\le n$}\\
0&\qquad\text{sonst}.
\end{cases}
\]
Jede der Funktionen $f_n$ ist integrierbar, denn sie weichen nur an
endlich vielen, genauer an $n$ Stellen von $0$ ab.
Ihr Integral ist daher $0$.
Der Abstand zwischen zwei Funktionen der Folge ist $\| f_n-f_m\|_2 = 0$,
da auch die Differenz nur an endlich vielen Stellen von $0$ verschieden ist.
Trotzdem kann man nicht sagen, dass die Folge eine Grenzfunktion hat.
Punktweise konvergiert die Folge $f_n$ gegen die Funktion
\[
f_{\mathbb Q}(t) = \begin{cases}
1&\qquad t\in\mathbb Q\\
0&\qquad\text{sonst}
\end{cases}
\]
Diese Funktion ist aber nicht einmal integrierbar im Sinne des
Riemann-Integrals.
\end{beispiel}

Das Beispiel illustriert die Unzulänglichkeit des Riemann-Integrals.
Die Lösung besteht darin, den Integralbegriff zu erweitern.
Dies ist Henri Lebesgue in seiner Disseration 1902 gelungen.
Das Lebesgue-Integral ist für alle Riemann-integrierbaren Funktionen
definiert und ergibt denselben Wert.
Es ist also für konkrete Berechnungen nicht relevant.
Es erweitert die Menge der integrierbaren Funktionen und stellt insbesondere
sicher, dass die Grenzfunktionen von Folgen unter genügend allgemeinen
Voraussetzungen integrierbar sind und dass Integral und Grenzwert vertauschbar
sind.
Damit ist die Klasse gross genug für das Approximationsproblem.

% XXX Fubini
% XXX Fatou
% XXX Dominated convergence

Das Lebesgue-Integral betrachtet die Funktion $f_{\mathbb Q}$ 
als integrierbar mit Integralwert $0$.
Für eine beliebige integrierbare Funktion $g$ ist daher auch 
$g+f_{\mathbb Q}$ integrierbar und hat den gleichen Integralwert
wie $g$.
In $\mathcal{L}^p$ sind daher sehr viele Funktionen nicht unterscheidbar,
da sie sich nur um eine Funktion mit Integral $0$ unterscheiden.

\begin{definition}
Die Menge $L^p(I)$ besteht aus Äquivalenzklassen von Funktionen in
$\mathcal{L}^p(I)$, die sich durch eine Nullfunktion unterscheiden.
Zwei Funktionen $f_1,f_2\in L^p(I)$ werden als gleich betrachtet,
wenn
\[
\int_I
|f_2-f_1|
\,dx
=0
\]
ist.
\end{definition}

Die Menge $L^p(I)$ erbt von $\mathcal{L}^p(I)$ die Struktur eines
Vektorraums mit der Norm $\|\mathstrut\cdot\mathstrut\|_p$.
Der Raum $L^2(I)$ wird dank der oben formulierten Grenzwertsätze zu
einem Hilbertraum.
Damit ist der Raum $L^2(I)$ als die Bühne für die nachfolgend zu diskutierende
Approximationstheorie bereitgestellt.

Die Definition von $L^p(I)$ als Menge von Äquivalenzklassen von Funktionen
ist etwas schwerfällig und ungewohnt, für die Praxis aber kaum von Bedeutung.
Jegliche Rechnungen mit Funktionen finden immer mit einem
Riemann-integrierbaren Repräsentaten statt.


\subsection{$L^2$ und $L^1$}
Eine integrierbare Funktion ist nicht automatisch
quadratintegrierbar.
Wenn eine Funktion gerade noch langsam genug anwächst, dass ihr
Integral existiert, kann das Quadrat der Funktion bereits zu schnell
anwachsen, so dass das Quadrat nicht mehr integrierbar ist.

\begin{beispiel}
Auf dem Interval $I=[0,1]$ ist die Funktion
\[
f\colon I\to \mathbb R: t\mapsto \begin{cases} 0&\qquad t=0\\
t^{-\alpha}&\qquad t > 0
\end{cases}
\]
gegeben, der genaue Wert von $\alpha$ wird später festgelegt.
Die Integrale von $f$ und $f^2$ sind
\begin{align*}
\int_0^1 |f(t)|\,dt
&=
\int_0^1 t^{-\alpha}\,dt
=
\biggl[\frac{1}{1-\alpha}t^{1-\alpha}\biggr]_0^1
=
\begin{cases}
\frac{1}{1-\alpha}&\qquad \alpha < 1\\
\infty&\qquad \alpha \ge 1
\end{cases}
\\
\int_0^1|f(t)|^2\,dt
&=
\int_0^1 t^{-2\alpha}\,dt
=
\begin{cases}
\frac{1}{1-2\alpha}&\qquad \alpha < \frac12\\
\infty&\qquad \alpha \ge \frac12
\end{cases}
\end{align*}
Für $\frac12<\alpha<1$ tritt also die Situation ein, dass das Integral
von $f$ existiert, das von $f^2$ aber nicht\footnote{In der Rechnung in
diesem und im nächsten Beispiel wurde der Fall $\alpha=1$ nicht sorgfältig
nachgerechnet.
In diesem Fall ist die Stammfunktion nämlich $\log|t|$, der jedoch
auch für $t\to 0$ und $t\to\infty$ divergiert.
Die Behauptungen sind daher auch in diesem Fall richtig.}.
\end{beispiel}

Die umgekehrte Situation kann für Funktionen auf $\mathbb R$ eintreten,
die zu langsam abfallen, um integrierbar zu sein. 
Da das Quadrieren kleine Werte noch kleiner macht, kann das Quadrat
der Funktion schnell genug abfallen, so dass es integrierbar ist.
Eine solche Funktion ist in $L^2(\mathbb R)$, aber nicht in $L^1(\mathbb R)$.

\begin{beispiel}
Auf dem Interval $J=[1,\infty)$ ist die Funktion
$f(t)=t^{-\alpha}$ gegeben.
Die Integrale von $f$ und $f^2$ sind
\begin{align*}
\int_1^\infty |f(t)|\,dt
&=
\int_1^\infty t^{-\alpha}\,dt
=
\biggl[\frac{t^{1-\alpha}}{1-\alpha}\biggr]_1^\infty
=
\begin{cases}
\frac{1}{\alpha-1}&\qquad \alpha > 1\\
\infty &\qquad \alpha \le 1
\end{cases}
\\
\int_1^\infty |f(t)|^2\,dt
&=
\int_1^\infty t^{-2\alpha}\,dt
=
\biggl[\frac{t^{1-2\alpha}}{1-2\alpha}\biggr]_1^\infty
=
\begin{cases}
\frac{1}{2\alpha-1}&\qquad \alpha > \frac12\\
\infty &\qquad \alpha \le \frac12
\end{cases}
\end{align*}
Man liest daraus ab, dass für $\frac12<\alpha < 1$ die Funktion $f$ zwar
in $L^2([0,\infty))$, nicht aber in $L^1([1,\infty))$ ist.
\end{beispiel}

Der Fall des letzten Beispiels kann vermieden werden, wenn man den
Definitionsbereich der Funktion auf ein kompaktes Interval beschränkt.
Dann folgt, der folgende Satz.

\begin{satz}
\label{satz:l2inl1}
Ist $I$ ein kompaktes Interval, dann ist $L^2(I)\subset L^1(I)$, oder:
jede quadratintegrierbare Funktion auf einem kompakten Interval ist 
integrierbar.
\end{satz}

\begin{proof}[Beweis]
Wir zerlegen die Funktion $f$ in eine Summe $f=f_-+f_+$. 
Dabei setzen wir:
\begin{align*}
f_+(t) &= 
\begin{cases}
f(t)&\qquad |f(t)| > 1\\
0   &\qquad\text{sonst}
\end{cases}
&&\text{und}
&
f_-(t)
&=
\begin{cases}
0   &\qquad |f(t)| > 1\\
f(t)&\qquad f(t)\le 1
\end{cases}
\end{align*}
Die Teilfunktion $f_+$ sammelt also all jene Werte, die beim Quadrieren
grösser werden, die Teilfunktion $f_-$ dagegen diejenigen Werte, die
beim Quadrieren kleiner werden.
Wir schätzen jetzt das Integral von $|f|$ ab, indem wir es in die Summanden
$f_+$ und $f_-$ zerlegen:
\begin{align*}
\int_I |f(t)|\,dt
&=
\int_I |f_+(t)|\,dt + \int_I |f_-(t)|\,dt
\le
\int_I |f_+(t)|^2\,dt + \int_I 1\,dt
\le 
\int_I |f(t)|^2\,dt + |I|
\le \|f\|^2 + |I|.
\end{align*}
Da nach Voraussetzung $f^2$ integrierbar ist und das Interval $I$ kompakt ist
und damit beschränkte Länge $|I|$ länge hat, ist das Integral von $|f|$
ebenfalls beschränkt.
\end{proof}

\subsection{Die Operatoren für Translation und Dilatation
\label{subsection:translation-dilatation2}}
Für das Verständnis von Wavelets müssen wir die Wirkung der Operatoren
$T_b$ und $D_a$ auf das Skalarprodukt und die $L^2$-Norm verstehen.

\begin{satz}
Für Funktionen $f,g\in\mathbb R$ gilt
\begin{align}
&\qquad&
\langle T_bf,T_bg\rangle
&=
\langle f,g\rangle
&
\| T_bf\|^2 &= \|f\|^2
&\qquad&
\label{normTb}
\\
&&
\langle D_af,D_ag\rangle
&=
|a|\cdot
\langle f,g\rangle
&
\| D_af\|^2 &= |a|\cdot \|f\|^2.
&&
\label{normDa}
\end{align}
\end{satz}

\begin{proof}[Beweis]
Die Translation ändert die Norm nicht, weil das Lebesgue-Integral 
translationsinvariant ist, wie man sich durch die Rechnung
\begin{align*}
\int_{-\infty}^\infty (T_bf)(t)\,\overline{(T_bg)(t)}\,dt
&=
\int_{\infty}^\infty f(\underbrace{t-b}_{\displaystyle=t'})\bar{g}(t-b)\,dt
=
\int_{\infty}^\infty f(t')\bar{g}(t')\,dt'
=
\langle f,g\rangle
\\
\int_{-\infty}^\infty |(T_bf)(t)|^2\,dt
&=
\int_{-\infty}^\infty |f(\underbrace{t-b}_{\displaystyle=t'})|^2\,dt
=
\int_{-\infty}^\infty |f(t')|^2\,dt'
=
\|f\|^2
\end{align*}
überzeugen kann.
Für $\tilde{D}_a$ findet man ähnlich
\begin{align*}
\langle \tilde{D}_af,\tilde{D}_ag\rangle
&=
\int_{-\infty}^\infty (\tilde{D}_af)(t)\,\overline{(\tilde{D}_ag)(t)}\,dt
=
\int_{-\infty}^\infty
f(\underbrace{t/a}_{\displaystyle=t'})
\overline{g(t/a)}
\,dt
=
\int_{-\infty}^\infty f(t')\bar{g}(t')\,|a|\,dt'
\\
&=
|a|\cdot \langle f,g\rangle
\end{align*}
mit der Substitution $t'=t/a$ bzw.~$t=at'$.
Daraus ergibt sich auch die Beziehung
\[
\|\tilde{D}_af\|^2
=
\langle \tilde{D}_af,\tilde{D}_af\rangle
=
|a|\cdot \langle f,f\rangle
=
|a|\cdot \|f\|^2
\]
für die Norm.
\end{proof}

Der Operator $\tilde{D}_a$ erhält nach \eqref{normDa} die Norm nicht.
Eine um den Faktor $|a|$ ``gestreckte'' Funktion hat eine um $\sqrt{|a|}$
grössere Norm.
Dies lässt sich aber leicht korrigieren, indem man den Dilationsoperator
neu skaliert, wie in der folgenden Definition.


%In Abschnitt~\ref{subsection:translation-dilatation} haben wir die
%Operationen $T_b$ und $\tilde{D}_a$ definiert.
%Es ist ganz offensichtlich, dass die Operation $T_b$ die Norm einer
%Funktion $f(t)$ nicht ändert, denn
%\[
%\|T_bf\|
%=
%\int_{-\infty}^{\infty} |T_bf(t)|^2\,dt
%=
%\int_{-\infty}^{\infty} |f(t-b)|^2\,dt
%=
%\int_{-\infty}^{\infty} |f(t)|^2\,dt
%=
%\|f\|^2.
%\]
%Die Dilatation $\tilde{D}_a$ hingegen ändert die Norm, es gilt nämlich
%\begin{align*}
%\|\tilde{D}_af\|^2
%&=
%\int_{-\infty}^\infty |\tilde{D}_af(t)|^2\,dt
%=
%\int_{-\infty}^\infty |f(t/a)|^2 \,dt
%=
%\int_{-\infty}^\infty |f(\tau)|^2 \, a\,d\tau
%=
%\int_{-\infty}^\infty |\sqrt{a} f(\tau)|^2 \,d\tau
%=
%\| \sqrt{a} f \|^2,
%\end{align*}
%wobei wir die Subsitution $\tau = t/a$ verwendet haben.
%Wir können daraus ablesen, wie wir die Definition von $\tilde{D}_a$
%modifizieren müssen, dass auch die Dilatation die Norm erhält.

\begin{definition}
Die Dilatation $D_af$ der Funktion $f\colon\mathbb R\to\mathbb C$ ist 
\[
(D_af)(t)
=
\frac{1}{\sqrt{a}} \tilde{D}_af(t)
=
\frac{1}{\sqrt{a}} f\biggl(\frac{t}{a}\biggr).
\]
\end{definition}
Der Operator $D_a$ ist im Gegensatz zu $\tilde{D}_b$ eine Isometrie.

\begin{satz}
Die Operationen $T_b$ und $D_a$ sind Isometrien, sie erhalten die
Norm:
\[
\begin{aligned}
\|T_bf\| &= \|f\|
&&\qquad&
\|D_af\| &= \|f\|.
\end{aligned}
\]
Ausserdem gelten die Rechenregeln
\begin{align*}
T_{b_1}T_{b_2}&=T_{b_1+b_2}
&
D_{a_1} D_{a_2}&=D_{a_1a_2}
&
D_aT_b
&=
T_{ab}D_a
\end{align*}
\end{satz}

\begin{proof}[Beweis]
Die Eigenschaften der Operatoren bezügich der Norm und die Rechenregeln
folgen unmittelbar aus den Definitionen und
den früher bewiesenen Regeln, insbesondere Satz~\ref{satz:kommutator}.
\end{proof}

\begin{figure}
\centering
\includegraphics{chapters/2-fourier/images/kommutatorD.pdf}
\caption{Vertauschungsregel für die Operationen $T_b$ und $D_a$
(vergleiche auch Abbildung~\ref{geometrie:kommutator:image} für
die Operationen $T_b$ und $\tilde{D}_a$).
Die Graphen der Bildfunktionen unter $D_a$ sind gegenüber denen der
Bildfunktionen von $\tilde{D}_a$ vertikal um den Faktor $\sqrt{a}$
gestaucht.
\label{geometrie:kommutatorD:image}}
\end{figure}

Die Vertauschungsregel für $T_b$ und $D_a$ ist in
Abbildung~\ref{geometrie:kommutatorD:image} visualisiert.






%
% reihen.tex
%
% (c) 2019 Prof Dr Andreas Müller, Hochschule Rapperswil
%
\section{Fourier-Reihen
\label{section:fourier-reihen}}
\rhead{Fourier-Reihen}


\subsection{Reelle Fourier-Reihen}

\subsection{Komplexe Fourier-Reihen}

\subsection{Plancherel-Bedingung}


%
% integral.tex -- Fourier-Integral
%
% (c) 2019 Prof Dr Andreas Müller, Hochschule Rapperswil
%
\section{Fourier-Integral
\label{section:fourier-integral}}
\rhead{Fourier-Integral}
Für Funktionen $f\in L^2(\mathbb R)$ ist die Fourier-Transformation
\[
\hat{f}(\omega)
=
\frac{1}{\sqrt{2\pi}}
\int_{-\infty}^\infty f(t) e^{-i\omega t}\,dt
\]
wohldefiniert.
Wir schreiben die Fourier-Transformation auch
\[
\mathcal{F}\colon f\mapsto \hat{f}.
\]
Sie verallgemeinert die Eigenschaften der Fourier-Koeffizienten $c_k$
für Funktionen auf $\mathbb R$.
Für die Rechungen in den folgenden Kapiteln stellen wir hier die
wichtigsen Formeln für die Fourier-Transformation zusammen.

\begin{satz}
Für $\hat{f}\in L^2(\mathbb R)$ gilt die Umkehrformel
\[
f(t)
=
\frac{1}{\sqrt{2\pi}}
\int_{-\infty}^{\infty} \hat{f}(\omega)e^{i\omega t}\,d\omega.
\]
\end{satz}

Die Fourier-Transformation ist also eine invertierbare lineare Abbildung,
$\mathcal{F}^{-1}\!\left\lbrace\hat{f}\right\rbrace = f$.

\begin{satz}
Die Fourier-Transformation $\mathcal F\colon f\mapsto f$ ist eine linear
Abbildung, die sich mit Translation, Dilatation und Ableitung wie folgt
verträgt:
\begin{align*}
\widehat{T_bf}(\omega)
&=
e^{-i\omega b}\hat{f}(\omega).
\\
\widehat{D_af}(\omega)
&=
a \hat{f}(a\omega)
\\
\widehat{e^{ibt}f}(\omega)
&=
(T_b\hat{f})(\omega).
\end{align*}
\end{satz}

\begin{proof}[Beweis]
Durch direkte Rechnung finden wir:
\begin{align*}
\widehat{T_bf}(\omega)
&=
\int_{-\infty}^{\infty} (T_bf)(t)e^{-i\omega t}\,dt
=
\int_{-\infty}^{\infty} f(\underbrace{t-b}_{\displaystyle=t'})e^{-i\omega t}\,dt
=
\int_{-\infty}^{\infty} f(t')e^{-i\omega(t'+b)}\,dt'
\\
&=
e^{-i\omega b}
\int_{-\infty}^{\infty} f(t')e^{-i\omega t'}\,dt'
=
e^{-i\omega b}\hat{f}(\omega).
\\
\widehat{D_af}(\omega)
&=
\int_{-\infty}^\infty (D_af)(t)e^{-i\omega t}\,dt
=
\int_{-\infty}^\infty f(\underbrace{t/a}_{\displaystyle=t'})e^{-i\omega t}\,dt
=
\int_{-\infty}^\infty f(t')e^{-i\omega at'}\,a\,dt'
=
a \hat{f}(a\omega)
\\
\widehat{e^{ibt}f}(\omega)
&=
\int_{-\infty}^\infty e^{ibt}f(t)e^{-i\omega t}\,dt
=
\int_{-\infty}^\infty f(t)e^{-i(\omega -b)t}\,dt
=
\hat{f}(\omega-b)
=
(T_b\hat{f})(\omega).
\end{align*}
Damit sind alle drei Identitäten bewiesen.
\end{proof}

Schreibt man $M_b$ für den Operator, der eine Funktion mit dem
Faktor $e^{ibt}$ multipliziert, also
\[
(M_bf)(t) = e^{ibt}f(t),
\]
dann kann man die Relationen noch etwas kompakter schreiben:
\begin{align*}
\widehat{T_bf}
&=
M_{-b}\hat{f}
\\
\widehat{M_bf}
&=
T_b\hat{f},
\end{align*}
oder mit der Schreibweise $\mathcal{F}f$ für die Fourier-Transformation
\[
\begin{aligned}
\mathcal{F}T_b f &= M_{-b}\mathcal F f
&&\Rightarrow &
\mathcal{F}T_b &= M_{-b}\mathcal{F}
\\
\mathcal{F}M_b f&=T_b\mathcal{F}T_bf
&&\Rightarrow &
\mathcal{F}M_b&=T_b\mathcal{F}T_b
\\
\mathcal{F}D_af&=a D_{1/a} \mathcal F f
&&\Rightarrow &
\mathcal{F}D_a&=a D_{1/a} \mathcal F 
\end{aligned}
\]

\begin{satz}
Für zwei Funktionen $f,g\in L^2(\mathbb R)$ gilt die Plancherel-Formel
\begin{align*}
\langle f,g\rangle
&=
\langle \hat{f},\hat{g}\rangle.
\\
\|f\|&=\|\hat{f}\|
\end{align*}
Die Fourier-Transformation $\mathcal{F}\colon f\mapsto \hat{f}$ ist
also eine Isometrie.
\end{satz}

\begin{satz}
Für differenzierbare Funktionen $f$ gilt
\[
\widehat{f'}(\omega) = i\omega \hat{f}(\omega)
\]
Falls für $g\in L^2(\mathbb R)$ mit $\|tg\|^2<\infty$, dann ist $\hat{g}$
fast überall differenzierbar und es gilt
\[
-\widehat{i t f}(\omega) = \hat{f}'(\omega).
\]
\end{satz}

\begin{proof}[Beweis]
Die Formeln hängen davon ab, ob partiell integiert oder die Ableitung
unter des Integral genommen werden darf:
\begin{align*}
\widehat{f'}(\omega)
&=
\frac{1}{\sqrt{2\pi}}
\int_{-\infty}^\infty \underbrace{f'(t)}_{\uparrow}\underbrace{e^{-i\omega t}}_{\downarrow}\,dt
=
\frac{1}{\sqrt{2\pi}}
\biggl[
f(t)e^{-i\omega t}\,dt
\biggr]_{-\infty}^\infty
+
\frac{1}{\sqrt{2\pi}}
\cdot
i\omega
\int_{-\infty}^\infty f(t)e^{-i\omega t}\,dt
=
i\omega\hat{f}(\omega),
\\
\hat{f}'(\omega)
&=
\frac{d}{d\omega}
\frac{1}{\sqrt{2\pi}} \int_{-\infty}^\infty
f(t) e^{-i\omega t}\,dt
=
\frac{1}{\sqrt{2\pi}} \int_{-\infty}^\infty
-it f(t) e^{-i\omega t}\,dt
=
-\widehat{itf}(\omega)
\qedhere
\end{align*}
\end{proof}

Schreibt man $\partial_t$ für die Ableitung nach $t$ und $\partial_\omega$
für die Ableitung nach $\omega$ und $\mu_{t}$ bzw.~$\mu_{\omega}$ für
die Multiplikation $t$ bzw.~$\omega$, dann kann man die Regeln wieder
kompakter schreiben:
\[
\begin{aligned}
\mathcal{F}\partial_t f &= i\mu_{\omega}\mathcal{F}f
&&\Rightarrow&
\mathcal{F}\partial_t &= i\mu_{\omega}\mathcal{F}
\\
-i\mathcal{F}\mu_t f &= \partial_\omega\mathcal{F} f
&&\Rightarrow&
-i\mathcal{F}\mu_t &= \partial_\omega\mathcal{F}.
\end{aligned}
\]








%
% windowed.tex
%
% (c) 2019 Prof Dr Andreas Müller, Hochschule Rapperswil
%
\section{Gefensterte Fourier-Transformation
\label{section:gefenstert}}
\rhead{Gefensterte Fourier-Transformation}
\begin{figure}
\centering
\includegraphics{chapters/2-fourier/images/wft.pdf}
\caption{Aufteilung der Frequenz-Zeit-Ebene für die gefensterte
Fourier-Transformation.
\label{wft:ftplane}}
\end{figure}
Ändert man eine $2\pi$-periodische Funktion in der Umgebung eines Punktes,
dann ändern praktisch alle Fourier-Koeffizienten.
Da die Fourier-Koeffizienten linear von der Funktion abhängen,
reicht es, die Koeffizienten der Änderung zu bestimmen.
Für eine Dirac $\delta$-Distribution im Punkt $t_0$ sind die Koeffizienten
\[
c_k
=
\frac{1}{2\pi} \int_0^{2\pi} \delta(t-t_0) e^{ikt}\,dt
=
\frac{1}{2\pi} e^{ikt_0},
\]
insbesondere ändert jeder Koeffizient um eine Zahl vom Betrag $1/2\pi$.

Die Position einer Störung äussert sich also nur in der Phase, nicht
im Betrag der Störung.
Man könnte sagen, die Information über die Störung ist im Amplitudenspektrum
vollständig delokalisiert.
Diese Eigenschaft der Fourier-Reihen bedeutet, dass transiente Ereignisse
nur sehr beschränkt mit Fourier-Reihen analysiert werden können.

Noch drastischer ist die Situation für die Fourier-Transformation.
Eine $\delta$-Störung in einem Punkt $t_0$ irgendwo auf der reellen Achse
ändert die Transformierte um
\[
\mathcal{\delta_{t_0}}(\omega)
=
\frac{1}{\sqrt{2\pi}} \int_{-\infty}^\infty \delta(t-t_0) e^{-i\omega t}\,dt
=
\frac1{\sqrt{2\pi}} e^{-i\omega t_0}.
\]
Auch in diesem Fall schlägt sich der Ort der Störung nur in der Phase nieder,
nicht in der Amplitude.
Das Amptlitudenspektrum sagt also nichts über die Position der
Störung aus.

Die {\em gefensterte Fouriertransformation} ermöglich, etwas Orts-Information
in die Fourier-Koeffizienten zu rechnen.
Zu diesem Zweck wird das Definitionsgebiet in kleinere Intervalle gleicher
Grösser unterteilt.
In jedem Teil-Intervall kann das vorgegebene Signal in eine Fourier-Reihe
entwickelt werden.
Die so ermittelten Fourier-Koeffizienten können sich dann nur auf den Teil
der Funktion im Teilintervall bezienen.
Eine Störung im Punkt $t_0$ wirkt sich nur auf die Fourier-Koeffizienten
für das Intervall aus, welches $t_0$ enthält.

Die Unterteilung in kleinere Intervalle erhöht aber auch die Frequenz
der Analysefunktionen.
Wird das Intervall $[0,2\pi]$ in $n$ Intervalle unterteilt, dann wird in
jedem Teilintervall mit den Funktionen $e^{inkt}$ für $k\in\mathbb Z$
analysiert.
Die Koeffizienten in den Teilintervallen repräsentieren also ein $n$-mal
grössers Frequenz-Intervall, wie dies in Abbildung~\ref{wft:ftplane}
dargestellt ist.

Die Unterteilung in Teilintervalle löst zwar das Problem der Lokalisierung,
ist aber aus anderen Gründen nicht optimal.
Die Berechnung der Fourier-Reihe für das Teilintervall $[2\pi k/n, 2\pi(k+1)/n]$
läuft darauf hinaus, die Fourier-Reihe im Intervall $[0,2\pi]$ der Funktion
$f(t) \chi_{[2\pi k/n, 2\pi(k+1)/n]}(t)$
zu berechnen, wobei 
\[
\chi_{[2\pi k/n, 2\pi(k+1)/n]}(t)
=
\begin{cases}
1&\qquad t \in [2\pi k/n, 2\pi(k+1)/n]\\
0&\qquad\text{sonst}
\end{cases}
\]
die charakteristische Funktion des Intervalls $[2\pi k/n, 2\pi(k+1)/n]$
ist.
Die Schwierigkeiten rühren daher, dass die charakteristische
Funktion Sprünge aufweist.
Bessere Resultate kann man daher erreichen, wenn man statt einer
charakterisischen Funktion eines Intervalls eine glatte Funktion
verwendet, deren Träger das Intervall enthält.
Die Wahl dieser sogenannten Fensterfunktion beeinflusst die
Analyse-Koeffizienten, bei sorgfältiger Wahl können aber mindestens
die Glattheits-Eigenschaften der analysierten Funktion und die zugehörigen
Eigenschaften der Koeffizienten erhalten werden.

Unbefriedigend bleibt an der gefensterten Fourier-Transformation aber,
dass für jede Frequenz die gleiche Fensterbreite verwendet wird.
Selbst wenn man also zum Beispiel die Sampling-Rate erhöht und damit
die Auflösung verbessert, kann man Transienten in den
Funktionskoeffizienten trotzdem nur mit der Auflösung der Fensterbreite
lokalisieren.
Eine mögliche Verbesserung besteht darin, die Fensterbreite mit zunehmender
Frequenz zu verkleinern.
Genau dies ist, was die Wavelet-Transformation erreicht, die in
Kapitel~\ref{chapter:haar-wavelet} am Beispiel des Haar-Wavelets und in
Kapitel~\ref{chapter:cwt} in voller Allgemeinheit eingeführt wird.


%
% heisenberg.tex
%
% (c) 2019 Prof Dr Andreas Müller, Hochschule Rapperswil
%
\section{Heisenbergsche Unschärfe-Relation
\label{section:heisenberg}}
\rhead{Heisenbergsche Unschärferelation}
Die Fourier-Transformation eines in der Zeit gut lokalisierten
Signals $f(t)$ zeigt, wie gut das selbe Signal im Frequenzraum
lokalisiert ist.
Um diese Idee zu quantifizieren brauchen wir zunächst ein Mass für
die Lokalisierung von $f(t)$ bzw.~$\hat{f}(\omega)$.
Wäre $f$ eine Wahrscheinlichkeitsdichte, dann wäre die Varianz ein
naheliegendes Mass dafür.
Die Funktion darf aber nicht $\le 0$ sein. Man könnte das Problem korrigeren, indem man stattdessen $|f(t)|^2$ verwendet.
Die Masse für die Lokalisierung in Zeit und Frequenz sind daher
\begin{align*}
\int_{-\infty}^\infty
t^2 \, |f(t)|^2\,dt
&=
\int_{-\infty}^\infty
|t\,f(t)|^2\,dt
=
\|tf\|^2
\qquad\text{bzw.}
\\
\int_{-\infty}^\infty
\omega^2\,|\hat{f}(\omega)|^2\,d\omega
&=
\int_{-\infty}^\infty
|\omega\,\hat{f}(\omega)|^2\,d\omega
=
\|\omega \hat{f}\|^2.
\end{align*}
Je grösser diese Normen werden, desto schlechter ist das Signal in der
entsprechenden Variable lokalisiert.

Die Heisenbergsche Unschärferelation besagt, dass ein Signal nicht
in Zeit und Frequenz gleichzeit beliebig gut lokalisiert sein kann.
Sie drückt dies dadurch aus, dass das Produkt der beiden Normen 
eine untere Schranke hat.

\begin{satz}[Heisenberg]
\label{satz:heisenberg}
Ist $f\in L^2(\mathbb R)$ derart, dass 
$\|tf\|<\infty$ und $\|\omega\hat{f}\|<\infty$, dann gilt
\begin{equation}
\| tf \| \cdot \| \omega \hat{f}\| \ge \frac12\| f\|^2.
\label{heisenberg:gleichung}
\end{equation}
Die untere Schranke wir erreicht für Gauss-Funktionen, also Funktionen
der Form $De^{-t^2/2\sigma^2}$.
\end{satz}

\begin{proof}[Beweis]
Die Voraussetzungen garantieren, dass die Fourier-Transformation existiert.
Ausserdem bedeutet $\|\omega\hat{f}\|<\infty$, dass $\omega\hat{f}(\omega)$ 
eine $L^2$-Funktion ist.
Die Rechenregeln für die Fourier-Transformation besagen, dass 
$f$ fast überall differenzierbar sein muss, denn es gilt
\[
i\omega \hat{f}(\omega) = \widehat {f'}(\omega).
\]
Damit kann man die Lokalisierung von $\hat{f}$ durch $f$ ausdrücken:
\begin{align*}
\|\omega \hat{f}\|^2
&=
\int_{-\infty}^\infty |i\omega \hat{f}(\omega)|^2\,d\omega
=
\int_{-\infty}^\infty |\widehat{f'}(\omega)|^2\,d\omega
\\
&=
\int_{-\infty}^\infty f'(t)|^2\,dt.
\end{align*}
Jetzt kann man die Cauchy-Schwarz-Ungleichung auf die Funktionen $f'$ und $tf$
anwenden:
\begin{align*}
\|tf\| \cdot \| f'\|
&\ge
| \operatorname{Re}\langle tf,f'\rangle |
=
\biggl|
\int_{-\infty}^\infty
\frac 12\bigl( tf(t)\bar{f}'(t)  + t\bar{f}(t)f'(t)\bigr)\,dt
\biggr|
\\
&=
\biggl|
\int_{-\infty}^\infty t\cdot \frac{d}{dt}\biggl(\frac12|f(t)|^2\biggr)\,dt
\biggr|
=
\biggl|
\underbrace{
\biggl[
\frac12 t\,|f(t)|^2
\biggr]_{-\infty}^{\infty}
}_{\displaystyle=0}
-
\frac12\underbrace{\int_{-\infty}^\infty |f(t)|^2\,dt}_{\displaystyle=\|f\|^2}
\biggr|
\\
&=
\frac12 \|f\|^2.
\end{align*}
Damit ist die Ungleichung~\eqref{heisenberg:gleichung} bewiesen.

Gleichheit wird erreicht, wenn die beiden Faktoren in der 
Ungleichung~\eqref{heisenberg:gleichung} linear abhängig sein.
Es gibt also einen Proportionalitätsfaktor $c\in\mathbb C$ derart,
dass
\[
\begin{aligned}
&&
tf(t)&=cf'(t)
\\
&\Rightarrow&
\frac{t}{c}&=\frac{d}{dt}\log f(t)
\\
&\Rightarrow&
\frac{t^2}{2c}&=\log f(t) + C
\\
&\Rightarrow&
f(t)&=De^{t^2/2c}.
\end{aligned}
\]
Da die Funktion $f$ in $L^2(\mathbb R)$ sein muss, kommen nur negative
Konstanten $c$ in Frage, wir bezeichnen sie mit $c=-\sigma^2$.
Gleichheit in der Ungleichung~\eqref{heisenberg:gleichung} tritt also
genau dann auf, wenn $f$ die Form
\[
f(t) = D e^{-\frac{t^2}{2\sigma^2}}
\]
hat.
\end{proof}

In der Nachrichtentechnik ist die Bezieung von Satz~\ref{satz:heisenberg}
\index{Nachrichtentechnik}%
\index{Küpfmüllersche Unbestimmtheitsrelation}
auch als die Küpfmüllersche Unbestimmtheitsrelation bekannt nach
Karl Küpfmüller, der sie 1924 formulierte.
\index{Küpfmüller, Karl}%
Die Erkenntnis von Werner Heisenberg war, dass diese Relation in
der Quantenmechanik eine die grundsätzliche Unmöglichkeit zur Folge
hat, Ort und Impuls eines Teilchens gleichzeitig beliebig genau
zu kennen.
\index{Heisenberg, Werner}%


