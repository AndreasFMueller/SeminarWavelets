%
% l2.tex -- L2 Räume 
%
% (c) 2019 Prof Dr Andreas Müller, Hochschule Rapperswil
%
\section{Der Hilbertraum $L^2$
\label{section:l2}}
\rhead{Der Hilbertraum $L^2$}

\subsection{Definition des Skalarproduktes}

\subsection{Lebesgue-Integral und Vollständigkeit}

\subsection{$L^2$ und $L^1$}
Eine integrierbare Funktion ist nicht automatisch
quadratintegrierbar.
Wenn eine Funktion gerade noch langsam genug anwächst, dass ihr
Integral existiert, kann das Quadrat der Funktion bereits zu schnell
anwachsen, so dass das Quadrat nicht mehr integrierbar ist.

\begin{beispiel}
Auf dem Interval $I=[0,1]$ ist die Funktion
\[
f\colon I\to \mathbb R: t\mapsto \begin{cases} 0&\qquad t=0\\
t^{-\alpha}&\qquad t > 0
\end{cases}
\]
gegeben, der genaue Wert von $\alpha$ wird später festgelegt.
Die Integrale von $f$ und $f^2$ sind
\begin{align*}
\int_0^1 |f(t)|\,dt
&=
\int_0^1 t^{-\alpha}\,dt
=
\biggl[\frac{1}{1-\alpha}t^{1-\alpha}\biggr]_0^1
=
\begin{cases}
\frac{1}{1-\alpha}&\qquad \alpha < 1\\
\infty&\qquad \alpha \ge 1
\end{cases}
\\
\int_0^1|f(t)|^2\,dt
&=
\int_0^1 t^{-2\alpha}\,dt
=
\begin{cases}
\frac{1}{1-2\alpha}&\qquad \alpha < \frac12\\
\infty&\qquad \alpha \ge \frac12
\end{cases}
\end{align*}
Für $\frac12<\alpha<1$ tritt also die Situation ein, dass das Integral
von $f$ existiert, das von $f^2$ aber nicht\footnote{In der Rechnung in
diesem und im nächsten Beispiel wurde der Fall $\alpha=1$ nicht sorgfältig
nachgerechnet.
In diesem Fall ist die Stammfunktion nämlich $\log|t|$, der jedoch
auch für $t\to 0$ und $t\infty$ divergiert.
Die Behauptungen sind daher auch in diesem Fall richtig.}.
\end{beispiel}

Die umgekehrte Situation kann für Funktionen auf $\mathbb R$ eintreten,
die zu langsam abfallen, um integrierbar zu sein. 
Da das Quadrieren kleine Werte noch kleiner macht, kann das Quadrat
der Funktion schnell genug abfallen, so dass es integrierbar ist.
Eine solche Funktion ist in $L^2(\mathbb R)$, aber nicht in $L^2(\mathbb R)$.

\begin{beispiel}
Auf dem Interval $J=[1,\infty)$ ist die Funktion
$f(t)=t^{-\alpha}$ gegeben.
Die Integrale von $f$ und $f^2$ sind
\begin{align*}
\int_1^\infty |f(t)|\,dt
&=
\int_1^\infty t^{-\alpha}\,dt
=
\biggl[\frac{t^{1-\alpha}}{1-\alpha}\biggr]_1^\infty
=
\begin{cases}
\frac{1}{\alpha-1}&\qquad \alpha > 1\\
\infty &\qquad \alpha \le 1
\end{cases}
\\
\int_1^\infty |f(t)|^2\,dt
&=
\int_1^\infty t^{-2\alpha}\,dt
=
\biggl[\frac{t^{1-2\alpha}}{1-2\alpha}\biggr]_1^\infty
=
\begin{cases}
\frac{1}{2\alpha-1}&\qquad \alpha > \frac12\\
\infty &\qquad \alpha \le \frac12
\end{cases}
\end{align*}
Man liest daraus ab, dass für $\frac12<\alpha < 1$ die Funktion $f$ zwar
in $L^2([0,\infty))$, nicht aber in $L^1([1,\infty))$ ist.
\end{beispiel}

Der Fall des letzten Beispiels kann vermieden werden, wenn man den
Definitionsbereich der Funktion auf ein kompaktes Interval beschränkt.
Dann folgt, der folgende Satz.

\begin{satz}
\label{satz:l2inl1}
Ist $I$ ein kompaktes Interval, dann ist $L^2(I)\subset L^1(I)$, oder:
jede quadratintegrierbare Funktion auf einem kompakten Interval ist 
integrierbar.
\end{satz}

\begin{proof}[Beweis]
Wir zerlegen die Funktion $f$ in eine Summe $f=f_-+f_+$. 
Dabei setzen wir:
\begin{align*}
f_+(t) &= 
\begin{cases}
f(t)&\qquad |f(t)| > 1\\
0   &\qquad\text{sonst}
\end{cases}
&&\text{und}
&
f_(t)
&=
\begin{cases}
0   &\qquad |f(t)| > 1\\
f(t)&\qquad f(t)\le 1
\end{cases}
\end{align*}
Die Teilfunktion $f_+$ sammelt also all jene Werte, die beim Quadrieren
grösser werden, die Teilfunktion $f_-$ dagegen diejenigen Werte, die
beim Quadrieren kleiner werden.
Wir schätzen jetzt das Integral von $|f|$ ab, indem wir es in die Summanden
$f_+$ und $f_-$ zerlegen:
\begin{align*}
\int_I |f(t)|\,dt
&=
\int_I |f_+(t)|\,dt + \int_I |f_-(t)|\,dt
\le
\int_I |f_+(t)|^2\,dt + \int_I 1\,dt
\le 
\int_I |f(t)|^2\,dt + |I|
\le \|f\|^2 + |I|.
\end{align*}
Da nach Voraussetzung $f^2$ integrierbar ist und das Interval $I$ kompakt ist
und damit beschränkte Länge $|I|$ länge hat, ist das Integral von $|f|$
ebenfalls beschränkt.
\end{proof}

