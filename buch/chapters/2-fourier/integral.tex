%
% integral.tex -- Fourier-Integral
%
% (c) 2019 Prof Dr Andreas Müller, Hochschule Rapperswil
%
\section{Fourier-Integral
\label{section:fourier-integral}}
\rhead{Fourier-Integral}
Für Funktionen $f\in L^2(\mathbb R)$ ist die Fourier-Transformation
\[
\hat{f}(\omega)
=
\frac{1}{\sqrt{2\pi}}
\int_{-\infty}^\infty f(t) e^{-i\omega t}\,dt
\]
wohldefiniert.
Wir schreiben die Fourier-Transformation auch
\[
\mathcal{F}\colon f\mapsto \hat{f}.
\]
Sie verallgemeinert die Eigenschaften der Fourier-Koeffizienten $c_k$
für Funktionen auf $\mathbb R$.
Für die Rechungen in den folgenden Kapiteln stellen wir hier die
wichtigsen Formeln für die Fourier-Transformation zusammen.

\begin{satz}
Für $\hat{f}\in L^2(\mathbb R)$ gilt die Umkehrformel
\[
f(t)
=
\frac{1}{\sqrt{2\pi}}
\int_{-\infty}^{\infty} \hat{f}(\omega)e^{i\omega t}\,d\omega.
\]
\end{satz}

Die Fourier-Transformation ist also eine invertierbare lineare Abbildung,
$\mathcal{F}^{-1}\!\left\lbrace\hat{f}\right\rbrace = f$.

\begin{satz}
Die Fourier-Transformation $\mathcal F\colon f\mapsto f$ ist eine linear
Abbildung, die sich mit Translation, Dilatation und Ableitung wie folgt
verträgt:
\begin{align*}
\widehat{T_bf}(\omega)
&=
e^{-i\omega b}\hat{f}(\omega).
\\
\widehat{D_af}(\omega)
&=
a \hat{f}(a\omega)
\\
\widehat{e^{ibt}f}(\omega)
&=
(T_b\hat{f})(\omega).
\end{align*}
\end{satz}

\begin{proof}[Beweis]
Durch direkte Rechnung finden wir:
\begin{align*}
\widehat{T_bf}(\omega)
&=
\int_{-\infty}^{\infty} (T_bf)(t)e^{-i\omega t}\,dt
=
\int_{-\infty}^{\infty} f(\underbrace{t-b}_{\displaystyle=t'})e^{-i\omega t}\,dt
=
\int_{-\infty}^{\infty} f(t')e^{-i\omega(t'+b)}\,dt'
\\
&=
e^{-i\omega b}
\int_{-\infty}^{\infty} f(t')e^{-i\omega t'}\,dt'
=
e^{-i\omega b}\hat{f}(\omega).
\\
\widehat{D_af}(\omega)
&=
\int_{-\infty}^\infty (D_af)(t)e^{-i\omega t}\,dt
=
\int_{-\infty}^\infty f(\underbrace{t/a}_{\displaystyle=t'})e^{-i\omega t}\,dt
=
\int_{-\infty}^\infty f(t')e^{-i\omega at'}\,a\,dt'
=
a \hat{f}(a\omega)
\\
\widehat{e^{ibt}f}(\omega)
&=
\int_{-\infty}^\infty e^{ibt}f(t)e^{-i\omega t}\,dt
=
\int_{-\infty}^\infty f(t)e^{-i(\omega -b)t}\,dt
=
\hat{f}(\omega-b)
=
(T_b\hat{f})(\omega).
\end{align*}
Damit sind alle drei Identitäten bewiesen.
\end{proof}

Schreibt man $M_b$ für den Operator, der eine Funktion mit dem
Faktor $e^{ibt}$ multipliziert, also
\[
(M_bf)(t) = e^{ibt}f(t),
\]
dann kann man die Relationen noch etwas kompakter schreiben:
\begin{align*}
\widehat{T_bf}
&=
M_{-b}\hat{f}
\\
\widehat{M_bf}
&=
T_b\hat{f},
\end{align*}
oder mit der Schreibweise $\mathcal{F}f$ für die Fourier-Transformation
\[
\begin{aligned}
\mathcal{F}T_b f &= M_{-b}\mathcal F f
&&\Rightarrow &
\mathcal{F}T_b &= M_{-b}\mathcal{F}
\\
\mathcal{F}M_b f&=T_b\mathcal{F}T_bf
&&\Rightarrow &
\mathcal{F}M_b&=T_b\mathcal{F}T_b
\\
\mathcal{F}D_af&=a D_{1/a} \mathcal F f
&&\Rightarrow &
\mathcal{F}D_a&=a D_{1/a} \mathcal F 
\end{aligned}
\]

\begin{satz}
Für zwei Funktionen $f,g\in L^2(\mathbb R)$ gilt die Plancherel-Formel
\begin{align*}
\langle f,g\rangle
&=
\langle \hat{f},\hat{g}\rangle.
\\
\|f\|&=\|\hat{f}\|
\end{align*}
Die Fourier-Transformation $\mathcal{F}\colon f\mapsto \hat{f}$ ist
also eine Isometrie.
\end{satz}

\begin{satz}
Für differenzierbare Funktionen $f$ gilt
\[
\widehat{f'}(\omega) = i\omega \hat{f}(\omega)
\]
Falls für $g\in L^2(\mathbb R)$ mit $\|tg\|^2<\infty$, dann ist $\hat{g}$
fast überall differenzierbar und es gilt
\[
-\widehat{i t f}(\omega) = \hat{f}'(\omega).
\]
\end{satz}

\begin{proof}[Beweis]
Die Formeln hängen davon ab, ob partiell integiert oder die Ableitung
unter des Integral genommen werden darf:
\begin{align*}
\widehat{f'}(\omega)
&=
\frac{1}{\sqrt{2\pi}}
\int_{-\infty}^\infty \underbrace{f'(t)}_{\uparrow}\underbrace{e^{-i\omega t}}_{\downarrow}\,dt
=
\frac{1}{\sqrt{2\pi}}
\biggl[
f(t)e^{-i\omega t}\,dt
\biggr]_{-\infty}^\infty
+
\frac{1}{\sqrt{2\pi}}
\cdot
i\omega
\int_{-\infty}^\infty f(t)e^{-i\omega t}\,dt
=
i\omega\hat{f}(\omega),
\\
\hat{f}'(\omega)
&=
\frac{d}{d\omega}
\frac{1}{\sqrt{2\pi}} \int_{-\infty}^\infty
f(t) e^{-i\omega t}\,dt
=
\frac{1}{\sqrt{2\pi}} \int_{-\infty}^\infty
-it f(t) e^{-i\omega t}\,dt
=
-\widehat{itf}(\omega)
\qedhere
\end{align*}
\end{proof}

Schreibt man $\partial_t$ für die Ableitung nach $t$ und $\partial_\omega$
für die Ableitung nach $\omega$ und $\mu_{t}$ bzw.~$\mu_{\omega}$ für
die Multiplikation $t$ bzw.~$\omega$, dann kann man die Regeln wieder
kompakter schreiben:
\[
\begin{aligned}
\mathcal{F}\partial_t f &= i\mu_{\omega}\mathcal{F}f
&&\Rightarrow&
\mathcal{F}\partial_t &= i\mu_{\omega}\mathcal{F}
\\
-i\mathcal{F}\mu_t f &= \partial_\omega\mathcal{F} f
&&\Rightarrow&
-i\mathcal{F}\mu_t &= \partial_\omega\mathcal{F}.
\end{aligned}
\]







