%
% heisenberg.tex
%
% (c) 2019 Prof Dr Andreas Müller, Hochschule Rapperswil
%
\section{Heisenbergsche Unschärferelation
\label{section:heisenberg}}
\rhead{Heisenbergsche Unschärferelation}
Die Fourier-Transformation eines in der Zeit gut lokalisierten
Signals $f(t)$ zeigt, wie gut das selbe Signal im Frequenzraum
lokalisiert ist.
Um diese Idee zu quantifizieren brauchen wir zunächst ein Mass für
die Lokalisierung von $f(t)$ bzw.~$\hat{f}(\omega)$.
Wäre $f$ eine Wahrscheinlichkeitsdichte, dann wäre die Varianz ein
naheliegendes Mass dafür.
Die Funktion darf aber nicht $\le 0$ sein. Man könnte das Problem korrigeren, indem man stattdessen $|f(t)|^2$ verwendet.
Die Masse für die Lokalisierung in Zeit und Frequenz sind daher
\begin{align*}
\int_{-\infty}^\infty
t^2 \, |f(t)|^2\,dt
&=
\int_{-\infty}^\infty
|t\,f(t)|^2\,dt
=
\|tf\|^2
\qquad\text{bzw.}
\\
\int_{-\infty}^\infty
\omega^2\,|\hat{f}(\omega)|^2\,d\omega
&=
\int_{-\infty}^\infty
|\omega\,\hat{f}(\omega)|^2\,d\omega
=
\|\omega \hat{f}\|^2.
\end{align*}
Je grösser diese Normen werden, desto schlechter ist das Signal in der
entsprechenden Variable lokalisiert.

Die Heisenbergsche Unschärferelation besagt, dass ein Signal nicht
in Zeit und Frequenz gleichzeit beliebig gut lokalisiert sein kann.
Sie drückt dies dadurch aus, dass das Produkt der beiden Normen 
eine untere Schranke hat.

\begin{satz}[Heisenberg]
\label{satz:heisenberg}
Ist $f\in L^2(\mathbb R)$ derart, dass 
$\|tf\|<\infty$ und $\|\omega\hat{f}\|<\infty$, dann gilt
\begin{equation}
\| tf \| \cdot \| \omega \hat{f}\| \ge \frac12\| f\|^2.
\label{heisenberg:gleichung}
\end{equation}
Die untere Schranke wir erreicht für Gauss-Funktionen, also Funktionen
der Form $De^{-t^2/2\sigma^2}$.
\end{satz}

\begin{proof}[Beweis]
Die Voraussetzungen garantieren, dass die Fourier-Transformation existiert.
Ausserdem bedeutet $\|\omega\hat{f}\|<\infty$, dass $\omega\hat{f}(\omega)$ 
eine $L^2$-Funktion ist.
Die Rechenregeln für die Fourier-Transformation besagen, dass 
$f$ fast überall differenzierbar sein muss, denn es gilt
\[
i\omega \hat{f}(\omega) = \widehat {f'}(\omega).
\]
Damit kann man die Lokalisierung von $\hat{f}$ durch $f$ ausdrücken:
\begin{align*}
\|\omega \hat{f}\|^2
&=
\int_{-\infty}^\infty |i\omega \hat{f}(\omega)|^2\,d\omega
=
\int_{-\infty}^\infty |\widehat{f'}(\omega)|^2\,d\omega
\\
&=
\int_{-\infty}^\infty f'(t)|^2\,dt.
\end{align*}
Jetzt kann man die Cauchy-Schwarz-Ungleichung auf die Funktionen $f'$ und $tf$
anwenden:
\begin{align*}
\|tf\| \cdot \| f'\|
&\ge
| \operatorname{Re}\langle tf,f'\rangle |
=
\biggl|
\int_{-\infty}^\infty
\frac 12\bigl( tf(t)\bar{f}'(t)  + t\bar{f}(t)f'(t)\bigr)\,dt
\biggr|
\\
&=
\biggl|
\int_{-\infty}^\infty t\cdot \frac{d}{dt}\biggl(\frac12|f(t)|^2\biggr)\,dt
\biggr|
=
\biggl|
\underbrace{
\biggl[
\frac12 t\,|f(t)|^2
\biggr]_{-\infty}^{\infty}
}_{\displaystyle=0}
-
\frac12\underbrace{\int_{-\infty}^\infty |f(t)|^2\,dt}_{\displaystyle=\|f\|^2}
\biggr|
\\
&=
\frac12 \|f\|^2.
\end{align*}
Damit ist die Ungleichung~\eqref{heisenberg:gleichung} bewiesen.

Gleichheit wird erreicht, wenn die beiden Faktoren in der 
Ungleichung~\eqref{heisenberg:gleichung} linear abhängig sein.
Es gibt also einen Proportionalitätsfaktor $c\in\mathbb C$ derart,
dass
\[
\begin{aligned}
&&
tf(t)&=cf'(t)
\\
&\Rightarrow&
\frac{t}{c}&=\frac{d}{dt}\log f(t)
\\
&\Rightarrow&
\frac{t^2}{2c}&=\log f(t) + C
\\
&\Rightarrow&
f(t)&=De^{t^2/2c}.
\end{aligned}
\]
Da die Funktion $f$ in $L^2(\mathbb R)$ sein muss, kommen nur negative
Konstanten $c$ in Frage, wir bezeichnen sie mit $c=-\sigma^2$.
Gleichheit in der Ungleichung~\eqref{heisenberg:gleichung} tritt also
genau dann auf, wenn $f$ die Form
\[
f(t) = D e^{-\frac{t^2}{2\sigma^2}}
\]
hat.
\end{proof}

In der Nachrichtentechnik ist die Bezieung von Satz~\ref{satz:heisenberg}
\index{Nachrichtentechnik}%
\index{Küpfmüllersche Unbestimmtheitsrelation}
auch als die Küpfmüllersche Unbestimmtheitsrelation bekannt nach
Karl Küpfmüller, der sie 1924 formulierte.
\index{Küpfmüller, Karl}%
Die Erkenntnis von Werner Heisenberg war, dass diese Relation in
der Quantenmechanik eine die grundsätzliche Unmöglichkeit zur Folge
hat, Ort und Impuls eines Teilchens gleichzeitig beliebig genau
zu kennen.
\index{Heisenberg, Werner}%

