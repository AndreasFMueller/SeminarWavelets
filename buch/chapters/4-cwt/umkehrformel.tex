%
% umkehrformel.tex
%
% (c) 2019 Prof Dr Andreas Müller, Hochschule Rapperswil
%
\section{Umkehrformel%
\label{section:cwt:umkehrformel}}
\rhead{Umkehrformel}
Die Plancherel-Formel von Satz~\ref{satz:wplancherel} zeigt, dass die
Wavelettransformation~\eqref{cwt:definition:eq} bis auf den Faktor
$C_{\psi}$ eine Isometrie ist.
In Abschnitt~\ref{section:plancherel} wurde dies die Plancherel-Eigenschaft
der stetigen Wavelet-Transformation genannt und gezeigt, dass daraus
eine einfache Umkehrformel folgt.
Zu jeder Wavelet-Transformierten sollten daher die ursprüngliche
Funktionen wiedergewonnen werden können.
Ziel dieses Abschnitts ist die unabhängige Konstruktion einer Umkehrformel, 
die $f$ aus $\mathcal{W}f$ wiedergewinnt.
Es wird sich herausstellen, dass diese Umkehrformel genau der Formel
entspricht, die in Satz~\ref{satz:plancherel-prinzip} ermittelt wurde.

\begin{satz}
Unter geeigneten Stetitkeitsvoraussetzungen an die Funktion $f$ gilt
\begin{equation}
f(t) = \frac{1}{C_{\psi}}\int_{\mathbb R^*}\int_{-\infty}^\infty
\mathcal{W}f(a,b) \psi_{a,b}(t)
\,db\,\frac{da}{|a|^2}
\label{cwt:umkehr}
\end{equation}
\end{satz}

\begin{proof}[Beweis]
Wir haben zu zeigen, dass die Funktion
\begin{equation}
\tilde{f}(t) = \frac{1}{C_{\psi}}\int_{\mathbb R^*}\int_{-\infty}^\infty
\mathcal{W}f(a,b) \psi_{a,b}(t)
\,db\,\frac{da}{|a|^2}
\end{equation}
wohldefiniert ist, in $L^2(\mathbb R)$ liegt  und mit $f$ übereinstimmt.

Wir nehmen zunächst an, dass $\tilde{f}$ tatsächlich wohldefiniert
ist und eine $L^2$-Funktion ist.
Um zu zeigen, dass $\tilde{f}$ mit $f$ übereinstimmt, genügt es nach
dem Satz von Riesz~\ref{geometrie:satz:riesz}, zu zeigen, das
$\tilde{f}$ und $f$ das selbe Skalarprodukt mit beliebigen Funktionen
$g\in L^2$ hat.
Durch Vertauschung der Integrationsreihenfolge in~\eqref{cwt:umkehr:beweis1}
und Verwendung der Plancherel-Formel in~\eqref{cwt:umkehr:beweis2}
finden wir
\begin{align}
\langle \tilde{f},g\rangle
&=
\int_{-\infty}^\infty
\frac1{C_{\psi}} \int_{\mathbb R^*}\int_{-\infty}^\infty
\mathcal{W}f(a,b)
\psi_{a,b}(t)
\,db\,\frac{da}{|a|^2}
\bar{g}(t)
\,dt
\notag
\\
&=
\frac1{C_{\psi}} \int_{\mathbb R^*}\int_{-\infty}^\infty
\mathcal{W}f(a,b)
\int_{-\infty}^\infty
\bar{g}(t)
\psi_{a,b}(t)
\,dt
\,db\,\frac{da}{|a|^2}
\label{cwt:umkehr:beweis1}
\\
&=
\frac1{C_{\psi}} \int_{\mathbb R^*}\int_{-\infty}^\infty
\mathcal{W}f(a,b)
\underbrace{
\overline{
\int_{-\infty}^\infty
g(t)
\bar{\psi}_{a,b}(t)
\,dt
}}_{\displaystyle\mathcal{W}g(a,b)}
\,db\,\frac{da}{|a|^2}
\notag
\\
&=
\frac{1}{C_{\psi}}
\langle \mathcal{W}f, \mathcal{W}g \rangle_H
=
\langle f, g\rangle.
\label{cwt:umkehr:beweis2}
\end{align}
Die Funktion $\tilde{f}$ hat mit allen Funktion $g$ das gleiche Skalarprodukt
wie $f$ und daher
\[
\langle \tilde{f},g\rangle = \langle f,g\rangle
\quad\Rightarrow\quad
\langle \tilde{f}-f,g\rangle = 0.
\]
Für $g=\tilde{f}-f$ findet man
\[
=
\langle \tilde{f}-f,\tilde{f}-f\rangle
=
\|\tilde{f}-f\|^2
\quad\Rightarrow\quad
\tilde{f}=f.
\]

Die bisherigen Überlegungen haben nur gezeigt, dass die
Formel~\label{cwt:umkehr} eine sogenannte schwache Lösung des Umkehrproblems
liefern kann.
Der Nachweis, dass die Funktion $\tilde{f}$ existiert,
ist etwas technisch und kann
in \cite{buch:daubechies} nachgelesen werden.
\end{proof}



