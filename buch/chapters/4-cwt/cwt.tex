%
% cwt.tex
%
% (c) 2019 Prof Dr Andreas Müller, Hochschule Rapperswil
%
\section{Stetige Wavelet-Transformation
\label{sextion:cwt}}
\rhead{Stetige Wavelet-Transformation}
Ein Wavelet soll nun dazu verwendet werden, ein Signal $f(t)$
abzutasten.
Da das Wavelet lokalisiert ist, müssen wir es der $t$-Achse entlang
verschieben, um jeden Abschnitt des Signals sinnvoll abtasten zu
können.
Da das Wavelet auch im Frequenzbereich lokalisiert ist, müssen wir
es ausserdem skalieren, um sowohl kurzwellige wie auch langwellige
Details des Signals zu erfassen.

Sei also $\psi$ ein Wavelet, also eine Funktion, die die
Zulässigkeitsbedingung~\eqref{cwt:zulaessig} erfüllt\footnote{Die nachfolgenden
Definitionen sind zum Teil auch sinnvoll, wenn die Zulässigkeitsbedingung
nicht erfüllt ist, doch ist die so entstehende Transformation nicht unbedingt
stetig oder umkehrbar.}.
Wir setzen daher
\[
\psi_{a,b}(t) = \frac{1}{\sqrt{|a|}} \psi\biggl(\frac{t-b}a\biggr).
\]
Den Spezialfall $b=0$ kürzen wir $\psi_a = \psi_{a,0}$ ab.

\begin{lemma}
Die verschobenen und gestreckten Kopien $\psi_{a,b}$ von $\psi$ haben alle
die gleiche Norm: $\|\psi_{a,b}\|=1$.
\end{lemma}

\begin{proof}[Beweis]
Zunächst ist klar, dass die Verschiebung um $b$ die Norm nicht ändert.
Ebenso kann man sich überlegen, dass ein negatives Vorzeichen von $a$
ausser der Skalierung um den Betrag $|a|$ die Funktion am Nullpunkt spiegelt,
was die Norm ebenfalls unverändert lässt.
Es ist also nur genauer zu untersuchen, ob sich die Norm mit $a>0$ ändern
kann.
Dazu berechnen wir
\begin{align*}
\| \psi_{a,b} \|
&=
\| \psi_a \|
=
\int_{-\infty}^\infty |\psi_a(t)|^2 \, dt
=
\int_{-\infty}^\infty
\biggl|\psi\biggl(\frac{t}{a}\biggr)\biggr|^2\cdot \frac1{|a|}
\,dt
\\
\intertext{Darin substituieren wir $s=t/a$ und erhalten}
&=
\int_{-\infty}^\infty |\psi(s)|^2 \,ds
=
\|\psi\|=1.
\end{align*}
Darin haben wir verwendet, dass $a>0$.
Damit ist gezeigt, dass sich die Norm nicht ändert.
\end{proof}

\begin{beispiel}
Bei den Haar-Wavelets haben wir als Streckungsfaktoren die Zweierpotenzen
$2^j$ mit $j\in\mathbb Z$ verwendet.
Damit alle Wavelets die gleiche Norm bekamen, haben wir mit dem Faktor
$2^{-j/2}$ kompensiert.

Selbstverständlich können wir aber auch das Haar Mutter-Wavelet
\[
\psi_{\text{Haar}} = \chi_{[0,\frac12)} - \chi_{[\frac12,1)}
\]
als Ausgangspunkt wählen.
Dann ist $\psi_{\text{Haar},a,b}$ eine stückweise konstante Funktion,
die für $a>0$ beim Punkt $b$ auf den Wert $1/\sqrt{|a|}$ springt,
beim Punkt $b+a/2$ auf $-1/\sqrt{|a|}$ und ab $b+a$ wieder verschwindet.
Der Träger der Funktion $\psi_{\text{Haar},a,b}$ ist also das Interval
$[b,b+a]$.
% XXX Abbildung für das skalierte  Haar-Wavelet
\end{beispiel}

\begin{beispiel}
Sei $\psi=\frac1{\sqrt{2\pi}} e^{-t^2/2}$ die Wahrscheinlichkeitsdichte
der Standardnormalverteilung.
Man kann nachrechnen, dass $\psi$ die Zulässigkeitsbedingung erfüllt.
Die $\sigma$ skalierten und um $\mu$ verschobenen Funktionen sind
\[
\psi_{\sigma,\mu}(t)
=
\frac{1}{\sqrt{2\pi\sigma}} e^{-(t-b)^2/2\sigma^2},
\]
die Wahrscheinlichkeitsdichte einer Normalverteilung mit Erwartungswert $\mu$
und Varianz $\sigma^2$.
\end{beispiel}

Die Funktionen $\psi_{a,b}$ können jetzt als Analyse-Funktionen für das
Signal dienen.

\begin{definition}
Die stetige Wavelet-Transformation des Signals $f(t)$ mit dem Wavelet
$\psi$ ist die Funktion
\[
\mathcal{W}f (a,b)
=
\langle f,\psi_{a,b}\rangle
=
\frac{1}{\sqrt{|a|}}\int_{-\infty}^\infty f(t)\,\overline{
\psi\biggl(\frac{t-b}{a}\biggr)}\,dt.
\]
Der Definitionsbereich dieser Funktion ist die Menge
\[
\mathbb R^2_-
=
\mathbb R^*\times \mathbb R
=
\mathbb R^2 - (\{0\}\times \mathbb R)
\]
die Ebene ohne die Achse $a=0$.
\end{definition}

Die Wavelet-Transformation liefert also eine Funktion von {\em zwei}
Variablen.
Die beiden Parameter erlauben, unabhängig voneinander eine bestimmte
Stelle des Signals genauer anzuschauen durch Wahl von $b$ sowie die
Details genauer aufzulösen durch Vergrösserung von $a$.

\begin{beispiel}
Wir betrachten die Funktion
\[
\psi(t) = \begin{cases}
-\frac1{\sqrt{2}}&\qquad -1\le t< 0\\
\frac1{\sqrt{2}}&\qquad 0\le t< 1\\
0&\qquad\text{sonst}
\end{cases}
\]
Dies ist eine gestreckte und verschobene Version des Haar-Wavelets,
und erfüllt daher automatisch die Zulässigkeitsbedingung für ein Wavelet.
Ausserdem gilt $\|\psi\|=1$.
Gegenüber dem Haar-Mutter-Wavelet hat diese Funktion den Vorteil, dass 
sie antisymmetrisch ist, so dass auch die stetige Wavelettransformation
einer antisymmetrischen Funktion wieder symmetrisch sein wird.

Wir berechnen jetzt die stetige Wavelet-Transformat des Signals $f(t)=\sin t$.
Nach Definition ist
\begin{align*}
\mathcal{W}f(a,b)
&=
\int_{-\infty}^\infty f(t)
\frac{1}{\sqrt{|a|}}\overline{\psi\biggl(\frac{t-b}{a}\biggr)}
\,dt.
\\
\intertext{Da sich das Integral über ganz $\mathbb R$ erstreckt, können
wir das Integral um $b$ verschieben, und erhalten}
&=
\frac{1}{\sqrt{|a|}}
\int_{-\infty}^\infty f(t+b)\overline{\psi\biggl(\frac{t}{a}\biggr)}\,dt.
\\
\intertext{Substituieren wir $as=t$, können wir auch den Skalierungsfaktor
von $\psi$ auf das Signal verschieben:}
&=
\sqrt{|a|}
\int_{-\infty}^\infty f(as+b)\overline{\psi(s)}\,ds.
\\
\intertext{Für $a<0$ bekommt man ein negatives Vorzeichen, aber es tauschen
auch die Integrationsgrenzen ihre Plätze.
Jetzt können wir das Signal $f$ einsetzen und mit der Definition der
Funktion $\psi$ das Integral vereinfachen}
&=
\sqrt{\frac{|a|}{2}}
\biggl(
-
\int_{-1}^0 \sin(as+b)\,dt
+
\int_{0}^1 \sin(as+b)\,dt
\biggr)
\\
&=
\sqrt{\frac{|a|}{2}}
\biggl(
-
\biggl[-\frac{\cos(as+b)}{a}\biggr]_{-1}^0
+
\biggl[-\frac{\cos(as+b)}{a}\biggr]_{0}^1
\biggr)
\\
&=
\frac{1}{\sqrt{2|a|}}
(
\cos(b) - \cos(-a+b) -\cos(a+b)+\cos(b)
)
\\
&=
\frac{1}{\sqrt{2|a|}}
(2\cos b - \cos(b+a) - \cos(b-a))
\end{align*}
Wir betrachten einige einfach zu berechnende Werte von $\mathcal{W}f$.
Die Kosinus-Funktion ist antisymmetrisch bezüglich ungeraden Vielfachen
von $\pi/2$.
Wir setzen daher $b=(2k+1)\frac{\pi}2$ und berechnen den Wert von
\begin{align*}
\mathcal{W}f(a,b)
&=
\mathcal{W}f\biggl(a,(2k+1)\frac{\pi}2\biggr)
=
\pm
\frac{1}{\sqrt{2|a|}}2\sin a.
\end{align*}
\end{beispiel}

\begin{beispiel}
Sei wieder $\psi=\psi_{\text{Haar}}$ das Haar-Wavelet.
Wir berechnen die stetige Wavelet-Transformation der Sinus-Funktion
$f(t)=\sin t$.
Der Einfachheit halber führen wir die Berechnung der Wavelet-Transformation
zunächst für $a>0$ durch.
Nach Definition ist gilt
\begin{align*}
\mathcal{W}f (a,b)
&=
\frac{1}{\sqrt{|a|}}
\int_{-\infty}^\infty \sin t\cdot \psi\biggl(\frac{t-b}{a}\biggr)\,dt.
\\
\intertext{Da sich das Integral über die ganze reelle Achse erstreckt,
können wir $t$ durch $t+b$ ersetzen, ohne dass sich das Integral ändert.
Wir erhalten daher}
&=
\frac{1}{\sqrt{|a|}}
\int_{-\infty}^\infty \sin (t+b)\cdot \psi\biggl(\frac{t}{a}\biggr)\,dt
=
\frac{1}{\sqrt{|a|}}
\int_{-\infty}^\infty \sin (as+b)\cdot \psi(s) a\,ds.
\\
\intertext{Dabei haben wir $t=as$ substituiert.
Jetzt lässt sich das Integral explizit ausrechnen}
&=
\sqrt{a}
\int_{0}^{\frac12} \sin(as+b)\,ds
-
\sqrt{a}
\int_{\frac12}^{1} \sin(as+b)\,ds
\\
&=
\sqrt{a}
\biggl[-\frac{\cos(as+b)}{a}\biggr]_{0}^{\frac12}
-
\sqrt{a}
\biggl[-\frac{\cos(as+b)}{a}\biggr]_{\frac12}^1
=
\frac{1}{\sqrt{a}}\biggl(
-1 + 2\cos({\textstyle\frac12}a+b) - \cos(a+b)
\biggr)
\end{align*}
\end{beispiel}

