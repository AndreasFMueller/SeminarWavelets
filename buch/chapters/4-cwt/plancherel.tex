%
% plancherel.tex
%
% (c) 2019 Prof Dr Andreas Müller, Hochschule Rapperswil
%
\section{Plancherel-Formel%
\label{section:cwt:plancherel}}
\rhead{Plancherel-Formel}
Die Plancherel-Formel der Fourier-Theorie besagt, dass die
Fourier-Transformation eine Isometrie ist.
Das Skalarprodukt zweier Funktionen und das Skalarprodukt der
Fourier-Transformierten ist dasselbe.
Daraus folgt unmittelbar, dass die Transformation stetig und
umkehrbar ist.
Es ist daher erstrebenswert, wenn eine ähnliche Formel auch
auch für die stetige Wavelet-Transformation gilt.

\subsection{Das Skalarprodukt auf $H$}
Der Definnitionsbereich $H=\mathbb R^* \times \mathbb R$ der
Wavelet-Transformierten kan auf verschiedene Arten mit einer Integration
ausgestattet werden.
Wir versuchen das ``richtige'' Mass zu erraten, im
Abschnitt~\ref{subsection:haarmass} werden wir eine abstrakte Begründung
geben.

Die Variable $b$ verhält sich genau wie dei Variable $t$ bei Integration
einer Funktion von $t$.
Funktionen werden durch den Parameter $b$ verschoben, dabei darf sich
das Integral nicht ändern.
Das ist genau die Eigenschaft, mit deren Hilfe das Lebesgue-Mass
definiert wurde.

Die Variable $a$ funktioniert dagegen als Skalierungsparameter.
Grössere Werte von $a$ ziehen den Träger des Integranden um den Faktor $a$ 
in die Länge.

\begin{definition}
Das Mass auf $H$ ist definiert duch das Integral
\[
\int_H F(a,b)\,db \,\frac{da}{a^2}
\]
und das Skalaprodukt von Funktionen auf $H$ ist
\[
\langle F,G\rangle_H
=
\int_{H} F(a,b)\overline{G(a,b)}\,db \,\frac{da}{a^2}.
\]
\end{definition}

Dieses Mass ist im folgenden Sinn das richtige Mass.
Wenden wir auf die Argumente einer Funktion auf $H$ die Transformation
\[
(a,b) \mapsto (\alpha a, \beta + \alpha b)
\]
an, ändert sich das Integral wie folgt:
\begin{align*}
\int_H F(\alpha a,\beta + \alpha b)\,d(\beta + \alpha b)\,\alpha\frac{da}{|\alpha a|^2}
&=
\int_H F(\alpha a,\beta + \alpha b) \alpha\,db \ \frac{da}{|a|^2},
\end{align*}
die Transformation der Funktion ändert also das Integral nicht.
Die Wahl des Integrals ist so, dass die Skalierung und Translation
das Integral nicht ändert.

\subsection{Die Plancherel-Formel}
Ziel dieses Abschnitts ist, in Satz~\ref{satz:wplancherel} die
Plancherel-Formel für die Wavelet-Transformation zu beweisen.
Dabei können wir die Plancherel-Formel für die Fourier-Transformation
verwenden, denn es gilt
\[
\mathcal{W}f (a,b)
=
\langle f,\psi_{a,b} \rangle
=
\langle \hat{f}, \hat{\psi}_{a,b}\rangle.
\]
Wir müssen daher die Fourier-Transformation von $\psi_{a,b}$ besser
verstehen.
Nach den Rechenregeln für die Fourier-Transformation gilt
\[
\hat{\psi}_{a,b}(\omega)
=
e^{-i\omega b} \hat{\psi}_a(\omega)
=
\sqrt{|a|}e^{-i\omega b} \hat{\psi}(a\omega).
\]
Die Wavelet-Transformation kann daher geschrieben werden als
\begin{align*}
\mathcal{W}f (a,b)
&=
\int_{-\infty}^\infty 
\hat{f}(\omega)
\sqrt{|a|}e^{i\omega b} \overline{\hat{\psi}(a\omega)}
\,d\omega
=
\int_{-\infty}^\infty
\bigl(
\sqrt{|a|}
\hat{f}(\omega)
\overline{\hat{\psi}(a\omega)}
\bigr)
e^{i\omega b}
\,d\omega
\\
&=
\frac1{2\pi}
\int_{-\infty}^\infty
\underbrace{
2\pi
\sqrt{|a|}
\hat{f}(\omega)
\overline{\hat{\psi}(a\omega)}}_{\displaystyle F_a(\omega)}
e^{i\omega b}
\,d\omega
\end{align*}
Daraus kann man ablesen, dass $F_a(\omega)$ die Fourier-Umkehrtransformierte
der Funktion $b\mapsto \mathcal{W}f(a,b)$ ist.

% XXX Formel beweisen
\begin{equation}
\hat{F}_a(-b) = \mathcal{W}f(a, b)
\end{equation}

\begin{satz}
\label{satz:wplancherel}
Sei $\psi$ ein Wavelet mit der Konstanten $C_{\psi}$, wie sie in
der Zulässigkeitsbedingung auftritt.
Dann gilt
\begin{equation}
\langle \mathcal{W}f,\mathcal{W}g\rangle
=
C_{\psi}\langle f,g\rangle
\end{equation}
für beliebige Funktionen $f,g\in L^2$.
\end{satz}

\begin{proof}[Beweis]
Wir verwenden 
\begin{align}
\langle \mathcal{W}f, \mathcal{W}g\rangle_H
&=
\int_{\mathbb R^*} \int_{-\infty}^\infty
\mathcal{W}f (a,b)
\overline{\mathcal{W}g (a,b)} \,db \,\frac{da}{|a|^2}
\notag
\\
&=
\int_{\mathbb R^*} \int_{-\infty}^\infty
\hat{F}_a(-b) \overline{\hat{G}_a(-b)}\,db\,\frac{da}{|a|^2}
\notag
\\
&=
\int_{\mathbb R^*} 
\langle
\hat{F}_a,\hat{G}_a
\rangle
\,\frac{da}{|a|^2}
\notag
\\
&=
\int_{\mathbb R^*} \langle F_a, G_a\rangle \,\frac{da}{|a|^2}
\notag
\\
&=
\int_{\mathbb R^*}
\int_{-\infty}^\infty
(2\pi)^2 |a|\hat{f}(\omega) \overline{\hat{g}(\omega)} |\hat{\psi}(a\omega)|^2
\,d\omega
\,\frac{da}{|a|^2}
\notag
\\
&=
\int_{-\infty}^\infty
(2\pi)^2 \hat{f}(\omega) \overline{\hat{g}(\omega)}
\int_{\mathbb R^*}
|\hat{\psi}(a\omega)|^2
\,\frac{da}{|a|}
\,d\omega.
\label{cwtplancherel:p1}
\end{align}
Das Produkt $\hat{f}(\omega)\overline{\hat{g}(\omega)}$ deutet darauf hin,
dass dies fast schon ein Skalarprodukt ist.
Es wird tatsächlich das Skalarprodukt, wenn das innere Integral konstant ist.
Wir betrachten dieses separat und setzen $a\omega = s$, was
\begin{align*}
\int_{\mathbb R^*} |\hat{\psi}(a\omega)|^2 \,\frac{da}{|a|}
&=
\int_{\mathbb R^*} |\hat{\psi}(s)|^2 \,\frac{ds/\omega}{|s/\omega|}
=
\int_{\mathbb R^*} |\hat{\psi}(s)|^2 \,\frac{ds}{|s|}
=
C_{\psi}
\end{align*}
ergibt.
Eingesetzt \eqref{cwtplancherel:p1} finden wir
\begin{align*}
\langle \mathcal{W}f, \mathcal{W}g\rangle_H
&=
(2\pi)^2
C_{\psi}
\int_{-\infty}^\infty \hat{f}(\omega)\overline{\hat{g}(\omega)}\,d\omega
=
(2\pi)^2
C_{\psi}
\langle \hat{f},\hat{g}\rangle
=
(2\pi)^2
C_{\psi}
\langle f,g\rangle.
\end{align*}
Damit ist die Plancherel-Formel bewiesen.
\end{proof}

\subsection{Das Haar-Mass auf $\mathbb R^* \ltimes \mathbb R$
\label{subsection:haar-mass}}








