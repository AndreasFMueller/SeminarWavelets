%
% wsin.tex -- Wavelet-Transformation des Sinus
%
% (c) 2019 Prof Dr Andreas Müller, Hochschule Rapperswil
%
\documentclass[tikz]{standalone}
\usepackage{amsmath}
\usepackage{times}
\usepackage{txfonts}
\usepackage{pgfplots}
\usepackage{csvsimple}
\usetikzlibrary{arrows,intersections,math}
\begin{document}
\begin{tikzpicture}[>=latex,scale=0.95]

\def\punkt#1#2#3{
\fill[color=red!#3] ({#2-0.05},{#1-0.05}) rectangle ({#2+0.05},{#1+0.05});
}

\input{wsindots.tex}

\draw[->,line width=1pt] (-2.05,0)--(12.3,0) coordinate[label={$b$}];
\draw[->,line width=1pt] (0,-0.1)--(0,6.5) coordinate[label={right:$a$}];

\foreach \a in {2,4,...,8}{
	\draw[line width=0.1pt] (-2.05,{\a*3.1415/5})--(12.05,{\a*3.1415/5});
	\draw[line width=1pt] (-0.1,{\a*3.1415/5})--(0.1,{\a*3.1415/5});
	\node at (0,{\a*3.1415/5}) [above left] {$\a \pi$};
}
\foreach \b in {2,3}{
	\draw[line width=0.1pt] ({\b*3.1415},0)--({\b*3.1415},6);
	\draw[line width=1pt] ({\b*3.1415},-0.1)--({\b*3.1415},0.1);
	\node at ({\b*3.1415},0) [below] {$\mathstrut\b\pi$};
}
\draw[line width=0.1pt] ({3.1415},0)--({3.1415},6);
\draw[line width=1pt] ({3.1415},-0.1)--({3.1415},0.1);
\node at ({3.1415},0) [below] {$\mathstrut\pi$};
\node at (0,0) [below] {$0\mathstrut$};

\end{tikzpicture}
\end{document}

