%
% psisin.tex -- Wavelet-Transformation des Sinus
%
% (c) 2019 Prof Dr Andreas Müller, Hochschule Rapperswil
%
\documentclass[tikz]{standalone}
\usepackage{amsmath}
\usepackage{times}
\usepackage{txfonts}
\usepackage{pgfplots}
\usepackage{csvsimple}
\usetikzlibrary{arrows,intersections,math}
\begin{document}
\begin{tikzpicture}[>=latex]

\def\ascale{0.5}
\def\pxs{3.1415/60}

\def\rotpunkt#1#2#3{
\fill[color=red!#3] ({#2-\pxs},{(\ascale*#1)-\pxs}) 
	rectangle ({#2+\pxs},{(\ascale*#1)+\pxs});
}
\def\blaupunkt#1#2#3{
\fill[color=blue!#3] ({#2-\pxs},{(\ascale*#1)-\pxs}) 
	rectangle ({#2+\pxs},{(\ascale*#1)+\pxs});
}

\def\punkt#1#2#3{
\ifnum #3 > 0
	\rotpunkt{#1}{#2}{#3}
\fi
\ifnum #3 < 0
	\pgfmathparse{-#3}
	\xdef\f{\pgfmathresult}
	\blaupunkt{#1}{#2}{\f}
\fi
}

\input{psidots.tex}

\draw[->,line width=1pt] ({-41*\pxs},0)--(12.3,0) coordinate[label={$b$}];
\draw[->,line width=1pt] (0,-0.2)--(0,10.4) coordinate[label={right:$a$}];

\foreach \y in {1,2,...,6}{
	\draw[line width=0.1pt] ({-41*\pxs},{\y*3.1415/2})--({229*\pxs},{\y*3.1415/2});
	\draw[line width=1pt] (-0.1,{\y*3.1415/2})--(0.1,{\y*3.1415/2});
}
\foreach \y in {-1,...,7}{
	\draw[line width=0.1pt] ({\y*3.1415/2},0)--({\y*3.1415/2},{10+\pxs});
	\draw[line width=1pt] ({\y*3.1415/2},-0.1)--({\y*3.14159/2},0.1);
}

\node at (0,{3.1415/2}) [above left] {$\pi$};
\foreach \y in {2,...,6}{
	\node at (0,{\y*3.1415/2}) [above left] {$\y\pi$};
}

\node at ({3.1415},0) [below] {$\mathstrut\pi$};
\foreach \y in {2,...,3}{
	\node at ({\y*3.1415},0) [below] {$\y\pi\mathstrut$};
}

\node at (0,0) [below left] {$0$};

\end{tikzpicture}
\end{document}

