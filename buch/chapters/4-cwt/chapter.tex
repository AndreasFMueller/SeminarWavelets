%
% chapter.tex -- Kapitel über stetige Wavelet-Transformation
%
% (c) 2019 Prof Dr Andreas Müller, Hochschule Rapperswil
%
\chapter{Stetige Wavelet-Transformation
\label{chapter:cwt}}
\lhead{Stetige Wavelet-Transformation}
Im Kapitel~\ref{chapter:haar-wavelet} wurde am Beispiel des Haar-Wavelets
gezeigt, wie jede beliebige stetige Funktion als Linearkombination 
von skalierten und verschobenen Versionen einer einzigen Ausgangsfunktion
$\psi(t)$ geschrieben werden kann.
Das Haar-Wavelet hat einen Ausweg aus der Schwierigkeit der
Fourier-Transformation gewiesen, Ereignisse sowohl auf der Zeitachse
wie auch bezüglich ihrer Frequenz zu lokalisieren.
Es bleibt aber eine Reihe von Schwierigkeiten, die vom Haar-Wavelet nicht
adressiert werden:
\begin{enumerate}
\item
Die Funktion $\psi$ ist nicht stetig und alle daraus aufgebauten
Approximationsfunktionen sind nicht stetig
und erst recht nicht differenzierbar.
\item
Es werden nur Frequenzen verwendet, die $2^n$-fache einer Grundfrequenz
sind.
Die Analyse kann also nur Oktaven unterscheiden, während
\index{Oktave}%
Töne auf einer Tonleiter Frequenzverhältnisse von $\sqrt[12]{2}$ haben.
\index{Tonleiter}%
\end{enumerate}
Ziel dieses Kapitels ist, vernünftige Kriterien für Waveletfunktionen
$\psi(t)$ zu finden, so dass die Analyse und Synthese ähnlich einfach
wie für Haar-Wavelets möglich bleibt.
Ausserdem soll die stetige Wavelet-Transformation diskutiert werden,
welche das Problem der ``Zwischenfrequenzen'' löst.
\index{Haar-Wavelet}%

% !TeX spellcheck = de_CH_frami

\section{Spectral Graph Wavelet Transformation\label{sec:sgwt:wavelets}}
\rhead{SGWT}

Hier wollen wir nun das Bisherige zur Spectral Graph Wavelet Transformation 
vereinen.

\subsection{Graph Fourier Transformation\label{subsec:sgwt:gft}}

Bevor wir uns auf die Graph Wavelets st\"urzen, wollen wir zuerst versuchen die 
Fourier Theorie auf Funktionen auf Graphen anzuwenden. Dazu nehmen wir das 
Beispiel eines Dirac-Stosses $\delta(x)$, siehe~\cref{fig:sgwt:gft:dirac}.
\begin{figure}
    \centering
    \begin{minipage}[b]{0.49\textwidth}
    \includegraphics[
    angle=-90,
    origin=c,
    width=\textwidth
    ]{papers/sgwt/images/dirac.pdf}
    \vspace{-45pt}
    \caption{Darstellung eines Dirac-Stoss $\delta(x)$ mit maximal Wert $1$. 
        \label{fig:sgwt:gft:dirac}}
    \end{minipage}
    ~
    \begin{minipage}[b]{0.49\textwidth}
    \includegraphics[
    angle=-90,
    origin=c,
    width=\textwidth
    ]{papers/sgwt/images/dirac_fft.pdf}
    \vspace{-45pt}
    \caption{Fourier Transformation des Dirac-Stosses $\hat{\delta(x)}$ 
        aus~\cref{fig:sgwt:gft:dirac}.\label{fig:sgwt:gft:fftdirac}}
    \end{minipage}
\end{figure}
Wenn wir davon die Fourier Transformation berechnen, zu sehen 
in~\cref{fig:sgwt:gft:fftdirac}, stellen wir fest, das nicht nur alle 
Frequenzen vorhanden sondern auch alle gleich stark vorhanden sind. Die 
Fouriertransformierte $\hat{\delta}$ ist somit vollst\"andig delokalisiert.

$g(\lambda)\cdot\hat{\delta}(\lambda)$

\subsubsection{Kernelfunktionen}

Um dieses Problem zu l\"osen nehmen wir eine Kernelfunktion $g(\lambda)$, 
welche die Eigenschaften $g(0) = 0$ und $\lim_{\lambda\to\infty} g(x) = 0$ 
erf\"ullt. Mit Hilfe dieses Bandpasses k\"onnen wir die Eigenwerte bewusst 
delokalisieren. Wenn wir nun zus\"atzlich die Eigenwerte mit einem 
Skalierungsfaktor $t$ multiplizieren, erhalten wir auch eine 
M\"oglichkeit, das Spektrum zu verschieben. Obwohl $t$ ein kontinuierlicher 
Faktor ist, werden wir uns in der Praxis auf $J$ Faktoren, also eine endliche 
Anzahl $\{t_j\}^J_{j=1}$, beschr\"anken m\"ussen.

Wir haben jetzt nur noch das Problem, dass $\lambda_0 = 0$ gilt. Wir verlieren 
also den konstanten Anteil der zu analysierenden und synthetisierenden 
Funktion. Um das zu umgehen nehmen brauchen wir eine zus\"atzliche 
Kernelfunktion $h(\lambda)$ mit den Eigenschaften $h(0) > 0$ und 
$\lim_{\lambda\to\infty} = 0$.

Die in~\cref{sec:sgwt:spectralanalysis} vorgestellten Eigenfunktionen 
k\"onnen wir direkt nutzen um eine Graph Fourier Transformation zu definieren. 
Wir ersetzen bei der Fourier Transformation einfach die Exponentialfunktion 
oder Kugelfunktionen durch die Eigenvektoren der Laplace Matrix
\begin{equation*}
\hat{f} = \langle \chi, f \rangle = \sum_{n = 1}^{N} \chi^*(n)f(n).
\end{equation*}
Wenn wir, wie zum Beispiel in \texttt{octave} \"ublich, die Eigenvektoren als 
Spalten einer Matrix
\begin{equation}
\chi = 
\left[
\begin{pmatrix}\\\chi_0\\\\\end{pmatrix}
\begin{pmatrix}\\\chi_1\\\\\end{pmatrix}
\begin{pmatrix}\\\chi_2\\\\\end{pmatrix}
\cdots
\begin{pmatrix}\\\chi_{N-1}\\\\\end{pmatrix}
\right]
\end{equation}
zur Verf\"ugung haben, k\"onnen wir die Transformation 
einfach durch eine Matrix-Vektor Multiplikation ersetzen
\begin{equation*}
\hat{f} = \chi^* f.
\end{equation*}
Die R\"ucktransformation ist dann auch wieder analog der Fouriertheorie
\begin{equation*}
f = \chi \hat{f}.
\end{equation*}

\subsection{Graph Wavelets: Lokalisierung\label{subsec:sgwt:gwt:localizing}}

Mit der Graph Fourier Transformation k\"onnen wir nun Funktionen auf Graphen 
analysieren und dann auch wieder synthetisieren. Mit Hilfe von Wavelets haben 
wir aber bereits gesehen, dass damit auch eine Lokalisierung m\"oglich ist. 
Graphen haben da den Nachteil, dass die Knoten keine inh\"arente Reihenfolge 
haben. Es ist also nicht klar was mit $f(x - h)$ gemeint ist. Wir haben 
allerdings in~\cref{eq:sgwt:lambda:series} eine Reihenfolge f\"ur die 
Eigenwerte der Laplace Matrix eines Graphen definiert. Die Eigenwerte sind 
allerdings komplett lokalisiert, sie kommen also einem Dirac-Stoss an 
einem Knoten gleich.

\subsection{Graph Wavelets: Skalierung\label{subsec:sgwt:gwt:scaling}}

\subsection{Frames}

Mit $h(\lambda)$ und den $J$ skalierten $g(t\lambda)$ haben wir also $J + 1$ 
Kernelfunktionen f\"ur die Analyse und Synthese zur Verf\"ugung. Wir arbeiten 
also mit einem Frame. Die Grenzen sind dabei gegeben als
\begin{align*}
A &= \min_{\lambda \in \left[0, \lambda_{N-1}\right]} f(\lambda) \\
B &= \max_{\lambda \in \left[0, \lambda_{N-1}\right]} f(\lambda) \\
\text{mit } f(\lambda) &= h(\lambda)^2 + \sum_{j = 1}^{J} g(t_j\lambda)^2.
\end{align*}

\subsection{\texorpdfstring{$\psi_j$}{psi} und \texorpdfstring{$\phi$}{phi}}
Damit lassen sich nun unsere Wavelets $\psi_j$ und $\phi$ wie folgt konstruieren
\begin{align}
\psi_j = \chi \diag{g(t_j\lambda)} \chi' 
\label{eq:sgwt:psi}
\\
\phi = \chi \diag{h(\lambda)} \chi'.
\label{eq:sgwt:phi}
\end{align}
Zur Veranschaulichung sehen wir hier die Wavelets eines Kreisgraphen 
in~\cref{fig:sgwt:wavelets:ring0,fig:sgwt:wavelets:ring1,fig:sgwt:wavelets:ring2,fig:sgwt:wavelets:ring3,fig:sgwt:wavelets:ring4,fig:sgwt:wavelets:ring5},
 eines Streckengraphen 
in~\cref{fig:sgwt:wavelets:line0,fig:sgwt:wavelets:line1,fig:sgwt:wavelets:line2,fig:sgwt:wavelets:line3,fig:sgwt:wavelets:line4,fig:sgwt:wavelets:line5}
 und eines Kugelgraphen 
in~\cref{fig:sgwt:wavelets:sphere0,fig:sgwt:wavelets:sphere1,fig:sgwt:wavelets:sphere2,fig:sgwt:wavelets:sphere3,fig:sgwt:wavelets:sphere4,fig:sgwt:wavelets:sphere5}.
Gut zu erkennen ist dabei, dass der Kreisgraph die wohl beste Approximation der 
bisherigen Wavelettheorie ist. Beim Streckengraph fehlt die Periodisierung, die 
wir durch die Verbindungskante des Start- und Endknoten beim Kreisgraphen 
erreicht haben. Auch beim Kugelgraphen wird klar, dass die Pole, aufgrund ihres 
viel gr\"osseren Grades, deutlich st\"arker gewichtet werden und es daher zu 
einer Verzerrung der Wavelets in Richtung der Pole kommt.

\begin{figure}
    \centering
    \begin{minipage}[b]{0.49\textwidth}
        \includegraphics[
        angle=-90,
        origin=c,
        width=\textwidth]{papers/sgwt/images/wavelets-psi-1-10.pdf}
        \vspace{-45pt}
        \caption{$\psi_1(v_{10})$-Wavelets eines Kreisgraphen mit 100 Knoten.}
        \label{fig:sgwt:wavelets:ring0}
    \end{minipage}
    ~
    \begin{minipage}[b]{0.49\textwidth}
        \includegraphics[
        angle=-90,
        origin=c,
        width=\textwidth]{papers/sgwt/images/wavelets-psi-2-10.pdf}
        \vspace{-45pt}
        \caption{$\psi_2(v_{10})$-Wavelets eines Kreisgraphen mit 100 Knoten.}
        \label{fig:sgwt:wavelets:ring1}
    \end{minipage}
    ~
    \begin{minipage}[b]{0.49\textwidth}
        \includegraphics[
        angle=-90,
        origin=c,
        width=\textwidth]{papers/sgwt/images/wavelets-psi-3-10.pdf}
        \vspace{-45pt}
        \caption{$\psi_3(v_{10})$-Wavelets eines Kreisgraphen mit 100 Knoten.}
        \label{fig:sgwt:wavelets:ring2}
    \end{minipage}
    ~
    \begin{minipage}[b]{0.49\textwidth}
        \includegraphics[
        angle=-90,
        origin=c,
        width=\textwidth]{papers/sgwt/images/wavelets-psi-4-10.pdf}
        \vspace{-45pt}
        \caption{$\psi_4(v_{10})$-Wavelets eines Kreisgraphen mit 100 Knoten.}
        \label{fig:sgwt:wavelets:ring3}
    \end{minipage}
    ~
    \begin{minipage}[b]{0.49\textwidth}
        \includegraphics[
        angle=-90,
        origin=c,
        width=\textwidth]{papers/sgwt/images/wavelets-psi-5-10.pdf}
        \vspace{-45pt}
        \caption{$\psi_5(v_{10})$-Wavelets eines Kreisgraphen mit 100 Knoten.}
        \label{fig:sgwt:wavelets:ring4}
    \end{minipage}
    ~
    \begin{minipage}[b]{0.49\textwidth}
        \includegraphics[
        angle=-90,
        origin=c,
        width=\textwidth]{papers/sgwt/images/wavelets-phi-10.pdf}
        \vspace{-45pt}
        \caption{$\phi(v_{10})$-Wavelets eines Kreisgraphen mit 100 Knoten.}
        \label{fig:sgwt:wavelets:ring5}
    \end{minipage}
\end{figure}

\begin{figure}
    \centering
    \begin{minipage}[b]{0.49\textwidth}
        \includegraphics[
        angle=-90,
        origin=c,
        width=\textwidth]{papers/sgwt/images/wavelets-psi-line-1-10.pdf}
        \vspace{-45pt}
        \caption{$\psi_1(v_{10})$-Wavelets eines Streckengraphen mit 100 
        Knoten.}
        \label{fig:sgwt:wavelets:line0}
    \end{minipage}
    ~
    \begin{minipage}[b]{0.49\textwidth}
        \includegraphics[
        angle=-90,
        origin=c,
        width=\textwidth]{papers/sgwt/images/wavelets-psi-line-2-10.pdf}
        \vspace{-45pt}
        \caption{$\psi_2(v_{10})$-Wavelets eines Streckengraphen mit 100 
        Knoten.}
        \label{fig:sgwt:wavelets:line1}
    \end{minipage}
    ~
    \begin{minipage}[b]{0.49\textwidth}
        \includegraphics[
        angle=-90,
        origin=c,
        width=\textwidth]{papers/sgwt/images/wavelets-psi-line-3-10.pdf}
        \vspace{-45pt}
        \caption{$\psi_3(v_{10})$-Wavelets eines Streckengraphen mit 100 
        Knoten.}
        \label{fig:sgwt:wavelets:line2}
    \end{minipage}
    ~
    \begin{minipage}[b]{0.49\textwidth}
        \includegraphics[
        angle=-90,
        origin=c,
        width=\textwidth]{papers/sgwt/images/wavelets-psi-line-4-10.pdf}
        \vspace{-45pt}
        \caption{$\psi_4(v_{10})$-Wavelets eines Streckengraphen mit 100 
        Knoten.}
        \label{fig:sgwt:wavelets:line3}
    \end{minipage}
    ~
    \begin{minipage}[b]{0.49\textwidth}
        \includegraphics[
        angle=-90,
        origin=c,
        width=\textwidth]{papers/sgwt/images/wavelets-psi-line-5-10.pdf}
        \vspace{-45pt}
        \caption{$\psi_5(v_{10})$-Wavelets eines Streckengraphen mit 100 
        Knoten.}
        \label{fig:sgwt:wavelets:line4}
    \end{minipage}
    ~
    \begin{minipage}[b]{0.49\textwidth}
        \includegraphics[
        angle=-90,
        origin=c,
        width=\textwidth]{papers/sgwt/images/wavelets-phi-line-10.pdf}
        \vspace{-45pt}
        \caption{$\phi(v_{10})$-Wavelets eines Streckengraphen mit 100 Knoten.}
        \label{fig:sgwt:wavelets:line5}
    \end{minipage}
\end{figure}

\begin{figure}
    \centering
    \begin{minipage}[b]{0.49\textwidth}
        \includegraphics[
        angle=-90,
        origin=c,
        width=\textwidth]{papers/sgwt/images/wavelets-psi-1-sphere-334.pdf}
        \vspace{-45pt}
        \caption{$\psi_1(v_{334})$ eines ungewichteten Kugelgraphen mit 
        1252 Knoten.}
        \label{fig:sgwt:wavelets:sphere0}
    \end{minipage}
    ~
    \begin{minipage}[b]{0.49\textwidth}
        \includegraphics[
        angle=-90,
        origin=c,
        width=\textwidth]{papers/sgwt/images/wavelets-psi-2-sphere-334.pdf}
        \vspace{-45pt}
        \caption{$\psi_2(v_{334})$ eines ungewichteten Kugelgraphen mit 
        1252 Knoten.}
        \label{fig:sgwt:wavelets:sphere1}
    \end{minipage}
    ~
    \begin{minipage}[b]{0.49\textwidth}
        \includegraphics[
        angle=-90,
        origin=c,
        width=\textwidth]{papers/sgwt/images/wavelets-psi-3-sphere-334.pdf}
        \vspace{-45pt}
        \caption{$\psi_3(v_{334})$ eines ungewichteten Kugelgraphen mit 
        1252 Knoten.}
        \label{fig:sgwt:wavelets:sphere2}
    \end{minipage}
    ~
    \begin{minipage}[b]{0.49\textwidth}
        \includegraphics[
        angle=-90,
        origin=c,
        width=\textwidth]{papers/sgwt/images/wavelets-psi-4-sphere-334.pdf}
        \vspace{-45pt}
        \caption{$\psi_4(v_{334})$ eines nicht gewichteten Kugelgraphen mit 
        1252 Knoten.}
        \label{fig:sgwt:wavelets:sphere3}
    \end{minipage}
    ~
    \begin{minipage}[b]{0.49\textwidth}
        \includegraphics[
        angle=-90,
        origin=c,
        width=\textwidth]{papers/sgwt/images/wavelets-psi-5-sphere-334.pdf}
        \vspace{-45pt}
        \caption{$\psi_4(v_{334})$ eines ungewichteten Kugelgraphen mit 
        1252 Knoten.}
        \label{fig:sgwt:wavelets:sphere4}
    \end{minipage}
    ~
    \begin{minipage}[b]{0.49\textwidth}
        \includegraphics[
        angle=-90,
        origin=c,
        width=\textwidth]{papers/sgwt/images/wavelets-phi-sphere-334.pdf}
        \vspace{-45pt}
        \caption{$\phi(v_{334})$ eines ungewichteten Kugelgraphen mit 1252 
        Knoten.}
        \label{fig:sgwt:wavelets:sphere5}
    \end{minipage}
\end{figure}

Die Wavelets k\"onnen wir dann wieder ein die $N(J+1)\text{x}N$-Matrix $T$ 
packen um dann die Wavelet Koeffizienten wie folgt zu berechnen
\begin{equation}
\hat{f} = T f.
\label{eq:sgwt:hatf}
\end{equation}

\subsection{SGWT Analyse und Synthese}

Um nun eine Funktion $f(v)$ auf einem Graphen $G$ zu analysieren gehen wir 
also wie folgt vor:
\begin{itemize}
    \item[1.] Generierung der \laplaceL{} Matrix aus dem Graphen $G$.
    \item[2.] Berechnung der Eigenwerte $\lambda$ und Eigenvektoren $\chi$.
    \item[3.] Berechnung der $\psi_j$ und $\phi$ Wavelets 
    mit~\cref{eq:sgwt:psi,eq:sgwt:phi}.
    \item[4.] Berechnung der Wavelet Koeffizienten $\hat{f}$ 
    mit~\cref{eq:sgwt:hatf}.
\end{itemize}

F\"ur die Synthese nehmen wir als Eingabe die bei der Analyse berechnet und 
danach m\"oglicherweise bearbeiteten $\hat{f}$. Zus\"atzlich brauchen wir die 
$T$ Matrix, mit deren Inversen wir wieder die Funktion
\begin{equation}
f = T^{-1} \hat{f}
\end{equation}
synthetisieren. Da diese Matrix aber nicht mehr quadratisch ist, kann sie nicht 
mehr so einfach invertiert werden. Wir nehmen uns daher das Pseudoinverse $L = 
(T^*T)^{-1}T^*$ zur Hilfe und erhalten
\begin{equation}
f = L \hat{f} = (T^*T)^{-1}T^* \hat{f}.
\end{equation}

\section{Effiziente Berechnung der CWT}
\rhead{Berechnung der CWT}
Als erstes möchten wir herausfinden, wie sich die CWT effizient berechnen lässt.
Als zweites werden wir zwei Beispielsignale definieren und die CWT mittels Haar-Wavelet betrachten.
Abschliessend rechnen wir die CWT noch mit dem Gabor-Wavelet und vergleichen die beiden Bilder.

\subsection{CWT als Faltung}
Betrachten wir zuerst die folgende Gleichung
\begin{equation}
\Wave f (a,b)
=
\langle f,\psi_{a,b}\rangle
=
\frac{1}{\sqrt{|a|}}\int_{-\infty}^\infty f(t)\,
	\overline{\psi}\biggl(\frac{t-b}{a}\biggr)\,\mathrm{d}t,\label{complex:CWT}
\end{equation}
durch welche in Definition~\ref{cwt:definition} die kontinuierliche Wavelet-Transformation eingeführt wurde.
Dieses Integral entspricht der Faltung zwischen $f(t)$ und 
\begin{equation} 
    g(t) 
    = \frac{1}{\sqrt{|a|}} \overline\psi\biggl(\frac{-t}{a}\biggr).
\end{equation}
Der Standard-Trick zur effizienten Berechnung einer Faltung ist die Multiplikation im Frequenzbereich.
\begin{equation} 
\mathcal{W}f (a,b) = (f*g)(t) = \mathcal{F}^{-1}\biggl\lbrace\hat f(\omega) \hat g (\omega) \biggr\rbrace.
\end{equation}
Dafür benötigen wir die Fouriertransformierte $\hat g (\omega)$:
\begin{align*}
	\hat g (\omega) = 
    \Four\,\biggl\lbrace \frac{1}{\sqrt{|a|}} \overline\psi \biggl(\frac{-t}{a}\biggr) \biggr\rbrace 
	&= \frac{1}{\sqrt{|a|}} \int_{-\infty}^{\infty}\overline\psi\biggl(\frac{-t}{a}\biggr) \, e^{-i\omega t}\,\mathrm{d}t\\
	&= \frac{1}{\sqrt{|a|}} \overline{\int_{-\infty}^{\infty}\psi \biggl(\frac{-t}{a}\biggr) \, e^{i\omega t}\,\mathrm{d}t}  
    & \biggl(\text{Substitution } t' = \frac{-t}{a}\biggr)\\
	&= \frac{1}{\sqrt{|a|}} \overline{\int_{-\infty}^{\infty}\psi(t') \, e^{-ia\omega t'} |a|\,\mathrm{d}t'}\\
	&= \sqrt{|a|} \, \overline{\hat{\psi}}(a\omega).
\end{align*}
Gleichung~\eqref{complex:CWT} lässt sich somit schreiben als
\begin{equation}
\Wave f(a,b)
= \mathcal{F}^{-1}\bigl\lbrace\hat{f}(\omega) \sqrt{|a|}\, \overline{\hat{\psi}}(a\omega)\bigr\rbrace. \label{complex:fcwt}
\end{equation}

Mittels Fourier-Transformation lässt sich die Wavelet-Transformation folglich besonders elegant berechnen.
Kontinuierliche Funktionen sind für numerische Systeme jedoch ungeeignet.
Die CWT muss in $a$ und $b$ diskretisiert werden.
Die Diskretisierung von $b$ entspricht vorteilhaft gerade derjenigen des Signals selbst.
Dann lässt sich die Fourier-Transformation mittels FFT effizient berechnen und Gleichung~\eqref{complex:fcwt} wird zu
\begin{equation}
	\mathcal{W}f(a,b) = \text{IFFT}\bigl(\text{FFT}(f) \, \overline{\hat{\psi}}(a\omega)\bigr). \label{complex:ffcwt}
\end{equation}

Der Faktor $\sqrt{|a|}$ wurde hierbei weggelassen.
Hierdurch werden die hohen Frequenzen stärker gewichtet und $|\!\Wave f(a,b)|$ ist gerade proportional zur Amplitude der analysierten Signalkomponente.
Zudem erzielen wir im Diskreten nicht exakt die Faltung, sondern die zirkuläre Version davon. 
Mehr dazu im Abschnitt~\ref{complex:circ-conv-padding}.

Gleichung~\eqref{complex:ffcwt} muss für jedes $a$ einzeln gelöst werden.
Sie wird besonders interessant, wenn das Wavelet im Frequenzbereich eine geschlossene, analytische Form besitzt.
Dann benötigt man nur eine FFT für das Signal, so wie für jedes $a$ eine inverse FFT und eine punktweise Multiplikation zwischen Signal und Wavelet.

\clearpage
\subsection{Das Haar-Wavelet}
\rhead{Haar-Wavelet}
\begin{figure}
	\centering
	\includegraphics{papers/complex/images/signals.pdf}
	\caption{Die beiden Beispielsignale $x_1(t)$ und $x_2(t)$}
\end{figure}
Rechnen wir das erste Beispiel.
Hierfür benötigen wir zwei Dinge: Signal und Wavelet.
Als Signale nehmen wir zwei Sinus-Schwingungen, eine mit linear ansteigender und eine mit stückweise konstanter Frequenz.
\begin{align}
    x_1(t) &= \sin\left( \int_{0}^{t} 2\pi f_1(t')\,\mathrm{d}t'\right) & f_1(t) &= 2 + 6/4 \cdot t \\
    x_2(t) &= \sin\left( \int_{0}^{t} 2\pi f_2(t')\,\mathrm{d}t'\right) & f_2(t) &= \left\lbrace \begin{matrix}
    4, & &t& < 1\\
    8, & 1.0 \le &t& < 2.0\\
    4, & 2.0 \le &t& < 3.0\\
    8, & 3.0 \le &t&\\
    \end{matrix}\right.
\end{align}
Das Haar-Wavelet sei in diesem Abschnitt zentriert um $t=0$.
Die daraus resultierende Symmetrie wird sich in der Berechnung der Fourier-Transformation als hilfreich erweisen.

\begin{definition}
	\label{complex:def-haar-wavelet}
	Das Haar-Wavelet besitzt folgende Gestalt:
	\[
	\psi_{\text{Haar}}(t) = \left\lbrace\begin{matrix*}[r]
	1 & -\frac{1}{2} \le t < 0  \\
	-1 & 0 \le t < \frac{1}{2} \\
	0 & \text{sonst}.
	\end{matrix*} \right.\label{complex:def-haar}
	\]
\end{definition}
Die Fourier-Transformierte von $\psi_{\text{Haar}}$ berechnet sich wie folgt:
\begin{align}
	\Four \psi_\text{Haar}  
	&= \frac{1}{\sqrt{2\pi}}\int_{-\infty}^{\infty} \psi_\text{Haar} e^{-i\omega t} \,\mathrm{d}t\nonumber\\
	&= \frac{1}{\sqrt{2\pi}}\Biggl( \int_{-1/2}^{0} e^{-i\omega t} \,\mathrm{d}t - \int_{0}^{1/2} e^{-i\omega t}\,\mathrm{d}t \Biggr) \nonumber\\
	&= \frac{i}{\sqrt{2\pi}\omega}\bigl( \bigl[ e^{-i\omega t}\bigr]_{-1/2}^0  - \bigl[ e^{-i\omega t}\bigr]_{0}^{1/2} \bigr)\nonumber\\
	&= \frac{i}{\sqrt{2\pi}} \frac{1-\cos(\omega/2)}{\omega/2}\label{complex:f-psi-haar}
\end{align}
Das Haar-Wavelet ist also nicht nur im Zeit-, sondern auch im Frequenzbereich besonders einfach.
Insbesondere lässt sich die mit $a$ skalierte Version des Wavelets durch Satz~\ref{four-int:trans-dial} direkt im Frequenzbereich berechnen.
Abbildung~\ref{complex:haar} zeigt das Haar-Wavelet im Zeit- und Frequenzbereich.
Auffallend ist, dass das im Zeitbereich besonders gut lokalisierte Haar-Wavelet in der Frequenz sehr schlecht lokalisiert ist.
\begin{figure}
	\centering
	\includegraphics{papers/complex/images/haar.pdf}
	\caption{Das Haar-Wavelet}
	\label{complex:haar}
\end{figure}

\begin{figure}
	\centering
	\includegraphics{papers/complex/images/haar_dom.pdf}
	
	\caption{Blau: $\psi_\text{Haar}$, Rot: $\sin ({\color{red}\omega_\psi}\cdot t)$, Gelb: $\sin ({\color{yellow}1.0}\cdot 2\pi t)$, Grün: $\sin ({\color{green}0.5}\cdot 2\pi t)$}
	\label{complex:dom-freq}
\end{figure}
An dieser Stelle definieren wir noch die \emph{dominante Frequenz} eines Wavelets.
\begin{definition}
	Die Fourier-Transformierte eines Wavelets erreicht den maximalen Betrag bei der \emph{dominanten Frequenz $\omega_\psi$}.
	\begin{equation}
		\omega_\psi \coloneqq \underset{\omega}{\text{\emph{argmax}}} \, |\hat\psi(\omega)|
	\end{equation}
	
\end{definition}

Die dominante Frequenz erlaubt es, die $a$-Achse der Wavelet-Transformation als Frequenz-Achse zu interpretieren.
Für die Momentanfrequenz gilt
\[
	\omega(b) \approx \frac{\omega_\psi}{a_\text{max}(b)},
	\quad 
	a_\text{max}(b)
	= 
	\underset{a}{\text{argmax}} \, |\!\Wave f(a,b)|.
\]
Diese Interpretation ist natürlich nur zulässig, wenn das Signal zum betrachteten Zeitpunkt nur eine dominante Frequenz-Komponente beinhaltet.
Bei unseren Beispielsignalen ist dies der Fall.
Abbildung~\ref{complex:dom-freq} illustriert die Bedeutung von $\omega_\psi$ für das Haar-Wavelet.
Es ist die Frequenz, bei welcher das Skalarprodukt mit dem Wavelet maximal wird.

Somit haben wir für unser Beispiel alles zusammen.
Nach einer Diskretisierung der Variablen überlassen wir die Arbeit dem Computer.
Dies liefert die Bilder aus Abbildung~\ref{complex:haar-ex}.

\begin{figure}
	\centering
	\includegraphics{papers/complex/images/chirp_haar.pdf}
	\includegraphics{papers/complex/images/square_haar.pdf}
	\caption{Wavelet-Transformationen der beiden Beispielsignale mit dem Haar-Wavelet. 
		Die Lokalisierung in der Zeit ist sehr gut, aber die momentane Frequenz ist kaum ersichtlich. 
		Zudem resultiert das periodische Signal in einer periodischen Helligkeit. 
		(Zur Erinnerung: bei reellen Werten entspricht die Farbe dem Vorzeichen, Blau: $+$, Gelb $-$)
	}
	\label{complex:haar-ex}
\end{figure}

Wie erwartet ist die Lokalisierung in der Frequenz ziemlich schlecht.
Das Haar-Wavelet gibt den Zeitpunkt einer Änderung der Frequenz zwar sehr genau wieder, die Frequenz selbst ist jedoch kaum ablesbar.
Als Orientierungshilfe sind $a_\text{max} (b) = \max_a{|\!\Wave x_n(a,b)|}$ weiss hervorgehoben.
Sie weichen um $\omega_\psi$ von der Signal-Frequenz ab, welche als Schwingung in der Amplitude gut erkennbar ist.
Dieses An- und Abschwellen des Betrags der Skalarprodukte verhindert es, $a_\text{max}(b)$ einfach zu folgen.
Dies werden wir im Abschnitt~\ref{complex:separate} durch komplexe Wavelets beheben.
Zuerst kümmern wir uns aber um die Lokalisierung in der Frequenz.

%
% plancherel.tex
%
% (c) 2019 Prof Dr Andreas Müller, Hochschule Rapperswil
%

%
% Verifikation der Umkehrformel
%
\begin{frame}
\frametitle{Vergleichsprinzip}

\begin{vergleich}
Für zwei Funktionen $f_1$ und $f_2$ in $L^2(\mathbb R)$ gilt
\[
\left.
\begin{aligned}
f_1&=f_2
&&\Leftrightarrow&
\langle f_1,g\rangle &= \langle f_2,g\rangle
\\
f_1-f_2&=0
&&\Leftrightarrow&
\langle f_1-f_2,g\rangle &= 0
\end{aligned}
\quad\right\}
\quad
\text{für alle $g\in L^2(\mathbb R)$}
\]
\end{vergleich}
\vspace*{-10pt}

\begin{proof}[Beweis]
Richtung $\boxed{\Rightarrow}$\;: Verwende Cauchy-Schwarz-Ungleichung
\begin{align*}
f_1&=f_2
&&\Rightarrow&
f_1-f_2&=0
&&\Rightarrow&
|\langle f_1-f_2,g\rangle|
&\le
\underbrace{\|f_1-f_2\|}_{\displaystyle=0}\cdot \|g\|
\end{align*}
\vspace{-20pt}

Richtung $\boxed{\Leftarrow}$\;: Verwende $g=f_1-f_2$
\begin{align*}
\langle f_1-f_2,g\rangle&=0
&&\Rightarrow&
\|f_1-f_2\|
&=
\langle f_1-f_2,f_1-f_2\rangle
=
0
&&\Rightarrow&
f_1-f_2&=0
\end{align*}
\end{proof}

\end{frame}

%
% Verifikation der Umkehrformel
%
\begin{frame}
\frametitle{Verifikation der Umkehrformel $\mathbb R^n$}
Umkehrformel muss gleiche Skalarprodukte mit $\vec{w}$ haben wie $\vec{v}$:
\[
\langle \vec{v},\vec{w}\rangle
\stackrel{?}{=}
\langle \text{Umkehrformel für $\vec{v}$},\vec{w}\rangle
\qquad\text{für alle $\vec{w}\in\mathbb R^n$}
\]
\uncover<2->{
Nachrechnen
\begin{align*}
&\hbox to8cm{\hfill}\\[-20pt]
\langle\vec{v},\vec{w}\rangle
&\stackrel{?}{=}
\ifthenelse{\boolean{presentation}}{
\only<3>{
\biggl\langle
\sum_{j=1}^n \langle \vec{v},\vec{b}_j\rangle\,\vec{b}_j,\vec{w}
\biggr\rangle
}
\only<4>{
\sum_{j=1}^n \langle \vec{v},\vec{b}_j\rangle\,\langle\vec{b}_j,\vec{w}\rangle
}
\only<5>{
\sum_{j=1}^n \langle \vec{v},\vec{b}_j\rangle\,
\overline{\langle\vec{w},\vec{b}_j\rangle}
}
\only<6->{
\sum_{j=1}^n v_j\bar{w}_j
}
\only<7->{
\qquad\text{Parseval-/Plancherel-Formel}
}}{
\sum_{j=1}^n v_j\bar{w}_j
\qquad\text{Parseval-/Plancherel-Formel}
}
\end{align*}
\uncover<8->{%
\begin{parseval}
$f,g$ $2\pi$-periodische Formeln mit Fourierkoeffizienten 
$a_k^f,a_k^g$ und $b_k^f,b_k^g$, dann gilt
\vspace{-10pt}

\[
\langle f,g\rangle
=
\frac12
a_0^f \overline{a_0^g}
+
\sum_{k=1}^\infty
(
a_k^f \overline{a_k^g}
+
b_k^f \overline{b_k^g}
)
\]
\end{parseval}}
}
\vspace*{-10pt}
\uncover<9->{%
\begin{plancherel}
Für $f,g\in L^2(\mathbb R)$ gilt
\[
\langle f, g\rangle
=
\langle \mathcal{F}f,\mathcal{F}g\rangle
=
\langle \hat{f},\hat{g}\rangle
\]
\end{plancherel}}

\end{frame}

%
% Verifikation der Umkehrformel
%
\begin{frame}
\frametitle{Verifikation der Umkehrformel $\mathcal{W}f$}

Umkehrformel muss gleiche Skalarprodukte mit $g$ haben wie $f$:
\[
\langle f,g\rangle
\stackrel{?}{=}
\langle \text{Umkehrformel für $f$},g\rangle
\qquad\text{für alle $g\in L^2$}
\]
\uncover<2->{
Nachrechnen
\begin{align*}
&\hbox to8cm{\hfill}\\[-20pt]
\langle f, g\rangle
&
\stackrel{?}{=}
\ifthenelse{\boolean{presentation}}{
\only<3>{
\biggl\langle
\int_{\mathbb R^+}\int_{-\infty}^\infty
\mathcal{W}f(a,b)\,\psi_{a,b}
\, db\,\frac{da}{|a|^2},
g
\biggr\rangle}
\only<4>{
\int_{-\infty}^\infty
\int_{\mathbb R^+}\int_{-\infty}^\infty
\mathcal{W}f(a,b)\,\psi_{a,b}(t)
\, db\,\frac{da}{|a|^2}
\,
\overline{g(t)}
\,dt
}
\only<5>{
\int_{\mathbb R^+}\int_{-\infty}^\infty
\mathcal{W}f(a,b)
\int_{-\infty}^\infty
\psi_{a,b}(t)
\,
\overline{g(t)}
\,dt
\, db\,\frac{da}{|a|^2}
}
\only<6>{
\int_{\mathbb R^+}\int_{-\infty}^\infty
\mathcal{W}f(a,b)
\langle
\psi_{a,b},
g
\rangle
\, db\,\frac{da}{|a|^2}
}
\only<7>{
\int_{\mathbb R^+}\int_{-\infty}^\infty
\mathcal{W}f(a,b)
\overline{
\langle
g,
\psi_{a,b}
\rangle}
\, db\,\frac{da}{|a|^2}
}
\only<8-10>{
\int_{\mathbb R^+}\int_{-\infty}^\infty
\mathcal{W}f(a,b)
\overline{\mathcal{W}g(a,b)}
\, db\,\frac{da}{|a|^2}}
\only<11->{
\langle 
\mathcal{W}f(a,b),
\mathcal{W}g(a,b)
\rangle_H}
\only<12->{\qquad\text{Plancherel-Formel}}}{
\langle 
\mathcal{W}f(a,b),
\mathcal{W}g(a,b)
\rangle_H
\qquad\text{Plancherel-Formel}
}
\end{align*}
}

\uncover<9->{
\begin{definition}
Die Menge $H=\{(a,b)\,|\, a\in\mathbb R^*\wedge b\in\mathbb R\}$
heisst die {\em Heisenberg-Gruppe}.
\uncover<10->{%
Für Funktionen auf $H$ gilt das Skalarprodukt
\[
\langle u,v\rangle_H
=
\int_{\mathbb R^*}\int_{-\infty}^\infty
u(a,b)\overline{v(a,b)}
\,db\,\frac{da}{|a|^2}
\]
}
\end{definition}
}

\end{frame}


%
% umkehrformel.tex
%
% (c) 2019 Prof Dr Andreas Müller, Hochschule Rapperswil
%
\section{Umkehrformel%
\label{section:cwt:umkehrformel}}
\rhead{Umkehrformel}
Die Plancherel-Formel von Satz~\ref{satz:wplancherel} zeigt, dass die
Wavelettransformation~\eqref{cwt:definition:eq} bis auf den Faktor
$C_{\psi}$ eine Isometrie ist.
In Abschnitt~\ref{section:plancherel} wurde dies die Plancherel-Eigenschaft
der stetigen Wavelet-Transformation genannt und gezeigt, dass daraus
eine einfache Umkehrformel folgt.
Zu jeder Wavelet-Transformierten sollten daher die ursprüngliche
Funktionen wiedergewonnen werden können.
Ziel dieses Abschnitts ist die unabhängige Konstruktion einer Umkehrformel, 
die $f$ aus $\mathcal{W}f$ wiedergewinnt.
Es wird sich herausstellen, dass diese Umkehrformel genau der Formel
entspricht, die in Satz~\ref{satz:plancherel-prinzip} ermittelt wurde.

\begin{satz}
Unter geeigneten Stetitkeitsvoraussetzungen an die Funktion $f$ gilt
\begin{equation}
f(t) = \frac{1}{C_{\psi}}\int_{\mathbb R^*}\int_{-\infty}^\infty
\mathcal{W}f(a,b) \psi_{a,b}(t)
\,db\,\frac{da}{|a|^2}
\label{cwt:umkehr}
\end{equation}
\end{satz}

\begin{proof}[Beweis]
Wir haben zu zeigen, dass die Funktion
\begin{equation}
\tilde{f}(t) = \frac{1}{C_{\psi}}\int_{\mathbb R^*}\int_{-\infty}^\infty
\mathcal{W}f(a,b) \psi_{a,b}(t)
\,db\,\frac{da}{|a|^2}
\end{equation}
wohldefiniert ist, in $L^2(\mathbb R)$ liegt  und mit $f$ übereinstimmt.

Wir nehmen zunächst an, dass $\tilde{f}$ tatsächlich wohldefiniert
ist und eine $L^2$-Funktion ist.
Um zu zeigen, dass $\tilde{f}$ mit $f$ übereinstimmt, genügt es nach
dem Satz von Riesz~\ref{geometrie:satz:riesz}, zu zeigen, das
$\tilde{f}$ und $f$ das selbe Skalarprodukt mit beliebigen Funktionen
$g\in L^2$ hat.
Durch Vertauschung der Integrationsreihenfolge in~\eqref{cwt:umkehr:beweis1}
und Verwendung der Plancherel-Formel in~\eqref{cwt:umkehr:beweis2}
finden wir
\begin{align}
\langle \tilde{f},g\rangle
&=
\int_{-\infty}^\infty
\frac1{C_{\psi}} \int_{\mathbb R^*}\int_{-\infty}^\infty
\mathcal{W}f(a,b)
\psi_{a,b}(t)
\,db\,\frac{da}{|a|^2}
\bar{g}(t)
\,dt
\notag
\\
&=
\frac1{C_{\psi}} \int_{\mathbb R^*}\int_{-\infty}^\infty
\mathcal{W}f(a,b)
\int_{-\infty}^\infty
\bar{g}(t)
\psi_{a,b}(t)
\,dt
\,db\,\frac{da}{|a|^2}
\label{cwt:umkehr:beweis1}
\\
&=
\frac1{C_{\psi}} \int_{\mathbb R^*}\int_{-\infty}^\infty
\mathcal{W}f(a,b)
\underbrace{
\overline{
\int_{-\infty}^\infty
g(t)
\bar{\psi}_{a,b}(t)
\,dt
}}_{\displaystyle\mathcal{W}g(a,b)}
\,db\,\frac{da}{|a|^2}
\notag
\\
&=
\frac{1}{C_{\psi}}
\langle \mathcal{W}f, \mathcal{W}g \rangle_H
=
\langle f, g\rangle.
\label{cwt:umkehr:beweis2}
\end{align}
Die Funktion $\tilde{f}$ hat mit allen Funktion $g$ das gleiche Skalarprodukt
wie $f$ und daher
\[
\langle \tilde{f},g\rangle = \langle f,g\rangle
\quad\Rightarrow\quad
\langle \tilde{f}-f,g\rangle = 0.
\]
Für $g=\tilde{f}-f$ findet man
\[
=
\langle \tilde{f}-f,\tilde{f}-f\rangle
=
\|\tilde{f}-f\|^2
\quad\Rightarrow\quad
\tilde{f}=f.
\]

Die bisherigen Überlegungen haben nur gezeigt, dass die
Formel~\label{cwt:umkehr} eine sogenannte schwache Lösung des Umkehrproblems
liefern kann.
Der Nachweis, dass die Funktion $\tilde{f}$ existiert,
ist etwas technisch und kann
in \cite{buch:daubechies} nachgelesen werden.
\end{proof}





\section*{Übungsaufgaben}
\rhead{Übungsaufgaben}
\uebungsaufgabe{04001}

