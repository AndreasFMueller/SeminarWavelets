%
% muttervater.tex
%
% (c) 2019 Prof Dr Andreas Müller, Hochschule Rapperswil
%
\section{Mutterwavelet $\psi$ aus dem Vaterwavelet $\varphi$
\label{section:mutter-aus-vater}}
Lemma~\ref{lemma:msa:psivorschlag} stellt einen Zusammenhang zwischen
den Fouriertransformation von $\psi$ und $\varphi$ und der Skalierungsrelation
von $\varphi$ her.
Daraus können wir auch eine Relation gewinnen, die $\psi$ als
Linearkombination der $\varphi$ festlegt.
Dazu erinnern wir daran, dass per Definition
\[
H(\omega) = \sum_{k\in\mathbb Z} h_k e^{-ik\omega} 
\quad\Rightarrow\quad
\bar{H}(\omega)
=
\frac{1}{\sqrt{2}}
\sum_{k\in\mathbb Z} \bar{h}_k e^{ik\omega} 
\]
ist.
Setzen wir dies in die Relation~\eqref{msa:psivorschlag} ein, erhalten wir
\begin{align*}
\hat{\psi}(\omega)
&=
e^{i\omega/2}
\bar{H}\biggl(\frac{\omega}2+\pi\biggr)
\hat{\varphi}\biggl(\frac{\omega}2\biggr)
=
e^{i\omega/2}
\frac{1}{\sqrt{2}}
\sum_{k\in\mathbb Z} \bar{h}_k e^{ik\omega/2 + ik\pi} 
\hat{\varphi}\biggl(\frac{\omega}2\biggr)
=
\frac{1}{\sqrt{2}}
\sum_{k\in\mathbb Z} \bar{h}_k e^{i(k+1)\omega/2} (-1)^k
\hat{\varphi}\biggl(\frac{\omega}2\biggr)
\\
&=
\frac{1}{\sqrt{2}}
\sum_{k\in\mathbb Z}
(-1)^{l+1}
\bar{h}_{-l-1} e^{-il\omega/2}
\hat{\varphi}\biggl(\frac{\omega}2\biggr)
\\
\intertext{In dieser Form kann die Fourier-Transformation umgekehrt werden,
da die letzten zwei Faktoren Skalierungen und Translation von $\varphi$
beschreiben.
Es ergibt sich}
\psi(t)
&=
\sqrt{2}\sum_{l\in\mathbb Z} (-1)^{l+1}\bar{h}_{-l-1} \varphi(2t-k).
\end{align*}

\begin{lemma}
\label{lemma:msa:psirelation}
Das Mutterwavelet definiert durch die Fourier-Transformierte
\eqref{msa:psivorschlag}
ist die Linearkombination
\begin{equation}
\psi(t)
=
\sqrt{2}\sum_{l\in\mathbb Z} (-1)^{l+1}\bar{h}_{-l-1} \varphi(2t-k)
\end{equation}
von Translaten des Vaterwavelets $\varphi$.
\end{lemma}

Durch Vergleich mit~\eqref{msa:skalrel-g} kann die $g$-Koeffizienten als
\[
g_k = (-1)^{k+1} \bar{h}_{-k-1}
\]
ausdrücken.

Früher wurde bereits darauf hingewiesen, dass diese Wahl des Mutterwavelets
nicht eindeutig ist.
Zum Beispiel ist jede ganzzahlig verschobene Version dieses Mutterwavelets
ebenfalls ein adäquates Mutterwavelet.
Dies folgt auch schon aus der Definition der Multiskalen-Analyse.
Diese Freiheit ermöglicht zum Beispiel bei Wavelets mit kompaktem Träger,
den Träger von $\varphi$ mit dem Träger von $\psi$ zur Deckung zu
bringen.
Wir werden auf diese Möglichkeit in Kapitel~\ref{chapter:kompakt}
zurück kommen.


