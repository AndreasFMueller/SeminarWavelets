%
% orthonormalisierung.tex
%
% (c) 2019 Prof Dr Andreas Müller, Hochschule Rapperswil
%
\section{Orthonormalisierung
\label{section:orthonormalisierung}}
\rhead{Orthonormalisierung}
Die Funktion
\[
\varphi_{B^2}(t)
=
\begin{cases}
  t&\qquad 0\le t \le 1\\
2-t&\qquad 1\le t \le 2\\
  0&\qquad \text{sonst}
\end{cases}
\]
erfüllt die Skalierungsgleichung
\[
\varphi_{B^2}(t)
= 
\frac12\varphi_{B^2}(2t)
+
\varphi_{B^2}(2t-1)
+
\frac12\varphi_{B^2}(2t-2).
\]
Wir werden in Kapitel~\ref{chapter:spline} zeigen, dass sich daraus ein
geeignetes Mutterwavelet konstruieren lässt.
Eine grundlegende Schwierigkeit, die dabei gelöst werden muss, ist, dass
die Translate $\varphi_{B^2}(t-b)$ für $b\in\mathbb N$ nicht orthonormiert
sind.
Vielmehr ist das Skalarprodukt
\begin{align*}
\langle \varphi_{B^2}, T_{-1}\varphi_{B^2}\rangle
&=
\int_0^1 t\cdot (1-t)\,dt
=
\biggl[
\frac{t^2}{2}-\frac{t^3}{3}
\biggl]_0^1
=
\frac12-\frac13 = \frac16\ne 0
\end{align*}

% XXX TODO bild zu \varphi_{B^2} und der Berechnung des Skalarproduktes

In diesem Abschnitt soll daher die folgende Aufgabe gelöst werden
\begin{aufgabe}
\label{aufgabe:orthonormalisierung}
Gegeben $\varphi\in L^2(\mathbb R)$ konstruiere eine Funktion 
$\varphi_1\in L^2(\mathbb R)$, die eine Linearkombination von 
Translaten von $\varphi$ ist, also
\[
\varphi_1(t) = \sum_{k=-\infty}^\infty c_k \varphi(t-k)
\]
und so, dass die Translate orthonormiert sind.
\end{aufgabe}

In der Linearen Algebra lernt man üblicherweise das Verfahren von
Gram-Schmidt zur Orthonormalisierung einer Basis eines Vektorraumes.
Es beginnt beim ersten Vektor der nur in seiner Länge verändert wird.
Die folgenden Vektoren werden dann sukzessive modifiziert, so dass sie
auf den bisher behandelten Vektoren senkrecht stehen.
Dieses Vorgehen ist im vorliegenden Fall nicht angebracht, da die
entstehenden Vektoren nicht alle Translate des ersten Vektor sind.

\begin{proof}[Lösung]
Zunächst können wir annehmen, dass $\|\varphi\|=1$ ist.
\end{proof}


