%
% skalfour.tex
%
% (c) 2019 Prof Dr Andreas Müller, Hochschule Rapperswil
%
\section{Skalierungsrelation und Fouriertransformation
\label{section:skalfour}}
\rhead{Skalierungsrelation und Fouriertransformation}

% XXX TODO Funktion H und Skalierungsrelation

Gibt es überhaupt eine Funktion $\varphi$ derart, dass die Translate
$T_b\varphi$ alle orthogonal sind?
Ausser natürlich der charakteristischen Funktion des Einheitsintervals,
welche das Vater-Wavelet des Haar-Wavelets ist.
In diesem Abschnitt untersuchen wir diese Eigenschaft mit Hilfe der 
Fourier-Theorie.

%
% Interval-Trick: Aufteilung eines Fourier-Integrals in Teilintervalle
%                 für ganzzahlige Argumente
%
\subsection{Der Interval-Trick}
Wir werden im Folgenden wiederholt einenn Trick verwenden, mit dem
Fourier-Integrale über $\mathbb R$ in gewöhnliche Integrale
über ein endliches Interval umgeformt werden können.
Sei $f$ eine Funktion auf $\mathbb R$ und $b\in\mathbb Z$.
Da $b$ eine ganze Zahl ist, ist
\[
e^{ib(t+2\pi k)}
=
e^{ibt}\underbrace{e^{2\pi i bk}}_{\displaystyle = 1}
=
e^{ibt},
\]
da $bk$ ebenfalls eine ganze Zahl ist.
Dann kann man den Integrationsbereich $\mathbb R$ in Intervalle der
Länge $2\pi$ jeweils zwischen $2\pi k$ und $2\pi(k+1)$ aufteilen.
Für das Fourier-Integral folgt dann
\begin{align}
\int_{-\infty}^\infty f(t) e^{ibt}\,dt
&=
\sum_{k\in\mathbb Z} \int_{2\pi k}^{2\pi(k+1)} f(t) e^{ibt}\,dt
=
\sum_{k\in\mathbb Z} \int_{0}^{2\pi} f(t+2\pi k) e^{ibt}\,dt
=
\int_{0}^{2\pi} \biggl(\sum_{k\in\mathbb Z} f(t+2\pi k)\biggr) e^{ibt}\,dt
\label{msa:intervaltrick}
\end{align}
Der Klammerausdruck ist eine $2\pi$-periodische Funktion auf $\mathbb R$.
Wir schreiben dafür.
\[
\mathcal{P}f(t) = \sum_{k\in\mathbb Z} f(t+2\pi k).
\]
Da $b\in\mathbb Z$, berechnet das Integral~\eqref{msa:intervaltrick}
bis auf einen Faktor $2\pi$ die Fourier-Koeffizienten von $\mathcal{P}f$:
\[
\widehat{\mathcal{P}f}(b)
=
\frac1{2\pi}\int_{0}^{2\pi} \mathcal{P}f(t) e^{ibt}\,dt
=
\frac1{2\pi}
\int_{-\infty}^\infty f(t) e^{ibt}\,dt.
\]
Die Fourier-Koeffizienten der periodisch gemachten Funktion $\mathcal{P}f$ 
hängen also mit den Werte der Fourier-Transformation an ganzzahligen
Punkten zusammen.

%
% Orthogonalitätsbedingung für die Translate des Vater-Wavelets
%
\subsection{Orthogonalität der Translate von $\varphi$}
Die Funktion $\varphi$ hat die Bedingung zu erfüllen, dass
\[
\langle \varphi, T_b\varphi\rangle = \delta_{b0}
\qquad\forall b\in\mathbb Z.
\]
Wir verwenden die Plancherel-Formel für die Fourier-Transformation,
um dieses Skalarprodukt zu vereinfachen:
\begin{align*}
\langle \varphi,T_b\varphi\rangle
&=
\langle \hat{\varphi},\widehat{T_b\varphi}\rangle
=
\int_{-\infty}^\infty
\hat{\varphi}(\omega) e^{i\omega b}\overline{\hat{\varphi}(\omega)}
\,d\omega
=
\int_{-\infty}^\infty
|\hat{\varphi}(\omega)|^2 e^{i\omega b}
\,d\omega.
\\
\intertext{Dies ist genau ein Integral der Art~\eqref{msa:intervaltrick}.
Wir können es daher durch Fourier-Koeffizienten der periodische 
gemachten Funktion $\mathcal{P}|\hat{\varphi}|^2$ ausdrücken und
erhalten
}
&=
\int_0^{2\pi}
\mathcal{P}|\hat{\varphi}|^2(\omega) e^{ib\omega}
\,d\omega
=
\int_0^{2\pi}
\biggl(
\sum_{k\in\mathbb Z}
|\hat{\varphi}(\omega + 2\pi k)|^2\biggr)
e^{i\omega b}
\,d\omega.
\end{align*}
Die Orthogonalitätsbedingung lautet jetzt, dass 
\[
\langle \varphi,T_b\varphi\rangle
=
2\pi
\widehat{\mathcal{P}|\varphi|^2}(b)
=
\delta_{0b}.
\]
Die Funktion $\mathcal{P}|\varphi|^2$ ist also $2\pi$-periodisch und
derart, dass alle Fourier-Koeffizienten auss der Koeffizienten für
$b=0$ verschwinden.
Diese Funktion muss daher eine Konstante sein.

\begin{satz}
\label{satz:msa:orthogonalitaetsbedingung}
Damit die ganzzahligen Translate einer Funktion $\varphi$ alle orthogonal
sind, ist notwendig und hinreichend, dass 
\begin{equation}
\mathcal{P}|\hat{\varphi}|^2(\omega)
=
\sum_{k\in\mathbb Z} |\hat{\varphi}(\omega + 2\pi k)|^2
=
\frac1{2\pi}
\label{msa:orthogonalitaetsbedingung}
\end{equation}
für fast alle $\omega\in\mathbb R$.
\end{satz}

%
% Skalierungsrelation
%
\subsection{Die Skalierungsrelation}
Das Vater-Wavelet einer Multiskalen-Analyse muss einer Skalierungsrelation
genügen, die wir in der Form
\begin{equation}
\varphi(t)
=
\sqrt{2} \sum_{k\in\mathbb Z} h_k \varphi(2t-k)
\label{msa:skalt}
\end{equation}
schreiben.
Wir wenden darauf die Fourier-Transformation an und erhalten
\[
\hat{\varphi}(t)
=
\sqrt{2} \sum_{k\in\mathbb Z} h_k e^{-ik\omega/2} \frac12\hat{\varphi}\biggl(\frac{\omega}{2}\biggr)
=
\frac1{\sqrt{2}}
\biggl(\sum_{k\in\mathbb Z}h_ke^{-ik\omega/2}\biggr)
\hat{\varphi}\biggl(\frac{\omega}2\biggr)
\]

\begin{definition}
Schreiben wir 
\[
H(s)
=
\frac1{\sqrt{2}}
\sum_{k\in\mathbb Z}h_ke^{-iks},
\]
sie heisst die {\em erzeugende Funktion} einer Multiskalen-Analyse.
\end{definition}

Mit der erzeugenden Funktion $H(s)$ wird die
Skalierungsbedingung~\eqref{msa:skalt} im Zeitbereich
für das Vater-Wavelet zu der Bedingung
\begin{equation}
\hat{\varphi}(\omega) 
=
H\biggl(\frac{\omega}2\biggr)\,\hat{\varphi}\biggl(\frac{\omega}2\biggr)
\label{msa:skalomega}
\end{equation}
im Frequenzbereich.
Wir haben in \eqref{msa:skalomega}
also eine zusätzliche Bedingung zur Orthogonalitätsbedingung
\eqref{msa:orthogonalitaetsbedingung}.

Man beachte, dass die Funktion $H(s)$ ausschliesslich durch die
Koeffizienten der Skalierungsrelation bestimmt ist.
Die Funktionalgleichung~\eqref{msa:skalomega} deutet bereits an,
dass diese Koeffizienten die Funktion $\varphi$ eindeutig bestimmen
könnten.

Ein weiterer Schritt in diese Richtung ist, dass sich die
Orthogonalitätsbedingung~\eqref{msa:orthogonalitaetsbedingung}
Durch $H$ ausdrücken lässt.

\begin{satz}
Die erzeugende Funktion $H(s)$ einer Multiskalen-Analyse erfüllt die
Bedingung
\begin{equation}
|H(\omega)|^2 + |H(\omega+\pi)|^2 = 1
\label{msa:Hbed}
\end{equation}
für fast alle $\omega\in\mathbb R$.
\end{satz}

\begin{proof}[Beweis]
Wir wenden spalten die Summe in der Orthogonalitätsbedingung in zwei
Teilsummen mit geraden und ungeraden $k$ auf:
\begin{align*}
\frac{1}{2\pi}
&=
\sum_{k\in\mathbb Z} |\hat{\varphi}(\omega + 2\pi k)|^2
=
\sum_{k\in\mathbb Z} |\hat{\varphi}(\omega + 4\pi k)|^2
+
\sum_{k\in\mathbb Z} |\hat{\varphi}(\omega + 4\pi k + 2\pi)|^2
\\
\intertext{und wenden auf jeden Summanden die 
Skalierungsrelation~\eqref{msa:skalomega} an:}
&=
\sum_{k\in\mathbb Z}
\biggl|
H\biggl(\frac{\omega}2+2\pi k\biggr)
\hat\varphi\biggl(\frac{\omega}2 + 2\pi k\biggr)
\biggr|^2
+
\sum_{k\in\mathbb Z}
\biggl|
H\biggl(\frac{\omega}2 + \pi + 2\pi k\biggr)
\hat\varphi\biggl(\frac{\omega}2 + \pi+ 2\pi k\biggr)
\biggr|^2
\\
&=
\biggl|
H\biggl(\frac{\omega}2+2\pi k\biggr)
\biggr|^2
\sum_{k\in\mathbb Z}
\biggl|
\hat\varphi\biggl(\frac{\omega}2 + 2\pi k\biggr)
\biggr|^2
+
\biggl|
H\biggl(\frac{\omega}2 + \pi + 2\pi k\biggr)
\biggr|^2
\sum_{k\in\mathbb Z}
\biggl|
\hat\varphi\biggl(\frac{\omega}2 + \pi+ 2\pi k\biggr)
\biggr|^2
\intertext{Darin können wir die Summanden $2\pi k$ in den Argumenten von
$H$ weglassen, weil $H$ $2\pi$-periodisch ist.
$k$ verschwindet damit aus den Faktoren $H$, die daher aus der
Summe ausgeklammert werden können:}
\\
&=
\biggl|
H\biggl(\frac{\omega}2\biggr)
\biggr|^2
\mathcal{P}|\hat{\varphi}|^2\biggl(\frac{\omega}2\biggr)
+
\biggl|
H\biggl(\frac{\omega}2 + \pi\biggr)
\biggr|^2
\mathcal{P}|\hat{\varphi}|^2\biggl(\frac{\omega}2+\pi\biggr)
\\
\intertext{Aus der Orthogonalitätsbedingung folgt, dass die Terme
$\mathcal{P}|\hat{\varphi}|^2$ fast überall konstant sind:}
&=
\biggl|
H\biggl(\frac{\omega}2\biggr)
\biggr|^2
\cdot
\frac1{2\pi}
+
\biggl|
H\biggl(\frac{\omega}2 + \pi\biggr)
\biggr|^2
\cdot
\frac1{2\pi}.
\end{align*}
Dies gilt für fast alle $\omega$, also müssen die beiden $H$-Terme zusammen
$1$ geben.
Damit ist die Aussage \eqref{msa:Hbed} bewiesen.
\end{proof}

