%
% skalfour.tex
%
% (c) 2019 Prof Dr Andreas Müller, Hochschule Rapperswil
%
\section{Skalierungsrelation und Fouriertransformation
\label{section:skalfour}}
\rhead{Skalierungsrelation und Fouriertransformation}

Gibt es überhaupt eine Funktion $\varphi$ derart, dass die Translate
$T_b\varphi$ mit $b\ne 0$ alle orthogonal sind?
Lange Zeit war nicht klar, dass es ausser dem schon lange bekannten
Haar-Wavelet tatsächlich solche Funktionen gibt.
Yves Meyer versuchte sogar erst zu zeigen, dass das Haar-Wavelet das
einzige mit dieser Eigenschaft ist.
In diesem Abschnitt untersuchen wir diese Eigenschaft mit Hilfe der 
Fourier-Theorie.
In Abschnitt~\ref{section:orthonormalisierung} werden wir später sehen,
dass sich für fast beliebige Funktionen $f$ eine Linearkombination
$\varphi$ von Translaten von $f$ finden lässt, so dass die Translate
von $\varphi$ orthonormiert sind.

%
% Intervall-Trick: Aufteilung eines Fourier-Integrals in Teilintervalle
%                 für ganzzahlige Argumente
%
\subsection{Der Intervalltrick und der Periodisierungsoperator}
Wir werden im Folgenden wiederholt einen Trick verwenden, mit dem
Fourier-Integrale über $\mathbb R$ in gewöhnliche Integrale
über ein endliches Intervall umgeformt werden können.
Sei $f$ eine Funktion auf $\mathbb R$ und $b,\,k\in\mathbb Z$.
Da $b,\,k$ ganze Zahlen sind, ist
\[
e^{ib(t+2\pi k)}
=
e^{ibt}\underbrace{e^{2\pi i bk}}_{\displaystyle = 1}
=
e^{ibt},
\]
da $bk$ ebenfalls eine ganze Zahl ist.
Dann kann man den Integrationsbereich $\mathbb R$ in Intervalle der
Länge $2\pi$ jeweils zwischen $2\pi k$ und $2\pi(k+1)$ aufteilen.
Für das Fourier-Integral folgt dann
\begin{align}
\int_{-\infty}^\infty f(t) e^{-ibt}\,dt
&=
\sum_{k\in\mathbb Z} \int_{2\pi k}^{2\pi(k+1)} f(t) e^{-ibt}\,dt
=
\sum_{k\in\mathbb Z} \int_{0}^{2\pi} f(t+2\pi k) e^{-ibt}\,dt
\notag
\\
&=
\int_{0}^{2\pi} \biggl(\sum_{k\in\mathbb Z} f(t+2\pi k)\biggr) e^{-ibt}\,dt.
\label{msa:intervalltrick}
\end{align}
Der Klammerausdruck ist eine $2\pi$-periodische Funktion auf $\mathbb R$.
Wir werden diese Operation wiederholt benötigen und spendieren ihr daher
eine formelle Definition

\begin{definition}
\label{msa:peri}
Für eine Funktion $f\in L^2(\mathbb R)$ ist die periodisch gemachte Funktion
\[
\mathcal{P}f(t) = \sum_{k\in\mathbb Z} f(t+2\pi k)
\]
eine Funktion in $L^2([0,2\pi])$.
$\mathcal{P}$ heisst Periodisierungsoperator.
\end{definition}
\index{Periodisierungsoperator}%

\begin{proof}[Beweis]
Die Definition behauptet, dass die Operation $\mathcal{P}$ für
quadratintegrierbare Funktionen auf $\mathbb R$ eine quadratintegrierbare
Funktion $2\pi$-periodische Funktion ergibt.
Wir rechnen dazu nach
\begin{align*}
\int_{0}^{2\pi}
| \mathcal{P}f(t)|^2
\,dt
&=
\int_{0}^{2\pi}
\biggl|
\sum_{k\in\mathbb Z}f(t+2\pi k)
\biggr|^2
\,dt
\le
\int_{-\infty}^\infty |f(t)|^2 \,dt
=
\| f\|^2.
\qedhere
\end{align*}
\end{proof}

Da $b\in\mathbb Z$ ist, berechnet das Integral~\eqref{msa:intervalltrick}
bis auf einen Faktor $2\pi$ die Fourier-Koeffizienten von $\mathcal{P}f$.
Wir haben daher den folgenden Satz gefunden:

\begin{satz}
\label{msa:Pfourier}
Ist $f\in L^2(\mathbb R)$, dann sind die Fourier-Koeffizienten von
$\mathcal{P}f$
\[
\widehat{\mathcal{P}f}(b)
=
\frac1{2\pi}\int_{0}^{2\pi} \mathcal{P}f(t) e^{-ibt}\,dt
=
\frac1{2\pi}
\int_{-\infty}^\infty f(t) e^{-ibt}\,dt
=
\frac{1}{\sqrt{2\pi}} \hat{f}(b).
\]
\end{satz}

Die Fourier-Koeffizienten der periodisch gemachten Funktion $\mathcal{P}f$ 
hängen also mit den Werten der Fourier-Transformation an ganzzahligen
Punkten zusammen.

\begin{lemma}
Der Periodisierungsoperator $\mathcal{P}$ ist linear:
\begin{align*}
\mathcal{P}(f+g)(t)
&=
\mathcal{P}f(t) + \mathcal{P}g(t),
\\
\mathcal{P}(\lambda f)(t)
&=
\lambda \mathcal{P}f(t).
\end{align*}
\end{lemma}

\begin{proof}[Beweis]
Diese Identitäten können durch Nachrechnen verifiziert werden:
\begin{align*}
\mathcal{P}(f+g)(t)
&=
\sum_{k\in\mathbb Z}(f+g)(t+2\pi k)
=
\sum_{k\in\mathbb Z}\bigl(f(t+2\pi k)+ g(t+2\pi k)\bigr)
\\
&=
\sum_{k\in\mathbb Z}\bigl(f(t+2\pi k)
+
\sum_{k\in\mathbb Z}g(t+2\pi k)\bigr)
\\
&=
\mathcal{P}f(t)
+
\mathcal{P}g(t),
\\
\mathcal{P}(\lambda f)(t)
&=
\sum_{k\in\mathbb Z}(\lambda f)(t+2\pi k)
=
\sum_{k\in\mathbb Z}\lambda f(t+2\pi k)
=
\lambda
\sum_{k\in\mathbb Z}f(t+2\pi k)
=
\lambda \mathcal{P}f(t).
\qedhere
\end{align*}
\end{proof}

%
% Orthogonalitätsbedingung für die Translate des Vaterwavelets
%
\subsection{Orthogonalität der Translate von $\varphi$}
Die Funktion $\varphi$ hat die Bedingung zu erfüllen, dass
\[
\langle \varphi, T_b\varphi\rangle = \delta_{b0}
\qquad\forall b\in\mathbb Z.
\]
Wir verwenden die Plancherel-Formel für die Fourier-Transformation,
um dieses Skalarprodukt zu vereinfachen:
\begin{align*}
\langle \varphi,T_b\varphi\rangle
&=
\langle \hat{\varphi},\widehat{T_b\varphi}\rangle
=
\int_{-\infty}^\infty
\hat{\varphi}(\omega) e^{i\omega b}\overline{\hat{\varphi}(\omega)}
\,d\omega
=
\int_{-\infty}^\infty
|\hat{\varphi}(\omega)|^2 e^{i\omega b}
\,d\omega.
\\
\intertext{Dies ist genau ein Integral der Art~\eqref{msa:intervalltrick}.
Wir können es daher durch Fourier-Koeffizienten der periodische 
gemachten Funktion $\mathcal{P}|\hat{\varphi}|^2$ ausdrücken und
erhalten
}
&=
\int_0^{2\pi}
\mathcal{P}|\hat{\varphi}|^2(\omega) e^{-ib\omega}
\,d\omega
=
\int_0^{2\pi}
\biggl(
\sum_{k\in\mathbb Z}
|\hat{\varphi}(\omega + 2\pi k)|^2\biggr)
e^{-i\omega b}
\,d\omega.
\intertext{Die Orthogonalitätsbedingung lautet jetzt}
\langle \varphi,T_b\varphi\rangle
&=
2\pi
\widehat{\mathcal{P}|\hat{\varphi}|^2}(-b)
=
\delta_{0b}.
\end{align*}
\index{Orthgonalitätsbedingung}%
Die Funktion $\mathcal{P}|\hat{\varphi}|^2$ ist also $2\pi$-periodisch und
derart, dass alle Fourier-Koeffizienten ausser dem Koeffizienten für
$b=0$ verschwinden.
Diese Funktion muss daher eine Konstante sein.
Damit haben wir den folgenden Satz bewiesen.

\begin{satz}
\label{satz:msa:orthogonalitaetsbedingung}
Damit die ganzzahligen Translate einer Funktion $\varphi\in L^2$ alle
orthogonal sind, ist notwendig und hinreichend, dass 
\begin{equation}
\mathcal{P}|\hat{\varphi}|^2(\omega)
=
\sum_{k\in\mathbb Z} |\hat{\varphi}(\omega + 2\pi k)|^2
=
\frac1{2\pi}
\label{msa:orthogonalitaetsbedingung}
\end{equation}
ist für fast alle $\omega\in\mathbb R$.
\end{satz}
\index{Orthogonalitätsbedingung}

Dieselbe Idee lässt sich auch auf Funktionen anwenden, die orthogonal
sind zu allen Translaten.

\begin{satz}
\label{satz:msa:alleorthogonal}
Seien $\varphi,\psi\in L^2$, dann ist $\varphi$ orthogonal zu allen
ganzzahligen Translaten $T_b\psi$  von $\psi$ genau dann, wenn
\begin{equation}
\sum_{k\in\mathbb Z} \hat{\varphi}(\omega+2\pi k)\overline{\hat{\psi}(\omega+2\pi k)}
=
0
\end{equation}
für fast alle $\omega\in\mathbb R$.
\end{satz}

\begin{proof}[Beweis]
Nach Voraussetzung gilt für alle $b\in\mathbb Z$
\begin{align*}
0
&=
\langle \varphi,T_b\psi\rangle
=
\langle \hat{\varphi}, \widehat{T_b\psi}\rangle
=
\int_{-\infty}^\infty
\hat{\varphi}(\omega)\, \overline{\hat{\psi}(\omega)} e^{ib\omega}
\,d\omega
\\
&=
2\pi
\widehat{\mathcal{P}(\hat{\varphi}\bar{\hat{\psi}})}(b).
\end{align*}
Die $2\pi$-periodische Funktion $\mathcal{P}(\hat{\varphi}\bar{\hat{\psi}})$ 
hat also lauter verschwindende Fourier-Koeffizienten, also verschwindet
sie.
Nach Definition ist
\[
\mathcal{P}(\hat{\varphi}\bar{\hat{\psi}})
=
\sum_{k\in\mathbb Z}
\hat{\varphi}(\omega + 2\pi k)
\bar{\hat{\psi}}(\omega + 2\pi k)
=0.
\]
Dies beweist die Aussage.
\end{proof}

%
% Skalierungsrelation
%
\subsection{Die Skalierungsrelation
\label{msa:skal}}
Das Vaterwavelet einer Multiskalenanalyse muss einer Skalierungsrelation
genügen, die wir in der Form
\begin{equation}
\varphi(t)
=
\sqrt{2} \sum_{k\in\mathbb Z} h_k \varphi(2t-k)
\label{msa:skalt}
\end{equation}
schreiben.
\index{Skalierungsrelation}
Wir wenden darauf die Fourier-Transformation an und erhalten
\[
\hat{\varphi}(t)
=
\sqrt{2} \sum_{k\in\mathbb Z} h_k e^{-ik\omega/2} \frac12\hat{\varphi}\biggl(\frac{\omega}{2}\biggr)
=
\frac1{\sqrt{2}}
\biggl(\sum_{k\in\mathbb Z}h_ke^{-ik\omega/2}\biggr)
\hat{\varphi}\biggl(\frac{\omega}2\biggr).
\]

\begin{definition}
\label{definition:erzeugende-funktion-msa}
Wir schreiben 
\[
H(s)
=
\frac1{\sqrt{2}}
\sum_{k\in\mathbb Z}h_ke^{-iks},
\]
für die sogenannte {\em erzeugende Funktion} einer Multiskalenanalyse.
\end{definition}
\index{erzeugende Funktion einer MSA}

Mit der erzeugenden Funktion $H(s)$ wird die
Skalierungsbedingung~\eqref{msa:skalt} im Zeitbereich
für das Vaterwavelet zu der Bedingung
\begin{equation}
\hat{\varphi}(\omega) 
=
H\biggl(\frac{\omega}2\biggr)\,\hat{\varphi}\biggl(\frac{\omega}2\biggr)
\label{msa:skalomega}
\end{equation}
im Frequenzbereich.
Wir haben in \eqref{msa:skalomega}
also eine zusätzliche Bedingung zur Orthogonalitätsbedingung
\eqref{msa:orthogonalitaetsbedingung}.

Man beachte, dass die Funktion $H(s)$ ausschliesslich durch die
Koeffizienten der Skalierungsrelation bestimmt ist.
Die Funktionalgleichung~\eqref{msa:skalomega} deutet bereits an,
dass diese Koeffizienten die Funktion $\varphi$ eindeutig bestimmen
könnten.

Ein weiterer Schritt in diese Richtung ist, dass sich
aus der Orthogonalität der $T_b\varphi$ und der
Orthogonalitätsbedingung~\eqref{msa:orthogonalitaetsbedingung}
eine Orthogonalitätsbedingung für $H$ gewinnen lässt.
Dies geschieht im folgenden Satz.

\begin{satz}
\label{satz:Hbed}
Die erzeugende Funktion $H(s)$ einer Multiskalenanalyse erfüllt die
Bedingung
\begin{equation}
|H(\omega)|^2 + |H(\omega+\pi)|^2 = 1
\label{msa:Hbed}
\end{equation}
für fast alle $\omega\in\mathbb R$.
\end{satz}

\begin{proof}[Beweis]
Wir spalten die Summe in der Orthogonalitätsbedingung in zwei
Teilsummen mit geraden und ungeraden $k$ auf:
\begin{align*}
\frac{1}{2\pi}
&=
\sum_{k\in\mathbb Z} |\hat{\varphi}(\omega + 2\pi k)|^2
=
\sum_{k\in\mathbb Z} |\hat{\varphi}(\omega + 4\pi k)|^2
+
\sum_{k\in\mathbb Z} |\hat{\varphi}(\omega + 4\pi k + 2\pi)|^2
\\
\intertext{und wenden auf jeden Summanden die 
Skalierungsrelation~\eqref{msa:skalomega} an:}
\frac{1}{2\pi}
&=
\sum_{k\in\mathbb Z}
\biggl|
H\biggl(\frac{\omega}2+2\pi k\biggr)
\hat\varphi\biggl(\frac{\omega}2 + 2\pi k\biggr)
\biggr|^2
+
\sum_{k\in\mathbb Z}
\biggl|
H\biggl(\frac{\omega}2 + \pi + 2\pi k\biggr)
\hat\varphi\biggl(\frac{\omega}2 + \pi+ 2\pi k\biggr)
\biggr|^2
\\
&=
\biggl|
H\biggl(\frac{\omega}2\biggr)
\biggr|^2
\sum_{k\in\mathbb Z}
\biggl|
\hat\varphi\biggl(\frac{\omega}2 + 2\pi k\biggr)
\biggr|^2
+
\biggl|
H\biggl(\frac{\omega}2 + \pi\biggr)
\biggr|^2
\sum_{k\in\mathbb Z}
\biggl|
\hat\varphi\biggl(\frac{\omega}2 + \pi+ 2\pi k\biggr)
\biggr|^2.
\intertext{Darin können wir die Summanden $2\pi k$ in den Argumenten von
$H$ weglassen, weil $H$ $2\pi$-periodisch ist.
$k$ verschwindet damit aus den Faktoren $H$, die daher aus der
Summe ausgeklammert werden können:}
\frac{1}{2\pi}
&=
\biggl|
H\biggl(\frac{\omega}2\biggr)
\biggr|^2
\mathcal{P}|\hat{\varphi}|^2\biggl(\frac{\omega}2\biggr)
+
\biggl|
H\biggl(\frac{\omega}2 + \pi\biggr)
\biggr|^2
\mathcal{P}|\hat{\varphi}|^2\biggl(\frac{\omega}2+\pi\biggr).
\\
\intertext{Aus der
Orthogonalitätsbedingung~\eqref{msa:orthogonalitaetsbedingung} folgt,
dass die Terme
$\mathcal{P}|\hat{\varphi}|^2$ fast überall konstant sind:}
\frac{1}{2\pi}
&=
\biggl|
H\biggl(\frac{\omega}2\biggr)
\biggr|^2
\cdot
\frac1{2\pi}
+
\biggl|
H\biggl(\frac{\omega}2 + \pi\biggr)
\biggr|^2
\cdot
\frac1{2\pi}.
\end{align*}
Dies gilt für fast alle $\omega$, also müssen die beiden $H$-Terme zusammen
$1$ geben.
Damit ist die Aussage \eqref{msa:Hbed} bewiesen.
\end{proof}

\subsection{Skalierungsrelationen für $f\in V_1$
\label{msa:skalv1}}
Die Multiskalenanalyse besagt, dass $V_0 \oplus W_0 = V_1$ und dass
$\varphi$ ein orthogonale Basis von $V_0$ ist.
Eine Funktion $f\in V_1$ muss dann eine Linearkombination von
skalierten Funktionen $\varphi_{1,b}(t)=\varphi(2t-b)$ sein.
Es gibt also $f_b\in\mathbb C$ mit
\[
f(t) = \sum_{b\in\mathbb Z} f_b \varphi_{1,b}(t).
\]
Die Fourier-Transformierte davon ist
\begin{equation}
\hat{f}(\omega)
=
\sum_{b\in\mathbb Z}
f_k
\frac{1}{\sqrt{2}}
e^{ib\omega/2}
\hat{\varphi}
\biggl(\frac{\omega}2\biggr).
\label{msa:skalfour:fhat}
\end{equation}
Der Faktor $\hat{\varphi}$ hängt nicht von $b$ ab und kann ausgeklammert
werden.
Für die verbleibende Funktion führen wir folgende Bezeichnung ein.

\begin{definition}
Für eine $f\in V_1$ sei $m_f$ definiert durch
\[
m_f(\omega)
=
\frac{1}{\sqrt{2}} \sum_{b\in\mathbb Z} f_b e^{-ib\omega}.
\]
\end{definition}

Die Gleichung~\eqref{msa:skalfour:fhat} lässt sich jetzt als Produkt
schreiben.

\begin{lemma}
\label{msa:mfskal}
Für eine Funktion $f\in V_1$ gilt
\begin{equation}
\hat{f}(\omega)
=
m_f\biggl(\frac{\omega}2\biggr)
\hat{\varphi}\biggl(\frac{\omega}2\biggr)
\label{msa:mfskaleq}
\end{equation}
für fast alle $\omega\in\mathbb R$.
\end{lemma}

Die Funktion $H$ ist natürlich nichts anders als der Spezialfall
des Vaterwavelets $f=\varphi\in V_0$, also $H=m_{\varphi}$.

In der Multiskalenanalyse bilden die Translate des Mutterwavelet $\psi$
ein orthonormierte Basis von $W_0$, insbesondere sind sie orthogonal
zu $V_0$.
Die Funktion $m_{\psi}$ erfüllt daher nicht nur \eqref{msa:mfskaleq},
es muss ausserdem eine Relation geben, die ausdrückt, dass alle Translate auf
$V_0$ orthogonal sind.
Dieser Zusammenhang wird gegeben durch das folgende Lemma.

\begin{lemma}
Eine Funktion $f\in V_1$ ist genau dann in $W_0$, wenn 
\[
m_f(\omega + \pi)\overline{H(\omega + \pi)}
+
m_f(\omega)\overline{H(\omega)}
=
0
\]
für fast alle $\omega\in\mathbb R$.
\end{lemma}

\begin{proof}[Beweis]
Die Funktion $f\in V_1$ ist genau dann in $W_0$, wenn $f$ orthogonal ist
auf allen Translaten von $\varphi$.
Nach Satz~\ref{satz:msa:alleorthogonal} ist dies gleichbedeutend mit
\[
\sum_{k\in\mathbb Z}
\hat{f}(\omega+2\pi k)
\overline{\hat{\varphi}(\omega+2\pi k)}
=
0.
\]
Die Summe auf der rechten Seite kann wieder in je eine Summe für die
Geraden und die ungeraden $k$ aufgeteilt werden.
\begin{align*}
0
&=
\sum_{k\in\mathbb Z}
\hat{f}(\omega+2\pi+4\pi k)
\overline{\hat{\varphi}(\omega+2\pi+4\pi k)}
+
\sum_{k\in\mathbb Z}
\hat{f}(\omega+4\pi k)
\overline{\hat{\varphi}(\omega+4\pi k)}.
\\
\intertext{Die Terme auf der rechten Seite können mit Hilfe von
Lemma~\ref{msa:skalomega} für $H$ und Lemma~\ref{msa:mfskal} für $m_f$
durch Ausdrücke mit $H$ und $m_f$ ersetzt werden.
Da sowohl $m_f$ als auch $H$ $2\pi$-periodisch sind, können sie aus
der Summe ausgeklammert werden:}
&=
\sum_{k\in\mathbb Z}
m_f\biggl(\frac{\omega}2+\pi\biggr)
\hat{\varphi}\biggl(\frac{\omega}2+\pi+2\pi k\biggr)
\bar{H}\biggl(\frac{\omega}2+\pi\biggr)
\bar{\hat{\varphi}}\biggl(\frac{\omega}2+\pi+2\pi k\biggr)
\\
&\qquad
+
\sum_{k\in\mathbb Z}
m_f\biggl(\frac{\omega}2\biggr)
\hat{\varphi}\biggl(\frac{\omega}2+2\pi k\biggr)
\bar{H}\biggl(\frac{\omega}2\biggr)
\bar{\hat{\varphi}}\biggl(\frac{\omega}2+2\pi k\biggr)
\\
&=
m_f\biggl(\frac{\omega}2+\pi\biggr)
\bar{H}\biggl(\frac{\omega}2+\pi\biggr)
\underbrace{
\sum_{k\in\mathbb Z}
\biggl|\hat{\varphi}\biggl(\frac{\omega}2+\pi+2\pi k\biggr)\biggr|^2
}_{\displaystyle = \frac{1}{2\pi}}
+
m_f\biggl(\frac{\omega}2\biggr)
\bar{H}\biggl(\frac{\omega}2\biggr)
\underbrace{
\sum_{k\in\mathbb Z}
\biggl|\hat{\varphi}\biggl(\frac{\omega}2+2\pi k\biggr)\biggr|^2
}_{\displaystyle = \frac{1}{2\pi}}
\\
&=
\biggl(
m_f\biggl(\frac{\omega}2+\pi\biggr)
\bar{H}\biggl(\frac{\omega}2+\pi\biggr)
+
m_f\biggl(\frac{\omega}2\biggr)
\bar{H}\biggl(\frac{\omega}2\biggr)
\biggr)
\cdot
\frac{1}{2\pi},
\end{align*}
woraus die Behauptung folgt.
\end{proof}

%
% Skalierungsrelation für \psi
%
\subsection{Skalierungsrelation für $\psi$}
Die Axiome der Multiskalenanalyse besagen, dass es eine Funktion $\psi$
gibt derart, dass die Translate von $\psi$ eine orthonormierte Basis
von $W_0$ sind.
In den Sätzen dieses Abschnitts wurde genau dieser Sachverhalt
bereits
in Abschnitt~\ref{msa:skal}
für das Vaterwavelet $\varphi$
sowie
in Abschnitt~\ref{msa:skalv1}
ganz allgemein für eine Funktion $f\in V_1$
untersucht.
Für das Mutterwavelet $\psi$ müssen diese Eigenschaften alle
auch gelten.
Wir fassen diese im folgenden Satz zusammen.

\begin{satz}
Die Funktion $m_\psi$, die zum Mutterwavelet einer Multiskalenanalyse 
gehört, erfüllt
\[
m_{\psi}\biggl(\frac{\omega}2+\pi\biggr)
\bar{H}\biggl(\frac{\omega}2+\pi\biggr)
+
m_{\psi}\biggl(\frac{\omega}2\biggr)
\bar{H}\biggl(\frac{\omega}2\biggr)
=
0.
\]
Wenn die Translate von $\psi$ orthonormiert sind, dann gilt
zusätzlich
\[
\biggl|m_\psi\biggl(\frac{\omega}2+\pi\biggr)\biggr|^2
+
\biggl|m_\psi\biggl(\frac{\omega}2\biggr)\biggr|^2
=
1.
\]
\end{satz}

Ausserdem gilt natürlich immer noch die Relation~\eqref{msa:Hbed}
für die Funktion $H(\omega)$.
Diese Bedingungen schränken stark ein, was $\psi$ überhaupt sein
kann.
Wir zeigen zunächst, wie sich aus $\varphi$ eine Funktion $\psi$
konstruieren lässt, die als Mutterwavelet in Frage kommt.

\begin{lemma}
\label{lemma:msa:psivorschlag}
Ist $\varphi$ das Vaterwavelet einer Multiskalenanalyse, dann sind die
Translate der Funktion mit der Fourier-Transformierten
\begin{equation}
\hat{\psi}(\omega)
=
e^{i\omega/2}
\overline{H\biggl(\frac{\omega}2+\pi\biggr)}
\hat{\varphi}\biggl(\frac{\omega}2\biggr)
\label{msa:psivorschlag}
\end{equation}
orthonormiert.
\end{lemma}

Bis auf weiteres meinen wir im Folgenden immer dieses $\psi$, wenn wir
vom Mutterwavelet einer Multiskalenanalyse sprechen.
Wir werden weiter unten untersuchen, wie andere mögliche Funktionen 
$\psi$ aussehen könnten.

\begin{proof}[Beweis]
Die Translate sind orthonormiert, wenn die
Bedingung~\eqref{msa:orthogonalitaetsbedingung}
für $\psi$ erfüllt ist.
Setzen wir \eqref{msa:psivorschlag} ein, erhalten wir
\begin{align*}
\sum_{k\in\mathbb Z} |\hat{\psi}(\omega+2\pi k)|^2
&=
\sum_{k\in\mathbb Z} |\hat{\psi}(\omega + 4\pi k)|^2
+
\sum_{k\in\mathbb Z} |\hat{\psi}(\omega + 4\pi k + 2\pi)|^2
\\
&=
\sum_{k\in\mathbb Z}
\biggl| H\biggl(\frac{\omega}2+2\pi k\biggr)\biggr|^2
\biggl| \hat{\varphi}\biggl(\frac{\omega}2 + 2\pi k\biggr) \biggr|^2
+
\sum_{k\in\mathbb Z}
\biggl|H\biggl(\frac{\omega}2+2\pi k+ \pi\biggr) \biggr|^2
\biggl|\hat{\varphi}\biggl(\frac{\omega}2 + 2\pi k + \pi\biggr) \biggr|^2
\\
&=
\biggl|H\biggl(\frac{\omega}2\biggr) \biggr|^2
\underbrace{
\sum_{k\in\mathbb Z}
\biggl|\hat{\varphi}\biggl(\frac{\omega}2 + 2\pi k\biggr)  \biggr|^2
}_{\displaystyle\frac{1}{2\pi}}
+
\biggl|H\biggl(\frac{\omega}2+\pi\biggr) \biggr|^2
\underbrace{
\sum_{k\in\mathbb Z}
\biggl|\hat{\varphi}\biggl(\frac{\omega}2 + 2\pi k + \pi\biggr) \biggr|^2
}_{\displaystyle\frac{1}{2\pi}}
\\
&=
\biggl(
\biggl|H\biggl(\frac{\omega}2\biggr) \biggr|^2
+
\biggl|H\biggl(\frac{\omega}2+\pi\biggr) \biggr|^2
\biggr)
\cdot
\frac{1}{2\pi}
=
\frac{1}{2\pi}
\end{align*}
fast überall in $\mathbb R$.
Dabei haben wir auf der dritten Zeile die
Orthonormalitätsbedingung~\eqref{msa:orthogonalitaetsbedingung} verwendet.
Folglich erfüllt auch $\psi$
die Orthonormalitätsbedingung~\eqref{msa:orthogonalitaetsbedingung}.
\end{proof}

Für eine Multiskalenanalyse muss aber noch mehr gezeigt werden.
Es muss gezeigt werden, dass die Translate von $\psi$ eine
Basis von $W_0$ bilden.
Ausserdem ist zu untersuchen, wieviel Freiheit bei der Wahl von 
$\hat{\psi}$ in \eqref{msa:psivorschlag} besteht.

Die Relation zwischen $H$ und $m_f$ mit $f\in W_0$, wozu auch $f=\psi$
zu zählen ist, kann etwas kompakter ausgedrückt werden, in dem man sie
als komplexe zweidimensionalen Vektoren
\[
\vec{m}_{\psi}(\omega)
=
\begin{pmatrix}
m_{f}(\omega)\\
m_{f}(\omega+\pi)\\
\end{pmatrix}
\qquad\text{und}\qquad
\vec{H}(\omega)
=
\begin{pmatrix}
H(\omega)\\
H(\omega + \pi)
\end{pmatrix}
\]
schreibt.
Die Relationen für $H$ und $m_f$ sind dann nichts anderes  als
Orthogonalitätsrelationen für die beiden Vektoren $\vec{H}$ und
$\vec{m}_f$:
\begin{align}
|\vec{H}(\omega)|^2
&=
|H(\omega)|^2 + |H(\omega+\pi)|^2 = 1,
\label{msa:vektorskalar1}
\\
|\vec{m}_{\psi}|^2
&=
|m_{\psi}(\omega)|^2 + |m_{\psi}(\omega+\pi)|^2 = 1\quad\text{und}
\label{msa:vektorskalar2}
\\
\vec{m}_{\psi}\cdot\vec{H}
&=
m_{f}(\omega)\bar{H}(\omega)
+
m_{f}(\omega+\pi)\bar{H}(\omega+\pi)
=
0
\label{msa:vektorskalar3}
\end{align}
für fast alle $\omega\in\mathbb R$.
Die Orthonormierungsbedingung~\eqref{msa:vektorskalar2} gilt natürlich
nur für solche Funktionen in $W_0$, deren Translate orthonormiert sind,
eben zum Beispiel für $\psi$.
Für $\psi$ kommen also nur Funktionen in Frage, deren $m_{\psi}$ 
fast überall zu einem auf $\vec{H}$ orthogonalen Einheitsvektor
$\vec{m}_\psi$ Anlass geben.

Wir suchen daher zu einem vorgegebenen Vektor $\vec{v}\in\mathbb C^2$
alle Vektoren $\vec{u}\in\mathbb C^2$, die auf $\vec{v}$ senkrecht stehen.
Solche Vektoren erfüllen $\vec{u}\cdot\vec{v}=0$, also die homogene
lineare Gleichung
\[
\bar{v}_1 {\color{red}u_1} + \bar{v}_2 {\color{red} u_2} =0 
\]
(die Unbekannte ist {\color{red}rot} hervorgehoben).
Falls $v_1\ne 0$ ist, kann man nach $u_1$ auflösen und erhält
$u_1= -(\bar{v}_2/\bar{v}_1)\cdot u_2$.
Andernfalls folgt $u_2=0$ in beiden Fällen kann man die Lösungsmenge
als
\[
\biggl\{
\lambda
\begin{pmatrix}\bar{v}_2\\-\bar{v}_1\end{pmatrix}
\,\bigg|
\,\lambda\in\mathbb C
\biggr\}
\]
schreiben.

\begin{lemma}
\label{lemma:msa:pperiodisch}
Für $f\in W_0$ gibt es eine $\pi$-periodische Funktion $p(\omega)$ derart,
dass 
\begin{equation*}
\hat{f}(\omega)
=
p(\omega) \hat{\psi}(\omega).
\end{equation*}
\end{lemma}

\begin{proof}[Beweis]
Zunächst folgt aus obigen Überlegungen zu den Vektoren $\vec{m}_f$ und
$\vec{H}$, dass es einen Faktor $\lambda(\omega)$ mit
$m_f(\omega)=\lambda(\omega)\bar{H}(\omega)$ geben muss, der die
Bedingung $\lambda(\omega)=-\lambda(\omega+\pi)$ erfüllen muss.
Eine möglich Lösung für diese Funktionalgleichung ist
$\lambda(\omega)=e^{i\omega}$.
Setzen wir $p(\omega) = \lambda(\omega)e^{-i\omega}$, dann folgt
\[
p(\omega+\pi) = \lambda(\omega+\pi)e^{-i\omega-i\pi}
=-\lambda(\omega)e^{-i\omega} (-1) = p(\omega),
\]
die Funktion $p(\omega)$ ist daher $\pi$-periodisch.
Aus der Festlegung~\eqref{msa:psivorschlag} für $\hat{\psi}$ folgt
\[
\hat{f}(\omega)= p(\omega)\hat{\psi}(\omega),
\]
wie behauptet.
\end{proof}

\begin{lemma}
Jede Funktion $f\in W_0$ ist eine Linearkombination von Translaten von $\psi$.
Es gibt also Koeffizienten $p_k\in\mathbb Z$ derart, dass 
\[
f = \sum_{k\in\mathbb Z} p_k\,T_k\psi.
\]
\end{lemma}

\begin{proof}[Beweis]
Nach Lemma~\ref{lemma:msa:pperiodisch} gibt es eine $\pi$-periodische
Funktion $p(\omega)$ derart, dass $\hat{f}(\omega)=p(\omega)\hat{\psi}(\omega)$.
Diese lässt sich in eine Fourier-Reihe 
\[
p(\omega) = \sum_{k\in\mathbb Z} p_k e^{ik\omega} 
\]
entwickeln.
Setzt man dies ein, erhält man
\[
\hat{f}(\omega)
=
\sum_{k\in\mathbb Z} p_k e^{ik\omega} \hat{\psi}(\omega)
=
\sum_{k\in\mathbb Z} p_k \widehat{T_k\psi}(\omega)
\]
oder nach Fourier-Rücktransformation
\[
f(t) = \sum_{k\in\mathbb Z} p_k T_k\psi(t),
\]
wie behauptet.
\end{proof}

\begin{lemma}
Die Translate von $\psi$ bilden eine Basis von $W_0$.
\end{lemma}

Wenn wir in obiger Überlegung zu den auf $\vec{v}$ in $\mathbb C^2$
orthogonalen Vektoren nur an Einheitsvektoren $\vec{u}$
interessiert sind,
dann muss der Faktor $\lambda$ eine Einheit sein.
Der Faktor $p(\omega)$ muss daher eine $\pi$-periodische Funktion
sein, die überall den Betrag $1$ hat.

%
% Eindeutigkeit von $\varphi$
%
\subsection{Eindeutigkeit von $\varphi$}
Die Skalierungsrelation \eqref{msa:skalt} für die Funktion $\varphi$ hat auf
die Relation \eqref{msa:skalomega} für deren Fouriertransformation
$\hat{\varphi}$ geführt.
Iteriert man \eqref{msa:skalomega}, erhält man
\begin{align*}
\hat{\varphi}(\omega)
&=
H\biggl(\frac{\omega}2\biggr)
\hat{\varphi}\biggl(\frac{\omega}2\biggr)
=
H\biggl(\frac{\omega}2\biggr)
H\biggl(\frac{\omega}4\biggr)
\hat{\varphi}\biggl(\frac{\omega}4\biggr)
=
H\biggl(\frac{\omega}2\biggr)
H\biggl(\frac{\omega}4\biggr)
H\biggl(\frac{\omega}8\biggr)
\hat{\varphi}\biggl(\frac{\omega}8\biggr)
\\
&=
H\biggl(\frac{\omega}2\biggr)
\dots
H\biggl(\frac{\omega}{2^n}\biggr)
\hat{\varphi}\biggl(\frac{\omega}{2^n}\biggr)
=
\hat{\varphi}\biggl(\frac{\omega}{2^n}\biggr)
\prod_{k=1}^n 
H\biggl(\frac{\omega}{2^k}\biggr).
\end{align*}
In vielen Fällen ist $\hat{\varphi}$ eine stetige Funktion, so dass
$\hat{\varphi}(\omega 2^{-n})$ gegen $\hat{\varphi}(0)=1$.
Daher konvergiert das Produkt und es gilt
\[
\hat{\varphi}(\omega) = \prod_{k=1}^\infty H(\omega 2^{-k}).
\]
Insbesondere ist die Funktion $\hat{\varphi}$ eindeutig bestimmt 
durch die Funktion $H(\omega)$, die wiederum eindeutig bestimmt ist
durch die Koeffizienten $h_k$ der Skalierungsrelation.

