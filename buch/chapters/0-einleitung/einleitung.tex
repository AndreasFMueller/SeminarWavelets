%
% einleitung.tex
%
% (c) 2018 Prof Dr Andreas Müller
%
\chapter*{Einleitung\label{chapter:einleitung}}
\lhead{Einleitung}
\addcontentsline{toc}{chapter}{Einleitung}
Zeitabhängige Signale können verstanden werden als reellwertige
Funktionen $t\mapsto f(t)$ der Zeit $t\in\mathbb R$.
Die Analysis lehrt eine Reihe von Methoden, wie man mit solchen
Funktionen arbeiten kann.
Dies ist jedoch eine mathematische Idealisierung.
In der Praxis kennt man nur das Resultat eines Abtastprozesses, wo
der Funktionswert $f(t_k)$ für diskrete Werte $t_i=t_0 + k\Delta t$
ermittelt wird.
Insbesondere gehen alle Informationen über den Verlauf einer Funktion $f(t)$
zwischen den Abtastpunkten $t_k$ verloren.
Trotzdem führt daran nichts vorbei, denn nur auf diese Art ist es möglich,
Signale in einem digitalen System zu verarbeiten.

\index{Abtastung}%
Die Methoden der Analysis sind für Folgen von Abtastwerten $x_k=f(t_k)$
fast nutzlos.
Sogar das Konzept der Frequenz ist fragwürdig, denn ein Sinussignal
$s(t)=\sin(2\pi t/\Delta t)$ nimmt auf allen Abtastpunkten $t_k$
den Wert $s(t_k)=\sin(2\pi t_0/\Delta t+2\pi k)=\sin(2\pi t_0/\Delta t)$,
also den gleichen Wert an.
Allgemeiner: ein Signal mit der gleichen Frequenz wie die Abtastfrequenz
ist nicht von einem konstanten Signal unterscheidbar.
Dieses als Aliasing bekannte Phänomen findet seine allgemeine Erklärung
\index{Aliasing}%
im Nyquist-Shannon-Abtasttheorem, welches besagt, dass ein Signal
\index{Abtasttheorem}%
\index{Shannon, Claude}%
\index{Nyquist, Harry}%
aus den Abtastwerten nur exakt rekonstruiert werden kann, wenn es
keine Frequenzen grösser als die halbe Abtastfrequenz enthält.
Die Signalverarbeitung entwickelt daher einen Satz von Werk\-zeugen, die
einerseits mit dieser Schwierigkeit fertig werden, aber andererseits auch
in der Lage sind gut verstandene technische Prozesse, wie das Ausfiltern
bestimmter Frequenzbereiche, adäquat zu beschreiben.

Das übliche und für den Ingenieur naheliegende Werkzeug ist die 
Fourier-Theorie, die von Joseph Fourier ursprünglich als analytisches
\index{Fourier, Joseph}%
\index{Fourier-Theorie}%
Hilfsmittel zur Lösung von Differentialgleichungen entwickelt wurde.
Sie machte erstmals möglich, ein Signal über die darin vertretenen
Frequenzen zu charakterisieren und zu manipulieren.
Mit der Entwickung der schnellen Fourier-Transformation
(Fast Fourier Transform, FFT) wurde ihr praktischer Einsatz in digitalen
Problemlösungen möglich.
\index{Fourier-Transformation!schnelle}%
\index{Fast Fourier Transform}%
Sie bringt die neue Idee in die Diskussion, ein Signal als Überlagerung
von Beispielfunktionen zu betrachten, die besonderes einfach zu
verstehen sind.
Konkret sind dies die Funktionen $\sin kt$ und $\cos kt$ mit
$k\in\mathbb N$ bei der Analyse periodischer Signale,
oder allgemeiner die Funktionen $t\mapsto e^{i\omega t}$ mit
$\omega\in\mathbb R$ bei beliebigen Signalen.
Die daraus entwickelte Theorie ist zwar sehr reichhaltig und auch erfolgreich,
aber doch in einem Punkt unbefriedigend.
Die Fourier-Transformation liefert zu einem Signal detaillierte Information
über die darin enthaltenen Frequenzen, dafür gehen aber Informationen
darüber verloren, wann interessante Ereignisse eingetreten sind.
Die Information ist nicht ganz verloren, da sie immer noch in den Phasen
der Koeffizienten stecken, aber sie sind für viele praktischen Zwecke 
nicht mehr nutzbar, da nur über eine Rücktransformation wieder erschliessbar.

Es stellt sich daher die Frage, ob es Möglichkeiten der Analyse von Signalen
gibt, die nicht so radikal sind.
Enthält ein Signal hohe Frequenzen, dann ändert sich das Signal rasch.
Ein kurzes Ereignis passt auf diese Beschreibung, aber eine wesentliche
Eigenschaft ist damit nicht beschrieben, nämlich wann es statt gefunden hat.
Kann man Informationen über die in einem Signal vertretenen Frequenzen gewinnen,
ohne auf Information darüber verzichten zu müssen, wann diese Ereignisse
eingetreten sind?

Für eine umfassende Antwort müssen mehrere Teilfragen untersucht werden.
\begin{enumerate}
\item
Welche Arten von Vergleichssignalen soll man verwenden, um solche kurzlebigen
Ereignisse zu beschreiben?
Die trigonometrischen Funktionen sind zweifellos naheliegend und ihre
analytischen Eigenchaften einladend, doch ihre unendliche Ausdehnung entlang
der Zeitachse torpediert von vornherein die Möglichkeit der zeitlichen
Lokalisierung von interessanten Ereignissen.
Gibt es Funktionen, die ähnliche oszillatorische Form haben
aber trotzdem über hilfreiche analytische Eigenschaften haben?
\item
Wie vergleicht man ein gegebenes Signal mit einem Vergleichssignal?
In der Fourier-Theorie lernt man einen ausgedehnten Satz von Integralformeln
zur Berechnung der Fourier-Koeffizien\-ten.
Wie müssen diese verallgemeinert werden?
\item
Wie rekonstruiert man das Signal aus dem Resultat der Analyse?
In der Fourier-Theorie lernt man Summenformeln für die Fourierreihen
und Integralformeln für ausgedehnte Signale, doch wie ist das für 
die neuen Vergleichsfunktionen umzusetzen?
\item
Die Praxis arbeitet mit diskreten Signalen. 
Wie lässt sich die Analyse diskretisieren?
Unter welchen Voraussetzungen ist die Rekonstruktion des Signals
aus der diskreten Analyse möglich?
\item
Das Verfahren ist nur dann praktisch einsetzbar, wenn sowohl die Analyse als
auch die Synthese mit effizienten numerischen Algorithmen möglich sind.
Diese Algorithmen müssen ausserdem stabil sein, d.~h.~kleine Störungen in
den Inputdaten oder Rundungsfehler während der Durchführung des Verfahrens
dürfen sich nicht zu unbrauchbaren Resultaten aufschaukeln.
\item
Welche Vergleichssignale sind für welchen Anwendungszweck geeignet?
\end{enumerate}
Die nachstehenden Kapitel versuchen, Antworten auf diese Fragen zu
geben.

In Kapitel~\ref{chapter:geometrie} wird gezeigt, wie der Vergleich
von Funktionen gleichbedeutend ist mit dem Skalarprodukt von Vektoren.
Daraus entwickelt sich dann die Theorie der Hilberträume, der
\index{Hilbertraum}%
unendlichdimensionalen komplexen Vektorräume mit Skalarprodukt.
% Diese bilden den geometrischen Rahmen, in dem man die ganze 

In der Vektorgeometrie lernt man, Vektoren in Vektorräumen mit Hilfe
orthonormierter Basen zu beschreiben.
Oft erzwingt die Wahl einer orthonormierten Basis die Zerstörung
der Symmetrie der ursprünglichen Problemstellung, was die Lösung
erwschweren kann.
Einen Ausweg aus diesem Dilemma stellen die Frames dar, die in
Kapitel~\ref{chapter:frames} beschrieben werden.
\index{Frame}%
Ein Frame beschreibt Vektoren auf redundante Art, erlaubt aber eventuell
vorhandene Symmetrien zu erhalten.

Die bekannte Fourier-Theorie wird im Kapitel~\ref{chapter:fourier} in
den neuen Rahmen des Hilbertraumes eingeordnet.
Es demonstriert, wie die Hilbertraumtheorie die Vielfalt der 
verschiedenen Varianten der Fouriertheorie vereinheitlicht.

Die Basisfunktionen der Fourier-Theorie sind darauf ausgelegt,
die Translationssymmetrie der Zeitachse auszunützen.
Die Tatsache, dass sich die Zeitachse auch dehnen lässt, ist von
untergeordneter Bedeutung.
Waveletbasen oder Waveletframes stellen sicher, dass die Signalanalyse
sowohl bei gewissen Translationen als auch bei gewissen Dilatationen
erhalten bleibt.
\index{Translation}%
\index{Dilatation}%
Das Haar-Wavelet ist das älteste bekannte Beispiel dieser Art.
\index{Haar-Wavelet}
Es ist einfach und übersichtlich und kann als Muster für die
Phänomene dienen, die bei komplizierteren Wavelets nicht so leicht
zu verstehen sind.
Es wird in Kapitel~\ref{chapter:haar-wavelet} im Detail besprochen.

Die in Kapitel~\ref{chapter:cwt} definierte stetige Wavelettransformation
\index{Wavelettransformation, stetig}%
demonstriert die Fähigkeit, mit Wavelets Ereignisse sowohl in der Zeit
als auch in der Frequenz zu lokalisieren.
Sie analysiert ein Signal jedoch auf höchst redundante Art.
Die Hilbert\-raum\-theorie wird in diesem Kapitel erstmals allgemeiner
angewendet.
Wir lernen die Be\-ding\-ungen kennen, die nötig sind, damit ein Signal
aus seiner Wavelettransformation rekonstruierbar ist.
Wir geben sogar eine Formel für die Rekonstruktion an.

Die diskrete Wavelettransformation überträgt die stetige
Wavelettransformation auf abgetastete Signale.
Der Begriff der Multiskalenanalyse wird in Kapitel~\ref{chapter:msa}
dazu verwendet, die angemessene Diskretisierung im ``Frequenzraum''
zu finden.
Aus der Multiskalenanalyse lassen sich auch die schnellen Algorithmen
\index{Multiskalenanalyse}%
ableiten, welche Wavelets überhaupt erst praktisch nützlich machen.
Diese Algorithmen werden in Kapitel~\ref{chapter:algo} hergeleitet.

Selbst mit einer Multiskalenanalyse brauchen die Algorithmen noch nicht
perfekt zu sein.
Schon bei der Fourier-Transformation war das Problem, dass zur Berechnung
der Fourier-Trans\-for\-ma\-tion alle Inputdaten notwendig sind.
Dies bedeutet, dass das Transformationsresultat erst verfügbar ist,
wenn das ganze Signal abgetastet ist.
Damit sind Verarbeitungen in Echtzeit von vornherein ausgeschlossen.
Ein Ausweg besteht darin, dass die Transformationsoperationen sich auf
einer nur sehr kleinen Anzahl Samples ausführen lassen.
Genau dies wird erreicht, wenn die verwendeten Wavelets einen kompakten Träger
haben.
Solche Wavelets wurden von Ingrid Daubechies in den neun\-zehn\-achtziger
\index{Daubechies, Ingrid}%
\index{Ingrid Daubechies}%
Jahren konstruiert und waren massgebend für den schnellen Siegeszug 
der Wavelettransformation in den Anwendungen.

Natürlich sind die Daubechies-Wavelets nicht die einzigen praktisch
\index{Daubechies-Wavelet}%
nützlichen Wavelets.
Die verbleibenden Kapitel zeigen eine Reihe von alternativen Wavelets,
Kapitel~\ref{chapter:spline} zum Beispiel beschreibt die sogenannten
Spline-Wavelets, die sich durch eine interesssante Faltungskonstruktion
\index{Spline-Wavelet}%
auszeichnen.
Dies gibt uns Gelegenheit, die Faltung etwas besser kennen zu lernen.
Das Haar-Wavelet ist sowohl das einfachste Daubechies-Wavelet als auch
das primäre Spline-Wavelet.









