%
% ordnung.tex
%
% (c) 2019 prof Dr Andreas Müller, Hochschule Rapperswil
%
\section{Ordnung\label{section:ordnung}}
\rhead{Ordnung}
Bis jetzt haben wir nur die Bedingungen \eqref{buch:kompakt:hsumme}
und \eqref{buch:kompakt:orthorel} an die Koeffizienten 
$h_k$ der Skalierungsrelation.
Wir haben ebenfalls bereits gesehen, dass für grösseren Träger die
Bedinungen \eqref{buch:kompakt:orthorel} nicht ausreichen, die
Koeffizienten festzulegen.
Es werden daher zusätzliche Bedingungen nötig sein, um die
Multiskalenanalyse eindeutig zu bestimmen.

\begin{definition}
Das Wavelet $\psi(t)$ heisst von Ordnung $N$ wenn
$t^N\psi(t)\in L^1(\mathbb R)$ und
\[
\int_{\mathbb R} t^k\psi(t)\,dt=0\quad \text{für $k<N$.}
\]
\end{definition}

Für ein Wavelet der Ordnung $N$ folgt mit Hilfe der Ableitung unter
dem Integralzeichen
\begin{align*}
\frac{d}{d\omega}
\hat{\psi}(\omega)
&=
\frac{d}{d\omega}
\frac{1}{\sqrt{2\pi}}
\int_{\mathbb R}
\psi(t) e^{-i\omega t}\,dt
=
\frac{1}{\sqrt{2\pi}}
\int_{\mathbb R}
(-it)
\psi(t) e^{-i\omega t}\,dt
\\
\frac{d^k}{d\omega^k}
\hat{\psi}(\omega)
&=
\frac{1}{\sqrt{2\pi}}
\int_{\mathbb R}
(-it)^k
\psi(t) e^{-i\omega t}\,dt.
\end{align*}
Für ein Wavelet der Ordnung $N$ sind die Ableitungen von $\hat{\psi}(\omega)$
im Nullpunkt
\[
\frac{d^k}{d\omega^k}
\hat{\psi}(0)
=
\frac{(-i)^k}{\sqrt{2\pi}}
\int_{\mathbb R} t^k \psi(t) \,dt = 0
\]

für $k<N$. 
Die Taylorreihe von $\hat{\psi}$ hat daher die Form
\[
\hat{\psi}(\omega)
=
\frac{\hat{\psi}^{(N)}(0)}{N!} \omega^N + \text{Terme höherer Ordnung}
\]
Insbesondere hat $\hat{\psi}$ eine Nullstelle $N$-ter Ordnung bei $\omega=0$.

Aus dem Zusammenhang zwischen Vater- und Mutter-Wavelet im Frequenzbereich
\[
\hat{\psi}(\omega)
=
e^{i\omega/2}
\overline{H\biggl(\frac{\omega}+\pi\biggr)}
\hat{\varphi}\biggl(\frac{\omega}2\biggr)
\]
folgt daher, dass auch die Funktion $H$ eine Nullstelle $N$-ter
Ordnung haben muss.
Da $\hat{\varphi}(0)\ne 0$ und $e^{i\omega}\ne 0$ ist, muss daher
$H$ eine Nullstelle $N$-ter Ordnung bei $\pi$ haben.

Das trigonometrische Polynom
\[
\biggl(
\frac{1+e^{-i\omega}}{2}
\biggr)^N
\]
hat eine $N$-fache Nullstelle bei $\pi$, also muss es möglich sein,
\begin{equation}
H(\omega)
= 
\biggl(
\frac{1+e^{-i\omega}}{2}
\biggr)^N B(\omega)
\label{buch:kompakt:HB}
\end{equation}
zu schreiben, wobei $B(\omega)$ ein trigonometrisches Polynom ist,
welches keine Nullstelle bei $\omega=\pi$ hat.

\begin{beispiel}
Für das Haarwavelet ist
\[
H(\omega)
=
\frac{ 1+e^{-i\omega}}2
\]
und damit folgt $B(\omega)=1$.
Wie früher gezeigt wurde, bestimmen die Relationen 
\eqref{buch:kompakt:hsumme} und \eqref{buch:kompakt:orthorel}
das Haar-Wavelet bereits eindeutig, so dass die Vorgabe der Ordnung
nicht nötig ist.
\end{beispiel}

Das trigonometrische Polynom $B(\omega)$ soll in den nachfolgenden
Abschnitten bestimmt werden.


