%
% ordnung.tex
%
% (c) 2019 prof Dr Andreas Müller, Hochschule Rapperswil
%
\section{Ordnung\label{section:ordnung}}
\rhead{Ordnung}
Bis jetzt haben wir nur die Bedingungen \eqref{buch:kompakt:hsumme}
und \eqref{buch:kompakt:orthorel} an die Koeffizienten 
$h_k$ der Skalierungsrelation gefunden.
Wir haben ebenfalls bereits gesehen, dass für grösseren Träger die
Bedinungen \eqref{buch:kompakt:orthorel} nicht ausreichen, die
Koeffizienten festzulegen.
Es werden daher zusätzliche Bedingungen nötig sein, um die
Multiskalenanalyse eindeutig zu bestimmen.

Die Art dieser zusätzlichen Bedingungen ist mehr oder weniger willkürlich.
Sie kann durch beabsichtigte Anwendungen motiviert sein oder durch
interessante mathematische Eigenschaften der resultierenden Funktionen.
Im vorliegenden Fall ist es die {\em Ordnung}, die wie folgt definiert ist.

\begin{definition}
\label{kompakt:ordnung:definition}
\index{Ordnung}%
Das Wavelet $\psi(t)$ heisst von {\em Ordnung} $N$ wenn
$t^N\psi(t)\in L^1(\mathbb R)$ und
\begin{equation}
\int_{\mathbb R} t^k\psi(t)\,dt=0\quad \text{für $k<N$.}
\label{kompakt:ordnung:formel}
\end{equation}
\label{definition:ordnung}
\end{definition}

Für ein Wavelet der Ordnung $N$ folgt mit Hilfe der Ableitung unter
dem Integralzeichen oder mit Hilfe von Satz~\ref{four-int:trans-dial}
\begin{align*}
\frac{d}{d\omega}
\hat{\psi}(\omega)
&=
\frac{d}{d\omega}
\frac{1}{\sqrt{2\pi}}
\int_{\mathbb R}
\psi(t) e^{-i\omega t}\,dt
=
\frac{1}{\sqrt{2\pi}}
\int_{\mathbb R}
(-it)
\psi(t) e^{-i\omega t}\,dt
\\
\frac{d^k}{d\omega^k}
\hat{\psi}(\omega)
&=
\frac{1}{\sqrt{2\pi}}
\int_{\mathbb R}
(-it)^k
\psi(t) e^{-i\omega t}\,dt.
\end{align*}
Für ein Wavelet der Ordnung $N$ sind die Ableitungen von $\hat{\psi}(\omega)$
im Nullpunkt gegeben durch
\[
\frac{d^k}{d\omega^k}
\hat{\psi}(0)
=
\frac{(-i)^k}{\sqrt{2\pi}}
\int_{\mathbb R} t^k \psi(t) \,dt = 0
\]
für $k<N$. 
Die Taylorreihe von $\hat{\psi}$ hat daher die Form
\[
\hat{\psi}(\omega)
=
\frac{\hat{\psi}^{(N)}(0)}{N!} \omega^N + \text{Terme höherer Ordnung}
\]
Insbesondere hat $\hat{\psi}$ eine Nullstelle $N$-ter Ordnung bei $\omega=0$.
Die Ordnung sagt also etwas über die Regularität von $\hat{\psi}$ aus.

Die Eigenschaft, Ordnung $N$ zu haben, verträgt sich mit den Operatoren
$D_a$ und $T_b$.
Man kann formal die Bedingung \eqref{kompakt:ordnung:formel} aus
Defintion~\ref{kompakt:ordnung:definition} als Skalarprodukt mit der
Funktion $t^k$ betrachten:
\[
\langle t^k,\psi\rangle = \int_{\mathbb{R}} t^k\psi(t)\,dt.
\]
Für eine skalierte Funktion folgt dann
\[
\langle t^k,D_a\psi\rangle 
=
\langle D_{1/a}t^k,\psi\rangle
=
\langle \sqrt{|a|}(at)^k,\psi\rangle
=
a^k|a|^{\frac12}\langle t^k,\psi\rangle = 0
\qquad
\forall k<N.
\]
Für den Operator $T_b$ ist zu berücksichtigen, dass $T_bt^k$ das
Polynom
\[
T_bt^k
=
(t-b)^k
=
\sum_{i=0}^k (-1)^i \binom{k}{i}t^ib^{k-i}
\]
ist.
Damit kann auch die Ordnung von $T_b\psi$ berechnet werden.
Es gilt
\[
\langle t^k,T_b\psi\rangle
=
\langle T_{-b}t^k,\psi\rangle
=
\biggl\langle \sum_{i=0}^k \binom{k}{i}t^ib^{k-i},\psi\biggr\rangle
=
\sum_{i=0}^k b^{k-i} \underbrace{\langle t^i,\psi\rangle}_{\displaystyle=0}
=
0.
\]
Mit $\psi$ hat daher auch $D_a\psi$ und $T_b\psi$ die Ordnung $n$.
Die Forderung nach einer bestimmten Ordnung ist damit eine ``natürliche''
Forderung für ein Wavelet.

\begin{forderung}
Das gesuchte Wavelet hat Ordnung $N$.
\end{forderung}

Das Integral der Skalierungsfunktion $\varphi$ darf nicht verschwinden,
es ist also $\hat{\varphi}(0)\ne 0$.
Für ein Wavelet der Ordnung $N$ hat $\hat{\psi}$ bei $0$ eine Nullstelle
$N$-ter Ordnung.
Aus dem Zusammenhang zwischen Vater- und Mutterwavelet im Frequenzbereich
\[
\hat{\psi}(\omega)
=
e^{i\omega/2}
\overline{H\biggl(\frac{\omega}2+\pi\biggr)}
\hat{\varphi}\biggl(\frac{\omega}2\biggr)
\]
folgt, dass die rechte Seite eine Nullstelle $N$-ter Ordnung hat.
Da $e^{i\omega/2}$ nicht verschwinden kann und $\hat{\varphi}(0)\ne 0$ folgt,
dass die Funktion $H$ bei $\pi$ eine Nullstelle $N$-ter
Ordnung haben muss.

\begin{konsequenz}
Die erzeugende Funktion $H(\omega)$ einer Multiskalenanalyse mit einem
Wavelet der Ordnung $N$ ist ein trigonometrisches Polynom mit einer
Nullstelle $N$-ter Ordnung bei $\omega=\pi$.
\end{konsequenz}

Da die erzeugende Funktion ein Polynom ist, können wir für die
bekannten Nullstellen als Linearfaktoren abspalten.
Das trigonometrische Polynom
\[
\biggl(
\frac{1+e^{-i\omega}}{2}
\biggr)^N
\]
hat eine $N$-fache Nullstelle bei $\pi$, also muss es möglich sein,
\begin{equation}
H(\omega)
= 
\biggl(
\frac{1+e^{-i\omega}}{2}
\biggr)^N B(\omega)
\label{buch:kompakt:HB}
\end{equation}
zu schreiben, wobei $B(\omega)$ ein trigonometrisches Polynom ist,
welches keine Nullstelle bei $\omega=\pi$ hat.

\begin{beispiel}
Für das Haar-Wavelet ist
\[
H(\omega)
=
\frac{ 1+e^{-i\omega}}2
\]
und damit folgt $B(\omega)=1$.
Wie früher gezeigt wurde, bestimmen die Relationen 
\eqref{buch:kompakt:hsumme} und \eqref{buch:kompakt:orthorel}
das Haar-Wavelet bereits eindeutig, so dass die Vorgabe der Ordnung
nicht nötig ist.
\end{beispiel}

Das trigonometrische Polynom $B(\omega)$ soll in den nachfolgenden
Abschnitten bestimmt werden.


