%
% chapter.tex -- Kapitel über Wavelets mit kompaktem Träger
%
% (c) 2019 Prof Dr Andreas Müller, Hochschule Rapperswil
%
\chapter{Wavelets mit kompaktem Träger
\label{chapter:kompakt}}
\lhead{Wavelets mit kompaktem Träger}
\rhead{}
Für die Praxis sind nur Wavelets interessant, für die die
Wavelet-Transformation und die Rücktransformation mit beschränktem Aufwand,
effizient und stabil berechnet werden können.
Die Untersuchungen in Kapitel~\ref{chapter:algo} haben gezeigt, dass 
die diskrete Wavelet-Transformation mit einem einfachen kaskadierten
Faltungsfilter berechnet werden kann.
Ein solcher Filter kann aber nur dann effizient berechnet werden,
wenn nur endlich viele der Filterkoeffizienten von 0 verschieden sind.

Das Haar-Wavelet hat diese Eigenschaft, genau zwei der Filterkoeffizienten
sind von 0 verschieden.
Lange Zeit war nicht klar, aber es überhaupt andere Wavelets mit
nur endlich vielen nicht verschwindenden Filterkoeffizienten gibt.
Yves Meyer versuchte sogar zu zeigen, dass es gar keine solche Funktionen
$\varphi$ und $\psi$ gibt, wie sie in der Multiskalen-Analyse verlangt
werden, also erst recht keine mit endlich vielen nicht verschwindenden
Koeffizienten.
Als es ihm dann 1985 gelang, solche Wavelets zu finden, stellte sich
die Frage nach Wavelets mit kompaktem Träger erneut.
Schliesslich war Ingrid Daubechies in der Konstruktion solcher Wavelets
erfolgreich.

%
% bedingungen.tex
%
% (c) 2019 Prof Dr Andreas Müller, Hochschule Rapperswil
%
\section{Bedingungen für die Koeffizienten\label{section:bedingungen}}
\rhead{Bedingungen für die Koeffizienten}
Bis jetzt ist ausser dem Haar-Wavelet kein Beispiel einer
\index{Haar-Wavelet}
Multiskalenanalyse konstruiert worden, in dem die Funktionen 
$\varphi$ und $\psi$ kompakten Träger haben.
Wir untersuchen in diesem Abschnitt, welche Eigenschaften an die 
Koeffizienten sich bereits aus den Axiomen einer Multiskalenanalyse
ergeben.
Wir fordern daher:

\begin{forderung}
\label{forderung:msa}
Das Wavelet $\psi$ und die Skalierungsfunktion $\varphi$ gehören zu
einer Multiskalenanalyse.
\end{forderung}

\subsection{Kompakter Träger}
\index{Träger}%
\index{kompakter Träger}%
Wie in der Einleitung motiviert, wollen wir aus Gründen der numerischen
Berechenbarkeit nur Wavelets betrachten, die eine ``Endlichkeitsbedingung''
erfüllen.
Für die Synthese eines Signals in $V_j$ in einem Punkt sollen nur endlich
viele Summanden nötig sein.
Dies führt auf die folgende Forderung:

\begin{forderung}
\label{forderung:kompakt}
Die Skalierungsfunktion $\varphi$ hat kompakten Träger.
\end{forderung}

Wir gehen also davon aus, dass der Träger der Funktion $\varphi$ im Intervall
$[a,b]$ enthalten ist.
Dann ist der Träger des Translates $T_k\varphi$ in $[a+k,b+k]$ enthalten
und der Träger der skalierten Funktion $D_{\frac12}T_k\varphi$ ist
enthalten in $[\frac12(a+k),\frac12(b+k)]$.

Die Koeffizienten der Skalierungsrelation können mit Hilfe des
Skalarproduktes erhalten werden, denn es gilt:
\begin{align*}
\langle D_{\frac12}T_r\varphi, \varphi\rangle
&=
\biggl\langle
D_{\frac12}T_r\varphi, \sum_{k\in\mathbb Z} D_{\frac12}T_k\varphi
\biggr\rangle
=
\sum_{k\in\mathbb Z}
\langle
D_{\frac12}T_r\varphi,
h_kD_{\frac12}T_k\varphi
\rangle
\\
&=
\sum_{k\in\mathbb Z}
\bar{h}_k
\langle
D_{\frac12}T_r\varphi,
D_{\frac12}T_k\varphi
\rangle
=
\sum_{k\in\mathbb Z}
\bar{h}_k
\langle
T_r\varphi,
T_k\varphi
\rangle
=
\sum_{k\in\mathbb Z}
\bar{h}_k
\delta_{kr}
=
\bar{h}_r,
\end{align*}
wobei wir verwendet haben, dass $T_k\varphi$ und $T_r\varphi$ orthogonal
sind für $k\ne r$.

Das Skalarprodukt
$\langle D_{\frac12}T_k\varphi, \varphi \rangle$
ist ein Integral über das Produkt der Funktion $\varphi$ mit Träger im Intervall
$[a,b]$ und der Funktion $D_{\frac12}T_r\varphi$  mit Träger in
$[\frac12(a+r),\frac12(b+r)]$.
Dieses Integral verschwindet, wenn die Intervalle nicht überlappen.
Dieser Fall tritt ein, wenn
\[
\begin{aligned}
b&<\frac12(a+r) &&\text{oder}& \frac12(b+r) &< a,
\\
2b-a&<r         &&           &            r &< 2a-b.
\end{aligned}
\]
Die einzigen Koeffizienten $h_r$, die nicht verschwinden können, erfüllen
\[
2b-a < r < 2a-b,
\]
es können also nur endlich viele Koeffizienten $h_r$ von $0$ verschieden
sein.

\begin{konsequenz}
Die erzeugende Funktion $H(\omega)$ einer Multiskalenanalyse, in der
$\varphi$ kompakten Träger hat, ist ein Polynom in $e^{i\omega}$ oder auch
trigonometrisches Polynom.
\end{konsequenz}

\subsection{Normierung}
\index{Normierungskonvention für Wavelet-Koeffizienten}%
Wir gehen wieder von einer Multiskalenanalyse mit einem Vaterwavelet
$\varphi$ mit kompaktem Träger aus.
Die Analyse des konstanten Signals $f(t)=1$ erfolgt mit Hilfe des
Skalarprodukts $\langle f,\varphi\rangle$.
Anwendung der Skalierungsrelation auf $\varphi$ liefert
\begin{align*}
\langle 1,\varphi\rangle
&=
\biggl\langle
1,\sum_{k\in\mathbb Z} h_kD_{\frac12}T_k\varphi
\biggr\rangle
=
\sum_{k\in\mathbb Z}
\bar{h}_k
\langle
1,
D_{\frac12}T_k\varphi
\rangle
=
\sum_{k\in\mathbb Z}
\bar{h}_k
\langle
D_21,
T_k\varphi
\rangle
=
\sum_{k\in\mathbb Z}
\bar{h}_k
\langle
D_21,
T_k\varphi
\rangle
\\
&=
\sum_{k\in\mathbb Z}
\bar{h}_k
\frac{1}{\sqrt{2}}
\langle 1,\varphi\rangle.
\end{align*}
Da $\langle 1,\varphi\rangle\ne 0$ ist, können wir durch
$\langle 1,\varphi\rangle$ dividieren und es folgt
\begin{equation}
\sum_{k\in\mathbb Z} \bar{h}_k = \sqrt{2}
\quad\Rightarrow\quad
\sum_{k\in\mathbb Z} h_k = \sqrt{2}.
\label{buch:kompakt:hsumme}
\end{equation}
Diese Identität kann nützlich sein, wenn man Koeffizienten aus einer
nicht genauer bekannten Quelle benützen will und nicht sicher ist,
ob sie die gleiche Normierungskonvention für die Skalierungsrelation
verwendet.

\begin{beispiel}
Für das Haar-Wavelet sind nur zwei Koeffizienten $h_k$ von $0$ verschieden
und beide sind gleich gross.
Aus der Relation \eqref{buch:kompakt:hsumme} folgt
\begin{equation}
\sqrt{2}
=
h_0+h_1
=2h_0
\qquad\Rightarrow\qquad
h_0 = h_1 = \frac{\sqrt{2}}{2}=\frac{1}{\sqrt{2}}.
\label{kompakt:haar:normierung}
\end{equation}
\qedhere
\end{beispiel}

\subsection{Orthogonalität}
Aus der Orthonormalität der Funktionen $T_k\varphi$ einer Multiskalenanalyse
folgen die Relationen
\begin{align}
\delta_{0k}
=
\langle \varphi,T_k \varphi\rangle
&=
\biggl\langle
\sum_{l\in\mathbb Z} h_lT_l\varphi,
\sum_{r\in\mathbb Z} h_rT_{r+2k}\varphi
\biggr\rangle
\notag
\\
&=
\sum_{l,r\in\mathbb Z} h_l\bar{h}_r \langle T_l\varphi, T_{r+2k}\varphi\rangle
\\
&=
\sum_{l,r\in\mathbb Z} h_l\bar{h}_r \delta_{l,r+2k}
=
\sum_{r\in\mathbb Z} h_{r+2k}\bar{h}_r.
\label{buch:kompakt:orthorel}
\end{align}
Nehmen wir an, dass genau die Koeffizienten $h_k$ mit $k$ zwischen
$0$ und $n$ von $0$ verschieden sind,
dann wird die Summe~\eqref{buch:kompakt:orthorel} trivialerweise
verschwinden, wenn $2k>n$.
Dies liefert maximal $n/2$ Bedinungen für die $n+1$ Koeffizienten
$h_k$, sie werden daher im Allgemeinen noch nicht bestimmt sein.
Weitere Bedingungen müssen formuliert werden, um die Koeffizienten
festzulegen.

\begin{beispiel}
Die Skalierungsrelation des Haar-Wavelets enthält nur zwei Koeffizienten
$h_0$ und $h_1$, also $n=1$.
Sie wurden im Beispiel auf Seite~\pageref{kompakt:haar:normierung}
in \eqref{kompakt:haar:normierung} bereits bestimmt.
Von den Relationen \eqref{buch:kompakt:orthorel} ist nur jene mit $k=0$
nichttrivial, es gilt
\[
1
=
\delta_{0k}
=
h_0\bar{h}_0 + h_1\bar{h}_1
=
|h_0|^2+|h_1|^2
=
\frac12+\frac12
=1.
\qedhere
\]
\end{beispiel}

Die bisher gesammelten Forderungen reichen bereits, die Multiskalenanalyse
vollständig festzulegen, wenn wir nicht zu viele Koeffizienten zu
bestimmen haben.
Das folgende Beispiel zeigt, dass im Falle $n=1$, in dem genau zwei 
Koeffizienten bestimmt werden müssen, alles bereits festgelegt ist
und auf das Haar-Wavelet führt.

\begin{beispiel}
Gesucht ist die Skalierungsrelation für eine Multiskalenanalyse derart,
dass $\varphi$ Träger im Intervall $[0,1]$ hat.
Daraus kann man zunächst schliessen, dass die Skalierungsrelation
genau zwei von $0$ verschiedene Koeffizienten $h_0$ und $h_1$ hat.
Aus~\eqref{buch:kompakt:hsumme} und \eqref{buch:kompakt:orthorel}
folgen die Gleichungen
\begin{align*}
h_0+h_1&=\sqrt{2},
\\
|h_0|^2+|h_1|^2&=1.
\end{align*}
Wir suchen eine Lösung in der Form $h_0=a+bi$ und $h_1=\sqrt{2}-a-bi$.
Die Orthonormalisierungsrelation liefert dann
\begin{align}
1
&=
a^2 + b^2
+
(\sqrt{2}-a)^2 + b^2
\notag
\\
&=
2a^2 + 2b^2 - 2\sqrt{2}a + 2
\notag
\\
0
&=
a^2 - \sqrt{2}a + \frac12 + b^2.
\label{buch:kompakt:quadr}
\end{align}
Dies ist eine quadratische Gleichung für $a$.
Für $b=0$ wird sie zu
\[
0
=
a^2 - \sqrt{2}a + \frac12
=
\biggl(a-\frac1{\sqrt{2}}\biggr)^2,
\]
also ist in diesem Fall $a=\frac1{\sqrt{2}}$ und damit
\[
h_0=h_1=\frac{1}{\sqrt{2}},
\]
dies sind die Koeffizienten der Skalierungsrelation des Haar-Wavelets.

Gibt es weitere Lösungen?
Die quadratische Gleichung \eqref{buch:kompakt:quadr} hat nur dann
relle Lösungen für $a$, wenn die Diskriminante
$D=\sqrt{2}^2 - 4\cdot (b^2+\frac12) \ge 0$ ist:
\[
D=2-2-4b^2 =-4b^2\le 0.
\]
Es folgt, dass $b=0$ die einzige Lösung ist.
\end{beispiel}

Es gibt also nur eine einzige Multiskalenanalyse mit einem Wavelet,
dessen Träger im Intervall $[0,1]$ enthalten ist, nämlich das Haar-Wavelet.
Da $n=1$ in diesem Fall sehr klein ist, ist das nicht überraschen, wir
haben wenige Koeffizienten und genügend einschränkende Bedingungen.
Für grösseres $n$ bleibt das Problem offen.


%
% ordnung.tex
%
% (c) 2019 prof Dr Andreas Müller, Hochschule Rapperswil
%
\section{Ordnung\label{section:ordnung}}
\rhead{Ordnung}
Bis jetzt haben wir nur die Bedingungen \eqref{buch:kompakt:hsumme}
und \eqref{buch:kompakt:orthorel} an die Koeffizienten 
$h_k$ der Skalierungsrelation gefunden.
Wir haben ebenfalls bereits gesehen, dass für grösseren Träger die
Bedinungen \eqref{buch:kompakt:orthorel} nicht ausreichen, die
Koeffizienten festzulegen.
Es werden daher zusätzliche Bedingungen nötig sein, um die
Multiskalenanalyse eindeutig zu bestimmen.

Die Art dieser zusätzlichen Bedingungen ist mehr oder weniger willkürlich.
Sie kann durch beabsichtigte Anwendungen motiviert sein oder durch
interessante mathematische Eigenschaften der resultierenden Funktionen.
Im vorliegenden Fall ist es die {\em Ordnung}, die wie folgt definiert ist.

\begin{definition}
\label{kompakt:ordnung:definition}
\index{Ordnung}%
Das Wavelet $\psi(t)$ heisst von {\em Ordnung} $N$ wenn
$t^N\psi(t)\in L^1(\mathbb R)$ und
\begin{equation}
\int_{\mathbb R} t^k\psi(t)\,dt=0\quad \text{für $k<N$.}
\label{kompakt:ordnung:formel}
\end{equation}
\label{definition:ordnung}
\end{definition}

Für ein Wavelet der Ordnung $N$ folgt mit Hilfe der Ableitung unter
dem Integralzeichen oder mit Hilfe von Satz~\ref{four-int:trans-dial}
\begin{align*}
\frac{d}{d\omega}
\hat{\psi}(\omega)
&=
\frac{d}{d\omega}
\frac{1}{\sqrt{2\pi}}
\int_{\mathbb R}
\psi(t) e^{-i\omega t}\,dt
=
\frac{1}{\sqrt{2\pi}}
\int_{\mathbb R}
(-it)
\psi(t) e^{-i\omega t}\,dt
\\
\frac{d^k}{d\omega^k}
\hat{\psi}(\omega)
&=
\frac{1}{\sqrt{2\pi}}
\int_{\mathbb R}
(-it)^k
\psi(t) e^{-i\omega t}\,dt.
\end{align*}
Für ein Wavelet der Ordnung $N$ sind die Ableitungen von $\hat{\psi}(\omega)$
im Nullpunkt gegeben durch
\[
\frac{d^k}{d\omega^k}
\hat{\psi}(0)
=
\frac{(-i)^k}{\sqrt{2\pi}}
\int_{\mathbb R} t^k \psi(t) \,dt = 0
\]
für $k<N$. 
Die Taylorreihe von $\hat{\psi}$ hat daher die Form
\[
\hat{\psi}(\omega)
=
\frac{\hat{\psi}^{(N)}(0)}{N!} \omega^N + \text{Terme höherer Ordnung}
\]
Insbesondere hat $\hat{\psi}$ eine Nullstelle $N$-ter Ordnung bei $\omega=0$.
Die Ordnung sagt also etwas über die Regularität von $\hat{\psi}$ aus.

Die Eigenschaft, Ordnung $N$ zu haben, verträgt sich mit den Operatoren
$D_a$ und $T_b$.
Man kann formal die Bedingung \eqref{kompakt:ordnung:formel} aus
Defintion~\ref{kompakt:ordnung:definition} als Skalarprodukt mit der
Funktion $t^k$ betrachten:
\[
\langle t^k,\psi\rangle = \int_{\mathbb{R}} t^k\psi(t)\,dt.
\]
Für eine skalierte Funktion folgt dann
\[
\langle t^k,D_a\psi\rangle 
=
\langle D_{1/a}t^k,\psi\rangle
=
\langle \sqrt{a}(at)^k,\psi\rangle
=
a^{k+\frac12}\langle t^k,\psi\rangle = 0
\qquad
\forall k<N.
\]
Für den Operator $T_b$ ist zu berücksichtigen, dass $T_bt^k$ das
Polynom
\[
T_bt^k
=
(t-b)^k
=
\sum_{i=0}^k (-1)^i \binom{k}{i}t^ib^{k-i}
\]
ist.
Damit kann auch die Ordnung von $T_b\psi$ berechnet werden.
Es gilt
\[
\langle t^k,T_b\psi\rangle
=
\langle T_{-b}t^k,\psi\rangle
=
\biggl\langle \sum_{i=0}^k \binom{k}{i}t^ib^{k-i},\psi\biggr\rangle
=
\sum_{i=0}^k b^{k-i} \underbrace{\langle t^i,\psi\rangle}_{\displaystyle=0}
=
0.
\]
Mit $\psi$ hat daher auch $D_a\psi$ und $T_b\psi$ die Ordnung $n$.
Die Forderung nach einer bestimmten Ordnung ist damit eine ``natürliche''
Forderung für ein Wavelet.

\begin{forderung}
Das gesuchte Wavelet hat Ordnung $N$.
\end{forderung}

Das Integral der Skalierungsfunktion $\varphi$ darf nicht verschwinden,
es ist also $\hat{\varphi}(0)\ne 0$.
Für ein Wavelet der Ordnung $N$ hat $\hat{\psi}$ bei $0$ eine Nullstelle
$N$-ter Ordnung.
Aus dem Zusammenhang zwischen Vater- und Mutterwavelet im Frequenzbereich
\[
\hat{\psi}(\omega)
=
e^{i\omega/2}
\overline{H\biggl(\frac{\omega}2+\pi\biggr)}
\hat{\varphi}\biggl(\frac{\omega}2\biggr)
\]
folgt, dass die rechte Seite eine Nullstelle $N$-ter Ordnung hat.
Da $e^{i\omega/2}$ nicht verschwinden kann und $\hat{\varphi}(0)\ne 0$ folgt,
dass die Funktion $H$ bei $\pi$ eine Nullstelle $N$-ter
Ordnung haben muss.

\begin{konsequenz}
Die erzeugende Funktion $H(\omega)$ einer Multiskalenanalyse mit einem
Wavelet der Ordnung $N$ ist ein trigonometrisches Polynom mit einer
Nullstelle $N$-ter Ordnung bei $\omega=\pi$.
\end{konsequenz}

Da die erzeugende Funktion ein Polynom ist, können wir für die
bekannten Nullstellen als Linearfaktoren abspalten.
Das trigonometrische Polynom
\[
\biggl(
\frac{1+e^{-i\omega}}{2}
\biggr)^N
\]
hat eine $N$-fache Nullstelle bei $\pi$, also muss es möglich sein,
\begin{equation}
H(\omega)
= 
\biggl(
\frac{1+e^{-i\omega}}{2}
\biggr)^N B(\omega)
\label{buch:kompakt:HB}
\end{equation}
zu schreiben, wobei $B(\omega)$ ein trigonometrisches Polynom ist,
welches keine Nullstelle bei $\omega=\pi$ hat.

\begin{beispiel}
Für das Haar-Wavelet ist
\[
H(\omega)
=
\frac{ 1+e^{-i\omega}}2
\]
und damit folgt $B(\omega)=1$.
Wie früher gezeigt wurde, bestimmen die Relationen 
\eqref{buch:kompakt:hsumme} und \eqref{buch:kompakt:orthorel}
das Haar-Wavelet bereits eindeutig, so dass die Vorgabe der Ordnung
nicht nötig ist.
\end{beispiel}

Das trigonometrische Polynom $B(\omega)$ soll in den nachfolgenden
Abschnitten bestimmt werden.



%
% betrag.tex
%
% (c) 2019 prof Dr Andreas Müller, Hochschule Rapperswil
%
\section{Betrag der erzeugenden Funktion\label{section:betrag}}
\rhead{Betrag der erzeugenden Funktion}
Die Koeffizienten der Skalierungsrelation sollen reell sein, damit
der Transformationsalgorithmus vollständig in $\mathbb R$ gerechnet
werden kann.

\begin{forderung}
\label{forderung:reell}
Die Koeffizienten $h_k$ der Skalierungsrelation sind reell.
\end{forderung}

Um die Funktion $H(\omega)$ zu bestimmen untersuchen wir erst den
Betrag $|H(\omega)|^2$.
Die Forderung~\ref{forderung:reell} hat zur Folge, dass
\[
\overline{H(\omega})
=
\overline{
\frac1{\sqrt{2}}
\sum_{k\in\mathbb Z} h_ke^{-ik\omega}
}
=
\frac1{\sqrt{2}}
\sum_{k\in\mathbb Z} \bar{h}_ke^{ik\omega}
=
\frac1{\sqrt{2}}
\sum_{k\in\mathbb Z} h_ke^{-ik(-\omega)}
=
H(-\omega).
\]
Damit ist
\[
M(\omega)
=
|H(\omega)|^2
=
H(\omega)
\overline{H(\omega)}
=
H(\omega)H(-\omega).
\]
Dieser Ausdruck ändert sich nicht, wenn man $\omega$ durch $-\omega$
ersetzt, die Funktion $M(\omega)$ ist daher eine gerade Funktion.
Da sie ausserdem ein trigonometrisches Polynom ist, muss sie
als ein Polynom in $\cos\omega$ geschrieben werden können.

\begin{konsequenz}
Die Funktion $M(\omega)=|H(\omega)|^2$ ist eine gerade Funktion
und kann also Polynom in $\cos\omega$ geschrieben werden.
\end{konsequenz}

Die Faktorisierung \eqref{buch:kompakt:HB} von $H(\omega)$
zeigt, dass $M(\omega)$ den Faktor
\[
\biggl(\frac{1+e^{-i\omega}}2\biggr)^N
\biggl(\frac{1+e^{i\omega}}2\biggr)^N
=
\biggl( \frac{1+\cos\omega}2\biggr)^N
=
\biggl(
\cos^2\frac{\omega}2
\biggr)^N
\]
enthält.
Oder für $M(\omega)$
\begin{equation}
M(\omega)
= 
\biggl(
\cos^2\frac{\omega}2
\biggr)^N
B(\omega)B(-\omega)
\label{buch:kompakt:MB}
\end{equation}
Um $B(\omega)$ zu finden ist daher zunächst
$A(\omega)=B(\omega)B(-\omega)$
zu bestimmen, in einem zweiten Schritt kann man dann eine
Faktorisierung von $A(\omega)$ mit Hilfe von $B(\omega)$ suchen.

Der Faktor $A(\omega)$ ist wieder in trigonometrisches Polynom,
welches als Funktion von $\omega$ gerade ist.
Es muss also wieder als Polynom in $\cos\omega$
in der Form $A(\omega)=\tilde{P}(\cos\omega)$
geschrieben werden können.
Aus der Halbwinkelformel
\[
\cos^2\frac{\omega}2
=
\frac{1+\cos\omega}2,
\qquad
\text{für den Kosinus folgt}
\qquad
\cos\omega
=
2\cos^2\frac{\omega}2 - 1,
\]
so dass das Polynom $A(\omega)$ auch durch $\cos^2\frac{\omega}2$
ausgedrückt werden kann.

\begin{konsequenz}
Die Funktion $A(\omega)$ ist ein Polynom in $\cos^2\frac{\omega}2$.
\end{konsequenz}

Wir verwenden im Folgenden die Variable 
\[
y=\sin^2\frac{\omega}2
\qquad\Rightarrow\qquad
1-y=\cos^2\frac{\omega}2.
\]
Damit wird auch 
\[
\cos\omega
=
2\cos^2\frac{\omega}2
-1
=
2(1-y)-1
=
1-2y.
\]
Die Funktion $A(\omega)$ als Polynom in $\cos\omega$ wird
dann zu $\tilde{P}(1-2y) = P(y)$.

Um $A(\omega)$ zu bestimmen ist also ein Polynom $P(y)$ zu finden.
Für die Funktion $M(\omega)$ folgt jetzt
\[
M(\omega)
=
\biggl(\cos\frac{\omega}2\biggr)^N
A(\omega)
=
(1-y)^N P(y).
\]

\begin{konsequenz}
Die Funktion $M(\omega)$ ist eine Polynom der Form
$(1-y)^NP(y)$ mit $y=\sin^2\omega/2$.
\end{konsequenz}

Die Funktion $M(\omega)$ genügt auch der Relation
\begin{equation}
1
=
|H(\omega)|^2 + |H(\omega+\pi)|^2
=
M(\omega) + M(\omega+\pi),
\label{kompakt:orthogonalitaetsbedingung}
\end{equation}
die wir bis jetzt noch nicht genutzt haben.
In der Variablen $y$ ausgedrückt ist
\[
\cos^2\frac{\omega+\pi}2
=
\sin^2\frac{\omega}2
=
y.
\]
Die Addition von $\pi$ im Argument $\omega$ entspricht also dem
Übergang von $y$ zu $1-y$.
Wenden wir dies auf die Orthogonalitätsbedingung
\eqref{kompakt:orthogonalitaetsbedingung} an, entsteht die Relation
\begin{equation}
1
=
M(\omega) + M(\omega+\pi)
=
(1-y)^N P(y) + y^N P(1-y).
\end{equation}
Um $A(\omega)$ zu finden ist also ein Polynom
$P(y)$ zu finden, welches die Relation
\eqref{buch:kompakt:Prel} erfüllt.

\begin{konsequenz}
Die Funktion $A(\omega)$ kann aus einem Polynom $P(y)$ konstruiert werden,
welches die Funktionalgleichung
\begin{equation}
(1-y)^N P(y) + y^N P(1-y) = 1
\label{buch:kompakt:Prel}
\end{equation}
erfüllt.
\end{konsequenz}


%
% partial.tex -- 
%
% (c) 2019 Prof Dr Andreas Müller, Hochschule Rapperswil
%

\section{Partialbruchzerlegung\label{section:partialbruch}}
\rhead{Partialbruchzerlegung}
Es muss ein Polynom $P(y)$ gefunden werden mit der Eigenschaft, dass
\begin{equation}
y^N P(1-y) + (1-y)^N P(y) = 1.
\label{buch:kompakt:bedingung}
\end{equation}
Division durch $y^N(1-y)^N$ macht daraus
\[
Q(y)
=
\frac{1}{y^N(1-y)^N}
=
\frac{P(1-y)}{(1-y)^N}
+
\frac{P(y)}{y^N}
\]
Die beiden Terme auf der rechten Seite sind rationale Funktionen, die im
Nenner ausschliesslich Potenzen von $1-y$ im ersten und $y$ im zweiten
Term haben.
Dies sind genau die Nenner, die man in der Partialbruchzerlegung der linken
Seite findet.

Wir berechnen daher die Partialbruchzerlegung
\begin{equation}
\frac{1}{y^N(1-y)^N}
=
\sum_{k=1}^N\frac{C_k}{y^k}
+
\sum_{k=1}^N\frac{C'_k}{(1-y)^k}
\label{buch:kompakt:pbruch}
\end{equation}
von $Q(y)$.
Der Ausdruck $Q(y)$ ist symmetrisch bezüglich bezüglich der Abbildung
$y\leftrightarrow 1-y$.
Da die Koeffizienten der Partialbruchzerlegung eindeutig bestimmt sind,
müssen die Koeffizienten der beiden Summen in \eqref{buch:kompakt:pbruch}
übereinstimmen: $C'_k=C_k$ für $k=1,\dots,N$.

Multipliziert man \eqref{buch:kompakt:pbruch} wieder mit $y^N(1-y)^N$,
erhält man
\begin{align*}
1
&=
(1-y)^N
\underbrace{\sum_{k=1}^\infty C_k y^{N-k}}_{\displaystyle =P(y)}
\mathstrut
+
y^N
\sum_{k=1}^\infty C_k (1-y)^{N-k}
\\
&=
(1-y)^N
P_N(y)
+
y^N
P_N(1-y).
\end{align*}
Das Polynom $P_N(y)$, gebildet mit den Koeffizienten der Partialbruchzerlegung
von $Q(y)$ löst also das eingangs gestellte Problem.

\subsection{Die Fälle $N=2$ und $N=3$}
Wir berechnen die Partialbruchzerlegung für kleine Werte von $N$.
Für $N=2$ ist
\begin{align*}
\frac{1}{y^2(1-y)^2}
&=
\frac{C_1}{y}
+
\frac{C_2}{y^2}
+
\frac{C_1}{(1-y)}
+
\frac{C_2}{(1-y)^2}
\\
&=
\frac{C_1y+C_2}{y^2}
+
\frac{C_1(1-y)+C_2}{(1-y)^2}
\\
&=
\frac{1}{y^2(1-y)^2}
\bigl(
(C_1y+C_2)(1-2y+y^2)
+
(C_1+C_2-C_1y)y^2
\bigr)
\intertext{oder nach Multiplikation mit $y^N(1-y)^N$}
1
&=
C_2
+
(C_1-2C_2) y
+
(C_2-2C_1) y^2
+
C_1 y^3
+
(C_1+C_2)y^2
-C_1y^3
\\
&=
C_2 + (C_1-2C_2) y + (2C_2-C_1)y^2.
\end{align*}
Durch Koeffizientenvergleich findet man die Bedingungen
\[
\begin{aligned}
C_2&=1,
&
C_1-2C_2&=0
&&
\text{und}
&
2C_2-C_1&=0.
\end{aligned}
\]
Die dritte Bedingung ist identisch mit der zweiten.
Aus der ersten und der zweiten Bedingung folgt $C_1=2$.
Es folgt
\[
P_2(y) = C_1y + C_2
=
2y+1.
\]

Die analoge Rechnung für $N=3$ liefert die Bedingungen
\[
\begin{aligned}
C_3&=1
\\
C_2-3C_3&=0     &&\Rightarrow&C_2&=3C_3=3
\\
3C_3-3C_2+C_1&=0&&\Rightarrow&C_1&=3C_2-3C_3=6
\\
4C_2-2C_1&=0    &&\Rightarrow&C_1&=2C_2=6
\\
C_1-2C_2&=0     &&\Rightarrow&C_1&=2C_2=6
\end{aligned}
\]
Man liest daraus das Polynom
\[
P_3(y) = 1+3y+6y^2
\]
ab.
Auf die gleiche Weise findet man auch
\begin{align*}
P_4(y) &=
1 + 4y + 10y^2 + 20y^3
\end{align*}
Daraus kann man die Vermutung ablesen, dass
\begin{equation}
P_N(y)
=
\sum_{k=0}^{N-1}
\binom{N+k-1}{k}
y^k
\label{buch:kompakt:vermutung}
\end{equation}
sein könnte.

\subsection{Der allgemeine Fall}
Das Polynom $P_N(y)$ ist die einzige Lösung vom Grad $N-1$ der Gleichung
$
(1-y)^NP(y)+y^NP(1-y)=1.
$
Die Bedingung ist gleichbedeutend mit
\begin{equation}
P(y) = (1-y)^{-N} (1-y^NP(y)).
\label{buch:kompakt:produkt}
\end{equation}
Das Ziel dieses Abschnitts ist, $P_N(y)$ explizit zu bestimmen.

Um mehr über $P_N(y)$ herauszufinden, können wir beide Seiten von
\eqref{buch:kompakt:produkt} in Taylor-Reihen um den Punkt $y=0$
entwickeln.
Da wir nur ein Polynom vom Grad $N-1$ suchen, können wir die Taylor-Reihen
nach $y^{N-1}$ abbrechen.
Der Ausdruck $y^Np(y)$ auf der rechten Seite von \eqref{buch:kompakt:produkt}
enthält nur Terme vom Grad mindestens $N$ in $y$,
für die Terme vom Grad $<N$ spielt er daher keine Rolle.
Eine Lösung vom Grad $N-1$ erhält man daher, indem man die Reihe
für $(1-y)^{-N}$ nach dem Terme vom Grade $N-1$ abbricht.

Auf der rechten Seite von \eqref{buch:kompakt:produkt}
kann man die Newtonsche Potenzreihe
\begin{equation*}
(1-y)^{\alpha} = \sum_{k=0}^{\infty} \binom{\alpha}{k} (-y)^k
\end{equation*}
verwenden.
Im vorliegenden Fall ist $\alpha=-N$ und damit sind die Koeffizienten
\begin{align*}
\binom{-N}{k}
&=
\frac{(-N)\cdot(-N-1)\cdot\dots\cdot (-N-k+1)}{k\cdot (k-1)\cdot\dots\cdot 1}
=
(-1)^k \frac{N\cdot(N+1)\cdot\dots\cdots (N+k-1)}{k\cdot(k-1)\cdot\dots\cdot 1}
\\
&=
(-1)^k \binom{N+k-1}{k}.
\end{align*}
Daraus können wir das Polynom $P_N(y)$ ablesen.
Wegen $(-y)^k(-1)^k=y^k$ folgt das folgende Lemma,
welches auch die Vermutung \eqref{buch:kompakt:vermutung} bestätigt.

\begin{lemma}
\label{buch:kompakt:lemma-partial}
Das Polynom
\begin{equation}
P_N(y) = \sum_{k=0}^{N-1} \binom{N+k-1}{k}y^k,
\end{equation}
ist das einzige Polynom vom Grad $N$, welches die Gleichung
\[
(1-y)^NP_N(y) + y^NP_N(1-y)=1
\]
erfüllt.
\end{lemma}

\subsection{Weitere Lösungen von höherem Grad}
Das Polynom $P_N(y)$ ist nicht die einzige Lösung der Gleichung
\eqref{buch:kompakt:bedingung}.
Weitere Lösungen $P(y)$ haben aber notwendigerweise Grad mindestens $N$.
Die Differenz $D(y) = P(y)-P_N(y)$ erfüllt die Bedingung
\[
(1-y)^N D(y) + y^N D(1-y) = 0
\qquad\Leftrightarrow\qquad
(1-y)^N D(y) = -y^N D(y-1).
\]
Die rechte Seite hat eine $N$-fache Nullstelle bei $y=0$, also muss
$D(y)$ ebenfalls eine $N$-fache Nullstelle haben. 
Der term niedrigsten Grades in $D(y)$ hat daher mindestens den Grad $N$.
Wir schreiben daher $D(y) = y^NR(y)$.

Damit die Bedingung \eqref{buch:kompakt:bedingung} erfüllt ist, muss
für $R(y)$ die Bedingung
\[
(1-y)^N
D(y)
=
(1-y)^N
y^N
R(y)
=
-y^N
(1-y)^N
R(1-y)
\qquad\Rightarrow\qquad
R(y) = -R(1-y)
\]
erfüllt sein.
Es genügt daher, dass $R(y)$ ein Polynom ist, welches bezüglich $y=\frac12$
antisymmetrisch ist.

\begin{lemma}
Eine beliebige Lösung $P(y)$ der Gleichung 
\[
(1-y)^N P(y) + y^N P(1-y)=1
\]
hat die Form
\[
P(y) = P_N(y) + y^N R(y),
\]
wobei $R(y)$ ein bezüglich $y=\frac12$ antisymmetrisches Polynom ist,
also $R(1-y)=-R(y)$.
\end{lemma}




%
% riesz.tex -- 
%
% (c) 2019 Prof Dr Andreas Müller, Hochschule Rapperswil
%

%
% Lemma von Riesz
%
\section{Das Lemma von Riesz\label{section:riesz}}
\rhead{Das Lemma von Riesz}
Mit Hilfe der Partialbruchzerlegung war es im
Abschnitt~\ref{section:partialbruch}
möglich, die Funktion $M(\omega)$ als Polynom in $\cos\omega$ zu finden,
welches die Identität
\[
M(\omega) + M(\omega + \pi) = 1
\]
erfüllt.
Gesucht wird aber eine Funktion $H(\omega)$ mit
$M(\omega)=H(\omega)H(-\omega)$.
Diese Faktorisierung wird ermöglicht dank des in diesem Abschnitt
formulierten und beschriebenen Lemmas von Riesz.

\subsection{Das Lemma\label{subsection:riesz}}
Das Lemma von Riesz ist die folgende etwas überraschende Aussage:

\begin{lemma}[Riesz]
Ist
\[
A(\omega)
=
\sum_{k=0}^n
a_k \cos^k \omega,
\qquad
a_n\ne 0
\]
mit $A(0)=1$ und $A(\omega)\ge 0$ für alle $\omega$.
Dann gibt es eine Funktion
\[
B(\omega)
=
\sum_{k=0}^n b_ke^{-ik\omega}
\]
mit $B(0)=1$ und $A(\omega)=B(\omega)B(-\omega)$.
\end{lemma}

Im Laufe des nachfolgenden Beweises werden wir wiederholt die folgende
Identität verwenden:
\begin{align}
\frac{z+z^{-1}}2-\frac{s+s^{-1}}2
&=
\frac1{2s}(sz+sz^{-1}-s^2-1)
=
-
\frac1{2s}(1-sz-sz^{-1}+s^2)
\notag
\\
&=
-
\frac1{2s}(z-s)(z^{-1}-s)
\label{buch:kompakt:cosh}
\end{align}

\begin{proof}[Beweis]
Die Funktion $A(\omega)$ kann geschrieben werden als
$A(\omega) = p(\cos\omega)$, wobei $p(x)$ ein Polynom vom
Grad $n$ ist.
Das Polynom $p(x)$ hat $n$ möglicherweise komplexe Nullstellen
$c_1,\dots,c_n$, wobei komplexe Nullstellen in konjugiert komplexen
Paaren auftreten.
Das Polynom $p(x)$ kann daher faktorisiert werden in das Produkt
\begin{align*}
p(x)
&=
a_n(x-c_1)(x-c_2)\dots(x-c_n)
=
a_n \prod_{j=1}^n (x-c_j)
\\
A(\omega) = p(\cos\omega)
&=
a_n \prod_{j=1}^n (\cos \omega - c_j)
\end{align*}
Mit der Abkürzung $z=e^{-i\omega}$ können die Faktoren im Produkt als
\[
\cos\omega -c_j = \frac{z+z^{-1}}2-c_j
\]
geschrieben werden.

Wir versuchen, das gesuchte Polynom $B(\omega)$ aus Lösungen des Problems
für die einzelnen Faktoren von $p(x)$ aufzubauen.
Ein einzelner Faktor $A_j(\omega=)=\cos\omega - c_j$ erfüllt die Bedingungen an
$A$ im Lemma im Allgemeinen nicht.
Ist zum Beispiel $c_j$ reell zwischen $-1$ und $1$, dann hat
$\cos\omega-c_j$ eine Nullstelle hat daher auch negative Werte.
Wenn $c_j$ komplex ist, dann ist $\cos\omega-c_j$ keine rellwertige
Funktion.
Wir unterscheiden daher die folgenden drei Fälle:
\begin{enumerate}
\item
Fall $|c_j|\ge 1$ und $c_j\in\mathbb R$:
In diesem Fall kann $c_j$ geschrieben werden als
\[
c_j = \frac{s+s^{-1}}2
\quad
\text{mit $s=\operatorname{sign}(c_j)\cdot \operatorname{arsinh}|c_j|$.}
\]
Dann wird
\[
A_j(\omega)
=
\cos\omega - c_j
=
\frac{z+z^{-1}}2 - \frac{s+s^{-1}}2
=
-\frac1{2s} (z-s)(z^{-1}-s)
\]
mit Hilfe von \eqref{buch:kompakt:cosh}.
Falls $c_j < 0$ ist auch $s<0$ und wir können dem Produkt $B(\omega)$
den Faktor $B_j(\omega)=(z-s)/\sqrt{2s}$ hinzufügen, für den gilt
$B_j(\omega)B_j(-\omega)=A_j(\omega)$.
Falls $c_j > 0$ ist auch $s>0$ und wir können dem Produkt $B(\omega)$
den Faktor $B_j(\omega) =(z-s)/\sqrt{-2s}$ hinzufügen, für den
gilt $B_j(\omega)B_j(-\omega)=-A_j(\omega)$.
Das Vorzeichen im zweiten Fall stört nicht, weil wegen $A(\omega)\ge 0$
noch ein weiterer Faktor $A_{j'}(\omega)$ vorhanden sein muss, der ebenfalls
$A_{j'}(\omega)\le 0$ ist für alle $\omega$ und der daher ebenfalls
mit dem ``falschen'' Vorzeichen in das Produkt eingeht, wodurch das
Vorzeichen wieder korrigiert wird.
\item
Fall $|c_j|\ge 1$ und $c_j\in\mathbb R$:
In diesem Fall hat $A_j(\omega)$ und damit $A(\omega)$ eine Nullstelle
für jedes $\omega$ mit $\cos\omega=c_j$.
Wegen $A(\omega)\ge 0$ müssen alle solchen Nullstellen in gerader
Anzahl auftreten.
Sei $\alpha$ so, dass $\cos\alpha = c_j$ und damit insbesondere auch
\[
c_j
=
\cos\alpha
=
\frac{s+s^{-1}}2
\quad\text{mit $s=e^{i\alpha}$}.
\]
Wegen der Symmetrie der Funktion $\cos\omega$ gilt auch
\[
c_{j'}=
\cos(-\alpha)
=
\frac{e^{-i\alpha}+e^{i\alpha}}2.
\]
Damit kann $A_j(\omega)^2$ geschrieben werden als
\begin{align*}
A_j(\omega)^2
&=
\biggl(
\frac{z+z^{-1}}2-\frac{e^{i\alpha}+e^{-i\alpha}}2
\biggr)
\biggl(
\frac{z+z^{-1}}2-\frac{e^{-i\alpha}+e^{i\alpha}}2
\biggr)
\\
&=
\frac{1}{2e^{i\alpha}}
(z-e^{i\alpha})
(z^{-1}-e^{i\alpha})
\frac{1}{2e^{-i\alpha}}
(z-e^{-i\alpha})
(z^{-1}-e^{-i\alpha})
\\
&=
\frac{1}{2}
(z-e^{i\alpha})
(z-e^{-i\alpha})
\cdot
\frac{1}{2}
(z^{-1}-e^{i\alpha})
(z^{-1}-e^{-i\alpha})
\\
&=
\frac{1}{2}
(z^2-2z\cos\alpha +1)
\cdot
\frac{1}{2}
(z^{-2}-2z^{-1}\cos\alpha +1)
\end{align*}
Wir fügen daher der Funktion $B(\omega)$ den Faktor
$B_j(\omega)=\frac12(z^2-2z\cos\alpha +1)$ hinzu, für
den 
$B_j(\omega)B_j(-\omega)=A_j(\omega)^2$ gilt.
\item
Fall $c_j\in\mathbb C\setminus\mathbb R$:
In diesem Fall gibt es einen zweiten, komplex konjugierten Faktor mit
$c_{j'}=\bar{c}_j$.
Wir möchten $c_j$ wieder als
\begin{equation}
c_j
=
\frac{s+s^{-1}}2
\label{buch:kompakt:cj}
\end{equation}
darstellen.
Dazu multiplizieren wir \eqref{buch:kompakt:cj} mit $s$ und erhalten die
quadratische Gleichung
\[
s^2-2c_j s+1=0
\]
mit der Lösung
\[
s=c_j\pm\sqrt{c_j^2-1},
\]
die immer mindestens eine Lösung hat.
Wir dürfen daher annehmen, dass
\[
c_j = \frac{s+s^{-1}}2
\qquad\text{und}\qquad
c_{j'} = \bar{c}_j = \frac{\bar{s}+\bar{s}^{-1}}2.
\]
Damit können wir jetzt das Produkt $A_j(\omega)A_{j'}(\omega)$ 
wie folgt faktorisieren:
\begin{align*}
A_j(\omega)A_{j'}(\omega)
&=
\biggl(
\frac{z+z^{-1}}2 - \frac{s+s^{-1}}2
\biggr)
\biggl(
\frac{z+z^{-1}}2 - \frac{\bar{s}+\bar{s}^{-1}}2
\biggr)
\\
&=
\frac{1}{2s}
(z-s)(z^{-1}-s)
\cdot
\frac{1}{2\bar{s}}
(z-\bar{s})(z^{-1}-\bar{s})
\\
&=
\frac{1}{4|s|^2}(z-s)(z^{-1}-s)(z-\bar{s})(z^{-1}-\bar{s})
\\
&=
\frac1{2|s|}
(z-s)
(z-\bar{s})
\cdot
\frac1{2|s|}
(z^{-1}-s)
(z^{-1}-\bar{s})
\\
&=
\frac1{2|s|}(z^2-2z\operatorname{Re}s -|s|^2)
\cdot
\frac1{2|s|}(z^{-2}-2z^{-1}\operatorname{Re}s -|s|^2).
\end{align*}
Wir fügen in diesem Fall den Faktor $B_j(\omega)=(z^2-2z\operatorname{Re}s-|s|^2)/2|s|$ hinzu.
\end{enumerate}
In allen drei Fällen haben wir für die behandelten Faktoren $A_j(\omega)$
und der eventuell zugehörigen Faktoren $A_{j'}(\omega)$ eine Funktion
$B_j(\omega)$ gefunden, so dass das Produkt all dieser Faktoren
\[
B(\omega) = \prod_j B_j(\omega)
\]
genau die im Lemma versprochenen Eigenschaften hat.
\end{proof}

\subsection{Haar-Wavelet}

\subsection{Daubechies-Wavelets}





%
% daubechies.tex
%
% (c) 2019 Prof Dr Andreas Müller, Hochschule Rapperswil
%
\section{Daubechies-Wavelets\label{section:daubechies}}
\rhead{Daubechies-Wavelets}
In den vorangegangenen Abschnitten wurde gezeigt, wie die erzeugende
Funktion $H(s)$ einer Multi\-skalen-Analyse mit Wavelets beliebiger
Ordnung bestimmt werden kann.
Es wurde auch klar, dass das Haar-Wavelet eine Lösung dieser Aufgabe
ist.
Nachdem die Existenz solcher Multiskalen-Ana\-ly\-sen nun etabliert ist,
sollen in diesem Abschnitt die Koeffizienten $h_k$ dieser Wavelets
explizit konstruiert werden.

In Lemma~\ref{buch:kompakt:lemma-partial} wurden die Lösungen für das
Polynom $P(y)$ bereits charakterisiert.
Um die erzeugende Funktion zu bestimmen, sind noch folgende Schritte
durchzuführen.
\begin{enumerate}
\item
Die Funktion $A(\omega)=P(y)$ mit $y=\sin^2\frac{\omega}2$ muss
bestimmt und als Polynom in $\cos\omega=1-2y$ ausgedrückt werden.
\item
Die Funktion $A(\omega)$ muss mit Hilfe des Lemmas \ref{lemma:riesz}
von Riesz in das Produkt $A(\omega)=B(\omega)B(-\omega)$ zerlegt
werden.
\item
Die erzeugende Funktion ist dann
\[
H(\omega)
=
\biggl(
\frac{1+e^{-i\omega}}2
\biggr)^N
B(\omega),
\]
aus ihr müssen die Koeffizienten $h_k$ abgelesen werden.
\end{enumerate}

Wir führen den Prozess im folgenden für $N=2$ und $N=3$ durch.

\subsection{Der Fall $N=2$}
Für $N=2$ is $P_N(y)=P_2(y)=1+2y$.
Wegen $y=(1-\cos \omega)/2$ folgt
\[
A(\omega) = 1 + 2\frac{1-\cos\omega}2 = 2-\cos\omega
= 2- \cos\omega
\]
Wegen $\cos\omega\le 1$ ist $A(\omega) \ge 0$, die Voraussetzungen des
Riesz-Lemmas~\ref{lemma:riesz} sind damit erfüllt.

Im Beispiel auf Seite~\pageref{buch:kompakt:db2riesz} wird bereits
die Funktion $B(z)$ gefunden, die nach dem Lemma~\ref{lemma:riesz}
von Riesz existieren muss.
Es war
\[
B(\omega)
=
\frac{1+\sqrt{3}}2 + \frac{1-\sqrt{3}}2e^{-i\omega}.
\]
Daraus kann man jetzt die erzeugende Funktion ableiten
\begin{align*}
H(\omega)
&=
\biggl(\frac{1+e^{-i\omega}}2\biggr)^2 B(\omega)
=
\frac14(1+2e^{-i\omega}+e^{-2i\omega})
\biggl(
\frac{1+\sqrt{3}}2 + \frac{1-\sqrt{3}}2e^{-i\omega}
\biggr).
\\
&=
\frac18(
(1+\sqrt{3})
+
(2(1+\sqrt{3})+1-\sqrt{3}) e^{-i\omega}
+
(1+\sqrt{3}+2(1-\sqrt{3})) e^{-2i\omega}
+
(1-\sqrt{3})e^{-3i\omega}
)
\\
&=
\frac14\biggl(
\frac{1+\sqrt{3}}2
+
\frac{3+\sqrt{3}}2 e^{-i\omega}
+
\frac{3-\sqrt{3}}2 e^{-2i\omega}
+
\frac{1-\sqrt{3}}2 e^{-3i\omega}
\biggr)
\\
&=
\frac1{\sqrt{2}}
\biggl(
\frac{1+\sqrt{3}}{4\sqrt{2}}
+
\frac{3+\sqrt{3}}{4\sqrt{2}} e^{-i\omega}
+
\frac{3-\sqrt{3}}{4\sqrt{2}} e^{-2i\omega}
+
\frac{1-\sqrt{3}}{4\sqrt{2}} e^{-3i\omega}
\biggr)
\end{align*}
Daraus kann man die Koeffizienten der Skalierungsrelation ablesen:
\begin{align*}
h_0
&=
\frac{1+\sqrt{3}}{4\sqrt{2}}
=
\phantom{-}
0.482962913144534
\\
h_1
&=
\frac{3+\sqrt{3}}{4\sqrt{2}}
=
\phantom{-}
0.836516303737808
\\
h_2
&=
\frac{3-\sqrt{3}}{4\sqrt{2}}
=
\phantom{-}
0.224143868042013
\\
h_3
&=
\frac{1-\sqrt{3}}{4\sqrt{2}}
=
-0.129409522551260
\end{align*}


\subsection{Der Fall $N=3$}
Für $N=3$ ist $P_N(y)=P_3(y)=1+3y+6y^2$.
Wir ersetzen wieder $y=(1-\cos\omega)/2$ 
\begin{align*}
A(\omega)
&=
1+3(1-\cos\omega)/2 + 6(1-\cos\omega)^2/4
\\
&=
1+\frac32-\frac32\cos\omega
+
\frac32(1-2\cos\omega+\cos^2\omega)
\\
&=
4
-
\frac92\cos\omega
+
\frac32\cos^2\omega
\\
&=
\frac32(\cos^2\omega -3\cos\omega + \frac{8}3)
\end{align*}
Jetzt muss eine Funktion $B(\omega)$ als Polynom in $e^{i\omega}$
gefunden werden derart, dass $A(\omega)=B(\omega)B(-\omega)$.
Der Ansatz
\[
B(\omega)
=
b_0 + b_1 e^{i\omega} + b_2 e^{2i\omega}
\]
führt auf
\begin{align*}
A(\omega)
=
B(\omega)B(-\omega)
&=
(b_0 + b_1 e^{i\omega} + b_2 e^{2i\omega})
(b_0 + b_1 e^{-i\omega} + b_2 e^{-2i\omega})
\\
&=
b_0^2 + b_1^2 + b_2^2
+
(b_0b_1 + b_2b_1)
(e^{i\omega}+e^{-i\omega})
+
b_0b_2
(e^{2i\omega}+e^{-2i\omega})
\\
&=
b_0^2 + b_1^2 + b_2^2
+
2(b_0b_1 + b_2b_1)
\frac{e^{i\omega}+e^{-i\omega}}2
+
2b_0b_2
\frac{e^{2i\omega}+e^{-2i\omega}}2
\\
&=
b_0^2 + b_1^2 + b_2^2
+
2(b_0b_1 + b_2b_1)
\cos\omega
+
2b_0b_2
\cos 2\omega
\\
&=
b_0^2 + b_1^2 + b_2^2
+
2(b_0b_1 + b_2b_1)
\cos\omega
+
2b_0b_2
(2\cos^2\omega-1)
\\
&=
b_0^2 + b_1^2 + b_2^2 -2b_0b_2
+
2(b_0b_1 + b_2b_1)
\cos\omega
+
4b_0b_2
\cos^2\omega
\end{align*}
Durch Koeffizientenvergleich lesen wir die Gleichungen
\begin{align*}
b_0^2+b_1^2+b_2^2-2b_0b_2&=\frac83
\\
(b_0+b_2)b_1 &= -\frac94
\\
b_0b_2
&=
\frac{3}{8}
\end{align*}








