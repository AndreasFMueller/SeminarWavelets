%
% chapter.tex -- Kapitel über Wavelets mit kompaktem Träger
%
% (c) 2019 Prof Dr Andreas Müller, Hochschule Rapperswil
%
\chapter{Wavelets mit kompaktem Träger
\label{chapter:kompakt}}
\lhead{Wavelets mit kompaktem Träger}
Für die Praxis sind nur Wavelets interessant, für die die
Wavelet-Transformation und die Rücktransformation mit beschränktem Aufwand,
effizient und stabil berechnet werden können.
Die Untersuchungen in Kapitel~\ref{chapter:algo} haben gezeigt, dass 
die diskrete Wavelet-Transformation mit einem einfachen kaskadierten
Faltungsfilter berechnet werden kann.
Ein solcher Filter kann aber nur dann effizient berechnet werden,
wenn nur endlich viele der Filterkoeffizienten von 0 verschieden sind.

Das Haar-Wavelet hat diese Eigenschaft, genau zwei der Filterkoeffizienten
sind von 0 verschieden.
Lange Zeit war nicht klar, aber es überhaupt andere Wavelets mit
nur endlich vielen nicht verschwindenden Filterkoeffizienten gibt.
Yves Meyer versuchte sogar zu zeigen, dass es gar keine solche Funktione
$\varphi$ und $\psi$ gibt, wie sie in der Multiskalen-Analyse verlangt
werden, also erst recht keine mit endlich vielen nicht verschwindenden
Koeffizienten.
Als es ihm dann 1985 gelang, solche Wavelets zu finden, stellte sich
die Frage nach Wavelets mit kompaktem Träger erneut.
Schliesslich gelang es Ingrid Daubechies, solche Wavelets zu konstruieren.







