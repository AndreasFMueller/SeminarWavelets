%
% bedingungen.tex
%
% (c) 2019 Prof Dr Andreas Müller, Hochschule Rapperswil
%
\section{Bedingungen für die Koeffizienten\label{section:bedingungen}}
\rhead{Bedingungen für die Koeffizienten}
Bis jetzt ist ausser dem Haar-Wavelet kein Beispiel einer
Multiskalen-Analyse konstruiert worden, in dem die Funktionen 
$\varphi$ und $\psi$ kompakten Träger haben.

\subsection{Kompakter Träger}
Wir gehen davon aus, dass der Träger der Funktion $\varphi$ im Interval
$[a,b]$ enthalten ist.
Dann ist der Träger der Translate $T_k\varphi$ in $[a+k,b+k]$ enthalten
und der Träger der skalierten Funktion $D_{\frac12}T_k\varphi$ ist
enthalten in $[\frac12(a+k),\frac12(b+k)]$.

Die Koeffizienten der Skalierungsrelation können mit Hilfe des
Skalarproduktes erhalten werden, denn es gilt:
\begin{align*}
\langle D_{\frac12}T_r\varphi, \varphi\rangle
&=
\biggl\langle
D_{\frac12}T_r\varphi, \sum_{k\in\mathbb Z} D_{\frac12}T_k\varphi
\biggr\rangle
=
\sum_{k\in\mathbb Z}
\langle
D_{\frac12}T_r\varphi,
h_kD_{\frac12}T_k\varphi
\rangle
\\
&=
\sum_{k\in\mathbb Z}
\bar{h}_k
\langle
D_{\frac12}T_r\varphi,
D_{\frac12}T_k\varphi
\rangle
=
\sum_{k\in\mathbb Z}
\bar{h}_k
\delta_{kr}
=
\bar{h}_r.
\end{align*}
Das Skalarprodukt
$\langle \varphi, D_{\frac12}T_k\varphi\rangle$
ist ein Integral über das Produkt der Funktion $\varphi$ mit Träger im Interval
$[a,b]$ und der Funktion $D_{\frac12}T_r\varphi$  mit Träger in
$[\frac12(a+r),\frac12(b+r)]$.
Dieses Integral verschwindet, wenn die Intervalle nicht überlappen.
Dieser Fall tritt ein, wenn
\[
\begin{aligned}
b&<\frac12(a+r) &&\text{oder}& \frac12(b+r) < a
\\
2b-a&<r         &&           & r < 2a-b
\end{aligned}
\]
Die einzigen Koeffizienten $h_r$, die nicht verschwinden können, erfüllen
\[
2b-a < r < 2a-b,
\]
es können also nur endlich viele Koeffizienten $h_r$ von $0$ verschieden
sein.

\subsection{Normierung}
Wir gehen wieder von einer Multiskalen-Analyse mit einem Vater-Wavelet
$\varphi$ mit kompaktem Träger aus.
Die Analyse des konstanten Signals $f(t)=1$ erfolgt mit Hilfe des
Skalarprodukts $\langle f,\varphi\rangle$.
Anwendung der Skalierungsrelation auf $\varphi$ liefert
\begin{align*}
\langle 1,\varphi\rangle
&=
\biggl\langle
1,\sum_{k\in\mathbb Z} h_kD_{\frac12}T_k\varphi
\biggr\rangle
=
\sum_{k\in\mathbb Z}
\bar{h}_k
\langle
1,
D_{\frac12}T_k\varphi
\rangle
\\
&=
\sum_{k\in\mathbb Z}
\bar{h}_k
\int_{-\infty}^\infty \sqrt{2}\varphi(2t-k)\,dt
=
\sum_{k\in\mathbb Z}
\bar{h}_k
\int_{-\infty}^\infty \sqrt{2} \varphi(\tau) \,\frac{d\tau}{2}
=
\sum_{k\in\mathbb Z}
\bar{h}_k
\frac{1}{\sqrt{2}}
\int_{-\infty}^\infty \varphi(\tau) \,d\tau
\\
&=
\sum_{k\in\mathbb Z}
\bar{h}_k
\frac{1}{\sqrt{2}}
\langle 1,\varphi\rangle.
\end{align*}
Da $\langle 1,\varphi\rangle\ne 0$ ist, können wir durch
$\langle 1,\varphi\rangle$ dividieren und es folgt
\begin{equation}
\sum_{k\in\mathbb Z} \bar{h}_k = \sqrt{2}
\quad\Rightarrow\quad
\sum_{k\in\mathbb Z} h_k = \sqrt{2}.
\label{buch:kompakt:hsumme}
\end{equation}
Diese Identität kann nützlich sein, wenn man Koeffizienten aus einer
nicht genauer bekannten Quelle benützen will und nicht sicher ist,
ob sie die gleiche Normierungskonvention für die Skalierungsrelation
verwendet.

\begin{beispiel}
Für das Haarwavelet sind nur zwei Koeffizienten $h_k$ von $0$ verschieden
und beide sind gleich gross.
Aus der Relation \eqref{buch:kompakt:hsumme} folgt
\[
\sqrt{2}
=
h_0+h_1
=2h_0
\qquad\Rightarrow\qquad
h_0 = h_1 = \frac{\sqrt{2}}{2}=\frac{1}{\sqrt{2}}.
\qedhere
\]
\end{beispiel}

\subsection{Orthogonalität}
Aus der Orthonormalität der Funktionen $T_k\varphi$ einer Multiskalenanalyse
folgen die Relationen
\begin{align}
\delta_{0k}
=
\langle \varphi,T_k \varphi\rangle
&=
\biggl\langle
\sum_{l\in\mathbb Z} h_lT_l\varphi,
\sum_{r\in\mathbb Z} h_rT_{r+2k}\varphi
\biggr\rangle
\notag
\\
&=
\sum_{l,r\in\mathbb Z} h_l\bar{h}_r \langle T_l\varphi, T_{r+2k}\varphi\rangle
\\
&=
\sum_{l,r\in\mathbb Z} h_l\bar{h}_r \delta_{l,r+2k}
=
\sum_{r\in\mathbb Z} h_{r+2k}\bar{h}_r
\label{buch:kompakt:orthorel}
\end{align}
Nehmen wir an, dass genau die Koeffizienten $h_k$ mit $k$ zwischen
$0$ und $N$ von $0$ verschieden sind.
Dann wird die Summe~\eqref{buch:kompakt:orthorel} trivialerweise
verschwinden, wenn $2k>N$.
Dies liefert maximal $N/2$ Bedinungen für die $N+1$ Koeffizienten
$h_k$, sie werden daher im Allgemeinen noch nicht bestimmt sein.
Weitere Bedingungen müssen formuliert werden, um die Koeffizienten
festzulegen.

\begin{beispiel}
Die Skalierungsrelation des Haar-Wavelets enthält nur zwei Koeffizienten
$h_0$ und $h_1$, also $N=2$.
Von den Relationen \eqref{buch:kompakt:orthorel} ist nur jene mit $k=0$
nichttrivial, es gilt
\[
1
=
\delta_{0k}
=
h_0\bar{h}_0 + h_1\bar{h}_1
=
|h_0|^2+|h_1|^2
=
\frac12+\frac12
=1.
\qedhere
\]
\end{beispiel}

\begin{beispiel}
Gesucht ist die Skalierungsrelation für eine Multiskalenanalyse derart,
dass $\varphi$ Träger im Interval $[0,1]$ hat.
Daraus kann man zunächst schliessen, dass die Skalierungsrelation
genau zwei von $0$ verschiedene Koeffizienten $h_0$ und $h_1$ hat.
Aus~\eqref{buch:kompakt:hsumme} und \eqref{buch:kompakt:orthorel}
folgen die Gleichungen
\begin{align*}
h_0+h_1&=\sqrt{2},
\\
|h_0|^2+|h_1|^2&=1.
\end{align*}
Wir suchen eine Lösung in der Form $h_0=a+bi$ und $h_1=\sqrt{2}-a-bi$.
Die Orthonormalisierungsrelation liefert dann
\begin{align}
1
&=
a^2 + b^2
+
(\sqrt{2}-a)^2 + b^2
\notag
\\
&=
2a^2 + 2b^2 - 2\sqrt{2}a + 2
\notag
\\
0
&=
a^2 - \sqrt{2}a + \frac12 + b^2
\label{buch:kompakt:quadr}
\end{align}
Dies ist eine quadratische Gleichung für $a$.
Für $b=0$ wird sie zu
\[
0
=
a^2 - \sqrt{2}a + \frac12
=
\biggl(a-\frac1{\sqrt{2}}\biggr)^2,
\]
also ist in diesem Fall $a=\frac1{\sqrt{2}}$ und damit
\[
h_0=h_1=\frac{1}{\sqrt{2}},
\]
also die Koeffizienten der Skalierungsrelation des Haar-Wavelets.

Die quadratische Gleichung \eqref{buch:kompakt:quadr} hat nur dann
relle Lösungen für $a$, wenn die Diskriminante
$D=\sqrt{2}^2 - 4\cdot (b^2+\frac12) \ge 0$ ist:
\[
D=2-2-4b^2 =-4b^2\le 0.
\]
Es folgt, dass $b=0$ die einzige Lösung ist.
\end{beispiel}

Das Beispiel zeigt, dass das Haar-Wavelet durch die bisher gefundenen
Relationen bereits eindeutig festgelegt ist.
Es gibt also nur eine einzige Multiskalen-Analyse mit einem Wavelet,
dessen Träger im Interval $[0,1]$ enthalten ist, nämlich das Haar-Wavelet.


