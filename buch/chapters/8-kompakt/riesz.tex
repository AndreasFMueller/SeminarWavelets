%
% riesz.tex -- 
%
% (c) 2019 Prof Dr Andreas Müller, Hochschule Rapperswil
%

%
% Lemma von Riesz
%
\section{Das Lemma von Riesz\label{section:riesz}}
\rhead{Das Lemma von Riesz}
Mit Hilfe der Partialbruchzerlegung war es im
Abschnitt~\ref{section:partialbruch}
möglich, die Funktion $M(\omega)$ als Polynom in $\cos\omega$ zu finden,
welches die Identität
\[
M(\omega) + M(\omega + \pi) = 1
\]
erfüllt.
Gesucht wird aber eine Funktion $H(\omega)$ mit
$M(\omega)=H(\omega)H(-\omega)$.
Diese Faktorisierung wird ermöglicht dank des in diesem Abschnitt
formulierten und beschriebenen Lemmas von Riesz.
Das Lemma von Riesz ist die folgende etwas überraschende Aussage:

\begin{lemma}[Riesz]
\label{lemma:riesz}
Ist
\[
A(\omega)
=
\sum_{k=0}^n
a_k \cos^k \omega,
\qquad
a_n\ne 0
\]
mit $A(0)=1$ und $A(\omega)\ge 0$ für alle $\omega$.
Dann gibt es eine Funktion
\[
B(\omega)
=
\sum_{k=0}^n b_ke^{-ik\omega}
\]
mit $B(0)=1$ und $A(\omega)=B(\omega)B(-\omega)$.
\end{lemma}

Im Laufe des nachfolgenden Beweises werden wir wiederholt die folgende
Identität verwenden:
\begin{align}
\frac{z+z^{-1}}2-\frac{s+s^{-1}}2
&=
\frac1{2s}(sz+sz^{-1}-s^2-1)
=
-
\frac1{2s}(1-sz-sz^{-1}+s^2)
\notag
\\
&=
-
\frac1{2s}(z-s)(z^{-1}-s)
\label{buch:kompakt:cosh}
\end{align}

\begin{proof}[Beweis]
Die Funktion $A(\omega)$ kann geschrieben werden als
$A(\omega) = p(\cos\omega)$, wobei $p(x)$ ein Polynom vom
Grad $n$ ist.
Das Polynom $p(x)$ hat $n$ möglicherweise komplexe Nullstellen
$c_1,\dots,c_n$, wobei komplexe Nullstellen in konjugiert komplexen
Paaren auftreten.
Das Polynom $p(x)$ kann daher faktorisiert werden in das Produkt
\begin{align*}
p(x)
&=
a_n(x-c_1)(x-c_2)\dots(x-c_n)
=
a_n \prod_{j=1}^n (x-c_j)
\\
A(\omega) = p(\cos\omega)
&=
a_n \prod_{j=1}^n (\cos \omega - c_j)
\end{align*}
Mit der Abkürzung $z=e^{-i\omega}$ können die Faktoren im Produkt als
\[
\cos\omega -c_j = \frac{z+z^{-1}}2-c_j
\]
geschrieben werden.

Wir versuchen, das gesuchte Polynom $B(\omega)$ aus Lösungen des Problems
für die einzelnen Faktoren von $p(x)$ aufzubauen.
Ein einzelner Faktor $A_j(\omega=)=\cos\omega - c_j$ erfüllt die Bedingungen an
$A$ im Lemma im Allgemeinen nicht.
Ist zum Beispiel $c_j$ reell zwischen $-1$ und $1$, dann hat
$\cos\omega-c_j$ eine Nullstelle hat daher auch negative Werte.
Wenn $c_j$ komplex ist, dann ist $\cos\omega-c_j$ keine rellwertige
Funktion.
Wir unterscheiden daher die folgenden drei Fälle:
\begin{enumerate}
\item
Fall $|c_j|\ge 1$ und $c_j\in\mathbb R$:
In diesem Fall kann $c_j$ geschrieben werden als
\[
c_j = \frac{s+s^{-1}}2
\quad
\text{mit $s=\operatorname{sign}(c_j)\cdot \operatorname{arsinh}|c_j|$.}
\]
Dann wird
\[
A_j(\omega)
=
\cos\omega - c_j
=
\frac{z+z^{-1}}2 - \frac{s+s^{-1}}2
=
-\frac1{2s} (z-s)(z^{-1}-s)
\]
mit Hilfe von \eqref{buch:kompakt:cosh}.
Falls $c_j < 0$ ist auch $s<0$ und wir können dem Produkt $B(\omega)$
den Faktor $B_j(\omega)=(z-s)/\sqrt{2s}$ hinzufügen, für den gilt
$B_j(\omega)B_j(-\omega)=A_j(\omega)$.
Falls $c_j > 0$ ist auch $s>0$ und wir können dem Produkt $B(\omega)$
den Faktor $B_j(\omega) =(z-s)/\sqrt{-2s}$ hinzufügen, für den
gilt $B_j(\omega)B_j(-\omega)=-A_j(\omega)$.
Das Vorzeichen im zweiten Fall stört nicht, weil wegen $A(\omega)\ge 0$
noch ein weiterer Faktor $A_{j'}(\omega)$ vorhanden sein muss, der ebenfalls
$A_{j'}(\omega)\le 0$ ist für alle $\omega$ und der daher ebenfalls
mit dem ``falschen'' Vorzeichen in das Produkt eingeht, wodurch das
Vorzeichen wieder korrigiert wird.
\item
Fall $|c_j|\ge 1$ und $c_j\in\mathbb R$:
In diesem Fall hat $A_j(\omega)$ und damit $A(\omega)$ eine Nullstelle
für jedes $\omega$ mit $\cos\omega=c_j$.
Wegen $A(\omega)\ge 0$ müssen alle solchen Nullstellen in gerader
Anzahl auftreten.
Sei $\alpha$ so, dass $\cos\alpha = c_j$ und damit insbesondere auch
\[
c_j
=
\cos\alpha
=
\frac{s+s^{-1}}2
\quad\text{mit $s=e^{i\alpha}$}.
\]
Wegen der Symmetrie der Funktion $\cos\omega$ gilt auch
\[
c_{j'}=
\cos(-\alpha)
=
\frac{e^{-i\alpha}+e^{i\alpha}}2.
\]
Damit kann $A_j(\omega)^2$ geschrieben werden als
\begin{align*}
A_j(\omega)^2
&=
\biggl(
\frac{z+z^{-1}}2-\frac{e^{i\alpha}+e^{-i\alpha}}2
\biggr)
\biggl(
\frac{z+z^{-1}}2-\frac{e^{-i\alpha}+e^{i\alpha}}2
\biggr)
\\
&=
\frac{1}{2e^{i\alpha}}
(z-e^{i\alpha})
(z^{-1}-e^{i\alpha})
\frac{1}{2e^{-i\alpha}}
(z-e^{-i\alpha})
(z^{-1}-e^{-i\alpha})
\\
&=
\frac{1}{2}
(z-e^{i\alpha})
(z-e^{-i\alpha})
\cdot
\frac{1}{2}
(z^{-1}-e^{i\alpha})
(z^{-1}-e^{-i\alpha})
\\
&=
\frac{1}{2}
(z^2-2z\cos\alpha +1)
\cdot
\frac{1}{2}
(z^{-2}-2z^{-1}\cos\alpha +1)
\end{align*}
Wir fügen daher der Funktion $B(\omega)$ den Faktor
$B_j(\omega)=\frac12(z^2-2z\cos\alpha +1)$ hinzu, für
den 
$B_j(\omega)B_j(-\omega)=A_j(\omega)^2$ gilt.
\item
Fall $c_j\in\mathbb C\setminus\mathbb R$:
In diesem Fall gibt es einen zweiten, komplex konjugierten Faktor mit
$c_{j'}=\bar{c}_j$.
Wir möchten $c_j$ wieder als
\begin{equation}
c_j
=
\frac{s+s^{-1}}2
\label{buch:kompakt:cj}
\end{equation}
darstellen.
Dazu multiplizieren wir \eqref{buch:kompakt:cj} mit $s$ und erhalten die
quadratische Gleichung
\[
s^2-2c_j s+1=0
\]
mit der Lösung
\[
s=c_j\pm\sqrt{c_j^2-1},
\]
die immer mindestens eine Lösung hat.
Wir dürfen daher annehmen, dass
\[
c_j = \frac{s+s^{-1}}2
\qquad\text{und}\qquad
c_{j'} = \bar{c}_j = \frac{\bar{s}+\bar{s}^{-1}}2.
\]
Damit können wir jetzt das Produkt $A_j(\omega)A_{j'}(\omega)$ 
wie folgt faktorisieren:
\begin{align*}
A_j(\omega)A_{j'}(\omega)
&=
\biggl(
\frac{z+z^{-1}}2 - \frac{s+s^{-1}}2
\biggr)
\biggl(
\frac{z+z^{-1}}2 - \frac{\bar{s}+\bar{s}^{-1}}2
\biggr)
\\
&=
\frac{1}{2s}
(z-s)(z^{-1}-s)
\cdot
\frac{1}{2\bar{s}}
(z-\bar{s})(z^{-1}-\bar{s})
\\
&=
\frac{1}{4|s|^2}(z-s)(z^{-1}-s)(z-\bar{s})(z^{-1}-\bar{s})
\\
&=
\frac1{2|s|}
(z-s)
(z-\bar{s})
\cdot
\frac1{2|s|}
(z^{-1}-s)
(z^{-1}-\bar{s})
\\
&=
\frac1{2|s|}(z^2-2z\operatorname{Re}s -|s|^2)
\cdot
\frac1{2|s|}(z^{-2}-2z^{-1}\operatorname{Re}s -|s|^2).
\end{align*}
Wir fügen in diesem Fall den Faktor $B_j(\omega)=(z^2-2z\operatorname{Re}s-|s|^2)/2|s|$ hinzu.
\end{enumerate}
In allen drei Fällen haben wir für die behandelten Faktoren $A_j(\omega)$
und der eventuell zugehörigen Faktoren $A_{j'}(\omega)$ eine Funktion
$B_j(\omega)$ gefunden, so dass das Produkt all dieser Faktoren
\[
B(\omega) = \prod_j B_j(\omega)
\]
genau die im Lemma versprochenen Eigenschaften hat.
\end{proof}

Aus dem Beweis des Lemmas lässt sich auch ein Algorithmus ablesen, mit
dem das Polynom $B(\omega)$ konstruiert werden kann.
\begin{enumerate}
\item
Bestimme die Nullstellen von $A(\omega) = P((1-\cos \omega)/2)$ 
als Polynom in $\omega$ und schreibe $A(\omega)$ als Produkt
\[
A(\omega) = \prod_j (\cos\omega - c_j)
\]
\item Für jede Nullstelle $c_j$ mit $|c_j|>1$ füge einen Faktor 
\[
B_j(z)=\frac{1}{\sqrt{-2s}}(z-s),\qquad s = \operatorname{sign}c_j\cdot \operatorname{arsinh}|c_j|
\]
zu $B(z)$ hinzu.
\item
Jede Nullstelle $c_j \in\mathbb R$ mit $|c_j|<1$ kommt eine Gerade Anzahl mal
vor, füge für jedes Paar einen Faktor
\[
B_j(z) = \frac12(z^2 -2 z\cos \alpha +1),
\qquad\text{mit $c_j=\cos\alpha$}
\]
hinzu.
\item
Für jede Nullstelle $c_j\in \mathbb C\setminus\mathbb R$ gibt es
eine konjugiert komplexe Nullstelle $c_{j'}=\bar{c}_j$.
Füge für beide einen Faktor
\[
B_j(z) = \frac{1}{2|s|}(z^2 -2z\operatorname{Re}s-|s|^2),
\qquad
\text{wobei $s$ eine Lösung von $s^2 -2c_js+1=0$ ist,}
\]
hinzu.
\end{enumerate}

\begin{beispiel}
\label{buch:kompakt:db2riesz}
Als Beispiel für den im Beweis des Lemmas von Riesz formulierten Algorithmus
wenden wir ihn auf das Polynom
\[
A(\omega)
=
2-\cos\omega
=
2-\frac12(e^{i\omega}+e^{-i\omega})
\]
an.
Das Polynom $-(\cos\omega -2)$ hat nur einen einzigen Faktor mit $c_0=2$.
Der Faktor $B_j(z)$ muss von der Form $B_j(z)=b_0+b_1z$ sein.
Setze man dies für $B_j$ ein, erhält man
\[
A(\omega)
=
2-\cos\omega
=
(b_0+b_1e^{i\omega})
(b_0+b_1e^{-i\omega})
=
b_0^2 + b_1^2 +b_0b_1\cos\alpha 
\]
Durch Koeffizientengleich liest man ab
\begin{align*}
2&=b_0^2 + b_1^2
\\
-\frac12&=b_0b_1.
\end{align*}
Setzt man die zweite Gleichung in die erste ein, erhält man die biquadratische
Gleichung
\[
2=b_0^2 + \frac{1}{4b_0^2}
\qquad\Leftrightarrow\qquad
b_0^4-2b_0^2 + \frac14 = 0
\]
mit den Lösungen
\[
b_0^2 = 1\pm \sqrt{1-\frac14} = \frac{2\pm\sqrt{3}}{2}
\qquad\Rightarrow\qquad
b_0 = \pm \frac{1\pm\sqrt{3}}2.
\]
Tatsächlich ist
\[
\biggl(
\frac{1\pm\sqrt{3}}2
\biggr)^2
=
\frac{1\pm 2\sqrt{3}+3}{4}
=
\frac{2\pm\sqrt{3}}2.
\]
Die zugehörigen Wert von $b_1$ sind
\[
b_1
=
\mp
\frac{1}{2}\frac{2}{1\pm\sqrt{3}}
=
\mp
\frac{1}{1\pm\sqrt{3}}
=
\mp
\frac{1\mp \sqrt{3}}{1-3}
=
\pm
\frac{1\mp\sqrt{3}}{2}.
\]
Wählt man in allen Fällen das obere Zeichen, wird
\[
B(\omega)
= 
\frac{1+\sqrt{3}}2 + \frac{1-\sqrt{3}}2 e^{i\omega}.
\]
\end{beispiel}


