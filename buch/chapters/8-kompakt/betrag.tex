%
% betrag.tex
%
% (c) 2019 prof Dr Andreas Müller, Hochschule Rapperswil
%
\section{Betrag der erzeugenden Funktion\label{section:betrag}}
\rhead{Betrag der erzeugenden Funktion}
Die Koeffizienten der Skalierungsrelation sollen reell sein, damit
der Transformationsalgorithmus vollständig in $\mathbb R$ gerechnet
werden kann.

\begin{forderung}
\label{forderung:reell}
Die Koeffizienten $h_k$ der Skalierungsrelation sind reell.
\end{forderung}

Um die Funktion $H(\omega)$ zu bestimmen untersuchen wir erst den
Betrag $|H(\omega)|^2$.
Die Forderung~\ref{forderung:reell} hat zur Folge, dass
\[
\overline{H(\omega})
=
\overline{
\frac1{\sqrt{2}}
\sum_{k\in\mathbb Z} h_ke^{-ik\omega}
}
=
\frac1{\sqrt{2}}
\sum_{k\in\mathbb Z} \bar{h}_ke^{ik\omega}
=
\frac1{\sqrt{2}}
\sum_{k\in\mathbb Z} h_ke^{-ik(-\omega)}
=
H(-\omega).
\]
Damit ist
\[
M(\omega)
=
|H(\omega)|^2
=
H(\omega)
\overline{H(\omega)}
=
H(\omega)H(-\omega).
\]
Dieser Ausdruck ändert sich nicht, wenn man $\omega$ durch $-\omega$
ersetzt, die Funktion $M(\omega)$ ist daher eine gerade Funktion.
Da sie ausserdem ein trigonometrisches Polynom ist, muss sie
als ein Polynom in $\cos\omega$ geschrieben werden können.

\begin{konsequenz}
Die Funktion $M(\omega)=|H(\omega)|^2$ ist eine gerade Funktion
und kann also Polynom in $\cos\omega$ geschrieben werden.
\end{konsequenz}

Die Faktorisierung \eqref{buch:kompakt:HB} von $H(\omega)$
zeigt, dass $M(\omega)$ den Faktor
\[
\biggl(\frac{1+e^{-i\omega}}2\biggr)^N
\biggl(\frac{1+e^{i\omega}}2\biggr)^N
=
\biggl( \frac{1+\cos\omega}2\biggr)^N
=
\biggl(
\cos^2\frac{\omega}2
\biggr)^N
\]
enthält.
Oder für $M(\omega)$
\begin{equation}
M(\omega)
= 
\biggl(
\cos^2\frac{\omega}2
\biggr)^N
B(\omega)B(-\omega)
\label{buch:kompakt:MB}
\end{equation}
Um $B(\omega)$ zu finden ist daher zunächst
$A(\omega)=B(\omega)B(-\omega)$
zu bestimmen, in einem zweiten Schritt kann man dann eine
Faktorisierung von $A(\omega)$ mit Hilfe von $B(\omega)$ suchen.

Der Faktor $A(\omega)$ ist wieder in trigonometrisches Polynom,
welches als Funktion von $\omega$ gerade ist.
Es muss also wieder als Polynom in $\cos\omega$
in der Form $A(\omega)=\tilde{P}(\cos\omega)$
geschrieben werden können.
Aus der Halbwinkelformel
\[
\cos^2\frac{\omega}2
=
\frac{1+\cos\omega}2,
\qquad
\text{für den Kosinus folgt}
\qquad
\cos\omega
=
2\cos^2\frac{\omega}2 - 1,
\]
so dass das Polynom $A(\omega)$ auch durch $\cos^2\frac{\omega}2$
ausgedrückt werden kann.

\begin{konsequenz}
Die Funktion $A(\omega)$ ist ein Polynom in $\cos^2\frac{\omega}2$.
\end{konsequenz}

Wir verwenden im Folgenden die Variable 
\[
y=\sin^2\frac{\omega}2
\qquad\Rightarrow\qquad
1-y=\cos^2\frac{\omega}2.
\]
Damit wird auch 
\[
\cos\omega
=
2\cos^2\frac{\omega}2
-1
=
2(1-y)-1
=
1-2y.
\]
Die Funktion $A(\omega)$ als Polynom in $\cos\omega$ wird
dann zu $\tilde{P}(1-2y) = P(y)$.

Um $A(\omega)$ zu bestimmen ist also ein Polynom $P(y)$ zu finden.
Für die Funktion $M(\omega)$ folgt jetzt
\[
M(\omega)
=
\biggl(\cos\frac{\omega}2\biggr)^N
A(\omega)
=
(1-y)^N P(y).
\]

\begin{konsequenz}
Die Funktion $M(\omega)$ ist eine Polynom der Form
$(1-y)^NP(y)$ mit $y=\sin^2\omega/2$.
\end{konsequenz}

Die Funktion $M(\omega)$ genügt auch der Relation
\begin{equation}
1
=
|H(\omega)|^2 + |H(\omega+\pi)|^2
=
M(\omega) + M(\omega+\pi),
\label{kompakt:orthogonalitaetsbedingung}
\end{equation}
die wir bis jetzt noch nicht genutzt haben.
In der Variablen $y$ ausgedrückt ist
\[
\cos^2\frac{\omega+\pi}2
=
\sin^2\frac{\omega}2
=
y.
\]
Die Addition von $\pi$ im Argument $\omega$ entspricht also dem
Übergang von $y$ zu $1-y$.
Wenden wir dies auf die Orthogonalitätsbedingung
\eqref{kompakt:orthogonalitaetsbedingung} an, entsteht die Relation
\begin{equation}
1
=
M(\omega) + M(\omega+\pi)
=
(1-y)^N P(y) + y^N P(1-y).
\end{equation}
Um $A(\omega)$ zu finden ist also ein Polynom
$P(y)$ zu finden, welches die Relation
\eqref{buch:kompakt:Prel} erfüllt.

\begin{konsequenz}
Die Funktion $A(\omega)$ kann aus einem Polynom $P(y)$ konstruiert werden,
welches die Funktionalgleichung
\begin{equation}
(1-y)^N P(y) + y^N P(1-y) = 1
\label{buch:kompakt:Prel}
\end{equation}
erfüllt.
\end{konsequenz}

