%
% algebra.tex -- 
%
% (c) 2019 Prof Dr Andreas Müller, Hochschule Rapperswil
%
\section{Algebraische Konstruktionen}
\rhead{Algebraische Konstruktionen}

%
% Satz von Riesz
%
\subsection{Satz von Riesz}

%
% Partialbruchzerlegung
%
\subsection{Partialbruchzerlegung}
Es muss ein Polynom $P(y)$ gefunden werden mit der Eigenschaft, dass
\[
y^N P(1-y) + (1-y)^N P(y) = 1
\]
Division durch $y^N(1-y)^N$ macht daraus
\[
Q(y)
=
\frac{1}{y^N(1-y)^N}
=
\frac{P(1-y)}{(1-y)^N}
+
\frac{P(y)}{y^N}
\]
Die beiden Terme auf der rechten Seite sind rationale Funktionen, die im
Nenner ausschliesslich Potenzen von $1-y$ im ersten und $y$ im zweiten
Term haben.
Dies sind genau die Nenner, die man in der Partialbruchzerlegung der linken
Seite findet.

Wir berechnen daher die Partialbruchzerlegung
\begin{equation}
\frac{1}{y^N(1-y)^N}
=
\sum_{k=1}^N\frac{C_k}{y^k}
+
\sum_{k=1}^N\frac{C'_k}{(1-y)^k}
\label{buch:kompakt:pbruch}
\end{equation}
von $Q(y)$.
Der Ausdruck $Q(y)$ ist symmetrisch bezüglich bezüglich der Abbildung
$y\leftrightarrow 1-y$.
Da die Koeffizienten der Partialbruchzerlegung eindeutig bestimmt sind,
müssen die Koeffizienten der beiden Summen in \eqref{buch:kompakt:pbruch}
übereinstimmen: $C'_k=C_k$ für $k=1,\dots,N$.

Multipliziert man \eqref{buch:kompakt:pbruch} wieder mit $y^N(1-y)^N$,
erhält man
\begin{align*}
1
&=
(1-y)^N
\underbrace{\sum_{k=1}^\infty C_k y^{N-k}}_{\displaystyle =P(y)}
\mathstrut
+
y^N
\sum_{k=1}^\infty C_k (1-y)^{N-k}
\\
&=
(1-y)^N
P_N(y)
+
y^N
P_N(1-y).
\end{align*}
Das Polynom $P_N(y)$, gebildet mit den Koeffizienten der Partialbruchzerlegung
von $Q(y)$ löst also das eingangs gestellte Problem.

\subsubsection{Die Fälle $N=2$ und $N=3$}
Wir berechnen die Partialbruchzerlegung für kleine Werte von $N$.
Für $N=2$ ist
\begin{align*}
\frac{1}{y^2(1-y)^2}
&=
\frac{C_1}{y}
+
\frac{C_2}{y^2}
+
\frac{C_1}{(1-y)}
+
\frac{C_2}{(1-y)^2}
\\
&=
\frac{C_1y+C_2}{y^2}
+
\frac{C_1(1-y)+C_2}{(1-y)^2}
\\
&=
\frac{1}{y^2(1-y)^2}
\bigl(
(C_1y+C_2)(1-2y+y^2)
+
(C_1+C_2-C_1y)y^2
\bigr)
\intertext{oder nach Multiplikation mit $y^N(1-y)^N$}
\\
1
&=
C_2
+
(C_1-2C_2) y
+
(C_2-2C_1) y^2
+
C_1 y^3
+
(C_1+C_2)y^2
-C_1y^3
\\
&=
C_2 + (C_1-2C_2) y + (2C_2-C_1)y^2.
\end{align*}
Durch Koeffizientenvergleich findet man die Bedingungen
\[
\begin{aligned}
C_2&=1,
&
C_1-2C_2&=0
&&
\text{und}
&
2C_2-C_1&=0.
\end{aligned}
\]
Die dritte Bedingung ist identisch mit der zweiten.
Aus der ersten und der zweiten Bedingung folgt $C_1=2$.
Es folgt
\[
P_2(y) = C_1y + C_2
=
2y+1.
\]

Die analoge Rechnung für $N=3$ liefert die Bedingungen
\[
\begin{aligned}
C_3&=1
\\
C_2-3C_3&=0     &&\Rightarrow&C_2&=3C_3=3
\\
3C_3-3C_2+C_1&=0&&\Rightarrow&C_1&=3C_2-3C_3=6
\\
4C_2-2C_1&=0    &&\Rightarrow&C_1&=2C_2=6
\\
C_1-2C_2&=0     &&\Rightarrow&C_1&=2C_2=6
\end{aligned}
\]
Man liest daraus das Polynom
\[
P_3(y) = 1+3y+6y^2
\]
ab.
Auf die gleiche Weise findet man auch
\begin{align*}
P_4(y) &=
1 + 4y + 10y^2 + 20y^3
\end{align*}
Daraus kann man die Vermutung ablesen, dass
\[
P_N(y)
=
\sum_{k=0}^N
\binom{N-k}{k}
y^k
\]
sein könnte.

\subsubsection{Der allgemeine Fall}




