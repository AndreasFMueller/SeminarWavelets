%
% algebra.tex -- 
%
% (c) 2019 Prof Dr Andreas Müller, Hochschule Rapperswil
%
\section{Algebraische Konstruktionen}
\rhead{Algebraische Konstruktionen}

%
% partial.tex -- 
%
% (c) 2019 Prof Dr Andreas Müller, Hochschule Rapperswil
%

\section{Lösung der Funktionalgleichung für $P(y)$
\label{section:partialbruch}}
%\rhead{Lösung der Funktionalgleichung für $P(y)$}
Es muss ein Polynom $P(y)$ gefunden werden mit der Eigenschaft, dass
\begin{equation}
y^N P(1-y) + (1-y)^N P(y) = 1.
\label{buch:kompakt:bedingung}
\end{equation}
Wir zeigen zunächst, wie sich die Funktionalgleichung
\eqref{buch:kompakt:bedingung} mit Hilfe der Partialbruchzerlegung lösen
lässt mit einem Polynom $P_N(y)$ vom Grad $N-1$.
In Abschnitt~\ref{subsection:weitere loesungen} werden wir dann
zeigen, dass weiter Lösungen $P(y)$ möglich sind, aber höheren Grad haben.

% moved here
\rhead{Lösung der Funktionalgleichung für $P(y)$}

\subsection{Partialbruchzerlegung und die Lösung $P_N(y)$}
\index{Partialbruchzerlegung}%
Division der Funktionalgleichung \eqref{buch:kompakt:bedingung} für $P(y)$
durch $y^N(1-y)^N$ macht daraus
\[
\frac{P(1-y)}{(1-y)^N}
+
\frac{P(y)}{y^N}
=
\frac{1}{y^N(1-y)^N}
=:
Q(y).
\]
Die beiden Terme auf der linken Seite sind rationale Funktionen, die im
Nenner ausschliesslich Potenzen von $1-y$ im ersten und $y$ im zweiten
Term haben.
Dies sind genau die Nenner, die man in der Partialbruchzerlegung der rechten
Seite findet.

Wir berechnen daher die Partialbruchzerlegung von $Q(y)$ mit dem Ansatz
\begin{equation}
\frac{1}{y^N(1-y)^N}
=
\sum_{k=1}^N\frac{C_k}{y^k}
+
\sum_{k=1}^N\frac{C'_k}{(1-y)^k}.
\label{buch:kompakt:pbruch}
\end{equation}
Der Ausdruck $Q(y)$ ist symmetrisch bezüglich der Vertauschung
$y\leftrightarrow 1-y$.
Da die Koeffizienten der Partialbruchzerlegung eindeutig bestimmt sind,
müssen die Koeffizienten der beiden Summen in \eqref{buch:kompakt:pbruch}
übereinstimmen: $C'_k=C_k$ für $k=1,\dots,N$.

Multipliziert man \eqref{buch:kompakt:pbruch} wieder mit $y^N(1-y)^N$,
erhält man
\begin{align*}
1
&=
(1-y)^N
\underbrace{\sum_{k=1}^\infty C_k y^{N-k}}_{\displaystyle =P(y)}
\mathstrut
+
y^N
\sum_{k=1}^\infty C_k (1-y)^{N-k}
\\
&=
(1-y)^N
P_N(y)
+
y^N
P_N(1-y).
\end{align*}
Das Polynom $P_N(y)$, gebildet mit den Koeffizienten der Partialbruchzerlegung
von $Q(y)$ löst also das eingangs gestellte Problem.

\subsection{Der Fall $N=2$
\label{subsection:falln=2}}
Wir berechnen die Partialbruchzerlegung für kleine Werte von $N$.
Für $N=2$ ist
\begin{align*}
\frac{1}{y^2(1-y)^2}
&=
\frac{C_1}{y}
+
\frac{C_2}{y^2}
+
\frac{C_1}{(1-y)}
+
\frac{C_2}{(1-y)^2}
\\
&=
\frac{C_1y+C_2}{y^2}
+
\frac{C_1(1-y)+C_2}{(1-y)^2}
\\
&=
\frac{1}{y^2(1-y)^2}
\bigl(
(C_1y+C_2)(1-2y+y^2)
+
(C_1+C_2-C_1y)y^2
\bigr)
\intertext{oder nach Multiplikation mit $y^N(1-y)^N$}
1
&=
C_2
+
(C_1-2C_2) y
+
(C_2-2C_1) y^2
+
C_1 y^3
+
(C_1+C_2)y^2
-C_1y^3
\\
&=
C_2 + (C_1-2C_2) y + (2C_2-C_1)y^2.
\end{align*}
Durch Koeffizientenvergleich findet man die Bedingungen
\[
\begin{aligned}
C_2&=1,
&
C_1-2C_2&=0
&&
\text{und}
&
2C_2-C_1&=0.
\end{aligned}
\]
Die dritte Bedingung ist identisch mit der zweiten.
Aus der ersten und der zweiten Bedingung folgt $C_1=2$.
Es folgt
\[
P_2(y) = C_1y + C_2
=
2y+1.
\]

\subsection{Der Fall $N=3$
\label{subsection:falln=3}}
Die analoge Rechnung für $N=3$ liefert die Bedingungen
\[
\begin{aligned}
C_3&=1
\\
C_2-3C_3&=0     &&\Rightarrow&C_2&=3C_3=3
\\
3C_3-3C_2+C_1&=0&&\Rightarrow&C_1&=3C_2-3C_3=6
\\
4C_2-2C_1&=0    &&\Rightarrow&C_1&=2C_2=6
\\
C_1-2C_2&=0     &&\Rightarrow&C_1&=2C_2=6
\end{aligned}
\]
Man liest daraus das Polynom
\[
P_3(y) = 1+3y+6y^2
\]
ab.
Auf die gleiche Weise findet man auch
\begin{align*}
P_4(y) &=
1 + 4y + 10y^2 + 20y^3.
\end{align*}

\subsection{Der allgemeine Fall}
Aus den Spezialfällen $N=2$ in Abschnitt~\ref{subsection:falln=2}
und $N=3$ in Abschnitt~\ref{subsection:falln=3}
kann man die Vermutung ablesen, dass
\begin{equation}
P_N(y)
=
\sum_{k=0}^{N-1}
\binom{N+k-1}{k}
y^k
\label{buch:kompakt:vermutung}
\end{equation}
sein könnte.

In diesem Abschnitt wollen wir zeigen, dass $P_N(y)$ die einzige Lösung vom
Grad $N-1$ der Gleichung
$
(1-y)^NP(y)+y^NP(1-y)=1
$
ist.
Die Bedingung ist gleichbedeutend mit
\begin{equation}
P(y) = (1-y)^{-N} (1-y^NP(y)).
\label{buch:kompakt:produkt}
\end{equation}

Um mehr über $P_N(y)$ herauszufinden, können wir beide Seiten von
\eqref{buch:kompakt:produkt} in Taylor-Reihen um den Punkt $y=0$
entwickeln.
Da wir nur ein Polynom vom Grad $N-1$ suchen, können wir die Taylor-Reihen
nach $y^{N-1}$ abbrechen.
Der Ausdruck $y^NP(y)$ auf der rechten Seite von \eqref{buch:kompakt:produkt}
enthält nur Terme vom Grad mindestens $N$ in $y$,
für die Terme vom Grad $<N$ spielt er daher keine Rolle, wir lassen ihn weg.
Eine Lösung vom Grad $N-1$ erhält man daher, indem man die Reihe
für $(1-y)^{-N}$ nach dem Term vom Grade $N-1$ abbricht.

Zur Berechnung der rechten Seite von \eqref{buch:kompakt:produkt}
kann man die Newtonsche Potenzreihe
\index{Newtonsche Reihe}%
\index{Reihe, newtonsche}%
\begin{equation*}
(1-y)^{\alpha} = \sum_{k=0}^{\infty} \binom{\alpha}{k} (-y)^k
\end{equation*}
verwenden.
Im vorliegenden Fall ist $\alpha=-N$ und damit sind die Koeffizienten
\begin{align*}
\binom{-N}{k}
&=
\frac{(-N)\cdot(-N-1)\cdot\dots\cdot (-N-k+1)}{k\cdot (k-1)\cdot\dots\cdot 1}
=
(-1)^k \frac{N\cdot(N+1)\cdot\dots\cdots (N+k-1)}{k\cdot(k-1)\cdot\dots\cdot 1}
\\
&=
(-1)^k \binom{N+k-1}{k}.
\end{align*}
Daraus können wir das Polynom $P_N(y)$ ablesen.
Wegen $(-y)^k(-1)^k=y^k$ folgt das folgende Resultat,
welches auch die Vermutung \eqref{buch:kompakt:vermutung} bestätigt.

\begin{konsequenz}
\label{buch:kompakt:lemma-partial}
Das Polynom
\begin{equation}
P_N(y) = \sum_{k=0}^{N-1} \binom{N+k-1}{k}y^k,
\end{equation}
ist das einzige Polynom vom Grad $N$, welches die Gleichung
\[
(1-y)^NP_N(y) + y^NP_N(1-y)=1
\]
erfüllt.
\end{konsequenz}

\subsection{Weitere Lösungen von höherem Grad
\label{subsection:weitere loesungen}}
Das Polynom $P_N(y)$ ist nicht die einzige Lösung der Gleichung
\eqref{buch:kompakt:bedingung}.
Weitere Lösungen $P(y)$ haben aber notwendigerweise Grad mindestens $N$.
Die Differenz $D(y) = P(y)-P_N(y)$ erfüllt die Bedingung
\[
(1-y)^N D(y) + y^N D(1-y) = 0
\qquad\Leftrightarrow\qquad
(1-y)^N D(y) = -y^N D(y-1).
\]
Die rechte Seite hat eine $N$-fache Nullstelle bei $y=0$, also muss
$D(y)$ ebenfalls eine $N$-fache Nullstelle haben. 
Der Term niedrigsten Grades in $D(y)$ hat daher mindestens den Grad $N$.
Wir schreiben daher $D(y) = y^NR(y)$.

Damit die Bedingung \eqref{buch:kompakt:bedingung} erfüllt ist, muss
für $R(y)$ die Bedingung
\[
(1-y)^N
D(y)
=
(1-y)^N
y^N
R(y)
=
-y^N
(1-y)^N
R(1-y)
\qquad\Rightarrow\qquad
R(y) = -R(1-y)
\]
erfüllt sein.
Es genügt daher, dass $R(y)$ ein Polynom ist, welches bezüglich $y=\frac12$
antisymmetrisch ist.

\begin{konsequenz}
Eine beliebige Lösung $P(y)$ der Gleichung 
\[
(1-y)^N P(y) + y^N P(1-y)=1
\]
hat die Form
\[
P(y) = P_N(y) + y^N R(y),
\]
wobei $R(y)$ ein bezüglich $y=\frac12$ antisymmetrisches Polynom ist,
also $R(1-y)=-R(y)$.
$P_N(y)$ ist das in Konsequenz~\label{buch:kompakt:lemma-partial}
definierte Polynom.
\end{konsequenz}




%
% riesz.tex -- 
%
% (c) 2019 Prof Dr Andreas Müller, Hochschule Rapperswil
%

%
% Lemma von Riesz
%
\section{Das Lemma von Riesz\label{section:riesz}}
\rhead{Das Lemma von Riesz}
Mit Hilfe der Partialbruchzerlegung war es im
Abschnitt~\ref{section:partialbruch}
möglich, die Funktion $M(\omega)$ als Polynom in $\cos\omega$ zu finden,
welches die Identität
\[
M(\omega) + M(\omega + \pi) = 1
\]
erfüllt.
Gesucht wird aber eine Funktion $H(\omega)$ mit
$M(\omega)=H(\omega)H(-\omega)$.
Diese Faktorisierung wird ermöglicht dank des in diesem Abschnitt
formulierten und beschriebenen Lemmas von Riesz.
Das Lemma von Riesz ist die folgende etwas überraschende Aussage:

\begin{lemma}[Riesz]
Ist
\[
A(\omega)
=
\sum_{k=0}^n
a_k \cos^k \omega,
\qquad
a_n\ne 0
\]
mit $A(0)=1$ und $A(\omega)\ge 0$ für alle $\omega$.
Dann gibt es eine Funktion
\[
B(\omega)
=
\sum_{k=0}^n b_ke^{-ik\omega}
\]
mit $B(0)=1$ und $A(\omega)=B(\omega)B(-\omega)$.
\end{lemma}

Im Laufe des nachfolgenden Beweises werden wir wiederholt die folgende
Identität verwenden:
\begin{align}
\frac{z+z^{-1}}2-\frac{s+s^{-1}}2
&=
\frac1{2s}(sz+sz^{-1}-s^2-1)
=
-
\frac1{2s}(1-sz-sz^{-1}+s^2)
\notag
\\
&=
-
\frac1{2s}(z-s)(z^{-1}-s)
\label{buch:kompakt:cosh}
\end{align}

\begin{proof}[Beweis]
Die Funktion $A(\omega)$ kann geschrieben werden als
$A(\omega) = p(\cos\omega)$, wobei $p(x)$ ein Polynom vom
Grad $n$ ist.
Das Polynom $p(x)$ hat $n$ möglicherweise komplexe Nullstellen
$c_1,\dots,c_n$, wobei komplexe Nullstellen in konjugiert komplexen
Paaren auftreten.
Das Polynom $p(x)$ kann daher faktorisiert werden in das Produkt
\begin{align*}
p(x)
&=
a_n(x-c_1)(x-c_2)\dots(x-c_n)
=
a_n \prod_{j=1}^n (x-c_j)
\\
A(\omega) = p(\cos\omega)
&=
a_n \prod_{j=1}^n (\cos \omega - c_j)
\end{align*}
Mit der Abkürzung $z=e^{-i\omega}$ können die Faktoren im Produkt als
\[
\cos\omega -c_j = \frac{z+z^{-1}}2-c_j
\]
geschrieben werden.

Wir versuchen, das gesuchte Polynom $B(\omega)$ aus Lösungen des Problems
für die einzelnen Faktoren von $p(x)$ aufzubauen.
Ein einzelner Faktor $A_j(\omega=)=\cos\omega - c_j$ erfüllt die Bedingungen an
$A$ im Lemma im Allgemeinen nicht.
Ist zum Beispiel $c_j$ reell zwischen $-1$ und $1$, dann hat
$\cos\omega-c_j$ eine Nullstelle hat daher auch negative Werte.
Wenn $c_j$ komplex ist, dann ist $\cos\omega-c_j$ keine rellwertige
Funktion.
Wir unterscheiden daher die folgenden drei Fälle:
\begin{enumerate}
\item
Fall $|c_j|\ge 1$ und $c_j\in\mathbb R$:
In diesem Fall kann $c_j$ geschrieben werden als
\[
c_j = \frac{s+s^{-1}}2
\quad
\text{mit $s=\operatorname{sign}(c_j)\cdot \operatorname{arsinh}|c_j|$.}
\]
Dann wird
\[
A_j(\omega)
=
\cos\omega - c_j
=
\frac{z+z^{-1}}2 - \frac{s+s^{-1}}2
=
-\frac1{2s} (z-s)(z^{-1}-s)
\]
mit Hilfe von \eqref{buch:kompakt:cosh}.
Falls $c_j < 0$ ist auch $s<0$ und wir können dem Produkt $B(\omega)$
den Faktor $B_j(\omega)=(z-s)/\sqrt{2s}$ hinzufügen, für den gilt
$B_j(\omega)B_j(-\omega)=A_j(\omega)$.
Falls $c_j > 0$ ist auch $s>0$ und wir können dem Produkt $B(\omega)$
den Faktor $B_j(\omega) =(z-s)/\sqrt{-2s}$ hinzufügen, für den
gilt $B_j(\omega)B_j(-\omega)=-A_j(\omega)$.
Das Vorzeichen im zweiten Fall stört nicht, weil wegen $A(\omega)\ge 0$
noch ein weiterer Faktor $A_{j'}(\omega)$ vorhanden sein muss, der ebenfalls
$A_{j'}(\omega)\le 0$ ist für alle $\omega$ und der daher ebenfalls
mit dem ``falschen'' Vorzeichen in das Produkt eingeht, wodurch das
Vorzeichen wieder korrigiert wird.
\item
Fall $|c_j|\ge 1$ und $c_j\in\mathbb R$:
In diesem Fall hat $A_j(\omega)$ und damit $A(\omega)$ eine Nullstelle
für jedes $\omega$ mit $\cos\omega=c_j$.
Wegen $A(\omega)\ge 0$ müssen alle solchen Nullstellen in gerader
Anzahl auftreten.
Sei $\alpha$ so, dass $\cos\alpha = c_j$ und damit insbesondere auch
\[
c_j
=
\cos\alpha
=
\frac{s+s^{-1}}2
\quad\text{mit $s=e^{i\alpha}$}.
\]
Wegen der Symmetrie der Funktion $\cos\omega$ gilt auch
\[
c_{j'}=
\cos(-\alpha)
=
\frac{e^{-i\alpha}+e^{i\alpha}}2.
\]
Damit kann $A_j(\omega)^2$ geschrieben werden als
\begin{align*}
A_j(\omega)^2
&=
\biggl(
\frac{z+z^{-1}}2-\frac{e^{i\alpha}+e^{-i\alpha}}2
\biggr)
\biggl(
\frac{z+z^{-1}}2-\frac{e^{-i\alpha}+e^{i\alpha}}2
\biggr)
\\
&=
\frac{1}{2e^{i\alpha}}
(z-e^{i\alpha})
(z^{-1}-e^{i\alpha})
\frac{1}{2e^{-i\alpha}}
(z-e^{-i\alpha})
(z^{-1}-e^{-i\alpha})
\\
&=
\frac{1}{2}
(z-e^{i\alpha})
(z-e^{-i\alpha})
\cdot
\frac{1}{2}
(z^{-1}-e^{i\alpha})
(z^{-1}-e^{-i\alpha})
\\
&=
\frac{1}{2}
(z^2-2z\cos\alpha +1)
\cdot
\frac{1}{2}
(z^{-2}-2z^{-1}\cos\alpha +1)
\end{align*}
Wir fügen daher der Funktion $B(\omega)$ den Faktor
$B_j(\omega)=\frac12(z^2-2z\cos\alpha +1)$ hinzu, für
den 
$B_j(\omega)B_j(-\omega)=A_j(\omega)^2$ gilt.
\item
Fall $c_j\in\mathbb C\setminus\mathbb R$:
In diesem Fall gibt es einen zweiten, komplex konjugierten Faktor mit
$c_{j'}=\bar{c}_j$.
Wir möchten $c_j$ wieder als
\begin{equation}
c_j
=
\frac{s+s^{-1}}2
\label{buch:kompakt:cj}
\end{equation}
darstellen.
Dazu multiplizieren wir \eqref{buch:kompakt:cj} mit $s$ und erhalten die
quadratische Gleichung
\[
s^2-2c_j s+1=0
\]
mit der Lösung
\[
s=c_j\pm\sqrt{c_j^2-1},
\]
die immer mindestens eine Lösung hat.
Wir dürfen daher annehmen, dass
\[
c_j = \frac{s+s^{-1}}2
\qquad\text{und}\qquad
c_{j'} = \bar{c}_j = \frac{\bar{s}+\bar{s}^{-1}}2.
\]
Damit können wir jetzt das Produkt $A_j(\omega)A_{j'}(\omega)$ 
wie folgt faktorisieren:
\begin{align*}
A_j(\omega)A_{j'}(\omega)
&=
\biggl(
\frac{z+z^{-1}}2 - \frac{s+s^{-1}}2
\biggr)
\biggl(
\frac{z+z^{-1}}2 - \frac{\bar{s}+\bar{s}^{-1}}2
\biggr)
\\
&=
\frac{1}{2s}
(z-s)(z^{-1}-s)
\cdot
\frac{1}{2\bar{s}}
(z-\bar{s})(z^{-1}-\bar{s})
\\
&=
\frac{1}{4|s|^2}(z-s)(z^{-1}-s)(z-\bar{s})(z^{-1}-\bar{s})
\\
&=
\frac1{2|s|}
(z-s)
(z-\bar{s})
\cdot
\frac1{2|s|}
(z^{-1}-s)
(z^{-1}-\bar{s})
\\
&=
\frac1{2|s|}(z^2-2z\operatorname{Re}s -|s|^2)
\cdot
\frac1{2|s|}(z^{-2}-2z^{-1}\operatorname{Re}s -|s|^2).
\end{align*}
Wir fügen in diesem Fall den Faktor $B_j(\omega)=(z^2-2z\operatorname{Re}s-|s|^2)/2|s|$ hinzu.
\end{enumerate}
In allen drei Fällen haben wir für die behandelten Faktoren $A_j(\omega)$
und der eventuell zugehörigen Faktoren $A_{j'}(\omega)$ eine Funktion
$B_j(\omega)$ gefunden, so dass das Produkt all dieser Faktoren
\[
B(\omega) = \prod_j B_j(\omega)
\]
genau die im Lemma versprochenen Eigenschaften hat.
\end{proof}



%%
%% Satz von Riesz
%%
%\subsection{Das Lemma von Riesz}
%
%\begin{lemma}[Riesz]
%Ist
%\[
%A(\omega)
%=
%\sum_{k=0}^n
%a_k \cos^k \omega,
%\qquad
%a_n\ne 0
%\]
%mit $A(0)=1$ und $A(\omega)\ge 0$ für alle $\omega$.
%Dann gibt es eine Funktion
%\[
%B(\omega)
%=
%\sum_{k=0}^n b_ke^{-ik\omega}
%\]
%mit $B(0)=1$ und $A(\omega)=B(\omega)B(-\omega)$.
%\end{lemma}
%
%Im Laufe des nachfolgenden Beweises werden wir wiederholt die folgende
%Identität verwenden:
%\begin{align}
%\frac{z+z^{-1}}2-\frac{s+s^{-1}}2
%&=
%\frac1{2s}(sz+sz^{-1}-s^2-1)
%=
%-
%\frac1{2s}(1-sz-sz^{-1}+s^2)
%\notag
%\\
%&=
%-
%\frac1{2s}(z-s)(z^{-1}-s)
%\label{buch:kompakt:cosh}
%\end{align}
%
%\begin{proof}[Beweis]
%Die Funktion $A(\omega)$ kann geschrieben werden als
%$A(\omega) = p(\cos\omega)$, wobei $p(x)$ ein Polynom vom
%Grad $n$ ist.
%Das Polynom $p(x)$ hat $n$ möglicherweise komplexe Nullstellen
%$c_1,\dots,c_n$, wobei komplexe Nullstellen in konjugiert komplexen
%Paaren auftreten.
%Das Polynom $p(x)$ kann daher faktorisiert werden in das Produkt
%\begin{align*}
%p(x)
%&=
%a_n(x-c_1)(x-c_2)\dots(x-c_n)
%=
%a_n \prod_{j=1}^n (x-c_j)
%\\
%A(\omega) = p(\cos\omega)
%&=
%a_n \prod_{j=1}^n (\cos \omega - c_j)
%\end{align*}
%Mit der Abkürzung $z=e^{-i\omega}$ können die Faktoren im Produkt als
%\[
%\cos\omega -c_j = \frac{z+z^{-1}}2-c_j
%\]
%geschrieben werden.
%
%Wir versuchen, das gesuchte Polynom $B(\omega)$ aus Lösungen des Problems
%für die einzelnen Faktoren von $p(x)$ aufzubauen.
%Ein einzelner Faktor $A_j(\omega=)=\cos\omega - c_j$ erfüllt die Bedingungen an
%$A$ im Lemma im Allgemeinen nicht.
%Ist zum Beispiel $c_j$ reell zwischen $-1$ und $1$, dann hat
%$\cos\omega-c_j$ eine Nullstelle hat daher auch negative Werte.
%Wenn $c_j$ komplex ist, dann ist $\cos\omega-c_j$ keine rellwertige
%Funktion.
%Wir unterscheiden daher die folgenden drei Fälle:
%\begin{enumerate}
%\item
%Fall $|c_j|\ge 1$ und $c_j\in\mathbb R$:
%In diesem Fall kann $c_j$ geschrieben werden als
%\[
%c_j = \frac{s+s^{-1}}2
%\quad
%\text{mit $s=\operatorname{sign}(c_j)\cdot \operatorname{arsinh}|c_j|$.}
%\]
%Dann wird
%\[
%A_j(\omega)
%=
%\cos\omega - c_j
%=
%\frac{z+z^{-1}}2 - \frac{s+s^{-1}}2
%=
%-\frac1{2s} (z-s)(z^{-1}-s)
%\]
%mit Hilfe von \eqref{buch:kompakt:cosh}.
%Falls $c_j < 0$ ist auch $s<0$ und wir können dem Produkt $B(\omega)$
%den Faktor $B_j(\omega)=(z-s)/\sqrt{2s}$ hinzufügen, für den gilt
%$B_j(\omega)B_j(-\omega)=A_j(\omega)$.
%Falls $c_j > 0$ ist auch $s>0$ und wir können dem Produkt $B(\omega)$
%den Faktor $B_j(\omega) =(z-s)/\sqrt{-2s}$ hinzufügen, für den
%gilt $B_j(\omega)B_j(-\omega)=-A_j(\omega)$.
%Das Vorzeichen im zweiten Fall stört nicht, weil wegen $A(\omega)\ge 0$
%noch ein weiterer Faktor $A_{j'}(\omega)$ vorhanden sein muss, der ebenfalls
%$A_{j'}(\omega)\le 0$ ist für alle $\omega$ und der daher ebenfalls
%mit dem ``falschen'' Vorzeichen in das Produkt eingeht, wodurch das
%Vorzeichen wieder korrigiert wird.
%\item
%Fall $|c_j|\ge 1$ und $c_j\in\mathbb R$:
%In diesem Fall hat $A_j(\omega)$ und damit $A(\omega)$ eine Nullstelle
%für jedes $\omega$ mit $\cos\omega=c_j$.
%Wegen $A(\omega)\ge 0$ müssen alle solchen Nullstellen in gerader
%Anzahl auftreten.
%Sei $\alpha$ so, dass $\cos\alpha = c_j$ und damit insbesondere auch
%\[
%c_j
%=
%\cos\alpha
%=
%\frac{s+s^{-1}}2
%\quad\text{mit $s=e^{i\alpha}$}.
%\]
%Wegen der Symmetrie der Funktion $\cos\omega$ gilt auch
%\[
%c_{j'}=
%\cos(-\alpha)
%=
%\frac{e^{-i\alpha}+e^{i\alpha}}2.
%\]
%Damit kann $A_j(\omega)^2$ geschrieben werden als
%\begin{align*}
%A_j(\omega)^2
%&=
%\biggl(
%\frac{z+z^{-1}}2-\frac{e^{i\alpha}+e^{-i\alpha}}2
%\biggr)
%\biggl(
%\frac{z+z^{-1}}2-\frac{e^{-i\alpha}+e^{i\alpha}}2
%\biggr)
%\\
%&=
%\frac{1}{2e^{i\alpha}}
%(z-e^{i\alpha})
%(z^{-1}-e^{i\alpha})
%\frac{1}{2e^{-i\alpha}}
%(z-e^{-i\alpha})
%(z^{-1}-e^{-i\alpha})
%\\
%&=
%\frac{1}{2}
%(z-e^{i\alpha})
%(z-e^{-i\alpha})
%\cdot
%\frac{1}{2}
%(z^{-1}-e^{i\alpha})
%(z^{-1}-e^{-i\alpha})
%\\
%&=
%\frac{1}{2}
%(z^2-2z\cos\alpha +1)
%\cdot
%\frac{1}{2}
%(z^{-2}-2z^{-1}\cos\alpha +1)
%\end{align*}
%Wir fügen daher der Funktion $B(\omega)$ den Faktor
%$B_j(\omega)=\frac12(z^2-2z\cos\alpha +1)$ hinzu, für
%den 
%$B_j(\omega)B_j(-\omega)=A_j(\omega)^2$ gilt.
%\item
%Fall $c_j\in\mathbb C\setminus\mathbb R$:
%In diesem Fall gibt es einen zweiten, komplex konjugierten Faktor mit
%$c_{j'}=\bar{c}_j$.
%Wir möchten $c_j$ wieder als
%\begin{equation}
%c_j
%=
%\frac{s+s^{-1}}2
%\label{buch:kompakt:cj}
%\end{equation}
%darstellen.
%Dazu multiplizieren wir \eqref{buch:kompakt:cj} mit $s$ und erhalten die
%quadratische Gleichung
%\[
%s^2-2c_j s+1=0
%\]
%mit der Lösung
%\[
%s=c_j\pm\sqrt{c_j^2-1},
%\]
%die immer mindestens eine Lösung hat.
%Wir dürfen daher annehmen, dass
%\[
%c_j = \frac{s+s^{-1}}2
%\qquad\text{und}\qquad
%c_{j'} = \bar{c}_j = \frac{\bar{s}+\bar{s}^{-1}}2.
%\]
%Damit können wir jetzt das Produkt $A_j(\omega)A_{j'}(\omega)$ 
%wie folgt faktorisieren:
%\begin{align*}
%A_j(\omega)A_{j'}(\omega)
%&=
%\biggl(
%\frac{z+z^{-1}}2 - \frac{s+s^{-1}}2
%\biggr)
%\biggl(
%\frac{z+z^{-1}}2 - \frac{\bar{s}+\bar{s}^{-1}}2
%\biggr)
%\\
%&=
%\frac{1}{2s}
%(z-s)(z^{-1}-s)
%\cdot
%\frac{1}{2\bar{s}}
%(z-\bar{s})(z^{-1}-\bar{s})
%\\
%&=
%\frac{1}{4|s|^2}(z-s)(z^{-1}-s)(z-\bar{s})(z^{-1}-\bar{s})
%\\
%&=
%\frac1{2|s|}
%(z-s)
%(z-\bar{s})
%\cdot
%\frac1{2|s|}
%(z^{-1}-s)
%(z^{-1}-\bar{s})
%\\
%&=
%\frac1{2|s|}(z^2-2z\operatorname{Re}s -|s|^2)
%\cdot
%\frac1{2|s|}(z^{-2}-2z^{-1}\operatorname{Re}s -|s|^2).
%\end{align*}
%Wir fügen in diesem Fall den Faktor $B_j(\omega)=(z^2-2z\operatorname{Re}s-|s|^2)/2|s|$ hinzu.
%\end{enumerate}
%In allen drei Fällen haben wir für die behandelten Faktoren $A_j(\omega)$
%und der eventuell zugehörigen Faktoren $A_{j'}(\omega)$ eine Funktion
%$B_j(\omega)$ gefunden, so dass das Projekt all dieser Faktoren
%\[
%B(\omega) = \prod_j B_j(\omega)
%\]
%genau die im Lemma versprochenen Eigenschaften hat.
%\end{proof}
%
%%
%% Partialbruchzerlegung
%%
%\subsection{Partialbruchzerlegung}
%Es muss ein Polynom $P(y)$ gefunden werden mit der Eigenschaft, dass
%\begin{equation}
%y^N P(1-y) + (1-y)^N P(y) = 1
%\label{buch:kompakt:bedingung}
%\end{equation}
%Division durch $y^N(1-y)^N$ macht daraus
%\[
%Q(y)
%=
%\frac{1}{y^N(1-y)^N}
%=
%\frac{P(1-y)}{(1-y)^N}
%+
%\frac{P(y)}{y^N}
%\]
%Die beiden Terme auf der rechten Seite sind rationale Funktionen, die im
%Nenner ausschliesslich Potenzen von $1-y$ im ersten und $y$ im zweiten
%Term haben.
%Dies sind genau die Nenner, die man in der Partialbruchzerlegung der linken
%Seite findet.
%
%Wir berechnen daher die Partialbruchzerlegung
%\begin{equation}
%\frac{1}{y^N(1-y)^N}
%=
%\sum_{k=1}^N\frac{C_k}{y^k}
%+
%\sum_{k=1}^N\frac{C'_k}{(1-y)^k}
%\label{buch:kompakt:pbruch}
%\end{equation}
%von $Q(y)$.
%Der Ausdruck $Q(y)$ ist symmetrisch bezüglich bezüglich der Abbildung
%$y\leftrightarrow 1-y$.
%Da die Koeffizienten der Partialbruchzerlegung eindeutig bestimmt sind,
%müssen die Koeffizienten der beiden Summen in \eqref{buch:kompakt:pbruch}
%übereinstimmen: $C'_k=C_k$ für $k=1,\dots,N$.
%
%Multipliziert man \eqref{buch:kompakt:pbruch} wieder mit $y^N(1-y)^N$,
%erhält man
%\begin{align*}
%1
%&=
%(1-y)^N
%\underbrace{\sum_{k=1}^\infty C_k y^{N-k}}_{\displaystyle =P(y)}
%\mathstrut
%+
%y^N
%\sum_{k=1}^\infty C_k (1-y)^{N-k}
%\\
%&=
%(1-y)^N
%P_N(y)
%+
%y^N
%P_N(1-y).
%\end{align*}
%Das Polynom $P_N(y)$, gebildet mit den Koeffizienten der Partialbruchzerlegung
%von $Q(y)$ löst also das eingangs gestellte Problem.
%
%\subsubsection{Die Fälle $N=2$ und $N=3$}
%Wir berechnen die Partialbruchzerlegung für kleine Werte von $N$.
%Für $N=2$ ist
%\begin{align*}
%\frac{1}{y^2(1-y)^2}
%&=
%\frac{C_1}{y}
%+
%\frac{C_2}{y^2}
%+
%\frac{C_1}{(1-y)}
%+
%\frac{C_2}{(1-y)^2}
%\\
%&=
%\frac{C_1y+C_2}{y^2}
%+
%\frac{C_1(1-y)+C_2}{(1-y)^2}
%\\
%&=
%\frac{1}{y^2(1-y)^2}
%\bigl(
%(C_1y+C_2)(1-2y+y^2)
%+
%(C_1+C_2-C_1y)y^2
%\bigr)
%\intertext{oder nach Multiplikation mit $y^N(1-y)^N$}
%1
%&=
%C_2
%+
%(C_1-2C_2) y
%+
%(C_2-2C_1) y^2
%+
%C_1 y^3
%+
%(C_1+C_2)y^2
%-C_1y^3
%\\
%&=
%C_2 + (C_1-2C_2) y + (2C_2-C_1)y^2.
%\end{align*}
%Durch Koeffizientenvergleich findet man die Bedingungen
%\[
%\begin{aligned}
%C_2&=1,
%&
%C_1-2C_2&=0
%&&
%\text{und}
%&
%2C_2-C_1&=0.
%\end{aligned}
%\]
%Die dritte Bedingung ist identisch mit der zweiten.
%Aus der ersten und der zweiten Bedingung folgt $C_1=2$.
%Es folgt
%\[
%P_2(y) = C_1y + C_2
%=
%2y+1.
%\]
%
%Die analoge Rechnung für $N=3$ liefert die Bedingungen
%\[
%\begin{aligned}
%C_3&=1
%\\
%C_2-3C_3&=0     &&\Rightarrow&C_2&=3C_3=3
%\\
%3C_3-3C_2+C_1&=0&&\Rightarrow&C_1&=3C_2-3C_3=6
%\\
%4C_2-2C_1&=0    &&\Rightarrow&C_1&=2C_2=6
%\\
%C_1-2C_2&=0     &&\Rightarrow&C_1&=2C_2=6
%\end{aligned}
%\]
%Man liest daraus das Polynom
%\[
%P_3(y) = 1+3y+6y^2
%\]
%ab.
%Auf die gleiche Weise findet man auch
%\begin{align*}
%P_4(y) &=
%1 + 4y + 10y^2 + 20y^3
%\end{align*}
%Daraus kann man die Vermutung ablesen, dass
%\begin{equation}
%P_N(y)
%=
%\sum_{k=0}^{N-1}
%\binom{N+k-1}{k}
%y^k
%\label{buch:kompakt:vermutung}
%\end{equation}
%sein könnte.
%
%\subsubsection{Der allgemeine Fall}
%Das Polynom $P_N(y)$ ist die einzige Lösung vom Grad $N-1$ der Gleichung
%$
%(1-y)^NP(y)+y^NP(1-y)=1.
%$
%Die Bedingung ist gleichbedeutend mit
%\begin{equation}
%P(y) = (1-y)^{-N} (1-y^NP(y)).
%\label{buch:kompakt:produkt}
%\end{equation}
%Das Ziel dieses Abschnitts ist, $P_N(y)$ explizit zu bestimmen.
%
%Um mehr über $P_N(y)$ herauszufinden, können wir beide Seiten von
%\eqref{buch:kompakt:produkt} in Taylor-Reihen um den Punkt $y=0$
%entwickeln.
%Da wir nur ein Polynom vom Grad $N-1$ suchen, können wir die Taylor-Reihen
%nach $y^{N-1}$ abbrechen.
%Der Ausdruck $y^Np(y)$ auf der rechten Seite von \eqref{buch:kompakt:produkt}
%enthält nur Terme vom Grad mindestens $N$ in $y$,
%für die Terme vom Grad $<N$ spielt er daher keine Rolle.
%Eine Lösung vom Grad $N-1$ erhält man daher, indem man die Reihe
%für $(1-y)^{-N}$ nach dem Terme vom Grade $N-1$ abbricht.
%
%Auf der rechten Seite von \eqref{buch:kompakt:produkt}
%kann man die Newtonsche Potenzreihe
%\begin{equation*}
%(1-y)^{\alpha} = \sum_{k=0}^{\infty} \binom{\alpha}{k} (-y)^k
%\end{equation*}
%verwenden.
%Im vorliegenden Fall ist $\alpha=-N$ und damit sind die Koeffizienten
%\begin{align*}
%\binom{-N}{k}
%&=
%\frac{(-N)\cdot(-N-1)\cdot\dots\cdot (-N-k+1)}{k\cdot (k-1)\cdot\dots\cdot 1}
%=
%(-1)^k \frac{N\cdot(N+1)\cdot\dots\cdots (N+k-1)}{k\cdot(k-1)\cdot\dots\cdot 1}
%\\
%&=
%(-1)^k \binom{N+k-1}{k}.
%\end{align*}
%Daraus können wir das Polynom $P_N(y)$ ablesen.
%Wegen $(-y)^k(-1)^k=y^k$ folgt
%\begin{equation}
%P_N(y) = \sum_{k=0}^{N-1} \binom{N+k-1}{k}y^k,
%\end{equation}
%was die Vermutung \eqref{buch:kompakt:vermutung} bestätigt.
%
%\subsubsection{Weitere Lösungen von höherem Grad}
%Das Polynom $P_N(y)$ ist nicht die einzige Lösung der Gleichung
%\eqref{buch:kompakt:bedingung}.
%Weitere Lösungen $P(y)$ haben aber notwendigerweise Grad mindestens $N$.
%Die Differenz $D(y) = P(y)-P_N(y)$ erfüllt die Bedingung
%\[
%(1-y)^N D(y) + y^N D(1-y) = 0
%\qquad\Leftrightarrow\qquad
%(1-y)^N D(y) = -y^N D(y-1).
%\]
%Die rechte Seite hat eine $N$-fache Nullstelle bei $y=0$, also muss
%$D(y)$ ebenfalls eine $N$-fache Nullstelle haben. 
%Der term niedrigsten Grades in $D(y)$ hat daher mindestens den Grad $N$.
%Wir schreiben daher $D(y) = y^NR(y)$.
%
%Damit die Bedingung \eqref{buch:kompakt:bedingung} erfüllt ist, muss
%für $R(y)$ die Bedingung
%\[
%(1-y)^N
%D(y)
%=
%(1-y)^N
%y^N
%R(y)
%=
%-y^N
%(1-y)^N
%R(1-y)
%\qquad\Rightarrow\qquad
%R(y) = -R(1-y)
%\]
%erfüllt sein.
%Es genügt daher, dass $R(y)$ ein Polynom ist, welches bezüglich $y=\frac12$
%antisymmetrisch ist.
%
%
%
%
%
