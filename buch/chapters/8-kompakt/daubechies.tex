%
% daubechies.tex
%
% (c) 2019 Prof Dr Andreas Müller, Hochschule Rapperswil
%
\section{Daubechies-Wavelets\label{section:daubechies}}
\rhead{Daubechies-Wavelets}
In den vorangegangenen Abschnitten wurde gezeigt, wie die erzeugende
Funktion $H(s)$ einer Multi\-skalenanalyse für Wavelets beliebiger
Ordnung bestimmt werden kann.
Es wurde auch klar, dass das Haar-Wavelet eine Lösung dieser Aufgabe
ist.
Nachdem die Existenz solcher Multiskalenana\-ly\-sen nun etabliert ist,
sollen in diesem Abschnitt die Koeffizienten $h_k$ dieser Wavelets
explizit konstruiert werden.

In Lemma~\ref{buch:kompakt:lemma-partial} wurden die Lösungen für das
Polynom $P(y)$ bereits charakterisiert.
Um die erzeugende Funktion zu bestimmen, sind noch folgende Schritte
durchzuführen.
\begin{enumerate}
\item
Die Funktion $A(\omega)=P(y)$ mit $y=\sin^2\frac{\omega}2$ muss
bestimmt und als Polynom in $\cos\omega=1-2y$ ausgedrückt werden.
\item
Die Funktion $A(\omega)$ muss mit Hilfe des Lemmas \ref{lemma:riesz}
von Riesz in das Produkt $A(\omega)=B(\omega)B(-\omega)$ zerlegt
werden.
\item
Die erzeugende Funktion ist dann
\[
H(\omega)
=
\biggl(
\frac{1+e^{-i\omega}}2
\biggr)^N
B(\omega),
\]
aus ihr müssen die Koeffizienten $h_k$ abgelesen werden.
\end{enumerate}
Auf diese Weise entsteht für jede Ordnung $N$ ein Wavelet, welches
genau $2N$ Koeffizienten verwendet und Träger im Interval
$[0,N]$ hat.
Es heisst das Daubechies-Wavelet
\index{Daubechies-Wavelet}
nach Ingrid Daubechies, die diese Konstruktion als erste durchgeführt hat.
\index{Ingrid Daubechies}
Das Daubechies-Wavelet der Ordnung $N$ wird mit $\texttt{db}N$
bezeichnet.
Das Haar-Wavelet ist $\texttt{db}1$.

\subsection{Der Fall $N=2$}
Im Fall $N=2$ ergeben sich vier Koeffizienten, die das Daubechies-Wavelet
\texttt{db4} definieren.
\index{db4}%

Es ist $P_N(y)=P_2(y)=1+2y$.
Wegen $y=(1-\cos \omega)/2$ folgt
\[
A(\omega) = 1 + 2\frac{1-\cos\omega}2 = 2-\cos\omega
= 2- \cos\omega
\]
Wegen $\cos\omega\le 1$ ist $A(\omega) \ge 0$, die Voraussetzungen des
Riesz-Lemmas~\ref{lemma:riesz} sind damit erfüllt.

Im Beispiel auf Seite~\pageref{buch:kompakt:db2riesz} wird bereits
die Funktion $B(z)$ gefunden, die nach dem Lemma~\ref{lemma:riesz}
von Riesz existieren muss.
Es war
\[
B(\omega)
=
\frac{1+\sqrt{3}}2 + \frac{1-\sqrt{3}}2e^{-i\omega}.
\]
Daraus kann man jetzt die erzeugende Funktion ableiten
\begin{align*}
H(\omega)
&=
\biggl(\frac{1+e^{-i\omega}}2\biggr)^2 B(\omega)
=
\frac14(1+2e^{-i\omega}+e^{-2i\omega})
\biggl(
\frac{1+\sqrt{3}}2 + \frac{1-\sqrt{3}}2e^{-i\omega}
\biggr).
\\
&=
\frac18(
(1+\sqrt{3})
+
(2(1+\sqrt{3})+1-\sqrt{3}) e^{-i\omega}
+
(1+\sqrt{3}+2(1-\sqrt{3})) e^{-2i\omega}
+
(1-\sqrt{3})e^{-3i\omega}
)
\\
&=
\frac14\biggl(
\frac{1+\sqrt{3}}2
+
\frac{3+\sqrt{3}}2 e^{-i\omega}
+
\frac{3-\sqrt{3}}2 e^{-2i\omega}
+
\frac{1-\sqrt{3}}2 e^{-3i\omega}
\biggr)
\\
&=
\frac1{\sqrt{2}}
\biggl(
\frac{1+\sqrt{3}}{4\sqrt{2}}
+
\frac{3+\sqrt{3}}{4\sqrt{2}} e^{-i\omega}
+
\frac{3-\sqrt{3}}{4\sqrt{2}} e^{-2i\omega}
+
\frac{1-\sqrt{3}}{4\sqrt{2}} e^{-3i\omega}
\biggr)
\end{align*}
Daraus kann man die Koeffizienten der Skalierungsrelation ablesen:
\begin{align*}
h_0
&=
\frac{1+\sqrt{3}}{4\sqrt{2}}
=
\phantom{-}
0.482962913144534
\\
h_1
&=
\frac{3+\sqrt{3}}{4\sqrt{2}}
=
\phantom{-}
0.836516303737808
\\
h_2
&=
\frac{3-\sqrt{3}}{4\sqrt{2}}
=
\phantom{-}
0.224143868042013
\\
h_3
&=
\frac{1-\sqrt{3}}{4\sqrt{2}}
=
-0.129409522551260
\end{align*}
Diese Koeffizienten sind für die Rekonstruktion von
Skalierungsfunktion und Wavelet in Abbildung~\ref{buch:algo:db1}
verwendet worden.

\subsection{Der Fall $N=3$}
Für $N=3$ ist $P_N(y)=P_3(y)=1+3y+6y^2$.
Wir ersetzen wieder $y=(1-\cos\omega)/2$ 
\begin{align*}
A(\omega)
&=
1+3(1-\cos\omega)/2 + 6(1-\cos\omega)^2/4
\\
&=
1+\frac32-\frac32\cos\omega
+
\frac32(1-2\cos\omega+\cos^2\omega)
\\
&=
4
-
\frac92\cos\omega
+
\frac32\cos^2\omega
\\
&=
\frac32(\cos^2\omega -3\cos\omega + \frac{8}3)
\end{align*}
Jetzt muss eine Funktion $B(\omega)$ als Polynom in $e^{i\omega}$
gefunden werden derart, dass $A(\omega)=B(\omega)B(-\omega)$.
Der Ansatz
\[
B(\omega)
=
b_0 + b_1 e^{i\omega} + b_2 e^{2i\omega}
\]
führt auf
\begin{align*}
A(\omega)
=
B(\omega)B(-\omega)
&=
(b_0 + b_1 e^{i\omega} + b_2 e^{2i\omega})
(b_0 + b_1 e^{-i\omega} + b_2 e^{-2i\omega})
\\
&=
b_0^2 + b_1^2 + b_2^2
+
(b_0b_1 + b_2b_1)
(e^{i\omega}+e^{-i\omega})
+
b_0b_2
(e^{2i\omega}+e^{-2i\omega})
\\
&=
b_0^2 + b_1^2 + b_2^2
+
2(b_0b_1 + b_2b_1)
\frac{e^{i\omega}+e^{-i\omega}}2
+
2b_0b_2
\frac{e^{2i\omega}+e^{-2i\omega}}2
\\
&=
b_0^2 + b_1^2 + b_2^2
+
2(b_0b_1 + b_2b_1)
\cos\omega
+
2b_0b_2
\cos 2\omega
\\
&=
b_0^2 + b_1^2 + b_2^2
+
2(b_0b_1 + b_2b_1)
\cos\omega
+
2b_0b_2
(2\cos^2\omega-1)
\\
&=
b_0^2 + b_1^2 + b_2^2 -2b_0b_2
+
2(b_0b_1 + b_2b_1)
\cos\omega
+
4b_0b_2
\cos^2\omega.
\end{align*}
Durch Koeffizientenvergleich lesen wir die Gleichungen
\begin{align*}
b_0^2+b_1^2+b_2^2-2b_0b_2&=\phantom{-}\frac83
\\
(b_0+b_2)b_1 &= -\frac94
\\
b_0b_2
&=
\phantom{-}
\frac{3}{8}
\end{align*}
Dies sind drei quadratische Gleichungen für die Koeffizienten
$b_0$, $b_1$ und $b_2$, mit denen die Werte für die Koeffizienten
gefunden werden können.



