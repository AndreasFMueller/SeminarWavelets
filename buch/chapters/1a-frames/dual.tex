%
% dual.tex -- duale frames
%
% (c) 2019 Prof Dr Andreas Müller, Hochschule Rapperswil
%
\section{Das duale Frame\label{section:dual}}
\rhead{Das duale Frame}
In diesem Abschnitt gehen wird davon aus, dass $\mathcal{B}$ ein
Frame in $\mathbb{R}^n$ ist mit Framekonstanten $A$ und $B$.
Die Transformation $\mathcal{T}$, die zu diesem Frame gehört, liefert 
zu einem Vektor $v$ die Analyse-Koeffizienten $\hat{v} = \mathcal{T}v$.
Die Konstruktion eines Filters auf der Basis einer solchen Analyse verlangt
jedoch auch, dass die Abbildung $\mathcal{T}$ umgekehrt werden kann.
Von vornherein ist aber nicht klar, ob die Abbildung überhaupt umkehrbar ist,
denn es wurde ja nicht verlangt, dass die Framevektoren linear unabhängig
sind.
Im besten Fall kann man also Linksinverse von $\mathcal{T}$ erwarten.
\index{Linksinverse}
Eine solche ordnet jedem Koeffizienten $\hat{v}_k$ einen Vektor $v_k'$
zu, das sogenannte duale Frame, welches in diesem Abschnitt konstruiert
werden soll.

Wir betrachten in diesem Abschnitt immer ein Frame $\mathcal{B}$
mit Framevektoren $b_k$ und Framekonstanten $A$ und $B$.
Der Frameoperator $\mathcal{T}$ ordnet dem Vektor $v\in V$ den Vektor
mit Koordinaten $\hat{v}_k$ zu.

\subsection{Der Gram-Operator}
Das Frame $\mathcal{B}$ mit den Frame-Konstanten $A$ und $B$ erfüllt
\[
A\|v\|^2 \le \sum_{k=1}^m |\langle v,b_k\rangle|^2 \le B\|v\|^2
\qquad\forall v\in \mathbb R^n.
\]
Die Quadratsumme in der Mitte ist genau die Norm des Vektors
$\hat{v}$ in $\mathbb R^m$.
Unterscheiden wir für den Moment die verschiedenen Skalarprodukte
und Normen dadurch, dass wir die Vektorräume als Indizes hinzufügen,
können wir die Frame-Ungleichung schreiben als
\[
A \| v\|_{\mathbb R^n}^2
\le
\| Tv\|_{\mathbb R^m}^2
\le
B \| v\|_{\mathbb R^n}^2.
\]
Die Norm von $Tv$ in $\mathbb R^m$ ist
\[
\| Tv\|_{\mathbb R^m}^2
=
\langle Tv,Tv\rangle_{\mathbb R^m}
=
(Tv)^tTv
=
v^t(T^tT)v
=
\langle T^tTv,v\rangle_{\mathbb R^n}.
\]
Man beachte, dass $T^tT$ eine symmetrische $n\times n$-Matrix ist,
die codiert, wie stark die einzelnen Vektoren $v\in\mathbb R^n$ durch
den Frame-Operator $\mathcal{T}$ verzerrt werden.
Dies rechtfertigt eine eigene Definition.

\begin{definition}
\label{definition:gram-operator}
\index{Gram-Operator}
Ist $T$ die Matrix des Frame-Operators $\mathcal{T}$ eines Frames
$\mathcal{B}$, dann heisst $G=T^tT$ der {\em Gram-Operator} des Frames
$\mathcal{B}$.
\end{definition}

Die Frame-Bedingung lässt sich mit dem Gram-Operator $T^tT$ als
\begin{equation}
A \|v\|^2
\le
\langle T^t T v,v\rangle
\le
B \|v\|^2
\label{framebedingung:ttt}
\end{equation}
schreiben.
Man kann daraus schliessen, dass der Gram-Operator $T^tT$ regulär ist.

Die symmetrische Matrix $T^tT$ kann durch Wahl einer geeigneten Basis
in $\mathbb R^n$ diagonalisiert werden.
Die Eigenvektoren können sogar orthonormiert gewählt werden.
Seien daher $u_1,\dots,u_n$ orthonormierte Eigenvektoren von $T^tT$
mit Eigenwerten $\lambda_1\le\dots\le\lambda_n$.
Die Framebedingung~\eqref{framebedingung:ttt}
für den Vektore $v=u_k$ sagt dann
\[
A \| u_k\|^2
=
A
\le
\langle T^tTu_k,u_k\rangle
=
\lambda_k\langle u_k,u_k\rangle
=
\lambda_k
=
B \| u_k\|^2
=
B.
\]
Die Eigenwerte von $T^tT$ sind also alle zwischen $A$ und $B$.
Man kann sogar noch mehr sagen:

\begin{satz}
Sei $\mathcal{B}\subset\mathbb R^n$ eine Menge von Vektoren derart,
dass die zugehörige Matrix $T^tT$ regulär ist.
Seien weiter $\lambda_1\le\dots\le \lambda_n$ die Eigenwerte
(mit Vielfachheiten) von $T^tT$.
Dann ist $\mathcal{B}$ ein Frame mit Framekonstanten
$A=\lambda_1$ und $B=\lambda_n$.
\end{satz}

\begin{beispiel}
Im Beispiel von Abschnitt~\ref{subsection:hexagon} ist die Matrix $T$
gegeben in \eqref{beispielTmatrix}.
Der zugehörige Gram-Operator ist
\[
T^tT
=
\begin{pmatrix}
1& -\frac12        &-\frac12          \\[2pt]
0& \frac{\sqrt{3}}2& -\frac{\sqrt{3}}2
\end{pmatrix}
\begin{pmatrix}
1&0\\
-\frac12&\frac{\sqrt{3}}2\\[2pt]
-\frac12&-\frac{\sqrt{3}}2
\end{pmatrix}
=
\begin{pmatrix}
1+\frac14+\frac14 & 0-\frac{\sqrt{3}}4+\frac{\sqrt{3}}4\\[2pt]
0-\frac{\sqrt{3}}4+\frac{\sqrt{3}}4&0+\frac{3}{4}+\frac{3}{4}
\end{pmatrix}
=
\begin{pmatrix}
\frac32&0\\
0&\frac32
\end{pmatrix}
=
\frac{3}{2}E
\]
Da die Matrix $T^tT$ bereits diagonal ist, können die Eigenwerte
$\lambda_1=\lambda_2=\frac32$ direkt abgelesen werden.
\end{beispiel}

Das Frame des Beispiels ist straff und der Gram-Operator ist ein
Vielfaches der Einheitsmatrix.
Dies ist kein Zufall, wie das folgende Korollar zeigt.

\begin{korollar}
Das Frame $\mathcal{B}$ ist genau dann straff, wenn der zugehörige
Gram-Operator $T^tT$ ein Vielfaches der Einheitsmatrix ist.
\end{korollar}

\begin{figure}
\centering
\includegraphics{chapters/1-geometrie/images/beispiel2.pdf}
\caption{Beispiel eines nicht straffen Frames mit rationalen Framevektoren
$b_k$ (rot).
Die Vektoren $\tilde{b}_k$ des dualen Frames
(siehe Definition~\ref{definition:dualesframe}) sind blau eingezeichnet.
\label{frame:beispiel2}}
\end{figure}

\begin{beispiel}
Die Vektoren
\[
\mathcal{B} 
=
\biggl\{
\frac15
\begin{pmatrix} 3\\4\end{pmatrix},
\frac1{13}
\begin{pmatrix} -12\\5\end{pmatrix},
\frac1{17}
\begin{pmatrix} 8\\-15\end{pmatrix}
\biggr\}
\subset \mathbb R^2
\]
erzeugen den ganzen Raum $\mathbb R^2$, sie bilden daher ein
Frame (Abbildung~\ref{frame:beispiel2}).
Wir berechnen den Frame-Operator, den Gram-Operator und seine Eigenwerte,
um die Frame-Konstanten zu ermiteln.

Der Frame-Operator ist die Matrix
\[
T
=
\begin{pmatrix}
 \frac{ 3}{ 5}& \frac{ 4}{ 5}\\[2pt]
-\frac{12}{13}& \frac{ 5}{13}\\[2pt]
 \frac{ 8}{17}&-\frac{15}{17}
\end{pmatrix}.
\]
Der Gram-Operator wird daher
\[
G=T^tT
=
\begin{pmatrix}
 \frac{ 3}{ 5}&-\frac{12}{13}&  \frac{ 8}{17}\\[2pt]
 \frac{ 4}{ 5}& \frac{ 5}{13}& -\frac{15}{17}
\end{pmatrix}
\begin{pmatrix}
 \frac{ 3}{ 5}& \frac{ 4}{ 5}\\[2pt]
-\frac{12}{13}& \frac{ 5}{13}\\[2pt]
 \frac{ 8}{17}&-\frac{15}{17}
\end{pmatrix}
=
\frac{1}{1221025}
\begin{pmatrix}
1750369&  286808 \\
 286808& 1676106
\end{pmatrix}
\]
Die Eigenwerte dieser $2\times 2$-Matrix können als Nullstellen des
charakteristischen Polynoms mit der Lösungsformel für quadratische
Gleichungen berechnet werden.
Man findet zum Beispiel mit Hilfe eines Computer-Algebra-Systems
\[
\lambda_{\pm}
=
\frac{52715\pm 41\sqrt{47105}}{37570}
\quad\Rightarrow\quad
\left\{
\begin{aligned}
A&=
1.166262672548251
\\
B&=
1.639965701154171.
\end{aligned}
\right.
\]
Da die beiden Konstanten nicht übereinstimmen, ist dieses Frame
nicht straff.
\end{beispiel}

\begin{proof}[Beweis des Korollars]
Ein straffes Frame hat $A=B$, was gleichbedeutend damit ist, dass
die Eigenwerte des Frame-Operators $T^tT$ mit 
$A=B=\lambda_1=\dots=\lambda_n$ übereinstimmen.
Dies wiederum ist gleichbedeutend damit, dass alle Vektoren Eigenvektoren
zum gleichen Eigenwert sind und $T^tT=AE=BE$.
\end{proof}

\subsection{Die inverse Abbildung des Frame-Operators}
Für ein straffes Frame gilt $T^tT  = AE$.
Bis auf den Faktor $A$ ist daher $T^t$ bereits eine Linksinverse von $T$.
Genauer, mit
\[
S=\frac1{A} T^t
\qquad\text{folgt}\qquad
ST
=
\frac1{A} T^tT
=
\frac1{A} AE
=
E.
\]
Eine Linksinverse von $\mathcal{T}$ ist also für ein straffes Frame
leicht zu finden.

Für ein beliebiges Frame lässt sich ebenfalls eine Linksinverse angeben.
Mit $G=T^tT$ ist
\[
v
=
\underbrace{G^{-1}T^t}_{\displaystyle=S}Tv
=
STv,
\]
die Matrix $S=G^{-1}T^t$ ist also die gesuchte Linksinverse von $T$.

\subsection{Das duale Frame}
\index{duales Frame}
\index{Frame!duales}
Um den Frame-Operator umzukehren, muss man $G^{-1}T^t$ berechnen
können.
Auf einen Vektor $\hat{v} = (\hat{v}_k)_{1\le k\le m} = Tv$ angewendet
bedeutet dies, dass man $G^{-1}$ auf
\[
T^t \hat{v} = \sum_{k=1}^m \hat{v}_k  b_k
\]
anwenden muss, dabei wurde verwendet, dass $T^t$ als Spalten die
Vektoren $b_k$ enthält.
Anwendung von $G^{-1}$ liefert
\[
v
=
G^{-1} T^t \hat{v}
=
\sum_{k=1}^m \hat{v}_k G^{-1}b_k.
\]
Den Vektor $v$ erhält man also als Linearkombination der Vektoren
\[
\tilde{\mathcal{B}}
=
\{ G^{-1}b_1,\dots,G^{-1}b_n\}
=
\{ \tilde{b}_k = G^{-1}b_k\,|\, 1\le k\le m\}.
\]

\begin{definition}
\label{definition:dualesframe}
Ist $\mathcal{B}=\{b_1,\dots,b_m\}$, dann heisst
$\tilde{\mathcal{B}} = \{\tilde{b}_1,\dots,\tilde{b}_m\}$ 
mit
$\tilde{b}_k=G^{-1} b_k$
das zu $\mathcal{B}$ duale Frame.
\end{definition}

\begin{beispiel}
Das duale Frame zum Beispiel zu Abbildung~\ref{frame:beispiel2} kann
ebenfalls mit Hilfe eines Computer-Algebra-Systems berechnet werden aus
der bereits früher berechneten Matrix bestimmt werden.
Aus 
\[
G^{-1}T^t
=
\begin{pmatrix}
\frac{4259}{8053}
	&-\frac{718432}{1167685}
		&\frac{284716}{1167685}\\[2pt]
\frac{5422}{8053}
	&\frac{408473}{2335370}
		&-\frac{1203549}{2335370}
\end{pmatrix}
=
\begin{pmatrix}
0.52887&-0.61526&\phantom{-} 0.24383\\
0.67329&\phantom{-} 0.17490&-0.51536
\end{pmatrix}
\]
findet man das duale Frame
\[
\tilde{\mathcal{B}}
=
\left\{
\begin{pmatrix}
\frac{4259}{8053}\\[2pt]
\frac{5422}{8053}
\end{pmatrix},
\begin{pmatrix}
-\frac{718432}{1167685}\\[2pt]
\frac{408473}{2335370}
\end{pmatrix},
\begin{pmatrix}
\frac{284716}{1167685}\\[2pt]
-\frac{1203549}{2335370}
\end{pmatrix}
\right\}
=
\left\{
\begin{pmatrix} 0.52887\\ 0.67329 \end{pmatrix},
\begin{pmatrix}-0.61526\\ \phantom{-}0.17490 \end{pmatrix},
\begin{pmatrix}\phantom{-}0.24383\\ -0.51536 \end{pmatrix}
\right\}.
\]
Die Vektoren des dualen Frames sind in Abbildung~\ref{frame:beispiel2}
blau eingezeichnet.
\end{beispiel}

\begin{korollar}
Ist $\mathcal{B}=\{b_1,\dots,b_m\}$ ein straffes Frame mit Framekonstanten
$A=B$, dann ist
\[
\tilde{\mathcal{B}}
=
\biggl\{\tilde{b}_k = \frac1Ab_k\,\bigg|\,1\le k\le m\biggr\}
\]
das dazu duale Frame.
\end{korollar}

\begin{proof}[Beweis]
Für ein straffes Frame ist $T^tT=AE$, also ist $G^{-1}=\frac1A E$.
Die Vektoren des dualen Frames sind daher
$\tilde{b}_k = \frac1AEb_k=\frac1Ab_k$.
\end{proof}

Die Linksinverse $S$ von $\mathcal{T}$ kann mit Hilfe des dualen Frames
sehr leicht formuliert werden.

\begin{satz}
Ist $\mathcal{B}$ ein Frame und $\tilde{\mathcal{B}}$ das dazu duale
Frame und ist $\hat{v} = Tv$, dann ist die Linksinverse $S$ von $T$
gegeben durch
\[
v
=
S\hat{v}
=
\sum_{k=1}^m \hat{v}_k \tilde{b}_k.
\]
Linearkombination der Vektoren des dualen Frames mit den Analysekoeffizienten
$\hat{v}_k$ liefert also das ursprüngliche Signal zurück.
\end{satz}

\begin{beispiel}
\label{beispiel3}
In diesem Beispiel betrachten wir das Frame bestehend aus den
Signalen, die auf einem von maximal elf Samples konstant sind
und sonst verschwinden.
Das Signal $b_k$ verschwindet für Samples $i>k$ und $i<k-10$
und ist so normiert, dass $\|b_k\|=1$.
Die zugehörigen dualen Signale können direkt berechnet werden,
zwei Beispiele sind in den Abbildungen \ref{b3-01} und \ref{b3-05}
dargestellt.
\def\beispieldrei#1#2{
\begin{figure}
\centering
\includegraphics{chapters/1-geometrie/images/b3-#1.pdf}
\caption{Signal $b_{#2}$ aus dem Beispiel von Seite~\pageref{beispiel3}
und zugehöriges duales Signal $\tilde{b}_{#2}$.
\label{b3-#1}}
\end{figure}
}
\beispieldrei{01}{1}
%\beispieldrei{02}{3}
%\beispieldrei{03}{6}
%\beispieldrei{04}{10}
\beispieldrei{05}{20}
%\beispieldrei{06}{30}
%\beispieldrei{07}{40}
%\beispieldrei{08}{50}
%\beispieldrei{09}{60}
%\beispieldrei{10}{70}
\end{beispiel}

In der Definition~\ref{definition:dualesframe} wird das duale Frame
definiert ohne zu rechtfertigen, dass die Vektoren $\tilde{b}_k$ 
tatsächlich ein Frame bilden.
Wegen $b_k=G\tilde{b}_k$ ist aber klar, dass die Vektoren $\tilde{b}_k$
den ganzen Vektorraum $\mathbb R^n$ erzeugen.
Insbesondere bilden sie ein Frame.
Damit sind jedoch die Framekonstanten noch nicht bestimmt.
Dies wird im folgenden Satz nachgeholt.

\begin{satz}
Ist $\mathcal{B}=\{b_1,\dots,b_m\}$ ein Frame mit Framekonstanten
$A$ und $B$, dann hat das duale Frame die Framekonstanten $1/B$ und $1/A$.
\end{satz}

\begin{proof}[Beweis]
Um die Frame-Konstanten des dualen Frames zu bestimmen,
müssen der Frame-Operator $\tilde{T}$ und Gram-Operator $\tilde{G}$
des dualen Frames bestimmt werden.
Das duale Frame besteht aus den Spaltenvektoren von $S=G^{-1}T^t$,
also ist der Frameoperator $\tilde{T}=S^t$ und damit ist der Gram-Operator
\[
\tilde{G}
=
\tilde{T}^t\tilde{T}
=
(S^t)^tS^t
=
SS^t
=
G^{-1}T^tTG^{-1}
=
G^{-1}.
\]
Der Gram-Operator des dualen Frames ist daher die Inverse des
Gram-Operators des ursprünglichen Frames.
Sind $A=\lambda_1\le \dots\le \lambda_n=B$ die Eigenwerte  von $G$,
dann sind
$1/B\le \lambda_n^{-1} \le \dots \le \lambda_1^{-1}\le 1/A$ 
die Eigenwerte von $\tilde{G}=G^{-1}$.
Daraus folgen die behaupteten Framekonstanten.
\end{proof}

%
% Allgemeine Frames
%
\subsection{Allgemeine Frames}
Die Wahl der Indexmenge $K$ in der Definition~\ref{definition:frame}
war einigermassen willkürlich.
Schon bei der Fouriertransformation ist eine solche diskrete Menge
für die Indizierung der Vergleichsfunktionen nicht mehr ausreichend.
Dort werden nämlich die Funktion $e^{i\omega t}$ mit $\omega\in\mathbb R$
verwendet.
Auch die geplante Anwendung auf Wavelets ist davon betroffen.
Dort wollen wir mit Funktionen $\psi_{a,b}$ vergleichen, die 
skalierte und verschobene Versionen eines Mutter-Wavelets $\psi$ sind,
mit Skalierungsfaktor $a\in\mathbb R^*$ und die Translationsdistanz
$b\in \mathbb R$.
\index{Skalierungsfaktor}%
\index{Translationsdistanz}%

Lässt man eine beliebige Indexmenge zu, ist die Definition der
Transformation $T$
\[
k
\mapsto
(Tv)(k) = \langle v,e_k\rangle
\]
als komplexwertige Funktion auf $K$ immer noch sinnvoll.
Für eine überabzählbare Indexmenge $K$ ist die Summe 
\[
\sum_{k\in K} |\langle v,e_k\rangle|^2,
\]
die in der Definition eines Frames auftritt, schlicht sinnlos.
Wir stehen hier also vor einem ähnlichen Problem wie bei der Frage,
wie man aus dem Raum der Signale auf $\mathbb R$ einen Hilbertraum machen kann.
Diese Frage wird in Kapitel~\ref{chapter:fourier} im Detail beantwortet.

Nehmen wir für den Moment an, dass es gelungen ist, eine Hilbertraum $H$
von Funktionen auf $K$ zu konstruieren.
Die Frame-Ungleichung kann dann mit Hilfe der Norm von $H$ formuliert
werden, sie lautet
\[
A\|v\|^2 \le \|Tv\|^2 \le B\|v\|^2.
\]
Die Norm in der Mitte ist als Norm in $H$ zu lesen.
Diese Ungleichungen sagen immer noch aus, dass kein Vektor $v\in V$ bei
der Transformation $T$ ``unsichtbar'' wird.
Wäre nämlich $Tv=0$ in $H$, dann wäre auch $\|Tv\|=0$ und damit
$\|v\|=0$.
Die Frame-Ungleichung stellt also sicher, dass die Abbildung $T$ 
injektiv und damit potentiell invertierbar ist.

Es ist aber keinesfalls garantiert, dass das Bild der Transformation $T$
den ganzen Hilbertraum $T$ abdeckt, ganz im Gegenteil.
Schon im Beispiel in Abschnitt~\ref{subsection:hexagon} wurde gezeigt,
dass die mit Hilfe eines Frames gefundenen Koeffizienten redundant sind.
Alle Vektoren der Menge
\[
\left\{
\left.
\begin{pmatrix}\hat{v}_1\\\hat{v}_2\\\hat{v}_3\end{pmatrix}
+\alpha\begin{pmatrix}1\\1\\1\end{pmatrix}
\,
\right|\,
\alpha \in\mathbb R
\right\}
\]
beschreiben den gleichen Punkt in der Ebene, aber nur einer davon 
wird von der Transformation $T$ erreicht.
Dies entspricht natürlich genau dem, was man erwartet: der Bildraum
der Ebene $\mathbb R^2$ unter der Abbildung $T$ ist ein zweidimensionaler
Teilraum des $\mathbb R^3$.

Die Abbildung $T$ wird daher im Allgemeinen nicht invertierbar sein, aber
wir dürfen hoffen, dass es eine Formel gibt, mit der man aus $Tv$ den 
Vektor $v$ rekonstruieren kann.
Im Falle des Beispiels war dies die Formel
\[
v = \frac23 \sum_{k=1}^3 \langle v,e_k\rangle \, e_k.
\]
Für ein beliebiges Frame ist so eine Formel natürlich wieder wegen
der Summe nicht sinnvoll.
Dach in Kapitel~\ref{chapter:fourier} werden wir sehen, dass sie sich
oft durch eine Integralformel ersetzen lässt.
Die Konstruktion eines solchen vektorwertigen Integrals ist allerdings
etwas subtil, wird kehren zu dieser Problematik im Kapitel~\ref{chapter:cwt}
zurück.


