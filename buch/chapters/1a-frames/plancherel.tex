%
% plancherel.tex
%
% (c) 2019 Prof Dr Andreas Müller, Hochschule Rapperswil
%
\section{Das Plancherel-Prinzip
\label{section:plancherel}}
\rhead{Das Plancherel-Prinzip}
Im Abschnitt~\ref{section:ginv} wurde gezeigt, wie der Gram-Operator
für ein beliebiges Frame numerisch invertiert werden kann.
In vielen praktischen Fällen ist dies jedoch gar nicht nötig, denn
die Frame-Konstanten stimmen überein, wie zum Beispiel im einführenden
Beispiel dieses Kapitels.
In diesem Fall ist der Gram-Operator ein Vielfaches der Einheitsmatrix.
Die Linksinverse $S$ des Frame-Operators $T$ eines Frames mit $A=B$
ist $A^{-1}T^*$.
Diese Aussagen folgen aus der bisher entwickelten Theorie sofort
für endlichdimensionale Vektorräume.

Etwas allgemeiner formuliert erfolgt die Analyse von Vektoren
mit einer diskreten Menge von Analyse-Vektoren, den Vektoren $(b_k)_{k\in I}$
des Frames.
Die Situation, in der die Transformation besonders leicht umgekehrt werden
kann, war gekennzeichnet dadurch, dass die Norm bei der Transformation
immer um den gleichen Faktor grösser wird,
ausgedrückt durch die Formel
\begin{equation}
\sum_{k\in I} |\langle v,b_k\rangle|^2
=
\sum_{k\in I} |\hat{v}_k|^2
=
A\| v\|^2.
\label{plancherel:equation:diskret}
\end{equation}
Die Umkehrtransformation ist dann ganz einfach:
Gegeben die Transformationskoeffizienten $a_k$ mit $k\in I$ wird der Vektor
\[
u = \sum_{k\in I} a_k v_k
\]
durch die Transformation wieder auf die gegeben Koeffizienten abgebildet,
also $\hat{u}_k = a_k\;\forall k\in I$.

Die in KapiteL~\ref{chapter:cwt} definierte stetige Wavelet-Transformation
verwendet jedoch eine Menge von Vektoren für die Analyse, die sehr viel
grösser ist.
Ausgehend von einer Funktion $\psi\colon\mathbb R\to\mathbb R$
wirt für jedes Paar $(a,b)\in \mathbb{R}^*\times \mathbb R$ mit
$\psi_{a,b}=T_bD_a\psi$ analysiert.
Die Menge der Analyse-Vektoren $\psi_{a,b}$ ist also überabzähltbar.
Trotzdem lässt sich die Rücktransformation verallgemeinern, wenn 
man die Eigenschaft~\eqref{plancherel:equation:diskret} auf die
stetige Situation verallgemeinern kann.

Für eine diskrete Menge $I$ ist das Resultat der Transformation des Vektors
$v$ eine auf $I$ definierte Funktion $k\mapsto \hat{v}_k$.
Die Funktionen auf $I$ beilden einen Hilbertraum mit dem Skalarprodukt
\[
\langle \hat{v},\hat{w}\rangle
=
\sum_{k\in I} \hat{v}_k \overline{\hat{w}_k}.
\]
Die Bedingung~\eqref{plancherel:equation:diskret} ist also nichts anderes
als die Aussage, dass die Transformation die Hilbertraum-Norm immer um
den gleichen Faktor $A$ skaliert:
\[
\| Tv \|^2 = A \| v\|^2.
\]
Die naheliegende Verallgemeinerung für eine beliebige Menge von
Analysefunktionen wie im Fall der stetigen Wavelettransformation
besteht also darin, aus der Menge der Funktionen auf der Indexmenge
einen Hilbertraum zu machen, so dass wieder eine Bedingung der Form
\eqref{plancherel:equation:diskret} gilt.

\begin{definition}
Sei $H$ ein Hilbertraum und
$v\colon X\to H$ eine Funktion mit Werten in $H$.
Die Funktion $v$ definiert eine Transformation
\[
T\colon  H \to \mathbb C^X : u \mapsto \hat{u}
\quad\text{mit}\quad
\hat{u}(x) = \langle u,\hat{v(x)}\rangle.
\]
Sei ausserdem $\mu$ ein Mass auf $X$ und $L^2(X)$ der Hilbertraum
der Funktionen auf $X$ mit dem Skalarprodukt
\[
\langle f,g\rangle_{L^2(X)} = \int_X f(x) \bar{g}(x)\,d\mu(x).
\]
Man sagt, $T$ erfüllt die {\em Plancherel-Bedingung}, wenn $T$
eine Transformation $H\to L^2(X)$ ist und es eine Konstante $C$
gibt mit
\[
\| Tv \|_{L^2(X)}
=
\biggl(
\int_{X} |(Tv)(y)|^2 \, d\mu(y)
\biggr)^{\frac12}
=
C \|v\|_H.
\]
\end{definition}

Ein Frame mit Framevektoren $v_k, k\in\mathbb N$ mit
ist der Fall $X=\mathbb N$. Die Plancherel-Bedingung ist erfüllt, wenn
die Frame-Konstanten $A=B$ gleich sind.

\begin{satz}
\label{satz:plancherel-prinzip}
Falls $T$ die Plancherel-Bedingung erfüllt mit der Konstanten $C$, dann ist
die Abbildung
\[
S
\colon
L^2(X) \to H
:
a \mapsto \frac{1}{C}\int_X v_x a(x) \,d\mu(x)
\]
ist eine Inverse der Transformation $T$.
\end{satz}

\begin{proof}[Beweis]
Die Polar-Identität besagt, dass nicht nur die Normen sondern auch die
Skalarprodukte in den beiden Hilberträumen sich nur um einen Faktor
unterscheiden:
\[
\| Tu \| = C \| u \|
\qquad\Rightarrow\qquad
\langle Tu,Tw\rangle = C^2 \langle u,w\rangle.
\]
Geschrieben als Integral folgt
\[
\langle Tu,Tw\rangle
=
\int_X \hat{u}(x) \overline{\hat{w}(x)} \,d\mu(x)
=
\int_X \langle u,v_x\rangle \overline{\langle w,v_x\rangle}\,d\mu(x)
=
\biggl\langle
\int_X \langle u,v_x\rangle v_x\,d\mu(x),
w
\biggr\rangle
=
C^2
\langle u,w\rangle
\]
Da dies für jeden beliebigen Vektor $w$ gilt, folgt 
\[
\int_X \langle u,v_x\rangle v_x\,d\mu(x) = C^2 u
\]
der
\[
u = \frac{1}{C^2}\int_X \langle u,v_x\rangle v_x\,d\mu(x).
\]
Das Integral reproduziert also den Vektor $u$.
\end{proof}

Die Umkehrbarkeit der Transformation $T$ ist also nicht nur für gewöhliche
Frames mit $A=B$ möglich sondern auch in Fällen, wo mit einer überabzählbaren
Menge von Vektoren analyisiert wird.
Wir werden diese Eigenschaft im Kapitel~\ref{chapter:cwt} wieder antreffen,
wenn wir die Umkehrformel für die stetige Wavelet-Transformation finden.





