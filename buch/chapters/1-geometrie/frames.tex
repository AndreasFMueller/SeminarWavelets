%
% frames.tex
%
% (c) 2019 Prof Dr Andreas Müller, Hochschule Rapperswil
%
\section{Frames
\label{section:frames}}
Eine orthonormierte Basis eines Hilbertraumes ist sehr starr.
Es ist nicht möglich, auch nur einen einzigen Vektor ein kleines
Bisschen zu ändern, ohne die Eigenschaften, die zum Satz~\ref{satz:parseval}
geführt haben, zu zerstören.
Im Hinblick auf die numerische Behandlung von Signalen ist das
ein unerwünschter Zustand.
Rundungsfehler werden unvermeidlich dazu führen, dass solche strengen
Strukturen nur näherungsweise im Computer nachgebildet werden 
können.

Die Zerlegung eines Vektors $v$ in einer Orthonormalbasis enthält keine
Redundanz.
Geht einer der Koeffizienten $\hat{v}_k$ verloren, gibt es keine
Chance, den Vektor zu rekonstruieren.
Auch diese Situation ist unerwünscht, denn durch Rundungsfehler geht
mindestens ein Teil der Information in einem Koeffizienten verloren.
Wir suchen daher nach einer Verallgemeinerung des Basis-Begriffs, welche
auf kontrollierte Weise Redundanz in die Koeffizienten $\hat{v}_k$
bringt und damit eine robustere Konstruktion ermöglicht.

\subsection{Ein geometrisches Beispiel}
Wir suchen ein geeignetes Koordinatensystem, um ein Problem über
Bienenwaben in der Ebene zu lösen.
Selbstverständlich kann dafür das übliche rechtwinklige Koordinatensystem
verwendet werden, aber die Ecken eines Sechsecks shaben darin die nicht
sehr symmetrischen Koordinaten
\begin{align*}
&      &&\textstyle(-\frac12,\frac12\sqrt{3})&&\textstyle( \frac12,\frac12\sqrt{3})&     \\
&(-1,0)&&                          &&                          &(1,0)\\
&      &&\textstyle(-\frac12,-\frac12\sqrt{3})&&\textstyle( \frac12,-\frac12\sqrt{3})&     
\end{align*}
(Siehe auch Abbildung~\ref{buch:frame:hexagon}).
Eine bessere Variante ist das Koordinatensystem auf der Basis der beiden
Basisvektoren (in kartesischen Koordinaten)
\[
b_1 = \begin{pmatrix} 1\\0\end{pmatrix}
\qquad
\text{und}
\qquad
b_2 = \begin{pmatrix} -\frac12\\\frac12\sqrt{3}\end{pmatrix}.
\]
In diesem Koordinatensystem haben die Ecken des Sechsecks die Koordinaten
\begin{align*}
&      &&(0,1)  &&(-1,1)&&     \\
&(-1,0)&&       &&      &&(1,0)\\
&      &&(-1,-1)&&(0,-1)&&
\end{align*}
Was bereits viel besser aussieht.
Trotzdem ist auch dies noch nicht ganz zufriedenstellend. 
Zum Beispiel sind die Ecken links oben und rechts unten direkt durch den
Basisvektor $b_2$ darstellbar, die Ecken rechts oben und links unten
dagegen nur durch eine Linearkombination.
Wir könnten natürlich auch die linke untere Ecke als Basisvektor nehmen,
dann würde eiinfach die linke obere Ecke speziell.
In dieser Situation lässt es sich also mit einer Basis gar nicht erreichen,
dass alle Eckpunkte sich auf einfache Art darstellen lassen.

Verzichten wir jedoch daruf, dass die Vektoren linear unabhängig sein müssen,
können wir also ``Basis'' die drei Vektoren (in kartesischen Koordinaten)
\begin{align*}
b_1
&=
\begin{pmatrix}1\\0\end{pmatrix}
&
b_2
&=
\begin{pmatrix}-\frac12\\\frac12\sqrt{3}\end{pmatrix}
&
b_3
&=
\begin{pmatrix}-\frac12\\-\frac12\sqrt{3}\end{pmatrix}
\end{align*}
verwenden.
Die drei Vektoren haben alle die Länge $1$, aber sie sind nicht
orthogonal, sondern haben das Skalarprodukt
\[
\langle b_j,b_k\rangle
=
\begin{cases}
-\frac12&\qquad j\ne k\\
0&\qquad j=k.
\end{cases}
\]
Zu jedem Vektor $v\in\mathbb R^2$ können wir wieder eie Koeffizienten
$\hat{v}_k=\langle v,b_k\rangle$ berechnen und damit die Linearkombination
\[
v' = \sum_{k=1}^3 \hat{v}_k\,b_k
\]
bilden,
doch es ist $v\ne v'$.
Wir berechnen die Synthese für die Basisvektoren:
\begin{align*}
b_1'
&=
b_1 - \frac12 b_2 - \frac 12 b_3
=
\frac32b_1
\\
b_2'
&=
-\frac12 b_1 + b_2 -\frac12 b_3
=
\frac32b_2
\\
b_3'
&=
-\frac12b_1-\frac12 b_2 + b_3
=
\frac32b_3
\end{align*}
Die Synthese liefert also nicht den ursprünglichen Vektor, sondern das
$frac32$-fache davon zurück.
Dies ist bereits ein Ausdruck der Tatsache, dass die Information in den
Koeffizienten $\hat{v}_k$ redundant ist.

Diese Vektoren sind natürlich nicht mehr linear unabhängig, Vektoren
der Ebene können also auf verschieden Art linear aus den $b_k$ kombiniert
werden.
Da $b_1+b_2+b_3=0$ ist, kann man zu den Koeffizienten $\hat{v}_k$
eine beliebige Zahl $\alpha$ hinzuaddieren, und erhält
\[
\sum_{k=1}^3 (\hat{v}_k+\alpha)\, b_k
=
\sum_{k=1}^3 \hat{v}_k\,b_k
+\alpha
\sum_{k=1}^3 b_k
=
\sum_{k=1}^3 \hat{v}_k\,b_k
=
\frac32 v.
\]
Die modifizierten Koeffizienten ergeben also das gleichen synthetisierten
Vektor $\frac32 v$.

Für die Norm des synthetisierten Vektors gilt natürlich
\[
\|v'\|^2
=
\frac94\|v\|^2
=
\sum_{k=1}^3 \hat{v}_k^2 
-
\frac12\sum_{k\ne l} \hat{v}_k\hat{v}_l.
\]
Die modifizierten Koeffizienten ergeben natürlich dieselbe Norm, also
\begin{align*}
\|v'\|^2
&=
\sum_{k=1}^3 (\hat{v}_k+\alpha)^2 
-
\frac12\sum_{k\ne l} (\hat{v}_k+\alpha)(\hat{v}_l+\alpha).
\\
&=
\sum_{k=1}^3 \hat{v}_k^2
+2\alpha \sum_{k=1}^3 \hat{v}_k
+3\alpha^2
-
\frac12\sum_{k\ne l} \hat{v}_k\hat{v}_l
-
\sum_{k\ne l} \hat{v}_k \alpha
-
3\alpha^2
\end{align*}


Was zeichnet die Koeffizienten $\hat{v}_k$ gegenüber den modifizierten
Koeffiziente $\hat{v}_k$ aus?
Im Falle einer Orthonormalbasis konnte man dank der Parseval-Formel
die Norm eines Vektors mit Hilfe der Quadratsumme der Koeffizienten
berechnen.
Wir berechnen daher
\[
f(\alpha)
=
\sum_{k=1}^n (\hat{v}_k+\alpha)^2
=
\sum_{k=1}^n \hat{v}_k^2 + 2\alpha \sum_{k=1}^3 \hat{v}_k + 3\alpha^2
\]

\subsection{Definition eines Frames}




