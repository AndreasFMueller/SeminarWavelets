%
% abstand.tex -- Abstand von einer Geraden
%
% (c) 2019 Prof Dr Andreas Müller, Hochschule Rapperswil
%
\documentclass[tikz]{standalone}
\usepackage{amsmath}
\usepackage{times}
\usepackage{txfonts}
\usepackage{pgfplots}
\usepackage{csvsimple}
\usetikzlibrary{arrows,intersections,math}
\begin{document}
\begin{tikzpicture}[>=latex,scale=0.8]

\coordinate (O) at (0,0);
\coordinate (X) at (4,1);
\coordinate (Y) at (2,6);

\draw[->,line width=1pt] (O)--(X);
\draw[->,line width=1pt] (O)--(Y);

\pgfmathparse{14/17}
\xdef\t{\pgfmathresult}

\draw[line width=0.5pt] (-4,-1)--(8,2);
\draw[->,line width=1.4pt,color=red] (O)--({\t*4},{\t*1});

\node at (1,3) [left] {$y$};
\node at (X) [below right] {$x$};

\node[color=red] at ({\t*4},{\t-0.1}) [below] {$\tilde{x}=tx$};

\draw[->,line width=0.7pt] ({\t*4},{\t})--(Y);

\def\punkt#1{
	\fill[color=white] #1 circle[radius=0.08];
	\draw[line width=0.7pt] #1 circle[radius=0.08];
}

\punkt{(O)}
\punkt{(Y)}

\def\punktrot#1{
	\fill[color=white] #1 circle[radius=0.08];
	\draw[color=red,line width=0.7pt] #1 circle[radius=0.08];
}
\punktrot{({4*\t},{\t})}

\node at ({0.5*(2+4*\t)},{0.5*(6+\t)}) [right] {$y-\tilde{x}$};

\node at (O) [below] {$O$};

\end{tikzpicture}
\end{document}

