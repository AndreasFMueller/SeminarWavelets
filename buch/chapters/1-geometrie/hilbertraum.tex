%
% hilbertraum.tex
%
% (c) 2019 Prof Dr Andreas Müller, Hochschule Rapperswil
%
\section{Hilbertraum
\label{section:hilbertraum}}
Die bisher entwickelte Theorie ist aus zwei Gründen nicht ausreichend für das,
was wir vorhaben.

Ein endlichdimensionaler Vektorraum ist sicher ein geeigneter Rahmen zur
Beschreibung eines Signals, welches an endlich vielen Stellen abgetastet
wurde.
Für kontinuirliche Signale reicht er aber nicht.
Zunächst ist die Menge aller Funktionen zwar ein Vektorraum, aber es ist
aussichtslos, eine endliche orthonrmierte Basis zu finden.
Vielmehr werden wir sehen, dass bereits der Raum der stetigen Funktionen
auf einem Interval unendlich viele Funktionen enthält, die in einem noch
zu definierenden Sinn aufeinander senkrecht stehen.
Wir müssen daher den Begriff erweitern, so dass auch unendlichdimensionale
Vektorräume behandelt werden können.

In der Praxis tauchen nicht nur Signale mit rellen Werten auf, sondern
auch solche mit komplexen Werten.
An entscheidenden Stellen im vorangegangen Abschnitt, insbesondere bei
der Konstruktion der Norm, haben wir verwendet,
dass $x^2\le 0$ ist für $x\in\mathbb R$. Für $x\in\mathbb C$ ist dies
nicht mehr wahr.
Der bisher formulierte Begriff des Skalarproduktes funktioniert daher
nicht für komplexe Signale.

\subsection{Komplexe Vektorräume mit Skalarprodukt}

\subsection{Norm und Grenzwert in einem Hilbertraum}

\subsection{Basis eines Hilbertraumes}

