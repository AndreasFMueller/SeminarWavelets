%
% funktionen.tex
%
% (c) 2018 Prof Dr Andreas Müller
%
\section{Funktionenräume%
\label{section:funktionenraume}}
\rhead{Funktionräume}

\subsection{Vektorräume von Funktionen}

\subsection{Translation und Dilatation}
Der Definitionsbereich $\mathbb R$ zeichnet sich durch die darin zur
Verfügung stehenden Rechenoperationen aus.
Man kann zu einem Argument einen konstanten Wert hinzuaddieren oder mit
einer Zahl multiplizieren, bevor man eine Funktion auswertet.
Daraus ergeben sich die zwei Operationen der Translation und der Dilatation.

\begin{figure}
\centering
\includegraphics[width=\hsize]{chapters/1-geometrie/images/translation.pdf}
\caption{Wirkung des Operators $T_b$ auf das Gabor-Wavelet
$\psi(t) = e^{-t^2/2}\sin(6t)$,
der Graph von $T_b\psi$ ist um $b$ nach rechts verschoben.
\label{geometrie:Tb:image}}
\end{figure}
\begin{figure}
\centering
\includegraphics[width=\hsize]{chapters/1-geometrie/images/dilatation.pdf}
\caption{Wirkung des Operators $D_a$ auf ein im Punkt $c$ zentriertes
Wavelet $\psi$. Das Wavelet $D_a\psi$ ist zentriert im Punkt $a\cdot c$.
\label{geometrie:Da:image}}
\end{figure}

\begin{definition}
Sei $f$ eine Funktion auf $\mathbb R$ mit Werten in $Y$.
Dann setzt man
\begin{align*}
T_bf&\colon \mathbb R \to Y: t\mapsto f(t-b)&&\text{Translation}
\\
D_af&\colon \mathbb R \to Y: t\mapsto f(t/a)&&\text{Dilatation}
\end{align*}
\end{definition}

Die Wirkung der beiden Operatoren ist in den
Abbildungen~\ref{geometrie:Tb:image} und \ref{geometrie:Da:image} dargestellt.
Der Operator $D_a$ dehnt den Graphen der Funktion entlang der
$t$-Achse um den Faktor $a$ während der Operator $T_b$ den Graphen
der Funktion um den Betrag $b$ entlant der $t$-Achse verschiebt.

\begin{figure}
\centering
\includegraphics[width=\hsize]{chapters/1-geometrie/images/kommutator.pdf}
\caption{Die Operatoren $T_b$ und $D_a$ vertauschen nicht.
Der rote Pfad wendet erst $T_b$ und dann $D_a$ an, der blaue zuerst
$D_a$ und dann erst $T_b$.
In beiden Fällen erhält man ein um den Faktor $a$ gestrecktes
Wellenpaket.
Im Fall $T_b\circ D_a$ erhält man ein im Punkt $b$ zentriertes Wellenpaket
(blaue), während es im Falle $D_a\circ T_b$ im Punkt $ab$ zentriert ist (rot).
Wendet man in der unteren Zeile statt $T_b$ den Operator $T_{ab}$ an, 
kommen die beiden Wellenpakete zur Deckung, was die Vertauschungsregel
von Satz~\ref{satz:kommutator} bestätigt.
\label{geometrie:kommutator:image}}
\end{figure}

\begin{satz}
\label{satz:kommutator}
Translation $T_b$ und Dilatation $D_a$ sind lineare Abbildungen.
Die beiden Operatoren vertauschen nicht, vielmehr gilt
$T_{ab}D_a = D_aT_b$.
\end{satz}

\begin{proof}[Beweis]
Die Linearität der Operatoren wird durch Nachrechnen verifiziert.
Für eine Linearkombination $\lambda f+\mu g$ zweier Funktion $f$ und $g$ gilt:
\begin{align*}
D_a(\lambda f+\mu g)(t)
&=
(\lambda f+\mu g)(t/a)
=
\lambda f(t/a)+\mu g(t/a)
=
\lambda (D_af)(t)+\mu (D_ag)(t),
\\
T_b(\lambda f+ \mu g)(t)
&=
(\lambda f+\mu g)(t-b)
=
\lambda f(t-b)+\mu g(t-b)
=
\lambda (T_bf)(t)+\mu (T_bg)(t),
\end{align*}
Ohne Argumente geschrieben heisst dies
\begin{align*}
D_a(\lambda f+\mu g) &= \lambda D_af + \mu D_ag
\\
T_b(\lambda f+\mu g) &= \lambda T_bf + \mu T_bg
\end{align*}
Damit ist die Linearität bewiesen.

Für die Vertauschungsregel müssen die beiden Seiten der Regel
berechnet werden.
Die Wirkung der beiden Operatoren in verschiedener Reihenfolge
ist:
\begin{align*}
(T_bf)(t)
&=
f(t-b)
\\
(D_af)(t)
&=
f(t/a)
\\
(D_aT_bf)
&=
(T_bf)(t/a)
=
f(t/a-b)
=
f((t-ab)/a)
\\
(T_{ab}D_a f)(t)
&=
(D_af)(t - ab)
=
f((t-ab)/a)
\end{align*}
Daraus kann man ablesen, dass $T_{ab}D_a=D_aT_b$.
\end{proof}

Die Abbildung~\ref{geometrie:kommutator:image} illustriert die
Vertauschungsregel für die Operatoren $T_b$ und $D_a$ und liefert
einen ``graphischen'' Beweis für die Vertauschungsregel von
Satz~\ref{satz:kommutator}.

\begin{satz}
Die einzigen stetigen Funktionen, die Eigenvektoren von $T_b$ sind für
jeden beliebigen Wert von $b$ sind die Funktionen $t\mapsto e^{i\omega t}$.
\end{satz}

\begin{proof}[Beweis]
Sei $f$ eine Eigenfunktion aller Operatoren $T_b$ mit Eigenwert $\lambda(b)$.
 kann keine Nullstelle haben.
Wäre nämlich $f(t_0)=0$, dann würde folgen
\[
(T_bf)(t_0) = f(t_0-b) = \lambda(b) f(t_0) = 0,
\]
die Funktion würde identisch verschwinden.

Weiter kann man aus der Stetigkeit von $f$ schliessen, dass auch
\[
\lambda(b) = \frac{f(t)}{f(t-b)}
\]
eine stetige Funktion von $b$ ist, die keine Nullstellen hat.
\end{proof}

Dieser Satz erklärt die besondere Stellung, die den komplexen
Exponentialfunktionen in der Fourier-Theorie zukommt.
