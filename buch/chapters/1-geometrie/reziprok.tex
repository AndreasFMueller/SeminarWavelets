%
% reziprok.tex
%
% (c) 2019 Prof Dr Andreas Müller, Hochschule Rapperswil
%
\subsection{Iterative Berechnung eines reziproken Wertes}
Zu einer beliebigen Zahl $y$ mit $A < y < B$ muss der reziproke Wert $1/y$
bestimmt werden.
Um die Berechnung der inversen Matrix zu vermeiden, suchen wir nach einem
iterativen Verfahren der Form $x_{n+1}=f(x_n)$, welches für einen beliebigen
Startwert $x_0$ gegen $y^{-1}$ konvergiert.
Der Wert $y^{-1}$ muss also der einzige Fixpunkt der Funktion $f(x)$ sein:
$f(y^{-1}=y^{-1}$.

Die Funktion $f(x) = x + g(x)$ hat als einzigen Fixpunkt die Stelle $x=y^{-1}$
genau dann, wenn $(x)$ als einzige Nullstelle die Stelle $x=y^{-1}$ hat.
Ausserdem muss die Funktion $g(x)$ berechenbar sein ausschliesslich mit
Addition und Multiplikation von Konstanten und der Variablen $x$, Division
darf nicht verwendet werden.
Die einfachste Funktion dieser Art ist eine lineare Funktion, zum 
Beispiel $g(x) = 1- xy$.

Damit das Iterationsverfahren konvergiert, sind aber noch weitere
Bedingungen zu erfüllen.
% XXX für das nachfolgende ist möglicherweise eine Graphik nötig
Die Folge $x_{n+1} = f(x_n)$ konvergiert nämlich nur, wenn die Steigung
der Funktion $f(x)$ in der Nähe des Fixpunktes Betrag $<1$ hat, 
also $|f'(x)|<1$..
Für die eben konstruierte Funktion trifft dies im Allgemeinen nicht zu,
es ist nämlich
\[
f'(x) = 1 + g'(x) = 1 - y
\]
Für $y>2$ wird nämlich $|f'(x)|'=|1-y| = y-1 >1$, so dass keine Konvergenz
mehr möglich ist.
Allerdings ist Konvergenz auch nicht für alle Werte von $y$ nötig, sondern
nur für Werte zwischen $A$ und $B$.
Das kann man mit Hilfe eines Skalierungsfaktors $c$ in der Iteration
\[
f(x) = x + c(1-yx)
\]
erreichen.
Die Ableitung wird damit
\[
f'(x) = 1-cy.
\]
$c$  muss also so gewählt werden, dass $|f'(x)|<1$ gilt für alle $y$
zwischen $A$ und $B$.
Dies bedeutet, dass
\begin{align*}
-1 &< 1-cy < 1 \\
-2 &< -cy < 0 \\
2 &> cy > 0 \\
\frac1c &> \frac{y}2 \\
\end{align*}
Diese Ungleichung muss für alle möglichen Werte für $y$ erfüllt sein,
also sowohl für $y=A$ also auch für $y=B$.
Man kann dies zum Beispiel dadurch erreichen, dass man
\[
\frac1c = \frac{A}2 + \frac{B}2
\qquad\Rightarrow\qquad
c = \frac{2}{A+B}
\]
setzt.

Damit haben wir jetzt ein Iterationsverfahren zur Berechnung des 
reziproken Wertes, welches wir im folgenden Satz zusammenfassen.

\begin{satz}
\label{iteration:reziprok}
Ist $y$ eine reelle Zahl zwischen $A$ und $B$, $A\le y\le B$ dann
konvergiert die Folge
\[
x_{n+1} = x_n + \frac{2}{A+B}(1-yx_n), \quad x_0 = 0
\]
gegen den reziproken Wert $y^{-1}$ von $y$.
Die Differenz zwischen aufeinanderfolgenden Approximationen nimmt
in jedem Schritt um mindestens den Faktor $(B-A)/(B+A)<1$ ab.
\end{satz}

\begin{proof}[Beweis]
Wir müssen uns nur noch mit der Aussage über die Konvergenz befassen.
Dazu berechnen wir die Differenz zweier aufeinanderfolgender
Approximationen 
\begin{equation}
\left.
\begin{aligned}
x_{n+1} &= x_n + \frac{2}{A+B}(1-yx_n) \\
x_n &= x_{n-1} + \frac{2}{A+B}(1-yx_{n-1}) 
\end{aligned}
\right\}
\quad\Rightarrow\quad
x_{n+1}-x_n
=
x_n - x_{n-1} -\frac{2y}{A+B}(x_n-x_{n-+})
=
\underbrace{
\biggl(1-\frac{2y}{A+B}\biggr)
}_{\displaystyle=\vartheta}
(x_n - x_{n-1})
\end{equation}
Die Differenz wird also tatsächlich um den $\vartheta$ ab.
Wir schätzen diesen Faktor ab
\[
\vartheta
=
1-\frac{2y}{A+B} = \frac{A+B-2y}{A+B}
\qquad
\Rightarrow
\qquad
|\vartheta|
<
\biggl|\frac{A+B-2B}{A+B}\biggr|
=
\biggl|\frac{A-B}{A+B}\biggr|
=
\frac{B-A}{B+A}.
\]
Da $A>0$ folgt auch $\vartheta<1$.
\end{proof}

Aus der Abschätzung von $|x_{n+1}-x_n|$ kann auch der Fehler
abgeschätzt werden.
Es gilt nämlich
\[
\frac1y
-
x_n
=
\frac1y - x_{n+1} + (x_{n+1} - x_n)
=
\frac1y - x_{n+1} + (x_{n+1} - x_n)
=
\sum_{k=n}^\infty(x_{k+1}-x_k).
\]
Wegen
\[
|x_{k+1} - x_k|
\le
\vartheta | x_{k}-x_{k-1}|
\le
\vartheta^2 | x_{k-1}-x_{k-2}|
\le \dots
\le
\vartheta^{k-n} |x_{n+1}-x_{n}|
\]
wird der Fehler der Approximation
\begin{align*}
\biggl|
\frac1y
-
x_n
\biggr|
=
\biggl|
\sum_{k=n}^\infty(x_{k+1}-x_k).
\biggr|
&\le
\sum_{k=n}^\infty|x_{k+1}-x_k|
\\
&\le
\sum_{k=n}^\infty \vartheta^{k-n} \cdot |x_{n+1}-x_n|
\\
&=
\sum_{k=0}^\infty \vartheta^k \cdot |x_{n+1}-x_n|
\\
&=
\frac1{1-\vartheta}\cdot |x_{n+1}-x_n|
\le
\frac{\vartheta^n}{1-\vartheta} |x_1-x_0|.
\end{align*}
Daraus kann man auch ablesen, dass der Fehler wie $\vartheta^n$ gegen $0$
geht.

\begin{beispiel}
Wir verwenden das Verfahren, um den Wert von $1/3$ zu berechnen, also
für $y=3$.
Das Verfahren soll beliebige Werte zwischen $A=2$ und $B=5$ berechnen können.
Mit dem Startwert $x_0=0$ bekommen wir
\begin{center}
\begin{tabular}{>{$}r<{$}|>{$}l<{$}}
n&x_n\\
\hline
 0&0.000000000000000\\
 1&0.285714285714286\\
 2&0.\underline{3}26530612244898\\
 3&0.\underline{33}2361516034985\\
 4&0.\underline{333}194502290712\\
 5&0.\underline{3333}13500327245\\
 6&0.\underline{33333}0500046749\\
 7&0.\underline{33333}2928578107\\
 8&0.\underline{333333}275511158\\
 9&0.\underline{3333333}25073023\\
10&0.\underline{33333333}2153289\\
11&0.\underline{333333333}164756\\
12&0.\underline{3333333333}09251\\
13&0.\underline{3333333333}29893\\
14&0.\underline{33333333333}2842\\
15&0.\underline{333333333333}263\\
16&0.\underline{3333333333333}23\\
17&0.\underline{33333333333333}2\\
18&0.\underline{333333333333333}\\
\end{tabular}
\end{center}
In jeder Iteration gewinnt man die gleiche Anzahl korrekter Stellen
(unterstrichen).
Aus dem Beweis von Satz~\ref{iteration:reziprok} kann man ablesen,
dass die Konvergenzgeschwindigkeit
\[
\frac{A+B-2y}{A+B}
=
\frac{7-6}{7} = \frac17
\]
ist, man gewinnt also
$\log 7=0.84510$
Stellen in jeder Iteration, in Übereinstimmung mit den obigen numerischen
Resultaten.

Für jeden beliebigen Wert von $y$ im gegebenen Intervall ist die
Konvergenzgeschwindigkeit besser als
$\vartheta=3/7 = 0.42857142\dots$.
Die Anzahl Stellen, die man pro Iteration gewinnt ist daher 
$-\log\vartheta = 0.36798$.
Diese Konvergenzgeschwindigkeit erreicht man an den Enden des Intervalls.
Dazwischen kann man, wie oben gezeigt, deutlich schnellere Konvergenz haben.
\end{beispiel}

