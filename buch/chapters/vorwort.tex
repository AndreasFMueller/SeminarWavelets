%
% vorwort.tex -- Vorwort zum Buch zum Seminar
%
% (c) 2019 Prof Dr Andreas Mueller, Hochschule Rapperswil
%
\chapter*{Vorwort}
\lhead{Vorwort}
\rhead{}
Dieses Buch entstand im Rahmen des Mathematischen Seminars
im Frühjahrssemester 2019 an der Hochschule für Technik Rapperswil.
Die Teilnehmer, Studierende der Abteilungen für Elektrotechnik,
Informatik und Bauingenieurwesen der
HSR, erarbeiteten nach einer Einführung in das Themengebiet jeweils
einzelne Aspekte des Gebietes in Form einer Seminararbeit, über
deren Resultate sie auch in einem Vortrag informierten. 

Im Frühjahr 2019 war das Thema des Seminars die Theorie und die Anwendungen
von Wavelets.
Der Fokus lag auf einer knappen aber mathematisch vollständigen 
Darstellung der Theorie mit besonderer Berücksichtigung von Frames
und Multiskalen-Analysen.

Der Begriff des {\em Frames} erweitert die Idee der Basis auf eine
für die Praxis besonders nützliche Weise.
Die manchmal schwierig zu erreichende lineare Unabhängigkeit der
Basisvektoren und damit die Eindeutigkeit der Zerlegung wird aufgegeben
und ersetzt durch eine Bedingung, mit der sich immer noch eindeutige
Rekonstruktion erreichen lässt.

Eine {\em Multiskalen-Analyse} fasst die Eigenschaften einer Wavelet-Basis
in eine abstrakten Form zusammen.
Mit Hilfe der Fourier-Theorie können daraus konkrete Eigenschaften der
Wavelet-Funk\-tionen rekonstruiert werden.
Besonders wichtig für die Praxis ist aber, dass sich aus der
Multiskalen-Analyse ohne grossen Aufwand ein schneller Analyse und
Synthese-Algorithmus ableiten lässt.
Die Daubechies-Wavelets mit kompaktem Träger ergeben sich auf ganz
natürliche Weise aus dieser Theorie.

Im zweiten Teil dieses Skripts kommen dann die Teilnehmer selbst zu Wort.
Ihre Arbeiten wurden jeweils als einzelne
Kapitel mit meist nur typographischen Änderungen übernommen.
Diese weiterführenden Kapitel sind sehr verschiedenartig.
Eine Übersicht und Einführung findet sich in der Einleitung
zum zweiten Teil auf Seite~\pageref{buch:uebersicht}.

In einigen Arbeiten wurde auch Code zur Demonstration der 
besprochenen Methoden und Resultate geschrieben, soweit
möglich und sinnvoll wurde dieser Code im Github-Repository
dieses Kurses%
\footnote{\url{https://github.com/AndreasFMueller/SeminarWavelets.git}}
\cite{buch:repo}
abgelegt.

Im genannten Repository findet sich auch der Source-Code dieses
Skriptes, es wird hier unter einer Creative Commons Lizenz
zur Verfügung gestellt.




