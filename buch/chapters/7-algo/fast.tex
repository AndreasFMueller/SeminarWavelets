%
% fast.tex
%
% (c) 2019 Prof Dr Andreas Müller, Hochschule Rapperswil
%
\section{Schnelle Algorithmen
\label{section:fast}}
\rhead{Schnelle Algorithmen}
Die Basisfunktion $\psi_{j,k}$ geben die Details eines Signals nahe bei $t=k2^{-j}$ 
mit einer Auflösung von der Grössernordnung $2^{-j}$ wieder.
Da die $\psi_{j,k}$ nur eine Basis von $W_j$ sind, werden ausserdem die
Basisfunktion $\varphi_{j,k}$ in mindestens einem der Vektorräume
$V_j$ gebraucht.
Ein praktisch nützlicher Analyse-Algorithmus muss die Koeffizienten
\[
\begin{aligned}
a_{j,k} &= \langle f, \varphi_{j,k}\rangle
&&\text{und}&
b_{j,k} &= \langle f, \psi_{j,k}\rangle
\end{aligned}
\]
für jedes beliebige Signal $f$ auf effiziente Art berechnen können.
Dabei wird verwendet, dass die Funktionen  $\varphi_{j,k}$ eine
orthonormierte Basis von $V_j$ sind und die $\psi_{j,k}$ eine
orthonormierte Basis von $W_j$.

Ein Schlüsselelement einer Multiskalen-Analyse ist die Skalierungsrelation
\begin{align*}
\varphi(t) &= \frac{1}{\sqrt{2}} \sum_{l\in\mathbb Z} h_l \varphi(2t-k)
\\
\varphi    &= \frac{1}{\sqrt{2}} \sum_{l\in\mathbb Z} h_l D_{\frac12}T_k\varphi.
\end{align*}
Da die Skalierung $D_{\frac12}$ jeden der Vektorräume $V_j$ isometrisch auf $V_{j+1}$
abbildet, wird daraus eine Skalierungsrelation für alle Basisfunktionen $\varphi_{j,k}$,
die
\begin{equation}
\varphi_{j+1,k}
=
\sum_{l\in\mathbb Z} h_l D_{\frac12}T_{k+l}\varphi
\label{fast:phirelation}
\end{equation}
geschrieben werden kann.
Ausserdem folgt aus der Tatsache, dass $W_j\subset V_{j+1}$ ist, dass die Basisfunktionen
$\psi_{j,k}$ der Orthonormalbasis von $W_j$ Linearkombinationen der Funktionen
$\varphi_{j+1,l}$ sein müssen, folgt die Relation
\[
\psi(t) = \frac{1}{\sqrt{2}}\sum_{l\in\mathbb Z} g_l \varphi(2t-k),
\]
für das Mutter-Wavelet.
Wieder mit Hilfe der Skalierungs-Isometrien $D_{\frac12}$ folgt eine entsprechende
Relation
\begin{equation}
\psi_{j,k}
=
\sum_{l\in\mathbb Z} g_l D_{\frac12}T_{k+l}\varphi
\label{fast:psirelation}
\end{equation}
für die skalierten und verschoebenen Wavelets.

Wendet man die Relationen \eqref{fast:phirelation} und \eqref{fast:psirelation}
auf das Signal $f$ an, findet man Beziehungen für die
Koeffizienten $b_{j,k}$ und $a_{j,k}$.
\begin{align}
a_{j+1,k}
=
\langle f,\varphi_{j+1,k} \rangle
&=
\sum_{l\in\mathbb Z} \bar{h}_l \langle f,\varphi_{j,k+l}\rangle
=
\sum_{l\in\mathbb Z} \bar{h}_l a_{j,k+l}
\label{fast:akoefgleichung}
\\
b_{j+1,k}
=
\langle f,\psi_{j+1,k} \rangle
&=
\sum_{l\in\mathbb Z} \bar{g}_l \langle f,\varphi_{j,k+l}\rangle
=
\sum_{l\in\mathbb Z} \bar{g}_l a_{j,k+l}
\label{fast:bkoefgleichung}
\end{align}
Die Formeln drücken die Wavelet-Koeffizienten für die Auflösung $j$ durch
die Wavelet-Koeffizienten für die Auflösung $j+1$ aus.
Sobald die Koeffizienten $a_{N,k}$ bekannt sind, können alle Koeffizienten
$a_{j,k}$ mit $j<N$, berechnet werden.

Bleibt die Frage zu klären, woher denn die Koeffizienten $a_{N,k}$ in einer
praktischen Anwendung der Wavelet-Transformation kommen.
Für grosses $j$ ist die Funktion $\varphi_{j,k}$ in der Nähe von $2^{-j}k$
konzentriert.
Ändert das Signal $f$ in einer Umgebung von $2^{-j}k$ nur wenig, kann man
\begin{equation*}
a_{j,k}
=
\langle f,\varphi_{j,k}\rangle
=
\int_{-\infty}^\infty f(t)\, \varphi_{j,k}(t)\,dt
\simeq
f(2^{-j}k)
\int_{-\infty}^\infty \varphi_{j,k}(t)\,dt
\end{equation*}
approximieren.
Bis auf einen möglichen skalaren Faktor gegeben durch das Integral, sind
die Koeffizienten $a_{j,k}$ für genügend grosses $j$ also durch
Sample-Werte $f(2^{-j}k)$ ausreichend genau approximiert.

\begin{beispiel}
Die Skalierungsrelation des Haar-Wavelets hat die Koeffizienten
$h_0=h_1=1$, $g_0=1$ und $g_1=-1$.
Die Berechnung der Koeffizienten gemäss \eqref{fast:akoefgleichung}
und \eqref{fast:bkoefgleichung} führt auf das folgende Berechnungsschema:
\[
\xymatrix{
\dots
	&a_{j,-2} \ar[d] \ar[dr]
		&a_{j,-1} {\color{red}\ar[d]^{\cdot (-1)}} \ar[dl]
			&a_{j,0} \ar[d] \ar[dr]
				&a_{j,1} {\color{red}\ar[d]^{\cdot (-1)}} \ar[dl]
					&a_{j,2} \ar[d] \ar[dr]
						&a_{j,3} {\color{red}\ar[d]^{\cdot (-1)}} \ar[dl]
							&\dots
\\
\dots
	&a_{j-1,-1} {\color{red}\ar[d]^{\cdot(-1)}}
		&b_{j-1,-1}
			&a_{j-1,0}\ar[d] \ar[drr]
				&b_{j-1,0}
					&a_{j-1,1} {\color{red}\ar[d]^{\cdot (-1)}} \ar[dll]
						&b_{j-1,1}
							&\dots
\\
\dots &b_{j-2,-1} 
		&
			&a_{j-2,0} \ar[d] \ar[drrrr]
				&
					&b_{j-2,0}
						&
							&\dots \ar[dllll]
\\
\dots &
		&
			&a_{j-3,0}
				&
					&
						&
							&\dots
}
\]
Wo immer zwei Pfeile zusammenkommen müssen die Terme, von denen die Pfeile ausgehen, 
summiert werden. Rote Pfeile multpilizieren die Ausgangsterme zusätzlich mit $-1$.
\end{beispiel}




