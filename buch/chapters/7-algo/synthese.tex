%
% synthese.tex -- synthese 
%
% (c) 2019 Prof Dr Andreas Müller, Hochschule Rapperswil
%
\section{Schnelle Synthese\label{section:schnelle-synthese}}
\rhead{Schnelle Synthese}
Im vorangegangenen Abschnitt wurde dargestellt, wie die Waveletkoeffizienten
$a_{j,k}$ mit einem schnellen Algorithmus aus den Waveletkoeffizienten
$a_{j+1,k}$ gewonnen werden können.
Dabei spielte die Skalierungsrelation für das Vaterwavelet $\varphi$ und die
Darstellung des Mutterwavelet $\psi$ als Linearkombination von $\varphi_{1,k}$
die entscheidende Rolle.
Die Orthogonalität der Basisvektoren war bisher nicht von Bedeutung.

Die Umkehrung einer linearen Transformation erfordert die
Invertierung einer Matrix, ein im Allgmeinen aufwendiger Prozess.
Doch für orthonormierte Basen ist die Umkehrung einfach.
In diesem Falle ist die Koordinationentransformationsmatrix
orthogonal und die inverse Matrix ist die Transponierte.
Da die aus einer Multiskalen-Analyse hervorgegangenen Wavelet-Basen
orthonormiert sind, erwarten wir daher eine ebenfalls leicht zu
berechnende schnelle Umkehrtransformation.
Sie soll im Folgenden dargestellt werden.

Die schnelle Wavelet-Transformation erzeugt die Detail-Koeffizienten
$b_{jk}$ aus den Samples $a_{Nk}$.
Dies entspricht einem Basiswechsel von $\varphi_{N,k}$ zu den
Vektoren $\psi_{j,k}$ für $j<N$.
Wir wollen diesen Basiswechsel als Transformationsmatrix für 
die Waveletkoeffizienten ausdrücken und schliesslich auch invertieren.

%
% Basiswechsel zwischen orthonormierten Basen
%
\subsection{Basiswechsel und Koordinatentransformationsmatrix}
Zunächst erinnern wir an die abstrakte Konstruktion der
Transformationsmatrix für die Koordinaten bei einem Basiswechsel.
Zu diesem Zweck seien in einem Vektorraum $V$ zwei Basen
$\mathcal{B}=\{b_i\in V\}$ und $\mathcal{C}=\{c_i\in V\}$ vorgegeben.
Gesucht wird die Umrechnung von Koordinaten bezüglich der Basis
$\mathcal{B}$ in Koordinaten bezüglich der Basis $\mathcal{C}$.

Da $\mathcal{C}$ eine Basis ist, lassen sich die Vektoren $b_i$ als
Linearkombinationen
\begin{equation}
b_i = \sum_{j}  t_{ji}c_j
\label{algo:basiswechsel}
\end{equation}
der $c_j$ schreiben.
Ein beliebiger Vektor $v\in V$ habe in der Basis $\mathcal{B}$ die
Koordinaten $\xi_i$, er lässt sich also als Linearkombination
\[
v=\sum_{i} \xi_i b_i
\]
schreiben.
Unter Verwendung der Koeffizienten $t_{ji}$ für den Basiswechsel
gemäss~\eqref{algo:basiswechsel} lassen
sich die Koordinaten von $v$ in der Basis $\mathcal{C}$ finden:
\[
\sum_{i} \xi_i b_i
=
\sum_{i,j} \underbrace{t_{ji}\xi_i}_{\displaystyle=\eta_j} c_j
\qquad\Rightarrow\qquad
\eta_j = \sum_{i} t_{ji} \xi _i.
\]
Die Notation ist so gewählt, dass die Matrix $T$ mit Einträgen $t_{ji}$ 
den Vektor $\xi$ der Koordinaten $\xi_i$ mittels eines Produktes $T\xi$
in den Vektor $\eta$ der Koordinaten $\eta_j$ der Koordinaten bezüglich
der Basis $\mathcal{C}$ transformiert.

\subsection{Transformationsmatrix für Wavelet-Koeffizienten}
Im Wavelet-Kontext liefert die Skalierungs-Relation den Zusammenhang
zwischen den verschiedenen Basen.
Die Multiskalen-Analyse verspricht, dass sich jeder der Vektorräume
$V_{j+1}$ auf zwei Arten mit einer Basis ausstatten lässt.
Einerseits bilden die Vektoren $D_{j+1}T_k\varphi$ eine orthonormierte
Basis von $V_{j+1}$.
Andererseits lässt sich $V_{j+1}$ schreiben als orthogonale Summe
$V_j\oplus W_j$, in der jeder Summand eine orthonormierte Basis hat.
In $V_j$ ist des die Basis der Vektoren $D_jT_k\varphi$, in $W_j$
sind es die $D_jT_k\psi$.

Die Skalierungsrelation für das Vaterwavelet und die Darstellung des
Mutterwavelets liefern die Beziehungen
\begin{align*}
\varphi_{j,k} &= \sum_{l} h_{l}\varphi_{j+1,l+2k}
\qquad\text{und}
\\
\psi_{j,k} &= \sum_{l} g_{l} \varphi_{j+1,l+2k}.
\end{align*}

Im Abschnitt~\ref{section:fast} haben wir daraus die Formeln für
die schnelle Wavelet-Transformation abgeleitet:
\begin{align*}
a_{j,k} &= \sum_{l} \bar{h}_l a_{j+1,l+2k} \qquad\text{und}
\\
b_{j,k} &= \sum_{l} \bar{g}_l a_{j+1,l+2k}
\end{align*}
abgeleitet.
Die Koeffizienten $a_{\cdot,\cdot}$ und $b_{\cdot,\cdot}$ entsprechen
den Koordinaten $\xi_{\cdot}$ und $\eta_{\cdot}$ weiter oben.

An der Verschiebungsposition $k$ sind in der Basis von $V_{j+1}$ 
primär die Koffizienten um $a_{j+1,2k}$ herum massgebend.
In $V_j$ und $W_j$ sind es dagegeben die Koeffizienten $a_{j,k}$ und
$b_{j,k}$.
Es liegt daher nahe, die Basisvektoren in $V_j$ und $W_j$ zu mischen
und die Reihenfolge
\[
\dots    
\varphi_{j,k-1},\psi_{j,k-1},
\varphi_{j,k},\psi_{j,k},
\varphi_{j,k+1},\psi_{j,k+1},
\varphi_{j,k+2},\psi_{j,k+2},
\dots
\]
zu verwenden.
Für die Umrechnung der Koeffzienten $a_{j+1,k}$ in $a_{j,k}$ und $b_{j,k}$
ist dann die folgende Matrix anzuwenden:
\begin{equation}
\bgroup
\def\arraystretch{1.5}
\begin{tabular}{>{$}c<{$}|
>{$}c<{$}
>{$}c<{$}
>{$}c<{$}
>{$}c<{$}
>{$}c<{$}
>{$}c<{$}
>{$}c<{$}
>{$}c<{$}
>{$}c<{$}
>{$}c<{$}|}
        &\dots &a_{j+1,-2}  &a_{j+1,-1}  &a_{j+1,0}   &a_{j+1,1}   &a_{j+1,2}   &a_{j+1,3}   &a_{j+1,4}&a_{j+1,5}&\dots \\
\hline
\vdots  &\ddots&\vdots      &\vdots      &\vdots      &\vdots      &\vdots      &\vdots      &\vdots   &\vdots   &\ddots\\
a_{j,-2}&\dots &\bar{h}_2   &\bar{h}_3   &\bar{h}_4   &\bar{h}_5   &\bar{h}_6   &\bar{h}_7   &\bar{h}_8&\bar{h}_9&\dots \\
b_{j,-2}&\dots &\bar{g}_2   &\bar{g}_3   &\bar{g}_4   &\bar{g}_5   &\bar{g}_6   &\bar{g}_7   &\bar{g}_8&\bar{g}_9&\dots \\
a_{j,-1}&\dots &\bar{h}_0   &\bar{h}_1   &\bar{h}_2   &\bar{h}_3   &\bar{h}_4   &\bar{h}_5   &\bar{h}_6&\bar{h}_7&\dots \\
b_{j,-1}&\dots &\bar{g}_0   &\bar{g}_1   &\bar{g}_2   &\bar{g}_3   &\bar{g}_4   &\bar{g}_5   &\bar{g}_6&\bar{g}_7&\dots \\
a_{j, 0}&\dots &\bar{h}_{-2}&\bar{h}_{-1}&\bar{h}_0   &\bar{h}_1   &\bar{h}_2   &\bar{h}_3   &\bar{h}_4&\bar{h}_5&\dots \\
b_{j, 0}&\dots &\bar{g}_{-2}&\bar{g}_{-1}&\bar{g}_0   &\bar{g}_1   &\bar{g}_2   &\bar{g}_3   &\bar{g}_4&\bar{g}_5&\dots \\
a_{j, 1}&\dots &\bar{h}_{-4}&\bar{h}_{-3}&\bar{h}_{-2}&\bar{h}_{-1}&\bar{h}_0   &\bar{h}_1   &\bar{h}_2&\bar{h}_3&\dots \\
b_{j, 1}&\dots &\bar{g}_{-4}&\bar{g}_{-3}&\bar{g}_{-2}&\bar{g}_{-1}&\bar{g}_0   &\bar{g}_1   &\bar{g}_2&\bar{g}_3&\dots \\
a_{j, 2}&\dots &\bar{h}_{-6}&\bar{h}_{-5}&\bar{h}_{-4}&\bar{h}_{-3}&\bar{h}_{-2}&\bar{h}_{-1}&\bar{h}_0&\bar{h}_1&\dots \\
b_{j, 2}&\dots &\bar{g}_{-6}&\bar{g}_{-5}&\bar{g}_{-4}&\bar{g}_{-3}&\bar{g}_{-2}&\bar{g}_{-1}&\bar{g}_0&\bar{g}_1&\dots \\
\vdots  &\ddots&\vdots      &\vdots      &\vdots      &\vdots      &\vdots      &\vdots      &\vdots   &\vdots   &\ddots\\
\hline
\end{tabular}
\egroup
\label{fast:transmatrix}
\end{equation}
%
% invertierung der Transformation
%
\subsection{Invertierung}
Wir nehmen jetzt zusätzlich an, dass die Basen $\mathcal{B}$ und $\mathcal{C}$
orthonormiert sind, eine Voraussetzung die für die skalierten Mutter- und
Vaterwavelets einer Multiskalenanalyse erfüllt ist.
Es gilt also
\[
\begin{aligned}
\langle b_i,b_j\rangle &=\delta_{ij}\;\forall i,j
&&\text{und}
&
\langle c_i,c_j\rangle &=\delta_{ij}\;\forall i,j.
\end{aligned}
\]
Auch diese Bedingung ist für die Basen, die aus einer Multiskalen-Analyse hervorgehen, erfüllt.
Die Koeffizienten $t_{ji}$ waren definiert durch die
Formel~\eqref{algo:basiswechsel}.
Die Orthogonalitätsbedingungen für die Vektoren $b_i$ werden damit
\[
\langle b_i,b_k\rangle
=
\biggl\langle
\sum_{j}t_{ji}c_j,\sum_{l}t_{lk}c_l
\biggr\rangle
=
\sum_{j,l} t_{ji}\bar{t}_{lk}
\underbrace{\langle c_j,c_l\rangle}_{\displaystyle=\delta_{jl}}
=
\sum_{j} t_{ji}\bar{t}_{jk}.
\]
In Matrix-Schreibweise ist dies
\[
E
=
\overline{T}^tT
\]
oder mit der Abkürzung $T^*$ für die Matrix mit dem $k$-$j$-Eintrag
$\bar{t}_{jk}$ (konjugiert und transponierte Matrix) 
\[
T^*T=E.
\]
Daraus kann man ablesen, dass $T^*$ die inverse Matrix zu $T$ ist.
Wie erwartet ist die Bestimmung der inversen Matrix für die Transformation 
zwischen orthonormierten Basen einfach.

\subsection{Schnelle Wavelet-Synthese}
Für die schnelle Wavelet-Transformation kann mit Hilfe der Matrix~\eqref{fast:transmatrix}
geschrieben werden.
Die Umkehrtransformation muss daher die transponierte und konjugierte Matrix verwenden:
\begin{equation}
\bgroup
\def\arraystretch{1.5}
\begin{tabular}{>{$}c<{$}|
>{$}c<{$}
>{$}c<{$}
>{$}c<{$}
>{$}c<{$}
>{$}c<{$}
>{$}c<{$}
>{$}c<{$}
>{$}c<{$}
>{$}c<{$}
>{$}c<{$}
>{$}c<{$}
>{$}c<{$}|}
          &\dots &a_{j,-2}&b_{j,-2}&a_{j,-1}&b_{j,-1}&a_{j,0}&b_{j,0}&a_{j,1}&b_{j,1}&a_{j,2}&b_{j,2}&\dots \\
\hline
\vdots    &\ddots&\vdots  &\vdots  &\vdots  &\vdots  &\vdots &\vdots &\vdots &\vdots &\vdots &\vdots &\ddots\\
a_{j+1,-2}&\dots &h_{ 2}  &g_{ 2}  &h_{ 0}  &g_{ 0}  &h_{-2} &g_{-2} &h_{-4} &g_{-4} &h_{-6} &g_{-6} &\dots \\
a_{j+1,-1}&\dots &h_{ 3}  &g_{ 3}  &h_{ 1}  &g_{ 1}  &h_{-1} &g_{-1} &h_{-3} &g_{-3} &h_{-5} &g_{-5} &\dots \\
a_{j+1, 0}&\dots &h_{ 4}  &g_{ 4}  &h_{ 2}  &g_{ 2}  &h_{ 0} &g_{ 0} &h_{-2} &g_{-2} &h_{-4} &g_{-4} &\dots \\
a_{j+1, 1}&\dots &h_{ 5}  &g_{ 5}  &h_{ 3}  &g_{ 3}  &h_{ 1} &g_{ 1} &h_{-1} &g_{-1} &h_{-3} &g_{-3} &\dots \\
a_{j+1, 2}&\dots &h_{ 6}  &g_{ 6}  &h_{ 4}  &g_{ 4}  &h_{ 2} &g_{ 2} &h_{ 0} &g_{ 0} &h_{-2} &g_{-2} &\dots \\
a_{j+1, 3}&\dots &h_{ 7}  &g_{ 7}  &h_{ 5}  &g_{ 5}  &h_{ 3} &g_{ 3} &h_{ 1} &g_{ 1} &h_{-1} &g_{-1} &\dots \\
a_{j+1, 4}&\dots &h_{ 8}  &g_{ 8}  &h_{ 6}  &g_{ 6}  &h_{ 4} &g_{ 4} &h_{ 2} &g_{ 2} &h_{ 0} &g_{ 0} &\dots \\
a_{j+1, 5}&\dots &h_{ 9}  &g_{ 9}  &h_{ 7}  &g_{ 7}  &h_{ 5} &g_{ 5} &h_{ 3} &g_{ 3} &h_{ 1} &g_{ 1} &\dots \\
\vdots    &\ddots&\vdots  &\vdots  &\vdots  &\vdots  &\vdots &\vdots &\vdots &\vdots &\vdots &\vdots &\ddots\\
\hline
\end{tabular}
\egroup
\end{equation}
Daraus kann man ablesen, wie die Koeffizienten $a_{j+1,k}$ aus $a_{j,k}$ und $b_{j,k}$
berechnet werden können.

\begin{satz}
Die Umkehrtransformation der schnellen Wavelet-Transformation berechnet die hochfrequenten
Wavelet-Koeffizienten $a_{j+1,k}$ aus den Koeffizienten
$a_{j,k}$ und $b_{j,k}$ gemäss
\begin{equation}
\begin{aligned}
a_{j+1,k}
&=
\dots
+
h_{k+4}
a_{j,-2}
+
g_{k+4}
b_{j,-2}
+
h_{k+2}
a_{j,-1}
+
g_{k+2}
b_{j,-1}
+
h_{k}
a_{j,0}
+
g_{k}
b_{j,0}
+
h_{k-2}
a_{j,1}
+
g_{k-2}
b_{j,1}
+
\dots
\\
&=
\sum_{l\in\mathbb Z}
h_{k-2l}
a_{j,l}
+
\sum_{l\in\mathbb Z}
g_{k-2l}
b_{j,l}.
\end{aligned}
\end{equation}
\end{satz}

\begin{beispiel}
Die Koeffizienten $h_{\cdot}$ und $g_{\cdot}$ für das Haar-Wavelet sind
wohlbekannt.
Sie sind $h_0=h_1=1$, $g_0=0$ and $g_1=-1$.
Die Transformations-Matrix wird damit zu
\begin{equation}
\bgroup
\def\arraystretch{1.5}
\begin{tabular}{>{$}c<{$}|
>{$}c<{$}
>{$}c<{$}
>{$}c<{$}
>{$}c<{$}
>{$}c<{$}
>{$}c<{$}
>{$}c<{$}
>{$}c<{$}
>{$}c<{$}
>{$}c<{$}
>{$}c<{$}
>{$}c<{$}|}
          &\dots &a_{j,-2}&b_{j,-2}&a_{j,-1}&b_{j,-1}&a_{j,0}&b_{j,0}&a_{j,1}&b_{j,1}&a_{j,2}&b_{j,2}&\dots \\
\hline
\vdots    &\ddots&\vdots  &\vdots  &\vdots  &\vdots  &\vdots &\vdots &\vdots &\vdots &\vdots &\vdots &\ddots\\
a_{j+1,-2}&\dots &    0   &    0   &    1   &    1   &    0  &    0  &    0  &    0  &    0  &    0  &\dots \\
a_{j+1,-1}&\dots &    0   &    0   &    1   &   -1   &    0  &    0  &    0  &    0  &    0  &    0  &\dots \\
a_{j+1, 0}&\dots &    0   &    0   &    0   &    0   &    1  &    1  &    0  &    0  &    0  &    0  &\dots \\
a_{j+1, 1}&\dots &    0   &    0   &    0   &    0   &    1  &   -1  &    0  &    0  &    0  &    0  &\dots \\
a_{j+1, 2}&\dots &    0   &    0   &    0   &    0   &    0  &    0  &    1  &    1  &    0  &    0  &\dots \\
a_{j+1, 3}&\dots &    0   &    0   &    0   &    0   &    0  &    0  &    1  &   -1  &    0  &    0  &\dots \\
a_{j+1, 4}&\dots &    0   &    0   &    0   &    0   &    0  &    0  &    0  &    0  &    1  &    1  &\dots \\
a_{j+1, 5}&\dots &    0   &    0   &    0   &    0   &    0  &    0  &    0  &    0  &    1  &   -1  &\dots \\
\vdots    &\ddots&\vdots  &\vdots  &\vdots  &\vdots  &\vdots &\vdots &\vdots &\vdots &\vdots &\vdots &\ddots\\
\hline
\end{tabular}
\egroup
\end{equation}
Damit werden die Umkehrformeln für das Haar-Wavelet zu
\begin{align*}
a_{j+1,2k\phantom{+1}}
&= 
a_{j,k} + b_{j,k}
\\
a_{j+1,2k         +1 }
&= 
a_{j,k} - b_{j,k}.
\end{align*}
In diesem Fall sind die Umkehrformeln also ähnlich einfach wie die Formeln
für die schnelle Vorwärts-Transformation.
\end{beispiel}

