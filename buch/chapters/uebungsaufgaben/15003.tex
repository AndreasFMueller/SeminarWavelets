Drücken Sie $\cos^3\varphi$ durch Kosinuswerte von $\varphi$ und seinen
Vielfachen aus.

\begin{loesung}
Wir berechnen die dritte Potenz von $z=e^{i\varphi}=\cos\varphi+i\sin\varphi$
auf zwei verschiedene Arten:
\begin{align*}
z^3
&=
\cos^3\varphi + 3i\cos^2\varphi\sin\varphi - 3 \cos\varphi\sin^2\varphi
-i\sin^3\varphi
\\
z^3
&=
e^{3i\varphi}=\cos3\varphi+i\sin3\varphi
\end{align*}
Die Real- und Imaginärteile müssen übereinstimmen:
\begin{align*}
\cos^3\varphi-3\cos\varphi\sin^2\varphi
&=
\cos3\varphi
&
3\cos^2\varphi\sin\varphi-\sin^3\varphi
&=
\sin3\varphi
\\
\cos^3\varphi-3\cos\varphi(1-\cos^2\varphi)
&=
\cos3\varphi
&
3(1-\sin^2\varphi)\sin\varphi-\sin^3\varphi
&=
\sin3\varphi
\\
\cos^3\varphi-3\cos\varphi + 3\cos^3\varphi
&=
\cos3\varphi
&
3\sin\varphi -\sin^3\varphi-\sin^3\varphi
&=
\sin3\varphi
\\
4\cos^3\varphi
&=
\cos3\varphi
+
3\cos\varphi
&
3\sin\varphi -4\sin^3\varphi
&=
\sin3\varphi
\\
\cos^3\varphi
&=
\frac34 \cos\varphi
+
\frac14 \cos3\varphi
&
\sin^3\varphi
&=
\frac34 \sin\varphi
-
\frac14\sin3\varphi.
\end{align*}
\end{loesung}

