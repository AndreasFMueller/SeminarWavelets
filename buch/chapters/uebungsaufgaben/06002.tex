Für eine Multiskalenanalyse gibt es immer Koeffizienten $g_k$,
die das Mutterwavelet linear aus skalierten Translaten des Vaterwavelets
$\varphi$ kombinieren:
\begin{equation}
\psi = \sum_{k\in\mathbb Z} D_{\frac12}T_k\varphi.
\label{06002:eq}
\end{equation}
Zeigen Sie, dass $\displaystyle\sum_{k\in\mathbb Z} g_k=0$ gilt.

\begin{loesung}
Da $\psi$ ein Wavelet ist, muss nach der Zulässigkeitsbedingung
\[
\int_{\mathbb R}\psi(t)\,dt
=
0
\]
sein.
Wir integrieren Gleichung~\eqref{06002:eq} und erhalten
\begin{align*}
0
=
\int_{\mathbb R}\psi(t)\,dt
&=
\sum_{k\in\mathbb Z} \biggl(g_k\int_{\mathbb R} (D_{\frac12}T_k \varphi)(t)\,dt\biggr)
\\
&=
\sum_{k\in\mathbb Z} \biggl(g_k \int_{\mathbb R} (D_{\frac12}\varphi)(t)\,dt\biggr)
=
\biggl(\sum_{k\in\mathbb Z} g_k\biggr)\cdot
\int_{\mathbb R} (D_{\frac12}\varphi)(t)\,dt.
\end{align*}
Die Integrale sind alle gleich und der Operator $T_k$ kann weggelassen
werden, weil das Lebesgue-Mass translationsinvariant ist.
Das Integral auf der rechten Seite verschwindet nicht, also muss die
Summe verschwinden:
\[
\sum_{k\in\mathbb Z} g_k = 0.
\]
Damit ist die Behauptung bewiesen.
\end{loesung}
