Auf der Website \url{http://wavelets.pybytes.com} findet man für das
Coiflet 1 Wavelet die folgenden Koeffizienten für die Skalierungsrelation:
\begin{align*}
h_0 &= -0.0156557281\\
h_1 &= -0.0727326195\\
h_2 &= \phantom{-} 0.3848648469\\
h_3 &= \phantom{-} 0.8525720202\\
h_4 &= \phantom{-} 0.3378976625\\
h_5 &= -0.0727326195
\end{align*}
\begin{teilaufgaben}
\item
Verwenden sie Aufgabe~\ref{06001} um zu verifizieren, ob für diese
Koeffizienten die gleiche Normierungskonvention verwendet wird wie in
diesem Buch.
\item
Verwenden Sie den schnellen Transformationsalgorithmus um eine 
erste Approximation der Funktionen $\varphi$ und $\psi$ zu bestimmen.
\end{teilaufgaben}

\begin{loesung}
\begin{teilaufgaben}
\item
Die Summe der Koeffiziente ist
\begin{align*}
\sum_{k\in\mathbb Z}h_k
&=
-0.0156557281 -0.0727326195 + 0.3848648469\\[-8pt]
&\phantom{=\mathstrut} + 0.8525720202 + 0.3378976625 -0.0727326195
\\
&=1.4142135625
\simeq\sqrt{2}.
\end{align*}
Die Koeffizienten verwenden also die gleiche Konvention.
\item
Für die Rücktransformation werden ausser den $h$-Koeffizienten auch die
$g$-Koeffizienten benötigt.
Diese sind
\begin{align*}
g_0 &= +0.0727326195\\
g_1 &= -0.3378976625\\
g_2 &= +0.8525720202\\
g_3 &= -0.3848648469\\
g_4 &= -0.0727326195\\
g_5 &= +0.0156557281\\
\end{align*}
Die Umkehrtransformation verwendet die Matrix 
\[
T^*
=
\begin{pmatrix}
&\vdots&\vdots&\vdots&\vdots&\vdots&\vdots&\\
\dots&h_0&g_0& 0 & 0 & 0 & 0 &\dots\\
\dots&h_1&g_1& 0 & 0 & 0 & 0 &\dots\\
\dots&h_2&g_2&h_0&g_0& 0 & 0 &\dots\\
\dots&h_3&g_3&h_1&g_1& 0 & 0 &\dots\\
\dots&h_4&g_4&h_2&g_2&h_0&g_0&\dots\\
\dots&h_5&g_5&h_3&g_3&h_1&g_1&\dots\\
\dots& 0 & 0 &h_4&g_4&h_2&g_2&\dots\\
\dots& 0 & 0 &h_5&g_5&h_3&g_3&\dots\\
&\vdots&\vdots&\vdots&\vdots&\vdots&\vdots&\\
\end{pmatrix}
\]
Daraus kann man eine erste Approximation for $\varphi$ dadurch bestimmen,
dass man mit einem Standardbasisvektor mit einer $1$ in einer Spalte
multipliziert, die $h$-Koeffizienten enthält.
Die Stufenfunktion mit den Werten $h_k$ ist daher eine erste Approximation 
für $\varphi$.
Analog ist das Produkt der Matrix $T^*$ mit einem Standardbasisvektor
mit einer $1$ in einer Spalte, die $g$-Koeffizienten enthält, eine erste
Approximation von $\psi$
\end{teilaufgaben}
\begin{figure}
\centering
\begin{tikzpicture}[>=latex,scale=2]
\draw[->,line width=0.7pt] (-0.1,0)--(6.2,0) coordinate[label={$t$}];
\draw[->,line width=0.7pt] (0,-0.7)--(0,1.3);
\draw[color=blue,line width=1pt] (0,-0.0156557281)--(1,-0.0156557281);
\draw[color=blue,line width=1pt] (1,-0.0727326195)--(2,-0.0727326195);
\draw[color=blue,line width=1pt] (2, 0.3848648469)--(3, 0.3848648469);
\draw[color=blue,line width=1pt] (3, 0.8525720202)--(4, 0.8525720202);
\draw[color=blue,line width=1pt] (4, 0.3378976625)--(5, 0.3378976625);
\draw[color=blue,line width=1pt] (5,-0.0727326195)--(6,-0.0727326195);
\draw[color=red,line width=1pt] (0,+0.0727326195)--(1,+0.0727326195);
\draw[color=red,line width=1pt] (1,-0.3378976625)--(2,-0.3378976625);
\draw[color=red,line width=1pt] (2,+0.8525720202)--(3,+0.8525720202);
\draw[color=red,line width=1pt] (3,-0.3848648469)--(4,-0.3848648469);
\draw[color=red,line width=1pt] (4,-0.0727326195)--(5,-0.0727326195);
\draw[color=red,line width=1pt] (5,+0.0156557281)--(6,+0.0156557281);
\input{chapters/7-algo/images/psi-coif.tex}
\input{chapters/7-algo/images/phi-coif.tex}
\end{tikzpicture}
\caption{Erste Approximation für das Coifflet 1 Vater-Wavelet (blau)
und das zugehörige Mutterwavelet (rot)
\label{07001:coifletimage}}
\end{figure}
\end{loesung}


