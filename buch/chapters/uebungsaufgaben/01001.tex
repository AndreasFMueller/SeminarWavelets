Betrachten Sie die Menge $\mathcal{B}=\{b_1,b_2,b_3,b_4\}$ bestehend aus 
den vier Vektoren
\[
b_1
=
\frac{1}{\sqrt{3}}
\begin{pmatrix}
1\\1\\1
\end{pmatrix},\quad
b_2
=
\frac{1}{\sqrt{3}}
\begin{pmatrix}
-1\\-1\\1
\end{pmatrix},\quad
b_3
=
\frac{1}{\sqrt{3}}
\begin{pmatrix}
1\\-1\\-1
\end{pmatrix}
\quad\text{und}\quad
b_4
=
\frac{1}{\sqrt{3}}
\begin{pmatrix}
-1\\1\\-1
\end{pmatrix}.
\]
\begin{teilaufgaben}
\item
Berechnen Sie die Wirkung des Gram-Operators $G$ für die Menge $\mathcal{B}$
auf den Standardbasisvektoren.
\item
Finden Sie die Matrix des Gram-Operators in der Standardbasis.
\item
Ist $\mathcal{B}$ ein Frame und wenn ja, wie gross sind die Frame-Konstanten?
Ist das Frame straff?
\end{teilaufgaben}

\begin{loesung}
\begin{teilaufgaben}
\item
Der Gram-Operator ist definiert durch
\[
Gx = \sum_{j=1}^4 \langle x,b_j\rangle b_j.
\]
Für $Ge_k$ sind also die Skalarprodukte $\langle e_k,b_j\rangle$
zu bestimmen.
Das Skalarprodukt $\langle e_k,b_j\rangle$ ist die $k$-te Komponente 
des Vektors $b_j$.
Man kann daher die Wirkung Gram-Operators direkt ablesen:
\begin{align*}
Ge_1
&=
\frac{1}{\sqrt{3}} b_1
-
\frac{1}{\sqrt{3}} b_2
+
\frac{1}{\sqrt{3}} b_3
-
\frac{1}{\sqrt{3}} b_4
=
\frac13
\begin{pmatrix}
4\\0\\0
\end{pmatrix}
=
\frac43 e_1
\\
Ge_2
&=
\frac{1}{\sqrt{3}} b_1
-
\frac{1}{\sqrt{3}} b_2
-
\frac{1}{\sqrt{3}} b_3
+
\frac{1}{\sqrt{3}} b_4
=
\frac13
\begin{pmatrix}
0\\4\\0
\end{pmatrix}
=
\frac43 e_2
\\
Ge_3
&=
\frac{1}{\sqrt{3}} b_1
+
\frac{1}{\sqrt{3}} b_2
-
\frac{1}{\sqrt{3}} b_3
-
\frac{1}{\sqrt{3}} b_4
=
\frac13
\begin{pmatrix}
0\\0\\4
\end{pmatrix}
=
\frac43 e_3
\end{align*}
\item
Aus den Resultaten von a) kann man ablesen, dass $G=\frac43*$.
Alternativ kann man die Gram-Matrix auch berechnen aus der Matrix 
\[
B
=
\frac{1}{\sqrt{3}}
\begin{pmatrix}
 1&-1& 1&-1\\
 1&-1&-1& 1\\
 1& 1&-1&-1
\end{pmatrix}.
\]
Die Skalarprodukte $\langle x,b_j\rangle$ für einen Vektor $x\in\mathbb R^3$
werden berechnet mit $xB^t$.
Die Linearkombination $Gx$ entsteht durch Linearkombination der vier Spalten
von $B$ mit Hilfe der vier Skalarprodukte bin $xB$.
Dazu müssen die Skalarprodukte als Spaltenvektor geschrieben werden, also
als $(x^tB)^t=B^tx$.
Somit ist
\[
Gx = BB^t x
\qquad\Rightarrow\qquad
G=BB^t
=
\frac{4}{3}
\begin{pmatrix}
1&0&0\\
0&1&0\\
0&0&1
\end{pmatrix}.
\]
\item
Da der Gram-Operator das $\frac43$-fache der Einheitsmatrix ist, ist
$\mathcal{B}$ sogar ein straffes Frame mit der Framekonstanten $A=B=\frac43$.
\qedhere
\end{teilaufgaben}
\end{loesung}

