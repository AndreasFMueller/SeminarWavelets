Betrachten Sie das Wavelet
\[
t\mapsto
\begin{cases}
\sin(\pi t)&\qquad -1 \le t\le 1\\
0&\qquad\text{sonst}
\end{cases}
\]
Mit diesem Wavelet soll das Signal
\[
f(t)
= 
\begin{cases}
1&\qquad 0\le t\le 1\\
0&\qquad\text{sonst}
\end{cases}
\]
analysiert werden.
\begin{teilaufgaben}
\item Rechnen Sie nach, dass $\|\psi\|=1$
\item Überprüfen Sie, dass $\int\psi\,dt=0$
\item Berechne Sie die stetige Wavelet-Transformation $\mathcal{W}f(\frac12,b)$.
\item Berechnen Sie die stetige Wavelet-Transformation $\mathcal{W}f(a,b)$
für $a\le\frac12$.
\item Berechne Sie die stetige Wavelet-Transformation $\mathcal{W}f(a,b)$
für $a >\frac12$.
\item Verifizieren Sie die Zulässigkeitsbedingung für das Wavelet $\psi$.
\end{teilaufgaben}

\begin{loesung}
\begin{figure}
\begin{tikzpicture}[>=latex,scale=2]
\draw[->,line width=0.7pt] (-3.1,0)--(3.3,0) coordinate[label={$t$}];
\draw[->,line width=0.7pt] (0,-1.1)--(0,1.3) coordinate[label={right:$f(t)$}];
\draw[line width=1pt,color=red] (-3,0)--(0,0);
\draw[line width=0.3pt,color=red] (0,0)--(0,1);
\draw[line width=1pt,color=red] (0,1)--(1,1);
\draw[line width=0.3pt,color=red] (1,0)--(1,1);
\draw[line width=1pt,color=red] (1,0)--(3,0);

\draw[color=blue,line width=1pt] (-3,0)--
	plot[domain=-180:180,samples=100] ({\x/180},{sin(\x)})
	--(3,0);
\end{tikzpicture}
\caption{Signal $f(t)$ (rot) und Wavelet $\psi$ (blau) für Aufgabe~4.1
\label{04001:funktionen}}
\end{figure}
Eine graphische Darstellung der Funktionen $f$ und $\psi$ ist in
Abbildung~\ref{04001:funktionen} zu finden.
\begin{teilaufgaben}
\item
Beweis durch Nachrechnen:
\begin{align*}
\|\psi\|^2
&=
\int_{-\infty}^\infty |\psi(t)|^2\,dt
=
\int_{-1}^1 \sin^2(\pi t)\,dt
=
\int_{-1}^1 \frac{1-\cos 2\pi t}{2}\,dt
\\
&=
\frac12\biggl[
t-\frac{1}{2\pi}\sin 2\pi t
\biggr]_{-1}^{1}
=\frac12(1-(-1))
=
1.
\end{align*}
Alternativ kann man aus Symmetriebetrachtungen erkennen, dass der Graph
von $\psi(t)^2$ das Rechteck mit Ecken $(-1,0)$ und $(1,1)$ halbiert.
Da das Rechteck den Flächeninhalt $2$ hat, folgt die Normierungsbedingung.
\item
Das Wavelet $\psi$ ist eine ungerade Funktion, daher verschwindet ihr
Integral.
\item
In diesem Fall ist das Wavelet
\[
\psi_{a,b}(t)
=
\psi_{\frac12,b}(t)
=
\begin{cases}
\sqrt{2}\sin 2\pi t&\qquad b-\frac12\le t\le b+\frac12
\\
0&\qquad\text{sonst.}
\end{cases}
\]
Es hat daher als Träger genau das Interval $[b-\frac12,b+\frac12]$.
Dies bedeutet, dass wir vier Fälle unterscheiden müssen:
\begin{enumerate}
\item
Fall $b < -\frac12$:  Der Träger von $\psi_{a,b}$ liegt ausserhalb des
intervals $[0,1]$, das Skalarprodukt
$\langle f,\psi_{a,b}\rangle=\mathcal{W}f(a,b)=0$ verschwindet.
\item
Fall $-\frac12\le b < 0$: In diesem Fall liegt das linke Ende des Trägers
von $\psi_{a,b}$ ausserhalb des Intervals $[0,1]$, das Integral zur Berechnung
von $\mathcal{W}f(a,b)$ ist also nur von $0$ bis $b+\frac12$ zu erstrecken:
\begin{align*}
\mathcal{W}f({\textstyle\frac12},b)
&=
\int_{0}^{b+\frac12} \sqrt{2} \sin 2\pi(t-b)\,dt
=
\biggl[
-
\frac{\sqrt{2}}{2\pi}
\cos 2\pi(t-b)
\biggr]_0^{b+\frac12}
\\
%&=
-\frac{\sqrt{2}}{2\pi}(\cos\pi - \cos(2\pi b))
=
\frac{\sqrt{2}}{2\pi}(1+\cos(2\pi b))
=
\frac{\sqrt{2}}{\pi}
\cos^2 \pi b.
\end{align*}
\item
Fall $0\le b \le \frac12$: In diesem Fall ist das Integral nur von $b-\frac12$
zu $1$ zu erstrecken:
\begin{align*}
\mathcal{W}f({\textstyle\frac12},b)
&=
\int_{b-\frac12}^1 \cos 2\pi (t-b)\,dt
=
\biggl[
-\frac{\sqrt{2}}{2\pi}\cos 2\pi(t-b)
\biggr]_0^{b-\frac12}
\\
&=
-\frac{\sqrt{2}}{2\pi}(
\cos 2\pi(1-b)
-
\cos \pi
)
=
-\frac{\sqrt{2}}{2\pi}(1+\cos 2\pi(1-b)).
=
-\frac{\sqrt{2}}{\pi}\cos^2\pi(1-b)
\end{align*}
\item
Fall $b > \frac12$:
Wie im ersten Fall verschwindet die Wavelet-Transformation auch in diesem
Fall.
\end{enumerate}
Der Graph von $b\mapsto \mathcal{W}f(\frac12,b)$ ist in
Abbildung~\ref{04001:wavelet2} dargestellt.
\begin{figure}
\centering
\begin{tikzpicture}[>=latex,scale=4]
\draw[->,line width=0.7pt] (-1.1,0)--(2.1,0) coordinate[label={$b$}];
\draw[->,line width=0.7pt] (0,-0.6)--(0,0.6) coordinate[label={right:$\mathcal{W}f(a,b)$}];
\draw[color=red,line width=1pt] (-1.1,0)--(-0.5,0)--
	plot[domain=-0.5:0.5,samples=100] ({\x},{sqrt(2)*(1+cos(2*\x*180))/(2*3.1415)})
	--
	plot[domain=0.5:1.5,samples=100] ({\x},{-sqrt(2)*(1+cos(2*(1-\x)*180))/(2*3.1415)})
	--(2.05,0)
;
\draw[line width=0.7pt] (-1,-0.025)--(-1,0.025);
\draw[line width=0.7pt] (1,-0.025)--(1,0.025);
\draw[line width=0.7pt] (2,-0.025)--(2,0.025);
\node at (-1,-0.025) [below] {$-1$};
\node at (1,-0.025) [below] {$1$};
\node at (2,-0.025) [below] {$2$};
\end{tikzpicture}
\caption{Werte der stetige Wavelet-Transformation $\mathcal{W}(\frac12,b)$
für das Signal $f$ und das Signal $\psi$ aus Abbildung~\ref{04001:funktionen}
\label{04001:wavelet2}}
\end{figure}
\item
Für $a\le\frac12$ wird das Wavelet von der $D_a$-Operation so stark gestaucht,
dass höchstens ein Endpunkt des Intervals $[0,1]$ im Träger der Funktion
$\psi_{a,b}$ liegen kann.
Der Träger des Wavelets liegt jetzt im Interval $[b-a,b+a]$.
Es gibt nun aber fünf Fälle zu berücksichtigen
\begin{enumerate}
\item Fall $b\le -a$:
Die Träger von $\psi_{a,b}$ und $f$ schneiden
sich nicht, $\mathcal{W}f(a,b)=0$.
\item Fall $-a\le b \le a$:
Wie im Fall $a=\frac12$ berechnet man
\begin{align*}
\mathcal{W}f(a,b)
&=
\int_{0}^{b+a} \frac{1}{\sqrt{|a|}} \sin \pi(t-b)/a\,dt
=
\biggl[
-
\frac{a}{\pi\sqrt{|a|}}
\cos \pi(t-b)/a
\biggr]_0^{b+a}
\\
&=
-\frac{a}{\pi\sqrt{|a|}}(\cos\pi - \cos(\pi b/a))
=
\frac{a}{\pi\sqrt{|a|}}(1+\cos(\pi b/a))
\\
&=
\frac{a}{2\pi\sqrt{|a|}}\cos^2\frac{\pi b}{2a}
\end{align*}
\item Fall $a\le b \le 1-a$:
Der Träger von $\psi_{a,b}$ liegt ganz innerhalb des Intervals $[0,1]$,
es wird also $\psi_{a,b}$ über den ganzen Träger integriert.
Dieses Integral verschwindet, also 
$\mathcal{W}f(a,b)=0$.
\item Fall $1-a\le b \le 1 +a$:
Wie im Fall $a=\frac12$ berechnet man
\begin{align*}
\mathcal{W}f(a,b)
&=
\int_{b-a}^{1} \frac{1}{\sqrt{|a|}} \sin \pi(t-b)/a\,dt
=
\biggl[
-
\frac{a}{\pi\sqrt{|a|}}
\cos \pi(t-b)/a
\biggr]_{b-a}^1
\\
&=
-\frac{a}{\pi\sqrt{|a|}}(
\cos(\pi (1-b)/a)
-
\cos\pi 
)
=
-\frac{a}{\pi\sqrt{|a|}}(1+\cos(\pi(1- b)/a))
\\
&=
-\frac{a}{2\pi\sqrt{|a|}}\cos^2\frac{\pi(1- b)}{2a}
\end{align*}
\item Fall $1+a\le b$:
Die Träger von $\psi_{a,b}$ und $f$ schneiden
sich nicht, $\mathcal{W}f(a,b)=0$.
\end{enumerate}
\item
\end{teilaufgaben}
\end{loesung}
