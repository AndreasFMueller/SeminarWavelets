Bilden sie die Vektoren von $\mathbb R^3$ deren Komponenten alle Permutationen
der Zahlen $1$, $2$ und $3$ sind.
Bilden diese Vektoren ein Frame?
Ist das Frame straff?

\begin{loesung}
Die Vektoren sind, geschrieben als Spalten einer Matrix,
\[
B
=
\begin{pmatrix}
1&2&1&2&3&3\\
2&1&3&3&1&2\\
3&3&2&1&2&1
\end{pmatrix}
\]
Der Gram-Operator ist
\[
G=BB^t
=
\begin{pmatrix}
28&22&22\\
22&28&22\\
22&22&28
\end{pmatrix}
\]
hat Rang $3$, also liegt ein Frame vor.
$G$ ist aber nicht ein Vielfaches der Einheitsmatrix, also ist das Frame
nicht straff.
Um die Frame-Konstanten zu bestimmen berechnen wir die Eigenwerte
mit Hilfe des charakteristischen Polynoms
\[
\chi_{G}(\lambda)
=
-\lambda^3 + 84\lambda^2 -900 \lambda + 2592.
\]
Das Polynom lässt sich faktorisieren in
\[
\chi_{G}(\lambda)
=
-(\lambda - 72)(\lambda - 6)^2.
\]
Daraus liest man die Frame-Konstanten $A=6$ und $B=72$ ab.
\end{loesung}

