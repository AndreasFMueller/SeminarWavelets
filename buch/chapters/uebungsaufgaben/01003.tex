Die zwölf Vektoren, die man aus 
\[
b_1=\begin{pmatrix} 0\\  1\\  \varphi \end{pmatrix},\quad
b_2=\begin{pmatrix} 0\\ -1\\  \varphi \end{pmatrix},\quad
b_3=\begin{pmatrix} 0\\  1\\ -\varphi \end{pmatrix}
\quad\text{und}\quad
b_4=\begin{pmatrix} 0\\ -1\\ -\varphi
\end{pmatrix},\quad
\]
durch zyklische Vertauschung der Komponenten erhalten kann, sind
die Ecken eines regulären Ikosaeders.
Dabei ist $\varphi=(1+\sqrt{5})/2$ das Verhältnis des goldenen
Schnittes.
Diese sind zum Beispiel
\[
\begin{aligned}
b_5&=\begin{pmatrix} 1\\\varphi\\0\end{pmatrix},&
b_6&=\begin{pmatrix}-1\\\varphi\\0\end{pmatrix},&
b_7&=\begin{pmatrix} 1\\-\varphi\\0\end{pmatrix},&
b_8&=\begin{pmatrix}-1\\-\varphi\\0\end{pmatrix},
\\
b_9   &=\begin{pmatrix} \varphi\\0\\ 1\end{pmatrix},&
b_{10}&=\begin{pmatrix} \varphi\\0\\-1\end{pmatrix},&
b_{11}&=\begin{pmatrix}-\varphi\\0\\ 1\end{pmatrix},&
b_{12}&=\begin{pmatrix}-\varphi\\0\\-1\end{pmatrix}.
\end{aligned}
\]
\begin{teilaufgaben}
\item
Bilden die Vektoren $\mathcal{B}=\{b_1,b_4,b_7,b_{10}\}$ ein Frame?
Wenn ja, berechnen Sie die Framekonstanten
\item
Die Vektoren $\mathcal{I}=\{b_1,\dots,b_{12}\}$ bilden ein straffes
Frame.
Berechnen Sie die Frame-Konstante.
\end{teilaufgaben}

\begin{loesung}
\begin{teilaufgaben}
\item
Wir berechnen den Gram-Operator aus der Matrix
\[
B=\begin{pmatrix}
      0&       0&       1& \varphi\\
      1&-      1&-\varphi&       0\\
\varphi&-\varphi&       0&-      1
\end{pmatrix}.
\]
Der Gram-Operator hat die Matrix
\[
G
=
BB^t
=
\begin{pmatrix}
   3.618034& -1.618034&-1.618034\\
  -1.618034&  4.618034& 3.236068\\
  -1.618034&  3.236068& 6.236068
\end{pmatrix}.
\]
Die Matrix hat Rang $3$, also ist $\mathcal{B}$ ein Frame, aber
weil $G$ nicht diagonal ist, ist das Frame nicht straff.
Die Frame-Konstanten sind die Werte des grössten und kleinsten Eigenwertes,
die man numerisch als
\[
A=2,\quad B=9.623122
\]
findet.
Man kann den Gram-Operator aber auch exakt finden.
Die Matrixmultiplikation mit Maxima ergibt den Gram-Operator in der
Form
\[
G=\begin{pmatrix}
\frac{\sqrt{5}+5}2 & -\varphi           & -\varphi    \\
-\varphi           & \frac{\sqrt{5}+7}2 & \sqrt{5} + 1\\
-\varphi           & \sqrt{5}+1         & \sqrt{5} + 4
\end{pmatrix}.
\]
Die Funktion \texttt{eigenvalues} von Maxima kann auch die Eigenwerte
in geschlossener Form finden:
\[
\lambda_1
=
2
<
\lambda_2
=
4+\sqrt{5} - \sqrt{2\sqrt{5}+7}
<
\lambda_3
=
4+\sqrt{5} + \sqrt{2\sqrt{5}+7}
\]
\item
Die Berechnung von $BB^t$ für die Matrix
\[
\setcounter{MaxMatrixCols}{12}
B=\begin{pmatrix}
      0&      0&       0&      0&
      1&     -1&       1&     -1&
\varphi&\varphi&-\varphi&\varphi
\\
      1&     -1&       1&     -1&
\varphi&\varphi&-\varphi&\varphi&
      0&      0&       0&      0
\\
\varphi&\varphi&-\varphi&\varphi&
      0&      0&       0&      0&
      1&     -1&       1&     -1
\end{pmatrix}
\]
mit Hilfe eines Computeralgebra-Systems wie zum Beispiel Maxima liefert
\[
G
=
(2\sqrt{5}+10)
\begin{pmatrix}
1&0&0\\
0&1&0\\
0&0&1
\end{pmatrix}.
\]
Daraus kann man ablesen, dass $\mathcal{I}$ ein straffes Frame mit 
Framekonstante $2\sqrt{5}+10$ ist.
\qedhere
\end{teilaufgaben}
\end{loesung}




