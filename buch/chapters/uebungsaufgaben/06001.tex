Verwenden Sie die Skalierungsrelation das Vaterwavelets um die
Summe der Koeffizienten $h_k$ zu berechnen.

\begin{loesung}
Die Skalierungsrelation ist
\[
\varphi(t) = \sqrt{2} \sum_{k\in\mathbb Z} h_k \varphi(2t-k)
\]
Wir analysieren das konstante Signal $1$ mit dem Vaterwavelet
\begin{align*}
\langle 1,\varphi\rangle
&=
\int_{-\infty}^\infty
\bar{\varphi}(t)\,dt
=
\sqrt{2}
\sum_{k\in\mathbb Z} 
h_k
\int_{-\infty}^\infty
\bar{\varphi}(2t-k)
\,dt
=
\sqrt{2}
\sum_{k\in\mathbb Z} 
h_k
\int_{-\infty}^\infty
\bar{\varphi}(\tau-k)
\frac12
\,d\tau
\\
&=
\frac{1}{\sqrt{2}}
\sum_{k\in\mathbb Z} 
h_k
\int_{-\infty}^\infty
\bar{\varphi}(\tau)
\,d\tau
=
\frac{1}{\sqrt{2}}
\langle 1,\varphi\rangle
\sum_{k\in\mathbb Z} h_k
\\
\sqrt{2}
&=
\sum_{k\in\mathbb Z} h_k.
\qedhere
\end{align*}
\end{loesung}


