Betrachten Sie die Menge $\mathcal{B}$ bestehend aus den vier Vektoren
\[
b_1
=
\frac{1}{\sqrt{3}}
\begin{pmatrix}
1\\1\\1
\end{pmatrix},\quad
b_2
=
\frac{1}{\sqrt{3}}
\begin{pmatrix}
-1\\-1\\1
\end{pmatrix},\quad
b_3
=
\frac{1}{\sqrt{3}}
\begin{pmatrix}
1\\-1\\1
\end{pmatrix}
\quad\text{und}\quad
b_4
=
\frac{1}{\sqrt{3}}
\begin{pmatrix}
-1\\1\\1
\end{pmatrix}.
\]
Bilden die Vektoren ein Frame?

\begin{loesung}
Wir berechnen wie in Aufgabe \ref{01001} den Gram-Operator aus der
Matrix
\[
B
=
\frac{1}{\sqrt{3}}
\begin{pmatrix}
1&-1& 1&-1\\
1&-1&-1& 1\\
1& 1& 1& 1
\end{pmatrix}.
\]
Der Gram-Operator ist
\[
G=BB^t
=
\frac43
\begin{pmatrix}
1&0&0\\
0&1&0\\
0&0&1
\end{pmatrix}
\]
Die Menge $\mathcal{B}$ ist also ein straffes Frame mit der Frame-Konstanten
$\frac43$.
Das Frame $\mathcal{B}$ ist bis auf die Vorzeichen einzelner Vektoren
identisch mit dem Frame von Aufgabe~\ref{01001}.
\end{loesung}


