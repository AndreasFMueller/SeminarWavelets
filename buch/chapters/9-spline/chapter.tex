%
% chapter.tex -- Kapitel über Spline Wavelets
%
% (c) 2019 Prof Dr Andreas Müller, Hochschule Rapperswil
%
\chapter{Spline-Wavelets
\label{chapter:spline}}
\lhead{Spline-Wavelets}
Das Haar-Wavelet ist charakterisiert durch die erzeugende Funktion $H_0$,
die in einer Funktionalgleichung der Fourier-Transformierten von $\varphi$
vorkommt.
Es stellt sich heraus, dass sich eine solche Funktionalgleichung für 
Faltungen von $\varphi$ mit sich selbst sofort ableiten lassen.
Allerdings bilden die verschobenen Kopien keine orthonormierte Basis,
dazu müssen sie erst orthonormiert werden.
Da sich dafür aber ein Verfahren angeben lässt, kann man eine Familie 
von Wavelets konstruieren, die sich dadurch auszeichnen, dass sie
stückweise polynomiell sind, genauso wie das Haar-Wavelet stückweise
konstant war.
Dies sind die B-Spline-Wavelets, die in diesem Kapitel dargestellt werden
sollen.

Die B-Spline-Wavelets haben nicht kompakten Träger, aber Koeffizienten
mit grossem Index sind sehr klein.
Ausserdem sind sie wie die Wavelet-Koeffizienten typischerweise irrational,
müssen also in einer Implementation notwendigerweise gerundet werden.
Die Stabilität der Wavelet-Transformation stellt sicher, dass die Rundung
nicht zu unbrauchbaren Resultaten führt.
Indem sehr kleine Koeffizienten zu $0$ gerundet werden, können auch die
B-Spline-Wavelets als Wavelets mit kompaktem Träger behandelt werden.
Dasselbe gilt natürlich auch für jede andere Form von Wavelet, bei dem
die Werte der Koeffizienten rasch gegen $0$ gehen.







