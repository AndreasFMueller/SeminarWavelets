%
% splines.tex
%
% (c) 2019 Prof Dr Andreas Müller, Hochschule Rapperswil
%
\section{Spline-Funktionen
\label{section:spline-funktionen}}
\rhead{Spline-Funktionen}
In diesem Abschnitt konstruieren wir ausgehend vom Haar-Wavelet mit Hilfe
der Faltung eine Familie $\varphi^{(n)}$ von Funktionen die zunehmend glatt
sind in dem Sinne, dass $\varphi^{(n)}$ für $n>1$ $n$ mal stetig
differenzierbar ist, und die alle eine Skalierungsrelation erfüllen.
Diese Funktionen werden wir dann im nächsten Abschnitt jeweils ein
Wavelet konstruieren.

\subsection{Das Haar-Wavelet
\label{subsection:spline:haar}}
Das Vater-Wavelet des Haar-Wavelets war die charakteristische Funktion
des Einheitsintervals
\begin{equation}
\varphi^{(0)}(t)
=
\chi_{[0,1)}(t)
=
\begin{cases}
1\qquad\qquad&0\le t < 1\\
0\qquad\qquad&\text{sonst}.
\end{cases}
\end{equation}
Aus $\varphi^{(0)}$ lässt sich wie in Kapitel~\ref{chapter:haar-wavelet}
ausgeführt in eine Multiskalen-Analyse verwandeln.
Eine solche garantiert zunächst einmal eine lineare Darstellung von
$\varphi^{(0)}$ duch skalierte und translierte Kopien von $\varphi^{(0)}$
in der Form
\begin{equation}
\varphi^{(0)}(t)
=
\sqrt{2}
\sum_{k\in\mathbb Z}
h_k\varphi(2t-k)
=
\sum_{k\in\mathbb Z}
h_k D_{\frac12}T_k\varphi^{(0)}(t).
\label{splines:skalierungsrelation}
\end{equation}
Aus der Skalierungsrelation~\eqref{splines:skalierungsrelation}
kann man dann das Mutter-Wavelet gewinnen, wie in
Abschnitt~\ref{section:mutter-aus-vater} gezeigt wird.
Im Falle des Haar-Wavelets sind sind die Koeffzienten der
Skalierungsrelation
\[
\varphi^{(0)}(t)
=
\sqrt{2}
\biggl(
\frac1{\sqrt{2}}
\varphi^{(0)}(2t)
+
\frac1{\sqrt{2}}
\varphi^{(0)}(2t-1)
\biggr)
\qquad
\Rightarrow
\quad
h_0 = h_1 = \frac{1}{\sqrt{2}}.
\]
Nach Lemma~\ref{lemma:msa:psirelation} ist dann das Mutterwavelet
\[
\psi^{(0)}(t)
=
\sqrt{2}
\biggl(
\frac{1}{\sqrt{2}}
\varphi^{(0)}(2t)
-
\frac{1}{\sqrt{2}}
\varphi^{(0)}(2t-1)
\biggr),
\]
wie ebenfalss bereits in Kapitel~\ref{chapter:haar-wavelet}.
Auch wurde bereits auf die schlechten Stetikeitseigenschaften und
die schlechte Frequenz-Lokalisierung des Haar-Wavelets hingewiesen,
welche den praktischen Nutzen des Haar-Wavelets limitiert.
Damit stellt sich das Problem, ob man Funktionen gewinnen kann, die
eine ähnlich einfach zu gewinnende Skalierungsrelation haben.

\subsection{Splines
\label{subsection:splines}}
Gesucht ist jetzt also eine Vater-Wavelet-Funktion, welche ähnlich
einfach ist wie das eben repetiert Haar-Wavelet, aber bessere 
Stetigkeitseigenschaften und damit bessere Lokalisierung im Frequenzbereich
hat.

Wir definieren die Funktionen $B_n=\varphi^{(n)}$ rekursiv wie folgt.

\begin{definition}
Die Funktion $\varphi^{(0)}$ ist wie vorhin die charakteristische Funktion
des Einheitsintervals.
Für $n>0$ setzt man
\[
\varphi^{(n)} = \varphi^{(0)} * \varphi^{(n-1)}.
\]
Wir verwenden auch die Notation $B_n=\varphi^{(n)}$.
Die Funktionen $B_n(t)$ heissen $B$-Splines vom Grad $n$.
\end{definition}

Man beachte, dass die Funktionen $\varphi^{(n)}$ zwar in $L^2(\mathbb R)$ 
sind, aber sie sind nicht normiert.
Erst recht sind die Translate nicht orthonormiert, wie man das für eine
Multiskalenanalyse erwarten würde.

Explizit bedeutet die Definition der Funktion $\varphi^{(n)}$
\begin{equation}
\varphi^{(n)}(t)
=
\int_{-\infty}^\infty
\varphi^{(0)}(s)
\varphi^{(n-1)}(t-s)
\,ds
=
\int_0^1
\varphi^{(n-1)}(t-s)
\,ds
=
-
\int_t^{t-1}
\varphi^{(n-1)}(\tau)
\,d\tau
=
\int_{t-1}^t \varphi^{(n-1)}(\tau)\,d\tau
\label{eq:varphin-berechnung}
\end{equation}
mit der Substitution $\tau=t-s$.

\begin{lemma}
\label{lemma:phidiffbar}
Die Funktion $\varphi^{(n)}$ ist für $n>1$ $n$ mal stetig differenzierbar.
\end{lemma}

\begin{proof}[Beweis]
Mit Hilfe der Formel~\eqref{eq:varphin-berechnung} kann man die Ableitung
\begin{align*}
\frac{d}{dt}
\varphi^{(n)}(t)
&=
\frac{d}{dt} \int_{t-1}^t \varphi^{(n-1)}(\tau)\,d\tau
=
\varphi^{(n-1)}(\tau)\bigg|_{\tau=t}
-
\varphi^{(n-1)}(\tau)\bigg|_{\tau=t-1}
=
\varphi^{(n-1)}(t)-\varphi^{(n-1)}(t-1)
\end{align*}
berechnen.
Für $n>1$ ist die Funktion $\varphi^{(n-1)}$ stetig, also auch die
Ableitung.
Damit ist $\varphi^{(n)}$ stetig differenzierbar.
\end{proof}

\begin{beispiel}
Für $n=1$ finde man zum Beispiel
\[
\varphi^{(1)}(t)
=
\int_0^1 \varphi^{(0)}(t-s)\,ds
=
\begin{cases}
t\qquad\qquad&0\le t<1\\
2-t\qquad\qquad&1\le t<2\\
0\qquad\qquad&\text{sonst}
\end{cases}
\]
Die $L^2$-Norm davon ist
\[
\| \varphi^{(1)}\|^2
=
2
\int_0^1 t^2\,dt
=
2\biggl[\frac13t^3\biggr]_0^1
=
\frac23.
\]
Die Funktion $\varphi^{(1)}$ ist in Abbildung~\ref{spline:phi1}
dargestellt.
\begin{figure}
\centering
\includegraphics{chapters/9-spline/images/phi1.pdf}
\caption{Der Graph der Funktion $\varphi^{(1)}(t)$ zeigt, dass die
Funktion $\varphi^{(1)}$ stetig ist.
\label{spline:phi1}}
\end{figure}
\end{beispiel}

\begin{beispiel}
\begin{figure}
\centering
\includegraphics{chapters/9-spline/images/phi2.pdf}
\caption{Graph der Funktion $\varphi^{(2)}$
\label{spline:phi2}}
\end{figure}
Die Berechnung von $\varphi^{(2)}$ ist etwas aufwendiger.
Nach \eqref{eq:varphin-berechnung} muss
für $t\in\mathbb R$ das Integral von $\varphi^{(1)}$ über
das Interval $[t-1,t]$ berechnet werden:
\begin{align*}
&t<0:
&
\varphi^{(2)}(t)
&=0
\\
0\le\,&t<1:
&
\varphi^{(2)}(t)
&=
\int_0^t\tau\,d\tau
=
\biggl[\frac12\tau^2\biggr]_0^t = \frac12t^2
\\
1\le\,&t<2:
&
\varphi^{(2)}(t)
&=
\int_{t-1}^1 \tau\,d\tau
+
\int_1^t 2-\tau\,d\tau
=
\biggl[\frac12\tau^2\biggr]_{t-1}^1
+
\biggl[2\tau-\frac12\tau^2\biggr]_1^t
%\\
%&&
%&=
%\frac12-\frac12(t-1)^2+2t-\frac12-\frac12t^2-2+\frac12
%\\
%&&
%&=
%\frac12-\frac12t^2+t-\frac12+2t-\frac12-\frac12t^2-2+\frac12
\\
&&
&=
-t^2+3t-\frac32
=
-\biggl(t-\frac32\biggr)^2+\frac34
\\
2\le\,&t<3:
&
\varphi^{(2)}(t)
&=
\int_{t-1}^2 2-\tau\,d\tau
=
\int_0^{3-t} s\,ds
=
\frac12(3-t)^2
\\
&t>3:
&
\varphi^{(2)}(t)
&=
0
\end{align*}
Der Graph der Funktion $\varphi^{(2)}$ ist in Abbildung~\ref{spline:phi2}
dargestellt.
In Übereinstimmung mit Lemma~\ref{lemma:phidiffbar} hat die Funktion
$\varphi^{(2)}$ keine offensichtlichen ``Knickpunkte''.
\end{beispiel}

\begin{beispiel}
\begin{figure}
\centering
\includegraphics{chapters/9-spline/images/phi3.pdf}
\caption{Graph der Funktion $\varphi^{(3)}$.
\label{spline:phi3}}
\end{figure}
Für $\varphi^{(3)}(t)$ stellen wir nur die Resultate
zusammen, die leicht mit Hilfe eines Computer-Algebra-Systems
gewonnen werden können:
\begin{equation}
\varphi^{(3)}(t)
=
\begin{cases}
\displaystyle
\phantom{-}
\frac16t^3
&\qquad
0\le t<1
\\[9pt]
\displaystyle
-\frac12t^3+2t^2-2t+\frac23
&\qquad
1\le t<2
\\[9pt]
\displaystyle
\phantom{-}
\frac12t^3-4t^2+10t-\frac{22}3
&\qquad
2\le t<3
\\[9pt]
\displaystyle
-\frac16(t-4)^3
&\qquad
3\le t<4
\\[9pt]
\displaystyle
\phantom{-}
0&\qquad\text{sonst}
\end{cases}
\label{phi3-polynome}
\end{equation}
Der Graph der Funktion $\varphi^{(3)}$ ist in Abbildung~\ref{spline:phi3}
dargestellt.
\end{beispiel}

Aus den Beispielen kann man ablesen, dass die Funktionen
$B_n(t)=\varphi^{(n)}(t)$ auf jedem Interval der Form $[k,k+1)$ 
ein Polynom vom Grad $n$ ist.
Wir sagen, $B_n$ sei stückweise {\em Grad-$n$-polynomiell}.
\index{Grad $n$-polynomiell}
Man kann zeigen, dass die vier Polynome auf der rechten
Seite von~\eqref{phi3-polynome} linear unabhängig sind.
Wir vermuten, dass die Translate von $B_n$ auf jedem Interval der Form
$[k,k+1)$ {\em linear unabhängige} Polynome vom Grad $n$ sind.
Dies bedeutet, dass durch Linearkombination von Translaten der
Funktionen $B_n$ jede beliebige stückweise Grad-$n$-polynomielle
Funktion dargestellt werden kann.
Die Funktionen $B_n$ bilden somit eine Basis des Raums der stückweise 
Grad-$n$-polynomiellen Funktionen.
Dies erklärt die Bedeutung der Funktionen $B_n$ in der Numerik.
Besonders die Funktionen $B_3$ spielen eine grosse Rolle in den
Anwendungen.
Zum Beispiel werden solche kubischen Splines für die Beschreibung
von Kurven in der Computergraphik verwendet.

\subsection{Skalierungsrelation für $\varphi^{(n)}$
\label{subsection:skalierungsrelation-phin}}
Die Funktionen $B_n=\varphi^{(n)}$ können nur dann Anlass zu einer
Multiskalen-Analyse geben, wenn sie die Skalierungseigenschaft
besitzen.
Wir wissen bereits aus Abschnitt~\ref{subsection:spline:haar},
dass das Haar-Wavelet $\varphi^{(0)}$ die Skalierungsrelation
mit den Koeffizienten $1/\sqrt{2}$ erfüllt.
Satz~\ref{satz:faltung-linearkombination} garantiert, dass mit
$\varphi^{(n-1)}$ auch $\varphi^{(n)}$ die Skalierungseigenschaft hat.
Wir haben in~\eqref{eq:faltung-linearkombination}
sogar eine Formel, mit der wir die Koeffizienten der Skalierungsrelation
berechnen können.

\begin{beispiel}
Wir bestimmen die Koeffizienten der Skalierungsrelation von $\varphi^{(1)}$.
Da $\varphi^{(1)}=\varphi^{(0)}*\varphi^{(0)}$ folgt mit den bekannten
Koeffizienten $\varphi^{(0)}_0=\varphi^{(0)}_1=1/\sqrt{2}$ die
Formel~\eqref{eq:faltung-linearkombination}
\[
\begin{aligned}
\varphi^{(1)}_0
&=
\frac1{\sqrt{2}}
\varphi^{(0)}_0
\cdot
\varphi^{(0)}_0
&&=
\frac{1}{2\sqrt{2}}
=\frac{1}{2\sqrt{2}}\binom{2}{0}
\\
\varphi^{(1)}_1
&=
\frac1{\sqrt{2}}
(
\varphi^{(0)}_0
\cdot
\varphi^{(0)}_1
+
\varphi^{(0)}_1
\cdot
\varphi^{(0)}_0
)
&&=
\frac{2}{2\sqrt{2}}
=\frac{1}{2\sqrt{2}}\binom{2}{1}
\\
\varphi^{(1)}_2
&=
\frac1{\sqrt{2}}
\varphi^{(0)}_1
\cdot
\varphi^{(0)}_1
&&=
\frac{1}{2\sqrt{2}}
=\frac{1}{2\sqrt{2}}\binom{2}{2}
\end{aligned}
\]
Diese Koeffizienten haben wir im Wesentlichen schon in
\eqref{eq:skalierung-phi1-proto} angetroffen.
\end{beispiel}

Das Beispiel lässt sich verallgemeinern und führt auf den folgenden Satz.

\begin{satz}
Die Funktion $\varphi^{(n)}$ hat die Skalierungseigenschaft  mit den
Koeffizienten
\[
\varphi^{(n)}_k = \frac{\sqrt{2}}{2^{n+1}}\binom{n+1}{k}.
\]
für $0\le k\le n$.
\end{satz}
Die zugehörige erzeugende Funktion Funktion ist
\begin{equation}
\tilde{H}_n(\omega)
=
\frac{1}{\sqrt{2}}
\sum_{k=0}^{n+1}
\varphi_k^{(n)} e^{-ik\omega}
=
\frac{1}{\sqrt{2}}
\sum_{k=0}^{n+1}
\frac{\sqrt{2}}{2^{n+1}}
\binom{n+1}{k}
e^{-ik\omega}
=
\sum_{k=0}^{n+1}
\frac{1}{2^{n+1}}
\binom{n+1}{k}
e^{ik\omega}
=
\biggl(
\frac{1+e^{-i\omega}}{2}
\biggr)^{n+1},
\end{equation}
wir werden sie in Abschnitt~\ref{subsection:spline-ft} verwenden,
um die erzeugende Funktion für die orthonormalisierten Spline-Wavelets
zu berechnen.


