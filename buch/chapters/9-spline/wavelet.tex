%
% wavelet.tex
%
% (c) 2019 Prof Dr Andreas Müller, Hochschule Rapperswil
%
\section{Wavelets
\label{section:spline-wavelets}}
\rhead{Spline-Wavelet}
Die Funktionen $B_n$ erfüllen eine Skalierungsrelation, es besteht also
die Hoffnung, dass daraus eine Multiskalen-Analyse konstruiert werden kann.
Dazu muss zunächst die Fourier-Transformierte von $B_n$ bestimmt werden,
da alle Konstruktionen von Abschnitt~ \ref{section:skalfour} im
Frequenzbereich stattgefunden haben.
Anschliessend kann das Orthonormalisierungsverfahren von
Abschnitt~\ref{section:orthonormalisierung} auf die Funktionen $B_n$
angewendet werden.
Schliesslich kann man aus den Koeffizienten der Skalierungsrelation
auch die Koeffizienten des Wavelets ableiten und so das Wavelet
bestimmen.

\subsection{Fourier-Transformierte $\hat{B}_n$
\label{subsection:spline-ft}}
Die Fourier-Transformierte von $B_n$ kann dank der Faltungsformel für die
Fourier-Transfomierte 
\begin{equation}
\hat{B}_n = \widehat{B_0 * B_{n-1}} = \sqrt{2\pi}\hat{B}_0 \cdot \hat{B}_{n-1}
\qquad
\hat{B}_n = (\!\sqrt{2\pi})^n \hat{B}_0^{n-1}
\label{Bn-ft}
\end{equation}
berechnet werden.
Es muss also nur $\hat{B}_0$ berechnet werden.

Die Fourier-Transformierte von $B_0$ ist
\begin{align*}
\hat{B}_0(\omega)
&=
\frac1{\sqrt{2\pi}}
\int_{-\infty}^{\infty}
B_0(t) e^{-i\omega t}\,dt
=
\frac1{\sqrt{2\pi}}
\int_0^1 e^{-i\omega t}\,dt
=
\frac1{\sqrt{2\pi}}
\biggl[
\frac{1}{-i\omega} 
e^{-i\omega t}
\biggr]_0^1
\\
&=
\frac1{\sqrt{2\pi}}
\frac{e^{-i\omega}-1}{-i\omega}
=
\frac1{\sqrt{2\pi}}
\frac{1-e^{-i\omega}}{i\omega}
=
\frac1{\sqrt{2\pi}}
\frac{2}{\omega}
e^{-i\omega/2}
\frac{e^{i\omega/2} - e^{-i\omega/2}}{2i}
\\
&=
\frac1{\sqrt{2\pi}}
e^{-i\omega/2}
\frac{\displaystyle\sin\frac{\omega}2}{\displaystyle\frac{\omega}2}
=
\frac1{\sqrt{2\pi}}
e^{-i\omega/2}
\operatorname{sinc}\frac{\omega}2.
\end{align*}
Daraus und aus \eqref{Bn-ft} folgt jetzt der folgende Satz

\begin{satz}
\label{satz:Bn-ft}
Die Fourier-Transformierte von $B_n$ ist
\[
\hat{B}_n(\omega)
=
(\!\sqrt{2\pi})^n
\hat{B}_0(\omega)^{n+1}
=
(\!\sqrt{2\pi})^n
\biggl(
\frac{1}{\sqrt{2\pi}}
e^{-i\omega/2} \operatorname{sinc}\frac{\omega}2
\biggr)^{n+1}
=
\frac{1}{\sqrt{2\pi}}
\biggl(
e^{-i\omega/2} \operatorname{sinc}\frac{\omega}2
\biggr)^{n+1}.
\]
\end{satz}

\subsection{Orthonormalisierung
\label{subsection:spline-orthonormalisierung}}
In Abschnitt~\ref{section:orthonormalisierung} wurde gezeigt, wie 
erreicht werden kann, dass die Translate einer Funktion orthonormiert
sind.
Dieses Verfahren soll jetzt auf die Funktionen $B_n$ angewendet werden.
Es wurde gezeigt, dass dazu zunächst die periodisierte Funktion
\[
\Phi_n(\omega)
=
2\pi \mathcal{P}|\hat{B}_n|^2(\omega)
\]
bestimmt werden muss. 
Den Periodisierungs-Operator kann man direkt berechnen:
\begin{align*}
\mathcal{P} |\hat{B}_n|^2(\omega)
&=
\sum_{l\in\mathbb Z} |\hat{B}_n(\omega + 2\pi l)|^2
=
\frac{1}{2\pi}
\sum_{l\in\mathbb Z}
\biggl|
e^{-i\omega/2 +i\pi l}
\operatorname{sinc}\biggl(\frac{\omega}2+l\pi\biggr)^{n+1}
\biggr|^2
\\
&=
\frac{1}{2\pi}
\sum_{l\in\mathbb Z}
\biggl|
\frac{\sin\frac{\omega}2}{\frac{\omega}2 + l\pi}
\biggr|^{2n+2}
=
\frac{1}{2\pi}
\biggl|\sin\frac{\omega}2\biggr|^{2n+2}
\sum_{l\in\mathbb Z}
\biggl|
\frac{1}{\frac{\omega}2 + l\pi}
\biggr|^2
\\
\Rightarrow\qquad
\Phi_n(\omega)
=
2\pi\mathcal{P}|\hat{B}_n|^2(\omega)
&=
\biggl|\sin\frac{\omega}2\biggr|^{2n+2}
\sum_{l\in\mathbb Z}
\biggl|
\frac{1}{\omega/2 + l\pi}
\biggr|^2.
\end{align*}
Da die Funktion $\operatorname{sinc}$ gerade ist, ist auch $\Phi_n(\omega)$
gerade.

Als nächstes muss die Funktion $1/\sqrt{\Phi_n(\omega)}$ in eine komplexe
Fourier-Reihe entwickelt werden.
Da $\Phi_n(\omega)$ gerade ist, muss dies auch mit einer reinen
Fourier-Kosinus-Reihe möglich sein.
Dies bedeutet, dass die Koeffizienten $c_k$ der komplexen Fourier-Reihe
von $1/\sqrt{\Phi_n(\omega)}$ reell sind und $c_k=c_{-k}$,
sie können mit 
der Formel
\begin{equation}
c_k = c_{-k}
=
\frac{1}{2\pi}
\int_{-\pi}^{\pi}
\frac{\cos(k\omega)}{\sqrt{\Phi_n(\omega)}}
\,d\omega
\label{sqrtPhincoef}
\end{equation}
berechnet werden.

Aus der Lösung von Aufgabe~\ref{aufgabe:orthonormalisierung} ist bekannt,
dass das gesuchte Wavelet mit Hilfe der Koeffizienten gebildet werden 
kann:
\[
\varphi_n(\omega)
=
\sum_{k\in\mathbb Z}
c_k B_n(t-k).
=
\sum_{k\in\mathbb Z}
c_k T_kB_n(t).
\]
Da die Funktionen $B_n$ stückweise Grad-$n$-polynomiell sind, ist auch
$\varphi_n(\omega)$ stückweise Grad-$n$-polynomiell.

In \cite{buch:blatter} wird sogar gezeigt, dass $\sqrt{\Phi_n(\omega)}$
als Polynom in $\cos\omega$ mit rationalen Koeffizienten geschrieben
werden kann.
Wir benötigen diese Eigenschaft im folgenden nicht und sie ist auch nicht
weiter hilfreich für die Berechnung der Koeffizienten $c_k$, die mit
Hilfe der Formel \eqref{sqrtPhincoef}
numerisch berechnet werden müssen.

\subsection{Koeffizienten der Skalierungsrelation
\label{subsection:spline-skalierungskoeffizienten}}

\subsection{Spline-Wavelet
\label{subsection:spline-wavelet}}





