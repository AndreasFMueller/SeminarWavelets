%
% faltung.tex
%
% (c) 2019 Prof Dr Andreas Müller, Hochschule Rapperswil
%
\section{Faltung
\label{section:faltung}}
\rhead{Faltung}
Die Faltung kann dazu verwendet werden, neue Funktionen zu erzeugen.
Dazu müssen wir verstehen, wie die Faltung und die Operatoren $T_b$ und
$D_b$ zusammenwirken.
Wir erinnern an die Definition~\ref{definition:faltung} und die
Formel \eqref{definition:formel:faltung}
\[
(f*g)(t)
=
\int_{-\infty}^\infty f(s)g(t-s)\,ds.
\]

\subsection{Translation und Dilatation
\label{subsection:translation-und-dialatation}}
Wir berechnen die Faltung der Translate $T_bf$ und $T_cg$:
\begin{align*}
(T_bf * T_cg)(t)
&=
\int_{-\infty}^\infty T_bf(s) T_cg(t-s)\,ds
\\
&=
\int_{-\infty}^\infty f(s-b) g(t-s-c)\,ds
\\
&=
\int_{-\infty}^\infty f(\sigma) g(t-\sigma-b-c)\,d\sigma
\\
&=
T_{b+c}(f*g)(t)
\end{align*}
mit der Substitution $\sigma=s-b$ oder $s=\sigma+b$.
Wir formulieren dieses Resultat als

\begin{satz}
Für beliebige Funktionen $f,g\in L^2(\mathbb R)$ gilt
\begin{equation}
T_bf*T_cg
=
T_{b+c}(f*g).
\label{eq:Tconv}
\end{equation}
\end{satz}

Für zwei Funktionen $f$ und $g$ berechnen wir die Faltung von
$D_af$ und $D_ag$:
\begin{align*}
(D_af * D_ag)(t)
&=
\int_{-\infty}^\infty D_af(s) D_ag(t-s)\,ds
\\
&=
\int_{-\infty}^\infty \frac{1}{a}f(s/a) g((t-s)/a)\,ds
\\
&=
\int_{-\infty}^\infty f(s/a) g((t-s)/a)\,\frac{ds}{a}
\\
&=
\int_{-\infty}^\infty f(\sigma) g(t/a- \sigma)\,d\sigma
\\
&=
(f*g)(t/a) = \sqrt{a} D_a(f*g)(t)
\end{align*}
mit der Substitution $\sigma=s/a$.
In der Skalierungsrelation einer Multiskalenanalyse kommen keine
gemischten Skalierungsfaktoren $a$ vor, daher reicht die Form
\eqref{eq:Dconv} für die nachfolgenden Rechnung aus.

\begin{satz}
Für beliebige Funktionen $f,g\in L^2(\mathbb R)$ gilt
\begin{align}
D_af * D_ag &= \sqrt{a}D_a(f*g)
&&\text{und}
&
\tilde{D}_af * \tilde{D}_ag &= a\tilde{D}_a(f*g).
\label{eq:Dconv}
\end{align}
für $a>0$.
\end{satz}

\begin{proof}[Beweis]
Wir müssen nur noch die zweite Formel für den Operator $\tilde{D}_a$
nachweisen.
Dazu verwenden wir die Relation $D_af = \sqrt{a}\tilde{D}_a$ und die
erste Beziehung in \eqref{eq:Dconv} und erhalten
\[
\tilde{D}_af * \tilde{D}_ag
=
\sqrt{a}D_af * \sqrt{a}D_ag
=
a
\sqrt{a}
D_a (f*g)
=
a
\tilde{D}_a(f*g).
\]
Damit ist auch die zweite Relation bewiesen.
\end{proof}


\subsection{Linearkombinationen
\label{subsection:linearkombinationen}}
Ein wesentlicher Aspekt einer Multiskalenanalyse ist, dass sich das
skalierte Vaterwavelet $D_2\varphi$ eine Linearkombination von
Translaten des Vater-Wavelets ist.

\begin{definition}
Wir sagen, eine Funktion $f$ erfüllt die {\em Skalierungseigenschaft},
wenn es Koeffizienten $f_k$ gibt derart, dass
\[
D_2f = \sum_{k\in\mathbb Z} f_k T_kf
\]
\end{definition}

\begin{satz}
\label{satz:faltung-linearkombination}
Sind $f$ und $g$ Funktionen, die beide die Skalierungseigenschaft
mit Koeffizienten $f_k$ und $g_l$
haben, dann hat auch $h=f*g$ die Skalierungseigenschaft
mit Koeffizienten
\begin{equation}
h_s = \frac1{\sqrt{2}} \sum_{k+l=s}f_kg_l.
\label{eq:faltung-linearkombination}
\end{equation}
\end{satz}

\begin{proof}[Beweis]
Seien $f_k$ und $g_k$ die Koeffizienten der Skalierungsrelation von $f$
bzw.~$g$, also
\begin{align*}
D_2f
&=
\sum_{k\in\mathbb Z} f_k T_kf
&
D_2g
&=
\sum_{k\in\mathbb Z} g_k T_kg.
\end{align*}
Nach \eqref{eq:Dconv} ist die Faltung der Funktionen $D_2f$ und $D_2g$
\[
D_2f * D_2g
=
\sqrt{2}
D_2(f*g)
=
\sqrt{2}
D_2h.
\]
Andererseits können wir auch die Faltung der Linearkombinationen 
berechnen:
\begin{align*}
D_2f * D_2g
&=
\biggl( \sum_{k\in\mathbb Z} f_kT_kf \biggr)
*
\biggl( \sum_{l\in\mathbb Z} g_lT_lg \biggr)
\\
&=
\sum_{s\in\mathbb Z}
\sum_{k+l=s} f_kg_l T_kf * T_lg
=
\sum_{s\in\mathbb Z}
\sum_{k+l=s} f_kg_l T_{k+l}(f * g)
=
\sum_{s\in\mathbb Z}
\biggl(\sum_{k+l=s} f_kg_l\biggr) T_sh
\end{align*}
Es folgt, dass
\[
D_2h
=
\frac1{\sqrt{2}} D_2f * D_2g
=
\sum_{s\in\mathbb Z}
\biggl(\frac1{\sqrt{2}}\sum_{k+l=s} f_kg_l\biggr) T_sh.
\]
Dies ist eine Skalierungsrelation für die Funktion $h$, die Koeffizienten
sind
\[
h_s = \frac1{\sqrt{2}} \sum_{k+l=s}f_kg_l.
\]
Damit ist alles bewiesen.
\end{proof}

\subsection{Faltung im Frequenzbereich
\label{subsection:faltung-im-frequenzbereich}}
Wir können dieses Resultat auch im Frequenzbereich formulieren.
Wir haben im Abschnitt~\ref{section:skalfour} die Skalierungsrelation
für die Fourier-Transformierte $\hat{\varphi}$ von $\varphi$ mit Hilfe
der erzeugenden Funktion
\[
H(\omega)
=
\frac{1}{\sqrt{2}}
\sum_{k\in\mathbb Z} h_ke^{ik\omega}
\]
(siehe Definition~\ref{definition:erzeugende-funktion-msa})
als
\begin{equation}
\hat{\varphi}(\omega)
=
H\biggl(\frac{\omega}2\biggr)
\,
\hat{\varphi}\biggl(\frac{\omega}2\biggr)
\label{faltung:skalierung}
\end{equation}
geschrieben.
Sind
$(\varphi_1,H_1)$ und $(\varphi_2,H_2)$ zwei Paare von Funktionen,
welche beide die Skalierungsrelation~\eqref{faltung:skalierung} erfüllen,
dann folgt
\begin{align*}
%\hat{\varphi}_1(\omega)
%&=
%H_1\biggl(\frac{\omega}2\biggr)
%\hat{\varphi}_1 \biggl(\frac{\omega}2\biggr)
%&&\wedge&
%\hat{\varphi}_1(\omega)
%=
%H_1\biggl(\frac{\omega}2\biggr)
%\hat{\varphi}_1\biggl(\frac{\omega}2\biggr)
%&&\Rightarrow&
\hat{\varphi}_1(\omega)
\cdot
\hat{\varphi}_2(\omega)
&=
H_1\biggl(\frac{\omega}2\biggr)
\cdot
H_2\biggl(\frac{\omega}2\biggr)
\cdot
\hat{\varphi}_1\biggl(\frac{\omega}2\biggr)
\cdot
\hat{\varphi}_2\biggl(\frac{\omega}2\biggr)
\\
%&&&&
%&\Rightarrow&
\Rightarrow\qquad
\widehat{\mathstrut\varphi_1*\varphi_2}(\omega)
&=
H_1\biggl(\frac{\omega}2\biggr)
H_2\biggl(\frac{\omega}2\biggr)
\,
\widehat{\mathstrut\varphi_1*\varphi_2}\biggl(\frac{\omega}2\biggr).
\end{align*}
Die Faltung $\varphi_1*\varphi_2$ erfüllt also die Skalierungsrelation
mit der erzeugenden Funktion $H_1\cdot H_2$, die wegen
\[
H_1(\omega)\,H_2(\omega)
=
\frac{1}{\sqrt{2}}
\sum_{k\in\mathbb Z} h_{1,k}e^{ik\omega}
\cdot
\frac{1}{\sqrt{2}}
\sum_{l\in\mathbb Z} h_{2,l}e^{il\omega}
=
\frac{1}{\sqrt{2}}
\sum_{s\in\mathbb Z} e^{is\omega}
\frac{1}{\sqrt{2}}
\sum_{k+l=s}h_{1,k}h_{2,l}
\]
die Koeffizienten
\[
h_s = \frac{1}{\sqrt{2}}\sum_{k+l=s} h_{1,k}h_{2,l}
\]
hat,
wie bereits im Satz~\ref{satz:faltung-linearkombination} gezeigt worden ist.

