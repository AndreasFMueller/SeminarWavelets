%
% faltung.tex
%
% (c) 2019 Prof Dr Andreas Müller, Hochschule Rapperswil
%
\section{Faltung
\label{section:faltung}}
\rhead{Faltung}
Die Faltung kann dazu verwendet werden, neue Funktionen zu erzeugen.
Dazu müssen wir verstehen, wie die Faltung und die Operatoren $T_b$ und
$D_b$ zusammenwirken.
Wir gehen von der Form
\[
(f*g)(t)
=
\int_{-\infty}^\infty f(s)g(t-s)\,ds
\]
aus.

\subsection{Translation und Dilatation
\label{subsection:translation-und-dialatation}}
Wir berechnen die Faltung von $T_bf$ und $T_cg$:
\begin{align*}
(T_bf * T_cg)(t)
&=
\int_{-\infty}^\infty T_bf(s) T_cg(t-s)\,ds
\\
&=
\int_{-\infty}^\infty f(s-b) g(t-s-c)\,ds
\\
&=
\int_{-\infty}^\infty f(\sigma) g(t-\sigma-b-c)\,d\sigma
\\
&=
T_{b+c}(f*g)(t)
\end{align*}
mit der Substituion $\sigma=s-b$ oder $s=\sigma+b$.

\begin{satz}
Für beliebige Funktionen $f,g\in L^2(\mathbb R)$ gilt
\begin{equation}
T_bf*T_cg
=
T_{b+c}(f*g).
\label{eq:Tconv}
\end{equation}
\end{satz}

Für zwei Funktionen $f$ und $g$ berechnen wir die Faltung von
$D_af$ und $D_ag$:
\begin{align*}
(D_af * D_ag)(t)
&=
\int_{-\infty}^\infty D_af(s) D_ag(t-s)\,ds
\\
&=
\int_{-\infty}^\infty \frac{1}{a}f(s/a) g((t-s)/a)\,ds
\\
&=
\int_{-\infty}^\infty f(\sigma) g(t/a- \sigma)\,d\sigma
\\
&=
(f*g)(t/a) = \sqrt{a} D_a(f*g)(t)
\end{align*}
mit der Substitution $\sigma=s/a$.

\begin{satz}
Für beliebige Funktionen $f,g\in L^2(\mathbb R)$ gilt
\begin{equation}
D_af * D_ag = \sqrt{a}D_a(f*g)
\label{eq:Dconv}
\end{equation}
\end{satz}

\subsection{Linearkombinationen
\label{subsection:linearkombinationen}}
Ein wesentlicher Aspekt einer Multiskalenanalyse ist, dass sich das
skalierte Vaterwavelet $D_2\varphi$ eine Linearkombination von
Translaten des Vater-Wavelets ist.

\begin{definition}
Wir sagen, eine Funktion $f$ erfüllt die {\em Skalierungseigenschaft},
wenn es Koeffizienten $f_k$ gibt derart, dass
\[
D_2f = \sum_{k\in\mathbb Z} f_k T_kf
\]
\end{definition}

\begin{satz}
\label{satz:faltung-linearkombination}
Sind $f$ und $g$ Funktionen, die beide die Skalierungseigenschaft
mit Koeffizienten $f_k$ und $g_l$
haben, dann hat auch $h=f*g$ die Skalierungseigenschaft
mit Koeffizienten
\begin{equation}
h_s = \frac1{\sqrt{2}} \sum_{k+l=s}f_kg_l.
\label{eq:faltung-linearkombination}
\end{equation}
\end{satz}

\begin{proof}[Beweis]
Seien $f_k$ und $g_k$ die Koeffizienten der Skalierungsrelation von $f$
bzw.~$g$, also
\begin{align*}
D_2f
&=
\sum_{k\in\mathbb Z} f_k T_kf
&
D_2g
&=
\sum_{k\in\mathbb Z} g_k T_kg.
\end{align*}
Nach \eqref{eq:Dconv} ist die Faltung der Funktionen $D_2f$ und $D_2g$
\[
D_2f * D_2g
=
\sqrt{2}
D_2(f*g)
=
\sqrt{2}
D_2h.
\]
Andererseits können wir auch die Faltung der Linearkombinationen 
berechnen:
\begin{align*}
D_2f * D_2g
&=
\biggl( \sum_{k\in\mathbb Z} f_kT_kf \biggr)
*
\biggl( \sum_{l\in\mathbb Z} g_lT_lg \biggr)
\\
&=
\sum_{s\in\mathbb Z}
\sum_{k+l=s} f_kg_l T_kf * T_lg
=
\sum_{s\in\mathbb Z}
\sum_{k+l=s} f_kg_l T_{k+l}(f * g)
=
\sum_{s\in\mathbb Z}
\biggl(\sum_{k+l=s} f_kg_l\biggr) T_sh
\end{align*}
Es folgt, dass
\[
D_2h
=
\frac1{\sqrt{2}} D_2f * D_2g
=
\sum_{s\in\mathbb Z}
\biggl(\frac1{\sqrt{2}}\sum_{k+l=s} f_kg_l\biggr) T_sh.
\]
Dies ist eine Skalierungsrelation für die Funktion $h$, die Koeffizienten
sind
\[
h_s = \frac1{\sqrt{2}} \sum_{k+l=s}f_kg_l.
\]
Damit ist alles bewiesen.
\end{proof}






