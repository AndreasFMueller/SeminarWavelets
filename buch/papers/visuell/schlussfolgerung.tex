\section{Schlussfolgerung}
\rhead{Schlussfolgerung}

In den vorherigen Abschnitten haben wir gezeigt, wie Gabor-Wavelets zur Verbesserung von CNNs im Bereich Bildklassifizierung genutzt werden können.
Die Motivation zu diesem Schritt kommt aus der Neurologie, wo mit Gabor-Wavelets die Funktionsweise von einfachen Zellen im primären visuellen Kortex modelliert wird.
Der Versuch hat an einem praktischen Beispiel gezeigt, wie mit einer simplen Gabor-Vorverarbeitung die Genauigkeit von einem CNN verbessert werden kann.
Diese Verbesserung konnte erreicht werden, ohne die Rechenleistung für Klassifizierungen zu erhöhen.
Auf Grund dieser Erkenntnisse drängt sich eine solche Vorverarbeitung bei vielen Problemen des maschinellen Lernens auf. 

Natürlich könnten diese Gabor-Ideen noch auf viele andere Arten implementiert werden.
Eine logische Erweiterung dieser Arbeit wäre zum Beispiel, die Parameter der zweidimensionalen Gabor-Wavelets während des Trainings automatisiert zu erlernen.
