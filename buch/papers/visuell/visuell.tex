\section{Visuelle Wahrnehmung}
\rhead{Visuelle Wahrnehmung}

In den vorherigen Abschnitten haben wir das zweidimensionale Gabor-Wavelet genauer betrachtet.
Dieses Wavelet ist einerseits der optimale Kompromiss zwischen Zeit- und Frequenzauflösung und andererseits kann es in verschiedene Richtungen ausgerichtet werden.
Diese speziellen Eigenschaften sind sehr nützlich und es zeigte sich, dass die Hirne von Säugetieren ähnliche Eigenschaften besitzen.
In den folgenden Abschnitten wird versucht zu zeigen, was die visuelle Wahrnehmung und Gabor-Wavelets gemeinsam haben. 

\subsection{Primärer Visueller Kortex}\label{subsec:v1}

Der primäre Visuelle Kortex (V1) verarbeitet die Bildinformationen welche vom Auge aufgenommen werden.
Als Ausgangsprodukt stellt er abstrakte Features zu Verfügung, welche dann vom Hirn zusammen mit weiteren Informationen (Kontextwissen, andere Sinne)  benützt werden, um Objekte zu erkennen.
Erste Analysen des V1 wurden von Hubel und Wiesel bereits 1959 veröffentlicht \cite{paper:hubelwiesel}.
Ihre Erkenntnisse haben sie anhand von Versuchen an Katzenhirnen gewonnen.

Jedes Neuron innerhalb des V1 bekommt Eingangssignale einer bestimmten Region der Netzhaut.
Eine solche Region wird rezeptives Feld genannt.
Diese rezeptiven Felder bestimmen also welcher Teil eines Bildes vom Neutron bearbeitet wird.
Es gibt verschiedene Arten von Neuronen innerhalb des V1.
Eine wichtige Gruppe davon sind die sogenannten einfachen Zellen.
Diese zeigen ein orientierungsspezifisches Antwortverhalten auf optische Inputs.
Ausserdem beinhalten die einfachen Zellen immer abwechselnde ON und OFF Regionen, welche ansprechen bei Licht (ON) oder eben keinem Licht (OFF).
Die zweite wichtige Gruppe der Neuronen sind die komplexen Zellen.
Diese können als Kombination von mehreren einfachen Zellen modelliert werden, wobei alle einfachen Zellen in die selbe Richtung orientiert sind.
Allerdings ist diese Erklärung der komplexen Zellen nicht überall anerkannt \cite{book:neuroscience}.


\subsection{Modellierung der Funktion des V1 mithilfe von Gabor-Wavelets}

Die Eigenschaften der einfachen Zellen sind sehr ähnlich derjenigen des Gabor-Wavelets.
Deshalb haben Forscher begonnen die einfachen Zellen mit Hilfe von Gabor-Wavelets zu modellieren \cite{paper:imgrep}.
Analog zu den einfachen Zellen können Gabor-Wavelets ebenfalls in beliebige Richtungen ausgerichtet und deren Wellenlänge variiert werden.
Ausserdem kann die Schwingung des Gabor-Wavelets als ON-OFF-Pattern interpretiert werden.
Dies ist eine sehr schöne Entdeckung, da es zeigt, dass sich der V1 während der Evolution hin zu einem optimalen Kompromiss aus Zeit- und Frequenzinformationen entwickelt hat.
Die Eigenschaften des Gabor-Wavelets sollten auch für die künstliche Bilderkennung sinnvoll sein, da es offensichtlich einen evolutionären Vorteil darstellt wenn Bilder in dieser Art vorverarbeitet werden.