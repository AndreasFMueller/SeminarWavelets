\section{Visuelle Wahrnehmung}
\rhead{Visuelle Wahrnehmung}

In den vorherigen Abschnitten haben wir das zweidimensionale Gabor-Wavelet genauer betrachtet.
Dieses Wavelet ist einerseits der optimale Kompromiss zwischen Zeit- und Frequenzauflösung und andererseits kann es in verschiedene Richtungen ausgerichtet werden.
Diese speziellen Eigenschaften sind sehr nützlich und es zeigte sich, dass das menschliche Hirn ähnliche Dinge gelernt hat über die Evolution.

\subsection{Primärer Visueller Kortex}

Der Primäre Visuelle Kortex (V1) verarbeitet die Bildinformationen welche vom Auge aufgenommen werden.
Als Ausgangsprodukt stellt er abstrakte Features zu Verfügung, welche dann vom Hirn zusammen mit weiteren Informationen (Kontextwissen, andere Sinne)  benützt werden um Dinge zu erkennen.
%TODO find out whats the point


\subsection{Modellierung der Funktion des V1 mithilfe von Gabor-Wavelets}

Diverse Forschungsarbeiten haben gezeigt das die Funktion des V1 mithilfe von Gabor-Wavelets modelliert werden kann. %TODO insert references, add more text
Analog zum V1 können Gabor-Wavelets ebenfalls in beliebige Richtungen ausgerichtet und deren Wellenlänge variiert werden.