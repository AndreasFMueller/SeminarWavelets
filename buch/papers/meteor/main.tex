%
% main.tex -- Paper zum Thema meteor
%
% (c) 2019 Hochschule Rapperswil
%
\chapter{Analyse von Meteor-Echos\label{chapter:meteor}}
\lhead{Analyse von Meteor-Echos}
\begin{refsection}
\chapterauthor{Dominic Hüppi}

\section{Vorwort}
\rhead{Vorwort}
Seit dem Beginn des Menschen blickt dieser dem Himmel entgegen. Bereits in der Steinzeit, der frühesten Epoche der Menschheitsgeschichte, nahm die Beobachtung von Sonne und Gestirn ihre Anfänge. Neben der Sonne und dem Mond wirkten Meteore oder umgangssprachlich 'Sternschnuppen' dabei schon immer eine Faszination auf ihre Beobachter aus. In der Antike galten sie als Vorzeichen für den Tod, im frühen Christentum waren sie ein Zeichen der Seelenerlösung und im Mittelalter überbrachten sie angeblich die Mahnungen Gottes. \\
Heutzutage, in Zeiten der modernen Astronomie, schauen wir den herabstürzenden Leuchterscheinungen etwas entspannter und objektiver (ausgenommen sind die Flacherdler) entgegen. Wenn wir heute eine 'Sternschnuppe' beobachten gehen einem die Wünsche in Erfüllung, aber bloss keinem seinen Wunsch verraten. Für Astronomen gelten Meteore als wichtige Zeugen aus der Frühzeit des Sonnensystems und für den heute ungemein wichtigen 'Influencer' bilden sie ideale Sujets für den nächsten Auftritt in den sozialen Medien. Und wer hat schon nicht fasziniert dem Himmelgestirn entgegengeblickt und den aufleuchtenden Lichtstrahl romantisiert. \\
Eine sehr interessantes Ereignis diese Meteore. Was sie genau sind und welche Rolle dabei 'Wavelets' einnehmen erfahren Sie in diesem Kapitel.

\section{Meteore}
\rhead{Abschnitt}


\section{Abschnitt}
\rhead{Abschnitt}

\section{Schlussfolgerung}
\rhead{Schlussfolgerung}

\printbibliography[heading=subbibliography]
\end{refsection}
