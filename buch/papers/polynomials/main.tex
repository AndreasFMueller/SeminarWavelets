%
% main.tex -- Paper zum Thema Polynome
%
% (c) 2019 Hochschule Rapperswil
%
\chapter{Wavelets und polynomiale Signale\label{chapter:thema}}
\lhead{Wavelets und polynomiale Signale}
\begin{refsection}
\chapterauthor{Raphael Nestler}

In der Literatur zu Wavelets findet man die folgende Aussage zu Daubechies
Wavelets:
\begin{displayquote}[\cite{wikipedia:daubechies}]
For example, $D2$, with one vanishing moment, easily encodes polynomials of one
coefficient, or constant signal components. $D4$ encodes polynomials with two
coefficients, i.e.\ constant and linear signal components; and $D6$ encodes
3-polynomials, i.e.\ constant, linear and quadratic signal components.
\end{displayquote}
Ein Daubechies Wavelet mit $A$ verschwindenden Momenten und Filterlänge $N=2A$
soll also ein Polynom der Ordnung $A-1$ einfach darstellen können. Wir wollen
erörtern, was das nun in der Praxis genau bedeutet und welche Anwendungen es
ermöglicht.

Wir werden im folgenden Daubechies Wavelets mit $A$ verschwindenden Momenten mit
db$A$ bezeichnen.

Die Analysen und Simulationen wurden jeweils mittels Python~\cite{python} und im
speziellem dem PyWavelets~\cite{gregory_r_lee_2019_2634243} Paket durchgeführt.
Der dazu verwendete Code wurde in einem GitHub-Repository%
\footnote{\url{https://github.com/rnestler/mathsem-FS2019/tree/paper}}%
~\cite{polynomials:repo}
abgelegt. Zusätzlich sind auch interaktive Jupyter Notebooks abgelegt um den
Einfluss von verschiedenen Parametern auszuprobieren. Mittels
Binder%
\footnote{\url{https://mybinder.org/v2/gh/rnestler/mathsem-FS2019/paper}}%
~\cite{project_jupyter-proc-scipy-2018}%
kann eine online Umgebung gestartet werden um diese Beispiele auszuführen.

\section{Analyse von polynomialen Signalen}
\rhead{Polynomiale Signale}

Wir werden als erstes die Signale in \autoref{polynomials:signals} mittels des db1
(Haar) Wavelets analysieren.

\begin{figure}
    \centering
    %% Creator: Matplotlib, PGF backend
%%
%% To include the figure in your LaTeX document, write
%%   \input{<filename>.pgf}
%%
%% Make sure the required packages are loaded in your preamble
%%   \usepackage{pgf}
%%
%% Figures using additional raster images can only be included by \input if
%% they are in the same directory as the main LaTeX file. For loading figures
%% from other directories you can use the `import` package
%%   \usepackage{import}
%% and then include the figures with
%%   \import{<path to file>}{<filename>.pgf}
%%
%% Matplotlib used the following preamble
%%   \usepackage{fontspec}
%%
\begingroup%
\makeatletter%
\begin{pgfpicture}%
\pgfpathrectangle{\pgfpointorigin}{\pgfqpoint{2.900000in}{3.000000in}}%
\pgfusepath{use as bounding box, clip}%
\begin{pgfscope}%
\pgfsetbuttcap%
\pgfsetmiterjoin%
\definecolor{currentfill}{rgb}{1.000000,1.000000,1.000000}%
\pgfsetfillcolor{currentfill}%
\pgfsetlinewidth{0.000000pt}%
\definecolor{currentstroke}{rgb}{1.000000,1.000000,1.000000}%
\pgfsetstrokecolor{currentstroke}%
\pgfsetdash{}{0pt}%
\pgfpathmoveto{\pgfqpoint{0.000000in}{0.000000in}}%
\pgfpathlineto{\pgfqpoint{2.900000in}{0.000000in}}%
\pgfpathlineto{\pgfqpoint{2.900000in}{3.000000in}}%
\pgfpathlineto{\pgfqpoint{0.000000in}{3.000000in}}%
\pgfpathclose%
\pgfusepath{fill}%
\end{pgfscope}%
\begin{pgfscope}%
\pgfsetbuttcap%
\pgfsetmiterjoin%
\definecolor{currentfill}{rgb}{1.000000,1.000000,1.000000}%
\pgfsetfillcolor{currentfill}%
\pgfsetlinewidth{0.000000pt}%
\definecolor{currentstroke}{rgb}{0.000000,0.000000,0.000000}%
\pgfsetstrokecolor{currentstroke}%
\pgfsetstrokeopacity{0.000000}%
\pgfsetdash{}{0pt}%
\pgfpathmoveto{\pgfqpoint{0.362500in}{0.375000in}}%
\pgfpathlineto{\pgfqpoint{2.610000in}{0.375000in}}%
\pgfpathlineto{\pgfqpoint{2.610000in}{2.640000in}}%
\pgfpathlineto{\pgfqpoint{0.362500in}{2.640000in}}%
\pgfpathclose%
\pgfusepath{fill}%
\end{pgfscope}%
\begin{pgfscope}%
\pgfsetbuttcap%
\pgfsetroundjoin%
\definecolor{currentfill}{rgb}{0.000000,0.000000,0.000000}%
\pgfsetfillcolor{currentfill}%
\pgfsetlinewidth{0.803000pt}%
\definecolor{currentstroke}{rgb}{0.000000,0.000000,0.000000}%
\pgfsetstrokecolor{currentstroke}%
\pgfsetdash{}{0pt}%
\pgfsys@defobject{currentmarker}{\pgfqpoint{0.000000in}{-0.048611in}}{\pgfqpoint{0.000000in}{0.000000in}}{%
\pgfpathmoveto{\pgfqpoint{0.000000in}{0.000000in}}%
\pgfpathlineto{\pgfqpoint{0.000000in}{-0.048611in}}%
\pgfusepath{stroke,fill}%
}%
\begin{pgfscope}%
\pgfsys@transformshift{0.464659in}{0.375000in}%
\pgfsys@useobject{currentmarker}{}%
\end{pgfscope}%
\end{pgfscope}%
\begin{pgfscope}%
\definecolor{textcolor}{rgb}{0.000000,0.000000,0.000000}%
\pgfsetstrokecolor{textcolor}%
\pgfsetfillcolor{textcolor}%
\pgftext[x=0.464659in,y=0.277778in,,top]{\color{textcolor}\rmfamily\fontsize{8.000000}{9.600000}\selectfont 0.0}%
\end{pgfscope}%
\begin{pgfscope}%
\pgfsetbuttcap%
\pgfsetroundjoin%
\definecolor{currentfill}{rgb}{0.000000,0.000000,0.000000}%
\pgfsetfillcolor{currentfill}%
\pgfsetlinewidth{0.803000pt}%
\definecolor{currentstroke}{rgb}{0.000000,0.000000,0.000000}%
\pgfsetstrokecolor{currentstroke}%
\pgfsetdash{}{0pt}%
\pgfsys@defobject{currentmarker}{\pgfqpoint{0.000000in}{-0.048611in}}{\pgfqpoint{0.000000in}{0.000000in}}{%
\pgfpathmoveto{\pgfqpoint{0.000000in}{0.000000in}}%
\pgfpathlineto{\pgfqpoint{0.000000in}{-0.048611in}}%
\pgfusepath{stroke,fill}%
}%
\begin{pgfscope}%
\pgfsys@transformshift{0.975455in}{0.375000in}%
\pgfsys@useobject{currentmarker}{}%
\end{pgfscope}%
\end{pgfscope}%
\begin{pgfscope}%
\definecolor{textcolor}{rgb}{0.000000,0.000000,0.000000}%
\pgfsetstrokecolor{textcolor}%
\pgfsetfillcolor{textcolor}%
\pgftext[x=0.975455in,y=0.277778in,,top]{\color{textcolor}\rmfamily\fontsize{8.000000}{9.600000}\selectfont 0.5}%
\end{pgfscope}%
\begin{pgfscope}%
\pgfsetbuttcap%
\pgfsetroundjoin%
\definecolor{currentfill}{rgb}{0.000000,0.000000,0.000000}%
\pgfsetfillcolor{currentfill}%
\pgfsetlinewidth{0.803000pt}%
\definecolor{currentstroke}{rgb}{0.000000,0.000000,0.000000}%
\pgfsetstrokecolor{currentstroke}%
\pgfsetdash{}{0pt}%
\pgfsys@defobject{currentmarker}{\pgfqpoint{0.000000in}{-0.048611in}}{\pgfqpoint{0.000000in}{0.000000in}}{%
\pgfpathmoveto{\pgfqpoint{0.000000in}{0.000000in}}%
\pgfpathlineto{\pgfqpoint{0.000000in}{-0.048611in}}%
\pgfusepath{stroke,fill}%
}%
\begin{pgfscope}%
\pgfsys@transformshift{1.486250in}{0.375000in}%
\pgfsys@useobject{currentmarker}{}%
\end{pgfscope}%
\end{pgfscope}%
\begin{pgfscope}%
\definecolor{textcolor}{rgb}{0.000000,0.000000,0.000000}%
\pgfsetstrokecolor{textcolor}%
\pgfsetfillcolor{textcolor}%
\pgftext[x=1.486250in,y=0.277778in,,top]{\color{textcolor}\rmfamily\fontsize{8.000000}{9.600000}\selectfont 1.0}%
\end{pgfscope}%
\begin{pgfscope}%
\pgfsetbuttcap%
\pgfsetroundjoin%
\definecolor{currentfill}{rgb}{0.000000,0.000000,0.000000}%
\pgfsetfillcolor{currentfill}%
\pgfsetlinewidth{0.803000pt}%
\definecolor{currentstroke}{rgb}{0.000000,0.000000,0.000000}%
\pgfsetstrokecolor{currentstroke}%
\pgfsetdash{}{0pt}%
\pgfsys@defobject{currentmarker}{\pgfqpoint{0.000000in}{-0.048611in}}{\pgfqpoint{0.000000in}{0.000000in}}{%
\pgfpathmoveto{\pgfqpoint{0.000000in}{0.000000in}}%
\pgfpathlineto{\pgfqpoint{0.000000in}{-0.048611in}}%
\pgfusepath{stroke,fill}%
}%
\begin{pgfscope}%
\pgfsys@transformshift{1.997045in}{0.375000in}%
\pgfsys@useobject{currentmarker}{}%
\end{pgfscope}%
\end{pgfscope}%
\begin{pgfscope}%
\definecolor{textcolor}{rgb}{0.000000,0.000000,0.000000}%
\pgfsetstrokecolor{textcolor}%
\pgfsetfillcolor{textcolor}%
\pgftext[x=1.997045in,y=0.277778in,,top]{\color{textcolor}\rmfamily\fontsize{8.000000}{9.600000}\selectfont 1.5}%
\end{pgfscope}%
\begin{pgfscope}%
\pgfsetbuttcap%
\pgfsetroundjoin%
\definecolor{currentfill}{rgb}{0.000000,0.000000,0.000000}%
\pgfsetfillcolor{currentfill}%
\pgfsetlinewidth{0.803000pt}%
\definecolor{currentstroke}{rgb}{0.000000,0.000000,0.000000}%
\pgfsetstrokecolor{currentstroke}%
\pgfsetdash{}{0pt}%
\pgfsys@defobject{currentmarker}{\pgfqpoint{0.000000in}{-0.048611in}}{\pgfqpoint{0.000000in}{0.000000in}}{%
\pgfpathmoveto{\pgfqpoint{0.000000in}{0.000000in}}%
\pgfpathlineto{\pgfqpoint{0.000000in}{-0.048611in}}%
\pgfusepath{stroke,fill}%
}%
\begin{pgfscope}%
\pgfsys@transformshift{2.507841in}{0.375000in}%
\pgfsys@useobject{currentmarker}{}%
\end{pgfscope}%
\end{pgfscope}%
\begin{pgfscope}%
\definecolor{textcolor}{rgb}{0.000000,0.000000,0.000000}%
\pgfsetstrokecolor{textcolor}%
\pgfsetfillcolor{textcolor}%
\pgftext[x=2.507841in,y=0.277778in,,top]{\color{textcolor}\rmfamily\fontsize{8.000000}{9.600000}\selectfont 2.0}%
\end{pgfscope}%
\begin{pgfscope}%
\pgfsetbuttcap%
\pgfsetroundjoin%
\definecolor{currentfill}{rgb}{0.000000,0.000000,0.000000}%
\pgfsetfillcolor{currentfill}%
\pgfsetlinewidth{0.803000pt}%
\definecolor{currentstroke}{rgb}{0.000000,0.000000,0.000000}%
\pgfsetstrokecolor{currentstroke}%
\pgfsetdash{}{0pt}%
\pgfsys@defobject{currentmarker}{\pgfqpoint{-0.048611in}{0.000000in}}{\pgfqpoint{0.000000in}{0.000000in}}{%
\pgfpathmoveto{\pgfqpoint{0.000000in}{0.000000in}}%
\pgfpathlineto{\pgfqpoint{-0.048611in}{0.000000in}}%
\pgfusepath{stroke,fill}%
}%
\begin{pgfscope}%
\pgfsys@transformshift{0.362500in}{0.477955in}%
\pgfsys@useobject{currentmarker}{}%
\end{pgfscope}%
\end{pgfscope}%
\begin{pgfscope}%
\definecolor{textcolor}{rgb}{0.000000,0.000000,0.000000}%
\pgfsetstrokecolor{textcolor}%
\pgfsetfillcolor{textcolor}%
\pgftext[x=0.206278in,y=0.439399in,left,base]{\color{textcolor}\rmfamily\fontsize{8.000000}{9.600000}\selectfont 0}%
\end{pgfscope}%
\begin{pgfscope}%
\pgfsetbuttcap%
\pgfsetroundjoin%
\definecolor{currentfill}{rgb}{0.000000,0.000000,0.000000}%
\pgfsetfillcolor{currentfill}%
\pgfsetlinewidth{0.803000pt}%
\definecolor{currentstroke}{rgb}{0.000000,0.000000,0.000000}%
\pgfsetstrokecolor{currentstroke}%
\pgfsetdash{}{0pt}%
\pgfsys@defobject{currentmarker}{\pgfqpoint{-0.048611in}{0.000000in}}{\pgfqpoint{0.000000in}{0.000000in}}{%
\pgfpathmoveto{\pgfqpoint{0.000000in}{0.000000in}}%
\pgfpathlineto{\pgfqpoint{-0.048611in}{0.000000in}}%
\pgfusepath{stroke,fill}%
}%
\begin{pgfscope}%
\pgfsys@transformshift{0.362500in}{0.735341in}%
\pgfsys@useobject{currentmarker}{}%
\end{pgfscope}%
\end{pgfscope}%
\begin{pgfscope}%
\definecolor{textcolor}{rgb}{0.000000,0.000000,0.000000}%
\pgfsetstrokecolor{textcolor}%
\pgfsetfillcolor{textcolor}%
\pgftext[x=0.206278in,y=0.696785in,left,base]{\color{textcolor}\rmfamily\fontsize{8.000000}{9.600000}\selectfont 1}%
\end{pgfscope}%
\begin{pgfscope}%
\pgfsetbuttcap%
\pgfsetroundjoin%
\definecolor{currentfill}{rgb}{0.000000,0.000000,0.000000}%
\pgfsetfillcolor{currentfill}%
\pgfsetlinewidth{0.803000pt}%
\definecolor{currentstroke}{rgb}{0.000000,0.000000,0.000000}%
\pgfsetstrokecolor{currentstroke}%
\pgfsetdash{}{0pt}%
\pgfsys@defobject{currentmarker}{\pgfqpoint{-0.048611in}{0.000000in}}{\pgfqpoint{0.000000in}{0.000000in}}{%
\pgfpathmoveto{\pgfqpoint{0.000000in}{0.000000in}}%
\pgfpathlineto{\pgfqpoint{-0.048611in}{0.000000in}}%
\pgfusepath{stroke,fill}%
}%
\begin{pgfscope}%
\pgfsys@transformshift{0.362500in}{0.992727in}%
\pgfsys@useobject{currentmarker}{}%
\end{pgfscope}%
\end{pgfscope}%
\begin{pgfscope}%
\definecolor{textcolor}{rgb}{0.000000,0.000000,0.000000}%
\pgfsetstrokecolor{textcolor}%
\pgfsetfillcolor{textcolor}%
\pgftext[x=0.206278in,y=0.954172in,left,base]{\color{textcolor}\rmfamily\fontsize{8.000000}{9.600000}\selectfont 2}%
\end{pgfscope}%
\begin{pgfscope}%
\pgfsetbuttcap%
\pgfsetroundjoin%
\definecolor{currentfill}{rgb}{0.000000,0.000000,0.000000}%
\pgfsetfillcolor{currentfill}%
\pgfsetlinewidth{0.803000pt}%
\definecolor{currentstroke}{rgb}{0.000000,0.000000,0.000000}%
\pgfsetstrokecolor{currentstroke}%
\pgfsetdash{}{0pt}%
\pgfsys@defobject{currentmarker}{\pgfqpoint{-0.048611in}{0.000000in}}{\pgfqpoint{0.000000in}{0.000000in}}{%
\pgfpathmoveto{\pgfqpoint{0.000000in}{0.000000in}}%
\pgfpathlineto{\pgfqpoint{-0.048611in}{0.000000in}}%
\pgfusepath{stroke,fill}%
}%
\begin{pgfscope}%
\pgfsys@transformshift{0.362500in}{1.250114in}%
\pgfsys@useobject{currentmarker}{}%
\end{pgfscope}%
\end{pgfscope}%
\begin{pgfscope}%
\definecolor{textcolor}{rgb}{0.000000,0.000000,0.000000}%
\pgfsetstrokecolor{textcolor}%
\pgfsetfillcolor{textcolor}%
\pgftext[x=0.206278in,y=1.211558in,left,base]{\color{textcolor}\rmfamily\fontsize{8.000000}{9.600000}\selectfont 3}%
\end{pgfscope}%
\begin{pgfscope}%
\pgfsetbuttcap%
\pgfsetroundjoin%
\definecolor{currentfill}{rgb}{0.000000,0.000000,0.000000}%
\pgfsetfillcolor{currentfill}%
\pgfsetlinewidth{0.803000pt}%
\definecolor{currentstroke}{rgb}{0.000000,0.000000,0.000000}%
\pgfsetstrokecolor{currentstroke}%
\pgfsetdash{}{0pt}%
\pgfsys@defobject{currentmarker}{\pgfqpoint{-0.048611in}{0.000000in}}{\pgfqpoint{0.000000in}{0.000000in}}{%
\pgfpathmoveto{\pgfqpoint{0.000000in}{0.000000in}}%
\pgfpathlineto{\pgfqpoint{-0.048611in}{0.000000in}}%
\pgfusepath{stroke,fill}%
}%
\begin{pgfscope}%
\pgfsys@transformshift{0.362500in}{1.507500in}%
\pgfsys@useobject{currentmarker}{}%
\end{pgfscope}%
\end{pgfscope}%
\begin{pgfscope}%
\definecolor{textcolor}{rgb}{0.000000,0.000000,0.000000}%
\pgfsetstrokecolor{textcolor}%
\pgfsetfillcolor{textcolor}%
\pgftext[x=0.206278in,y=1.468944in,left,base]{\color{textcolor}\rmfamily\fontsize{8.000000}{9.600000}\selectfont 4}%
\end{pgfscope}%
\begin{pgfscope}%
\pgfsetbuttcap%
\pgfsetroundjoin%
\definecolor{currentfill}{rgb}{0.000000,0.000000,0.000000}%
\pgfsetfillcolor{currentfill}%
\pgfsetlinewidth{0.803000pt}%
\definecolor{currentstroke}{rgb}{0.000000,0.000000,0.000000}%
\pgfsetstrokecolor{currentstroke}%
\pgfsetdash{}{0pt}%
\pgfsys@defobject{currentmarker}{\pgfqpoint{-0.048611in}{0.000000in}}{\pgfqpoint{0.000000in}{0.000000in}}{%
\pgfpathmoveto{\pgfqpoint{0.000000in}{0.000000in}}%
\pgfpathlineto{\pgfqpoint{-0.048611in}{0.000000in}}%
\pgfusepath{stroke,fill}%
}%
\begin{pgfscope}%
\pgfsys@transformshift{0.362500in}{1.764886in}%
\pgfsys@useobject{currentmarker}{}%
\end{pgfscope}%
\end{pgfscope}%
\begin{pgfscope}%
\definecolor{textcolor}{rgb}{0.000000,0.000000,0.000000}%
\pgfsetstrokecolor{textcolor}%
\pgfsetfillcolor{textcolor}%
\pgftext[x=0.206278in,y=1.726331in,left,base]{\color{textcolor}\rmfamily\fontsize{8.000000}{9.600000}\selectfont 5}%
\end{pgfscope}%
\begin{pgfscope}%
\pgfsetbuttcap%
\pgfsetroundjoin%
\definecolor{currentfill}{rgb}{0.000000,0.000000,0.000000}%
\pgfsetfillcolor{currentfill}%
\pgfsetlinewidth{0.803000pt}%
\definecolor{currentstroke}{rgb}{0.000000,0.000000,0.000000}%
\pgfsetstrokecolor{currentstroke}%
\pgfsetdash{}{0pt}%
\pgfsys@defobject{currentmarker}{\pgfqpoint{-0.048611in}{0.000000in}}{\pgfqpoint{0.000000in}{0.000000in}}{%
\pgfpathmoveto{\pgfqpoint{0.000000in}{0.000000in}}%
\pgfpathlineto{\pgfqpoint{-0.048611in}{0.000000in}}%
\pgfusepath{stroke,fill}%
}%
\begin{pgfscope}%
\pgfsys@transformshift{0.362500in}{2.022273in}%
\pgfsys@useobject{currentmarker}{}%
\end{pgfscope}%
\end{pgfscope}%
\begin{pgfscope}%
\definecolor{textcolor}{rgb}{0.000000,0.000000,0.000000}%
\pgfsetstrokecolor{textcolor}%
\pgfsetfillcolor{textcolor}%
\pgftext[x=0.206278in,y=1.983717in,left,base]{\color{textcolor}\rmfamily\fontsize{8.000000}{9.600000}\selectfont 6}%
\end{pgfscope}%
\begin{pgfscope}%
\pgfsetbuttcap%
\pgfsetroundjoin%
\definecolor{currentfill}{rgb}{0.000000,0.000000,0.000000}%
\pgfsetfillcolor{currentfill}%
\pgfsetlinewidth{0.803000pt}%
\definecolor{currentstroke}{rgb}{0.000000,0.000000,0.000000}%
\pgfsetstrokecolor{currentstroke}%
\pgfsetdash{}{0pt}%
\pgfsys@defobject{currentmarker}{\pgfqpoint{-0.048611in}{0.000000in}}{\pgfqpoint{0.000000in}{0.000000in}}{%
\pgfpathmoveto{\pgfqpoint{0.000000in}{0.000000in}}%
\pgfpathlineto{\pgfqpoint{-0.048611in}{0.000000in}}%
\pgfusepath{stroke,fill}%
}%
\begin{pgfscope}%
\pgfsys@transformshift{0.362500in}{2.279659in}%
\pgfsys@useobject{currentmarker}{}%
\end{pgfscope}%
\end{pgfscope}%
\begin{pgfscope}%
\definecolor{textcolor}{rgb}{0.000000,0.000000,0.000000}%
\pgfsetstrokecolor{textcolor}%
\pgfsetfillcolor{textcolor}%
\pgftext[x=0.206278in,y=2.241104in,left,base]{\color{textcolor}\rmfamily\fontsize{8.000000}{9.600000}\selectfont 7}%
\end{pgfscope}%
\begin{pgfscope}%
\pgfsetbuttcap%
\pgfsetroundjoin%
\definecolor{currentfill}{rgb}{0.000000,0.000000,0.000000}%
\pgfsetfillcolor{currentfill}%
\pgfsetlinewidth{0.803000pt}%
\definecolor{currentstroke}{rgb}{0.000000,0.000000,0.000000}%
\pgfsetstrokecolor{currentstroke}%
\pgfsetdash{}{0pt}%
\pgfsys@defobject{currentmarker}{\pgfqpoint{-0.048611in}{0.000000in}}{\pgfqpoint{0.000000in}{0.000000in}}{%
\pgfpathmoveto{\pgfqpoint{0.000000in}{0.000000in}}%
\pgfpathlineto{\pgfqpoint{-0.048611in}{0.000000in}}%
\pgfusepath{stroke,fill}%
}%
\begin{pgfscope}%
\pgfsys@transformshift{0.362500in}{2.537045in}%
\pgfsys@useobject{currentmarker}{}%
\end{pgfscope}%
\end{pgfscope}%
\begin{pgfscope}%
\definecolor{textcolor}{rgb}{0.000000,0.000000,0.000000}%
\pgfsetstrokecolor{textcolor}%
\pgfsetfillcolor{textcolor}%
\pgftext[x=0.206278in,y=2.498490in,left,base]{\color{textcolor}\rmfamily\fontsize{8.000000}{9.600000}\selectfont 8}%
\end{pgfscope}%
\begin{pgfscope}%
\pgfpathrectangle{\pgfqpoint{0.362500in}{0.375000in}}{\pgfqpoint{2.247500in}{2.265000in}}%
\pgfusepath{clip}%
\pgfsetrectcap%
\pgfsetroundjoin%
\pgfsetlinewidth{1.505625pt}%
\definecolor{currentstroke}{rgb}{0.121569,0.466667,0.705882}%
\pgfsetstrokecolor{currentstroke}%
\pgfsetdash{}{0pt}%
\pgfpathmoveto{\pgfqpoint{0.464659in}{0.477955in}}%
\pgfpathlineto{\pgfqpoint{0.472672in}{0.500749in}}%
\pgfpathlineto{\pgfqpoint{0.488697in}{0.517436in}}%
\pgfpathlineto{\pgfqpoint{0.504721in}{0.528925in}}%
\pgfpathlineto{\pgfqpoint{0.528759in}{0.542427in}}%
\pgfpathlineto{\pgfqpoint{0.560809in}{0.556917in}}%
\pgfpathlineto{\pgfqpoint{0.608884in}{0.574664in}}%
\pgfpathlineto{\pgfqpoint{0.664971in}{0.591927in}}%
\pgfpathlineto{\pgfqpoint{0.737083in}{0.610868in}}%
\pgfpathlineto{\pgfqpoint{0.825221in}{0.630865in}}%
\pgfpathlineto{\pgfqpoint{0.937395in}{0.653043in}}%
\pgfpathlineto{\pgfqpoint{1.073607in}{0.676673in}}%
\pgfpathlineto{\pgfqpoint{1.233857in}{0.701294in}}%
\pgfpathlineto{\pgfqpoint{1.426156in}{0.727656in}}%
\pgfpathlineto{\pgfqpoint{1.650506in}{0.755262in}}%
\pgfpathlineto{\pgfqpoint{1.914918in}{0.784623in}}%
\pgfpathlineto{\pgfqpoint{2.219392in}{0.815283in}}%
\pgfpathlineto{\pgfqpoint{2.507841in}{0.841954in}}%
\pgfpathlineto{\pgfqpoint{2.507841in}{0.841954in}}%
\pgfusepath{stroke}%
\end{pgfscope}%
\begin{pgfscope}%
\pgfpathrectangle{\pgfqpoint{0.362500in}{0.375000in}}{\pgfqpoint{2.247500in}{2.265000in}}%
\pgfusepath{clip}%
\pgfsetrectcap%
\pgfsetroundjoin%
\pgfsetlinewidth{1.505625pt}%
\definecolor{currentstroke}{rgb}{1.000000,0.498039,0.054902}%
\pgfsetstrokecolor{currentstroke}%
\pgfsetdash{}{0pt}%
\pgfpathmoveto{\pgfqpoint{0.464659in}{0.735341in}}%
\pgfpathlineto{\pgfqpoint{2.507841in}{0.735341in}}%
\pgfpathlineto{\pgfqpoint{2.507841in}{0.735341in}}%
\pgfusepath{stroke}%
\end{pgfscope}%
\begin{pgfscope}%
\pgfpathrectangle{\pgfqpoint{0.362500in}{0.375000in}}{\pgfqpoint{2.247500in}{2.265000in}}%
\pgfusepath{clip}%
\pgfsetrectcap%
\pgfsetroundjoin%
\pgfsetlinewidth{1.505625pt}%
\definecolor{currentstroke}{rgb}{0.172549,0.627451,0.172549}%
\pgfsetstrokecolor{currentstroke}%
\pgfsetdash{}{0pt}%
\pgfpathmoveto{\pgfqpoint{0.464659in}{0.477955in}}%
\pgfpathlineto{\pgfqpoint{2.507841in}{0.992727in}}%
\pgfpathlineto{\pgfqpoint{2.507841in}{0.992727in}}%
\pgfusepath{stroke}%
\end{pgfscope}%
\begin{pgfscope}%
\pgfpathrectangle{\pgfqpoint{0.362500in}{0.375000in}}{\pgfqpoint{2.247500in}{2.265000in}}%
\pgfusepath{clip}%
\pgfsetrectcap%
\pgfsetroundjoin%
\pgfsetlinewidth{1.505625pt}%
\definecolor{currentstroke}{rgb}{0.839216,0.152941,0.156863}%
\pgfsetstrokecolor{currentstroke}%
\pgfsetdash{}{0pt}%
\pgfpathmoveto{\pgfqpoint{0.464659in}{0.477955in}}%
\pgfpathlineto{\pgfqpoint{0.544784in}{0.479538in}}%
\pgfpathlineto{\pgfqpoint{0.624909in}{0.484288in}}%
\pgfpathlineto{\pgfqpoint{0.705033in}{0.492204in}}%
\pgfpathlineto{\pgfqpoint{0.785158in}{0.503287in}}%
\pgfpathlineto{\pgfqpoint{0.865283in}{0.517537in}}%
\pgfpathlineto{\pgfqpoint{0.945408in}{0.534954in}}%
\pgfpathlineto{\pgfqpoint{1.025533in}{0.555537in}}%
\pgfpathlineto{\pgfqpoint{1.105657in}{0.579286in}}%
\pgfpathlineto{\pgfqpoint{1.185782in}{0.606202in}}%
\pgfpathlineto{\pgfqpoint{1.265907in}{0.636285in}}%
\pgfpathlineto{\pgfqpoint{1.346032in}{0.669535in}}%
\pgfpathlineto{\pgfqpoint{1.426156in}{0.705951in}}%
\pgfpathlineto{\pgfqpoint{1.506281in}{0.745533in}}%
\pgfpathlineto{\pgfqpoint{1.586406in}{0.788283in}}%
\pgfpathlineto{\pgfqpoint{1.674543in}{0.838964in}}%
\pgfpathlineto{\pgfqpoint{1.762680in}{0.893478in}}%
\pgfpathlineto{\pgfqpoint{1.850818in}{0.951823in}}%
\pgfpathlineto{\pgfqpoint{1.938955in}{1.013999in}}%
\pgfpathlineto{\pgfqpoint{2.027092in}{1.080007in}}%
\pgfpathlineto{\pgfqpoint{2.115230in}{1.149847in}}%
\pgfpathlineto{\pgfqpoint{2.203367in}{1.223518in}}%
\pgfpathlineto{\pgfqpoint{2.291504in}{1.301021in}}%
\pgfpathlineto{\pgfqpoint{2.379641in}{1.382355in}}%
\pgfpathlineto{\pgfqpoint{2.475791in}{1.475454in}}%
\pgfpathlineto{\pgfqpoint{2.507841in}{1.507500in}}%
\pgfpathlineto{\pgfqpoint{2.507841in}{1.507500in}}%
\pgfusepath{stroke}%
\end{pgfscope}%
\begin{pgfscope}%
\pgfpathrectangle{\pgfqpoint{0.362500in}{0.375000in}}{\pgfqpoint{2.247500in}{2.265000in}}%
\pgfusepath{clip}%
\pgfsetrectcap%
\pgfsetroundjoin%
\pgfsetlinewidth{1.505625pt}%
\definecolor{currentstroke}{rgb}{0.580392,0.403922,0.741176}%
\pgfsetstrokecolor{currentstroke}%
\pgfsetdash{}{0pt}%
\pgfpathmoveto{\pgfqpoint{0.464659in}{0.477955in}}%
\pgfpathlineto{\pgfqpoint{0.648946in}{0.479465in}}%
\pgfpathlineto{\pgfqpoint{0.745096in}{0.483279in}}%
\pgfpathlineto{\pgfqpoint{0.825221in}{0.489271in}}%
\pgfpathlineto{\pgfqpoint{0.897333in}{0.497509in}}%
\pgfpathlineto{\pgfqpoint{0.969445in}{0.509006in}}%
\pgfpathlineto{\pgfqpoint{1.033545in}{0.522400in}}%
\pgfpathlineto{\pgfqpoint{1.097645in}{0.539181in}}%
\pgfpathlineto{\pgfqpoint{1.153732in}{0.556941in}}%
\pgfpathlineto{\pgfqpoint{1.209820in}{0.577840in}}%
\pgfpathlineto{\pgfqpoint{1.265907in}{0.602135in}}%
\pgfpathlineto{\pgfqpoint{1.321994in}{0.630082in}}%
\pgfpathlineto{\pgfqpoint{1.378082in}{0.661934in}}%
\pgfpathlineto{\pgfqpoint{1.426156in}{0.692539in}}%
\pgfpathlineto{\pgfqpoint{1.474231in}{0.726363in}}%
\pgfpathlineto{\pgfqpoint{1.522306in}{0.763567in}}%
\pgfpathlineto{\pgfqpoint{1.570381in}{0.804311in}}%
\pgfpathlineto{\pgfqpoint{1.618456in}{0.848757in}}%
\pgfpathlineto{\pgfqpoint{1.666531in}{0.897065in}}%
\pgfpathlineto{\pgfqpoint{1.714606in}{0.949397in}}%
\pgfpathlineto{\pgfqpoint{1.762680in}{1.005913in}}%
\pgfpathlineto{\pgfqpoint{1.810755in}{1.066775in}}%
\pgfpathlineto{\pgfqpoint{1.858830in}{1.132143in}}%
\pgfpathlineto{\pgfqpoint{1.906905in}{1.202178in}}%
\pgfpathlineto{\pgfqpoint{1.954980in}{1.277041in}}%
\pgfpathlineto{\pgfqpoint{2.003055in}{1.356893in}}%
\pgfpathlineto{\pgfqpoint{2.059142in}{1.456575in}}%
\pgfpathlineto{\pgfqpoint{2.115230in}{1.563522in}}%
\pgfpathlineto{\pgfqpoint{2.171317in}{1.677989in}}%
\pgfpathlineto{\pgfqpoint{2.227404in}{1.800233in}}%
\pgfpathlineto{\pgfqpoint{2.283492in}{1.930509in}}%
\pgfpathlineto{\pgfqpoint{2.339579in}{2.069073in}}%
\pgfpathlineto{\pgfqpoint{2.395666in}{2.216180in}}%
\pgfpathlineto{\pgfqpoint{2.451754in}{2.372086in}}%
\pgfpathlineto{\pgfqpoint{2.507841in}{2.537045in}}%
\pgfpathlineto{\pgfqpoint{2.507841in}{2.537045in}}%
\pgfusepath{stroke}%
\end{pgfscope}%
\begin{pgfscope}%
\pgfsetrectcap%
\pgfsetmiterjoin%
\pgfsetlinewidth{0.803000pt}%
\definecolor{currentstroke}{rgb}{0.000000,0.000000,0.000000}%
\pgfsetstrokecolor{currentstroke}%
\pgfsetdash{}{0pt}%
\pgfpathmoveto{\pgfqpoint{0.362500in}{0.375000in}}%
\pgfpathlineto{\pgfqpoint{0.362500in}{2.640000in}}%
\pgfusepath{stroke}%
\end{pgfscope}%
\begin{pgfscope}%
\pgfsetrectcap%
\pgfsetmiterjoin%
\pgfsetlinewidth{0.803000pt}%
\definecolor{currentstroke}{rgb}{0.000000,0.000000,0.000000}%
\pgfsetstrokecolor{currentstroke}%
\pgfsetdash{}{0pt}%
\pgfpathmoveto{\pgfqpoint{2.610000in}{0.375000in}}%
\pgfpathlineto{\pgfqpoint{2.610000in}{2.640000in}}%
\pgfusepath{stroke}%
\end{pgfscope}%
\begin{pgfscope}%
\pgfsetrectcap%
\pgfsetmiterjoin%
\pgfsetlinewidth{0.803000pt}%
\definecolor{currentstroke}{rgb}{0.000000,0.000000,0.000000}%
\pgfsetstrokecolor{currentstroke}%
\pgfsetdash{}{0pt}%
\pgfpathmoveto{\pgfqpoint{0.362500in}{0.375000in}}%
\pgfpathlineto{\pgfqpoint{2.610000in}{0.375000in}}%
\pgfusepath{stroke}%
\end{pgfscope}%
\begin{pgfscope}%
\pgfsetrectcap%
\pgfsetmiterjoin%
\pgfsetlinewidth{0.803000pt}%
\definecolor{currentstroke}{rgb}{0.000000,0.000000,0.000000}%
\pgfsetstrokecolor{currentstroke}%
\pgfsetdash{}{0pt}%
\pgfpathmoveto{\pgfqpoint{0.362500in}{2.640000in}}%
\pgfpathlineto{\pgfqpoint{2.610000in}{2.640000in}}%
\pgfusepath{stroke}%
\end{pgfscope}%
\begin{pgfscope}%
\pgfsetbuttcap%
\pgfsetmiterjoin%
\definecolor{currentfill}{rgb}{1.000000,1.000000,1.000000}%
\pgfsetfillcolor{currentfill}%
\pgfsetfillopacity{0.800000}%
\pgfsetlinewidth{1.003750pt}%
\definecolor{currentstroke}{rgb}{0.800000,0.800000,0.800000}%
\pgfsetstrokecolor{currentstroke}%
\pgfsetstrokeopacity{0.800000}%
\pgfsetdash{}{0pt}%
\pgfpathmoveto{\pgfqpoint{0.440278in}{1.774028in}}%
\pgfpathlineto{\pgfqpoint{1.002719in}{1.774028in}}%
\pgfpathquadraticcurveto{\pgfqpoint{1.024941in}{1.774028in}}{\pgfqpoint{1.024941in}{1.796251in}}%
\pgfpathlineto{\pgfqpoint{1.024941in}{2.562222in}}%
\pgfpathquadraticcurveto{\pgfqpoint{1.024941in}{2.584444in}}{\pgfqpoint{1.002719in}{2.584444in}}%
\pgfpathlineto{\pgfqpoint{0.440278in}{2.584444in}}%
\pgfpathquadraticcurveto{\pgfqpoint{0.418056in}{2.584444in}}{\pgfqpoint{0.418056in}{2.562222in}}%
\pgfpathlineto{\pgfqpoint{0.418056in}{1.796251in}}%
\pgfpathquadraticcurveto{\pgfqpoint{0.418056in}{1.774028in}}{\pgfqpoint{0.440278in}{1.774028in}}%
\pgfpathclose%
\pgfusepath{stroke,fill}%
\end{pgfscope}%
\begin{pgfscope}%
\pgfsetrectcap%
\pgfsetroundjoin%
\pgfsetlinewidth{1.505625pt}%
\definecolor{currentstroke}{rgb}{0.121569,0.466667,0.705882}%
\pgfsetstrokecolor{currentstroke}%
\pgfsetdash{}{0pt}%
\pgfpathmoveto{\pgfqpoint{0.462500in}{2.500583in}}%
\pgfpathlineto{\pgfqpoint{0.684722in}{2.500583in}}%
\pgfusepath{stroke}%
\end{pgfscope}%
\begin{pgfscope}%
\definecolor{textcolor}{rgb}{0.000000,0.000000,0.000000}%
\pgfsetstrokecolor{textcolor}%
\pgfsetfillcolor{textcolor}%
\pgftext[x=0.773611in,y=2.461694in,left,base]{\color{textcolor}\rmfamily\fontsize{8.000000}{9.600000}\selectfont \(\displaystyle x^{0.5}\)}%
\end{pgfscope}%
\begin{pgfscope}%
\pgfsetrectcap%
\pgfsetroundjoin%
\pgfsetlinewidth{1.505625pt}%
\definecolor{currentstroke}{rgb}{1.000000,0.498039,0.054902}%
\pgfsetstrokecolor{currentstroke}%
\pgfsetdash{}{0pt}%
\pgfpathmoveto{\pgfqpoint{0.462500in}{2.345167in}}%
\pgfpathlineto{\pgfqpoint{0.684722in}{2.345167in}}%
\pgfusepath{stroke}%
\end{pgfscope}%
\begin{pgfscope}%
\definecolor{textcolor}{rgb}{0.000000,0.000000,0.000000}%
\pgfsetstrokecolor{textcolor}%
\pgfsetfillcolor{textcolor}%
\pgftext[x=0.773611in,y=2.306278in,left,base]{\color{textcolor}\rmfamily\fontsize{8.000000}{9.600000}\selectfont \(\displaystyle x^{0}\)}%
\end{pgfscope}%
\begin{pgfscope}%
\pgfsetrectcap%
\pgfsetroundjoin%
\pgfsetlinewidth{1.505625pt}%
\definecolor{currentstroke}{rgb}{0.172549,0.627451,0.172549}%
\pgfsetstrokecolor{currentstroke}%
\pgfsetdash{}{0pt}%
\pgfpathmoveto{\pgfqpoint{0.462500in}{2.189750in}}%
\pgfpathlineto{\pgfqpoint{0.684722in}{2.189750in}}%
\pgfusepath{stroke}%
\end{pgfscope}%
\begin{pgfscope}%
\definecolor{textcolor}{rgb}{0.000000,0.000000,0.000000}%
\pgfsetstrokecolor{textcolor}%
\pgfsetfillcolor{textcolor}%
\pgftext[x=0.773611in,y=2.150861in,left,base]{\color{textcolor}\rmfamily\fontsize{8.000000}{9.600000}\selectfont \(\displaystyle x^{1}\)}%
\end{pgfscope}%
\begin{pgfscope}%
\pgfsetrectcap%
\pgfsetroundjoin%
\pgfsetlinewidth{1.505625pt}%
\definecolor{currentstroke}{rgb}{0.839216,0.152941,0.156863}%
\pgfsetstrokecolor{currentstroke}%
\pgfsetdash{}{0pt}%
\pgfpathmoveto{\pgfqpoint{0.462500in}{2.034334in}}%
\pgfpathlineto{\pgfqpoint{0.684722in}{2.034334in}}%
\pgfusepath{stroke}%
\end{pgfscope}%
\begin{pgfscope}%
\definecolor{textcolor}{rgb}{0.000000,0.000000,0.000000}%
\pgfsetstrokecolor{textcolor}%
\pgfsetfillcolor{textcolor}%
\pgftext[x=0.773611in,y=1.995445in,left,base]{\color{textcolor}\rmfamily\fontsize{8.000000}{9.600000}\selectfont \(\displaystyle x^{2}\)}%
\end{pgfscope}%
\begin{pgfscope}%
\pgfsetrectcap%
\pgfsetroundjoin%
\pgfsetlinewidth{1.505625pt}%
\definecolor{currentstroke}{rgb}{0.580392,0.403922,0.741176}%
\pgfsetstrokecolor{currentstroke}%
\pgfsetdash{}{0pt}%
\pgfpathmoveto{\pgfqpoint{0.462500in}{1.878917in}}%
\pgfpathlineto{\pgfqpoint{0.684722in}{1.878917in}}%
\pgfusepath{stroke}%
\end{pgfscope}%
\begin{pgfscope}%
\definecolor{textcolor}{rgb}{0.000000,0.000000,0.000000}%
\pgfsetstrokecolor{textcolor}%
\pgfsetfillcolor{textcolor}%
\pgftext[x=0.773611in,y=1.840028in,left,base]{\color{textcolor}\rmfamily\fontsize{8.000000}{9.600000}\selectfont \(\displaystyle x^{3}\)}%
\end{pgfscope}%
\end{pgfpicture}%
\makeatother%
\endgroup%

    \caption{Die verschiedenen zu analysierenden polynomialen Signale\label{polynomials:signals}}
\end{figure}

In~\autoref{polynomials:haar} sind die Approximations- und die Detailkoeffizienten
der Transformation zu sehen. Wir sehen, dass die Approximationskoeffizienten
uns das grobe Signal und im Fall von $x^0 = 1$ sogar das exakte Signal liefern. Die
Detailkoeffizienten scheinen uns etwas zu liefern was proportional zur ersten
Ableitung des Signals ist.

\begin{figure}
    \centering
    %% Creator: Matplotlib, PGF backend
%%
%% To include the figure in your LaTeX document, write
%%   \input{<filename>.pgf}
%%
%% Make sure the required packages are loaded in your preamble
%%   \usepackage{pgf}
%%
%% Figures using additional raster images can only be included by \input if
%% they are in the same directory as the main LaTeX file. For loading figures
%% from other directories you can use the `import` package
%%   \usepackage{import}
%% and then include the figures with
%%   \import{<path to file>}{<filename>.pgf}
%%
%% Matplotlib used the following preamble
%%   \usepackage{fontspec}
%%
\begingroup%
\makeatletter%
\begin{pgfpicture}%
\pgfpathrectangle{\pgfpointorigin}{\pgfqpoint{5.800000in}{3.300000in}}%
\pgfusepath{use as bounding box, clip}%
\begin{pgfscope}%
\pgfsetbuttcap%
\pgfsetmiterjoin%
\definecolor{currentfill}{rgb}{1.000000,1.000000,1.000000}%
\pgfsetfillcolor{currentfill}%
\pgfsetlinewidth{0.000000pt}%
\definecolor{currentstroke}{rgb}{1.000000,1.000000,1.000000}%
\pgfsetstrokecolor{currentstroke}%
\pgfsetdash{}{0pt}%
\pgfpathmoveto{\pgfqpoint{0.000000in}{0.000000in}}%
\pgfpathlineto{\pgfqpoint{5.800000in}{0.000000in}}%
\pgfpathlineto{\pgfqpoint{5.800000in}{3.300000in}}%
\pgfpathlineto{\pgfqpoint{0.000000in}{3.300000in}}%
\pgfpathclose%
\pgfusepath{fill}%
\end{pgfscope}%
\begin{pgfscope}%
\pgfsetbuttcap%
\pgfsetmiterjoin%
\definecolor{currentfill}{rgb}{1.000000,1.000000,1.000000}%
\pgfsetfillcolor{currentfill}%
\pgfsetlinewidth{0.000000pt}%
\definecolor{currentstroke}{rgb}{0.000000,0.000000,0.000000}%
\pgfsetstrokecolor{currentstroke}%
\pgfsetstrokeopacity{0.000000}%
\pgfsetdash{}{0pt}%
\pgfpathmoveto{\pgfqpoint{0.670972in}{1.861111in}}%
\pgfpathlineto{\pgfqpoint{4.930000in}{1.861111in}}%
\pgfpathlineto{\pgfqpoint{4.930000in}{2.926667in}}%
\pgfpathlineto{\pgfqpoint{0.670972in}{2.926667in}}%
\pgfpathclose%
\pgfusepath{fill}%
\end{pgfscope}%
\begin{pgfscope}%
\pgfsetbuttcap%
\pgfsetroundjoin%
\definecolor{currentfill}{rgb}{0.000000,0.000000,0.000000}%
\pgfsetfillcolor{currentfill}%
\pgfsetlinewidth{0.803000pt}%
\definecolor{currentstroke}{rgb}{0.000000,0.000000,0.000000}%
\pgfsetstrokecolor{currentstroke}%
\pgfsetdash{}{0pt}%
\pgfsys@defobject{currentmarker}{\pgfqpoint{0.000000in}{-0.048611in}}{\pgfqpoint{0.000000in}{0.000000in}}{%
\pgfpathmoveto{\pgfqpoint{0.000000in}{0.000000in}}%
\pgfpathlineto{\pgfqpoint{0.000000in}{-0.048611in}}%
\pgfusepath{stroke,fill}%
}%
\begin{pgfscope}%
\pgfsys@transformshift{0.864564in}{1.861111in}%
\pgfsys@useobject{currentmarker}{}%
\end{pgfscope}%
\end{pgfscope}%
\begin{pgfscope}%
\pgfsetbuttcap%
\pgfsetroundjoin%
\definecolor{currentfill}{rgb}{0.000000,0.000000,0.000000}%
\pgfsetfillcolor{currentfill}%
\pgfsetlinewidth{0.803000pt}%
\definecolor{currentstroke}{rgb}{0.000000,0.000000,0.000000}%
\pgfsetstrokecolor{currentstroke}%
\pgfsetdash{}{0pt}%
\pgfsys@defobject{currentmarker}{\pgfqpoint{0.000000in}{-0.048611in}}{\pgfqpoint{0.000000in}{0.000000in}}{%
\pgfpathmoveto{\pgfqpoint{0.000000in}{0.000000in}}%
\pgfpathlineto{\pgfqpoint{0.000000in}{-0.048611in}}%
\pgfusepath{stroke,fill}%
}%
\begin{pgfscope}%
\pgfsys@transformshift{1.474304in}{1.861111in}%
\pgfsys@useobject{currentmarker}{}%
\end{pgfscope}%
\end{pgfscope}%
\begin{pgfscope}%
\pgfsetbuttcap%
\pgfsetroundjoin%
\definecolor{currentfill}{rgb}{0.000000,0.000000,0.000000}%
\pgfsetfillcolor{currentfill}%
\pgfsetlinewidth{0.803000pt}%
\definecolor{currentstroke}{rgb}{0.000000,0.000000,0.000000}%
\pgfsetstrokecolor{currentstroke}%
\pgfsetdash{}{0pt}%
\pgfsys@defobject{currentmarker}{\pgfqpoint{0.000000in}{-0.048611in}}{\pgfqpoint{0.000000in}{0.000000in}}{%
\pgfpathmoveto{\pgfqpoint{0.000000in}{0.000000in}}%
\pgfpathlineto{\pgfqpoint{0.000000in}{-0.048611in}}%
\pgfusepath{stroke,fill}%
}%
\begin{pgfscope}%
\pgfsys@transformshift{2.084043in}{1.861111in}%
\pgfsys@useobject{currentmarker}{}%
\end{pgfscope}%
\end{pgfscope}%
\begin{pgfscope}%
\pgfsetbuttcap%
\pgfsetroundjoin%
\definecolor{currentfill}{rgb}{0.000000,0.000000,0.000000}%
\pgfsetfillcolor{currentfill}%
\pgfsetlinewidth{0.803000pt}%
\definecolor{currentstroke}{rgb}{0.000000,0.000000,0.000000}%
\pgfsetstrokecolor{currentstroke}%
\pgfsetdash{}{0pt}%
\pgfsys@defobject{currentmarker}{\pgfqpoint{0.000000in}{-0.048611in}}{\pgfqpoint{0.000000in}{0.000000in}}{%
\pgfpathmoveto{\pgfqpoint{0.000000in}{0.000000in}}%
\pgfpathlineto{\pgfqpoint{0.000000in}{-0.048611in}}%
\pgfusepath{stroke,fill}%
}%
\begin{pgfscope}%
\pgfsys@transformshift{2.693782in}{1.861111in}%
\pgfsys@useobject{currentmarker}{}%
\end{pgfscope}%
\end{pgfscope}%
\begin{pgfscope}%
\pgfsetbuttcap%
\pgfsetroundjoin%
\definecolor{currentfill}{rgb}{0.000000,0.000000,0.000000}%
\pgfsetfillcolor{currentfill}%
\pgfsetlinewidth{0.803000pt}%
\definecolor{currentstroke}{rgb}{0.000000,0.000000,0.000000}%
\pgfsetstrokecolor{currentstroke}%
\pgfsetdash{}{0pt}%
\pgfsys@defobject{currentmarker}{\pgfqpoint{0.000000in}{-0.048611in}}{\pgfqpoint{0.000000in}{0.000000in}}{%
\pgfpathmoveto{\pgfqpoint{0.000000in}{0.000000in}}%
\pgfpathlineto{\pgfqpoint{0.000000in}{-0.048611in}}%
\pgfusepath{stroke,fill}%
}%
\begin{pgfscope}%
\pgfsys@transformshift{3.303521in}{1.861111in}%
\pgfsys@useobject{currentmarker}{}%
\end{pgfscope}%
\end{pgfscope}%
\begin{pgfscope}%
\pgfsetbuttcap%
\pgfsetroundjoin%
\definecolor{currentfill}{rgb}{0.000000,0.000000,0.000000}%
\pgfsetfillcolor{currentfill}%
\pgfsetlinewidth{0.803000pt}%
\definecolor{currentstroke}{rgb}{0.000000,0.000000,0.000000}%
\pgfsetstrokecolor{currentstroke}%
\pgfsetdash{}{0pt}%
\pgfsys@defobject{currentmarker}{\pgfqpoint{0.000000in}{-0.048611in}}{\pgfqpoint{0.000000in}{0.000000in}}{%
\pgfpathmoveto{\pgfqpoint{0.000000in}{0.000000in}}%
\pgfpathlineto{\pgfqpoint{0.000000in}{-0.048611in}}%
\pgfusepath{stroke,fill}%
}%
\begin{pgfscope}%
\pgfsys@transformshift{3.913260in}{1.861111in}%
\pgfsys@useobject{currentmarker}{}%
\end{pgfscope}%
\end{pgfscope}%
\begin{pgfscope}%
\pgfsetbuttcap%
\pgfsetroundjoin%
\definecolor{currentfill}{rgb}{0.000000,0.000000,0.000000}%
\pgfsetfillcolor{currentfill}%
\pgfsetlinewidth{0.803000pt}%
\definecolor{currentstroke}{rgb}{0.000000,0.000000,0.000000}%
\pgfsetstrokecolor{currentstroke}%
\pgfsetdash{}{0pt}%
\pgfsys@defobject{currentmarker}{\pgfqpoint{0.000000in}{-0.048611in}}{\pgfqpoint{0.000000in}{0.000000in}}{%
\pgfpathmoveto{\pgfqpoint{0.000000in}{0.000000in}}%
\pgfpathlineto{\pgfqpoint{0.000000in}{-0.048611in}}%
\pgfusepath{stroke,fill}%
}%
\begin{pgfscope}%
\pgfsys@transformshift{4.522999in}{1.861111in}%
\pgfsys@useobject{currentmarker}{}%
\end{pgfscope}%
\end{pgfscope}%
\begin{pgfscope}%
\pgfsetbuttcap%
\pgfsetroundjoin%
\definecolor{currentfill}{rgb}{0.000000,0.000000,0.000000}%
\pgfsetfillcolor{currentfill}%
\pgfsetlinewidth{0.803000pt}%
\definecolor{currentstroke}{rgb}{0.000000,0.000000,0.000000}%
\pgfsetstrokecolor{currentstroke}%
\pgfsetdash{}{0pt}%
\pgfsys@defobject{currentmarker}{\pgfqpoint{-0.048611in}{0.000000in}}{\pgfqpoint{0.000000in}{0.000000in}}{%
\pgfpathmoveto{\pgfqpoint{0.000000in}{0.000000in}}%
\pgfpathlineto{\pgfqpoint{-0.048611in}{0.000000in}}%
\pgfusepath{stroke,fill}%
}%
\begin{pgfscope}%
\pgfsys@transformshift{0.670972in}{1.909545in}%
\pgfsys@useobject{currentmarker}{}%
\end{pgfscope}%
\end{pgfscope}%
\begin{pgfscope}%
\definecolor{textcolor}{rgb}{0.000000,0.000000,0.000000}%
\pgfsetstrokecolor{textcolor}%
\pgfsetfillcolor{textcolor}%
\pgftext[x=0.504306in,y=1.861351in,left,base]{\color{textcolor}\sffamily\fontsize{10.000000}{12.000000}\selectfont 0}%
\end{pgfscope}%
\begin{pgfscope}%
\pgfsetbuttcap%
\pgfsetroundjoin%
\definecolor{currentfill}{rgb}{0.000000,0.000000,0.000000}%
\pgfsetfillcolor{currentfill}%
\pgfsetlinewidth{0.803000pt}%
\definecolor{currentstroke}{rgb}{0.000000,0.000000,0.000000}%
\pgfsetstrokecolor{currentstroke}%
\pgfsetdash{}{0pt}%
\pgfsys@defobject{currentmarker}{\pgfqpoint{-0.048611in}{0.000000in}}{\pgfqpoint{0.000000in}{0.000000in}}{%
\pgfpathmoveto{\pgfqpoint{0.000000in}{0.000000in}}%
\pgfpathlineto{\pgfqpoint{-0.048611in}{0.000000in}}%
\pgfusepath{stroke,fill}%
}%
\begin{pgfscope}%
\pgfsys@transformshift{0.670972in}{2.340172in}%
\pgfsys@useobject{currentmarker}{}%
\end{pgfscope}%
\end{pgfscope}%
\begin{pgfscope}%
\definecolor{textcolor}{rgb}{0.000000,0.000000,0.000000}%
\pgfsetstrokecolor{textcolor}%
\pgfsetfillcolor{textcolor}%
\pgftext[x=0.504306in,y=2.291977in,left,base]{\color{textcolor}\sffamily\fontsize{10.000000}{12.000000}\selectfont 5}%
\end{pgfscope}%
\begin{pgfscope}%
\pgfsetbuttcap%
\pgfsetroundjoin%
\definecolor{currentfill}{rgb}{0.000000,0.000000,0.000000}%
\pgfsetfillcolor{currentfill}%
\pgfsetlinewidth{0.803000pt}%
\definecolor{currentstroke}{rgb}{0.000000,0.000000,0.000000}%
\pgfsetstrokecolor{currentstroke}%
\pgfsetdash{}{0pt}%
\pgfsys@defobject{currentmarker}{\pgfqpoint{-0.048611in}{0.000000in}}{\pgfqpoint{0.000000in}{0.000000in}}{%
\pgfpathmoveto{\pgfqpoint{0.000000in}{0.000000in}}%
\pgfpathlineto{\pgfqpoint{-0.048611in}{0.000000in}}%
\pgfusepath{stroke,fill}%
}%
\begin{pgfscope}%
\pgfsys@transformshift{0.670972in}{2.770798in}%
\pgfsys@useobject{currentmarker}{}%
\end{pgfscope}%
\end{pgfscope}%
\begin{pgfscope}%
\definecolor{textcolor}{rgb}{0.000000,0.000000,0.000000}%
\pgfsetstrokecolor{textcolor}%
\pgfsetfillcolor{textcolor}%
\pgftext[x=0.434861in,y=2.722604in,left,base]{\color{textcolor}\sffamily\fontsize{10.000000}{12.000000}\selectfont 10}%
\end{pgfscope}%
\begin{pgfscope}%
\pgfpathrectangle{\pgfqpoint{0.670972in}{1.861111in}}{\pgfqpoint{4.259028in}{1.065556in}}%
\pgfusepath{clip}%
\pgfsetrectcap%
\pgfsetroundjoin%
\pgfsetlinewidth{1.505625pt}%
\definecolor{currentstroke}{rgb}{0.121569,0.466667,0.705882}%
\pgfsetstrokecolor{currentstroke}%
\pgfsetdash{}{0pt}%
\pgfpathmoveto{\pgfqpoint{0.864564in}{1.914939in}}%
\pgfpathlineto{\pgfqpoint{0.895051in}{1.926514in}}%
\pgfpathlineto{\pgfqpoint{0.925538in}{1.932392in}}%
\pgfpathlineto{\pgfqpoint{0.986512in}{1.940980in}}%
\pgfpathlineto{\pgfqpoint{1.077973in}{1.950614in}}%
\pgfpathlineto{\pgfqpoint{1.199921in}{1.960708in}}%
\pgfpathlineto{\pgfqpoint{1.352356in}{1.971038in}}%
\pgfpathlineto{\pgfqpoint{1.565764in}{1.983100in}}%
\pgfpathlineto{\pgfqpoint{1.840147in}{1.996175in}}%
\pgfpathlineto{\pgfqpoint{2.175504in}{2.009868in}}%
\pgfpathlineto{\pgfqpoint{2.602321in}{2.024968in}}%
\pgfpathlineto{\pgfqpoint{3.120599in}{2.040993in}}%
\pgfpathlineto{\pgfqpoint{3.760825in}{2.058426in}}%
\pgfpathlineto{\pgfqpoint{4.522999in}{2.076827in}}%
\pgfpathlineto{\pgfqpoint{4.736408in}{2.081627in}}%
\pgfpathlineto{\pgfqpoint{4.736408in}{2.081627in}}%
\pgfusepath{stroke}%
\end{pgfscope}%
\begin{pgfscope}%
\pgfpathrectangle{\pgfqpoint{0.670972in}{1.861111in}}{\pgfqpoint{4.259028in}{1.065556in}}%
\pgfusepath{clip}%
\pgfsetrectcap%
\pgfsetroundjoin%
\pgfsetlinewidth{1.505625pt}%
\definecolor{currentstroke}{rgb}{1.000000,0.498039,0.054902}%
\pgfsetstrokecolor{currentstroke}%
\pgfsetdash{}{0pt}%
\pgfpathmoveto{\pgfqpoint{0.864564in}{2.031345in}}%
\pgfpathlineto{\pgfqpoint{4.736408in}{2.031345in}}%
\pgfpathlineto{\pgfqpoint{4.736408in}{2.031345in}}%
\pgfusepath{stroke}%
\end{pgfscope}%
\begin{pgfscope}%
\pgfpathrectangle{\pgfqpoint{0.670972in}{1.861111in}}{\pgfqpoint{4.259028in}{1.065556in}}%
\pgfusepath{clip}%
\pgfsetrectcap%
\pgfsetroundjoin%
\pgfsetlinewidth{1.505625pt}%
\definecolor{currentstroke}{rgb}{0.172549,0.627451,0.172549}%
\pgfsetstrokecolor{currentstroke}%
\pgfsetdash{}{0pt}%
\pgfpathmoveto{\pgfqpoint{0.864564in}{1.910023in}}%
\pgfpathlineto{\pgfqpoint{4.736408in}{2.152667in}}%
\pgfpathlineto{\pgfqpoint{4.736408in}{2.152667in}}%
\pgfusepath{stroke}%
\end{pgfscope}%
\begin{pgfscope}%
\pgfpathrectangle{\pgfqpoint{0.670972in}{1.861111in}}{\pgfqpoint{4.259028in}{1.065556in}}%
\pgfusepath{clip}%
\pgfsetrectcap%
\pgfsetroundjoin%
\pgfsetlinewidth{1.505625pt}%
\definecolor{currentstroke}{rgb}{0.839216,0.152941,0.156863}%
\pgfsetstrokecolor{currentstroke}%
\pgfsetdash{}{0pt}%
\pgfpathmoveto{\pgfqpoint{0.864564in}{1.909549in}}%
\pgfpathlineto{\pgfqpoint{1.047486in}{1.910718in}}%
\pgfpathlineto{\pgfqpoint{1.230408in}{1.914045in}}%
\pgfpathlineto{\pgfqpoint{1.413330in}{1.919529in}}%
\pgfpathlineto{\pgfqpoint{1.596251in}{1.927171in}}%
\pgfpathlineto{\pgfqpoint{1.779173in}{1.936972in}}%
\pgfpathlineto{\pgfqpoint{1.962095in}{1.948930in}}%
\pgfpathlineto{\pgfqpoint{2.145017in}{1.963045in}}%
\pgfpathlineto{\pgfqpoint{2.327938in}{1.979319in}}%
\pgfpathlineto{\pgfqpoint{2.510860in}{1.997751in}}%
\pgfpathlineto{\pgfqpoint{2.693782in}{2.018340in}}%
\pgfpathlineto{\pgfqpoint{2.876704in}{2.041087in}}%
\pgfpathlineto{\pgfqpoint{3.059625in}{2.065992in}}%
\pgfpathlineto{\pgfqpoint{3.242547in}{2.093055in}}%
\pgfpathlineto{\pgfqpoint{3.425469in}{2.122275in}}%
\pgfpathlineto{\pgfqpoint{3.608390in}{2.153654in}}%
\pgfpathlineto{\pgfqpoint{3.791312in}{2.187190in}}%
\pgfpathlineto{\pgfqpoint{3.974234in}{2.222884in}}%
\pgfpathlineto{\pgfqpoint{4.157156in}{2.260736in}}%
\pgfpathlineto{\pgfqpoint{4.340077in}{2.300746in}}%
\pgfpathlineto{\pgfqpoint{4.522999in}{2.342914in}}%
\pgfpathlineto{\pgfqpoint{4.705921in}{2.387239in}}%
\pgfpathlineto{\pgfqpoint{4.736408in}{2.394837in}}%
\pgfpathlineto{\pgfqpoint{4.736408in}{2.394837in}}%
\pgfusepath{stroke}%
\end{pgfscope}%
\begin{pgfscope}%
\pgfpathrectangle{\pgfqpoint{0.670972in}{1.861111in}}{\pgfqpoint{4.259028in}{1.065556in}}%
\pgfusepath{clip}%
\pgfsetrectcap%
\pgfsetroundjoin%
\pgfsetlinewidth{1.505625pt}%
\definecolor{currentstroke}{rgb}{0.580392,0.403922,0.741176}%
\pgfsetstrokecolor{currentstroke}%
\pgfsetdash{}{0pt}%
\pgfpathmoveto{\pgfqpoint{0.864564in}{1.909545in}}%
\pgfpathlineto{\pgfqpoint{1.260895in}{1.910640in}}%
\pgfpathlineto{\pgfqpoint{1.474304in}{1.913451in}}%
\pgfpathlineto{\pgfqpoint{1.657225in}{1.918051in}}%
\pgfpathlineto{\pgfqpoint{1.809660in}{1.923895in}}%
\pgfpathlineto{\pgfqpoint{1.962095in}{1.931942in}}%
\pgfpathlineto{\pgfqpoint{2.114530in}{1.942546in}}%
\pgfpathlineto{\pgfqpoint{2.236477in}{1.953107in}}%
\pgfpathlineto{\pgfqpoint{2.358425in}{1.965709in}}%
\pgfpathlineto{\pgfqpoint{2.480373in}{1.980535in}}%
\pgfpathlineto{\pgfqpoint{2.602321in}{1.997763in}}%
\pgfpathlineto{\pgfqpoint{2.724269in}{2.017576in}}%
\pgfpathlineto{\pgfqpoint{2.846217in}{2.040152in}}%
\pgfpathlineto{\pgfqpoint{2.968164in}{2.065674in}}%
\pgfpathlineto{\pgfqpoint{3.059625in}{2.086856in}}%
\pgfpathlineto{\pgfqpoint{3.151086in}{2.109872in}}%
\pgfpathlineto{\pgfqpoint{3.242547in}{2.134799in}}%
\pgfpathlineto{\pgfqpoint{3.334008in}{2.161712in}}%
\pgfpathlineto{\pgfqpoint{3.425469in}{2.190688in}}%
\pgfpathlineto{\pgfqpoint{3.516930in}{2.221802in}}%
\pgfpathlineto{\pgfqpoint{3.608390in}{2.255132in}}%
\pgfpathlineto{\pgfqpoint{3.699851in}{2.290752in}}%
\pgfpathlineto{\pgfqpoint{3.791312in}{2.328740in}}%
\pgfpathlineto{\pgfqpoint{3.882773in}{2.369172in}}%
\pgfpathlineto{\pgfqpoint{3.974234in}{2.412123in}}%
\pgfpathlineto{\pgfqpoint{4.065695in}{2.457669in}}%
\pgfpathlineto{\pgfqpoint{4.157156in}{2.505888in}}%
\pgfpathlineto{\pgfqpoint{4.248617in}{2.556854in}}%
\pgfpathlineto{\pgfqpoint{4.340077in}{2.610645in}}%
\pgfpathlineto{\pgfqpoint{4.431538in}{2.667337in}}%
\pgfpathlineto{\pgfqpoint{4.522999in}{2.727004in}}%
\pgfpathlineto{\pgfqpoint{4.614460in}{2.789725in}}%
\pgfpathlineto{\pgfqpoint{4.705921in}{2.855574in}}%
\pgfpathlineto{\pgfqpoint{4.736408in}{2.878232in}}%
\pgfpathlineto{\pgfqpoint{4.736408in}{2.878232in}}%
\pgfusepath{stroke}%
\end{pgfscope}%
\begin{pgfscope}%
\pgfsetrectcap%
\pgfsetmiterjoin%
\pgfsetlinewidth{0.803000pt}%
\definecolor{currentstroke}{rgb}{0.000000,0.000000,0.000000}%
\pgfsetstrokecolor{currentstroke}%
\pgfsetdash{}{0pt}%
\pgfpathmoveto{\pgfqpoint{0.670972in}{1.861111in}}%
\pgfpathlineto{\pgfqpoint{0.670972in}{2.926667in}}%
\pgfusepath{stroke}%
\end{pgfscope}%
\begin{pgfscope}%
\pgfsetrectcap%
\pgfsetmiterjoin%
\pgfsetlinewidth{0.803000pt}%
\definecolor{currentstroke}{rgb}{0.000000,0.000000,0.000000}%
\pgfsetstrokecolor{currentstroke}%
\pgfsetdash{}{0pt}%
\pgfpathmoveto{\pgfqpoint{4.930000in}{1.861111in}}%
\pgfpathlineto{\pgfqpoint{4.930000in}{2.926667in}}%
\pgfusepath{stroke}%
\end{pgfscope}%
\begin{pgfscope}%
\pgfsetrectcap%
\pgfsetmiterjoin%
\pgfsetlinewidth{0.803000pt}%
\definecolor{currentstroke}{rgb}{0.000000,0.000000,0.000000}%
\pgfsetstrokecolor{currentstroke}%
\pgfsetdash{}{0pt}%
\pgfpathmoveto{\pgfqpoint{0.670972in}{1.861111in}}%
\pgfpathlineto{\pgfqpoint{4.930000in}{1.861111in}}%
\pgfusepath{stroke}%
\end{pgfscope}%
\begin{pgfscope}%
\pgfsetrectcap%
\pgfsetmiterjoin%
\pgfsetlinewidth{0.803000pt}%
\definecolor{currentstroke}{rgb}{0.000000,0.000000,0.000000}%
\pgfsetstrokecolor{currentstroke}%
\pgfsetdash{}{0pt}%
\pgfpathmoveto{\pgfqpoint{0.670972in}{2.926667in}}%
\pgfpathlineto{\pgfqpoint{4.930000in}{2.926667in}}%
\pgfusepath{stroke}%
\end{pgfscope}%
\begin{pgfscope}%
\definecolor{textcolor}{rgb}{0.000000,0.000000,0.000000}%
\pgfsetstrokecolor{textcolor}%
\pgfsetfillcolor{textcolor}%
\pgftext[x=2.800486in,y=3.010000in,,base]{\color{textcolor}\sffamily\fontsize{12.000000}{14.400000}\selectfont Approximation Koeffizienten}%
\end{pgfscope}%
\begin{pgfscope}%
\pgfsetbuttcap%
\pgfsetmiterjoin%
\definecolor{currentfill}{rgb}{1.000000,1.000000,1.000000}%
\pgfsetfillcolor{currentfill}%
\pgfsetlinewidth{0.000000pt}%
\definecolor{currentstroke}{rgb}{0.000000,0.000000,0.000000}%
\pgfsetstrokecolor{currentstroke}%
\pgfsetstrokeopacity{0.000000}%
\pgfsetdash{}{0pt}%
\pgfpathmoveto{\pgfqpoint{0.670972in}{0.387222in}}%
\pgfpathlineto{\pgfqpoint{4.930000in}{0.387222in}}%
\pgfpathlineto{\pgfqpoint{4.930000in}{1.452778in}}%
\pgfpathlineto{\pgfqpoint{0.670972in}{1.452778in}}%
\pgfpathclose%
\pgfusepath{fill}%
\end{pgfscope}%
\begin{pgfscope}%
\pgfsetbuttcap%
\pgfsetroundjoin%
\definecolor{currentfill}{rgb}{0.000000,0.000000,0.000000}%
\pgfsetfillcolor{currentfill}%
\pgfsetlinewidth{0.803000pt}%
\definecolor{currentstroke}{rgb}{0.000000,0.000000,0.000000}%
\pgfsetstrokecolor{currentstroke}%
\pgfsetdash{}{0pt}%
\pgfsys@defobject{currentmarker}{\pgfqpoint{0.000000in}{-0.048611in}}{\pgfqpoint{0.000000in}{0.000000in}}{%
\pgfpathmoveto{\pgfqpoint{0.000000in}{0.000000in}}%
\pgfpathlineto{\pgfqpoint{0.000000in}{-0.048611in}}%
\pgfusepath{stroke,fill}%
}%
\begin{pgfscope}%
\pgfsys@transformshift{0.864564in}{0.387222in}%
\pgfsys@useobject{currentmarker}{}%
\end{pgfscope}%
\end{pgfscope}%
\begin{pgfscope}%
\definecolor{textcolor}{rgb}{0.000000,0.000000,0.000000}%
\pgfsetstrokecolor{textcolor}%
\pgfsetfillcolor{textcolor}%
\pgftext[x=0.864564in,y=0.290000in,,top]{\color{textcolor}\sffamily\fontsize{10.000000}{12.000000}\selectfont 0}%
\end{pgfscope}%
\begin{pgfscope}%
\pgfsetbuttcap%
\pgfsetroundjoin%
\definecolor{currentfill}{rgb}{0.000000,0.000000,0.000000}%
\pgfsetfillcolor{currentfill}%
\pgfsetlinewidth{0.803000pt}%
\definecolor{currentstroke}{rgb}{0.000000,0.000000,0.000000}%
\pgfsetstrokecolor{currentstroke}%
\pgfsetdash{}{0pt}%
\pgfsys@defobject{currentmarker}{\pgfqpoint{0.000000in}{-0.048611in}}{\pgfqpoint{0.000000in}{0.000000in}}{%
\pgfpathmoveto{\pgfqpoint{0.000000in}{0.000000in}}%
\pgfpathlineto{\pgfqpoint{0.000000in}{-0.048611in}}%
\pgfusepath{stroke,fill}%
}%
\begin{pgfscope}%
\pgfsys@transformshift{1.474304in}{0.387222in}%
\pgfsys@useobject{currentmarker}{}%
\end{pgfscope}%
\end{pgfscope}%
\begin{pgfscope}%
\definecolor{textcolor}{rgb}{0.000000,0.000000,0.000000}%
\pgfsetstrokecolor{textcolor}%
\pgfsetfillcolor{textcolor}%
\pgftext[x=1.474304in,y=0.290000in,,top]{\color{textcolor}\sffamily\fontsize{10.000000}{12.000000}\selectfont 20}%
\end{pgfscope}%
\begin{pgfscope}%
\pgfsetbuttcap%
\pgfsetroundjoin%
\definecolor{currentfill}{rgb}{0.000000,0.000000,0.000000}%
\pgfsetfillcolor{currentfill}%
\pgfsetlinewidth{0.803000pt}%
\definecolor{currentstroke}{rgb}{0.000000,0.000000,0.000000}%
\pgfsetstrokecolor{currentstroke}%
\pgfsetdash{}{0pt}%
\pgfsys@defobject{currentmarker}{\pgfqpoint{0.000000in}{-0.048611in}}{\pgfqpoint{0.000000in}{0.000000in}}{%
\pgfpathmoveto{\pgfqpoint{0.000000in}{0.000000in}}%
\pgfpathlineto{\pgfqpoint{0.000000in}{-0.048611in}}%
\pgfusepath{stroke,fill}%
}%
\begin{pgfscope}%
\pgfsys@transformshift{2.084043in}{0.387222in}%
\pgfsys@useobject{currentmarker}{}%
\end{pgfscope}%
\end{pgfscope}%
\begin{pgfscope}%
\definecolor{textcolor}{rgb}{0.000000,0.000000,0.000000}%
\pgfsetstrokecolor{textcolor}%
\pgfsetfillcolor{textcolor}%
\pgftext[x=2.084043in,y=0.290000in,,top]{\color{textcolor}\sffamily\fontsize{10.000000}{12.000000}\selectfont 40}%
\end{pgfscope}%
\begin{pgfscope}%
\pgfsetbuttcap%
\pgfsetroundjoin%
\definecolor{currentfill}{rgb}{0.000000,0.000000,0.000000}%
\pgfsetfillcolor{currentfill}%
\pgfsetlinewidth{0.803000pt}%
\definecolor{currentstroke}{rgb}{0.000000,0.000000,0.000000}%
\pgfsetstrokecolor{currentstroke}%
\pgfsetdash{}{0pt}%
\pgfsys@defobject{currentmarker}{\pgfqpoint{0.000000in}{-0.048611in}}{\pgfqpoint{0.000000in}{0.000000in}}{%
\pgfpathmoveto{\pgfqpoint{0.000000in}{0.000000in}}%
\pgfpathlineto{\pgfqpoint{0.000000in}{-0.048611in}}%
\pgfusepath{stroke,fill}%
}%
\begin{pgfscope}%
\pgfsys@transformshift{2.693782in}{0.387222in}%
\pgfsys@useobject{currentmarker}{}%
\end{pgfscope}%
\end{pgfscope}%
\begin{pgfscope}%
\definecolor{textcolor}{rgb}{0.000000,0.000000,0.000000}%
\pgfsetstrokecolor{textcolor}%
\pgfsetfillcolor{textcolor}%
\pgftext[x=2.693782in,y=0.290000in,,top]{\color{textcolor}\sffamily\fontsize{10.000000}{12.000000}\selectfont 60}%
\end{pgfscope}%
\begin{pgfscope}%
\pgfsetbuttcap%
\pgfsetroundjoin%
\definecolor{currentfill}{rgb}{0.000000,0.000000,0.000000}%
\pgfsetfillcolor{currentfill}%
\pgfsetlinewidth{0.803000pt}%
\definecolor{currentstroke}{rgb}{0.000000,0.000000,0.000000}%
\pgfsetstrokecolor{currentstroke}%
\pgfsetdash{}{0pt}%
\pgfsys@defobject{currentmarker}{\pgfqpoint{0.000000in}{-0.048611in}}{\pgfqpoint{0.000000in}{0.000000in}}{%
\pgfpathmoveto{\pgfqpoint{0.000000in}{0.000000in}}%
\pgfpathlineto{\pgfqpoint{0.000000in}{-0.048611in}}%
\pgfusepath{stroke,fill}%
}%
\begin{pgfscope}%
\pgfsys@transformshift{3.303521in}{0.387222in}%
\pgfsys@useobject{currentmarker}{}%
\end{pgfscope}%
\end{pgfscope}%
\begin{pgfscope}%
\definecolor{textcolor}{rgb}{0.000000,0.000000,0.000000}%
\pgfsetstrokecolor{textcolor}%
\pgfsetfillcolor{textcolor}%
\pgftext[x=3.303521in,y=0.290000in,,top]{\color{textcolor}\sffamily\fontsize{10.000000}{12.000000}\selectfont 80}%
\end{pgfscope}%
\begin{pgfscope}%
\pgfsetbuttcap%
\pgfsetroundjoin%
\definecolor{currentfill}{rgb}{0.000000,0.000000,0.000000}%
\pgfsetfillcolor{currentfill}%
\pgfsetlinewidth{0.803000pt}%
\definecolor{currentstroke}{rgb}{0.000000,0.000000,0.000000}%
\pgfsetstrokecolor{currentstroke}%
\pgfsetdash{}{0pt}%
\pgfsys@defobject{currentmarker}{\pgfqpoint{0.000000in}{-0.048611in}}{\pgfqpoint{0.000000in}{0.000000in}}{%
\pgfpathmoveto{\pgfqpoint{0.000000in}{0.000000in}}%
\pgfpathlineto{\pgfqpoint{0.000000in}{-0.048611in}}%
\pgfusepath{stroke,fill}%
}%
\begin{pgfscope}%
\pgfsys@transformshift{3.913260in}{0.387222in}%
\pgfsys@useobject{currentmarker}{}%
\end{pgfscope}%
\end{pgfscope}%
\begin{pgfscope}%
\definecolor{textcolor}{rgb}{0.000000,0.000000,0.000000}%
\pgfsetstrokecolor{textcolor}%
\pgfsetfillcolor{textcolor}%
\pgftext[x=3.913260in,y=0.290000in,,top]{\color{textcolor}\sffamily\fontsize{10.000000}{12.000000}\selectfont 100}%
\end{pgfscope}%
\begin{pgfscope}%
\pgfsetbuttcap%
\pgfsetroundjoin%
\definecolor{currentfill}{rgb}{0.000000,0.000000,0.000000}%
\pgfsetfillcolor{currentfill}%
\pgfsetlinewidth{0.803000pt}%
\definecolor{currentstroke}{rgb}{0.000000,0.000000,0.000000}%
\pgfsetstrokecolor{currentstroke}%
\pgfsetdash{}{0pt}%
\pgfsys@defobject{currentmarker}{\pgfqpoint{0.000000in}{-0.048611in}}{\pgfqpoint{0.000000in}{0.000000in}}{%
\pgfpathmoveto{\pgfqpoint{0.000000in}{0.000000in}}%
\pgfpathlineto{\pgfqpoint{0.000000in}{-0.048611in}}%
\pgfusepath{stroke,fill}%
}%
\begin{pgfscope}%
\pgfsys@transformshift{4.522999in}{0.387222in}%
\pgfsys@useobject{currentmarker}{}%
\end{pgfscope}%
\end{pgfscope}%
\begin{pgfscope}%
\definecolor{textcolor}{rgb}{0.000000,0.000000,0.000000}%
\pgfsetstrokecolor{textcolor}%
\pgfsetfillcolor{textcolor}%
\pgftext[x=4.522999in,y=0.290000in,,top]{\color{textcolor}\sffamily\fontsize{10.000000}{12.000000}\selectfont 120}%
\end{pgfscope}%
\begin{pgfscope}%
\pgfsetbuttcap%
\pgfsetroundjoin%
\definecolor{currentfill}{rgb}{0.000000,0.000000,0.000000}%
\pgfsetfillcolor{currentfill}%
\pgfsetlinewidth{0.803000pt}%
\definecolor{currentstroke}{rgb}{0.000000,0.000000,0.000000}%
\pgfsetstrokecolor{currentstroke}%
\pgfsetdash{}{0pt}%
\pgfsys@defobject{currentmarker}{\pgfqpoint{-0.048611in}{0.000000in}}{\pgfqpoint{0.000000in}{0.000000in}}{%
\pgfpathmoveto{\pgfqpoint{0.000000in}{0.000000in}}%
\pgfpathlineto{\pgfqpoint{-0.048611in}{0.000000in}}%
\pgfusepath{stroke,fill}%
}%
\begin{pgfscope}%
\pgfsys@transformshift{0.670972in}{0.673707in}%
\pgfsys@useobject{currentmarker}{}%
\end{pgfscope}%
\end{pgfscope}%
\begin{pgfscope}%
\definecolor{textcolor}{rgb}{0.000000,0.000000,0.000000}%
\pgfsetstrokecolor{textcolor}%
\pgfsetfillcolor{textcolor}%
\pgftext[x=0.149306in,y=0.625512in,left,base]{\color{textcolor}\sffamily\fontsize{10.000000}{12.000000}\selectfont −0.050}%
\end{pgfscope}%
\begin{pgfscope}%
\pgfsetbuttcap%
\pgfsetroundjoin%
\definecolor{currentfill}{rgb}{0.000000,0.000000,0.000000}%
\pgfsetfillcolor{currentfill}%
\pgfsetlinewidth{0.803000pt}%
\definecolor{currentstroke}{rgb}{0.000000,0.000000,0.000000}%
\pgfsetstrokecolor{currentstroke}%
\pgfsetdash{}{0pt}%
\pgfsys@defobject{currentmarker}{\pgfqpoint{-0.048611in}{0.000000in}}{\pgfqpoint{0.000000in}{0.000000in}}{%
\pgfpathmoveto{\pgfqpoint{0.000000in}{0.000000in}}%
\pgfpathlineto{\pgfqpoint{-0.048611in}{0.000000in}}%
\pgfusepath{stroke,fill}%
}%
\begin{pgfscope}%
\pgfsys@transformshift{0.670972in}{1.039025in}%
\pgfsys@useobject{currentmarker}{}%
\end{pgfscope}%
\end{pgfscope}%
\begin{pgfscope}%
\definecolor{textcolor}{rgb}{0.000000,0.000000,0.000000}%
\pgfsetstrokecolor{textcolor}%
\pgfsetfillcolor{textcolor}%
\pgftext[x=0.149306in,y=0.990831in,left,base]{\color{textcolor}\sffamily\fontsize{10.000000}{12.000000}\selectfont −0.025}%
\end{pgfscope}%
\begin{pgfscope}%
\pgfsetbuttcap%
\pgfsetroundjoin%
\definecolor{currentfill}{rgb}{0.000000,0.000000,0.000000}%
\pgfsetfillcolor{currentfill}%
\pgfsetlinewidth{0.803000pt}%
\definecolor{currentstroke}{rgb}{0.000000,0.000000,0.000000}%
\pgfsetstrokecolor{currentstroke}%
\pgfsetdash{}{0pt}%
\pgfsys@defobject{currentmarker}{\pgfqpoint{-0.048611in}{0.000000in}}{\pgfqpoint{0.000000in}{0.000000in}}{%
\pgfpathmoveto{\pgfqpoint{0.000000in}{0.000000in}}%
\pgfpathlineto{\pgfqpoint{-0.048611in}{0.000000in}}%
\pgfusepath{stroke,fill}%
}%
\begin{pgfscope}%
\pgfsys@transformshift{0.670972in}{1.404343in}%
\pgfsys@useobject{currentmarker}{}%
\end{pgfscope}%
\end{pgfscope}%
\begin{pgfscope}%
\definecolor{textcolor}{rgb}{0.000000,0.000000,0.000000}%
\pgfsetstrokecolor{textcolor}%
\pgfsetfillcolor{textcolor}%
\pgftext[x=0.257361in,y=1.356149in,left,base]{\color{textcolor}\sffamily\fontsize{10.000000}{12.000000}\selectfont 0.000}%
\end{pgfscope}%
\begin{pgfscope}%
\pgfpathrectangle{\pgfqpoint{0.670972in}{0.387222in}}{\pgfqpoint{4.259028in}{1.065556in}}%
\pgfusepath{clip}%
\pgfsetrectcap%
\pgfsetroundjoin%
\pgfsetlinewidth{1.505625pt}%
\definecolor{currentstroke}{rgb}{0.121569,0.466667,0.705882}%
\pgfsetstrokecolor{currentstroke}%
\pgfsetdash{}{0pt}%
\pgfpathmoveto{\pgfqpoint{0.864564in}{0.489258in}}%
\pgfpathlineto{\pgfqpoint{0.895051in}{1.113495in}}%
\pgfpathlineto{\pgfqpoint{0.925538in}{1.188321in}}%
\pgfpathlineto{\pgfqpoint{0.956025in}{1.224747in}}%
\pgfpathlineto{\pgfqpoint{0.986512in}{1.247340in}}%
\pgfpathlineto{\pgfqpoint{1.016999in}{1.263103in}}%
\pgfpathlineto{\pgfqpoint{1.047486in}{1.274905in}}%
\pgfpathlineto{\pgfqpoint{1.077973in}{1.284169in}}%
\pgfpathlineto{\pgfqpoint{1.108460in}{1.291691in}}%
\pgfpathlineto{\pgfqpoint{1.169434in}{1.303282in}}%
\pgfpathlineto{\pgfqpoint{1.230408in}{1.311901in}}%
\pgfpathlineto{\pgfqpoint{1.321869in}{1.321493in}}%
\pgfpathlineto{\pgfqpoint{1.413330in}{1.328609in}}%
\pgfpathlineto{\pgfqpoint{1.535277in}{1.335754in}}%
\pgfpathlineto{\pgfqpoint{1.687712in}{1.342365in}}%
\pgfpathlineto{\pgfqpoint{1.901121in}{1.349061in}}%
\pgfpathlineto{\pgfqpoint{2.175504in}{1.355148in}}%
\pgfpathlineto{\pgfqpoint{2.541347in}{1.360817in}}%
\pgfpathlineto{\pgfqpoint{3.029138in}{1.366015in}}%
\pgfpathlineto{\pgfqpoint{3.699851in}{1.370840in}}%
\pgfpathlineto{\pgfqpoint{4.675434in}{1.375435in}}%
\pgfpathlineto{\pgfqpoint{4.736408in}{1.375663in}}%
\pgfpathlineto{\pgfqpoint{4.736408in}{1.375663in}}%
\pgfusepath{stroke}%
\end{pgfscope}%
\begin{pgfscope}%
\pgfpathrectangle{\pgfqpoint{0.670972in}{0.387222in}}{\pgfqpoint{4.259028in}{1.065556in}}%
\pgfusepath{clip}%
\pgfsetrectcap%
\pgfsetroundjoin%
\pgfsetlinewidth{1.505625pt}%
\definecolor{currentstroke}{rgb}{1.000000,0.498039,0.054902}%
\pgfsetstrokecolor{currentstroke}%
\pgfsetdash{}{0pt}%
\pgfpathmoveto{\pgfqpoint{0.864564in}{1.404343in}}%
\pgfpathlineto{\pgfqpoint{4.736408in}{1.404343in}}%
\pgfpathlineto{\pgfqpoint{4.736408in}{1.404343in}}%
\pgfusepath{stroke}%
\end{pgfscope}%
\begin{pgfscope}%
\pgfpathrectangle{\pgfqpoint{0.670972in}{0.387222in}}{\pgfqpoint{4.259028in}{1.065556in}}%
\pgfusepath{clip}%
\pgfsetrectcap%
\pgfsetroundjoin%
\pgfsetlinewidth{1.505625pt}%
\definecolor{currentstroke}{rgb}{0.172549,0.627451,0.172549}%
\pgfsetstrokecolor{currentstroke}%
\pgfsetdash{}{0pt}%
\pgfpathmoveto{\pgfqpoint{0.864564in}{1.323302in}}%
\pgfpathlineto{\pgfqpoint{4.736408in}{1.323302in}}%
\pgfpathlineto{\pgfqpoint{4.736408in}{1.323302in}}%
\pgfusepath{stroke}%
\end{pgfscope}%
\begin{pgfscope}%
\pgfpathrectangle{\pgfqpoint{0.670972in}{0.387222in}}{\pgfqpoint{4.259028in}{1.065556in}}%
\pgfusepath{clip}%
\pgfsetrectcap%
\pgfsetroundjoin%
\pgfsetlinewidth{1.505625pt}%
\definecolor{currentstroke}{rgb}{0.839216,0.152941,0.156863}%
\pgfsetstrokecolor{currentstroke}%
\pgfsetdash{}{0pt}%
\pgfpathmoveto{\pgfqpoint{0.864564in}{1.403708in}}%
\pgfpathlineto{\pgfqpoint{4.736408in}{1.080814in}}%
\pgfpathlineto{\pgfqpoint{4.736408in}{1.080814in}}%
\pgfusepath{stroke}%
\end{pgfscope}%
\begin{pgfscope}%
\pgfpathrectangle{\pgfqpoint{0.670972in}{0.387222in}}{\pgfqpoint{4.259028in}{1.065556in}}%
\pgfusepath{clip}%
\pgfsetrectcap%
\pgfsetroundjoin%
\pgfsetlinewidth{1.505625pt}%
\definecolor{currentstroke}{rgb}{0.580392,0.403922,0.741176}%
\pgfsetstrokecolor{currentstroke}%
\pgfsetdash{}{0pt}%
\pgfpathmoveto{\pgfqpoint{0.864564in}{1.404338in}}%
\pgfpathlineto{\pgfqpoint{0.986512in}{1.403262in}}%
\pgfpathlineto{\pgfqpoint{1.108460in}{1.400270in}}%
\pgfpathlineto{\pgfqpoint{1.230408in}{1.395365in}}%
\pgfpathlineto{\pgfqpoint{1.352356in}{1.388545in}}%
\pgfpathlineto{\pgfqpoint{1.474304in}{1.379811in}}%
\pgfpathlineto{\pgfqpoint{1.596251in}{1.369163in}}%
\pgfpathlineto{\pgfqpoint{1.718199in}{1.356600in}}%
\pgfpathlineto{\pgfqpoint{1.840147in}{1.342123in}}%
\pgfpathlineto{\pgfqpoint{1.962095in}{1.325731in}}%
\pgfpathlineto{\pgfqpoint{2.084043in}{1.307425in}}%
\pgfpathlineto{\pgfqpoint{2.205990in}{1.287205in}}%
\pgfpathlineto{\pgfqpoint{2.327938in}{1.265071in}}%
\pgfpathlineto{\pgfqpoint{2.449886in}{1.241022in}}%
\pgfpathlineto{\pgfqpoint{2.571834in}{1.215059in}}%
\pgfpathlineto{\pgfqpoint{2.693782in}{1.187181in}}%
\pgfpathlineto{\pgfqpoint{2.815730in}{1.157390in}}%
\pgfpathlineto{\pgfqpoint{2.937677in}{1.125684in}}%
\pgfpathlineto{\pgfqpoint{3.059625in}{1.092063in}}%
\pgfpathlineto{\pgfqpoint{3.181573in}{1.056528in}}%
\pgfpathlineto{\pgfqpoint{3.303521in}{1.019079in}}%
\pgfpathlineto{\pgfqpoint{3.425469in}{0.979716in}}%
\pgfpathlineto{\pgfqpoint{3.547417in}{0.938438in}}%
\pgfpathlineto{\pgfqpoint{3.669364in}{0.895246in}}%
\pgfpathlineto{\pgfqpoint{3.791312in}{0.850139in}}%
\pgfpathlineto{\pgfqpoint{3.913260in}{0.803119in}}%
\pgfpathlineto{\pgfqpoint{4.035208in}{0.754183in}}%
\pgfpathlineto{\pgfqpoint{4.187643in}{0.690323in}}%
\pgfpathlineto{\pgfqpoint{4.340077in}{0.623470in}}%
\pgfpathlineto{\pgfqpoint{4.492512in}{0.553627in}}%
\pgfpathlineto{\pgfqpoint{4.644947in}{0.480793in}}%
\pgfpathlineto{\pgfqpoint{4.736408in}{0.435657in}}%
\pgfpathlineto{\pgfqpoint{4.736408in}{0.435657in}}%
\pgfusepath{stroke}%
\end{pgfscope}%
\begin{pgfscope}%
\pgfsetrectcap%
\pgfsetmiterjoin%
\pgfsetlinewidth{0.803000pt}%
\definecolor{currentstroke}{rgb}{0.000000,0.000000,0.000000}%
\pgfsetstrokecolor{currentstroke}%
\pgfsetdash{}{0pt}%
\pgfpathmoveto{\pgfqpoint{0.670972in}{0.387222in}}%
\pgfpathlineto{\pgfqpoint{0.670972in}{1.452778in}}%
\pgfusepath{stroke}%
\end{pgfscope}%
\begin{pgfscope}%
\pgfsetrectcap%
\pgfsetmiterjoin%
\pgfsetlinewidth{0.803000pt}%
\definecolor{currentstroke}{rgb}{0.000000,0.000000,0.000000}%
\pgfsetstrokecolor{currentstroke}%
\pgfsetdash{}{0pt}%
\pgfpathmoveto{\pgfqpoint{4.930000in}{0.387222in}}%
\pgfpathlineto{\pgfqpoint{4.930000in}{1.452778in}}%
\pgfusepath{stroke}%
\end{pgfscope}%
\begin{pgfscope}%
\pgfsetrectcap%
\pgfsetmiterjoin%
\pgfsetlinewidth{0.803000pt}%
\definecolor{currentstroke}{rgb}{0.000000,0.000000,0.000000}%
\pgfsetstrokecolor{currentstroke}%
\pgfsetdash{}{0pt}%
\pgfpathmoveto{\pgfqpoint{0.670972in}{0.387222in}}%
\pgfpathlineto{\pgfqpoint{4.930000in}{0.387222in}}%
\pgfusepath{stroke}%
\end{pgfscope}%
\begin{pgfscope}%
\pgfsetrectcap%
\pgfsetmiterjoin%
\pgfsetlinewidth{0.803000pt}%
\definecolor{currentstroke}{rgb}{0.000000,0.000000,0.000000}%
\pgfsetstrokecolor{currentstroke}%
\pgfsetdash{}{0pt}%
\pgfpathmoveto{\pgfqpoint{0.670972in}{1.452778in}}%
\pgfpathlineto{\pgfqpoint{4.930000in}{1.452778in}}%
\pgfusepath{stroke}%
\end{pgfscope}%
\begin{pgfscope}%
\definecolor{textcolor}{rgb}{0.000000,0.000000,0.000000}%
\pgfsetstrokecolor{textcolor}%
\pgfsetfillcolor{textcolor}%
\pgftext[x=2.800486in,y=1.536111in,,base]{\color{textcolor}\sffamily\fontsize{12.000000}{14.400000}\selectfont Detail Koeffizienten}%
\end{pgfscope}%
\begin{pgfscope}%
\pgfsetbuttcap%
\pgfsetmiterjoin%
\definecolor{currentfill}{rgb}{1.000000,1.000000,1.000000}%
\pgfsetfillcolor{currentfill}%
\pgfsetfillopacity{0.800000}%
\pgfsetlinewidth{1.003750pt}%
\definecolor{currentstroke}{rgb}{0.800000,0.800000,0.800000}%
\pgfsetstrokecolor{currentstroke}%
\pgfsetstrokeopacity{0.800000}%
\pgfsetdash{}{0pt}%
\pgfpathmoveto{\pgfqpoint{5.083854in}{1.145139in}}%
\pgfpathlineto{\pgfqpoint{5.758333in}{1.145139in}}%
\pgfpathquadraticcurveto{\pgfqpoint{5.786111in}{1.145139in}}{\pgfqpoint{5.786111in}{1.172917in}}%
\pgfpathlineto{\pgfqpoint{5.786111in}{2.127083in}}%
\pgfpathquadraticcurveto{\pgfqpoint{5.786111in}{2.154861in}}{\pgfqpoint{5.758333in}{2.154861in}}%
\pgfpathlineto{\pgfqpoint{5.083854in}{2.154861in}}%
\pgfpathquadraticcurveto{\pgfqpoint{5.056076in}{2.154861in}}{\pgfqpoint{5.056076in}{2.127083in}}%
\pgfpathlineto{\pgfqpoint{5.056076in}{1.172917in}}%
\pgfpathquadraticcurveto{\pgfqpoint{5.056076in}{1.145139in}}{\pgfqpoint{5.083854in}{1.145139in}}%
\pgfpathclose%
\pgfusepath{stroke,fill}%
\end{pgfscope}%
\begin{pgfscope}%
\pgfsetrectcap%
\pgfsetroundjoin%
\pgfsetlinewidth{1.505625pt}%
\definecolor{currentstroke}{rgb}{0.121569,0.466667,0.705882}%
\pgfsetstrokecolor{currentstroke}%
\pgfsetdash{}{0pt}%
\pgfpathmoveto{\pgfqpoint{5.111632in}{2.050694in}}%
\pgfpathlineto{\pgfqpoint{5.389409in}{2.050694in}}%
\pgfusepath{stroke}%
\end{pgfscope}%
\begin{pgfscope}%
\definecolor{textcolor}{rgb}{0.000000,0.000000,0.000000}%
\pgfsetstrokecolor{textcolor}%
\pgfsetfillcolor{textcolor}%
\pgftext[x=5.500520in,y=2.002083in,left,base]{\color{textcolor}\sffamily\fontsize{10.000000}{12.000000}\selectfont \(\displaystyle x^{0.5}\)}%
\end{pgfscope}%
\begin{pgfscope}%
\pgfsetrectcap%
\pgfsetroundjoin%
\pgfsetlinewidth{1.505625pt}%
\definecolor{currentstroke}{rgb}{1.000000,0.498039,0.054902}%
\pgfsetstrokecolor{currentstroke}%
\pgfsetdash{}{0pt}%
\pgfpathmoveto{\pgfqpoint{5.111632in}{1.857083in}}%
\pgfpathlineto{\pgfqpoint{5.389409in}{1.857083in}}%
\pgfusepath{stroke}%
\end{pgfscope}%
\begin{pgfscope}%
\definecolor{textcolor}{rgb}{0.000000,0.000000,0.000000}%
\pgfsetstrokecolor{textcolor}%
\pgfsetfillcolor{textcolor}%
\pgftext[x=5.500520in,y=1.808472in,left,base]{\color{textcolor}\sffamily\fontsize{10.000000}{12.000000}\selectfont \(\displaystyle x^{0}\)}%
\end{pgfscope}%
\begin{pgfscope}%
\pgfsetrectcap%
\pgfsetroundjoin%
\pgfsetlinewidth{1.505625pt}%
\definecolor{currentstroke}{rgb}{0.172549,0.627451,0.172549}%
\pgfsetstrokecolor{currentstroke}%
\pgfsetdash{}{0pt}%
\pgfpathmoveto{\pgfqpoint{5.111632in}{1.663472in}}%
\pgfpathlineto{\pgfqpoint{5.389409in}{1.663472in}}%
\pgfusepath{stroke}%
\end{pgfscope}%
\begin{pgfscope}%
\definecolor{textcolor}{rgb}{0.000000,0.000000,0.000000}%
\pgfsetstrokecolor{textcolor}%
\pgfsetfillcolor{textcolor}%
\pgftext[x=5.500520in,y=1.614861in,left,base]{\color{textcolor}\sffamily\fontsize{10.000000}{12.000000}\selectfont \(\displaystyle x^{1}\)}%
\end{pgfscope}%
\begin{pgfscope}%
\pgfsetrectcap%
\pgfsetroundjoin%
\pgfsetlinewidth{1.505625pt}%
\definecolor{currentstroke}{rgb}{0.839216,0.152941,0.156863}%
\pgfsetstrokecolor{currentstroke}%
\pgfsetdash{}{0pt}%
\pgfpathmoveto{\pgfqpoint{5.111632in}{1.469861in}}%
\pgfpathlineto{\pgfqpoint{5.389409in}{1.469861in}}%
\pgfusepath{stroke}%
\end{pgfscope}%
\begin{pgfscope}%
\definecolor{textcolor}{rgb}{0.000000,0.000000,0.000000}%
\pgfsetstrokecolor{textcolor}%
\pgfsetfillcolor{textcolor}%
\pgftext[x=5.500520in,y=1.421250in,left,base]{\color{textcolor}\sffamily\fontsize{10.000000}{12.000000}\selectfont \(\displaystyle x^{2}\)}%
\end{pgfscope}%
\begin{pgfscope}%
\pgfsetrectcap%
\pgfsetroundjoin%
\pgfsetlinewidth{1.505625pt}%
\definecolor{currentstroke}{rgb}{0.580392,0.403922,0.741176}%
\pgfsetstrokecolor{currentstroke}%
\pgfsetdash{}{0pt}%
\pgfpathmoveto{\pgfqpoint{5.111632in}{1.276250in}}%
\pgfpathlineto{\pgfqpoint{5.389409in}{1.276250in}}%
\pgfusepath{stroke}%
\end{pgfscope}%
\begin{pgfscope}%
\definecolor{textcolor}{rgb}{0.000000,0.000000,0.000000}%
\pgfsetstrokecolor{textcolor}%
\pgfsetfillcolor{textcolor}%
\pgftext[x=5.500520in,y=1.227639in,left,base]{\color{textcolor}\sffamily\fontsize{10.000000}{12.000000}\selectfont \(\displaystyle x^{3}\)}%
\end{pgfscope}%
\end{pgfpicture}%
\makeatother%
\endgroup%

    \caption{Analyse mit db1 (Haar) Wavelet\label{polynomials:haar}}
\end{figure}

In \autoref{polynomials:diff} sind zum Vergleich die Ableitungen der
verschiedenen Signale geben.

\begin{figure}
    \centering
    %% Creator: Matplotlib, PGF backend
%%
%% To include the figure in your LaTeX document, write
%%   \input{<filename>.pgf}
%%
%% Make sure the required packages are loaded in your preamble
%%   \usepackage{pgf}
%%
%% Figures using additional raster images can only be included by \input if
%% they are in the same directory as the main LaTeX file. For loading figures
%% from other directories you can use the `import` package
%%   \usepackage{import}
%% and then include the figures with
%%   \import{<path to file>}{<filename>.pgf}
%%
%% Matplotlib used the following preamble
%%   \usepackage{fontspec}
%%
\begingroup%
\makeatletter%
\begin{pgfpicture}%
\pgfpathrectangle{\pgfpointorigin}{\pgfqpoint{2.900000in}{3.000000in}}%
\pgfusepath{use as bounding box, clip}%
\begin{pgfscope}%
\pgfsetbuttcap%
\pgfsetmiterjoin%
\definecolor{currentfill}{rgb}{1.000000,1.000000,1.000000}%
\pgfsetfillcolor{currentfill}%
\pgfsetlinewidth{0.000000pt}%
\definecolor{currentstroke}{rgb}{1.000000,1.000000,1.000000}%
\pgfsetstrokecolor{currentstroke}%
\pgfsetdash{}{0pt}%
\pgfpathmoveto{\pgfqpoint{0.000000in}{0.000000in}}%
\pgfpathlineto{\pgfqpoint{2.900000in}{0.000000in}}%
\pgfpathlineto{\pgfqpoint{2.900000in}{3.000000in}}%
\pgfpathlineto{\pgfqpoint{0.000000in}{3.000000in}}%
\pgfpathclose%
\pgfusepath{fill}%
\end{pgfscope}%
\begin{pgfscope}%
\pgfsetbuttcap%
\pgfsetmiterjoin%
\definecolor{currentfill}{rgb}{1.000000,1.000000,1.000000}%
\pgfsetfillcolor{currentfill}%
\pgfsetlinewidth{0.000000pt}%
\definecolor{currentstroke}{rgb}{0.000000,0.000000,0.000000}%
\pgfsetstrokecolor{currentstroke}%
\pgfsetstrokeopacity{0.000000}%
\pgfsetdash{}{0pt}%
\pgfpathmoveto{\pgfqpoint{0.362500in}{0.375000in}}%
\pgfpathlineto{\pgfqpoint{2.610000in}{0.375000in}}%
\pgfpathlineto{\pgfqpoint{2.610000in}{2.640000in}}%
\pgfpathlineto{\pgfqpoint{0.362500in}{2.640000in}}%
\pgfpathclose%
\pgfusepath{fill}%
\end{pgfscope}%
\begin{pgfscope}%
\pgfpathrectangle{\pgfqpoint{0.362500in}{0.375000in}}{\pgfqpoint{2.247500in}{2.265000in}}%
\pgfusepath{clip}%
\pgfsetrectcap%
\pgfsetroundjoin%
\pgfsetlinewidth{0.803000pt}%
\definecolor{currentstroke}{rgb}{0.690196,0.690196,0.690196}%
\pgfsetstrokecolor{currentstroke}%
\pgfsetdash{}{0pt}%
\pgfpathmoveto{\pgfqpoint{0.464659in}{0.375000in}}%
\pgfpathlineto{\pgfqpoint{0.464659in}{2.640000in}}%
\pgfusepath{stroke}%
\end{pgfscope}%
\begin{pgfscope}%
\pgfsetbuttcap%
\pgfsetroundjoin%
\definecolor{currentfill}{rgb}{0.000000,0.000000,0.000000}%
\pgfsetfillcolor{currentfill}%
\pgfsetlinewidth{0.803000pt}%
\definecolor{currentstroke}{rgb}{0.000000,0.000000,0.000000}%
\pgfsetstrokecolor{currentstroke}%
\pgfsetdash{}{0pt}%
\pgfsys@defobject{currentmarker}{\pgfqpoint{0.000000in}{-0.048611in}}{\pgfqpoint{0.000000in}{0.000000in}}{%
\pgfpathmoveto{\pgfqpoint{0.000000in}{0.000000in}}%
\pgfpathlineto{\pgfqpoint{0.000000in}{-0.048611in}}%
\pgfusepath{stroke,fill}%
}%
\begin{pgfscope}%
\pgfsys@transformshift{0.464659in}{0.375000in}%
\pgfsys@useobject{currentmarker}{}%
\end{pgfscope}%
\end{pgfscope}%
\begin{pgfscope}%
\definecolor{textcolor}{rgb}{0.000000,0.000000,0.000000}%
\pgfsetstrokecolor{textcolor}%
\pgfsetfillcolor{textcolor}%
\pgftext[x=0.464659in,y=0.277778in,,top]{\color{textcolor}\rmfamily\fontsize{10.000000}{12.000000}\selectfont 0}%
\end{pgfscope}%
\begin{pgfscope}%
\pgfpathrectangle{\pgfqpoint{0.362500in}{0.375000in}}{\pgfqpoint{2.247500in}{2.265000in}}%
\pgfusepath{clip}%
\pgfsetrectcap%
\pgfsetroundjoin%
\pgfsetlinewidth{0.803000pt}%
\definecolor{currentstroke}{rgb}{0.690196,0.690196,0.690196}%
\pgfsetstrokecolor{currentstroke}%
\pgfsetdash{}{0pt}%
\pgfpathmoveto{\pgfqpoint{1.269061in}{0.375000in}}%
\pgfpathlineto{\pgfqpoint{1.269061in}{2.640000in}}%
\pgfusepath{stroke}%
\end{pgfscope}%
\begin{pgfscope}%
\pgfsetbuttcap%
\pgfsetroundjoin%
\definecolor{currentfill}{rgb}{0.000000,0.000000,0.000000}%
\pgfsetfillcolor{currentfill}%
\pgfsetlinewidth{0.803000pt}%
\definecolor{currentstroke}{rgb}{0.000000,0.000000,0.000000}%
\pgfsetstrokecolor{currentstroke}%
\pgfsetdash{}{0pt}%
\pgfsys@defobject{currentmarker}{\pgfqpoint{0.000000in}{-0.048611in}}{\pgfqpoint{0.000000in}{0.000000in}}{%
\pgfpathmoveto{\pgfqpoint{0.000000in}{0.000000in}}%
\pgfpathlineto{\pgfqpoint{0.000000in}{-0.048611in}}%
\pgfusepath{stroke,fill}%
}%
\begin{pgfscope}%
\pgfsys@transformshift{1.269061in}{0.375000in}%
\pgfsys@useobject{currentmarker}{}%
\end{pgfscope}%
\end{pgfscope}%
\begin{pgfscope}%
\definecolor{textcolor}{rgb}{0.000000,0.000000,0.000000}%
\pgfsetstrokecolor{textcolor}%
\pgfsetfillcolor{textcolor}%
\pgftext[x=1.269061in,y=0.277778in,,top]{\color{textcolor}\rmfamily\fontsize{10.000000}{12.000000}\selectfont 100}%
\end{pgfscope}%
\begin{pgfscope}%
\pgfpathrectangle{\pgfqpoint{0.362500in}{0.375000in}}{\pgfqpoint{2.247500in}{2.265000in}}%
\pgfusepath{clip}%
\pgfsetrectcap%
\pgfsetroundjoin%
\pgfsetlinewidth{0.803000pt}%
\definecolor{currentstroke}{rgb}{0.690196,0.690196,0.690196}%
\pgfsetstrokecolor{currentstroke}%
\pgfsetdash{}{0pt}%
\pgfpathmoveto{\pgfqpoint{2.073464in}{0.375000in}}%
\pgfpathlineto{\pgfqpoint{2.073464in}{2.640000in}}%
\pgfusepath{stroke}%
\end{pgfscope}%
\begin{pgfscope}%
\pgfsetbuttcap%
\pgfsetroundjoin%
\definecolor{currentfill}{rgb}{0.000000,0.000000,0.000000}%
\pgfsetfillcolor{currentfill}%
\pgfsetlinewidth{0.803000pt}%
\definecolor{currentstroke}{rgb}{0.000000,0.000000,0.000000}%
\pgfsetstrokecolor{currentstroke}%
\pgfsetdash{}{0pt}%
\pgfsys@defobject{currentmarker}{\pgfqpoint{0.000000in}{-0.048611in}}{\pgfqpoint{0.000000in}{0.000000in}}{%
\pgfpathmoveto{\pgfqpoint{0.000000in}{0.000000in}}%
\pgfpathlineto{\pgfqpoint{0.000000in}{-0.048611in}}%
\pgfusepath{stroke,fill}%
}%
\begin{pgfscope}%
\pgfsys@transformshift{2.073464in}{0.375000in}%
\pgfsys@useobject{currentmarker}{}%
\end{pgfscope}%
\end{pgfscope}%
\begin{pgfscope}%
\definecolor{textcolor}{rgb}{0.000000,0.000000,0.000000}%
\pgfsetstrokecolor{textcolor}%
\pgfsetfillcolor{textcolor}%
\pgftext[x=2.073464in,y=0.277778in,,top]{\color{textcolor}\rmfamily\fontsize{10.000000}{12.000000}\selectfont 200}%
\end{pgfscope}%
\begin{pgfscope}%
\pgfpathrectangle{\pgfqpoint{0.362500in}{0.375000in}}{\pgfqpoint{2.247500in}{2.265000in}}%
\pgfusepath{clip}%
\pgfsetrectcap%
\pgfsetroundjoin%
\pgfsetlinewidth{0.803000pt}%
\definecolor{currentstroke}{rgb}{0.690196,0.690196,0.690196}%
\pgfsetstrokecolor{currentstroke}%
\pgfsetdash{}{0pt}%
\pgfpathmoveto{\pgfqpoint{0.362500in}{0.477955in}}%
\pgfpathlineto{\pgfqpoint{2.610000in}{0.477955in}}%
\pgfusepath{stroke}%
\end{pgfscope}%
\begin{pgfscope}%
\pgfsetbuttcap%
\pgfsetroundjoin%
\definecolor{currentfill}{rgb}{0.000000,0.000000,0.000000}%
\pgfsetfillcolor{currentfill}%
\pgfsetlinewidth{0.803000pt}%
\definecolor{currentstroke}{rgb}{0.000000,0.000000,0.000000}%
\pgfsetstrokecolor{currentstroke}%
\pgfsetdash{}{0pt}%
\pgfsys@defobject{currentmarker}{\pgfqpoint{-0.048611in}{0.000000in}}{\pgfqpoint{0.000000in}{0.000000in}}{%
\pgfpathmoveto{\pgfqpoint{0.000000in}{0.000000in}}%
\pgfpathlineto{\pgfqpoint{-0.048611in}{0.000000in}}%
\pgfusepath{stroke,fill}%
}%
\begin{pgfscope}%
\pgfsys@transformshift{0.362500in}{0.477955in}%
\pgfsys@useobject{currentmarker}{}%
\end{pgfscope}%
\end{pgfscope}%
\begin{pgfscope}%
\definecolor{textcolor}{rgb}{0.000000,0.000000,0.000000}%
\pgfsetstrokecolor{textcolor}%
\pgfsetfillcolor{textcolor}%
\pgftext[x=0.018333in,y=0.429760in,left,base]{\color{textcolor}\rmfamily\fontsize{10.000000}{12.000000}\selectfont 0.00}%
\end{pgfscope}%
\begin{pgfscope}%
\pgfpathrectangle{\pgfqpoint{0.362500in}{0.375000in}}{\pgfqpoint{2.247500in}{2.265000in}}%
\pgfusepath{clip}%
\pgfsetrectcap%
\pgfsetroundjoin%
\pgfsetlinewidth{0.803000pt}%
\definecolor{currentstroke}{rgb}{0.690196,0.690196,0.690196}%
\pgfsetstrokecolor{currentstroke}%
\pgfsetdash{}{0pt}%
\pgfpathmoveto{\pgfqpoint{0.362500in}{0.917232in}}%
\pgfpathlineto{\pgfqpoint{2.610000in}{0.917232in}}%
\pgfusepath{stroke}%
\end{pgfscope}%
\begin{pgfscope}%
\pgfsetbuttcap%
\pgfsetroundjoin%
\definecolor{currentfill}{rgb}{0.000000,0.000000,0.000000}%
\pgfsetfillcolor{currentfill}%
\pgfsetlinewidth{0.803000pt}%
\definecolor{currentstroke}{rgb}{0.000000,0.000000,0.000000}%
\pgfsetstrokecolor{currentstroke}%
\pgfsetdash{}{0pt}%
\pgfsys@defobject{currentmarker}{\pgfqpoint{-0.048611in}{0.000000in}}{\pgfqpoint{0.000000in}{0.000000in}}{%
\pgfpathmoveto{\pgfqpoint{0.000000in}{0.000000in}}%
\pgfpathlineto{\pgfqpoint{-0.048611in}{0.000000in}}%
\pgfusepath{stroke,fill}%
}%
\begin{pgfscope}%
\pgfsys@transformshift{0.362500in}{0.917232in}%
\pgfsys@useobject{currentmarker}{}%
\end{pgfscope}%
\end{pgfscope}%
\begin{pgfscope}%
\definecolor{textcolor}{rgb}{0.000000,0.000000,0.000000}%
\pgfsetstrokecolor{textcolor}%
\pgfsetfillcolor{textcolor}%
\pgftext[x=0.018333in,y=0.869037in,left,base]{\color{textcolor}\rmfamily\fontsize{10.000000}{12.000000}\selectfont 0.02}%
\end{pgfscope}%
\begin{pgfscope}%
\pgfpathrectangle{\pgfqpoint{0.362500in}{0.375000in}}{\pgfqpoint{2.247500in}{2.265000in}}%
\pgfusepath{clip}%
\pgfsetrectcap%
\pgfsetroundjoin%
\pgfsetlinewidth{0.803000pt}%
\definecolor{currentstroke}{rgb}{0.690196,0.690196,0.690196}%
\pgfsetstrokecolor{currentstroke}%
\pgfsetdash{}{0pt}%
\pgfpathmoveto{\pgfqpoint{0.362500in}{1.356509in}}%
\pgfpathlineto{\pgfqpoint{2.610000in}{1.356509in}}%
\pgfusepath{stroke}%
\end{pgfscope}%
\begin{pgfscope}%
\pgfsetbuttcap%
\pgfsetroundjoin%
\definecolor{currentfill}{rgb}{0.000000,0.000000,0.000000}%
\pgfsetfillcolor{currentfill}%
\pgfsetlinewidth{0.803000pt}%
\definecolor{currentstroke}{rgb}{0.000000,0.000000,0.000000}%
\pgfsetstrokecolor{currentstroke}%
\pgfsetdash{}{0pt}%
\pgfsys@defobject{currentmarker}{\pgfqpoint{-0.048611in}{0.000000in}}{\pgfqpoint{0.000000in}{0.000000in}}{%
\pgfpathmoveto{\pgfqpoint{0.000000in}{0.000000in}}%
\pgfpathlineto{\pgfqpoint{-0.048611in}{0.000000in}}%
\pgfusepath{stroke,fill}%
}%
\begin{pgfscope}%
\pgfsys@transformshift{0.362500in}{1.356509in}%
\pgfsys@useobject{currentmarker}{}%
\end{pgfscope}%
\end{pgfscope}%
\begin{pgfscope}%
\definecolor{textcolor}{rgb}{0.000000,0.000000,0.000000}%
\pgfsetstrokecolor{textcolor}%
\pgfsetfillcolor{textcolor}%
\pgftext[x=0.018333in,y=1.308315in,left,base]{\color{textcolor}\rmfamily\fontsize{10.000000}{12.000000}\selectfont 0.04}%
\end{pgfscope}%
\begin{pgfscope}%
\pgfpathrectangle{\pgfqpoint{0.362500in}{0.375000in}}{\pgfqpoint{2.247500in}{2.265000in}}%
\pgfusepath{clip}%
\pgfsetrectcap%
\pgfsetroundjoin%
\pgfsetlinewidth{0.803000pt}%
\definecolor{currentstroke}{rgb}{0.690196,0.690196,0.690196}%
\pgfsetstrokecolor{currentstroke}%
\pgfsetdash{}{0pt}%
\pgfpathmoveto{\pgfqpoint{0.362500in}{1.795786in}}%
\pgfpathlineto{\pgfqpoint{2.610000in}{1.795786in}}%
\pgfusepath{stroke}%
\end{pgfscope}%
\begin{pgfscope}%
\pgfsetbuttcap%
\pgfsetroundjoin%
\definecolor{currentfill}{rgb}{0.000000,0.000000,0.000000}%
\pgfsetfillcolor{currentfill}%
\pgfsetlinewidth{0.803000pt}%
\definecolor{currentstroke}{rgb}{0.000000,0.000000,0.000000}%
\pgfsetstrokecolor{currentstroke}%
\pgfsetdash{}{0pt}%
\pgfsys@defobject{currentmarker}{\pgfqpoint{-0.048611in}{0.000000in}}{\pgfqpoint{0.000000in}{0.000000in}}{%
\pgfpathmoveto{\pgfqpoint{0.000000in}{0.000000in}}%
\pgfpathlineto{\pgfqpoint{-0.048611in}{0.000000in}}%
\pgfusepath{stroke,fill}%
}%
\begin{pgfscope}%
\pgfsys@transformshift{0.362500in}{1.795786in}%
\pgfsys@useobject{currentmarker}{}%
\end{pgfscope}%
\end{pgfscope}%
\begin{pgfscope}%
\definecolor{textcolor}{rgb}{0.000000,0.000000,0.000000}%
\pgfsetstrokecolor{textcolor}%
\pgfsetfillcolor{textcolor}%
\pgftext[x=0.018333in,y=1.747592in,left,base]{\color{textcolor}\rmfamily\fontsize{10.000000}{12.000000}\selectfont 0.06}%
\end{pgfscope}%
\begin{pgfscope}%
\pgfpathrectangle{\pgfqpoint{0.362500in}{0.375000in}}{\pgfqpoint{2.247500in}{2.265000in}}%
\pgfusepath{clip}%
\pgfsetrectcap%
\pgfsetroundjoin%
\pgfsetlinewidth{0.803000pt}%
\definecolor{currentstroke}{rgb}{0.690196,0.690196,0.690196}%
\pgfsetstrokecolor{currentstroke}%
\pgfsetdash{}{0pt}%
\pgfpathmoveto{\pgfqpoint{0.362500in}{2.235063in}}%
\pgfpathlineto{\pgfqpoint{2.610000in}{2.235063in}}%
\pgfusepath{stroke}%
\end{pgfscope}%
\begin{pgfscope}%
\pgfsetbuttcap%
\pgfsetroundjoin%
\definecolor{currentfill}{rgb}{0.000000,0.000000,0.000000}%
\pgfsetfillcolor{currentfill}%
\pgfsetlinewidth{0.803000pt}%
\definecolor{currentstroke}{rgb}{0.000000,0.000000,0.000000}%
\pgfsetstrokecolor{currentstroke}%
\pgfsetdash{}{0pt}%
\pgfsys@defobject{currentmarker}{\pgfqpoint{-0.048611in}{0.000000in}}{\pgfqpoint{0.000000in}{0.000000in}}{%
\pgfpathmoveto{\pgfqpoint{0.000000in}{0.000000in}}%
\pgfpathlineto{\pgfqpoint{-0.048611in}{0.000000in}}%
\pgfusepath{stroke,fill}%
}%
\begin{pgfscope}%
\pgfsys@transformshift{0.362500in}{2.235063in}%
\pgfsys@useobject{currentmarker}{}%
\end{pgfscope}%
\end{pgfscope}%
\begin{pgfscope}%
\definecolor{textcolor}{rgb}{0.000000,0.000000,0.000000}%
\pgfsetstrokecolor{textcolor}%
\pgfsetfillcolor{textcolor}%
\pgftext[x=0.018333in,y=2.186869in,left,base]{\color{textcolor}\rmfamily\fontsize{10.000000}{12.000000}\selectfont 0.08}%
\end{pgfscope}%
\begin{pgfscope}%
\pgfpathrectangle{\pgfqpoint{0.362500in}{0.375000in}}{\pgfqpoint{2.247500in}{2.265000in}}%
\pgfusepath{clip}%
\pgfsetrectcap%
\pgfsetroundjoin%
\pgfsetlinewidth{1.505625pt}%
\definecolor{currentstroke}{rgb}{0.121569,0.466667,0.705882}%
\pgfsetstrokecolor{currentstroke}%
\pgfsetdash{}{0pt}%
\pgfpathmoveto{\pgfqpoint{0.464659in}{2.423107in}}%
\pgfpathlineto{\pgfqpoint{0.472703in}{1.283663in}}%
\pgfpathlineto{\pgfqpoint{0.480747in}{1.096196in}}%
\pgfpathlineto{\pgfqpoint{0.488791in}{0.999157in}}%
\pgfpathlineto{\pgfqpoint{0.496835in}{0.937143in}}%
\pgfpathlineto{\pgfqpoint{0.504879in}{0.893092in}}%
\pgfpathlineto{\pgfqpoint{0.512923in}{0.859713in}}%
\pgfpathlineto{\pgfqpoint{0.529011in}{0.811690in}}%
\pgfpathlineto{\pgfqpoint{0.545099in}{0.778183in}}%
\pgfpathlineto{\pgfqpoint{0.561187in}{0.753096in}}%
\pgfpathlineto{\pgfqpoint{0.577275in}{0.733404in}}%
\pgfpathlineto{\pgfqpoint{0.601407in}{0.710468in}}%
\pgfpathlineto{\pgfqpoint{0.625540in}{0.692777in}}%
\pgfpathlineto{\pgfqpoint{0.649672in}{0.678593in}}%
\pgfpathlineto{\pgfqpoint{0.681848in}{0.663425in}}%
\pgfpathlineto{\pgfqpoint{0.722068in}{0.648561in}}%
\pgfpathlineto{\pgfqpoint{0.770332in}{0.634703in}}%
\pgfpathlineto{\pgfqpoint{0.826640in}{0.622141in}}%
\pgfpathlineto{\pgfqpoint{0.890992in}{0.610924in}}%
\pgfpathlineto{\pgfqpoint{0.971433in}{0.600005in}}%
\pgfpathlineto{\pgfqpoint{1.076005in}{0.589152in}}%
\pgfpathlineto{\pgfqpoint{1.204709in}{0.579079in}}%
\pgfpathlineto{\pgfqpoint{1.365590in}{0.569650in}}%
\pgfpathlineto{\pgfqpoint{1.574734in}{0.560596in}}%
\pgfpathlineto{\pgfqpoint{1.848231in}{0.552005in}}%
\pgfpathlineto{\pgfqpoint{2.210212in}{0.543901in}}%
\pgfpathlineto{\pgfqpoint{2.507841in}{0.538919in}}%
\pgfpathlineto{\pgfqpoint{2.507841in}{0.538919in}}%
\pgfusepath{stroke}%
\end{pgfscope}%
\begin{pgfscope}%
\pgfpathrectangle{\pgfqpoint{0.362500in}{0.375000in}}{\pgfqpoint{2.247500in}{2.265000in}}%
\pgfusepath{clip}%
\pgfsetrectcap%
\pgfsetroundjoin%
\pgfsetlinewidth{1.505625pt}%
\definecolor{currentstroke}{rgb}{1.000000,0.498039,0.054902}%
\pgfsetstrokecolor{currentstroke}%
\pgfsetdash{}{0pt}%
\pgfpathmoveto{\pgfqpoint{0.464659in}{0.477955in}}%
\pgfpathlineto{\pgfqpoint{2.507841in}{0.477955in}}%
\pgfpathlineto{\pgfqpoint{2.507841in}{0.477955in}}%
\pgfusepath{stroke}%
\end{pgfscope}%
\begin{pgfscope}%
\pgfpathrectangle{\pgfqpoint{0.362500in}{0.375000in}}{\pgfqpoint{2.247500in}{2.265000in}}%
\pgfusepath{clip}%
\pgfsetrectcap%
\pgfsetroundjoin%
\pgfsetlinewidth{1.505625pt}%
\definecolor{currentstroke}{rgb}{0.172549,0.627451,0.172549}%
\pgfsetstrokecolor{currentstroke}%
\pgfsetdash{}{0pt}%
\pgfpathmoveto{\pgfqpoint{0.464659in}{0.650220in}}%
\pgfpathlineto{\pgfqpoint{2.507841in}{0.650220in}}%
\pgfpathlineto{\pgfqpoint{2.507841in}{0.650220in}}%
\pgfusepath{stroke}%
\end{pgfscope}%
\begin{pgfscope}%
\pgfpathrectangle{\pgfqpoint{0.362500in}{0.375000in}}{\pgfqpoint{2.247500in}{2.265000in}}%
\pgfusepath{clip}%
\pgfsetrectcap%
\pgfsetroundjoin%
\pgfsetlinewidth{1.505625pt}%
\definecolor{currentstroke}{rgb}{0.839216,0.152941,0.156863}%
\pgfsetstrokecolor{currentstroke}%
\pgfsetdash{}{0pt}%
\pgfpathmoveto{\pgfqpoint{0.464659in}{0.479306in}}%
\pgfpathlineto{\pgfqpoint{2.507841in}{1.165666in}}%
\pgfpathlineto{\pgfqpoint{2.507841in}{1.165666in}}%
\pgfusepath{stroke}%
\end{pgfscope}%
\begin{pgfscope}%
\pgfpathrectangle{\pgfqpoint{0.362500in}{0.375000in}}{\pgfqpoint{2.247500in}{2.265000in}}%
\pgfusepath{clip}%
\pgfsetrectcap%
\pgfsetroundjoin%
\pgfsetlinewidth{1.505625pt}%
\definecolor{currentstroke}{rgb}{0.580392,0.403922,0.741176}%
\pgfsetstrokecolor{currentstroke}%
\pgfsetdash{}{0pt}%
\pgfpathmoveto{\pgfqpoint{0.464659in}{0.477965in}}%
\pgfpathlineto{\pgfqpoint{0.520967in}{0.479745in}}%
\pgfpathlineto{\pgfqpoint{0.577275in}{0.484641in}}%
\pgfpathlineto{\pgfqpoint{0.633584in}{0.492652in}}%
\pgfpathlineto{\pgfqpoint{0.689892in}{0.503779in}}%
\pgfpathlineto{\pgfqpoint{0.746200in}{0.518021in}}%
\pgfpathlineto{\pgfqpoint{0.802508in}{0.535379in}}%
\pgfpathlineto{\pgfqpoint{0.858816in}{0.555852in}}%
\pgfpathlineto{\pgfqpoint{0.915124in}{0.579441in}}%
\pgfpathlineto{\pgfqpoint{0.971433in}{0.606145in}}%
\pgfpathlineto{\pgfqpoint{1.027741in}{0.635965in}}%
\pgfpathlineto{\pgfqpoint{1.084049in}{0.668900in}}%
\pgfpathlineto{\pgfqpoint{1.148401in}{0.710355in}}%
\pgfpathlineto{\pgfqpoint{1.212753in}{0.755879in}}%
\pgfpathlineto{\pgfqpoint{1.277105in}{0.805472in}}%
\pgfpathlineto{\pgfqpoint{1.341458in}{0.859135in}}%
\pgfpathlineto{\pgfqpoint{1.405810in}{0.916867in}}%
\pgfpathlineto{\pgfqpoint{1.470162in}{0.978668in}}%
\pgfpathlineto{\pgfqpoint{1.534514in}{1.044538in}}%
\pgfpathlineto{\pgfqpoint{1.598866in}{1.114478in}}%
\pgfpathlineto{\pgfqpoint{1.671263in}{1.198023in}}%
\pgfpathlineto{\pgfqpoint{1.743659in}{1.286719in}}%
\pgfpathlineto{\pgfqpoint{1.816055in}{1.380565in}}%
\pgfpathlineto{\pgfqpoint{1.888451in}{1.479561in}}%
\pgfpathlineto{\pgfqpoint{1.960847in}{1.583707in}}%
\pgfpathlineto{\pgfqpoint{2.033244in}{1.693004in}}%
\pgfpathlineto{\pgfqpoint{2.105640in}{1.807450in}}%
\pgfpathlineto{\pgfqpoint{2.186080in}{1.940653in}}%
\pgfpathlineto{\pgfqpoint{2.266520in}{2.080214in}}%
\pgfpathlineto{\pgfqpoint{2.346960in}{2.226133in}}%
\pgfpathlineto{\pgfqpoint{2.427401in}{2.378410in}}%
\pgfpathlineto{\pgfqpoint{2.507841in}{2.537045in}}%
\pgfpathlineto{\pgfqpoint{2.507841in}{2.537045in}}%
\pgfusepath{stroke}%
\end{pgfscope}%
\begin{pgfscope}%
\pgfsetrectcap%
\pgfsetmiterjoin%
\pgfsetlinewidth{0.803000pt}%
\definecolor{currentstroke}{rgb}{0.000000,0.000000,0.000000}%
\pgfsetstrokecolor{currentstroke}%
\pgfsetdash{}{0pt}%
\pgfpathmoveto{\pgfqpoint{0.362500in}{0.375000in}}%
\pgfpathlineto{\pgfqpoint{0.362500in}{2.640000in}}%
\pgfusepath{stroke}%
\end{pgfscope}%
\begin{pgfscope}%
\pgfsetrectcap%
\pgfsetmiterjoin%
\pgfsetlinewidth{0.803000pt}%
\definecolor{currentstroke}{rgb}{0.000000,0.000000,0.000000}%
\pgfsetstrokecolor{currentstroke}%
\pgfsetdash{}{0pt}%
\pgfpathmoveto{\pgfqpoint{2.610000in}{0.375000in}}%
\pgfpathlineto{\pgfqpoint{2.610000in}{2.640000in}}%
\pgfusepath{stroke}%
\end{pgfscope}%
\begin{pgfscope}%
\pgfsetrectcap%
\pgfsetmiterjoin%
\pgfsetlinewidth{0.803000pt}%
\definecolor{currentstroke}{rgb}{0.000000,0.000000,0.000000}%
\pgfsetstrokecolor{currentstroke}%
\pgfsetdash{}{0pt}%
\pgfpathmoveto{\pgfqpoint{0.362500in}{0.375000in}}%
\pgfpathlineto{\pgfqpoint{2.610000in}{0.375000in}}%
\pgfusepath{stroke}%
\end{pgfscope}%
\begin{pgfscope}%
\pgfsetrectcap%
\pgfsetmiterjoin%
\pgfsetlinewidth{0.803000pt}%
\definecolor{currentstroke}{rgb}{0.000000,0.000000,0.000000}%
\pgfsetstrokecolor{currentstroke}%
\pgfsetdash{}{0pt}%
\pgfpathmoveto{\pgfqpoint{0.362500in}{2.640000in}}%
\pgfpathlineto{\pgfqpoint{2.610000in}{2.640000in}}%
\pgfusepath{stroke}%
\end{pgfscope}%
\begin{pgfscope}%
\pgfsetbuttcap%
\pgfsetmiterjoin%
\definecolor{currentfill}{rgb}{1.000000,1.000000,1.000000}%
\pgfsetfillcolor{currentfill}%
\pgfsetfillopacity{0.800000}%
\pgfsetlinewidth{1.003750pt}%
\definecolor{currentstroke}{rgb}{0.800000,0.800000,0.800000}%
\pgfsetstrokecolor{currentstroke}%
\pgfsetstrokeopacity{0.800000}%
\pgfsetdash{}{0pt}%
\pgfpathmoveto{\pgfqpoint{1.149010in}{1.560834in}}%
\pgfpathlineto{\pgfqpoint{1.823490in}{1.560834in}}%
\pgfpathquadraticcurveto{\pgfqpoint{1.851268in}{1.560834in}}{\pgfqpoint{1.851268in}{1.588612in}}%
\pgfpathlineto{\pgfqpoint{1.851268in}{2.542778in}}%
\pgfpathquadraticcurveto{\pgfqpoint{1.851268in}{2.570556in}}{\pgfqpoint{1.823490in}{2.570556in}}%
\pgfpathlineto{\pgfqpoint{1.149010in}{2.570556in}}%
\pgfpathquadraticcurveto{\pgfqpoint{1.121232in}{2.570556in}}{\pgfqpoint{1.121232in}{2.542778in}}%
\pgfpathlineto{\pgfqpoint{1.121232in}{1.588612in}}%
\pgfpathquadraticcurveto{\pgfqpoint{1.121232in}{1.560834in}}{\pgfqpoint{1.149010in}{1.560834in}}%
\pgfpathclose%
\pgfusepath{stroke,fill}%
\end{pgfscope}%
\begin{pgfscope}%
\pgfsetrectcap%
\pgfsetroundjoin%
\pgfsetlinewidth{1.505625pt}%
\definecolor{currentstroke}{rgb}{0.121569,0.466667,0.705882}%
\pgfsetstrokecolor{currentstroke}%
\pgfsetdash{}{0pt}%
\pgfpathmoveto{\pgfqpoint{1.176788in}{2.466389in}}%
\pgfpathlineto{\pgfqpoint{1.454566in}{2.466389in}}%
\pgfusepath{stroke}%
\end{pgfscope}%
\begin{pgfscope}%
\definecolor{textcolor}{rgb}{0.000000,0.000000,0.000000}%
\pgfsetstrokecolor{textcolor}%
\pgfsetfillcolor{textcolor}%
\pgftext[x=1.565677in,y=2.417778in,left,base]{\color{textcolor}\rmfamily\fontsize{10.000000}{12.000000}\selectfont \(\displaystyle x^{0.5}\)}%
\end{pgfscope}%
\begin{pgfscope}%
\pgfsetrectcap%
\pgfsetroundjoin%
\pgfsetlinewidth{1.505625pt}%
\definecolor{currentstroke}{rgb}{1.000000,0.498039,0.054902}%
\pgfsetstrokecolor{currentstroke}%
\pgfsetdash{}{0pt}%
\pgfpathmoveto{\pgfqpoint{1.176788in}{2.272778in}}%
\pgfpathlineto{\pgfqpoint{1.454566in}{2.272778in}}%
\pgfusepath{stroke}%
\end{pgfscope}%
\begin{pgfscope}%
\definecolor{textcolor}{rgb}{0.000000,0.000000,0.000000}%
\pgfsetstrokecolor{textcolor}%
\pgfsetfillcolor{textcolor}%
\pgftext[x=1.565677in,y=2.224167in,left,base]{\color{textcolor}\rmfamily\fontsize{10.000000}{12.000000}\selectfont \(\displaystyle x^{0}\)}%
\end{pgfscope}%
\begin{pgfscope}%
\pgfsetrectcap%
\pgfsetroundjoin%
\pgfsetlinewidth{1.505625pt}%
\definecolor{currentstroke}{rgb}{0.172549,0.627451,0.172549}%
\pgfsetstrokecolor{currentstroke}%
\pgfsetdash{}{0pt}%
\pgfpathmoveto{\pgfqpoint{1.176788in}{2.079167in}}%
\pgfpathlineto{\pgfqpoint{1.454566in}{2.079167in}}%
\pgfusepath{stroke}%
\end{pgfscope}%
\begin{pgfscope}%
\definecolor{textcolor}{rgb}{0.000000,0.000000,0.000000}%
\pgfsetstrokecolor{textcolor}%
\pgfsetfillcolor{textcolor}%
\pgftext[x=1.565677in,y=2.030556in,left,base]{\color{textcolor}\rmfamily\fontsize{10.000000}{12.000000}\selectfont \(\displaystyle x^{1}\)}%
\end{pgfscope}%
\begin{pgfscope}%
\pgfsetrectcap%
\pgfsetroundjoin%
\pgfsetlinewidth{1.505625pt}%
\definecolor{currentstroke}{rgb}{0.839216,0.152941,0.156863}%
\pgfsetstrokecolor{currentstroke}%
\pgfsetdash{}{0pt}%
\pgfpathmoveto{\pgfqpoint{1.176788in}{1.885556in}}%
\pgfpathlineto{\pgfqpoint{1.454566in}{1.885556in}}%
\pgfusepath{stroke}%
\end{pgfscope}%
\begin{pgfscope}%
\definecolor{textcolor}{rgb}{0.000000,0.000000,0.000000}%
\pgfsetstrokecolor{textcolor}%
\pgfsetfillcolor{textcolor}%
\pgftext[x=1.565677in,y=1.836945in,left,base]{\color{textcolor}\rmfamily\fontsize{10.000000}{12.000000}\selectfont \(\displaystyle x^{2}\)}%
\end{pgfscope}%
\begin{pgfscope}%
\pgfsetrectcap%
\pgfsetroundjoin%
\pgfsetlinewidth{1.505625pt}%
\definecolor{currentstroke}{rgb}{0.580392,0.403922,0.741176}%
\pgfsetstrokecolor{currentstroke}%
\pgfsetdash{}{0pt}%
\pgfpathmoveto{\pgfqpoint{1.176788in}{1.691945in}}%
\pgfpathlineto{\pgfqpoint{1.454566in}{1.691945in}}%
\pgfusepath{stroke}%
\end{pgfscope}%
\begin{pgfscope}%
\definecolor{textcolor}{rgb}{0.000000,0.000000,0.000000}%
\pgfsetstrokecolor{textcolor}%
\pgfsetfillcolor{textcolor}%
\pgftext[x=1.565677in,y=1.643334in,left,base]{\color{textcolor}\rmfamily\fontsize{10.000000}{12.000000}\selectfont \(\displaystyle x^{3}\)}%
\end{pgfscope}%
\end{pgfpicture}%
\makeatother%
\endgroup%

    \caption{Ableitungen der Polynome\label{polynomials:diff}}
\end{figure}

Dieses Verhalten ist zu erwarten wenn man bedenkt, dass das Haar Wavelet
jeweils die Summe und die Differenz zweier benachbarter Samples analysiert.
Siehe dazu auch \autoref{haar:allwavelets:image} im Kapitel zum Haar Wavelet.

Nun stellt sich die Frage was passiert wenn wir Daubechies Wavelets mit mehr
verschwindenden Momenten zur Analyse einsetzen. In \autoref{polynomials:db2_3}
sind jeweils die Detailkoeffizienten der Analyse mit db1 und db2 Wavelet zu
sehen.

\begin{figure}
    \centering
    %% Creator: Matplotlib, PGF backend
%%
%% To include the figure in your LaTeX document, write
%%   \input{<filename>.pgf}
%%
%% Make sure the required packages are loaded in your preamble
%%   \usepackage{pgf}
%%
%% Figures using additional raster images can only be included by \input if
%% they are in the same directory as the main LaTeX file. For loading figures
%% from other directories you can use the `import` package
%%   \usepackage{import}
%% and then include the figures with
%%   \import{<path to file>}{<filename>.pgf}
%%
%% Matplotlib used the following preamble
%%   \usepackage{fontspec}
%%
\begingroup%
\makeatletter%
\begin{pgfpicture}%
\pgfpathrectangle{\pgfpointorigin}{\pgfqpoint{5.800000in}{3.200000in}}%
\pgfusepath{use as bounding box, clip}%
\begin{pgfscope}%
\pgfsetbuttcap%
\pgfsetmiterjoin%
\definecolor{currentfill}{rgb}{1.000000,1.000000,1.000000}%
\pgfsetfillcolor{currentfill}%
\pgfsetlinewidth{0.000000pt}%
\definecolor{currentstroke}{rgb}{1.000000,1.000000,1.000000}%
\pgfsetstrokecolor{currentstroke}%
\pgfsetdash{}{0pt}%
\pgfpathmoveto{\pgfqpoint{0.000000in}{0.000000in}}%
\pgfpathlineto{\pgfqpoint{5.800000in}{0.000000in}}%
\pgfpathlineto{\pgfqpoint{5.800000in}{3.200000in}}%
\pgfpathlineto{\pgfqpoint{0.000000in}{3.200000in}}%
\pgfpathclose%
\pgfusepath{fill}%
\end{pgfscope}%
\begin{pgfscope}%
\pgfsetbuttcap%
\pgfsetmiterjoin%
\definecolor{currentfill}{rgb}{1.000000,1.000000,1.000000}%
\pgfsetfillcolor{currentfill}%
\pgfsetlinewidth{0.000000pt}%
\definecolor{currentstroke}{rgb}{0.000000,0.000000,0.000000}%
\pgfsetstrokecolor{currentstroke}%
\pgfsetstrokeopacity{0.000000}%
\pgfsetdash{}{0pt}%
\pgfpathmoveto{\pgfqpoint{0.583472in}{1.901944in}}%
\pgfpathlineto{\pgfqpoint{4.930000in}{1.901944in}}%
\pgfpathlineto{\pgfqpoint{4.930000in}{2.826667in}}%
\pgfpathlineto{\pgfqpoint{0.583472in}{2.826667in}}%
\pgfpathclose%
\pgfusepath{fill}%
\end{pgfscope}%
\begin{pgfscope}%
\pgfsetbuttcap%
\pgfsetroundjoin%
\definecolor{currentfill}{rgb}{0.000000,0.000000,0.000000}%
\pgfsetfillcolor{currentfill}%
\pgfsetlinewidth{0.803000pt}%
\definecolor{currentstroke}{rgb}{0.000000,0.000000,0.000000}%
\pgfsetstrokecolor{currentstroke}%
\pgfsetdash{}{0pt}%
\pgfsys@defobject{currentmarker}{\pgfqpoint{0.000000in}{-0.048611in}}{\pgfqpoint{0.000000in}{0.000000in}}{%
\pgfpathmoveto{\pgfqpoint{0.000000in}{0.000000in}}%
\pgfpathlineto{\pgfqpoint{0.000000in}{-0.048611in}}%
\pgfusepath{stroke,fill}%
}%
\begin{pgfscope}%
\pgfsys@transformshift{0.781042in}{1.901944in}%
\pgfsys@useobject{currentmarker}{}%
\end{pgfscope}%
\end{pgfscope}%
\begin{pgfscope}%
\pgfsetbuttcap%
\pgfsetroundjoin%
\definecolor{currentfill}{rgb}{0.000000,0.000000,0.000000}%
\pgfsetfillcolor{currentfill}%
\pgfsetlinewidth{0.803000pt}%
\definecolor{currentstroke}{rgb}{0.000000,0.000000,0.000000}%
\pgfsetstrokecolor{currentstroke}%
\pgfsetdash{}{0pt}%
\pgfsys@defobject{currentmarker}{\pgfqpoint{0.000000in}{-0.048611in}}{\pgfqpoint{0.000000in}{0.000000in}}{%
\pgfpathmoveto{\pgfqpoint{0.000000in}{0.000000in}}%
\pgfpathlineto{\pgfqpoint{0.000000in}{-0.048611in}}%
\pgfusepath{stroke,fill}%
}%
\begin{pgfscope}%
\pgfsys@transformshift{1.408246in}{1.901944in}%
\pgfsys@useobject{currentmarker}{}%
\end{pgfscope}%
\end{pgfscope}%
\begin{pgfscope}%
\pgfsetbuttcap%
\pgfsetroundjoin%
\definecolor{currentfill}{rgb}{0.000000,0.000000,0.000000}%
\pgfsetfillcolor{currentfill}%
\pgfsetlinewidth{0.803000pt}%
\definecolor{currentstroke}{rgb}{0.000000,0.000000,0.000000}%
\pgfsetstrokecolor{currentstroke}%
\pgfsetdash{}{0pt}%
\pgfsys@defobject{currentmarker}{\pgfqpoint{0.000000in}{-0.048611in}}{\pgfqpoint{0.000000in}{0.000000in}}{%
\pgfpathmoveto{\pgfqpoint{0.000000in}{0.000000in}}%
\pgfpathlineto{\pgfqpoint{0.000000in}{-0.048611in}}%
\pgfusepath{stroke,fill}%
}%
\begin{pgfscope}%
\pgfsys@transformshift{2.035451in}{1.901944in}%
\pgfsys@useobject{currentmarker}{}%
\end{pgfscope}%
\end{pgfscope}%
\begin{pgfscope}%
\pgfsetbuttcap%
\pgfsetroundjoin%
\definecolor{currentfill}{rgb}{0.000000,0.000000,0.000000}%
\pgfsetfillcolor{currentfill}%
\pgfsetlinewidth{0.803000pt}%
\definecolor{currentstroke}{rgb}{0.000000,0.000000,0.000000}%
\pgfsetstrokecolor{currentstroke}%
\pgfsetdash{}{0pt}%
\pgfsys@defobject{currentmarker}{\pgfqpoint{0.000000in}{-0.048611in}}{\pgfqpoint{0.000000in}{0.000000in}}{%
\pgfpathmoveto{\pgfqpoint{0.000000in}{0.000000in}}%
\pgfpathlineto{\pgfqpoint{0.000000in}{-0.048611in}}%
\pgfusepath{stroke,fill}%
}%
\begin{pgfscope}%
\pgfsys@transformshift{2.662655in}{1.901944in}%
\pgfsys@useobject{currentmarker}{}%
\end{pgfscope}%
\end{pgfscope}%
\begin{pgfscope}%
\pgfsetbuttcap%
\pgfsetroundjoin%
\definecolor{currentfill}{rgb}{0.000000,0.000000,0.000000}%
\pgfsetfillcolor{currentfill}%
\pgfsetlinewidth{0.803000pt}%
\definecolor{currentstroke}{rgb}{0.000000,0.000000,0.000000}%
\pgfsetstrokecolor{currentstroke}%
\pgfsetdash{}{0pt}%
\pgfsys@defobject{currentmarker}{\pgfqpoint{0.000000in}{-0.048611in}}{\pgfqpoint{0.000000in}{0.000000in}}{%
\pgfpathmoveto{\pgfqpoint{0.000000in}{0.000000in}}%
\pgfpathlineto{\pgfqpoint{0.000000in}{-0.048611in}}%
\pgfusepath{stroke,fill}%
}%
\begin{pgfscope}%
\pgfsys@transformshift{3.289860in}{1.901944in}%
\pgfsys@useobject{currentmarker}{}%
\end{pgfscope}%
\end{pgfscope}%
\begin{pgfscope}%
\pgfsetbuttcap%
\pgfsetroundjoin%
\definecolor{currentfill}{rgb}{0.000000,0.000000,0.000000}%
\pgfsetfillcolor{currentfill}%
\pgfsetlinewidth{0.803000pt}%
\definecolor{currentstroke}{rgb}{0.000000,0.000000,0.000000}%
\pgfsetstrokecolor{currentstroke}%
\pgfsetdash{}{0pt}%
\pgfsys@defobject{currentmarker}{\pgfqpoint{0.000000in}{-0.048611in}}{\pgfqpoint{0.000000in}{0.000000in}}{%
\pgfpathmoveto{\pgfqpoint{0.000000in}{0.000000in}}%
\pgfpathlineto{\pgfqpoint{0.000000in}{-0.048611in}}%
\pgfusepath{stroke,fill}%
}%
\begin{pgfscope}%
\pgfsys@transformshift{3.917065in}{1.901944in}%
\pgfsys@useobject{currentmarker}{}%
\end{pgfscope}%
\end{pgfscope}%
\begin{pgfscope}%
\pgfsetbuttcap%
\pgfsetroundjoin%
\definecolor{currentfill}{rgb}{0.000000,0.000000,0.000000}%
\pgfsetfillcolor{currentfill}%
\pgfsetlinewidth{0.803000pt}%
\definecolor{currentstroke}{rgb}{0.000000,0.000000,0.000000}%
\pgfsetstrokecolor{currentstroke}%
\pgfsetdash{}{0pt}%
\pgfsys@defobject{currentmarker}{\pgfqpoint{0.000000in}{-0.048611in}}{\pgfqpoint{0.000000in}{0.000000in}}{%
\pgfpathmoveto{\pgfqpoint{0.000000in}{0.000000in}}%
\pgfpathlineto{\pgfqpoint{0.000000in}{-0.048611in}}%
\pgfusepath{stroke,fill}%
}%
\begin{pgfscope}%
\pgfsys@transformshift{4.544269in}{1.901944in}%
\pgfsys@useobject{currentmarker}{}%
\end{pgfscope}%
\end{pgfscope}%
\begin{pgfscope}%
\pgfsetbuttcap%
\pgfsetroundjoin%
\definecolor{currentfill}{rgb}{0.000000,0.000000,0.000000}%
\pgfsetfillcolor{currentfill}%
\pgfsetlinewidth{0.803000pt}%
\definecolor{currentstroke}{rgb}{0.000000,0.000000,0.000000}%
\pgfsetstrokecolor{currentstroke}%
\pgfsetdash{}{0pt}%
\pgfsys@defobject{currentmarker}{\pgfqpoint{-0.048611in}{0.000000in}}{\pgfqpoint{0.000000in}{0.000000in}}{%
\pgfpathmoveto{\pgfqpoint{0.000000in}{0.000000in}}%
\pgfpathlineto{\pgfqpoint{-0.048611in}{0.000000in}}%
\pgfusepath{stroke,fill}%
}%
\begin{pgfscope}%
\pgfsys@transformshift{0.583472in}{2.037205in}%
\pgfsys@useobject{currentmarker}{}%
\end{pgfscope}%
\end{pgfscope}%
\begin{pgfscope}%
\definecolor{textcolor}{rgb}{0.000000,0.000000,0.000000}%
\pgfsetstrokecolor{textcolor}%
\pgfsetfillcolor{textcolor}%
\pgftext[x=0.308750in,y=1.989010in,left,base]{\color{textcolor}\sffamily\fontsize{10.000000}{12.000000}\selectfont −4}%
\end{pgfscope}%
\begin{pgfscope}%
\pgfsetbuttcap%
\pgfsetroundjoin%
\definecolor{currentfill}{rgb}{0.000000,0.000000,0.000000}%
\pgfsetfillcolor{currentfill}%
\pgfsetlinewidth{0.803000pt}%
\definecolor{currentstroke}{rgb}{0.000000,0.000000,0.000000}%
\pgfsetstrokecolor{currentstroke}%
\pgfsetdash{}{0pt}%
\pgfsys@defobject{currentmarker}{\pgfqpoint{-0.048611in}{0.000000in}}{\pgfqpoint{0.000000in}{0.000000in}}{%
\pgfpathmoveto{\pgfqpoint{0.000000in}{0.000000in}}%
\pgfpathlineto{\pgfqpoint{-0.048611in}{0.000000in}}%
\pgfusepath{stroke,fill}%
}%
\begin{pgfscope}%
\pgfsys@transformshift{0.583472in}{2.410919in}%
\pgfsys@useobject{currentmarker}{}%
\end{pgfscope}%
\end{pgfscope}%
\begin{pgfscope}%
\definecolor{textcolor}{rgb}{0.000000,0.000000,0.000000}%
\pgfsetstrokecolor{textcolor}%
\pgfsetfillcolor{textcolor}%
\pgftext[x=0.308750in,y=2.362725in,left,base]{\color{textcolor}\sffamily\fontsize{10.000000}{12.000000}\selectfont −2}%
\end{pgfscope}%
\begin{pgfscope}%
\pgfsetbuttcap%
\pgfsetroundjoin%
\definecolor{currentfill}{rgb}{0.000000,0.000000,0.000000}%
\pgfsetfillcolor{currentfill}%
\pgfsetlinewidth{0.803000pt}%
\definecolor{currentstroke}{rgb}{0.000000,0.000000,0.000000}%
\pgfsetstrokecolor{currentstroke}%
\pgfsetdash{}{0pt}%
\pgfsys@defobject{currentmarker}{\pgfqpoint{-0.048611in}{0.000000in}}{\pgfqpoint{0.000000in}{0.000000in}}{%
\pgfpathmoveto{\pgfqpoint{0.000000in}{0.000000in}}%
\pgfpathlineto{\pgfqpoint{-0.048611in}{0.000000in}}%
\pgfusepath{stroke,fill}%
}%
\begin{pgfscope}%
\pgfsys@transformshift{0.583472in}{2.784634in}%
\pgfsys@useobject{currentmarker}{}%
\end{pgfscope}%
\end{pgfscope}%
\begin{pgfscope}%
\definecolor{textcolor}{rgb}{0.000000,0.000000,0.000000}%
\pgfsetstrokecolor{textcolor}%
\pgfsetfillcolor{textcolor}%
\pgftext[x=0.416806in,y=2.736439in,left,base]{\color{textcolor}\sffamily\fontsize{10.000000}{12.000000}\selectfont 0}%
\end{pgfscope}%
\begin{pgfscope}%
\definecolor{textcolor}{rgb}{0.000000,0.000000,0.000000}%
\pgfsetstrokecolor{textcolor}%
\pgfsetfillcolor{textcolor}%
\pgftext[x=0.583472in,y=2.868333in,left,base]{\color{textcolor}\sffamily\fontsize{10.000000}{12.000000}\selectfont 1e−4}%
\end{pgfscope}%
\begin{pgfscope}%
\pgfpathrectangle{\pgfqpoint{0.583472in}{1.901944in}}{\pgfqpoint{4.346528in}{0.924722in}}%
\pgfusepath{clip}%
\pgfsetrectcap%
\pgfsetroundjoin%
\pgfsetlinewidth{1.505625pt}%
\definecolor{currentstroke}{rgb}{0.121569,0.466667,0.705882}%
\pgfsetstrokecolor{currentstroke}%
\pgfsetdash{}{0pt}%
\pgfpathmoveto{\pgfqpoint{0.781042in}{2.784634in}}%
\pgfpathlineto{\pgfqpoint{0.812402in}{2.784634in}}%
\pgfpathlineto{\pgfqpoint{0.843762in}{2.784634in}}%
\pgfpathlineto{\pgfqpoint{0.875122in}{2.784634in}}%
\pgfpathlineto{\pgfqpoint{0.906483in}{2.784634in}}%
\pgfpathlineto{\pgfqpoint{0.937843in}{2.784634in}}%
\pgfpathlineto{\pgfqpoint{0.969203in}{2.784634in}}%
\pgfpathlineto{\pgfqpoint{1.000563in}{2.784634in}}%
\pgfpathlineto{\pgfqpoint{1.031924in}{2.784634in}}%
\pgfpathlineto{\pgfqpoint{1.063284in}{2.784634in}}%
\pgfpathlineto{\pgfqpoint{1.094644in}{2.784634in}}%
\pgfpathlineto{\pgfqpoint{1.126004in}{2.784634in}}%
\pgfpathlineto{\pgfqpoint{1.157364in}{2.784634in}}%
\pgfpathlineto{\pgfqpoint{1.188725in}{2.784634in}}%
\pgfpathlineto{\pgfqpoint{1.220085in}{2.784634in}}%
\pgfpathlineto{\pgfqpoint{1.251445in}{2.784634in}}%
\pgfpathlineto{\pgfqpoint{1.282805in}{2.784634in}}%
\pgfpathlineto{\pgfqpoint{1.314166in}{2.784634in}}%
\pgfpathlineto{\pgfqpoint{1.345526in}{2.784634in}}%
\pgfpathlineto{\pgfqpoint{1.376886in}{2.784634in}}%
\pgfpathlineto{\pgfqpoint{1.408246in}{2.784634in}}%
\pgfpathlineto{\pgfqpoint{1.439606in}{2.784634in}}%
\pgfpathlineto{\pgfqpoint{1.470967in}{2.784634in}}%
\pgfpathlineto{\pgfqpoint{1.502327in}{2.784634in}}%
\pgfpathlineto{\pgfqpoint{1.533687in}{2.784634in}}%
\pgfpathlineto{\pgfqpoint{1.565047in}{2.784634in}}%
\pgfpathlineto{\pgfqpoint{1.596408in}{2.784634in}}%
\pgfpathlineto{\pgfqpoint{1.627768in}{2.784634in}}%
\pgfpathlineto{\pgfqpoint{1.659128in}{2.784634in}}%
\pgfpathlineto{\pgfqpoint{1.690488in}{2.784634in}}%
\pgfpathlineto{\pgfqpoint{1.721849in}{2.784634in}}%
\pgfpathlineto{\pgfqpoint{1.753209in}{2.784634in}}%
\pgfpathlineto{\pgfqpoint{1.784569in}{2.784634in}}%
\pgfpathlineto{\pgfqpoint{1.815929in}{2.784634in}}%
\pgfpathlineto{\pgfqpoint{1.847289in}{2.784634in}}%
\pgfpathlineto{\pgfqpoint{1.878650in}{2.784634in}}%
\pgfpathlineto{\pgfqpoint{1.910010in}{2.784634in}}%
\pgfpathlineto{\pgfqpoint{1.941370in}{2.784634in}}%
\pgfpathlineto{\pgfqpoint{1.972730in}{2.784634in}}%
\pgfpathlineto{\pgfqpoint{2.004091in}{2.784634in}}%
\pgfpathlineto{\pgfqpoint{2.035451in}{2.784634in}}%
\pgfpathlineto{\pgfqpoint{2.066811in}{2.784634in}}%
\pgfpathlineto{\pgfqpoint{2.098171in}{2.784634in}}%
\pgfpathlineto{\pgfqpoint{2.129532in}{2.784634in}}%
\pgfpathlineto{\pgfqpoint{2.160892in}{2.784634in}}%
\pgfpathlineto{\pgfqpoint{2.192252in}{2.784634in}}%
\pgfpathlineto{\pgfqpoint{2.223612in}{2.784634in}}%
\pgfpathlineto{\pgfqpoint{2.254972in}{2.784634in}}%
\pgfpathlineto{\pgfqpoint{2.286333in}{2.784634in}}%
\pgfpathlineto{\pgfqpoint{2.317693in}{2.784634in}}%
\pgfpathlineto{\pgfqpoint{2.349053in}{2.784634in}}%
\pgfpathlineto{\pgfqpoint{2.380413in}{2.784634in}}%
\pgfpathlineto{\pgfqpoint{2.411774in}{2.784634in}}%
\pgfpathlineto{\pgfqpoint{2.443134in}{2.784634in}}%
\pgfpathlineto{\pgfqpoint{2.474494in}{2.784634in}}%
\pgfpathlineto{\pgfqpoint{2.505854in}{2.784634in}}%
\pgfpathlineto{\pgfqpoint{2.537215in}{2.784634in}}%
\pgfpathlineto{\pgfqpoint{2.568575in}{2.784634in}}%
\pgfpathlineto{\pgfqpoint{2.599935in}{2.784634in}}%
\pgfpathlineto{\pgfqpoint{2.631295in}{2.784634in}}%
\pgfpathlineto{\pgfqpoint{2.662655in}{2.784634in}}%
\pgfpathlineto{\pgfqpoint{2.694016in}{2.784634in}}%
\pgfpathlineto{\pgfqpoint{2.725376in}{2.784634in}}%
\pgfpathlineto{\pgfqpoint{2.756736in}{2.784634in}}%
\pgfpathlineto{\pgfqpoint{2.788096in}{2.784634in}}%
\pgfpathlineto{\pgfqpoint{2.819457in}{2.784634in}}%
\pgfpathlineto{\pgfqpoint{2.850817in}{2.784634in}}%
\pgfpathlineto{\pgfqpoint{2.882177in}{2.784634in}}%
\pgfpathlineto{\pgfqpoint{2.913537in}{2.784634in}}%
\pgfpathlineto{\pgfqpoint{2.944897in}{2.784634in}}%
\pgfpathlineto{\pgfqpoint{2.976258in}{2.784634in}}%
\pgfpathlineto{\pgfqpoint{3.007618in}{2.784634in}}%
\pgfpathlineto{\pgfqpoint{3.038978in}{2.784634in}}%
\pgfpathlineto{\pgfqpoint{3.070338in}{2.784634in}}%
\pgfpathlineto{\pgfqpoint{3.101699in}{2.784634in}}%
\pgfpathlineto{\pgfqpoint{3.133059in}{2.784634in}}%
\pgfpathlineto{\pgfqpoint{3.164419in}{2.784634in}}%
\pgfpathlineto{\pgfqpoint{3.195779in}{2.784634in}}%
\pgfpathlineto{\pgfqpoint{3.227140in}{2.784634in}}%
\pgfpathlineto{\pgfqpoint{3.258500in}{2.784634in}}%
\pgfpathlineto{\pgfqpoint{3.289860in}{2.784634in}}%
\pgfpathlineto{\pgfqpoint{3.321220in}{2.784634in}}%
\pgfpathlineto{\pgfqpoint{3.352580in}{2.784634in}}%
\pgfpathlineto{\pgfqpoint{3.383941in}{2.784634in}}%
\pgfpathlineto{\pgfqpoint{3.415301in}{2.784634in}}%
\pgfpathlineto{\pgfqpoint{3.446661in}{2.784634in}}%
\pgfpathlineto{\pgfqpoint{3.478021in}{2.784634in}}%
\pgfpathlineto{\pgfqpoint{3.509382in}{2.784634in}}%
\pgfpathlineto{\pgfqpoint{3.540742in}{2.784634in}}%
\pgfpathlineto{\pgfqpoint{3.572102in}{2.784634in}}%
\pgfpathlineto{\pgfqpoint{3.603462in}{2.784634in}}%
\pgfpathlineto{\pgfqpoint{3.634823in}{2.784634in}}%
\pgfpathlineto{\pgfqpoint{3.666183in}{2.784634in}}%
\pgfpathlineto{\pgfqpoint{3.697543in}{2.784634in}}%
\pgfpathlineto{\pgfqpoint{3.728903in}{2.784634in}}%
\pgfpathlineto{\pgfqpoint{3.760263in}{2.784634in}}%
\pgfpathlineto{\pgfqpoint{3.791624in}{2.784634in}}%
\pgfpathlineto{\pgfqpoint{3.822984in}{2.784634in}}%
\pgfpathlineto{\pgfqpoint{3.854344in}{2.784634in}}%
\pgfpathlineto{\pgfqpoint{3.885704in}{2.784634in}}%
\pgfpathlineto{\pgfqpoint{3.917065in}{2.784634in}}%
\pgfpathlineto{\pgfqpoint{3.948425in}{2.784634in}}%
\pgfpathlineto{\pgfqpoint{3.979785in}{2.784634in}}%
\pgfpathlineto{\pgfqpoint{4.011145in}{2.784634in}}%
\pgfpathlineto{\pgfqpoint{4.042506in}{2.784634in}}%
\pgfpathlineto{\pgfqpoint{4.073866in}{2.784634in}}%
\pgfpathlineto{\pgfqpoint{4.105226in}{2.784634in}}%
\pgfpathlineto{\pgfqpoint{4.136586in}{2.784634in}}%
\pgfpathlineto{\pgfqpoint{4.167946in}{2.784634in}}%
\pgfpathlineto{\pgfqpoint{4.199307in}{2.784634in}}%
\pgfpathlineto{\pgfqpoint{4.230667in}{2.784634in}}%
\pgfpathlineto{\pgfqpoint{4.262027in}{2.784634in}}%
\pgfpathlineto{\pgfqpoint{4.293387in}{2.784634in}}%
\pgfpathlineto{\pgfqpoint{4.324748in}{2.784634in}}%
\pgfpathlineto{\pgfqpoint{4.356108in}{2.784634in}}%
\pgfpathlineto{\pgfqpoint{4.387468in}{2.784634in}}%
\pgfpathlineto{\pgfqpoint{4.418828in}{2.784634in}}%
\pgfpathlineto{\pgfqpoint{4.450188in}{2.784634in}}%
\pgfpathlineto{\pgfqpoint{4.481549in}{2.784634in}}%
\pgfpathlineto{\pgfqpoint{4.512909in}{2.784634in}}%
\pgfpathlineto{\pgfqpoint{4.544269in}{2.784634in}}%
\pgfpathlineto{\pgfqpoint{4.575629in}{2.784634in}}%
\pgfpathlineto{\pgfqpoint{4.606990in}{2.784634in}}%
\pgfpathlineto{\pgfqpoint{4.638350in}{2.784634in}}%
\pgfpathlineto{\pgfqpoint{4.669710in}{2.784634in}}%
\pgfpathlineto{\pgfqpoint{4.701070in}{2.784634in}}%
\pgfpathlineto{\pgfqpoint{4.732431in}{2.784634in}}%
\pgfusepath{stroke}%
\end{pgfscope}%
\begin{pgfscope}%
\pgfpathrectangle{\pgfqpoint{0.583472in}{1.901944in}}{\pgfqpoint{4.346528in}{0.924722in}}%
\pgfusepath{clip}%
\pgfsetrectcap%
\pgfsetroundjoin%
\pgfsetlinewidth{1.505625pt}%
\definecolor{currentstroke}{rgb}{1.000000,0.498039,0.054902}%
\pgfsetstrokecolor{currentstroke}%
\pgfsetdash{}{0pt}%
\pgfpathmoveto{\pgfqpoint{0.781042in}{2.784634in}}%
\pgfpathlineto{\pgfqpoint{0.812402in}{2.784634in}}%
\pgfpathlineto{\pgfqpoint{0.843762in}{2.784634in}}%
\pgfpathlineto{\pgfqpoint{0.875122in}{2.784634in}}%
\pgfpathlineto{\pgfqpoint{0.906483in}{2.784634in}}%
\pgfpathlineto{\pgfqpoint{0.937843in}{2.784634in}}%
\pgfpathlineto{\pgfqpoint{0.969203in}{2.784634in}}%
\pgfpathlineto{\pgfqpoint{1.000563in}{2.784634in}}%
\pgfpathlineto{\pgfqpoint{1.031924in}{2.784634in}}%
\pgfpathlineto{\pgfqpoint{1.063284in}{2.784634in}}%
\pgfpathlineto{\pgfqpoint{1.094644in}{2.784634in}}%
\pgfpathlineto{\pgfqpoint{1.126004in}{2.784634in}}%
\pgfpathlineto{\pgfqpoint{1.157364in}{2.784634in}}%
\pgfpathlineto{\pgfqpoint{1.188725in}{2.784634in}}%
\pgfpathlineto{\pgfqpoint{1.220085in}{2.784634in}}%
\pgfpathlineto{\pgfqpoint{1.251445in}{2.784634in}}%
\pgfpathlineto{\pgfqpoint{1.282805in}{2.784634in}}%
\pgfpathlineto{\pgfqpoint{1.314166in}{2.784634in}}%
\pgfpathlineto{\pgfqpoint{1.345526in}{2.784634in}}%
\pgfpathlineto{\pgfqpoint{1.376886in}{2.784634in}}%
\pgfpathlineto{\pgfqpoint{1.408246in}{2.784634in}}%
\pgfpathlineto{\pgfqpoint{1.439606in}{2.784634in}}%
\pgfpathlineto{\pgfqpoint{1.470967in}{2.784634in}}%
\pgfpathlineto{\pgfqpoint{1.502327in}{2.784634in}}%
\pgfpathlineto{\pgfqpoint{1.533687in}{2.784634in}}%
\pgfpathlineto{\pgfqpoint{1.565047in}{2.784634in}}%
\pgfpathlineto{\pgfqpoint{1.596408in}{2.784634in}}%
\pgfpathlineto{\pgfqpoint{1.627768in}{2.784634in}}%
\pgfpathlineto{\pgfqpoint{1.659128in}{2.784634in}}%
\pgfpathlineto{\pgfqpoint{1.690488in}{2.784634in}}%
\pgfpathlineto{\pgfqpoint{1.721849in}{2.784634in}}%
\pgfpathlineto{\pgfqpoint{1.753209in}{2.784634in}}%
\pgfpathlineto{\pgfqpoint{1.784569in}{2.784634in}}%
\pgfpathlineto{\pgfqpoint{1.815929in}{2.784634in}}%
\pgfpathlineto{\pgfqpoint{1.847289in}{2.784634in}}%
\pgfpathlineto{\pgfqpoint{1.878650in}{2.784634in}}%
\pgfpathlineto{\pgfqpoint{1.910010in}{2.784634in}}%
\pgfpathlineto{\pgfqpoint{1.941370in}{2.784634in}}%
\pgfpathlineto{\pgfqpoint{1.972730in}{2.784634in}}%
\pgfpathlineto{\pgfqpoint{2.004091in}{2.784634in}}%
\pgfpathlineto{\pgfqpoint{2.035451in}{2.784634in}}%
\pgfpathlineto{\pgfqpoint{2.066811in}{2.784634in}}%
\pgfpathlineto{\pgfqpoint{2.098171in}{2.784634in}}%
\pgfpathlineto{\pgfqpoint{2.129532in}{2.784634in}}%
\pgfpathlineto{\pgfqpoint{2.160892in}{2.784634in}}%
\pgfpathlineto{\pgfqpoint{2.192252in}{2.784634in}}%
\pgfpathlineto{\pgfqpoint{2.223612in}{2.784634in}}%
\pgfpathlineto{\pgfqpoint{2.254972in}{2.784634in}}%
\pgfpathlineto{\pgfqpoint{2.286333in}{2.784634in}}%
\pgfpathlineto{\pgfqpoint{2.317693in}{2.784634in}}%
\pgfpathlineto{\pgfqpoint{2.349053in}{2.784634in}}%
\pgfpathlineto{\pgfqpoint{2.380413in}{2.784634in}}%
\pgfpathlineto{\pgfqpoint{2.411774in}{2.784634in}}%
\pgfpathlineto{\pgfqpoint{2.443134in}{2.784634in}}%
\pgfpathlineto{\pgfqpoint{2.474494in}{2.784634in}}%
\pgfpathlineto{\pgfqpoint{2.505854in}{2.784634in}}%
\pgfpathlineto{\pgfqpoint{2.537215in}{2.784634in}}%
\pgfpathlineto{\pgfqpoint{2.568575in}{2.784634in}}%
\pgfpathlineto{\pgfqpoint{2.599935in}{2.784634in}}%
\pgfpathlineto{\pgfqpoint{2.631295in}{2.784634in}}%
\pgfpathlineto{\pgfqpoint{2.662655in}{2.784634in}}%
\pgfpathlineto{\pgfqpoint{2.694016in}{2.784634in}}%
\pgfpathlineto{\pgfqpoint{2.725376in}{2.784634in}}%
\pgfpathlineto{\pgfqpoint{2.756736in}{2.784634in}}%
\pgfpathlineto{\pgfqpoint{2.788096in}{2.784634in}}%
\pgfpathlineto{\pgfqpoint{2.819457in}{2.784634in}}%
\pgfpathlineto{\pgfqpoint{2.850817in}{2.784634in}}%
\pgfpathlineto{\pgfqpoint{2.882177in}{2.784634in}}%
\pgfpathlineto{\pgfqpoint{2.913537in}{2.784634in}}%
\pgfpathlineto{\pgfqpoint{2.944897in}{2.784634in}}%
\pgfpathlineto{\pgfqpoint{2.976258in}{2.784634in}}%
\pgfpathlineto{\pgfqpoint{3.007618in}{2.784634in}}%
\pgfpathlineto{\pgfqpoint{3.038978in}{2.784634in}}%
\pgfpathlineto{\pgfqpoint{3.070338in}{2.784634in}}%
\pgfpathlineto{\pgfqpoint{3.101699in}{2.784634in}}%
\pgfpathlineto{\pgfqpoint{3.133059in}{2.784634in}}%
\pgfpathlineto{\pgfqpoint{3.164419in}{2.784634in}}%
\pgfpathlineto{\pgfqpoint{3.195779in}{2.784634in}}%
\pgfpathlineto{\pgfqpoint{3.227140in}{2.784634in}}%
\pgfpathlineto{\pgfqpoint{3.258500in}{2.784634in}}%
\pgfpathlineto{\pgfqpoint{3.289860in}{2.784634in}}%
\pgfpathlineto{\pgfqpoint{3.321220in}{2.784634in}}%
\pgfpathlineto{\pgfqpoint{3.352580in}{2.784634in}}%
\pgfpathlineto{\pgfqpoint{3.383941in}{2.784634in}}%
\pgfpathlineto{\pgfqpoint{3.415301in}{2.784634in}}%
\pgfpathlineto{\pgfqpoint{3.446661in}{2.784634in}}%
\pgfpathlineto{\pgfqpoint{3.478021in}{2.784634in}}%
\pgfpathlineto{\pgfqpoint{3.509382in}{2.784634in}}%
\pgfpathlineto{\pgfqpoint{3.540742in}{2.784634in}}%
\pgfpathlineto{\pgfqpoint{3.572102in}{2.784634in}}%
\pgfpathlineto{\pgfqpoint{3.603462in}{2.784634in}}%
\pgfpathlineto{\pgfqpoint{3.634823in}{2.784634in}}%
\pgfpathlineto{\pgfqpoint{3.666183in}{2.784634in}}%
\pgfpathlineto{\pgfqpoint{3.697543in}{2.784634in}}%
\pgfpathlineto{\pgfqpoint{3.728903in}{2.784634in}}%
\pgfpathlineto{\pgfqpoint{3.760263in}{2.784634in}}%
\pgfpathlineto{\pgfqpoint{3.791624in}{2.784634in}}%
\pgfpathlineto{\pgfqpoint{3.822984in}{2.784634in}}%
\pgfpathlineto{\pgfqpoint{3.854344in}{2.784634in}}%
\pgfpathlineto{\pgfqpoint{3.885704in}{2.784634in}}%
\pgfpathlineto{\pgfqpoint{3.917065in}{2.784634in}}%
\pgfpathlineto{\pgfqpoint{3.948425in}{2.784634in}}%
\pgfpathlineto{\pgfqpoint{3.979785in}{2.784634in}}%
\pgfpathlineto{\pgfqpoint{4.011145in}{2.784634in}}%
\pgfpathlineto{\pgfqpoint{4.042506in}{2.784634in}}%
\pgfpathlineto{\pgfqpoint{4.073866in}{2.784634in}}%
\pgfpathlineto{\pgfqpoint{4.105226in}{2.784634in}}%
\pgfpathlineto{\pgfqpoint{4.136586in}{2.784634in}}%
\pgfpathlineto{\pgfqpoint{4.167946in}{2.784634in}}%
\pgfpathlineto{\pgfqpoint{4.199307in}{2.784634in}}%
\pgfpathlineto{\pgfqpoint{4.230667in}{2.784634in}}%
\pgfpathlineto{\pgfqpoint{4.262027in}{2.784634in}}%
\pgfpathlineto{\pgfqpoint{4.293387in}{2.784634in}}%
\pgfpathlineto{\pgfqpoint{4.324748in}{2.784634in}}%
\pgfpathlineto{\pgfqpoint{4.356108in}{2.784634in}}%
\pgfpathlineto{\pgfqpoint{4.387468in}{2.784634in}}%
\pgfpathlineto{\pgfqpoint{4.418828in}{2.784634in}}%
\pgfpathlineto{\pgfqpoint{4.450188in}{2.784634in}}%
\pgfpathlineto{\pgfqpoint{4.481549in}{2.784634in}}%
\pgfpathlineto{\pgfqpoint{4.512909in}{2.784634in}}%
\pgfpathlineto{\pgfqpoint{4.544269in}{2.784634in}}%
\pgfpathlineto{\pgfqpoint{4.575629in}{2.784634in}}%
\pgfpathlineto{\pgfqpoint{4.606990in}{2.784634in}}%
\pgfpathlineto{\pgfqpoint{4.638350in}{2.784634in}}%
\pgfpathlineto{\pgfqpoint{4.669710in}{2.784634in}}%
\pgfpathlineto{\pgfqpoint{4.701070in}{2.784634in}}%
\pgfpathlineto{\pgfqpoint{4.732431in}{2.784634in}}%
\pgfusepath{stroke}%
\end{pgfscope}%
\begin{pgfscope}%
\pgfpathrectangle{\pgfqpoint{0.583472in}{1.901944in}}{\pgfqpoint{4.346528in}{0.924722in}}%
\pgfusepath{clip}%
\pgfsetrectcap%
\pgfsetroundjoin%
\pgfsetlinewidth{1.505625pt}%
\definecolor{currentstroke}{rgb}{0.172549,0.627451,0.172549}%
\pgfsetstrokecolor{currentstroke}%
\pgfsetdash{}{0pt}%
\pgfpathmoveto{\pgfqpoint{0.781042in}{2.643856in}}%
\pgfpathlineto{\pgfqpoint{0.812402in}{2.643856in}}%
\pgfpathlineto{\pgfqpoint{0.843762in}{2.643856in}}%
\pgfpathlineto{\pgfqpoint{0.875122in}{2.643856in}}%
\pgfpathlineto{\pgfqpoint{0.906483in}{2.643856in}}%
\pgfpathlineto{\pgfqpoint{0.937843in}{2.643856in}}%
\pgfpathlineto{\pgfqpoint{0.969203in}{2.643856in}}%
\pgfpathlineto{\pgfqpoint{1.000563in}{2.643856in}}%
\pgfpathlineto{\pgfqpoint{1.031924in}{2.643856in}}%
\pgfpathlineto{\pgfqpoint{1.063284in}{2.643856in}}%
\pgfpathlineto{\pgfqpoint{1.094644in}{2.643856in}}%
\pgfpathlineto{\pgfqpoint{1.126004in}{2.643856in}}%
\pgfpathlineto{\pgfqpoint{1.157364in}{2.643856in}}%
\pgfpathlineto{\pgfqpoint{1.188725in}{2.643856in}}%
\pgfpathlineto{\pgfqpoint{1.220085in}{2.643856in}}%
\pgfpathlineto{\pgfqpoint{1.251445in}{2.643856in}}%
\pgfpathlineto{\pgfqpoint{1.282805in}{2.643856in}}%
\pgfpathlineto{\pgfqpoint{1.314166in}{2.643856in}}%
\pgfpathlineto{\pgfqpoint{1.345526in}{2.643856in}}%
\pgfpathlineto{\pgfqpoint{1.376886in}{2.643856in}}%
\pgfpathlineto{\pgfqpoint{1.408246in}{2.643856in}}%
\pgfpathlineto{\pgfqpoint{1.439606in}{2.643856in}}%
\pgfpathlineto{\pgfqpoint{1.470967in}{2.643856in}}%
\pgfpathlineto{\pgfqpoint{1.502327in}{2.643856in}}%
\pgfpathlineto{\pgfqpoint{1.533687in}{2.643856in}}%
\pgfpathlineto{\pgfqpoint{1.565047in}{2.643856in}}%
\pgfpathlineto{\pgfqpoint{1.596408in}{2.643856in}}%
\pgfpathlineto{\pgfqpoint{1.627768in}{2.643856in}}%
\pgfpathlineto{\pgfqpoint{1.659128in}{2.643856in}}%
\pgfpathlineto{\pgfqpoint{1.690488in}{2.643856in}}%
\pgfpathlineto{\pgfqpoint{1.721849in}{2.643856in}}%
\pgfpathlineto{\pgfqpoint{1.753209in}{2.643856in}}%
\pgfpathlineto{\pgfqpoint{1.784569in}{2.643856in}}%
\pgfpathlineto{\pgfqpoint{1.815929in}{2.643856in}}%
\pgfpathlineto{\pgfqpoint{1.847289in}{2.643856in}}%
\pgfpathlineto{\pgfqpoint{1.878650in}{2.643856in}}%
\pgfpathlineto{\pgfqpoint{1.910010in}{2.643856in}}%
\pgfpathlineto{\pgfqpoint{1.941370in}{2.643856in}}%
\pgfpathlineto{\pgfqpoint{1.972730in}{2.643856in}}%
\pgfpathlineto{\pgfqpoint{2.004091in}{2.643856in}}%
\pgfpathlineto{\pgfqpoint{2.035451in}{2.643856in}}%
\pgfpathlineto{\pgfqpoint{2.066811in}{2.643856in}}%
\pgfpathlineto{\pgfqpoint{2.098171in}{2.643856in}}%
\pgfpathlineto{\pgfqpoint{2.129532in}{2.643856in}}%
\pgfpathlineto{\pgfqpoint{2.160892in}{2.643856in}}%
\pgfpathlineto{\pgfqpoint{2.192252in}{2.643856in}}%
\pgfpathlineto{\pgfqpoint{2.223612in}{2.643856in}}%
\pgfpathlineto{\pgfqpoint{2.254972in}{2.643856in}}%
\pgfpathlineto{\pgfqpoint{2.286333in}{2.643856in}}%
\pgfpathlineto{\pgfqpoint{2.317693in}{2.643856in}}%
\pgfpathlineto{\pgfqpoint{2.349053in}{2.643856in}}%
\pgfpathlineto{\pgfqpoint{2.380413in}{2.643856in}}%
\pgfpathlineto{\pgfqpoint{2.411774in}{2.643856in}}%
\pgfpathlineto{\pgfqpoint{2.443134in}{2.643856in}}%
\pgfpathlineto{\pgfqpoint{2.474494in}{2.643856in}}%
\pgfpathlineto{\pgfqpoint{2.505854in}{2.643856in}}%
\pgfpathlineto{\pgfqpoint{2.537215in}{2.643856in}}%
\pgfpathlineto{\pgfqpoint{2.568575in}{2.643856in}}%
\pgfpathlineto{\pgfqpoint{2.599935in}{2.643856in}}%
\pgfpathlineto{\pgfqpoint{2.631295in}{2.643856in}}%
\pgfpathlineto{\pgfqpoint{2.662655in}{2.643856in}}%
\pgfpathlineto{\pgfqpoint{2.694016in}{2.643856in}}%
\pgfpathlineto{\pgfqpoint{2.725376in}{2.643856in}}%
\pgfpathlineto{\pgfqpoint{2.756736in}{2.643856in}}%
\pgfpathlineto{\pgfqpoint{2.788096in}{2.643856in}}%
\pgfpathlineto{\pgfqpoint{2.819457in}{2.643856in}}%
\pgfpathlineto{\pgfqpoint{2.850817in}{2.643856in}}%
\pgfpathlineto{\pgfqpoint{2.882177in}{2.643856in}}%
\pgfpathlineto{\pgfqpoint{2.913537in}{2.643856in}}%
\pgfpathlineto{\pgfqpoint{2.944897in}{2.643856in}}%
\pgfpathlineto{\pgfqpoint{2.976258in}{2.643856in}}%
\pgfpathlineto{\pgfqpoint{3.007618in}{2.643856in}}%
\pgfpathlineto{\pgfqpoint{3.038978in}{2.643856in}}%
\pgfpathlineto{\pgfqpoint{3.070338in}{2.643856in}}%
\pgfpathlineto{\pgfqpoint{3.101699in}{2.643856in}}%
\pgfpathlineto{\pgfqpoint{3.133059in}{2.643856in}}%
\pgfpathlineto{\pgfqpoint{3.164419in}{2.643856in}}%
\pgfpathlineto{\pgfqpoint{3.195779in}{2.643856in}}%
\pgfpathlineto{\pgfqpoint{3.227140in}{2.643856in}}%
\pgfpathlineto{\pgfqpoint{3.258500in}{2.643856in}}%
\pgfpathlineto{\pgfqpoint{3.289860in}{2.643856in}}%
\pgfpathlineto{\pgfqpoint{3.321220in}{2.643856in}}%
\pgfpathlineto{\pgfqpoint{3.352580in}{2.643856in}}%
\pgfpathlineto{\pgfqpoint{3.383941in}{2.643856in}}%
\pgfpathlineto{\pgfqpoint{3.415301in}{2.643856in}}%
\pgfpathlineto{\pgfqpoint{3.446661in}{2.643856in}}%
\pgfpathlineto{\pgfqpoint{3.478021in}{2.643856in}}%
\pgfpathlineto{\pgfqpoint{3.509382in}{2.643856in}}%
\pgfpathlineto{\pgfqpoint{3.540742in}{2.643856in}}%
\pgfpathlineto{\pgfqpoint{3.572102in}{2.643856in}}%
\pgfpathlineto{\pgfqpoint{3.603462in}{2.643856in}}%
\pgfpathlineto{\pgfqpoint{3.634823in}{2.643856in}}%
\pgfpathlineto{\pgfqpoint{3.666183in}{2.643856in}}%
\pgfpathlineto{\pgfqpoint{3.697543in}{2.643856in}}%
\pgfpathlineto{\pgfqpoint{3.728903in}{2.643856in}}%
\pgfpathlineto{\pgfqpoint{3.760263in}{2.643856in}}%
\pgfpathlineto{\pgfqpoint{3.791624in}{2.643856in}}%
\pgfpathlineto{\pgfqpoint{3.822984in}{2.643856in}}%
\pgfpathlineto{\pgfqpoint{3.854344in}{2.643856in}}%
\pgfpathlineto{\pgfqpoint{3.885704in}{2.643856in}}%
\pgfpathlineto{\pgfqpoint{3.917065in}{2.643856in}}%
\pgfpathlineto{\pgfqpoint{3.948425in}{2.643856in}}%
\pgfpathlineto{\pgfqpoint{3.979785in}{2.643856in}}%
\pgfpathlineto{\pgfqpoint{4.011145in}{2.643856in}}%
\pgfpathlineto{\pgfqpoint{4.042506in}{2.643856in}}%
\pgfpathlineto{\pgfqpoint{4.073866in}{2.643856in}}%
\pgfpathlineto{\pgfqpoint{4.105226in}{2.643856in}}%
\pgfpathlineto{\pgfqpoint{4.136586in}{2.643856in}}%
\pgfpathlineto{\pgfqpoint{4.167946in}{2.643856in}}%
\pgfpathlineto{\pgfqpoint{4.199307in}{2.643856in}}%
\pgfpathlineto{\pgfqpoint{4.230667in}{2.643856in}}%
\pgfpathlineto{\pgfqpoint{4.262027in}{2.643856in}}%
\pgfpathlineto{\pgfqpoint{4.293387in}{2.643856in}}%
\pgfpathlineto{\pgfqpoint{4.324748in}{2.643856in}}%
\pgfpathlineto{\pgfqpoint{4.356108in}{2.643856in}}%
\pgfpathlineto{\pgfqpoint{4.387468in}{2.643856in}}%
\pgfpathlineto{\pgfqpoint{4.418828in}{2.643856in}}%
\pgfpathlineto{\pgfqpoint{4.450188in}{2.643856in}}%
\pgfpathlineto{\pgfqpoint{4.481549in}{2.643856in}}%
\pgfpathlineto{\pgfqpoint{4.512909in}{2.643856in}}%
\pgfpathlineto{\pgfqpoint{4.544269in}{2.643856in}}%
\pgfpathlineto{\pgfqpoint{4.575629in}{2.643856in}}%
\pgfpathlineto{\pgfqpoint{4.606990in}{2.643856in}}%
\pgfpathlineto{\pgfqpoint{4.638350in}{2.643856in}}%
\pgfpathlineto{\pgfqpoint{4.669710in}{2.643856in}}%
\pgfpathlineto{\pgfqpoint{4.701070in}{2.643856in}}%
\pgfpathlineto{\pgfqpoint{4.732431in}{2.643856in}}%
\pgfusepath{stroke}%
\end{pgfscope}%
\begin{pgfscope}%
\pgfpathrectangle{\pgfqpoint{0.583472in}{1.901944in}}{\pgfqpoint{4.346528in}{0.924722in}}%
\pgfusepath{clip}%
\pgfsetrectcap%
\pgfsetroundjoin%
\pgfsetlinewidth{1.505625pt}%
\definecolor{currentstroke}{rgb}{0.839216,0.152941,0.156863}%
\pgfsetstrokecolor{currentstroke}%
\pgfsetdash{}{0pt}%
\pgfpathmoveto{\pgfqpoint{0.781042in}{2.778709in}}%
\pgfpathlineto{\pgfqpoint{0.812402in}{2.772084in}}%
\pgfpathlineto{\pgfqpoint{0.843762in}{2.765459in}}%
\pgfpathlineto{\pgfqpoint{0.875122in}{2.758834in}}%
\pgfpathlineto{\pgfqpoint{0.906483in}{2.752210in}}%
\pgfpathlineto{\pgfqpoint{0.937843in}{2.745585in}}%
\pgfpathlineto{\pgfqpoint{0.969203in}{2.738960in}}%
\pgfpathlineto{\pgfqpoint{1.000563in}{2.732335in}}%
\pgfpathlineto{\pgfqpoint{1.031924in}{2.725710in}}%
\pgfpathlineto{\pgfqpoint{1.063284in}{2.719085in}}%
\pgfpathlineto{\pgfqpoint{1.094644in}{2.712460in}}%
\pgfpathlineto{\pgfqpoint{1.126004in}{2.705836in}}%
\pgfpathlineto{\pgfqpoint{1.157364in}{2.699211in}}%
\pgfpathlineto{\pgfqpoint{1.188725in}{2.692586in}}%
\pgfpathlineto{\pgfqpoint{1.220085in}{2.685961in}}%
\pgfpathlineto{\pgfqpoint{1.251445in}{2.679336in}}%
\pgfpathlineto{\pgfqpoint{1.282805in}{2.672711in}}%
\pgfpathlineto{\pgfqpoint{1.314166in}{2.666086in}}%
\pgfpathlineto{\pgfqpoint{1.345526in}{2.659462in}}%
\pgfpathlineto{\pgfqpoint{1.376886in}{2.652837in}}%
\pgfpathlineto{\pgfqpoint{1.408246in}{2.646212in}}%
\pgfpathlineto{\pgfqpoint{1.439606in}{2.639587in}}%
\pgfpathlineto{\pgfqpoint{1.470967in}{2.632962in}}%
\pgfpathlineto{\pgfqpoint{1.502327in}{2.626337in}}%
\pgfpathlineto{\pgfqpoint{1.533687in}{2.619712in}}%
\pgfpathlineto{\pgfqpoint{1.565047in}{2.613088in}}%
\pgfpathlineto{\pgfqpoint{1.596408in}{2.606463in}}%
\pgfpathlineto{\pgfqpoint{1.627768in}{2.599838in}}%
\pgfpathlineto{\pgfqpoint{1.659128in}{2.593213in}}%
\pgfpathlineto{\pgfqpoint{1.690488in}{2.586588in}}%
\pgfpathlineto{\pgfqpoint{1.721849in}{2.579963in}}%
\pgfpathlineto{\pgfqpoint{1.753209in}{2.573338in}}%
\pgfpathlineto{\pgfqpoint{1.784569in}{2.566714in}}%
\pgfpathlineto{\pgfqpoint{1.815929in}{2.560089in}}%
\pgfpathlineto{\pgfqpoint{1.847289in}{2.553464in}}%
\pgfpathlineto{\pgfqpoint{1.878650in}{2.546839in}}%
\pgfpathlineto{\pgfqpoint{1.910010in}{2.540214in}}%
\pgfpathlineto{\pgfqpoint{1.941370in}{2.533589in}}%
\pgfpathlineto{\pgfqpoint{1.972730in}{2.526964in}}%
\pgfpathlineto{\pgfqpoint{2.004091in}{2.520340in}}%
\pgfpathlineto{\pgfqpoint{2.035451in}{2.513715in}}%
\pgfpathlineto{\pgfqpoint{2.066811in}{2.507090in}}%
\pgfpathlineto{\pgfqpoint{2.098171in}{2.500465in}}%
\pgfpathlineto{\pgfqpoint{2.129532in}{2.493840in}}%
\pgfpathlineto{\pgfqpoint{2.160892in}{2.487215in}}%
\pgfpathlineto{\pgfqpoint{2.192252in}{2.480591in}}%
\pgfpathlineto{\pgfqpoint{2.223612in}{2.473966in}}%
\pgfpathlineto{\pgfqpoint{2.254972in}{2.467341in}}%
\pgfpathlineto{\pgfqpoint{2.286333in}{2.460716in}}%
\pgfpathlineto{\pgfqpoint{2.317693in}{2.454091in}}%
\pgfpathlineto{\pgfqpoint{2.349053in}{2.447466in}}%
\pgfpathlineto{\pgfqpoint{2.380413in}{2.440841in}}%
\pgfpathlineto{\pgfqpoint{2.411774in}{2.434217in}}%
\pgfpathlineto{\pgfqpoint{2.443134in}{2.427592in}}%
\pgfpathlineto{\pgfqpoint{2.474494in}{2.420967in}}%
\pgfpathlineto{\pgfqpoint{2.505854in}{2.414342in}}%
\pgfpathlineto{\pgfqpoint{2.537215in}{2.407717in}}%
\pgfpathlineto{\pgfqpoint{2.568575in}{2.401092in}}%
\pgfpathlineto{\pgfqpoint{2.599935in}{2.394467in}}%
\pgfpathlineto{\pgfqpoint{2.631295in}{2.387843in}}%
\pgfpathlineto{\pgfqpoint{2.662655in}{2.381218in}}%
\pgfpathlineto{\pgfqpoint{2.694016in}{2.374593in}}%
\pgfpathlineto{\pgfqpoint{2.725376in}{2.367968in}}%
\pgfpathlineto{\pgfqpoint{2.756736in}{2.361343in}}%
\pgfpathlineto{\pgfqpoint{2.788096in}{2.354718in}}%
\pgfpathlineto{\pgfqpoint{2.819457in}{2.348093in}}%
\pgfpathlineto{\pgfqpoint{2.850817in}{2.341469in}}%
\pgfpathlineto{\pgfqpoint{2.882177in}{2.334844in}}%
\pgfpathlineto{\pgfqpoint{2.913537in}{2.328219in}}%
\pgfpathlineto{\pgfqpoint{2.944897in}{2.321594in}}%
\pgfpathlineto{\pgfqpoint{2.976258in}{2.314969in}}%
\pgfpathlineto{\pgfqpoint{3.007618in}{2.308344in}}%
\pgfpathlineto{\pgfqpoint{3.038978in}{2.301719in}}%
\pgfpathlineto{\pgfqpoint{3.070338in}{2.295095in}}%
\pgfpathlineto{\pgfqpoint{3.101699in}{2.288470in}}%
\pgfpathlineto{\pgfqpoint{3.133059in}{2.281845in}}%
\pgfpathlineto{\pgfqpoint{3.164419in}{2.275220in}}%
\pgfpathlineto{\pgfqpoint{3.195779in}{2.268595in}}%
\pgfpathlineto{\pgfqpoint{3.227140in}{2.261970in}}%
\pgfpathlineto{\pgfqpoint{3.258500in}{2.255345in}}%
\pgfpathlineto{\pgfqpoint{3.289860in}{2.248721in}}%
\pgfpathlineto{\pgfqpoint{3.321220in}{2.242096in}}%
\pgfpathlineto{\pgfqpoint{3.352580in}{2.235471in}}%
\pgfpathlineto{\pgfqpoint{3.383941in}{2.228846in}}%
\pgfpathlineto{\pgfqpoint{3.415301in}{2.222221in}}%
\pgfpathlineto{\pgfqpoint{3.446661in}{2.215596in}}%
\pgfpathlineto{\pgfqpoint{3.478021in}{2.208971in}}%
\pgfpathlineto{\pgfqpoint{3.509382in}{2.202347in}}%
\pgfpathlineto{\pgfqpoint{3.540742in}{2.195722in}}%
\pgfpathlineto{\pgfqpoint{3.572102in}{2.189097in}}%
\pgfpathlineto{\pgfqpoint{3.603462in}{2.182472in}}%
\pgfpathlineto{\pgfqpoint{3.634823in}{2.175847in}}%
\pgfpathlineto{\pgfqpoint{3.666183in}{2.169222in}}%
\pgfpathlineto{\pgfqpoint{3.697543in}{2.162597in}}%
\pgfpathlineto{\pgfqpoint{3.728903in}{2.155973in}}%
\pgfpathlineto{\pgfqpoint{3.760263in}{2.149348in}}%
\pgfpathlineto{\pgfqpoint{3.791624in}{2.142723in}}%
\pgfpathlineto{\pgfqpoint{3.822984in}{2.136098in}}%
\pgfpathlineto{\pgfqpoint{3.854344in}{2.129473in}}%
\pgfpathlineto{\pgfqpoint{3.885704in}{2.122848in}}%
\pgfpathlineto{\pgfqpoint{3.917065in}{2.116223in}}%
\pgfpathlineto{\pgfqpoint{3.948425in}{2.109599in}}%
\pgfpathlineto{\pgfqpoint{3.979785in}{2.102974in}}%
\pgfpathlineto{\pgfqpoint{4.011145in}{2.096349in}}%
\pgfpathlineto{\pgfqpoint{4.042506in}{2.089724in}}%
\pgfpathlineto{\pgfqpoint{4.073866in}{2.083099in}}%
\pgfpathlineto{\pgfqpoint{4.105226in}{2.076474in}}%
\pgfpathlineto{\pgfqpoint{4.136586in}{2.069850in}}%
\pgfpathlineto{\pgfqpoint{4.167946in}{2.063225in}}%
\pgfpathlineto{\pgfqpoint{4.199307in}{2.056600in}}%
\pgfpathlineto{\pgfqpoint{4.230667in}{2.049975in}}%
\pgfpathlineto{\pgfqpoint{4.262027in}{2.043350in}}%
\pgfpathlineto{\pgfqpoint{4.293387in}{2.036725in}}%
\pgfpathlineto{\pgfqpoint{4.324748in}{2.030100in}}%
\pgfpathlineto{\pgfqpoint{4.356108in}{2.023476in}}%
\pgfpathlineto{\pgfqpoint{4.387468in}{2.016851in}}%
\pgfpathlineto{\pgfqpoint{4.418828in}{2.010226in}}%
\pgfpathlineto{\pgfqpoint{4.450188in}{2.003601in}}%
\pgfpathlineto{\pgfqpoint{4.481549in}{1.996976in}}%
\pgfpathlineto{\pgfqpoint{4.512909in}{1.990351in}}%
\pgfpathlineto{\pgfqpoint{4.544269in}{1.983726in}}%
\pgfpathlineto{\pgfqpoint{4.575629in}{1.977102in}}%
\pgfpathlineto{\pgfqpoint{4.606990in}{1.970477in}}%
\pgfpathlineto{\pgfqpoint{4.638350in}{1.963852in}}%
\pgfpathlineto{\pgfqpoint{4.669710in}{1.957227in}}%
\pgfpathlineto{\pgfqpoint{4.701070in}{1.950602in}}%
\pgfpathlineto{\pgfqpoint{4.732431in}{1.943977in}}%
\pgfusepath{stroke}%
\end{pgfscope}%
\begin{pgfscope}%
\pgfsetrectcap%
\pgfsetmiterjoin%
\pgfsetlinewidth{0.803000pt}%
\definecolor{currentstroke}{rgb}{0.000000,0.000000,0.000000}%
\pgfsetstrokecolor{currentstroke}%
\pgfsetdash{}{0pt}%
\pgfpathmoveto{\pgfqpoint{0.583472in}{1.901944in}}%
\pgfpathlineto{\pgfqpoint{0.583472in}{2.826667in}}%
\pgfusepath{stroke}%
\end{pgfscope}%
\begin{pgfscope}%
\pgfsetrectcap%
\pgfsetmiterjoin%
\pgfsetlinewidth{0.803000pt}%
\definecolor{currentstroke}{rgb}{0.000000,0.000000,0.000000}%
\pgfsetstrokecolor{currentstroke}%
\pgfsetdash{}{0pt}%
\pgfpathmoveto{\pgfqpoint{4.930000in}{1.901944in}}%
\pgfpathlineto{\pgfqpoint{4.930000in}{2.826667in}}%
\pgfusepath{stroke}%
\end{pgfscope}%
\begin{pgfscope}%
\pgfsetrectcap%
\pgfsetmiterjoin%
\pgfsetlinewidth{0.803000pt}%
\definecolor{currentstroke}{rgb}{0.000000,0.000000,0.000000}%
\pgfsetstrokecolor{currentstroke}%
\pgfsetdash{}{0pt}%
\pgfpathmoveto{\pgfqpoint{0.583472in}{1.901944in}}%
\pgfpathlineto{\pgfqpoint{4.930000in}{1.901944in}}%
\pgfusepath{stroke}%
\end{pgfscope}%
\begin{pgfscope}%
\pgfsetrectcap%
\pgfsetmiterjoin%
\pgfsetlinewidth{0.803000pt}%
\definecolor{currentstroke}{rgb}{0.000000,0.000000,0.000000}%
\pgfsetstrokecolor{currentstroke}%
\pgfsetdash{}{0pt}%
\pgfpathmoveto{\pgfqpoint{0.583472in}{2.826667in}}%
\pgfpathlineto{\pgfqpoint{4.930000in}{2.826667in}}%
\pgfusepath{stroke}%
\end{pgfscope}%
\begin{pgfscope}%
\definecolor{textcolor}{rgb}{0.000000,0.000000,0.000000}%
\pgfsetstrokecolor{textcolor}%
\pgfsetfillcolor{textcolor}%
\pgftext[x=2.756736in,y=2.910000in,,base]{\color{textcolor}\sffamily\fontsize{12.000000}{14.400000}\selectfont db2}%
\end{pgfscope}%
\begin{pgfscope}%
\pgfsetbuttcap%
\pgfsetmiterjoin%
\definecolor{currentfill}{rgb}{1.000000,1.000000,1.000000}%
\pgfsetfillcolor{currentfill}%
\pgfsetlinewidth{0.000000pt}%
\definecolor{currentstroke}{rgb}{0.000000,0.000000,0.000000}%
\pgfsetstrokecolor{currentstroke}%
\pgfsetstrokeopacity{0.000000}%
\pgfsetdash{}{0pt}%
\pgfpathmoveto{\pgfqpoint{0.583472in}{0.568889in}}%
\pgfpathlineto{\pgfqpoint{4.930000in}{0.568889in}}%
\pgfpathlineto{\pgfqpoint{4.930000in}{1.493611in}}%
\pgfpathlineto{\pgfqpoint{0.583472in}{1.493611in}}%
\pgfpathclose%
\pgfusepath{fill}%
\end{pgfscope}%
\begin{pgfscope}%
\pgfsetbuttcap%
\pgfsetroundjoin%
\definecolor{currentfill}{rgb}{0.000000,0.000000,0.000000}%
\pgfsetfillcolor{currentfill}%
\pgfsetlinewidth{0.803000pt}%
\definecolor{currentstroke}{rgb}{0.000000,0.000000,0.000000}%
\pgfsetstrokecolor{currentstroke}%
\pgfsetdash{}{0pt}%
\pgfsys@defobject{currentmarker}{\pgfqpoint{0.000000in}{-0.048611in}}{\pgfqpoint{0.000000in}{0.000000in}}{%
\pgfpathmoveto{\pgfqpoint{0.000000in}{0.000000in}}%
\pgfpathlineto{\pgfqpoint{0.000000in}{-0.048611in}}%
\pgfusepath{stroke,fill}%
}%
\begin{pgfscope}%
\pgfsys@transformshift{0.781042in}{0.568889in}%
\pgfsys@useobject{currentmarker}{}%
\end{pgfscope}%
\end{pgfscope}%
\begin{pgfscope}%
\definecolor{textcolor}{rgb}{0.000000,0.000000,0.000000}%
\pgfsetstrokecolor{textcolor}%
\pgfsetfillcolor{textcolor}%
\pgftext[x=0.781042in,y=0.471667in,,top]{\color{textcolor}\sffamily\fontsize{10.000000}{12.000000}\selectfont 0.0}%
\end{pgfscope}%
\begin{pgfscope}%
\pgfsetbuttcap%
\pgfsetroundjoin%
\definecolor{currentfill}{rgb}{0.000000,0.000000,0.000000}%
\pgfsetfillcolor{currentfill}%
\pgfsetlinewidth{0.803000pt}%
\definecolor{currentstroke}{rgb}{0.000000,0.000000,0.000000}%
\pgfsetstrokecolor{currentstroke}%
\pgfsetdash{}{0pt}%
\pgfsys@defobject{currentmarker}{\pgfqpoint{0.000000in}{-0.048611in}}{\pgfqpoint{0.000000in}{0.000000in}}{%
\pgfpathmoveto{\pgfqpoint{0.000000in}{0.000000in}}%
\pgfpathlineto{\pgfqpoint{0.000000in}{-0.048611in}}%
\pgfusepath{stroke,fill}%
}%
\begin{pgfscope}%
\pgfsys@transformshift{1.408246in}{0.568889in}%
\pgfsys@useobject{currentmarker}{}%
\end{pgfscope}%
\end{pgfscope}%
\begin{pgfscope}%
\definecolor{textcolor}{rgb}{0.000000,0.000000,0.000000}%
\pgfsetstrokecolor{textcolor}%
\pgfsetfillcolor{textcolor}%
\pgftext[x=1.408246in,y=0.471667in,,top]{\color{textcolor}\sffamily\fontsize{10.000000}{12.000000}\selectfont 0.2}%
\end{pgfscope}%
\begin{pgfscope}%
\pgfsetbuttcap%
\pgfsetroundjoin%
\definecolor{currentfill}{rgb}{0.000000,0.000000,0.000000}%
\pgfsetfillcolor{currentfill}%
\pgfsetlinewidth{0.803000pt}%
\definecolor{currentstroke}{rgb}{0.000000,0.000000,0.000000}%
\pgfsetstrokecolor{currentstroke}%
\pgfsetdash{}{0pt}%
\pgfsys@defobject{currentmarker}{\pgfqpoint{0.000000in}{-0.048611in}}{\pgfqpoint{0.000000in}{0.000000in}}{%
\pgfpathmoveto{\pgfqpoint{0.000000in}{0.000000in}}%
\pgfpathlineto{\pgfqpoint{0.000000in}{-0.048611in}}%
\pgfusepath{stroke,fill}%
}%
\begin{pgfscope}%
\pgfsys@transformshift{2.035451in}{0.568889in}%
\pgfsys@useobject{currentmarker}{}%
\end{pgfscope}%
\end{pgfscope}%
\begin{pgfscope}%
\definecolor{textcolor}{rgb}{0.000000,0.000000,0.000000}%
\pgfsetstrokecolor{textcolor}%
\pgfsetfillcolor{textcolor}%
\pgftext[x=2.035451in,y=0.471667in,,top]{\color{textcolor}\sffamily\fontsize{10.000000}{12.000000}\selectfont 0.4}%
\end{pgfscope}%
\begin{pgfscope}%
\pgfsetbuttcap%
\pgfsetroundjoin%
\definecolor{currentfill}{rgb}{0.000000,0.000000,0.000000}%
\pgfsetfillcolor{currentfill}%
\pgfsetlinewidth{0.803000pt}%
\definecolor{currentstroke}{rgb}{0.000000,0.000000,0.000000}%
\pgfsetstrokecolor{currentstroke}%
\pgfsetdash{}{0pt}%
\pgfsys@defobject{currentmarker}{\pgfqpoint{0.000000in}{-0.048611in}}{\pgfqpoint{0.000000in}{0.000000in}}{%
\pgfpathmoveto{\pgfqpoint{0.000000in}{0.000000in}}%
\pgfpathlineto{\pgfqpoint{0.000000in}{-0.048611in}}%
\pgfusepath{stroke,fill}%
}%
\begin{pgfscope}%
\pgfsys@transformshift{2.662655in}{0.568889in}%
\pgfsys@useobject{currentmarker}{}%
\end{pgfscope}%
\end{pgfscope}%
\begin{pgfscope}%
\definecolor{textcolor}{rgb}{0.000000,0.000000,0.000000}%
\pgfsetstrokecolor{textcolor}%
\pgfsetfillcolor{textcolor}%
\pgftext[x=2.662655in,y=0.471667in,,top]{\color{textcolor}\sffamily\fontsize{10.000000}{12.000000}\selectfont 0.6}%
\end{pgfscope}%
\begin{pgfscope}%
\pgfsetbuttcap%
\pgfsetroundjoin%
\definecolor{currentfill}{rgb}{0.000000,0.000000,0.000000}%
\pgfsetfillcolor{currentfill}%
\pgfsetlinewidth{0.803000pt}%
\definecolor{currentstroke}{rgb}{0.000000,0.000000,0.000000}%
\pgfsetstrokecolor{currentstroke}%
\pgfsetdash{}{0pt}%
\pgfsys@defobject{currentmarker}{\pgfqpoint{0.000000in}{-0.048611in}}{\pgfqpoint{0.000000in}{0.000000in}}{%
\pgfpathmoveto{\pgfqpoint{0.000000in}{0.000000in}}%
\pgfpathlineto{\pgfqpoint{0.000000in}{-0.048611in}}%
\pgfusepath{stroke,fill}%
}%
\begin{pgfscope}%
\pgfsys@transformshift{3.289860in}{0.568889in}%
\pgfsys@useobject{currentmarker}{}%
\end{pgfscope}%
\end{pgfscope}%
\begin{pgfscope}%
\definecolor{textcolor}{rgb}{0.000000,0.000000,0.000000}%
\pgfsetstrokecolor{textcolor}%
\pgfsetfillcolor{textcolor}%
\pgftext[x=3.289860in,y=0.471667in,,top]{\color{textcolor}\sffamily\fontsize{10.000000}{12.000000}\selectfont 0.8}%
\end{pgfscope}%
\begin{pgfscope}%
\pgfsetbuttcap%
\pgfsetroundjoin%
\definecolor{currentfill}{rgb}{0.000000,0.000000,0.000000}%
\pgfsetfillcolor{currentfill}%
\pgfsetlinewidth{0.803000pt}%
\definecolor{currentstroke}{rgb}{0.000000,0.000000,0.000000}%
\pgfsetstrokecolor{currentstroke}%
\pgfsetdash{}{0pt}%
\pgfsys@defobject{currentmarker}{\pgfqpoint{0.000000in}{-0.048611in}}{\pgfqpoint{0.000000in}{0.000000in}}{%
\pgfpathmoveto{\pgfqpoint{0.000000in}{0.000000in}}%
\pgfpathlineto{\pgfqpoint{0.000000in}{-0.048611in}}%
\pgfusepath{stroke,fill}%
}%
\begin{pgfscope}%
\pgfsys@transformshift{3.917065in}{0.568889in}%
\pgfsys@useobject{currentmarker}{}%
\end{pgfscope}%
\end{pgfscope}%
\begin{pgfscope}%
\definecolor{textcolor}{rgb}{0.000000,0.000000,0.000000}%
\pgfsetstrokecolor{textcolor}%
\pgfsetfillcolor{textcolor}%
\pgftext[x=3.917065in,y=0.471667in,,top]{\color{textcolor}\sffamily\fontsize{10.000000}{12.000000}\selectfont 1.0}%
\end{pgfscope}%
\begin{pgfscope}%
\pgfsetbuttcap%
\pgfsetroundjoin%
\definecolor{currentfill}{rgb}{0.000000,0.000000,0.000000}%
\pgfsetfillcolor{currentfill}%
\pgfsetlinewidth{0.803000pt}%
\definecolor{currentstroke}{rgb}{0.000000,0.000000,0.000000}%
\pgfsetstrokecolor{currentstroke}%
\pgfsetdash{}{0pt}%
\pgfsys@defobject{currentmarker}{\pgfqpoint{0.000000in}{-0.048611in}}{\pgfqpoint{0.000000in}{0.000000in}}{%
\pgfpathmoveto{\pgfqpoint{0.000000in}{0.000000in}}%
\pgfpathlineto{\pgfqpoint{0.000000in}{-0.048611in}}%
\pgfusepath{stroke,fill}%
}%
\begin{pgfscope}%
\pgfsys@transformshift{4.544269in}{0.568889in}%
\pgfsys@useobject{currentmarker}{}%
\end{pgfscope}%
\end{pgfscope}%
\begin{pgfscope}%
\definecolor{textcolor}{rgb}{0.000000,0.000000,0.000000}%
\pgfsetstrokecolor{textcolor}%
\pgfsetfillcolor{textcolor}%
\pgftext[x=4.544269in,y=0.471667in,,top]{\color{textcolor}\sffamily\fontsize{10.000000}{12.000000}\selectfont 1.2}%
\end{pgfscope}%
\begin{pgfscope}%
\definecolor{textcolor}{rgb}{0.000000,0.000000,0.000000}%
\pgfsetstrokecolor{textcolor}%
\pgfsetfillcolor{textcolor}%
\pgftext[x=4.930000in,y=0.306667in,right,top]{\color{textcolor}\sffamily\fontsize{10.000000}{12.000000}\selectfont 1e2}%
\end{pgfscope}%
\begin{pgfscope}%
\pgfsetbuttcap%
\pgfsetroundjoin%
\definecolor{currentfill}{rgb}{0.000000,0.000000,0.000000}%
\pgfsetfillcolor{currentfill}%
\pgfsetlinewidth{0.803000pt}%
\definecolor{currentstroke}{rgb}{0.000000,0.000000,0.000000}%
\pgfsetstrokecolor{currentstroke}%
\pgfsetdash{}{0pt}%
\pgfsys@defobject{currentmarker}{\pgfqpoint{-0.048611in}{0.000000in}}{\pgfqpoint{0.000000in}{0.000000in}}{%
\pgfpathmoveto{\pgfqpoint{0.000000in}{0.000000in}}%
\pgfpathlineto{\pgfqpoint{-0.048611in}{0.000000in}}%
\pgfusepath{stroke,fill}%
}%
\begin{pgfscope}%
\pgfsys@transformshift{0.583472in}{0.932093in}%
\pgfsys@useobject{currentmarker}{}%
\end{pgfscope}%
\end{pgfscope}%
\begin{pgfscope}%
\definecolor{textcolor}{rgb}{0.000000,0.000000,0.000000}%
\pgfsetstrokecolor{textcolor}%
\pgfsetfillcolor{textcolor}%
\pgftext[x=0.308750in,y=0.883899in,left,base]{\color{textcolor}\sffamily\fontsize{10.000000}{12.000000}\selectfont −1}%
\end{pgfscope}%
\begin{pgfscope}%
\pgfsetbuttcap%
\pgfsetroundjoin%
\definecolor{currentfill}{rgb}{0.000000,0.000000,0.000000}%
\pgfsetfillcolor{currentfill}%
\pgfsetlinewidth{0.803000pt}%
\definecolor{currentstroke}{rgb}{0.000000,0.000000,0.000000}%
\pgfsetstrokecolor{currentstroke}%
\pgfsetdash{}{0pt}%
\pgfsys@defobject{currentmarker}{\pgfqpoint{-0.048611in}{0.000000in}}{\pgfqpoint{0.000000in}{0.000000in}}{%
\pgfpathmoveto{\pgfqpoint{0.000000in}{0.000000in}}%
\pgfpathlineto{\pgfqpoint{-0.048611in}{0.000000in}}%
\pgfusepath{stroke,fill}%
}%
\begin{pgfscope}%
\pgfsys@transformshift{0.583472in}{1.451578in}%
\pgfsys@useobject{currentmarker}{}%
\end{pgfscope}%
\end{pgfscope}%
\begin{pgfscope}%
\definecolor{textcolor}{rgb}{0.000000,0.000000,0.000000}%
\pgfsetstrokecolor{textcolor}%
\pgfsetfillcolor{textcolor}%
\pgftext[x=0.416806in,y=1.403384in,left,base]{\color{textcolor}\sffamily\fontsize{10.000000}{12.000000}\selectfont 0}%
\end{pgfscope}%
\begin{pgfscope}%
\definecolor{textcolor}{rgb}{0.000000,0.000000,0.000000}%
\pgfsetstrokecolor{textcolor}%
\pgfsetfillcolor{textcolor}%
\pgftext[x=0.583472in,y=1.535278in,left,base]{\color{textcolor}\sffamily\fontsize{10.000000}{12.000000}\selectfont 1e−6}%
\end{pgfscope}%
\begin{pgfscope}%
\pgfpathrectangle{\pgfqpoint{0.583472in}{0.568889in}}{\pgfqpoint{4.346528in}{0.924722in}}%
\pgfusepath{clip}%
\pgfsetrectcap%
\pgfsetroundjoin%
\pgfsetlinewidth{1.505625pt}%
\definecolor{currentstroke}{rgb}{0.121569,0.466667,0.705882}%
\pgfsetstrokecolor{currentstroke}%
\pgfsetdash{}{0pt}%
\pgfpathmoveto{\pgfqpoint{0.781042in}{1.451578in}}%
\pgfpathlineto{\pgfqpoint{0.812402in}{1.451578in}}%
\pgfpathlineto{\pgfqpoint{0.843762in}{1.451578in}}%
\pgfpathlineto{\pgfqpoint{0.875122in}{1.451578in}}%
\pgfpathlineto{\pgfqpoint{0.906483in}{1.451578in}}%
\pgfpathlineto{\pgfqpoint{0.937843in}{1.451578in}}%
\pgfpathlineto{\pgfqpoint{0.969203in}{1.451578in}}%
\pgfpathlineto{\pgfqpoint{1.000563in}{1.451578in}}%
\pgfpathlineto{\pgfqpoint{1.031924in}{1.451578in}}%
\pgfpathlineto{\pgfqpoint{1.063284in}{1.451578in}}%
\pgfpathlineto{\pgfqpoint{1.094644in}{1.451578in}}%
\pgfpathlineto{\pgfqpoint{1.126004in}{1.451578in}}%
\pgfpathlineto{\pgfqpoint{1.157364in}{1.451578in}}%
\pgfpathlineto{\pgfqpoint{1.188725in}{1.451578in}}%
\pgfpathlineto{\pgfqpoint{1.220085in}{1.451578in}}%
\pgfpathlineto{\pgfqpoint{1.251445in}{1.451578in}}%
\pgfpathlineto{\pgfqpoint{1.282805in}{1.451578in}}%
\pgfpathlineto{\pgfqpoint{1.314166in}{1.451578in}}%
\pgfpathlineto{\pgfqpoint{1.345526in}{1.451578in}}%
\pgfpathlineto{\pgfqpoint{1.376886in}{1.451578in}}%
\pgfpathlineto{\pgfqpoint{1.408246in}{1.451578in}}%
\pgfpathlineto{\pgfqpoint{1.439606in}{1.451578in}}%
\pgfpathlineto{\pgfqpoint{1.470967in}{1.451578in}}%
\pgfpathlineto{\pgfqpoint{1.502327in}{1.451578in}}%
\pgfpathlineto{\pgfqpoint{1.533687in}{1.451578in}}%
\pgfpathlineto{\pgfqpoint{1.565047in}{1.451578in}}%
\pgfpathlineto{\pgfqpoint{1.596408in}{1.451578in}}%
\pgfpathlineto{\pgfqpoint{1.627768in}{1.451578in}}%
\pgfpathlineto{\pgfqpoint{1.659128in}{1.451578in}}%
\pgfpathlineto{\pgfqpoint{1.690488in}{1.451578in}}%
\pgfpathlineto{\pgfqpoint{1.721849in}{1.451578in}}%
\pgfpathlineto{\pgfqpoint{1.753209in}{1.451578in}}%
\pgfpathlineto{\pgfqpoint{1.784569in}{1.451578in}}%
\pgfpathlineto{\pgfqpoint{1.815929in}{1.451578in}}%
\pgfpathlineto{\pgfqpoint{1.847289in}{1.451578in}}%
\pgfpathlineto{\pgfqpoint{1.878650in}{1.451578in}}%
\pgfpathlineto{\pgfqpoint{1.910010in}{1.451578in}}%
\pgfpathlineto{\pgfqpoint{1.941370in}{1.451578in}}%
\pgfpathlineto{\pgfqpoint{1.972730in}{1.451578in}}%
\pgfpathlineto{\pgfqpoint{2.004091in}{1.451578in}}%
\pgfpathlineto{\pgfqpoint{2.035451in}{1.451578in}}%
\pgfpathlineto{\pgfqpoint{2.066811in}{1.451578in}}%
\pgfpathlineto{\pgfqpoint{2.098171in}{1.451578in}}%
\pgfpathlineto{\pgfqpoint{2.129532in}{1.451578in}}%
\pgfpathlineto{\pgfqpoint{2.160892in}{1.451578in}}%
\pgfpathlineto{\pgfqpoint{2.192252in}{1.451578in}}%
\pgfpathlineto{\pgfqpoint{2.223612in}{1.451578in}}%
\pgfpathlineto{\pgfqpoint{2.254972in}{1.451578in}}%
\pgfpathlineto{\pgfqpoint{2.286333in}{1.451578in}}%
\pgfpathlineto{\pgfqpoint{2.317693in}{1.451578in}}%
\pgfpathlineto{\pgfqpoint{2.349053in}{1.451578in}}%
\pgfpathlineto{\pgfqpoint{2.380413in}{1.451578in}}%
\pgfpathlineto{\pgfqpoint{2.411774in}{1.451578in}}%
\pgfpathlineto{\pgfqpoint{2.443134in}{1.451578in}}%
\pgfpathlineto{\pgfqpoint{2.474494in}{1.451578in}}%
\pgfpathlineto{\pgfqpoint{2.505854in}{1.451578in}}%
\pgfpathlineto{\pgfqpoint{2.537215in}{1.451578in}}%
\pgfpathlineto{\pgfqpoint{2.568575in}{1.451578in}}%
\pgfpathlineto{\pgfqpoint{2.599935in}{1.451578in}}%
\pgfpathlineto{\pgfqpoint{2.631295in}{1.451578in}}%
\pgfpathlineto{\pgfqpoint{2.662655in}{1.451578in}}%
\pgfpathlineto{\pgfqpoint{2.694016in}{1.451578in}}%
\pgfpathlineto{\pgfqpoint{2.725376in}{1.451578in}}%
\pgfpathlineto{\pgfqpoint{2.756736in}{1.451578in}}%
\pgfpathlineto{\pgfqpoint{2.788096in}{1.451578in}}%
\pgfpathlineto{\pgfqpoint{2.819457in}{1.451578in}}%
\pgfpathlineto{\pgfqpoint{2.850817in}{1.451578in}}%
\pgfpathlineto{\pgfqpoint{2.882177in}{1.451578in}}%
\pgfpathlineto{\pgfqpoint{2.913537in}{1.451578in}}%
\pgfpathlineto{\pgfqpoint{2.944897in}{1.451578in}}%
\pgfpathlineto{\pgfqpoint{2.976258in}{1.451578in}}%
\pgfpathlineto{\pgfqpoint{3.007618in}{1.451578in}}%
\pgfpathlineto{\pgfqpoint{3.038978in}{1.451578in}}%
\pgfpathlineto{\pgfqpoint{3.070338in}{1.451578in}}%
\pgfpathlineto{\pgfqpoint{3.101699in}{1.451578in}}%
\pgfpathlineto{\pgfqpoint{3.133059in}{1.451578in}}%
\pgfpathlineto{\pgfqpoint{3.164419in}{1.451578in}}%
\pgfpathlineto{\pgfqpoint{3.195779in}{1.451578in}}%
\pgfpathlineto{\pgfqpoint{3.227140in}{1.451578in}}%
\pgfpathlineto{\pgfqpoint{3.258500in}{1.451578in}}%
\pgfpathlineto{\pgfqpoint{3.289860in}{1.451578in}}%
\pgfpathlineto{\pgfqpoint{3.321220in}{1.451578in}}%
\pgfpathlineto{\pgfqpoint{3.352580in}{1.451578in}}%
\pgfpathlineto{\pgfqpoint{3.383941in}{1.451578in}}%
\pgfpathlineto{\pgfqpoint{3.415301in}{1.451578in}}%
\pgfpathlineto{\pgfqpoint{3.446661in}{1.451578in}}%
\pgfpathlineto{\pgfqpoint{3.478021in}{1.451578in}}%
\pgfpathlineto{\pgfqpoint{3.509382in}{1.451578in}}%
\pgfpathlineto{\pgfqpoint{3.540742in}{1.451578in}}%
\pgfpathlineto{\pgfqpoint{3.572102in}{1.451578in}}%
\pgfpathlineto{\pgfqpoint{3.603462in}{1.451578in}}%
\pgfpathlineto{\pgfqpoint{3.634823in}{1.451578in}}%
\pgfpathlineto{\pgfqpoint{3.666183in}{1.451578in}}%
\pgfpathlineto{\pgfqpoint{3.697543in}{1.451578in}}%
\pgfpathlineto{\pgfqpoint{3.728903in}{1.451578in}}%
\pgfpathlineto{\pgfqpoint{3.760263in}{1.451578in}}%
\pgfpathlineto{\pgfqpoint{3.791624in}{1.451578in}}%
\pgfpathlineto{\pgfqpoint{3.822984in}{1.451578in}}%
\pgfpathlineto{\pgfqpoint{3.854344in}{1.451578in}}%
\pgfpathlineto{\pgfqpoint{3.885704in}{1.451578in}}%
\pgfpathlineto{\pgfqpoint{3.917065in}{1.451578in}}%
\pgfpathlineto{\pgfqpoint{3.948425in}{1.451578in}}%
\pgfpathlineto{\pgfqpoint{3.979785in}{1.451578in}}%
\pgfpathlineto{\pgfqpoint{4.011145in}{1.451578in}}%
\pgfpathlineto{\pgfqpoint{4.042506in}{1.451578in}}%
\pgfpathlineto{\pgfqpoint{4.073866in}{1.451578in}}%
\pgfpathlineto{\pgfqpoint{4.105226in}{1.451578in}}%
\pgfpathlineto{\pgfqpoint{4.136586in}{1.451578in}}%
\pgfpathlineto{\pgfqpoint{4.167946in}{1.451578in}}%
\pgfpathlineto{\pgfqpoint{4.199307in}{1.451578in}}%
\pgfpathlineto{\pgfqpoint{4.230667in}{1.451578in}}%
\pgfpathlineto{\pgfqpoint{4.262027in}{1.451578in}}%
\pgfpathlineto{\pgfqpoint{4.293387in}{1.451578in}}%
\pgfpathlineto{\pgfqpoint{4.324748in}{1.451578in}}%
\pgfpathlineto{\pgfqpoint{4.356108in}{1.451578in}}%
\pgfpathlineto{\pgfqpoint{4.387468in}{1.451578in}}%
\pgfpathlineto{\pgfqpoint{4.418828in}{1.451578in}}%
\pgfpathlineto{\pgfqpoint{4.450188in}{1.451578in}}%
\pgfpathlineto{\pgfqpoint{4.481549in}{1.451578in}}%
\pgfpathlineto{\pgfqpoint{4.512909in}{1.451578in}}%
\pgfpathlineto{\pgfqpoint{4.544269in}{1.451578in}}%
\pgfpathlineto{\pgfqpoint{4.575629in}{1.451578in}}%
\pgfpathlineto{\pgfqpoint{4.606990in}{1.451578in}}%
\pgfpathlineto{\pgfqpoint{4.638350in}{1.451578in}}%
\pgfpathlineto{\pgfqpoint{4.669710in}{1.451578in}}%
\pgfpathlineto{\pgfqpoint{4.701070in}{1.451578in}}%
\pgfusepath{stroke}%
\end{pgfscope}%
\begin{pgfscope}%
\pgfpathrectangle{\pgfqpoint{0.583472in}{0.568889in}}{\pgfqpoint{4.346528in}{0.924722in}}%
\pgfusepath{clip}%
\pgfsetrectcap%
\pgfsetroundjoin%
\pgfsetlinewidth{1.505625pt}%
\definecolor{currentstroke}{rgb}{1.000000,0.498039,0.054902}%
\pgfsetstrokecolor{currentstroke}%
\pgfsetdash{}{0pt}%
\pgfpathmoveto{\pgfqpoint{0.781042in}{1.451578in}}%
\pgfpathlineto{\pgfqpoint{0.812402in}{1.451578in}}%
\pgfpathlineto{\pgfqpoint{0.843762in}{1.451578in}}%
\pgfpathlineto{\pgfqpoint{0.875122in}{1.451578in}}%
\pgfpathlineto{\pgfqpoint{0.906483in}{1.451578in}}%
\pgfpathlineto{\pgfqpoint{0.937843in}{1.451578in}}%
\pgfpathlineto{\pgfqpoint{0.969203in}{1.451578in}}%
\pgfpathlineto{\pgfqpoint{1.000563in}{1.451578in}}%
\pgfpathlineto{\pgfqpoint{1.031924in}{1.451578in}}%
\pgfpathlineto{\pgfqpoint{1.063284in}{1.451578in}}%
\pgfpathlineto{\pgfqpoint{1.094644in}{1.451578in}}%
\pgfpathlineto{\pgfqpoint{1.126004in}{1.451578in}}%
\pgfpathlineto{\pgfqpoint{1.157364in}{1.451578in}}%
\pgfpathlineto{\pgfqpoint{1.188725in}{1.451578in}}%
\pgfpathlineto{\pgfqpoint{1.220085in}{1.451578in}}%
\pgfpathlineto{\pgfqpoint{1.251445in}{1.451578in}}%
\pgfpathlineto{\pgfqpoint{1.282805in}{1.451578in}}%
\pgfpathlineto{\pgfqpoint{1.314166in}{1.451578in}}%
\pgfpathlineto{\pgfqpoint{1.345526in}{1.451578in}}%
\pgfpathlineto{\pgfqpoint{1.376886in}{1.451578in}}%
\pgfpathlineto{\pgfqpoint{1.408246in}{1.451578in}}%
\pgfpathlineto{\pgfqpoint{1.439606in}{1.451578in}}%
\pgfpathlineto{\pgfqpoint{1.470967in}{1.451578in}}%
\pgfpathlineto{\pgfqpoint{1.502327in}{1.451578in}}%
\pgfpathlineto{\pgfqpoint{1.533687in}{1.451578in}}%
\pgfpathlineto{\pgfqpoint{1.565047in}{1.451578in}}%
\pgfpathlineto{\pgfqpoint{1.596408in}{1.451578in}}%
\pgfpathlineto{\pgfqpoint{1.627768in}{1.451578in}}%
\pgfpathlineto{\pgfqpoint{1.659128in}{1.451578in}}%
\pgfpathlineto{\pgfqpoint{1.690488in}{1.451578in}}%
\pgfpathlineto{\pgfqpoint{1.721849in}{1.451578in}}%
\pgfpathlineto{\pgfqpoint{1.753209in}{1.451578in}}%
\pgfpathlineto{\pgfqpoint{1.784569in}{1.451578in}}%
\pgfpathlineto{\pgfqpoint{1.815929in}{1.451578in}}%
\pgfpathlineto{\pgfqpoint{1.847289in}{1.451578in}}%
\pgfpathlineto{\pgfqpoint{1.878650in}{1.451578in}}%
\pgfpathlineto{\pgfqpoint{1.910010in}{1.451578in}}%
\pgfpathlineto{\pgfqpoint{1.941370in}{1.451578in}}%
\pgfpathlineto{\pgfqpoint{1.972730in}{1.451578in}}%
\pgfpathlineto{\pgfqpoint{2.004091in}{1.451578in}}%
\pgfpathlineto{\pgfqpoint{2.035451in}{1.451578in}}%
\pgfpathlineto{\pgfqpoint{2.066811in}{1.451578in}}%
\pgfpathlineto{\pgfqpoint{2.098171in}{1.451578in}}%
\pgfpathlineto{\pgfqpoint{2.129532in}{1.451578in}}%
\pgfpathlineto{\pgfqpoint{2.160892in}{1.451578in}}%
\pgfpathlineto{\pgfqpoint{2.192252in}{1.451578in}}%
\pgfpathlineto{\pgfqpoint{2.223612in}{1.451578in}}%
\pgfpathlineto{\pgfqpoint{2.254972in}{1.451578in}}%
\pgfpathlineto{\pgfqpoint{2.286333in}{1.451578in}}%
\pgfpathlineto{\pgfqpoint{2.317693in}{1.451578in}}%
\pgfpathlineto{\pgfqpoint{2.349053in}{1.451578in}}%
\pgfpathlineto{\pgfqpoint{2.380413in}{1.451578in}}%
\pgfpathlineto{\pgfqpoint{2.411774in}{1.451578in}}%
\pgfpathlineto{\pgfqpoint{2.443134in}{1.451578in}}%
\pgfpathlineto{\pgfqpoint{2.474494in}{1.451578in}}%
\pgfpathlineto{\pgfqpoint{2.505854in}{1.451578in}}%
\pgfpathlineto{\pgfqpoint{2.537215in}{1.451578in}}%
\pgfpathlineto{\pgfqpoint{2.568575in}{1.451578in}}%
\pgfpathlineto{\pgfqpoint{2.599935in}{1.451578in}}%
\pgfpathlineto{\pgfqpoint{2.631295in}{1.451578in}}%
\pgfpathlineto{\pgfqpoint{2.662655in}{1.451578in}}%
\pgfpathlineto{\pgfqpoint{2.694016in}{1.451578in}}%
\pgfpathlineto{\pgfqpoint{2.725376in}{1.451578in}}%
\pgfpathlineto{\pgfqpoint{2.756736in}{1.451578in}}%
\pgfpathlineto{\pgfqpoint{2.788096in}{1.451578in}}%
\pgfpathlineto{\pgfqpoint{2.819457in}{1.451578in}}%
\pgfpathlineto{\pgfqpoint{2.850817in}{1.451578in}}%
\pgfpathlineto{\pgfqpoint{2.882177in}{1.451578in}}%
\pgfpathlineto{\pgfqpoint{2.913537in}{1.451578in}}%
\pgfpathlineto{\pgfqpoint{2.944897in}{1.451578in}}%
\pgfpathlineto{\pgfqpoint{2.976258in}{1.451578in}}%
\pgfpathlineto{\pgfqpoint{3.007618in}{1.451578in}}%
\pgfpathlineto{\pgfqpoint{3.038978in}{1.451578in}}%
\pgfpathlineto{\pgfqpoint{3.070338in}{1.451578in}}%
\pgfpathlineto{\pgfqpoint{3.101699in}{1.451578in}}%
\pgfpathlineto{\pgfqpoint{3.133059in}{1.451578in}}%
\pgfpathlineto{\pgfqpoint{3.164419in}{1.451578in}}%
\pgfpathlineto{\pgfqpoint{3.195779in}{1.451578in}}%
\pgfpathlineto{\pgfqpoint{3.227140in}{1.451578in}}%
\pgfpathlineto{\pgfqpoint{3.258500in}{1.451578in}}%
\pgfpathlineto{\pgfqpoint{3.289860in}{1.451578in}}%
\pgfpathlineto{\pgfqpoint{3.321220in}{1.451578in}}%
\pgfpathlineto{\pgfqpoint{3.352580in}{1.451578in}}%
\pgfpathlineto{\pgfqpoint{3.383941in}{1.451578in}}%
\pgfpathlineto{\pgfqpoint{3.415301in}{1.451578in}}%
\pgfpathlineto{\pgfqpoint{3.446661in}{1.451578in}}%
\pgfpathlineto{\pgfqpoint{3.478021in}{1.451578in}}%
\pgfpathlineto{\pgfqpoint{3.509382in}{1.451578in}}%
\pgfpathlineto{\pgfqpoint{3.540742in}{1.451578in}}%
\pgfpathlineto{\pgfqpoint{3.572102in}{1.451578in}}%
\pgfpathlineto{\pgfqpoint{3.603462in}{1.451578in}}%
\pgfpathlineto{\pgfqpoint{3.634823in}{1.451578in}}%
\pgfpathlineto{\pgfqpoint{3.666183in}{1.451578in}}%
\pgfpathlineto{\pgfqpoint{3.697543in}{1.451578in}}%
\pgfpathlineto{\pgfqpoint{3.728903in}{1.451578in}}%
\pgfpathlineto{\pgfqpoint{3.760263in}{1.451578in}}%
\pgfpathlineto{\pgfqpoint{3.791624in}{1.451578in}}%
\pgfpathlineto{\pgfqpoint{3.822984in}{1.451578in}}%
\pgfpathlineto{\pgfqpoint{3.854344in}{1.451578in}}%
\pgfpathlineto{\pgfqpoint{3.885704in}{1.451578in}}%
\pgfpathlineto{\pgfqpoint{3.917065in}{1.451578in}}%
\pgfpathlineto{\pgfqpoint{3.948425in}{1.451578in}}%
\pgfpathlineto{\pgfqpoint{3.979785in}{1.451578in}}%
\pgfpathlineto{\pgfqpoint{4.011145in}{1.451578in}}%
\pgfpathlineto{\pgfqpoint{4.042506in}{1.451578in}}%
\pgfpathlineto{\pgfqpoint{4.073866in}{1.451578in}}%
\pgfpathlineto{\pgfqpoint{4.105226in}{1.451578in}}%
\pgfpathlineto{\pgfqpoint{4.136586in}{1.451578in}}%
\pgfpathlineto{\pgfqpoint{4.167946in}{1.451578in}}%
\pgfpathlineto{\pgfqpoint{4.199307in}{1.451578in}}%
\pgfpathlineto{\pgfqpoint{4.230667in}{1.451578in}}%
\pgfpathlineto{\pgfqpoint{4.262027in}{1.451578in}}%
\pgfpathlineto{\pgfqpoint{4.293387in}{1.451578in}}%
\pgfpathlineto{\pgfqpoint{4.324748in}{1.451578in}}%
\pgfpathlineto{\pgfqpoint{4.356108in}{1.451578in}}%
\pgfpathlineto{\pgfqpoint{4.387468in}{1.451578in}}%
\pgfpathlineto{\pgfqpoint{4.418828in}{1.451578in}}%
\pgfpathlineto{\pgfqpoint{4.450188in}{1.451578in}}%
\pgfpathlineto{\pgfqpoint{4.481549in}{1.451578in}}%
\pgfpathlineto{\pgfqpoint{4.512909in}{1.451578in}}%
\pgfpathlineto{\pgfqpoint{4.544269in}{1.451578in}}%
\pgfpathlineto{\pgfqpoint{4.575629in}{1.451578in}}%
\pgfpathlineto{\pgfqpoint{4.606990in}{1.451578in}}%
\pgfpathlineto{\pgfqpoint{4.638350in}{1.451578in}}%
\pgfpathlineto{\pgfqpoint{4.669710in}{1.451578in}}%
\pgfpathlineto{\pgfqpoint{4.701070in}{1.451578in}}%
\pgfusepath{stroke}%
\end{pgfscope}%
\begin{pgfscope}%
\pgfpathrectangle{\pgfqpoint{0.583472in}{0.568889in}}{\pgfqpoint{4.346528in}{0.924722in}}%
\pgfusepath{clip}%
\pgfsetrectcap%
\pgfsetroundjoin%
\pgfsetlinewidth{1.505625pt}%
\definecolor{currentstroke}{rgb}{0.172549,0.627451,0.172549}%
\pgfsetstrokecolor{currentstroke}%
\pgfsetdash{}{0pt}%
\pgfpathmoveto{\pgfqpoint{0.781042in}{1.451578in}}%
\pgfpathlineto{\pgfqpoint{0.812402in}{1.451578in}}%
\pgfpathlineto{\pgfqpoint{0.843762in}{1.451578in}}%
\pgfpathlineto{\pgfqpoint{0.875122in}{1.451578in}}%
\pgfpathlineto{\pgfqpoint{0.906483in}{1.451578in}}%
\pgfpathlineto{\pgfqpoint{0.937843in}{1.451578in}}%
\pgfpathlineto{\pgfqpoint{0.969203in}{1.451578in}}%
\pgfpathlineto{\pgfqpoint{1.000563in}{1.451578in}}%
\pgfpathlineto{\pgfqpoint{1.031924in}{1.451578in}}%
\pgfpathlineto{\pgfqpoint{1.063284in}{1.451578in}}%
\pgfpathlineto{\pgfqpoint{1.094644in}{1.451578in}}%
\pgfpathlineto{\pgfqpoint{1.126004in}{1.451578in}}%
\pgfpathlineto{\pgfqpoint{1.157364in}{1.451578in}}%
\pgfpathlineto{\pgfqpoint{1.188725in}{1.451578in}}%
\pgfpathlineto{\pgfqpoint{1.220085in}{1.451578in}}%
\pgfpathlineto{\pgfqpoint{1.251445in}{1.451578in}}%
\pgfpathlineto{\pgfqpoint{1.282805in}{1.451578in}}%
\pgfpathlineto{\pgfqpoint{1.314166in}{1.451578in}}%
\pgfpathlineto{\pgfqpoint{1.345526in}{1.451578in}}%
\pgfpathlineto{\pgfqpoint{1.376886in}{1.451578in}}%
\pgfpathlineto{\pgfqpoint{1.408246in}{1.451578in}}%
\pgfpathlineto{\pgfqpoint{1.439606in}{1.451578in}}%
\pgfpathlineto{\pgfqpoint{1.470967in}{1.451578in}}%
\pgfpathlineto{\pgfqpoint{1.502327in}{1.451578in}}%
\pgfpathlineto{\pgfqpoint{1.533687in}{1.451578in}}%
\pgfpathlineto{\pgfqpoint{1.565047in}{1.451578in}}%
\pgfpathlineto{\pgfqpoint{1.596408in}{1.451578in}}%
\pgfpathlineto{\pgfqpoint{1.627768in}{1.451578in}}%
\pgfpathlineto{\pgfqpoint{1.659128in}{1.451578in}}%
\pgfpathlineto{\pgfqpoint{1.690488in}{1.451578in}}%
\pgfpathlineto{\pgfqpoint{1.721849in}{1.451578in}}%
\pgfpathlineto{\pgfqpoint{1.753209in}{1.451578in}}%
\pgfpathlineto{\pgfqpoint{1.784569in}{1.451578in}}%
\pgfpathlineto{\pgfqpoint{1.815929in}{1.451578in}}%
\pgfpathlineto{\pgfqpoint{1.847289in}{1.451578in}}%
\pgfpathlineto{\pgfqpoint{1.878650in}{1.451578in}}%
\pgfpathlineto{\pgfqpoint{1.910010in}{1.451578in}}%
\pgfpathlineto{\pgfqpoint{1.941370in}{1.451578in}}%
\pgfpathlineto{\pgfqpoint{1.972730in}{1.451578in}}%
\pgfpathlineto{\pgfqpoint{2.004091in}{1.451578in}}%
\pgfpathlineto{\pgfqpoint{2.035451in}{1.451578in}}%
\pgfpathlineto{\pgfqpoint{2.066811in}{1.451578in}}%
\pgfpathlineto{\pgfqpoint{2.098171in}{1.451578in}}%
\pgfpathlineto{\pgfqpoint{2.129532in}{1.451578in}}%
\pgfpathlineto{\pgfqpoint{2.160892in}{1.451578in}}%
\pgfpathlineto{\pgfqpoint{2.192252in}{1.451578in}}%
\pgfpathlineto{\pgfqpoint{2.223612in}{1.451578in}}%
\pgfpathlineto{\pgfqpoint{2.254972in}{1.451578in}}%
\pgfpathlineto{\pgfqpoint{2.286333in}{1.451578in}}%
\pgfpathlineto{\pgfqpoint{2.317693in}{1.451578in}}%
\pgfpathlineto{\pgfqpoint{2.349053in}{1.451578in}}%
\pgfpathlineto{\pgfqpoint{2.380413in}{1.451578in}}%
\pgfpathlineto{\pgfqpoint{2.411774in}{1.451578in}}%
\pgfpathlineto{\pgfqpoint{2.443134in}{1.451578in}}%
\pgfpathlineto{\pgfqpoint{2.474494in}{1.451578in}}%
\pgfpathlineto{\pgfqpoint{2.505854in}{1.451578in}}%
\pgfpathlineto{\pgfqpoint{2.537215in}{1.451578in}}%
\pgfpathlineto{\pgfqpoint{2.568575in}{1.451578in}}%
\pgfpathlineto{\pgfqpoint{2.599935in}{1.451578in}}%
\pgfpathlineto{\pgfqpoint{2.631295in}{1.451578in}}%
\pgfpathlineto{\pgfqpoint{2.662655in}{1.451578in}}%
\pgfpathlineto{\pgfqpoint{2.694016in}{1.451578in}}%
\pgfpathlineto{\pgfqpoint{2.725376in}{1.451578in}}%
\pgfpathlineto{\pgfqpoint{2.756736in}{1.451578in}}%
\pgfpathlineto{\pgfqpoint{2.788096in}{1.451578in}}%
\pgfpathlineto{\pgfqpoint{2.819457in}{1.451578in}}%
\pgfpathlineto{\pgfqpoint{2.850817in}{1.451578in}}%
\pgfpathlineto{\pgfqpoint{2.882177in}{1.451578in}}%
\pgfpathlineto{\pgfqpoint{2.913537in}{1.451578in}}%
\pgfpathlineto{\pgfqpoint{2.944897in}{1.451578in}}%
\pgfpathlineto{\pgfqpoint{2.976258in}{1.451578in}}%
\pgfpathlineto{\pgfqpoint{3.007618in}{1.451578in}}%
\pgfpathlineto{\pgfqpoint{3.038978in}{1.451578in}}%
\pgfpathlineto{\pgfqpoint{3.070338in}{1.451578in}}%
\pgfpathlineto{\pgfqpoint{3.101699in}{1.451578in}}%
\pgfpathlineto{\pgfqpoint{3.133059in}{1.451578in}}%
\pgfpathlineto{\pgfqpoint{3.164419in}{1.451578in}}%
\pgfpathlineto{\pgfqpoint{3.195779in}{1.451578in}}%
\pgfpathlineto{\pgfqpoint{3.227140in}{1.451578in}}%
\pgfpathlineto{\pgfqpoint{3.258500in}{1.451578in}}%
\pgfpathlineto{\pgfqpoint{3.289860in}{1.451578in}}%
\pgfpathlineto{\pgfqpoint{3.321220in}{1.451578in}}%
\pgfpathlineto{\pgfqpoint{3.352580in}{1.451578in}}%
\pgfpathlineto{\pgfqpoint{3.383941in}{1.451578in}}%
\pgfpathlineto{\pgfqpoint{3.415301in}{1.451578in}}%
\pgfpathlineto{\pgfqpoint{3.446661in}{1.451578in}}%
\pgfpathlineto{\pgfqpoint{3.478021in}{1.451578in}}%
\pgfpathlineto{\pgfqpoint{3.509382in}{1.451578in}}%
\pgfpathlineto{\pgfqpoint{3.540742in}{1.451578in}}%
\pgfpathlineto{\pgfqpoint{3.572102in}{1.451578in}}%
\pgfpathlineto{\pgfqpoint{3.603462in}{1.451578in}}%
\pgfpathlineto{\pgfqpoint{3.634823in}{1.451578in}}%
\pgfpathlineto{\pgfqpoint{3.666183in}{1.451578in}}%
\pgfpathlineto{\pgfqpoint{3.697543in}{1.451578in}}%
\pgfpathlineto{\pgfqpoint{3.728903in}{1.451578in}}%
\pgfpathlineto{\pgfqpoint{3.760263in}{1.451578in}}%
\pgfpathlineto{\pgfqpoint{3.791624in}{1.451578in}}%
\pgfpathlineto{\pgfqpoint{3.822984in}{1.451578in}}%
\pgfpathlineto{\pgfqpoint{3.854344in}{1.451578in}}%
\pgfpathlineto{\pgfqpoint{3.885704in}{1.451578in}}%
\pgfpathlineto{\pgfqpoint{3.917065in}{1.451578in}}%
\pgfpathlineto{\pgfqpoint{3.948425in}{1.451578in}}%
\pgfpathlineto{\pgfqpoint{3.979785in}{1.451578in}}%
\pgfpathlineto{\pgfqpoint{4.011145in}{1.451578in}}%
\pgfpathlineto{\pgfqpoint{4.042506in}{1.451578in}}%
\pgfpathlineto{\pgfqpoint{4.073866in}{1.451578in}}%
\pgfpathlineto{\pgfqpoint{4.105226in}{1.451578in}}%
\pgfpathlineto{\pgfqpoint{4.136586in}{1.451578in}}%
\pgfpathlineto{\pgfqpoint{4.167946in}{1.451578in}}%
\pgfpathlineto{\pgfqpoint{4.199307in}{1.451578in}}%
\pgfpathlineto{\pgfqpoint{4.230667in}{1.451578in}}%
\pgfpathlineto{\pgfqpoint{4.262027in}{1.451578in}}%
\pgfpathlineto{\pgfqpoint{4.293387in}{1.451578in}}%
\pgfpathlineto{\pgfqpoint{4.324748in}{1.451578in}}%
\pgfpathlineto{\pgfqpoint{4.356108in}{1.451578in}}%
\pgfpathlineto{\pgfqpoint{4.387468in}{1.451578in}}%
\pgfpathlineto{\pgfqpoint{4.418828in}{1.451578in}}%
\pgfpathlineto{\pgfqpoint{4.450188in}{1.451578in}}%
\pgfpathlineto{\pgfqpoint{4.481549in}{1.451578in}}%
\pgfpathlineto{\pgfqpoint{4.512909in}{1.451578in}}%
\pgfpathlineto{\pgfqpoint{4.544269in}{1.451578in}}%
\pgfpathlineto{\pgfqpoint{4.575629in}{1.451578in}}%
\pgfpathlineto{\pgfqpoint{4.606990in}{1.451578in}}%
\pgfpathlineto{\pgfqpoint{4.638350in}{1.451578in}}%
\pgfpathlineto{\pgfqpoint{4.669710in}{1.451578in}}%
\pgfpathlineto{\pgfqpoint{4.701070in}{1.451578in}}%
\pgfusepath{stroke}%
\end{pgfscope}%
\begin{pgfscope}%
\pgfpathrectangle{\pgfqpoint{0.583472in}{0.568889in}}{\pgfqpoint{4.346528in}{0.924722in}}%
\pgfusepath{clip}%
\pgfsetrectcap%
\pgfsetroundjoin%
\pgfsetlinewidth{1.505625pt}%
\definecolor{currentstroke}{rgb}{0.839216,0.152941,0.156863}%
\pgfsetstrokecolor{currentstroke}%
\pgfsetdash{}{0pt}%
\pgfpathmoveto{\pgfqpoint{0.781042in}{0.610922in}}%
\pgfpathlineto{\pgfqpoint{0.812402in}{0.610922in}}%
\pgfpathlineto{\pgfqpoint{0.843762in}{0.610922in}}%
\pgfpathlineto{\pgfqpoint{0.875122in}{0.610922in}}%
\pgfpathlineto{\pgfqpoint{0.906483in}{0.610922in}}%
\pgfpathlineto{\pgfqpoint{0.937843in}{0.610922in}}%
\pgfpathlineto{\pgfqpoint{0.969203in}{0.610922in}}%
\pgfpathlineto{\pgfqpoint{1.000563in}{0.610922in}}%
\pgfpathlineto{\pgfqpoint{1.031924in}{0.610922in}}%
\pgfpathlineto{\pgfqpoint{1.063284in}{0.610922in}}%
\pgfpathlineto{\pgfqpoint{1.094644in}{0.610922in}}%
\pgfpathlineto{\pgfqpoint{1.126004in}{0.610922in}}%
\pgfpathlineto{\pgfqpoint{1.157364in}{0.610922in}}%
\pgfpathlineto{\pgfqpoint{1.188725in}{0.610922in}}%
\pgfpathlineto{\pgfqpoint{1.220085in}{0.610922in}}%
\pgfpathlineto{\pgfqpoint{1.251445in}{0.610922in}}%
\pgfpathlineto{\pgfqpoint{1.282805in}{0.610922in}}%
\pgfpathlineto{\pgfqpoint{1.314166in}{0.610922in}}%
\pgfpathlineto{\pgfqpoint{1.345526in}{0.610922in}}%
\pgfpathlineto{\pgfqpoint{1.376886in}{0.610922in}}%
\pgfpathlineto{\pgfqpoint{1.408246in}{0.610922in}}%
\pgfpathlineto{\pgfqpoint{1.439606in}{0.610922in}}%
\pgfpathlineto{\pgfqpoint{1.470967in}{0.610922in}}%
\pgfpathlineto{\pgfqpoint{1.502327in}{0.610922in}}%
\pgfpathlineto{\pgfqpoint{1.533687in}{0.610922in}}%
\pgfpathlineto{\pgfqpoint{1.565047in}{0.610922in}}%
\pgfpathlineto{\pgfqpoint{1.596408in}{0.610922in}}%
\pgfpathlineto{\pgfqpoint{1.627768in}{0.610922in}}%
\pgfpathlineto{\pgfqpoint{1.659128in}{0.610922in}}%
\pgfpathlineto{\pgfqpoint{1.690488in}{0.610922in}}%
\pgfpathlineto{\pgfqpoint{1.721849in}{0.610922in}}%
\pgfpathlineto{\pgfqpoint{1.753209in}{0.610922in}}%
\pgfpathlineto{\pgfqpoint{1.784569in}{0.610922in}}%
\pgfpathlineto{\pgfqpoint{1.815929in}{0.610922in}}%
\pgfpathlineto{\pgfqpoint{1.847289in}{0.610922in}}%
\pgfpathlineto{\pgfqpoint{1.878650in}{0.610922in}}%
\pgfpathlineto{\pgfqpoint{1.910010in}{0.610922in}}%
\pgfpathlineto{\pgfqpoint{1.941370in}{0.610922in}}%
\pgfpathlineto{\pgfqpoint{1.972730in}{0.610922in}}%
\pgfpathlineto{\pgfqpoint{2.004091in}{0.610922in}}%
\pgfpathlineto{\pgfqpoint{2.035451in}{0.610922in}}%
\pgfpathlineto{\pgfqpoint{2.066811in}{0.610922in}}%
\pgfpathlineto{\pgfqpoint{2.098171in}{0.610922in}}%
\pgfpathlineto{\pgfqpoint{2.129532in}{0.610922in}}%
\pgfpathlineto{\pgfqpoint{2.160892in}{0.610922in}}%
\pgfpathlineto{\pgfqpoint{2.192252in}{0.610922in}}%
\pgfpathlineto{\pgfqpoint{2.223612in}{0.610922in}}%
\pgfpathlineto{\pgfqpoint{2.254972in}{0.610922in}}%
\pgfpathlineto{\pgfqpoint{2.286333in}{0.610922in}}%
\pgfpathlineto{\pgfqpoint{2.317693in}{0.610922in}}%
\pgfpathlineto{\pgfqpoint{2.349053in}{0.610922in}}%
\pgfpathlineto{\pgfqpoint{2.380413in}{0.610922in}}%
\pgfpathlineto{\pgfqpoint{2.411774in}{0.610922in}}%
\pgfpathlineto{\pgfqpoint{2.443134in}{0.610922in}}%
\pgfpathlineto{\pgfqpoint{2.474494in}{0.610922in}}%
\pgfpathlineto{\pgfqpoint{2.505854in}{0.610922in}}%
\pgfpathlineto{\pgfqpoint{2.537215in}{0.610922in}}%
\pgfpathlineto{\pgfqpoint{2.568575in}{0.610922in}}%
\pgfpathlineto{\pgfqpoint{2.599935in}{0.610922in}}%
\pgfpathlineto{\pgfqpoint{2.631295in}{0.610922in}}%
\pgfpathlineto{\pgfqpoint{2.662655in}{0.610922in}}%
\pgfpathlineto{\pgfqpoint{2.694016in}{0.610922in}}%
\pgfpathlineto{\pgfqpoint{2.725376in}{0.610922in}}%
\pgfpathlineto{\pgfqpoint{2.756736in}{0.610922in}}%
\pgfpathlineto{\pgfqpoint{2.788096in}{0.610922in}}%
\pgfpathlineto{\pgfqpoint{2.819457in}{0.610922in}}%
\pgfpathlineto{\pgfqpoint{2.850817in}{0.610922in}}%
\pgfpathlineto{\pgfqpoint{2.882177in}{0.610922in}}%
\pgfpathlineto{\pgfqpoint{2.913537in}{0.610922in}}%
\pgfpathlineto{\pgfqpoint{2.944897in}{0.610922in}}%
\pgfpathlineto{\pgfqpoint{2.976258in}{0.610922in}}%
\pgfpathlineto{\pgfqpoint{3.007618in}{0.610922in}}%
\pgfpathlineto{\pgfqpoint{3.038978in}{0.610922in}}%
\pgfpathlineto{\pgfqpoint{3.070338in}{0.610922in}}%
\pgfpathlineto{\pgfqpoint{3.101699in}{0.610922in}}%
\pgfpathlineto{\pgfqpoint{3.133059in}{0.610922in}}%
\pgfpathlineto{\pgfqpoint{3.164419in}{0.610922in}}%
\pgfpathlineto{\pgfqpoint{3.195779in}{0.610922in}}%
\pgfpathlineto{\pgfqpoint{3.227140in}{0.610922in}}%
\pgfpathlineto{\pgfqpoint{3.258500in}{0.610922in}}%
\pgfpathlineto{\pgfqpoint{3.289860in}{0.610922in}}%
\pgfpathlineto{\pgfqpoint{3.321220in}{0.610922in}}%
\pgfpathlineto{\pgfqpoint{3.352580in}{0.610922in}}%
\pgfpathlineto{\pgfqpoint{3.383941in}{0.610922in}}%
\pgfpathlineto{\pgfqpoint{3.415301in}{0.610922in}}%
\pgfpathlineto{\pgfqpoint{3.446661in}{0.610922in}}%
\pgfpathlineto{\pgfqpoint{3.478021in}{0.610922in}}%
\pgfpathlineto{\pgfqpoint{3.509382in}{0.610922in}}%
\pgfpathlineto{\pgfqpoint{3.540742in}{0.610922in}}%
\pgfpathlineto{\pgfqpoint{3.572102in}{0.610922in}}%
\pgfpathlineto{\pgfqpoint{3.603462in}{0.610922in}}%
\pgfpathlineto{\pgfqpoint{3.634823in}{0.610922in}}%
\pgfpathlineto{\pgfqpoint{3.666183in}{0.610922in}}%
\pgfpathlineto{\pgfqpoint{3.697543in}{0.610922in}}%
\pgfpathlineto{\pgfqpoint{3.728903in}{0.610922in}}%
\pgfpathlineto{\pgfqpoint{3.760263in}{0.610922in}}%
\pgfpathlineto{\pgfqpoint{3.791624in}{0.610922in}}%
\pgfpathlineto{\pgfqpoint{3.822984in}{0.610922in}}%
\pgfpathlineto{\pgfqpoint{3.854344in}{0.610922in}}%
\pgfpathlineto{\pgfqpoint{3.885704in}{0.610922in}}%
\pgfpathlineto{\pgfqpoint{3.917065in}{0.610922in}}%
\pgfpathlineto{\pgfqpoint{3.948425in}{0.610922in}}%
\pgfpathlineto{\pgfqpoint{3.979785in}{0.610922in}}%
\pgfpathlineto{\pgfqpoint{4.011145in}{0.610922in}}%
\pgfpathlineto{\pgfqpoint{4.042506in}{0.610922in}}%
\pgfpathlineto{\pgfqpoint{4.073866in}{0.610922in}}%
\pgfpathlineto{\pgfqpoint{4.105226in}{0.610922in}}%
\pgfpathlineto{\pgfqpoint{4.136586in}{0.610922in}}%
\pgfpathlineto{\pgfqpoint{4.167946in}{0.610922in}}%
\pgfpathlineto{\pgfqpoint{4.199307in}{0.610922in}}%
\pgfpathlineto{\pgfqpoint{4.230667in}{0.610922in}}%
\pgfpathlineto{\pgfqpoint{4.262027in}{0.610922in}}%
\pgfpathlineto{\pgfqpoint{4.293387in}{0.610922in}}%
\pgfpathlineto{\pgfqpoint{4.324748in}{0.610922in}}%
\pgfpathlineto{\pgfqpoint{4.356108in}{0.610922in}}%
\pgfpathlineto{\pgfqpoint{4.387468in}{0.610922in}}%
\pgfpathlineto{\pgfqpoint{4.418828in}{0.610922in}}%
\pgfpathlineto{\pgfqpoint{4.450188in}{0.610922in}}%
\pgfpathlineto{\pgfqpoint{4.481549in}{0.610922in}}%
\pgfpathlineto{\pgfqpoint{4.512909in}{0.610922in}}%
\pgfpathlineto{\pgfqpoint{4.544269in}{0.610922in}}%
\pgfpathlineto{\pgfqpoint{4.575629in}{0.610922in}}%
\pgfpathlineto{\pgfqpoint{4.606990in}{0.610922in}}%
\pgfpathlineto{\pgfqpoint{4.638350in}{0.610922in}}%
\pgfpathlineto{\pgfqpoint{4.669710in}{0.610922in}}%
\pgfpathlineto{\pgfqpoint{4.701070in}{0.610922in}}%
\pgfusepath{stroke}%
\end{pgfscope}%
\begin{pgfscope}%
\pgfsetrectcap%
\pgfsetmiterjoin%
\pgfsetlinewidth{0.803000pt}%
\definecolor{currentstroke}{rgb}{0.000000,0.000000,0.000000}%
\pgfsetstrokecolor{currentstroke}%
\pgfsetdash{}{0pt}%
\pgfpathmoveto{\pgfqpoint{0.583472in}{0.568889in}}%
\pgfpathlineto{\pgfqpoint{0.583472in}{1.493611in}}%
\pgfusepath{stroke}%
\end{pgfscope}%
\begin{pgfscope}%
\pgfsetrectcap%
\pgfsetmiterjoin%
\pgfsetlinewidth{0.803000pt}%
\definecolor{currentstroke}{rgb}{0.000000,0.000000,0.000000}%
\pgfsetstrokecolor{currentstroke}%
\pgfsetdash{}{0pt}%
\pgfpathmoveto{\pgfqpoint{4.930000in}{0.568889in}}%
\pgfpathlineto{\pgfqpoint{4.930000in}{1.493611in}}%
\pgfusepath{stroke}%
\end{pgfscope}%
\begin{pgfscope}%
\pgfsetrectcap%
\pgfsetmiterjoin%
\pgfsetlinewidth{0.803000pt}%
\definecolor{currentstroke}{rgb}{0.000000,0.000000,0.000000}%
\pgfsetstrokecolor{currentstroke}%
\pgfsetdash{}{0pt}%
\pgfpathmoveto{\pgfqpoint{0.583472in}{0.568889in}}%
\pgfpathlineto{\pgfqpoint{4.930000in}{0.568889in}}%
\pgfusepath{stroke}%
\end{pgfscope}%
\begin{pgfscope}%
\pgfsetrectcap%
\pgfsetmiterjoin%
\pgfsetlinewidth{0.803000pt}%
\definecolor{currentstroke}{rgb}{0.000000,0.000000,0.000000}%
\pgfsetstrokecolor{currentstroke}%
\pgfsetdash{}{0pt}%
\pgfpathmoveto{\pgfqpoint{0.583472in}{1.493611in}}%
\pgfpathlineto{\pgfqpoint{4.930000in}{1.493611in}}%
\pgfusepath{stroke}%
\end{pgfscope}%
\begin{pgfscope}%
\definecolor{textcolor}{rgb}{0.000000,0.000000,0.000000}%
\pgfsetstrokecolor{textcolor}%
\pgfsetfillcolor{textcolor}%
\pgftext[x=2.756736in,y=1.576944in,,base]{\color{textcolor}\sffamily\fontsize{12.000000}{14.400000}\selectfont db3}%
\end{pgfscope}%
\begin{pgfscope}%
\pgfsetbuttcap%
\pgfsetmiterjoin%
\definecolor{currentfill}{rgb}{1.000000,1.000000,1.000000}%
\pgfsetfillcolor{currentfill}%
\pgfsetfillopacity{0.800000}%
\pgfsetlinewidth{1.003750pt}%
\definecolor{currentstroke}{rgb}{0.800000,0.800000,0.800000}%
\pgfsetstrokecolor{currentstroke}%
\pgfsetstrokeopacity{0.800000}%
\pgfsetdash{}{0pt}%
\pgfpathmoveto{\pgfqpoint{5.172203in}{1.191945in}}%
\pgfpathlineto{\pgfqpoint{5.758333in}{1.191945in}}%
\pgfpathquadraticcurveto{\pgfqpoint{5.786111in}{1.191945in}}{\pgfqpoint{5.786111in}{1.219722in}}%
\pgfpathlineto{\pgfqpoint{5.786111in}{1.980277in}}%
\pgfpathquadraticcurveto{\pgfqpoint{5.786111in}{2.008055in}}{\pgfqpoint{5.758333in}{2.008055in}}%
\pgfpathlineto{\pgfqpoint{5.172203in}{2.008055in}}%
\pgfpathquadraticcurveto{\pgfqpoint{5.144425in}{2.008055in}}{\pgfqpoint{5.144425in}{1.980277in}}%
\pgfpathlineto{\pgfqpoint{5.144425in}{1.219722in}}%
\pgfpathquadraticcurveto{\pgfqpoint{5.144425in}{1.191945in}}{\pgfqpoint{5.172203in}{1.191945in}}%
\pgfpathclose%
\pgfusepath{stroke,fill}%
\end{pgfscope}%
\begin{pgfscope}%
\pgfsetrectcap%
\pgfsetroundjoin%
\pgfsetlinewidth{1.505625pt}%
\definecolor{currentstroke}{rgb}{0.121569,0.466667,0.705882}%
\pgfsetstrokecolor{currentstroke}%
\pgfsetdash{}{0pt}%
\pgfpathmoveto{\pgfqpoint{5.199981in}{1.903889in}}%
\pgfpathlineto{\pgfqpoint{5.477758in}{1.903889in}}%
\pgfusepath{stroke}%
\end{pgfscope}%
\begin{pgfscope}%
\definecolor{textcolor}{rgb}{0.000000,0.000000,0.000000}%
\pgfsetstrokecolor{textcolor}%
\pgfsetfillcolor{textcolor}%
\pgftext[x=5.588869in,y=1.855277in,left,base]{\color{textcolor}\sffamily\fontsize{10.000000}{12.000000}\selectfont \(\displaystyle x^{0}\)}%
\end{pgfscope}%
\begin{pgfscope}%
\pgfsetrectcap%
\pgfsetroundjoin%
\pgfsetlinewidth{1.505625pt}%
\definecolor{currentstroke}{rgb}{1.000000,0.498039,0.054902}%
\pgfsetstrokecolor{currentstroke}%
\pgfsetdash{}{0pt}%
\pgfpathmoveto{\pgfqpoint{5.199981in}{1.710278in}}%
\pgfpathlineto{\pgfqpoint{5.477758in}{1.710278in}}%
\pgfusepath{stroke}%
\end{pgfscope}%
\begin{pgfscope}%
\definecolor{textcolor}{rgb}{0.000000,0.000000,0.000000}%
\pgfsetstrokecolor{textcolor}%
\pgfsetfillcolor{textcolor}%
\pgftext[x=5.588869in,y=1.661667in,left,base]{\color{textcolor}\sffamily\fontsize{10.000000}{12.000000}\selectfont \(\displaystyle x^{1}\)}%
\end{pgfscope}%
\begin{pgfscope}%
\pgfsetrectcap%
\pgfsetroundjoin%
\pgfsetlinewidth{1.505625pt}%
\definecolor{currentstroke}{rgb}{0.172549,0.627451,0.172549}%
\pgfsetstrokecolor{currentstroke}%
\pgfsetdash{}{0pt}%
\pgfpathmoveto{\pgfqpoint{5.199981in}{1.516667in}}%
\pgfpathlineto{\pgfqpoint{5.477758in}{1.516667in}}%
\pgfusepath{stroke}%
\end{pgfscope}%
\begin{pgfscope}%
\definecolor{textcolor}{rgb}{0.000000,0.000000,0.000000}%
\pgfsetstrokecolor{textcolor}%
\pgfsetfillcolor{textcolor}%
\pgftext[x=5.588869in,y=1.468056in,left,base]{\color{textcolor}\sffamily\fontsize{10.000000}{12.000000}\selectfont \(\displaystyle x^{2}\)}%
\end{pgfscope}%
\begin{pgfscope}%
\pgfsetrectcap%
\pgfsetroundjoin%
\pgfsetlinewidth{1.505625pt}%
\definecolor{currentstroke}{rgb}{0.839216,0.152941,0.156863}%
\pgfsetstrokecolor{currentstroke}%
\pgfsetdash{}{0pt}%
\pgfpathmoveto{\pgfqpoint{5.199981in}{1.323056in}}%
\pgfpathlineto{\pgfqpoint{5.477758in}{1.323056in}}%
\pgfusepath{stroke}%
\end{pgfscope}%
\begin{pgfscope}%
\definecolor{textcolor}{rgb}{0.000000,0.000000,0.000000}%
\pgfsetstrokecolor{textcolor}%
\pgfsetfillcolor{textcolor}%
\pgftext[x=5.588869in,y=1.274445in,left,base]{\color{textcolor}\sffamily\fontsize{10.000000}{12.000000}\selectfont \(\displaystyle x^{3}\)}%
\end{pgfscope}%
\end{pgfpicture}%
\makeatother%
\endgroup%

    \caption{Analyse mit db2/3 Wavelet\label{polynomials:db2_3}}
\end{figure}

Wie sich herausstellt liefern diese Wavelets jeweils die zweite und dritte
Ableitung unserer Signale. Daubechies Wavelets mit $A$ verschwindenden Momenten
können uns also direkt die $A$te Ableitung liefern. Dies im Gegensatz zum
Differenzieren welches mehrfach angewendet werden muss.

\section{Anwendung zur rauscharmen Ableitung}
\rhead{Rauscharme Ableitung}

\section{Hochfrequente Anteile in Polynomen}
\rhead{Hochfrequente Anteile in Polynomen}

\section{Schlussfolgerung}
\rhead{Schlussfolgerung}

\printbibliography[heading=subbibliography]
\end{refsection}
