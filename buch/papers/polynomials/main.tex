%
% main.tex -- Paper zum Thema Polynome
%
% (c) 2019 Hochschule Rapperswil
%
\chapter{Wavelets und polynomiale Signale\label{chapter:thema}}
\lhead{Wavelets und polynomiale Signale}
\begin{refsection}
\chapterauthor{Raphael Nestler}

In der Literatur zu Wavelets findet man die folgende Aussage zu Daubechies
Wavelets:
\begin{displayquote}[\cite{wikipedia:daubechies}]
For example, $D2$, with one vanishing moment, easily encodes polynomials of one
coefficient, or constant signal components. $D4$ encodes polynomials with two
coefficients, i.e.\ constant and linear signal components; and $D6$ encodes
3-polynomials, i.e.\ constant, linear and quadratic signal components.
\end{displayquote}
Ein Daubechies Wavelet mit $A$ verschwindenden Momenten und Filterlänge $N=2A$
soll also ein Polynom der Ordnung $A-1$ einfach darstellen können. Wir wollen
erörtern, was das nun in der Praxis genau bedeutet und welche Anwendungen es
ermöglicht.

Die Analysen und Simulationen wurden jeweils mittels Python \cite{python} und im
speziellem dem PyWavelets \cite{gregory_r_lee_2019_2634243} Paket durchgeführt.

Wir werden im folgenden Daubechies Wavelets mit $A$ verschwindenden Momenten mit
db$A$ bezeichnen.

\section{Analyse von polynomialen Signalen}
\rhead{Polynomiale Signale}

Wir werden als erstes die Signale in \autoref{polynomials:signals} mittels des db1
(Haar) Wavelets analysieren.

\begin{figure}
    \centering
    %% Creator: Matplotlib, PGF backend
%%
%% To include the figure in your LaTeX document, write
%%   \input{<filename>.pgf}
%%
%% Make sure the required packages are loaded in your preamble
%%   \usepackage{pgf}
%%
%% Figures using additional raster images can only be included by \input if
%% they are in the same directory as the main LaTeX file. For loading figures
%% from other directories you can use the `import` package
%%   \usepackage{import}
%% and then include the figures with
%%   \import{<path to file>}{<filename>.pgf}
%%
%% Matplotlib used the following preamble
%%   \usepackage{fontspec}
%%
\begingroup%
\makeatletter%
\begin{pgfpicture}%
\pgfpathrectangle{\pgfpointorigin}{\pgfqpoint{5.800000in}{3.300000in}}%
\pgfusepath{use as bounding box, clip}%
\begin{pgfscope}%
\pgfsetbuttcap%
\pgfsetmiterjoin%
\definecolor{currentfill}{rgb}{1.000000,1.000000,1.000000}%
\pgfsetfillcolor{currentfill}%
\pgfsetlinewidth{0.000000pt}%
\definecolor{currentstroke}{rgb}{1.000000,1.000000,1.000000}%
\pgfsetstrokecolor{currentstroke}%
\pgfsetdash{}{0pt}%
\pgfpathmoveto{\pgfqpoint{0.000000in}{0.000000in}}%
\pgfpathlineto{\pgfqpoint{5.800000in}{0.000000in}}%
\pgfpathlineto{\pgfqpoint{5.800000in}{3.300000in}}%
\pgfpathlineto{\pgfqpoint{0.000000in}{3.300000in}}%
\pgfpathclose%
\pgfusepath{fill}%
\end{pgfscope}%
\begin{pgfscope}%
\pgfsetbuttcap%
\pgfsetmiterjoin%
\definecolor{currentfill}{rgb}{1.000000,1.000000,1.000000}%
\pgfsetfillcolor{currentfill}%
\pgfsetlinewidth{0.000000pt}%
\definecolor{currentstroke}{rgb}{0.000000,0.000000,0.000000}%
\pgfsetstrokecolor{currentstroke}%
\pgfsetstrokeopacity{0.000000}%
\pgfsetdash{}{0pt}%
\pgfpathmoveto{\pgfqpoint{0.725000in}{0.363000in}}%
\pgfpathlineto{\pgfqpoint{5.220000in}{0.363000in}}%
\pgfpathlineto{\pgfqpoint{5.220000in}{2.904000in}}%
\pgfpathlineto{\pgfqpoint{0.725000in}{2.904000in}}%
\pgfpathclose%
\pgfusepath{fill}%
\end{pgfscope}%
\begin{pgfscope}%
\pgfsetbuttcap%
\pgfsetroundjoin%
\definecolor{currentfill}{rgb}{0.000000,0.000000,0.000000}%
\pgfsetfillcolor{currentfill}%
\pgfsetlinewidth{0.803000pt}%
\definecolor{currentstroke}{rgb}{0.000000,0.000000,0.000000}%
\pgfsetstrokecolor{currentstroke}%
\pgfsetdash{}{0pt}%
\pgfsys@defobject{currentmarker}{\pgfqpoint{0.000000in}{-0.048611in}}{\pgfqpoint{0.000000in}{0.000000in}}{%
\pgfpathmoveto{\pgfqpoint{0.000000in}{0.000000in}}%
\pgfpathlineto{\pgfqpoint{0.000000in}{-0.048611in}}%
\pgfusepath{stroke,fill}%
}%
\begin{pgfscope}%
\pgfsys@transformshift{0.929318in}{0.363000in}%
\pgfsys@useobject{currentmarker}{}%
\end{pgfscope}%
\end{pgfscope}%
\begin{pgfscope}%
\definecolor{textcolor}{rgb}{0.000000,0.000000,0.000000}%
\pgfsetstrokecolor{textcolor}%
\pgfsetfillcolor{textcolor}%
\pgftext[x=0.929318in,y=0.265778in,,top]{\color{textcolor}\sffamily\fontsize{10.000000}{12.000000}\selectfont 0.00}%
\end{pgfscope}%
\begin{pgfscope}%
\pgfsetbuttcap%
\pgfsetroundjoin%
\definecolor{currentfill}{rgb}{0.000000,0.000000,0.000000}%
\pgfsetfillcolor{currentfill}%
\pgfsetlinewidth{0.803000pt}%
\definecolor{currentstroke}{rgb}{0.000000,0.000000,0.000000}%
\pgfsetstrokecolor{currentstroke}%
\pgfsetdash{}{0pt}%
\pgfsys@defobject{currentmarker}{\pgfqpoint{0.000000in}{-0.048611in}}{\pgfqpoint{0.000000in}{0.000000in}}{%
\pgfpathmoveto{\pgfqpoint{0.000000in}{0.000000in}}%
\pgfpathlineto{\pgfqpoint{0.000000in}{-0.048611in}}%
\pgfusepath{stroke,fill}%
}%
\begin{pgfscope}%
\pgfsys@transformshift{1.440114in}{0.363000in}%
\pgfsys@useobject{currentmarker}{}%
\end{pgfscope}%
\end{pgfscope}%
\begin{pgfscope}%
\definecolor{textcolor}{rgb}{0.000000,0.000000,0.000000}%
\pgfsetstrokecolor{textcolor}%
\pgfsetfillcolor{textcolor}%
\pgftext[x=1.440114in,y=0.265778in,,top]{\color{textcolor}\sffamily\fontsize{10.000000}{12.000000}\selectfont 0.25}%
\end{pgfscope}%
\begin{pgfscope}%
\pgfsetbuttcap%
\pgfsetroundjoin%
\definecolor{currentfill}{rgb}{0.000000,0.000000,0.000000}%
\pgfsetfillcolor{currentfill}%
\pgfsetlinewidth{0.803000pt}%
\definecolor{currentstroke}{rgb}{0.000000,0.000000,0.000000}%
\pgfsetstrokecolor{currentstroke}%
\pgfsetdash{}{0pt}%
\pgfsys@defobject{currentmarker}{\pgfqpoint{0.000000in}{-0.048611in}}{\pgfqpoint{0.000000in}{0.000000in}}{%
\pgfpathmoveto{\pgfqpoint{0.000000in}{0.000000in}}%
\pgfpathlineto{\pgfqpoint{0.000000in}{-0.048611in}}%
\pgfusepath{stroke,fill}%
}%
\begin{pgfscope}%
\pgfsys@transformshift{1.950909in}{0.363000in}%
\pgfsys@useobject{currentmarker}{}%
\end{pgfscope}%
\end{pgfscope}%
\begin{pgfscope}%
\definecolor{textcolor}{rgb}{0.000000,0.000000,0.000000}%
\pgfsetstrokecolor{textcolor}%
\pgfsetfillcolor{textcolor}%
\pgftext[x=1.950909in,y=0.265778in,,top]{\color{textcolor}\sffamily\fontsize{10.000000}{12.000000}\selectfont 0.50}%
\end{pgfscope}%
\begin{pgfscope}%
\pgfsetbuttcap%
\pgfsetroundjoin%
\definecolor{currentfill}{rgb}{0.000000,0.000000,0.000000}%
\pgfsetfillcolor{currentfill}%
\pgfsetlinewidth{0.803000pt}%
\definecolor{currentstroke}{rgb}{0.000000,0.000000,0.000000}%
\pgfsetstrokecolor{currentstroke}%
\pgfsetdash{}{0pt}%
\pgfsys@defobject{currentmarker}{\pgfqpoint{0.000000in}{-0.048611in}}{\pgfqpoint{0.000000in}{0.000000in}}{%
\pgfpathmoveto{\pgfqpoint{0.000000in}{0.000000in}}%
\pgfpathlineto{\pgfqpoint{0.000000in}{-0.048611in}}%
\pgfusepath{stroke,fill}%
}%
\begin{pgfscope}%
\pgfsys@transformshift{2.461705in}{0.363000in}%
\pgfsys@useobject{currentmarker}{}%
\end{pgfscope}%
\end{pgfscope}%
\begin{pgfscope}%
\definecolor{textcolor}{rgb}{0.000000,0.000000,0.000000}%
\pgfsetstrokecolor{textcolor}%
\pgfsetfillcolor{textcolor}%
\pgftext[x=2.461705in,y=0.265778in,,top]{\color{textcolor}\sffamily\fontsize{10.000000}{12.000000}\selectfont 0.75}%
\end{pgfscope}%
\begin{pgfscope}%
\pgfsetbuttcap%
\pgfsetroundjoin%
\definecolor{currentfill}{rgb}{0.000000,0.000000,0.000000}%
\pgfsetfillcolor{currentfill}%
\pgfsetlinewidth{0.803000pt}%
\definecolor{currentstroke}{rgb}{0.000000,0.000000,0.000000}%
\pgfsetstrokecolor{currentstroke}%
\pgfsetdash{}{0pt}%
\pgfsys@defobject{currentmarker}{\pgfqpoint{0.000000in}{-0.048611in}}{\pgfqpoint{0.000000in}{0.000000in}}{%
\pgfpathmoveto{\pgfqpoint{0.000000in}{0.000000in}}%
\pgfpathlineto{\pgfqpoint{0.000000in}{-0.048611in}}%
\pgfusepath{stroke,fill}%
}%
\begin{pgfscope}%
\pgfsys@transformshift{2.972500in}{0.363000in}%
\pgfsys@useobject{currentmarker}{}%
\end{pgfscope}%
\end{pgfscope}%
\begin{pgfscope}%
\definecolor{textcolor}{rgb}{0.000000,0.000000,0.000000}%
\pgfsetstrokecolor{textcolor}%
\pgfsetfillcolor{textcolor}%
\pgftext[x=2.972500in,y=0.265778in,,top]{\color{textcolor}\sffamily\fontsize{10.000000}{12.000000}\selectfont 1.00}%
\end{pgfscope}%
\begin{pgfscope}%
\pgfsetbuttcap%
\pgfsetroundjoin%
\definecolor{currentfill}{rgb}{0.000000,0.000000,0.000000}%
\pgfsetfillcolor{currentfill}%
\pgfsetlinewidth{0.803000pt}%
\definecolor{currentstroke}{rgb}{0.000000,0.000000,0.000000}%
\pgfsetstrokecolor{currentstroke}%
\pgfsetdash{}{0pt}%
\pgfsys@defobject{currentmarker}{\pgfqpoint{0.000000in}{-0.048611in}}{\pgfqpoint{0.000000in}{0.000000in}}{%
\pgfpathmoveto{\pgfqpoint{0.000000in}{0.000000in}}%
\pgfpathlineto{\pgfqpoint{0.000000in}{-0.048611in}}%
\pgfusepath{stroke,fill}%
}%
\begin{pgfscope}%
\pgfsys@transformshift{3.483295in}{0.363000in}%
\pgfsys@useobject{currentmarker}{}%
\end{pgfscope}%
\end{pgfscope}%
\begin{pgfscope}%
\definecolor{textcolor}{rgb}{0.000000,0.000000,0.000000}%
\pgfsetstrokecolor{textcolor}%
\pgfsetfillcolor{textcolor}%
\pgftext[x=3.483295in,y=0.265778in,,top]{\color{textcolor}\sffamily\fontsize{10.000000}{12.000000}\selectfont 1.25}%
\end{pgfscope}%
\begin{pgfscope}%
\pgfsetbuttcap%
\pgfsetroundjoin%
\definecolor{currentfill}{rgb}{0.000000,0.000000,0.000000}%
\pgfsetfillcolor{currentfill}%
\pgfsetlinewidth{0.803000pt}%
\definecolor{currentstroke}{rgb}{0.000000,0.000000,0.000000}%
\pgfsetstrokecolor{currentstroke}%
\pgfsetdash{}{0pt}%
\pgfsys@defobject{currentmarker}{\pgfqpoint{0.000000in}{-0.048611in}}{\pgfqpoint{0.000000in}{0.000000in}}{%
\pgfpathmoveto{\pgfqpoint{0.000000in}{0.000000in}}%
\pgfpathlineto{\pgfqpoint{0.000000in}{-0.048611in}}%
\pgfusepath{stroke,fill}%
}%
\begin{pgfscope}%
\pgfsys@transformshift{3.994091in}{0.363000in}%
\pgfsys@useobject{currentmarker}{}%
\end{pgfscope}%
\end{pgfscope}%
\begin{pgfscope}%
\definecolor{textcolor}{rgb}{0.000000,0.000000,0.000000}%
\pgfsetstrokecolor{textcolor}%
\pgfsetfillcolor{textcolor}%
\pgftext[x=3.994091in,y=0.265778in,,top]{\color{textcolor}\sffamily\fontsize{10.000000}{12.000000}\selectfont 1.50}%
\end{pgfscope}%
\begin{pgfscope}%
\pgfsetbuttcap%
\pgfsetroundjoin%
\definecolor{currentfill}{rgb}{0.000000,0.000000,0.000000}%
\pgfsetfillcolor{currentfill}%
\pgfsetlinewidth{0.803000pt}%
\definecolor{currentstroke}{rgb}{0.000000,0.000000,0.000000}%
\pgfsetstrokecolor{currentstroke}%
\pgfsetdash{}{0pt}%
\pgfsys@defobject{currentmarker}{\pgfqpoint{0.000000in}{-0.048611in}}{\pgfqpoint{0.000000in}{0.000000in}}{%
\pgfpathmoveto{\pgfqpoint{0.000000in}{0.000000in}}%
\pgfpathlineto{\pgfqpoint{0.000000in}{-0.048611in}}%
\pgfusepath{stroke,fill}%
}%
\begin{pgfscope}%
\pgfsys@transformshift{4.504886in}{0.363000in}%
\pgfsys@useobject{currentmarker}{}%
\end{pgfscope}%
\end{pgfscope}%
\begin{pgfscope}%
\definecolor{textcolor}{rgb}{0.000000,0.000000,0.000000}%
\pgfsetstrokecolor{textcolor}%
\pgfsetfillcolor{textcolor}%
\pgftext[x=4.504886in,y=0.265778in,,top]{\color{textcolor}\sffamily\fontsize{10.000000}{12.000000}\selectfont 1.75}%
\end{pgfscope}%
\begin{pgfscope}%
\pgfsetbuttcap%
\pgfsetroundjoin%
\definecolor{currentfill}{rgb}{0.000000,0.000000,0.000000}%
\pgfsetfillcolor{currentfill}%
\pgfsetlinewidth{0.803000pt}%
\definecolor{currentstroke}{rgb}{0.000000,0.000000,0.000000}%
\pgfsetstrokecolor{currentstroke}%
\pgfsetdash{}{0pt}%
\pgfsys@defobject{currentmarker}{\pgfqpoint{0.000000in}{-0.048611in}}{\pgfqpoint{0.000000in}{0.000000in}}{%
\pgfpathmoveto{\pgfqpoint{0.000000in}{0.000000in}}%
\pgfpathlineto{\pgfqpoint{0.000000in}{-0.048611in}}%
\pgfusepath{stroke,fill}%
}%
\begin{pgfscope}%
\pgfsys@transformshift{5.015682in}{0.363000in}%
\pgfsys@useobject{currentmarker}{}%
\end{pgfscope}%
\end{pgfscope}%
\begin{pgfscope}%
\definecolor{textcolor}{rgb}{0.000000,0.000000,0.000000}%
\pgfsetstrokecolor{textcolor}%
\pgfsetfillcolor{textcolor}%
\pgftext[x=5.015682in,y=0.265778in,,top]{\color{textcolor}\sffamily\fontsize{10.000000}{12.000000}\selectfont 2.00}%
\end{pgfscope}%
\begin{pgfscope}%
\pgfsetbuttcap%
\pgfsetroundjoin%
\definecolor{currentfill}{rgb}{0.000000,0.000000,0.000000}%
\pgfsetfillcolor{currentfill}%
\pgfsetlinewidth{0.803000pt}%
\definecolor{currentstroke}{rgb}{0.000000,0.000000,0.000000}%
\pgfsetstrokecolor{currentstroke}%
\pgfsetdash{}{0pt}%
\pgfsys@defobject{currentmarker}{\pgfqpoint{-0.048611in}{0.000000in}}{\pgfqpoint{0.000000in}{0.000000in}}{%
\pgfpathmoveto{\pgfqpoint{0.000000in}{0.000000in}}%
\pgfpathlineto{\pgfqpoint{-0.048611in}{0.000000in}}%
\pgfusepath{stroke,fill}%
}%
\begin{pgfscope}%
\pgfsys@transformshift{0.725000in}{0.478500in}%
\pgfsys@useobject{currentmarker}{}%
\end{pgfscope}%
\end{pgfscope}%
\begin{pgfscope}%
\definecolor{textcolor}{rgb}{0.000000,0.000000,0.000000}%
\pgfsetstrokecolor{textcolor}%
\pgfsetfillcolor{textcolor}%
\pgftext[x=0.558333in,y=0.430306in,left,base]{\color{textcolor}\sffamily\fontsize{10.000000}{12.000000}\selectfont 0}%
\end{pgfscope}%
\begin{pgfscope}%
\pgfsetbuttcap%
\pgfsetroundjoin%
\definecolor{currentfill}{rgb}{0.000000,0.000000,0.000000}%
\pgfsetfillcolor{currentfill}%
\pgfsetlinewidth{0.803000pt}%
\definecolor{currentstroke}{rgb}{0.000000,0.000000,0.000000}%
\pgfsetstrokecolor{currentstroke}%
\pgfsetdash{}{0pt}%
\pgfsys@defobject{currentmarker}{\pgfqpoint{-0.048611in}{0.000000in}}{\pgfqpoint{0.000000in}{0.000000in}}{%
\pgfpathmoveto{\pgfqpoint{0.000000in}{0.000000in}}%
\pgfpathlineto{\pgfqpoint{-0.048611in}{0.000000in}}%
\pgfusepath{stroke,fill}%
}%
\begin{pgfscope}%
\pgfsys@transformshift{0.725000in}{0.767250in}%
\pgfsys@useobject{currentmarker}{}%
\end{pgfscope}%
\end{pgfscope}%
\begin{pgfscope}%
\definecolor{textcolor}{rgb}{0.000000,0.000000,0.000000}%
\pgfsetstrokecolor{textcolor}%
\pgfsetfillcolor{textcolor}%
\pgftext[x=0.558333in,y=0.719056in,left,base]{\color{textcolor}\sffamily\fontsize{10.000000}{12.000000}\selectfont 1}%
\end{pgfscope}%
\begin{pgfscope}%
\pgfsetbuttcap%
\pgfsetroundjoin%
\definecolor{currentfill}{rgb}{0.000000,0.000000,0.000000}%
\pgfsetfillcolor{currentfill}%
\pgfsetlinewidth{0.803000pt}%
\definecolor{currentstroke}{rgb}{0.000000,0.000000,0.000000}%
\pgfsetstrokecolor{currentstroke}%
\pgfsetdash{}{0pt}%
\pgfsys@defobject{currentmarker}{\pgfqpoint{-0.048611in}{0.000000in}}{\pgfqpoint{0.000000in}{0.000000in}}{%
\pgfpathmoveto{\pgfqpoint{0.000000in}{0.000000in}}%
\pgfpathlineto{\pgfqpoint{-0.048611in}{0.000000in}}%
\pgfusepath{stroke,fill}%
}%
\begin{pgfscope}%
\pgfsys@transformshift{0.725000in}{1.056000in}%
\pgfsys@useobject{currentmarker}{}%
\end{pgfscope}%
\end{pgfscope}%
\begin{pgfscope}%
\definecolor{textcolor}{rgb}{0.000000,0.000000,0.000000}%
\pgfsetstrokecolor{textcolor}%
\pgfsetfillcolor{textcolor}%
\pgftext[x=0.558333in,y=1.007806in,left,base]{\color{textcolor}\sffamily\fontsize{10.000000}{12.000000}\selectfont 2}%
\end{pgfscope}%
\begin{pgfscope}%
\pgfsetbuttcap%
\pgfsetroundjoin%
\definecolor{currentfill}{rgb}{0.000000,0.000000,0.000000}%
\pgfsetfillcolor{currentfill}%
\pgfsetlinewidth{0.803000pt}%
\definecolor{currentstroke}{rgb}{0.000000,0.000000,0.000000}%
\pgfsetstrokecolor{currentstroke}%
\pgfsetdash{}{0pt}%
\pgfsys@defobject{currentmarker}{\pgfqpoint{-0.048611in}{0.000000in}}{\pgfqpoint{0.000000in}{0.000000in}}{%
\pgfpathmoveto{\pgfqpoint{0.000000in}{0.000000in}}%
\pgfpathlineto{\pgfqpoint{-0.048611in}{0.000000in}}%
\pgfusepath{stroke,fill}%
}%
\begin{pgfscope}%
\pgfsys@transformshift{0.725000in}{1.344750in}%
\pgfsys@useobject{currentmarker}{}%
\end{pgfscope}%
\end{pgfscope}%
\begin{pgfscope}%
\definecolor{textcolor}{rgb}{0.000000,0.000000,0.000000}%
\pgfsetstrokecolor{textcolor}%
\pgfsetfillcolor{textcolor}%
\pgftext[x=0.558333in,y=1.296556in,left,base]{\color{textcolor}\sffamily\fontsize{10.000000}{12.000000}\selectfont 3}%
\end{pgfscope}%
\begin{pgfscope}%
\pgfsetbuttcap%
\pgfsetroundjoin%
\definecolor{currentfill}{rgb}{0.000000,0.000000,0.000000}%
\pgfsetfillcolor{currentfill}%
\pgfsetlinewidth{0.803000pt}%
\definecolor{currentstroke}{rgb}{0.000000,0.000000,0.000000}%
\pgfsetstrokecolor{currentstroke}%
\pgfsetdash{}{0pt}%
\pgfsys@defobject{currentmarker}{\pgfqpoint{-0.048611in}{0.000000in}}{\pgfqpoint{0.000000in}{0.000000in}}{%
\pgfpathmoveto{\pgfqpoint{0.000000in}{0.000000in}}%
\pgfpathlineto{\pgfqpoint{-0.048611in}{0.000000in}}%
\pgfusepath{stroke,fill}%
}%
\begin{pgfscope}%
\pgfsys@transformshift{0.725000in}{1.633500in}%
\pgfsys@useobject{currentmarker}{}%
\end{pgfscope}%
\end{pgfscope}%
\begin{pgfscope}%
\definecolor{textcolor}{rgb}{0.000000,0.000000,0.000000}%
\pgfsetstrokecolor{textcolor}%
\pgfsetfillcolor{textcolor}%
\pgftext[x=0.558333in,y=1.585306in,left,base]{\color{textcolor}\sffamily\fontsize{10.000000}{12.000000}\selectfont 4}%
\end{pgfscope}%
\begin{pgfscope}%
\pgfsetbuttcap%
\pgfsetroundjoin%
\definecolor{currentfill}{rgb}{0.000000,0.000000,0.000000}%
\pgfsetfillcolor{currentfill}%
\pgfsetlinewidth{0.803000pt}%
\definecolor{currentstroke}{rgb}{0.000000,0.000000,0.000000}%
\pgfsetstrokecolor{currentstroke}%
\pgfsetdash{}{0pt}%
\pgfsys@defobject{currentmarker}{\pgfqpoint{-0.048611in}{0.000000in}}{\pgfqpoint{0.000000in}{0.000000in}}{%
\pgfpathmoveto{\pgfqpoint{0.000000in}{0.000000in}}%
\pgfpathlineto{\pgfqpoint{-0.048611in}{0.000000in}}%
\pgfusepath{stroke,fill}%
}%
\begin{pgfscope}%
\pgfsys@transformshift{0.725000in}{1.922250in}%
\pgfsys@useobject{currentmarker}{}%
\end{pgfscope}%
\end{pgfscope}%
\begin{pgfscope}%
\definecolor{textcolor}{rgb}{0.000000,0.000000,0.000000}%
\pgfsetstrokecolor{textcolor}%
\pgfsetfillcolor{textcolor}%
\pgftext[x=0.558333in,y=1.874056in,left,base]{\color{textcolor}\sffamily\fontsize{10.000000}{12.000000}\selectfont 5}%
\end{pgfscope}%
\begin{pgfscope}%
\pgfsetbuttcap%
\pgfsetroundjoin%
\definecolor{currentfill}{rgb}{0.000000,0.000000,0.000000}%
\pgfsetfillcolor{currentfill}%
\pgfsetlinewidth{0.803000pt}%
\definecolor{currentstroke}{rgb}{0.000000,0.000000,0.000000}%
\pgfsetstrokecolor{currentstroke}%
\pgfsetdash{}{0pt}%
\pgfsys@defobject{currentmarker}{\pgfqpoint{-0.048611in}{0.000000in}}{\pgfqpoint{0.000000in}{0.000000in}}{%
\pgfpathmoveto{\pgfqpoint{0.000000in}{0.000000in}}%
\pgfpathlineto{\pgfqpoint{-0.048611in}{0.000000in}}%
\pgfusepath{stroke,fill}%
}%
\begin{pgfscope}%
\pgfsys@transformshift{0.725000in}{2.211000in}%
\pgfsys@useobject{currentmarker}{}%
\end{pgfscope}%
\end{pgfscope}%
\begin{pgfscope}%
\definecolor{textcolor}{rgb}{0.000000,0.000000,0.000000}%
\pgfsetstrokecolor{textcolor}%
\pgfsetfillcolor{textcolor}%
\pgftext[x=0.558333in,y=2.162806in,left,base]{\color{textcolor}\sffamily\fontsize{10.000000}{12.000000}\selectfont 6}%
\end{pgfscope}%
\begin{pgfscope}%
\pgfsetbuttcap%
\pgfsetroundjoin%
\definecolor{currentfill}{rgb}{0.000000,0.000000,0.000000}%
\pgfsetfillcolor{currentfill}%
\pgfsetlinewidth{0.803000pt}%
\definecolor{currentstroke}{rgb}{0.000000,0.000000,0.000000}%
\pgfsetstrokecolor{currentstroke}%
\pgfsetdash{}{0pt}%
\pgfsys@defobject{currentmarker}{\pgfqpoint{-0.048611in}{0.000000in}}{\pgfqpoint{0.000000in}{0.000000in}}{%
\pgfpathmoveto{\pgfqpoint{0.000000in}{0.000000in}}%
\pgfpathlineto{\pgfqpoint{-0.048611in}{0.000000in}}%
\pgfusepath{stroke,fill}%
}%
\begin{pgfscope}%
\pgfsys@transformshift{0.725000in}{2.499750in}%
\pgfsys@useobject{currentmarker}{}%
\end{pgfscope}%
\end{pgfscope}%
\begin{pgfscope}%
\definecolor{textcolor}{rgb}{0.000000,0.000000,0.000000}%
\pgfsetstrokecolor{textcolor}%
\pgfsetfillcolor{textcolor}%
\pgftext[x=0.558333in,y=2.451556in,left,base]{\color{textcolor}\sffamily\fontsize{10.000000}{12.000000}\selectfont 7}%
\end{pgfscope}%
\begin{pgfscope}%
\pgfsetbuttcap%
\pgfsetroundjoin%
\definecolor{currentfill}{rgb}{0.000000,0.000000,0.000000}%
\pgfsetfillcolor{currentfill}%
\pgfsetlinewidth{0.803000pt}%
\definecolor{currentstroke}{rgb}{0.000000,0.000000,0.000000}%
\pgfsetstrokecolor{currentstroke}%
\pgfsetdash{}{0pt}%
\pgfsys@defobject{currentmarker}{\pgfqpoint{-0.048611in}{0.000000in}}{\pgfqpoint{0.000000in}{0.000000in}}{%
\pgfpathmoveto{\pgfqpoint{0.000000in}{0.000000in}}%
\pgfpathlineto{\pgfqpoint{-0.048611in}{0.000000in}}%
\pgfusepath{stroke,fill}%
}%
\begin{pgfscope}%
\pgfsys@transformshift{0.725000in}{2.788500in}%
\pgfsys@useobject{currentmarker}{}%
\end{pgfscope}%
\end{pgfscope}%
\begin{pgfscope}%
\definecolor{textcolor}{rgb}{0.000000,0.000000,0.000000}%
\pgfsetstrokecolor{textcolor}%
\pgfsetfillcolor{textcolor}%
\pgftext[x=0.558333in,y=2.740306in,left,base]{\color{textcolor}\sffamily\fontsize{10.000000}{12.000000}\selectfont 8}%
\end{pgfscope}%
\begin{pgfscope}%
\pgfpathrectangle{\pgfqpoint{0.725000in}{0.363000in}}{\pgfqpoint{4.495000in}{2.541000in}}%
\pgfusepath{clip}%
\pgfsetrectcap%
\pgfsetroundjoin%
\pgfsetlinewidth{1.505625pt}%
\definecolor{currentstroke}{rgb}{0.121569,0.466667,0.705882}%
\pgfsetstrokecolor{currentstroke}%
\pgfsetdash{}{0pt}%
\pgfpathmoveto{\pgfqpoint{0.929318in}{0.478500in}}%
\pgfpathlineto{\pgfqpoint{0.945343in}{0.504072in}}%
\pgfpathlineto{\pgfqpoint{0.961368in}{0.514664in}}%
\pgfpathlineto{\pgfqpoint{0.977393in}{0.522792in}}%
\pgfpathlineto{\pgfqpoint{1.009443in}{0.535681in}}%
\pgfpathlineto{\pgfqpoint{1.041493in}{0.546157in}}%
\pgfpathlineto{\pgfqpoint{1.089568in}{0.559366in}}%
\pgfpathlineto{\pgfqpoint{1.153668in}{0.574182in}}%
\pgfpathlineto{\pgfqpoint{1.233792in}{0.589966in}}%
\pgfpathlineto{\pgfqpoint{1.329942in}{0.606361in}}%
\pgfpathlineto{\pgfqpoint{1.442117in}{0.623158in}}%
\pgfpathlineto{\pgfqpoint{1.586341in}{0.642242in}}%
\pgfpathlineto{\pgfqpoint{1.746591in}{0.661122in}}%
\pgfpathlineto{\pgfqpoint{1.938890in}{0.681472in}}%
\pgfpathlineto{\pgfqpoint{2.163240in}{0.702895in}}%
\pgfpathlineto{\pgfqpoint{2.419639in}{0.725109in}}%
\pgfpathlineto{\pgfqpoint{2.724113in}{0.749130in}}%
\pgfpathlineto{\pgfqpoint{3.060637in}{0.773412in}}%
\pgfpathlineto{\pgfqpoint{3.445236in}{0.798918in}}%
\pgfpathlineto{\pgfqpoint{3.877910in}{0.825377in}}%
\pgfpathlineto{\pgfqpoint{4.374684in}{0.853461in}}%
\pgfpathlineto{\pgfqpoint{4.919532in}{0.882021in}}%
\pgfpathlineto{\pgfqpoint{5.015682in}{0.886854in}}%
\pgfpathlineto{\pgfqpoint{5.015682in}{0.886854in}}%
\pgfusepath{stroke}%
\end{pgfscope}%
\begin{pgfscope}%
\pgfpathrectangle{\pgfqpoint{0.725000in}{0.363000in}}{\pgfqpoint{4.495000in}{2.541000in}}%
\pgfusepath{clip}%
\pgfsetrectcap%
\pgfsetroundjoin%
\pgfsetlinewidth{1.505625pt}%
\definecolor{currentstroke}{rgb}{1.000000,0.498039,0.054902}%
\pgfsetstrokecolor{currentstroke}%
\pgfsetdash{}{0pt}%
\pgfpathmoveto{\pgfqpoint{0.929318in}{0.767250in}}%
\pgfpathlineto{\pgfqpoint{5.015682in}{0.767250in}}%
\pgfpathlineto{\pgfqpoint{5.015682in}{0.767250in}}%
\pgfusepath{stroke}%
\end{pgfscope}%
\begin{pgfscope}%
\pgfpathrectangle{\pgfqpoint{0.725000in}{0.363000in}}{\pgfqpoint{4.495000in}{2.541000in}}%
\pgfusepath{clip}%
\pgfsetrectcap%
\pgfsetroundjoin%
\pgfsetlinewidth{1.505625pt}%
\definecolor{currentstroke}{rgb}{0.172549,0.627451,0.172549}%
\pgfsetstrokecolor{currentstroke}%
\pgfsetdash{}{0pt}%
\pgfpathmoveto{\pgfqpoint{0.929318in}{0.478500in}}%
\pgfpathlineto{\pgfqpoint{5.015682in}{1.056000in}}%
\pgfpathlineto{\pgfqpoint{5.015682in}{1.056000in}}%
\pgfusepath{stroke}%
\end{pgfscope}%
\begin{pgfscope}%
\pgfpathrectangle{\pgfqpoint{0.725000in}{0.363000in}}{\pgfqpoint{4.495000in}{2.541000in}}%
\pgfusepath{clip}%
\pgfsetrectcap%
\pgfsetroundjoin%
\pgfsetlinewidth{1.505625pt}%
\definecolor{currentstroke}{rgb}{0.839216,0.152941,0.156863}%
\pgfsetstrokecolor{currentstroke}%
\pgfsetdash{}{0pt}%
\pgfpathmoveto{\pgfqpoint{0.929318in}{0.478500in}}%
\pgfpathlineto{\pgfqpoint{1.057518in}{0.479637in}}%
\pgfpathlineto{\pgfqpoint{1.185717in}{0.483047in}}%
\pgfpathlineto{\pgfqpoint{1.313917in}{0.488731in}}%
\pgfpathlineto{\pgfqpoint{1.442117in}{0.496689in}}%
\pgfpathlineto{\pgfqpoint{1.570316in}{0.506920in}}%
\pgfpathlineto{\pgfqpoint{1.698516in}{0.519425in}}%
\pgfpathlineto{\pgfqpoint{1.826716in}{0.534203in}}%
\pgfpathlineto{\pgfqpoint{1.954915in}{0.551255in}}%
\pgfpathlineto{\pgfqpoint{2.083115in}{0.570580in}}%
\pgfpathlineto{\pgfqpoint{2.211315in}{0.592179in}}%
\pgfpathlineto{\pgfqpoint{2.339514in}{0.616052in}}%
\pgfpathlineto{\pgfqpoint{2.467714in}{0.642198in}}%
\pgfpathlineto{\pgfqpoint{2.595914in}{0.670618in}}%
\pgfpathlineto{\pgfqpoint{2.724113in}{0.701312in}}%
\pgfpathlineto{\pgfqpoint{2.852313in}{0.734279in}}%
\pgfpathlineto{\pgfqpoint{2.980512in}{0.769519in}}%
\pgfpathlineto{\pgfqpoint{3.108712in}{0.807033in}}%
\pgfpathlineto{\pgfqpoint{3.236912in}{0.846821in}}%
\pgfpathlineto{\pgfqpoint{3.365111in}{0.888882in}}%
\pgfpathlineto{\pgfqpoint{3.493311in}{0.933217in}}%
\pgfpathlineto{\pgfqpoint{3.621511in}{0.979826in}}%
\pgfpathlineto{\pgfqpoint{3.749710in}{1.028708in}}%
\pgfpathlineto{\pgfqpoint{3.877910in}{1.079864in}}%
\pgfpathlineto{\pgfqpoint{4.006110in}{1.133293in}}%
\pgfpathlineto{\pgfqpoint{4.134309in}{1.188996in}}%
\pgfpathlineto{\pgfqpoint{4.262509in}{1.246972in}}%
\pgfpathlineto{\pgfqpoint{4.390709in}{1.307222in}}%
\pgfpathlineto{\pgfqpoint{4.518908in}{1.369746in}}%
\pgfpathlineto{\pgfqpoint{4.647108in}{1.434543in}}%
\pgfpathlineto{\pgfqpoint{4.775307in}{1.501614in}}%
\pgfpathlineto{\pgfqpoint{4.903507in}{1.570959in}}%
\pgfpathlineto{\pgfqpoint{5.015682in}{1.633500in}}%
\pgfpathlineto{\pgfqpoint{5.015682in}{1.633500in}}%
\pgfusepath{stroke}%
\end{pgfscope}%
\begin{pgfscope}%
\pgfpathrectangle{\pgfqpoint{0.725000in}{0.363000in}}{\pgfqpoint{4.495000in}{2.541000in}}%
\pgfusepath{clip}%
\pgfsetrectcap%
\pgfsetroundjoin%
\pgfsetlinewidth{1.505625pt}%
\definecolor{currentstroke}{rgb}{0.580392,0.403922,0.741176}%
\pgfsetstrokecolor{currentstroke}%
\pgfsetdash{}{0pt}%
\pgfpathmoveto{\pgfqpoint{0.929318in}{0.478500in}}%
\pgfpathlineto{\pgfqpoint{1.233792in}{0.479456in}}%
\pgfpathlineto{\pgfqpoint{1.410067in}{0.482261in}}%
\pgfpathlineto{\pgfqpoint{1.554291in}{0.486764in}}%
\pgfpathlineto{\pgfqpoint{1.682491in}{0.492964in}}%
\pgfpathlineto{\pgfqpoint{1.794666in}{0.500437in}}%
\pgfpathlineto{\pgfqpoint{1.906840in}{0.510121in}}%
\pgfpathlineto{\pgfqpoint{2.002990in}{0.520400in}}%
\pgfpathlineto{\pgfqpoint{2.099140in}{0.532695in}}%
\pgfpathlineto{\pgfqpoint{2.195290in}{0.547187in}}%
\pgfpathlineto{\pgfqpoint{2.291439in}{0.564056in}}%
\pgfpathlineto{\pgfqpoint{2.387589in}{0.583482in}}%
\pgfpathlineto{\pgfqpoint{2.467714in}{0.601755in}}%
\pgfpathlineto{\pgfqpoint{2.547839in}{0.622034in}}%
\pgfpathlineto{\pgfqpoint{2.627963in}{0.644424in}}%
\pgfpathlineto{\pgfqpoint{2.708088in}{0.669029in}}%
\pgfpathlineto{\pgfqpoint{2.788213in}{0.695953in}}%
\pgfpathlineto{\pgfqpoint{2.868338in}{0.725301in}}%
\pgfpathlineto{\pgfqpoint{2.948463in}{0.757178in}}%
\pgfpathlineto{\pgfqpoint{3.028587in}{0.791688in}}%
\pgfpathlineto{\pgfqpoint{3.108712in}{0.828936in}}%
\pgfpathlineto{\pgfqpoint{3.188837in}{0.869025in}}%
\pgfpathlineto{\pgfqpoint{3.268962in}{0.912061in}}%
\pgfpathlineto{\pgfqpoint{3.349086in}{0.958148in}}%
\pgfpathlineto{\pgfqpoint{3.429211in}{1.007390in}}%
\pgfpathlineto{\pgfqpoint{3.493311in}{1.049126in}}%
\pgfpathlineto{\pgfqpoint{3.557411in}{1.093002in}}%
\pgfpathlineto{\pgfqpoint{3.621511in}{1.139071in}}%
\pgfpathlineto{\pgfqpoint{3.685611in}{1.187387in}}%
\pgfpathlineto{\pgfqpoint{3.749710in}{1.238003in}}%
\pgfpathlineto{\pgfqpoint{3.813810in}{1.290973in}}%
\pgfpathlineto{\pgfqpoint{3.877910in}{1.346350in}}%
\pgfpathlineto{\pgfqpoint{3.942010in}{1.404189in}}%
\pgfpathlineto{\pgfqpoint{4.006110in}{1.464541in}}%
\pgfpathlineto{\pgfqpoint{4.070209in}{1.527462in}}%
\pgfpathlineto{\pgfqpoint{4.134309in}{1.593003in}}%
\pgfpathlineto{\pgfqpoint{4.198409in}{1.661220in}}%
\pgfpathlineto{\pgfqpoint{4.262509in}{1.732165in}}%
\pgfpathlineto{\pgfqpoint{4.326609in}{1.805891in}}%
\pgfpathlineto{\pgfqpoint{4.390709in}{1.882453in}}%
\pgfpathlineto{\pgfqpoint{4.454808in}{1.961904in}}%
\pgfpathlineto{\pgfqpoint{4.518908in}{2.044297in}}%
\pgfpathlineto{\pgfqpoint{4.583008in}{2.129686in}}%
\pgfpathlineto{\pgfqpoint{4.647108in}{2.218124in}}%
\pgfpathlineto{\pgfqpoint{4.711208in}{2.309665in}}%
\pgfpathlineto{\pgfqpoint{4.775307in}{2.404362in}}%
\pgfpathlineto{\pgfqpoint{4.839407in}{2.502269in}}%
\pgfpathlineto{\pgfqpoint{4.919532in}{2.629248in}}%
\pgfpathlineto{\pgfqpoint{4.999657in}{2.761430in}}%
\pgfpathlineto{\pgfqpoint{5.015682in}{2.788500in}}%
\pgfpathlineto{\pgfqpoint{5.015682in}{2.788500in}}%
\pgfusepath{stroke}%
\end{pgfscope}%
\begin{pgfscope}%
\pgfsetrectcap%
\pgfsetmiterjoin%
\pgfsetlinewidth{0.803000pt}%
\definecolor{currentstroke}{rgb}{0.000000,0.000000,0.000000}%
\pgfsetstrokecolor{currentstroke}%
\pgfsetdash{}{0pt}%
\pgfpathmoveto{\pgfqpoint{0.725000in}{0.363000in}}%
\pgfpathlineto{\pgfqpoint{0.725000in}{2.904000in}}%
\pgfusepath{stroke}%
\end{pgfscope}%
\begin{pgfscope}%
\pgfsetrectcap%
\pgfsetmiterjoin%
\pgfsetlinewidth{0.803000pt}%
\definecolor{currentstroke}{rgb}{0.000000,0.000000,0.000000}%
\pgfsetstrokecolor{currentstroke}%
\pgfsetdash{}{0pt}%
\pgfpathmoveto{\pgfqpoint{5.220000in}{0.363000in}}%
\pgfpathlineto{\pgfqpoint{5.220000in}{2.904000in}}%
\pgfusepath{stroke}%
\end{pgfscope}%
\begin{pgfscope}%
\pgfsetrectcap%
\pgfsetmiterjoin%
\pgfsetlinewidth{0.803000pt}%
\definecolor{currentstroke}{rgb}{0.000000,0.000000,0.000000}%
\pgfsetstrokecolor{currentstroke}%
\pgfsetdash{}{0pt}%
\pgfpathmoveto{\pgfqpoint{0.725000in}{0.363000in}}%
\pgfpathlineto{\pgfqpoint{5.220000in}{0.363000in}}%
\pgfusepath{stroke}%
\end{pgfscope}%
\begin{pgfscope}%
\pgfsetrectcap%
\pgfsetmiterjoin%
\pgfsetlinewidth{0.803000pt}%
\definecolor{currentstroke}{rgb}{0.000000,0.000000,0.000000}%
\pgfsetstrokecolor{currentstroke}%
\pgfsetdash{}{0pt}%
\pgfpathmoveto{\pgfqpoint{0.725000in}{2.904000in}}%
\pgfpathlineto{\pgfqpoint{5.220000in}{2.904000in}}%
\pgfusepath{stroke}%
\end{pgfscope}%
\begin{pgfscope}%
\pgfsetbuttcap%
\pgfsetmiterjoin%
\definecolor{currentfill}{rgb}{1.000000,1.000000,1.000000}%
\pgfsetfillcolor{currentfill}%
\pgfsetfillopacity{0.800000}%
\pgfsetlinewidth{1.003750pt}%
\definecolor{currentstroke}{rgb}{0.800000,0.800000,0.800000}%
\pgfsetstrokecolor{currentstroke}%
\pgfsetstrokeopacity{0.800000}%
\pgfsetdash{}{0pt}%
\pgfpathmoveto{\pgfqpoint{0.822222in}{1.824834in}}%
\pgfpathlineto{\pgfqpoint{1.496702in}{1.824834in}}%
\pgfpathquadraticcurveto{\pgfqpoint{1.524480in}{1.824834in}}{\pgfqpoint{1.524480in}{1.852612in}}%
\pgfpathlineto{\pgfqpoint{1.524480in}{2.806778in}}%
\pgfpathquadraticcurveto{\pgfqpoint{1.524480in}{2.834556in}}{\pgfqpoint{1.496702in}{2.834556in}}%
\pgfpathlineto{\pgfqpoint{0.822222in}{2.834556in}}%
\pgfpathquadraticcurveto{\pgfqpoint{0.794444in}{2.834556in}}{\pgfqpoint{0.794444in}{2.806778in}}%
\pgfpathlineto{\pgfqpoint{0.794444in}{1.852612in}}%
\pgfpathquadraticcurveto{\pgfqpoint{0.794444in}{1.824834in}}{\pgfqpoint{0.822222in}{1.824834in}}%
\pgfpathclose%
\pgfusepath{stroke,fill}%
\end{pgfscope}%
\begin{pgfscope}%
\pgfsetrectcap%
\pgfsetroundjoin%
\pgfsetlinewidth{1.505625pt}%
\definecolor{currentstroke}{rgb}{0.121569,0.466667,0.705882}%
\pgfsetstrokecolor{currentstroke}%
\pgfsetdash{}{0pt}%
\pgfpathmoveto{\pgfqpoint{0.850000in}{2.730389in}}%
\pgfpathlineto{\pgfqpoint{1.127778in}{2.730389in}}%
\pgfusepath{stroke}%
\end{pgfscope}%
\begin{pgfscope}%
\definecolor{textcolor}{rgb}{0.000000,0.000000,0.000000}%
\pgfsetstrokecolor{textcolor}%
\pgfsetfillcolor{textcolor}%
\pgftext[x=1.238889in,y=2.681778in,left,base]{\color{textcolor}\sffamily\fontsize{10.000000}{12.000000}\selectfont \(\displaystyle x^{0.5}\)}%
\end{pgfscope}%
\begin{pgfscope}%
\pgfsetrectcap%
\pgfsetroundjoin%
\pgfsetlinewidth{1.505625pt}%
\definecolor{currentstroke}{rgb}{1.000000,0.498039,0.054902}%
\pgfsetstrokecolor{currentstroke}%
\pgfsetdash{}{0pt}%
\pgfpathmoveto{\pgfqpoint{0.850000in}{2.536778in}}%
\pgfpathlineto{\pgfqpoint{1.127778in}{2.536778in}}%
\pgfusepath{stroke}%
\end{pgfscope}%
\begin{pgfscope}%
\definecolor{textcolor}{rgb}{0.000000,0.000000,0.000000}%
\pgfsetstrokecolor{textcolor}%
\pgfsetfillcolor{textcolor}%
\pgftext[x=1.238889in,y=2.488167in,left,base]{\color{textcolor}\sffamily\fontsize{10.000000}{12.000000}\selectfont \(\displaystyle x^{0}\)}%
\end{pgfscope}%
\begin{pgfscope}%
\pgfsetrectcap%
\pgfsetroundjoin%
\pgfsetlinewidth{1.505625pt}%
\definecolor{currentstroke}{rgb}{0.172549,0.627451,0.172549}%
\pgfsetstrokecolor{currentstroke}%
\pgfsetdash{}{0pt}%
\pgfpathmoveto{\pgfqpoint{0.850000in}{2.343167in}}%
\pgfpathlineto{\pgfqpoint{1.127778in}{2.343167in}}%
\pgfusepath{stroke}%
\end{pgfscope}%
\begin{pgfscope}%
\definecolor{textcolor}{rgb}{0.000000,0.000000,0.000000}%
\pgfsetstrokecolor{textcolor}%
\pgfsetfillcolor{textcolor}%
\pgftext[x=1.238889in,y=2.294556in,left,base]{\color{textcolor}\sffamily\fontsize{10.000000}{12.000000}\selectfont \(\displaystyle x^{1}\)}%
\end{pgfscope}%
\begin{pgfscope}%
\pgfsetrectcap%
\pgfsetroundjoin%
\pgfsetlinewidth{1.505625pt}%
\definecolor{currentstroke}{rgb}{0.839216,0.152941,0.156863}%
\pgfsetstrokecolor{currentstroke}%
\pgfsetdash{}{0pt}%
\pgfpathmoveto{\pgfqpoint{0.850000in}{2.149556in}}%
\pgfpathlineto{\pgfqpoint{1.127778in}{2.149556in}}%
\pgfusepath{stroke}%
\end{pgfscope}%
\begin{pgfscope}%
\definecolor{textcolor}{rgb}{0.000000,0.000000,0.000000}%
\pgfsetstrokecolor{textcolor}%
\pgfsetfillcolor{textcolor}%
\pgftext[x=1.238889in,y=2.100945in,left,base]{\color{textcolor}\sffamily\fontsize{10.000000}{12.000000}\selectfont \(\displaystyle x^{2}\)}%
\end{pgfscope}%
\begin{pgfscope}%
\pgfsetrectcap%
\pgfsetroundjoin%
\pgfsetlinewidth{1.505625pt}%
\definecolor{currentstroke}{rgb}{0.580392,0.403922,0.741176}%
\pgfsetstrokecolor{currentstroke}%
\pgfsetdash{}{0pt}%
\pgfpathmoveto{\pgfqpoint{0.850000in}{1.955945in}}%
\pgfpathlineto{\pgfqpoint{1.127778in}{1.955945in}}%
\pgfusepath{stroke}%
\end{pgfscope}%
\begin{pgfscope}%
\definecolor{textcolor}{rgb}{0.000000,0.000000,0.000000}%
\pgfsetstrokecolor{textcolor}%
\pgfsetfillcolor{textcolor}%
\pgftext[x=1.238889in,y=1.907334in,left,base]{\color{textcolor}\sffamily\fontsize{10.000000}{12.000000}\selectfont \(\displaystyle x^{3}\)}%
\end{pgfscope}%
\end{pgfpicture}%
\makeatother%
\endgroup%

    \caption{Die verschiedenen zu analysierenden polynomialen Signale\label{polynomials:signals}}
\end{figure}

In~\autoref{polynomials:haar} sind die Approximations- und die Detailkoeffizienten
der Transformation zu sehen. Wir sehen, dass die Approximationskoeffizienten
uns das grobe Signal und im Fall von $x^0 = 1$ sogar das exakte Signal liefern. Die
Detail Koeffizienten scheinen uns etwas zu liefern was proportional zur ersten
Ableitung des Signals ist.

\begin{figure}
    \centering
    %% Creator: Matplotlib, PGF backend
%%
%% To include the figure in your LaTeX document, write
%%   \input{<filename>.pgf}
%%
%% Make sure the required packages are loaded in your preamble
%%   \usepackage{pgf}
%%
%% Figures using additional raster images can only be included by \input if
%% they are in the same directory as the main LaTeX file. For loading figures
%% from other directories you can use the `import` package
%%   \usepackage{import}
%% and then include the figures with
%%   \import{<path to file>}{<filename>.pgf}
%%
%% Matplotlib used the following preamble
%%   \usepackage{fontspec}
%%
\begingroup%
\makeatletter%
\begin{pgfpicture}%
\pgfpathrectangle{\pgfpointorigin}{\pgfqpoint{5.800000in}{3.300000in}}%
\pgfusepath{use as bounding box, clip}%
\begin{pgfscope}%
\pgfsetbuttcap%
\pgfsetmiterjoin%
\definecolor{currentfill}{rgb}{1.000000,1.000000,1.000000}%
\pgfsetfillcolor{currentfill}%
\pgfsetlinewidth{0.000000pt}%
\definecolor{currentstroke}{rgb}{1.000000,1.000000,1.000000}%
\pgfsetstrokecolor{currentstroke}%
\pgfsetdash{}{0pt}%
\pgfpathmoveto{\pgfqpoint{0.000000in}{0.000000in}}%
\pgfpathlineto{\pgfqpoint{5.800000in}{0.000000in}}%
\pgfpathlineto{\pgfqpoint{5.800000in}{3.300000in}}%
\pgfpathlineto{\pgfqpoint{0.000000in}{3.300000in}}%
\pgfpathclose%
\pgfusepath{fill}%
\end{pgfscope}%
\begin{pgfscope}%
\pgfsetbuttcap%
\pgfsetmiterjoin%
\definecolor{currentfill}{rgb}{1.000000,1.000000,1.000000}%
\pgfsetfillcolor{currentfill}%
\pgfsetlinewidth{0.000000pt}%
\definecolor{currentstroke}{rgb}{0.000000,0.000000,0.000000}%
\pgfsetstrokecolor{currentstroke}%
\pgfsetstrokeopacity{0.000000}%
\pgfsetdash{}{0pt}%
\pgfpathmoveto{\pgfqpoint{0.670972in}{1.861111in}}%
\pgfpathlineto{\pgfqpoint{4.930000in}{1.861111in}}%
\pgfpathlineto{\pgfqpoint{4.930000in}{2.926667in}}%
\pgfpathlineto{\pgfqpoint{0.670972in}{2.926667in}}%
\pgfpathclose%
\pgfusepath{fill}%
\end{pgfscope}%
\begin{pgfscope}%
\pgfsetbuttcap%
\pgfsetroundjoin%
\definecolor{currentfill}{rgb}{0.000000,0.000000,0.000000}%
\pgfsetfillcolor{currentfill}%
\pgfsetlinewidth{0.803000pt}%
\definecolor{currentstroke}{rgb}{0.000000,0.000000,0.000000}%
\pgfsetstrokecolor{currentstroke}%
\pgfsetdash{}{0pt}%
\pgfsys@defobject{currentmarker}{\pgfqpoint{0.000000in}{-0.048611in}}{\pgfqpoint{0.000000in}{0.000000in}}{%
\pgfpathmoveto{\pgfqpoint{0.000000in}{0.000000in}}%
\pgfpathlineto{\pgfqpoint{0.000000in}{-0.048611in}}%
\pgfusepath{stroke,fill}%
}%
\begin{pgfscope}%
\pgfsys@transformshift{0.864564in}{1.861111in}%
\pgfsys@useobject{currentmarker}{}%
\end{pgfscope}%
\end{pgfscope}%
\begin{pgfscope}%
\pgfsetbuttcap%
\pgfsetroundjoin%
\definecolor{currentfill}{rgb}{0.000000,0.000000,0.000000}%
\pgfsetfillcolor{currentfill}%
\pgfsetlinewidth{0.803000pt}%
\definecolor{currentstroke}{rgb}{0.000000,0.000000,0.000000}%
\pgfsetstrokecolor{currentstroke}%
\pgfsetdash{}{0pt}%
\pgfsys@defobject{currentmarker}{\pgfqpoint{0.000000in}{-0.048611in}}{\pgfqpoint{0.000000in}{0.000000in}}{%
\pgfpathmoveto{\pgfqpoint{0.000000in}{0.000000in}}%
\pgfpathlineto{\pgfqpoint{0.000000in}{-0.048611in}}%
\pgfusepath{stroke,fill}%
}%
\begin{pgfscope}%
\pgfsys@transformshift{1.474304in}{1.861111in}%
\pgfsys@useobject{currentmarker}{}%
\end{pgfscope}%
\end{pgfscope}%
\begin{pgfscope}%
\pgfsetbuttcap%
\pgfsetroundjoin%
\definecolor{currentfill}{rgb}{0.000000,0.000000,0.000000}%
\pgfsetfillcolor{currentfill}%
\pgfsetlinewidth{0.803000pt}%
\definecolor{currentstroke}{rgb}{0.000000,0.000000,0.000000}%
\pgfsetstrokecolor{currentstroke}%
\pgfsetdash{}{0pt}%
\pgfsys@defobject{currentmarker}{\pgfqpoint{0.000000in}{-0.048611in}}{\pgfqpoint{0.000000in}{0.000000in}}{%
\pgfpathmoveto{\pgfqpoint{0.000000in}{0.000000in}}%
\pgfpathlineto{\pgfqpoint{0.000000in}{-0.048611in}}%
\pgfusepath{stroke,fill}%
}%
\begin{pgfscope}%
\pgfsys@transformshift{2.084043in}{1.861111in}%
\pgfsys@useobject{currentmarker}{}%
\end{pgfscope}%
\end{pgfscope}%
\begin{pgfscope}%
\pgfsetbuttcap%
\pgfsetroundjoin%
\definecolor{currentfill}{rgb}{0.000000,0.000000,0.000000}%
\pgfsetfillcolor{currentfill}%
\pgfsetlinewidth{0.803000pt}%
\definecolor{currentstroke}{rgb}{0.000000,0.000000,0.000000}%
\pgfsetstrokecolor{currentstroke}%
\pgfsetdash{}{0pt}%
\pgfsys@defobject{currentmarker}{\pgfqpoint{0.000000in}{-0.048611in}}{\pgfqpoint{0.000000in}{0.000000in}}{%
\pgfpathmoveto{\pgfqpoint{0.000000in}{0.000000in}}%
\pgfpathlineto{\pgfqpoint{0.000000in}{-0.048611in}}%
\pgfusepath{stroke,fill}%
}%
\begin{pgfscope}%
\pgfsys@transformshift{2.693782in}{1.861111in}%
\pgfsys@useobject{currentmarker}{}%
\end{pgfscope}%
\end{pgfscope}%
\begin{pgfscope}%
\pgfsetbuttcap%
\pgfsetroundjoin%
\definecolor{currentfill}{rgb}{0.000000,0.000000,0.000000}%
\pgfsetfillcolor{currentfill}%
\pgfsetlinewidth{0.803000pt}%
\definecolor{currentstroke}{rgb}{0.000000,0.000000,0.000000}%
\pgfsetstrokecolor{currentstroke}%
\pgfsetdash{}{0pt}%
\pgfsys@defobject{currentmarker}{\pgfqpoint{0.000000in}{-0.048611in}}{\pgfqpoint{0.000000in}{0.000000in}}{%
\pgfpathmoveto{\pgfqpoint{0.000000in}{0.000000in}}%
\pgfpathlineto{\pgfqpoint{0.000000in}{-0.048611in}}%
\pgfusepath{stroke,fill}%
}%
\begin{pgfscope}%
\pgfsys@transformshift{3.303521in}{1.861111in}%
\pgfsys@useobject{currentmarker}{}%
\end{pgfscope}%
\end{pgfscope}%
\begin{pgfscope}%
\pgfsetbuttcap%
\pgfsetroundjoin%
\definecolor{currentfill}{rgb}{0.000000,0.000000,0.000000}%
\pgfsetfillcolor{currentfill}%
\pgfsetlinewidth{0.803000pt}%
\definecolor{currentstroke}{rgb}{0.000000,0.000000,0.000000}%
\pgfsetstrokecolor{currentstroke}%
\pgfsetdash{}{0pt}%
\pgfsys@defobject{currentmarker}{\pgfqpoint{0.000000in}{-0.048611in}}{\pgfqpoint{0.000000in}{0.000000in}}{%
\pgfpathmoveto{\pgfqpoint{0.000000in}{0.000000in}}%
\pgfpathlineto{\pgfqpoint{0.000000in}{-0.048611in}}%
\pgfusepath{stroke,fill}%
}%
\begin{pgfscope}%
\pgfsys@transformshift{3.913260in}{1.861111in}%
\pgfsys@useobject{currentmarker}{}%
\end{pgfscope}%
\end{pgfscope}%
\begin{pgfscope}%
\pgfsetbuttcap%
\pgfsetroundjoin%
\definecolor{currentfill}{rgb}{0.000000,0.000000,0.000000}%
\pgfsetfillcolor{currentfill}%
\pgfsetlinewidth{0.803000pt}%
\definecolor{currentstroke}{rgb}{0.000000,0.000000,0.000000}%
\pgfsetstrokecolor{currentstroke}%
\pgfsetdash{}{0pt}%
\pgfsys@defobject{currentmarker}{\pgfqpoint{0.000000in}{-0.048611in}}{\pgfqpoint{0.000000in}{0.000000in}}{%
\pgfpathmoveto{\pgfqpoint{0.000000in}{0.000000in}}%
\pgfpathlineto{\pgfqpoint{0.000000in}{-0.048611in}}%
\pgfusepath{stroke,fill}%
}%
\begin{pgfscope}%
\pgfsys@transformshift{4.522999in}{1.861111in}%
\pgfsys@useobject{currentmarker}{}%
\end{pgfscope}%
\end{pgfscope}%
\begin{pgfscope}%
\pgfsetbuttcap%
\pgfsetroundjoin%
\definecolor{currentfill}{rgb}{0.000000,0.000000,0.000000}%
\pgfsetfillcolor{currentfill}%
\pgfsetlinewidth{0.803000pt}%
\definecolor{currentstroke}{rgb}{0.000000,0.000000,0.000000}%
\pgfsetstrokecolor{currentstroke}%
\pgfsetdash{}{0pt}%
\pgfsys@defobject{currentmarker}{\pgfqpoint{-0.048611in}{0.000000in}}{\pgfqpoint{0.000000in}{0.000000in}}{%
\pgfpathmoveto{\pgfqpoint{0.000000in}{0.000000in}}%
\pgfpathlineto{\pgfqpoint{-0.048611in}{0.000000in}}%
\pgfusepath{stroke,fill}%
}%
\begin{pgfscope}%
\pgfsys@transformshift{0.670972in}{1.909545in}%
\pgfsys@useobject{currentmarker}{}%
\end{pgfscope}%
\end{pgfscope}%
\begin{pgfscope}%
\definecolor{textcolor}{rgb}{0.000000,0.000000,0.000000}%
\pgfsetstrokecolor{textcolor}%
\pgfsetfillcolor{textcolor}%
\pgftext[x=0.504306in,y=1.861351in,left,base]{\color{textcolor}\sffamily\fontsize{10.000000}{12.000000}\selectfont 0}%
\end{pgfscope}%
\begin{pgfscope}%
\pgfsetbuttcap%
\pgfsetroundjoin%
\definecolor{currentfill}{rgb}{0.000000,0.000000,0.000000}%
\pgfsetfillcolor{currentfill}%
\pgfsetlinewidth{0.803000pt}%
\definecolor{currentstroke}{rgb}{0.000000,0.000000,0.000000}%
\pgfsetstrokecolor{currentstroke}%
\pgfsetdash{}{0pt}%
\pgfsys@defobject{currentmarker}{\pgfqpoint{-0.048611in}{0.000000in}}{\pgfqpoint{0.000000in}{0.000000in}}{%
\pgfpathmoveto{\pgfqpoint{0.000000in}{0.000000in}}%
\pgfpathlineto{\pgfqpoint{-0.048611in}{0.000000in}}%
\pgfusepath{stroke,fill}%
}%
\begin{pgfscope}%
\pgfsys@transformshift{0.670972in}{2.340172in}%
\pgfsys@useobject{currentmarker}{}%
\end{pgfscope}%
\end{pgfscope}%
\begin{pgfscope}%
\definecolor{textcolor}{rgb}{0.000000,0.000000,0.000000}%
\pgfsetstrokecolor{textcolor}%
\pgfsetfillcolor{textcolor}%
\pgftext[x=0.504306in,y=2.291977in,left,base]{\color{textcolor}\sffamily\fontsize{10.000000}{12.000000}\selectfont 5}%
\end{pgfscope}%
\begin{pgfscope}%
\pgfsetbuttcap%
\pgfsetroundjoin%
\definecolor{currentfill}{rgb}{0.000000,0.000000,0.000000}%
\pgfsetfillcolor{currentfill}%
\pgfsetlinewidth{0.803000pt}%
\definecolor{currentstroke}{rgb}{0.000000,0.000000,0.000000}%
\pgfsetstrokecolor{currentstroke}%
\pgfsetdash{}{0pt}%
\pgfsys@defobject{currentmarker}{\pgfqpoint{-0.048611in}{0.000000in}}{\pgfqpoint{0.000000in}{0.000000in}}{%
\pgfpathmoveto{\pgfqpoint{0.000000in}{0.000000in}}%
\pgfpathlineto{\pgfqpoint{-0.048611in}{0.000000in}}%
\pgfusepath{stroke,fill}%
}%
\begin{pgfscope}%
\pgfsys@transformshift{0.670972in}{2.770798in}%
\pgfsys@useobject{currentmarker}{}%
\end{pgfscope}%
\end{pgfscope}%
\begin{pgfscope}%
\definecolor{textcolor}{rgb}{0.000000,0.000000,0.000000}%
\pgfsetstrokecolor{textcolor}%
\pgfsetfillcolor{textcolor}%
\pgftext[x=0.434861in,y=2.722604in,left,base]{\color{textcolor}\sffamily\fontsize{10.000000}{12.000000}\selectfont 10}%
\end{pgfscope}%
\begin{pgfscope}%
\pgfpathrectangle{\pgfqpoint{0.670972in}{1.861111in}}{\pgfqpoint{4.259028in}{1.065556in}}%
\pgfusepath{clip}%
\pgfsetrectcap%
\pgfsetroundjoin%
\pgfsetlinewidth{1.505625pt}%
\definecolor{currentstroke}{rgb}{0.121569,0.466667,0.705882}%
\pgfsetstrokecolor{currentstroke}%
\pgfsetdash{}{0pt}%
\pgfpathmoveto{\pgfqpoint{0.864564in}{1.914939in}}%
\pgfpathlineto{\pgfqpoint{0.895051in}{1.926514in}}%
\pgfpathlineto{\pgfqpoint{0.925538in}{1.932392in}}%
\pgfpathlineto{\pgfqpoint{0.986512in}{1.940980in}}%
\pgfpathlineto{\pgfqpoint{1.077973in}{1.950614in}}%
\pgfpathlineto{\pgfqpoint{1.199921in}{1.960708in}}%
\pgfpathlineto{\pgfqpoint{1.352356in}{1.971038in}}%
\pgfpathlineto{\pgfqpoint{1.565764in}{1.983100in}}%
\pgfpathlineto{\pgfqpoint{1.840147in}{1.996175in}}%
\pgfpathlineto{\pgfqpoint{2.175504in}{2.009868in}}%
\pgfpathlineto{\pgfqpoint{2.602321in}{2.024968in}}%
\pgfpathlineto{\pgfqpoint{3.120599in}{2.040993in}}%
\pgfpathlineto{\pgfqpoint{3.760825in}{2.058426in}}%
\pgfpathlineto{\pgfqpoint{4.522999in}{2.076827in}}%
\pgfpathlineto{\pgfqpoint{4.736408in}{2.081627in}}%
\pgfpathlineto{\pgfqpoint{4.736408in}{2.081627in}}%
\pgfusepath{stroke}%
\end{pgfscope}%
\begin{pgfscope}%
\pgfpathrectangle{\pgfqpoint{0.670972in}{1.861111in}}{\pgfqpoint{4.259028in}{1.065556in}}%
\pgfusepath{clip}%
\pgfsetrectcap%
\pgfsetroundjoin%
\pgfsetlinewidth{1.505625pt}%
\definecolor{currentstroke}{rgb}{1.000000,0.498039,0.054902}%
\pgfsetstrokecolor{currentstroke}%
\pgfsetdash{}{0pt}%
\pgfpathmoveto{\pgfqpoint{0.864564in}{2.031345in}}%
\pgfpathlineto{\pgfqpoint{4.736408in}{2.031345in}}%
\pgfpathlineto{\pgfqpoint{4.736408in}{2.031345in}}%
\pgfusepath{stroke}%
\end{pgfscope}%
\begin{pgfscope}%
\pgfpathrectangle{\pgfqpoint{0.670972in}{1.861111in}}{\pgfqpoint{4.259028in}{1.065556in}}%
\pgfusepath{clip}%
\pgfsetrectcap%
\pgfsetroundjoin%
\pgfsetlinewidth{1.505625pt}%
\definecolor{currentstroke}{rgb}{0.172549,0.627451,0.172549}%
\pgfsetstrokecolor{currentstroke}%
\pgfsetdash{}{0pt}%
\pgfpathmoveto{\pgfqpoint{0.864564in}{1.910023in}}%
\pgfpathlineto{\pgfqpoint{4.736408in}{2.152667in}}%
\pgfpathlineto{\pgfqpoint{4.736408in}{2.152667in}}%
\pgfusepath{stroke}%
\end{pgfscope}%
\begin{pgfscope}%
\pgfpathrectangle{\pgfqpoint{0.670972in}{1.861111in}}{\pgfqpoint{4.259028in}{1.065556in}}%
\pgfusepath{clip}%
\pgfsetrectcap%
\pgfsetroundjoin%
\pgfsetlinewidth{1.505625pt}%
\definecolor{currentstroke}{rgb}{0.839216,0.152941,0.156863}%
\pgfsetstrokecolor{currentstroke}%
\pgfsetdash{}{0pt}%
\pgfpathmoveto{\pgfqpoint{0.864564in}{1.909549in}}%
\pgfpathlineto{\pgfqpoint{1.047486in}{1.910718in}}%
\pgfpathlineto{\pgfqpoint{1.230408in}{1.914045in}}%
\pgfpathlineto{\pgfqpoint{1.413330in}{1.919529in}}%
\pgfpathlineto{\pgfqpoint{1.596251in}{1.927171in}}%
\pgfpathlineto{\pgfqpoint{1.779173in}{1.936972in}}%
\pgfpathlineto{\pgfqpoint{1.962095in}{1.948930in}}%
\pgfpathlineto{\pgfqpoint{2.145017in}{1.963045in}}%
\pgfpathlineto{\pgfqpoint{2.327938in}{1.979319in}}%
\pgfpathlineto{\pgfqpoint{2.510860in}{1.997751in}}%
\pgfpathlineto{\pgfqpoint{2.693782in}{2.018340in}}%
\pgfpathlineto{\pgfqpoint{2.876704in}{2.041087in}}%
\pgfpathlineto{\pgfqpoint{3.059625in}{2.065992in}}%
\pgfpathlineto{\pgfqpoint{3.242547in}{2.093055in}}%
\pgfpathlineto{\pgfqpoint{3.425469in}{2.122275in}}%
\pgfpathlineto{\pgfqpoint{3.608390in}{2.153654in}}%
\pgfpathlineto{\pgfqpoint{3.791312in}{2.187190in}}%
\pgfpathlineto{\pgfqpoint{3.974234in}{2.222884in}}%
\pgfpathlineto{\pgfqpoint{4.157156in}{2.260736in}}%
\pgfpathlineto{\pgfqpoint{4.340077in}{2.300746in}}%
\pgfpathlineto{\pgfqpoint{4.522999in}{2.342914in}}%
\pgfpathlineto{\pgfqpoint{4.705921in}{2.387239in}}%
\pgfpathlineto{\pgfqpoint{4.736408in}{2.394837in}}%
\pgfpathlineto{\pgfqpoint{4.736408in}{2.394837in}}%
\pgfusepath{stroke}%
\end{pgfscope}%
\begin{pgfscope}%
\pgfpathrectangle{\pgfqpoint{0.670972in}{1.861111in}}{\pgfqpoint{4.259028in}{1.065556in}}%
\pgfusepath{clip}%
\pgfsetrectcap%
\pgfsetroundjoin%
\pgfsetlinewidth{1.505625pt}%
\definecolor{currentstroke}{rgb}{0.580392,0.403922,0.741176}%
\pgfsetstrokecolor{currentstroke}%
\pgfsetdash{}{0pt}%
\pgfpathmoveto{\pgfqpoint{0.864564in}{1.909545in}}%
\pgfpathlineto{\pgfqpoint{1.260895in}{1.910640in}}%
\pgfpathlineto{\pgfqpoint{1.474304in}{1.913451in}}%
\pgfpathlineto{\pgfqpoint{1.657225in}{1.918051in}}%
\pgfpathlineto{\pgfqpoint{1.809660in}{1.923895in}}%
\pgfpathlineto{\pgfqpoint{1.962095in}{1.931942in}}%
\pgfpathlineto{\pgfqpoint{2.114530in}{1.942546in}}%
\pgfpathlineto{\pgfqpoint{2.236477in}{1.953107in}}%
\pgfpathlineto{\pgfqpoint{2.358425in}{1.965709in}}%
\pgfpathlineto{\pgfqpoint{2.480373in}{1.980535in}}%
\pgfpathlineto{\pgfqpoint{2.602321in}{1.997763in}}%
\pgfpathlineto{\pgfqpoint{2.724269in}{2.017576in}}%
\pgfpathlineto{\pgfqpoint{2.846217in}{2.040152in}}%
\pgfpathlineto{\pgfqpoint{2.968164in}{2.065674in}}%
\pgfpathlineto{\pgfqpoint{3.059625in}{2.086856in}}%
\pgfpathlineto{\pgfqpoint{3.151086in}{2.109872in}}%
\pgfpathlineto{\pgfqpoint{3.242547in}{2.134799in}}%
\pgfpathlineto{\pgfqpoint{3.334008in}{2.161712in}}%
\pgfpathlineto{\pgfqpoint{3.425469in}{2.190688in}}%
\pgfpathlineto{\pgfqpoint{3.516930in}{2.221802in}}%
\pgfpathlineto{\pgfqpoint{3.608390in}{2.255132in}}%
\pgfpathlineto{\pgfqpoint{3.699851in}{2.290752in}}%
\pgfpathlineto{\pgfqpoint{3.791312in}{2.328740in}}%
\pgfpathlineto{\pgfqpoint{3.882773in}{2.369172in}}%
\pgfpathlineto{\pgfqpoint{3.974234in}{2.412123in}}%
\pgfpathlineto{\pgfqpoint{4.065695in}{2.457669in}}%
\pgfpathlineto{\pgfqpoint{4.157156in}{2.505888in}}%
\pgfpathlineto{\pgfqpoint{4.248617in}{2.556854in}}%
\pgfpathlineto{\pgfqpoint{4.340077in}{2.610645in}}%
\pgfpathlineto{\pgfqpoint{4.431538in}{2.667337in}}%
\pgfpathlineto{\pgfqpoint{4.522999in}{2.727004in}}%
\pgfpathlineto{\pgfqpoint{4.614460in}{2.789725in}}%
\pgfpathlineto{\pgfqpoint{4.705921in}{2.855574in}}%
\pgfpathlineto{\pgfqpoint{4.736408in}{2.878232in}}%
\pgfpathlineto{\pgfqpoint{4.736408in}{2.878232in}}%
\pgfusepath{stroke}%
\end{pgfscope}%
\begin{pgfscope}%
\pgfsetrectcap%
\pgfsetmiterjoin%
\pgfsetlinewidth{0.803000pt}%
\definecolor{currentstroke}{rgb}{0.000000,0.000000,0.000000}%
\pgfsetstrokecolor{currentstroke}%
\pgfsetdash{}{0pt}%
\pgfpathmoveto{\pgfqpoint{0.670972in}{1.861111in}}%
\pgfpathlineto{\pgfqpoint{0.670972in}{2.926667in}}%
\pgfusepath{stroke}%
\end{pgfscope}%
\begin{pgfscope}%
\pgfsetrectcap%
\pgfsetmiterjoin%
\pgfsetlinewidth{0.803000pt}%
\definecolor{currentstroke}{rgb}{0.000000,0.000000,0.000000}%
\pgfsetstrokecolor{currentstroke}%
\pgfsetdash{}{0pt}%
\pgfpathmoveto{\pgfqpoint{4.930000in}{1.861111in}}%
\pgfpathlineto{\pgfqpoint{4.930000in}{2.926667in}}%
\pgfusepath{stroke}%
\end{pgfscope}%
\begin{pgfscope}%
\pgfsetrectcap%
\pgfsetmiterjoin%
\pgfsetlinewidth{0.803000pt}%
\definecolor{currentstroke}{rgb}{0.000000,0.000000,0.000000}%
\pgfsetstrokecolor{currentstroke}%
\pgfsetdash{}{0pt}%
\pgfpathmoveto{\pgfqpoint{0.670972in}{1.861111in}}%
\pgfpathlineto{\pgfqpoint{4.930000in}{1.861111in}}%
\pgfusepath{stroke}%
\end{pgfscope}%
\begin{pgfscope}%
\pgfsetrectcap%
\pgfsetmiterjoin%
\pgfsetlinewidth{0.803000pt}%
\definecolor{currentstroke}{rgb}{0.000000,0.000000,0.000000}%
\pgfsetstrokecolor{currentstroke}%
\pgfsetdash{}{0pt}%
\pgfpathmoveto{\pgfqpoint{0.670972in}{2.926667in}}%
\pgfpathlineto{\pgfqpoint{4.930000in}{2.926667in}}%
\pgfusepath{stroke}%
\end{pgfscope}%
\begin{pgfscope}%
\definecolor{textcolor}{rgb}{0.000000,0.000000,0.000000}%
\pgfsetstrokecolor{textcolor}%
\pgfsetfillcolor{textcolor}%
\pgftext[x=2.800486in,y=3.010000in,,base]{\color{textcolor}\sffamily\fontsize{12.000000}{14.400000}\selectfont Approximation Koeffizienten}%
\end{pgfscope}%
\begin{pgfscope}%
\pgfsetbuttcap%
\pgfsetmiterjoin%
\definecolor{currentfill}{rgb}{1.000000,1.000000,1.000000}%
\pgfsetfillcolor{currentfill}%
\pgfsetlinewidth{0.000000pt}%
\definecolor{currentstroke}{rgb}{0.000000,0.000000,0.000000}%
\pgfsetstrokecolor{currentstroke}%
\pgfsetstrokeopacity{0.000000}%
\pgfsetdash{}{0pt}%
\pgfpathmoveto{\pgfqpoint{0.670972in}{0.387222in}}%
\pgfpathlineto{\pgfqpoint{4.930000in}{0.387222in}}%
\pgfpathlineto{\pgfqpoint{4.930000in}{1.452778in}}%
\pgfpathlineto{\pgfqpoint{0.670972in}{1.452778in}}%
\pgfpathclose%
\pgfusepath{fill}%
\end{pgfscope}%
\begin{pgfscope}%
\pgfsetbuttcap%
\pgfsetroundjoin%
\definecolor{currentfill}{rgb}{0.000000,0.000000,0.000000}%
\pgfsetfillcolor{currentfill}%
\pgfsetlinewidth{0.803000pt}%
\definecolor{currentstroke}{rgb}{0.000000,0.000000,0.000000}%
\pgfsetstrokecolor{currentstroke}%
\pgfsetdash{}{0pt}%
\pgfsys@defobject{currentmarker}{\pgfqpoint{0.000000in}{-0.048611in}}{\pgfqpoint{0.000000in}{0.000000in}}{%
\pgfpathmoveto{\pgfqpoint{0.000000in}{0.000000in}}%
\pgfpathlineto{\pgfqpoint{0.000000in}{-0.048611in}}%
\pgfusepath{stroke,fill}%
}%
\begin{pgfscope}%
\pgfsys@transformshift{0.864564in}{0.387222in}%
\pgfsys@useobject{currentmarker}{}%
\end{pgfscope}%
\end{pgfscope}%
\begin{pgfscope}%
\definecolor{textcolor}{rgb}{0.000000,0.000000,0.000000}%
\pgfsetstrokecolor{textcolor}%
\pgfsetfillcolor{textcolor}%
\pgftext[x=0.864564in,y=0.290000in,,top]{\color{textcolor}\sffamily\fontsize{10.000000}{12.000000}\selectfont 0}%
\end{pgfscope}%
\begin{pgfscope}%
\pgfsetbuttcap%
\pgfsetroundjoin%
\definecolor{currentfill}{rgb}{0.000000,0.000000,0.000000}%
\pgfsetfillcolor{currentfill}%
\pgfsetlinewidth{0.803000pt}%
\definecolor{currentstroke}{rgb}{0.000000,0.000000,0.000000}%
\pgfsetstrokecolor{currentstroke}%
\pgfsetdash{}{0pt}%
\pgfsys@defobject{currentmarker}{\pgfqpoint{0.000000in}{-0.048611in}}{\pgfqpoint{0.000000in}{0.000000in}}{%
\pgfpathmoveto{\pgfqpoint{0.000000in}{0.000000in}}%
\pgfpathlineto{\pgfqpoint{0.000000in}{-0.048611in}}%
\pgfusepath{stroke,fill}%
}%
\begin{pgfscope}%
\pgfsys@transformshift{1.474304in}{0.387222in}%
\pgfsys@useobject{currentmarker}{}%
\end{pgfscope}%
\end{pgfscope}%
\begin{pgfscope}%
\definecolor{textcolor}{rgb}{0.000000,0.000000,0.000000}%
\pgfsetstrokecolor{textcolor}%
\pgfsetfillcolor{textcolor}%
\pgftext[x=1.474304in,y=0.290000in,,top]{\color{textcolor}\sffamily\fontsize{10.000000}{12.000000}\selectfont 20}%
\end{pgfscope}%
\begin{pgfscope}%
\pgfsetbuttcap%
\pgfsetroundjoin%
\definecolor{currentfill}{rgb}{0.000000,0.000000,0.000000}%
\pgfsetfillcolor{currentfill}%
\pgfsetlinewidth{0.803000pt}%
\definecolor{currentstroke}{rgb}{0.000000,0.000000,0.000000}%
\pgfsetstrokecolor{currentstroke}%
\pgfsetdash{}{0pt}%
\pgfsys@defobject{currentmarker}{\pgfqpoint{0.000000in}{-0.048611in}}{\pgfqpoint{0.000000in}{0.000000in}}{%
\pgfpathmoveto{\pgfqpoint{0.000000in}{0.000000in}}%
\pgfpathlineto{\pgfqpoint{0.000000in}{-0.048611in}}%
\pgfusepath{stroke,fill}%
}%
\begin{pgfscope}%
\pgfsys@transformshift{2.084043in}{0.387222in}%
\pgfsys@useobject{currentmarker}{}%
\end{pgfscope}%
\end{pgfscope}%
\begin{pgfscope}%
\definecolor{textcolor}{rgb}{0.000000,0.000000,0.000000}%
\pgfsetstrokecolor{textcolor}%
\pgfsetfillcolor{textcolor}%
\pgftext[x=2.084043in,y=0.290000in,,top]{\color{textcolor}\sffamily\fontsize{10.000000}{12.000000}\selectfont 40}%
\end{pgfscope}%
\begin{pgfscope}%
\pgfsetbuttcap%
\pgfsetroundjoin%
\definecolor{currentfill}{rgb}{0.000000,0.000000,0.000000}%
\pgfsetfillcolor{currentfill}%
\pgfsetlinewidth{0.803000pt}%
\definecolor{currentstroke}{rgb}{0.000000,0.000000,0.000000}%
\pgfsetstrokecolor{currentstroke}%
\pgfsetdash{}{0pt}%
\pgfsys@defobject{currentmarker}{\pgfqpoint{0.000000in}{-0.048611in}}{\pgfqpoint{0.000000in}{0.000000in}}{%
\pgfpathmoveto{\pgfqpoint{0.000000in}{0.000000in}}%
\pgfpathlineto{\pgfqpoint{0.000000in}{-0.048611in}}%
\pgfusepath{stroke,fill}%
}%
\begin{pgfscope}%
\pgfsys@transformshift{2.693782in}{0.387222in}%
\pgfsys@useobject{currentmarker}{}%
\end{pgfscope}%
\end{pgfscope}%
\begin{pgfscope}%
\definecolor{textcolor}{rgb}{0.000000,0.000000,0.000000}%
\pgfsetstrokecolor{textcolor}%
\pgfsetfillcolor{textcolor}%
\pgftext[x=2.693782in,y=0.290000in,,top]{\color{textcolor}\sffamily\fontsize{10.000000}{12.000000}\selectfont 60}%
\end{pgfscope}%
\begin{pgfscope}%
\pgfsetbuttcap%
\pgfsetroundjoin%
\definecolor{currentfill}{rgb}{0.000000,0.000000,0.000000}%
\pgfsetfillcolor{currentfill}%
\pgfsetlinewidth{0.803000pt}%
\definecolor{currentstroke}{rgb}{0.000000,0.000000,0.000000}%
\pgfsetstrokecolor{currentstroke}%
\pgfsetdash{}{0pt}%
\pgfsys@defobject{currentmarker}{\pgfqpoint{0.000000in}{-0.048611in}}{\pgfqpoint{0.000000in}{0.000000in}}{%
\pgfpathmoveto{\pgfqpoint{0.000000in}{0.000000in}}%
\pgfpathlineto{\pgfqpoint{0.000000in}{-0.048611in}}%
\pgfusepath{stroke,fill}%
}%
\begin{pgfscope}%
\pgfsys@transformshift{3.303521in}{0.387222in}%
\pgfsys@useobject{currentmarker}{}%
\end{pgfscope}%
\end{pgfscope}%
\begin{pgfscope}%
\definecolor{textcolor}{rgb}{0.000000,0.000000,0.000000}%
\pgfsetstrokecolor{textcolor}%
\pgfsetfillcolor{textcolor}%
\pgftext[x=3.303521in,y=0.290000in,,top]{\color{textcolor}\sffamily\fontsize{10.000000}{12.000000}\selectfont 80}%
\end{pgfscope}%
\begin{pgfscope}%
\pgfsetbuttcap%
\pgfsetroundjoin%
\definecolor{currentfill}{rgb}{0.000000,0.000000,0.000000}%
\pgfsetfillcolor{currentfill}%
\pgfsetlinewidth{0.803000pt}%
\definecolor{currentstroke}{rgb}{0.000000,0.000000,0.000000}%
\pgfsetstrokecolor{currentstroke}%
\pgfsetdash{}{0pt}%
\pgfsys@defobject{currentmarker}{\pgfqpoint{0.000000in}{-0.048611in}}{\pgfqpoint{0.000000in}{0.000000in}}{%
\pgfpathmoveto{\pgfqpoint{0.000000in}{0.000000in}}%
\pgfpathlineto{\pgfqpoint{0.000000in}{-0.048611in}}%
\pgfusepath{stroke,fill}%
}%
\begin{pgfscope}%
\pgfsys@transformshift{3.913260in}{0.387222in}%
\pgfsys@useobject{currentmarker}{}%
\end{pgfscope}%
\end{pgfscope}%
\begin{pgfscope}%
\definecolor{textcolor}{rgb}{0.000000,0.000000,0.000000}%
\pgfsetstrokecolor{textcolor}%
\pgfsetfillcolor{textcolor}%
\pgftext[x=3.913260in,y=0.290000in,,top]{\color{textcolor}\sffamily\fontsize{10.000000}{12.000000}\selectfont 100}%
\end{pgfscope}%
\begin{pgfscope}%
\pgfsetbuttcap%
\pgfsetroundjoin%
\definecolor{currentfill}{rgb}{0.000000,0.000000,0.000000}%
\pgfsetfillcolor{currentfill}%
\pgfsetlinewidth{0.803000pt}%
\definecolor{currentstroke}{rgb}{0.000000,0.000000,0.000000}%
\pgfsetstrokecolor{currentstroke}%
\pgfsetdash{}{0pt}%
\pgfsys@defobject{currentmarker}{\pgfqpoint{0.000000in}{-0.048611in}}{\pgfqpoint{0.000000in}{0.000000in}}{%
\pgfpathmoveto{\pgfqpoint{0.000000in}{0.000000in}}%
\pgfpathlineto{\pgfqpoint{0.000000in}{-0.048611in}}%
\pgfusepath{stroke,fill}%
}%
\begin{pgfscope}%
\pgfsys@transformshift{4.522999in}{0.387222in}%
\pgfsys@useobject{currentmarker}{}%
\end{pgfscope}%
\end{pgfscope}%
\begin{pgfscope}%
\definecolor{textcolor}{rgb}{0.000000,0.000000,0.000000}%
\pgfsetstrokecolor{textcolor}%
\pgfsetfillcolor{textcolor}%
\pgftext[x=4.522999in,y=0.290000in,,top]{\color{textcolor}\sffamily\fontsize{10.000000}{12.000000}\selectfont 120}%
\end{pgfscope}%
\begin{pgfscope}%
\pgfsetbuttcap%
\pgfsetroundjoin%
\definecolor{currentfill}{rgb}{0.000000,0.000000,0.000000}%
\pgfsetfillcolor{currentfill}%
\pgfsetlinewidth{0.803000pt}%
\definecolor{currentstroke}{rgb}{0.000000,0.000000,0.000000}%
\pgfsetstrokecolor{currentstroke}%
\pgfsetdash{}{0pt}%
\pgfsys@defobject{currentmarker}{\pgfqpoint{-0.048611in}{0.000000in}}{\pgfqpoint{0.000000in}{0.000000in}}{%
\pgfpathmoveto{\pgfqpoint{0.000000in}{0.000000in}}%
\pgfpathlineto{\pgfqpoint{-0.048611in}{0.000000in}}%
\pgfusepath{stroke,fill}%
}%
\begin{pgfscope}%
\pgfsys@transformshift{0.670972in}{0.673707in}%
\pgfsys@useobject{currentmarker}{}%
\end{pgfscope}%
\end{pgfscope}%
\begin{pgfscope}%
\definecolor{textcolor}{rgb}{0.000000,0.000000,0.000000}%
\pgfsetstrokecolor{textcolor}%
\pgfsetfillcolor{textcolor}%
\pgftext[x=0.149306in,y=0.625512in,left,base]{\color{textcolor}\sffamily\fontsize{10.000000}{12.000000}\selectfont −0.050}%
\end{pgfscope}%
\begin{pgfscope}%
\pgfsetbuttcap%
\pgfsetroundjoin%
\definecolor{currentfill}{rgb}{0.000000,0.000000,0.000000}%
\pgfsetfillcolor{currentfill}%
\pgfsetlinewidth{0.803000pt}%
\definecolor{currentstroke}{rgb}{0.000000,0.000000,0.000000}%
\pgfsetstrokecolor{currentstroke}%
\pgfsetdash{}{0pt}%
\pgfsys@defobject{currentmarker}{\pgfqpoint{-0.048611in}{0.000000in}}{\pgfqpoint{0.000000in}{0.000000in}}{%
\pgfpathmoveto{\pgfqpoint{0.000000in}{0.000000in}}%
\pgfpathlineto{\pgfqpoint{-0.048611in}{0.000000in}}%
\pgfusepath{stroke,fill}%
}%
\begin{pgfscope}%
\pgfsys@transformshift{0.670972in}{1.039025in}%
\pgfsys@useobject{currentmarker}{}%
\end{pgfscope}%
\end{pgfscope}%
\begin{pgfscope}%
\definecolor{textcolor}{rgb}{0.000000,0.000000,0.000000}%
\pgfsetstrokecolor{textcolor}%
\pgfsetfillcolor{textcolor}%
\pgftext[x=0.149306in,y=0.990831in,left,base]{\color{textcolor}\sffamily\fontsize{10.000000}{12.000000}\selectfont −0.025}%
\end{pgfscope}%
\begin{pgfscope}%
\pgfsetbuttcap%
\pgfsetroundjoin%
\definecolor{currentfill}{rgb}{0.000000,0.000000,0.000000}%
\pgfsetfillcolor{currentfill}%
\pgfsetlinewidth{0.803000pt}%
\definecolor{currentstroke}{rgb}{0.000000,0.000000,0.000000}%
\pgfsetstrokecolor{currentstroke}%
\pgfsetdash{}{0pt}%
\pgfsys@defobject{currentmarker}{\pgfqpoint{-0.048611in}{0.000000in}}{\pgfqpoint{0.000000in}{0.000000in}}{%
\pgfpathmoveto{\pgfqpoint{0.000000in}{0.000000in}}%
\pgfpathlineto{\pgfqpoint{-0.048611in}{0.000000in}}%
\pgfusepath{stroke,fill}%
}%
\begin{pgfscope}%
\pgfsys@transformshift{0.670972in}{1.404343in}%
\pgfsys@useobject{currentmarker}{}%
\end{pgfscope}%
\end{pgfscope}%
\begin{pgfscope}%
\definecolor{textcolor}{rgb}{0.000000,0.000000,0.000000}%
\pgfsetstrokecolor{textcolor}%
\pgfsetfillcolor{textcolor}%
\pgftext[x=0.257361in,y=1.356149in,left,base]{\color{textcolor}\sffamily\fontsize{10.000000}{12.000000}\selectfont 0.000}%
\end{pgfscope}%
\begin{pgfscope}%
\pgfpathrectangle{\pgfqpoint{0.670972in}{0.387222in}}{\pgfqpoint{4.259028in}{1.065556in}}%
\pgfusepath{clip}%
\pgfsetrectcap%
\pgfsetroundjoin%
\pgfsetlinewidth{1.505625pt}%
\definecolor{currentstroke}{rgb}{0.121569,0.466667,0.705882}%
\pgfsetstrokecolor{currentstroke}%
\pgfsetdash{}{0pt}%
\pgfpathmoveto{\pgfqpoint{0.864564in}{0.489258in}}%
\pgfpathlineto{\pgfqpoint{0.895051in}{1.113495in}}%
\pgfpathlineto{\pgfqpoint{0.925538in}{1.188321in}}%
\pgfpathlineto{\pgfqpoint{0.956025in}{1.224747in}}%
\pgfpathlineto{\pgfqpoint{0.986512in}{1.247340in}}%
\pgfpathlineto{\pgfqpoint{1.016999in}{1.263103in}}%
\pgfpathlineto{\pgfqpoint{1.047486in}{1.274905in}}%
\pgfpathlineto{\pgfqpoint{1.077973in}{1.284169in}}%
\pgfpathlineto{\pgfqpoint{1.108460in}{1.291691in}}%
\pgfpathlineto{\pgfqpoint{1.169434in}{1.303282in}}%
\pgfpathlineto{\pgfqpoint{1.230408in}{1.311901in}}%
\pgfpathlineto{\pgfqpoint{1.321869in}{1.321493in}}%
\pgfpathlineto{\pgfqpoint{1.413330in}{1.328609in}}%
\pgfpathlineto{\pgfqpoint{1.535277in}{1.335754in}}%
\pgfpathlineto{\pgfqpoint{1.687712in}{1.342365in}}%
\pgfpathlineto{\pgfqpoint{1.901121in}{1.349061in}}%
\pgfpathlineto{\pgfqpoint{2.175504in}{1.355148in}}%
\pgfpathlineto{\pgfqpoint{2.541347in}{1.360817in}}%
\pgfpathlineto{\pgfqpoint{3.029138in}{1.366015in}}%
\pgfpathlineto{\pgfqpoint{3.699851in}{1.370840in}}%
\pgfpathlineto{\pgfqpoint{4.675434in}{1.375435in}}%
\pgfpathlineto{\pgfqpoint{4.736408in}{1.375663in}}%
\pgfpathlineto{\pgfqpoint{4.736408in}{1.375663in}}%
\pgfusepath{stroke}%
\end{pgfscope}%
\begin{pgfscope}%
\pgfpathrectangle{\pgfqpoint{0.670972in}{0.387222in}}{\pgfqpoint{4.259028in}{1.065556in}}%
\pgfusepath{clip}%
\pgfsetrectcap%
\pgfsetroundjoin%
\pgfsetlinewidth{1.505625pt}%
\definecolor{currentstroke}{rgb}{1.000000,0.498039,0.054902}%
\pgfsetstrokecolor{currentstroke}%
\pgfsetdash{}{0pt}%
\pgfpathmoveto{\pgfqpoint{0.864564in}{1.404343in}}%
\pgfpathlineto{\pgfqpoint{4.736408in}{1.404343in}}%
\pgfpathlineto{\pgfqpoint{4.736408in}{1.404343in}}%
\pgfusepath{stroke}%
\end{pgfscope}%
\begin{pgfscope}%
\pgfpathrectangle{\pgfqpoint{0.670972in}{0.387222in}}{\pgfqpoint{4.259028in}{1.065556in}}%
\pgfusepath{clip}%
\pgfsetrectcap%
\pgfsetroundjoin%
\pgfsetlinewidth{1.505625pt}%
\definecolor{currentstroke}{rgb}{0.172549,0.627451,0.172549}%
\pgfsetstrokecolor{currentstroke}%
\pgfsetdash{}{0pt}%
\pgfpathmoveto{\pgfqpoint{0.864564in}{1.323302in}}%
\pgfpathlineto{\pgfqpoint{4.736408in}{1.323302in}}%
\pgfpathlineto{\pgfqpoint{4.736408in}{1.323302in}}%
\pgfusepath{stroke}%
\end{pgfscope}%
\begin{pgfscope}%
\pgfpathrectangle{\pgfqpoint{0.670972in}{0.387222in}}{\pgfqpoint{4.259028in}{1.065556in}}%
\pgfusepath{clip}%
\pgfsetrectcap%
\pgfsetroundjoin%
\pgfsetlinewidth{1.505625pt}%
\definecolor{currentstroke}{rgb}{0.839216,0.152941,0.156863}%
\pgfsetstrokecolor{currentstroke}%
\pgfsetdash{}{0pt}%
\pgfpathmoveto{\pgfqpoint{0.864564in}{1.403708in}}%
\pgfpathlineto{\pgfqpoint{4.736408in}{1.080814in}}%
\pgfpathlineto{\pgfqpoint{4.736408in}{1.080814in}}%
\pgfusepath{stroke}%
\end{pgfscope}%
\begin{pgfscope}%
\pgfpathrectangle{\pgfqpoint{0.670972in}{0.387222in}}{\pgfqpoint{4.259028in}{1.065556in}}%
\pgfusepath{clip}%
\pgfsetrectcap%
\pgfsetroundjoin%
\pgfsetlinewidth{1.505625pt}%
\definecolor{currentstroke}{rgb}{0.580392,0.403922,0.741176}%
\pgfsetstrokecolor{currentstroke}%
\pgfsetdash{}{0pt}%
\pgfpathmoveto{\pgfqpoint{0.864564in}{1.404338in}}%
\pgfpathlineto{\pgfqpoint{0.986512in}{1.403262in}}%
\pgfpathlineto{\pgfqpoint{1.108460in}{1.400270in}}%
\pgfpathlineto{\pgfqpoint{1.230408in}{1.395365in}}%
\pgfpathlineto{\pgfqpoint{1.352356in}{1.388545in}}%
\pgfpathlineto{\pgfqpoint{1.474304in}{1.379811in}}%
\pgfpathlineto{\pgfqpoint{1.596251in}{1.369163in}}%
\pgfpathlineto{\pgfqpoint{1.718199in}{1.356600in}}%
\pgfpathlineto{\pgfqpoint{1.840147in}{1.342123in}}%
\pgfpathlineto{\pgfqpoint{1.962095in}{1.325731in}}%
\pgfpathlineto{\pgfqpoint{2.084043in}{1.307425in}}%
\pgfpathlineto{\pgfqpoint{2.205990in}{1.287205in}}%
\pgfpathlineto{\pgfqpoint{2.327938in}{1.265071in}}%
\pgfpathlineto{\pgfqpoint{2.449886in}{1.241022in}}%
\pgfpathlineto{\pgfqpoint{2.571834in}{1.215059in}}%
\pgfpathlineto{\pgfqpoint{2.693782in}{1.187181in}}%
\pgfpathlineto{\pgfqpoint{2.815730in}{1.157390in}}%
\pgfpathlineto{\pgfqpoint{2.937677in}{1.125684in}}%
\pgfpathlineto{\pgfqpoint{3.059625in}{1.092063in}}%
\pgfpathlineto{\pgfqpoint{3.181573in}{1.056528in}}%
\pgfpathlineto{\pgfqpoint{3.303521in}{1.019079in}}%
\pgfpathlineto{\pgfqpoint{3.425469in}{0.979716in}}%
\pgfpathlineto{\pgfqpoint{3.547417in}{0.938438in}}%
\pgfpathlineto{\pgfqpoint{3.669364in}{0.895246in}}%
\pgfpathlineto{\pgfqpoint{3.791312in}{0.850139in}}%
\pgfpathlineto{\pgfqpoint{3.913260in}{0.803119in}}%
\pgfpathlineto{\pgfqpoint{4.035208in}{0.754183in}}%
\pgfpathlineto{\pgfqpoint{4.187643in}{0.690323in}}%
\pgfpathlineto{\pgfqpoint{4.340077in}{0.623470in}}%
\pgfpathlineto{\pgfqpoint{4.492512in}{0.553627in}}%
\pgfpathlineto{\pgfqpoint{4.644947in}{0.480793in}}%
\pgfpathlineto{\pgfqpoint{4.736408in}{0.435657in}}%
\pgfpathlineto{\pgfqpoint{4.736408in}{0.435657in}}%
\pgfusepath{stroke}%
\end{pgfscope}%
\begin{pgfscope}%
\pgfsetrectcap%
\pgfsetmiterjoin%
\pgfsetlinewidth{0.803000pt}%
\definecolor{currentstroke}{rgb}{0.000000,0.000000,0.000000}%
\pgfsetstrokecolor{currentstroke}%
\pgfsetdash{}{0pt}%
\pgfpathmoveto{\pgfqpoint{0.670972in}{0.387222in}}%
\pgfpathlineto{\pgfqpoint{0.670972in}{1.452778in}}%
\pgfusepath{stroke}%
\end{pgfscope}%
\begin{pgfscope}%
\pgfsetrectcap%
\pgfsetmiterjoin%
\pgfsetlinewidth{0.803000pt}%
\definecolor{currentstroke}{rgb}{0.000000,0.000000,0.000000}%
\pgfsetstrokecolor{currentstroke}%
\pgfsetdash{}{0pt}%
\pgfpathmoveto{\pgfqpoint{4.930000in}{0.387222in}}%
\pgfpathlineto{\pgfqpoint{4.930000in}{1.452778in}}%
\pgfusepath{stroke}%
\end{pgfscope}%
\begin{pgfscope}%
\pgfsetrectcap%
\pgfsetmiterjoin%
\pgfsetlinewidth{0.803000pt}%
\definecolor{currentstroke}{rgb}{0.000000,0.000000,0.000000}%
\pgfsetstrokecolor{currentstroke}%
\pgfsetdash{}{0pt}%
\pgfpathmoveto{\pgfqpoint{0.670972in}{0.387222in}}%
\pgfpathlineto{\pgfqpoint{4.930000in}{0.387222in}}%
\pgfusepath{stroke}%
\end{pgfscope}%
\begin{pgfscope}%
\pgfsetrectcap%
\pgfsetmiterjoin%
\pgfsetlinewidth{0.803000pt}%
\definecolor{currentstroke}{rgb}{0.000000,0.000000,0.000000}%
\pgfsetstrokecolor{currentstroke}%
\pgfsetdash{}{0pt}%
\pgfpathmoveto{\pgfqpoint{0.670972in}{1.452778in}}%
\pgfpathlineto{\pgfqpoint{4.930000in}{1.452778in}}%
\pgfusepath{stroke}%
\end{pgfscope}%
\begin{pgfscope}%
\definecolor{textcolor}{rgb}{0.000000,0.000000,0.000000}%
\pgfsetstrokecolor{textcolor}%
\pgfsetfillcolor{textcolor}%
\pgftext[x=2.800486in,y=1.536111in,,base]{\color{textcolor}\sffamily\fontsize{12.000000}{14.400000}\selectfont Detail Koeffizienten}%
\end{pgfscope}%
\begin{pgfscope}%
\pgfsetbuttcap%
\pgfsetmiterjoin%
\definecolor{currentfill}{rgb}{1.000000,1.000000,1.000000}%
\pgfsetfillcolor{currentfill}%
\pgfsetfillopacity{0.800000}%
\pgfsetlinewidth{1.003750pt}%
\definecolor{currentstroke}{rgb}{0.800000,0.800000,0.800000}%
\pgfsetstrokecolor{currentstroke}%
\pgfsetstrokeopacity{0.800000}%
\pgfsetdash{}{0pt}%
\pgfpathmoveto{\pgfqpoint{5.083854in}{1.145139in}}%
\pgfpathlineto{\pgfqpoint{5.758333in}{1.145139in}}%
\pgfpathquadraticcurveto{\pgfqpoint{5.786111in}{1.145139in}}{\pgfqpoint{5.786111in}{1.172917in}}%
\pgfpathlineto{\pgfqpoint{5.786111in}{2.127083in}}%
\pgfpathquadraticcurveto{\pgfqpoint{5.786111in}{2.154861in}}{\pgfqpoint{5.758333in}{2.154861in}}%
\pgfpathlineto{\pgfqpoint{5.083854in}{2.154861in}}%
\pgfpathquadraticcurveto{\pgfqpoint{5.056076in}{2.154861in}}{\pgfqpoint{5.056076in}{2.127083in}}%
\pgfpathlineto{\pgfqpoint{5.056076in}{1.172917in}}%
\pgfpathquadraticcurveto{\pgfqpoint{5.056076in}{1.145139in}}{\pgfqpoint{5.083854in}{1.145139in}}%
\pgfpathclose%
\pgfusepath{stroke,fill}%
\end{pgfscope}%
\begin{pgfscope}%
\pgfsetrectcap%
\pgfsetroundjoin%
\pgfsetlinewidth{1.505625pt}%
\definecolor{currentstroke}{rgb}{0.121569,0.466667,0.705882}%
\pgfsetstrokecolor{currentstroke}%
\pgfsetdash{}{0pt}%
\pgfpathmoveto{\pgfqpoint{5.111632in}{2.050694in}}%
\pgfpathlineto{\pgfqpoint{5.389409in}{2.050694in}}%
\pgfusepath{stroke}%
\end{pgfscope}%
\begin{pgfscope}%
\definecolor{textcolor}{rgb}{0.000000,0.000000,0.000000}%
\pgfsetstrokecolor{textcolor}%
\pgfsetfillcolor{textcolor}%
\pgftext[x=5.500520in,y=2.002083in,left,base]{\color{textcolor}\sffamily\fontsize{10.000000}{12.000000}\selectfont \(\displaystyle x^{0.5}\)}%
\end{pgfscope}%
\begin{pgfscope}%
\pgfsetrectcap%
\pgfsetroundjoin%
\pgfsetlinewidth{1.505625pt}%
\definecolor{currentstroke}{rgb}{1.000000,0.498039,0.054902}%
\pgfsetstrokecolor{currentstroke}%
\pgfsetdash{}{0pt}%
\pgfpathmoveto{\pgfqpoint{5.111632in}{1.857083in}}%
\pgfpathlineto{\pgfqpoint{5.389409in}{1.857083in}}%
\pgfusepath{stroke}%
\end{pgfscope}%
\begin{pgfscope}%
\definecolor{textcolor}{rgb}{0.000000,0.000000,0.000000}%
\pgfsetstrokecolor{textcolor}%
\pgfsetfillcolor{textcolor}%
\pgftext[x=5.500520in,y=1.808472in,left,base]{\color{textcolor}\sffamily\fontsize{10.000000}{12.000000}\selectfont \(\displaystyle x^{0}\)}%
\end{pgfscope}%
\begin{pgfscope}%
\pgfsetrectcap%
\pgfsetroundjoin%
\pgfsetlinewidth{1.505625pt}%
\definecolor{currentstroke}{rgb}{0.172549,0.627451,0.172549}%
\pgfsetstrokecolor{currentstroke}%
\pgfsetdash{}{0pt}%
\pgfpathmoveto{\pgfqpoint{5.111632in}{1.663472in}}%
\pgfpathlineto{\pgfqpoint{5.389409in}{1.663472in}}%
\pgfusepath{stroke}%
\end{pgfscope}%
\begin{pgfscope}%
\definecolor{textcolor}{rgb}{0.000000,0.000000,0.000000}%
\pgfsetstrokecolor{textcolor}%
\pgfsetfillcolor{textcolor}%
\pgftext[x=5.500520in,y=1.614861in,left,base]{\color{textcolor}\sffamily\fontsize{10.000000}{12.000000}\selectfont \(\displaystyle x^{1}\)}%
\end{pgfscope}%
\begin{pgfscope}%
\pgfsetrectcap%
\pgfsetroundjoin%
\pgfsetlinewidth{1.505625pt}%
\definecolor{currentstroke}{rgb}{0.839216,0.152941,0.156863}%
\pgfsetstrokecolor{currentstroke}%
\pgfsetdash{}{0pt}%
\pgfpathmoveto{\pgfqpoint{5.111632in}{1.469861in}}%
\pgfpathlineto{\pgfqpoint{5.389409in}{1.469861in}}%
\pgfusepath{stroke}%
\end{pgfscope}%
\begin{pgfscope}%
\definecolor{textcolor}{rgb}{0.000000,0.000000,0.000000}%
\pgfsetstrokecolor{textcolor}%
\pgfsetfillcolor{textcolor}%
\pgftext[x=5.500520in,y=1.421250in,left,base]{\color{textcolor}\sffamily\fontsize{10.000000}{12.000000}\selectfont \(\displaystyle x^{2}\)}%
\end{pgfscope}%
\begin{pgfscope}%
\pgfsetrectcap%
\pgfsetroundjoin%
\pgfsetlinewidth{1.505625pt}%
\definecolor{currentstroke}{rgb}{0.580392,0.403922,0.741176}%
\pgfsetstrokecolor{currentstroke}%
\pgfsetdash{}{0pt}%
\pgfpathmoveto{\pgfqpoint{5.111632in}{1.276250in}}%
\pgfpathlineto{\pgfqpoint{5.389409in}{1.276250in}}%
\pgfusepath{stroke}%
\end{pgfscope}%
\begin{pgfscope}%
\definecolor{textcolor}{rgb}{0.000000,0.000000,0.000000}%
\pgfsetstrokecolor{textcolor}%
\pgfsetfillcolor{textcolor}%
\pgftext[x=5.500520in,y=1.227639in,left,base]{\color{textcolor}\sffamily\fontsize{10.000000}{12.000000}\selectfont \(\displaystyle x^{3}\)}%
\end{pgfscope}%
\end{pgfpicture}%
\makeatother%
\endgroup%

    \caption{Analyse mit db1 (Haar) Wavelet\label{polynomials:haar}}
\end{figure}

In \autoref{polynomials:diff} sind zum Vergleich die Ableitungen der
verschiedenen Signale geben.

\begin{figure}
    \centering
    %% Creator: Matplotlib, PGF backend
%%
%% To include the figure in your LaTeX document, write
%%   \input{<filename>.pgf}
%%
%% Make sure the required packages are loaded in your preamble
%%   \usepackage{pgf}
%%
%% Figures using additional raster images can only be included by \input if
%% they are in the same directory as the main LaTeX file. For loading figures
%% from other directories you can use the `import` package
%%   \usepackage{import}
%% and then include the figures with
%%   \import{<path to file>}{<filename>.pgf}
%%
%% Matplotlib used the following preamble
%%   \usepackage{fontspec}
%%
\begingroup%
\makeatletter%
\begin{pgfpicture}%
\pgfpathrectangle{\pgfpointorigin}{\pgfqpoint{2.900000in}{3.000000in}}%
\pgfusepath{use as bounding box, clip}%
\begin{pgfscope}%
\pgfsetbuttcap%
\pgfsetmiterjoin%
\definecolor{currentfill}{rgb}{1.000000,1.000000,1.000000}%
\pgfsetfillcolor{currentfill}%
\pgfsetlinewidth{0.000000pt}%
\definecolor{currentstroke}{rgb}{1.000000,1.000000,1.000000}%
\pgfsetstrokecolor{currentstroke}%
\pgfsetdash{}{0pt}%
\pgfpathmoveto{\pgfqpoint{0.000000in}{0.000000in}}%
\pgfpathlineto{\pgfqpoint{2.900000in}{0.000000in}}%
\pgfpathlineto{\pgfqpoint{2.900000in}{3.000000in}}%
\pgfpathlineto{\pgfqpoint{0.000000in}{3.000000in}}%
\pgfpathclose%
\pgfusepath{fill}%
\end{pgfscope}%
\begin{pgfscope}%
\pgfsetbuttcap%
\pgfsetmiterjoin%
\definecolor{currentfill}{rgb}{1.000000,1.000000,1.000000}%
\pgfsetfillcolor{currentfill}%
\pgfsetlinewidth{0.000000pt}%
\definecolor{currentstroke}{rgb}{0.000000,0.000000,0.000000}%
\pgfsetstrokecolor{currentstroke}%
\pgfsetstrokeopacity{0.000000}%
\pgfsetdash{}{0pt}%
\pgfpathmoveto{\pgfqpoint{0.362500in}{0.375000in}}%
\pgfpathlineto{\pgfqpoint{2.610000in}{0.375000in}}%
\pgfpathlineto{\pgfqpoint{2.610000in}{2.640000in}}%
\pgfpathlineto{\pgfqpoint{0.362500in}{2.640000in}}%
\pgfpathclose%
\pgfusepath{fill}%
\end{pgfscope}%
\begin{pgfscope}%
\pgfpathrectangle{\pgfqpoint{0.362500in}{0.375000in}}{\pgfqpoint{2.247500in}{2.265000in}}%
\pgfusepath{clip}%
\pgfsetrectcap%
\pgfsetroundjoin%
\pgfsetlinewidth{0.803000pt}%
\definecolor{currentstroke}{rgb}{0.690196,0.690196,0.690196}%
\pgfsetstrokecolor{currentstroke}%
\pgfsetdash{}{0pt}%
\pgfpathmoveto{\pgfqpoint{0.464659in}{0.375000in}}%
\pgfpathlineto{\pgfqpoint{0.464659in}{2.640000in}}%
\pgfusepath{stroke}%
\end{pgfscope}%
\begin{pgfscope}%
\pgfsetbuttcap%
\pgfsetroundjoin%
\definecolor{currentfill}{rgb}{0.000000,0.000000,0.000000}%
\pgfsetfillcolor{currentfill}%
\pgfsetlinewidth{0.803000pt}%
\definecolor{currentstroke}{rgb}{0.000000,0.000000,0.000000}%
\pgfsetstrokecolor{currentstroke}%
\pgfsetdash{}{0pt}%
\pgfsys@defobject{currentmarker}{\pgfqpoint{0.000000in}{-0.048611in}}{\pgfqpoint{0.000000in}{0.000000in}}{%
\pgfpathmoveto{\pgfqpoint{0.000000in}{0.000000in}}%
\pgfpathlineto{\pgfqpoint{0.000000in}{-0.048611in}}%
\pgfusepath{stroke,fill}%
}%
\begin{pgfscope}%
\pgfsys@transformshift{0.464659in}{0.375000in}%
\pgfsys@useobject{currentmarker}{}%
\end{pgfscope}%
\end{pgfscope}%
\begin{pgfscope}%
\definecolor{textcolor}{rgb}{0.000000,0.000000,0.000000}%
\pgfsetstrokecolor{textcolor}%
\pgfsetfillcolor{textcolor}%
\pgftext[x=0.464659in,y=0.277778in,,top]{\color{textcolor}\rmfamily\fontsize{10.000000}{12.000000}\selectfont 0}%
\end{pgfscope}%
\begin{pgfscope}%
\pgfpathrectangle{\pgfqpoint{0.362500in}{0.375000in}}{\pgfqpoint{2.247500in}{2.265000in}}%
\pgfusepath{clip}%
\pgfsetrectcap%
\pgfsetroundjoin%
\pgfsetlinewidth{0.803000pt}%
\definecolor{currentstroke}{rgb}{0.690196,0.690196,0.690196}%
\pgfsetstrokecolor{currentstroke}%
\pgfsetdash{}{0pt}%
\pgfpathmoveto{\pgfqpoint{1.269061in}{0.375000in}}%
\pgfpathlineto{\pgfqpoint{1.269061in}{2.640000in}}%
\pgfusepath{stroke}%
\end{pgfscope}%
\begin{pgfscope}%
\pgfsetbuttcap%
\pgfsetroundjoin%
\definecolor{currentfill}{rgb}{0.000000,0.000000,0.000000}%
\pgfsetfillcolor{currentfill}%
\pgfsetlinewidth{0.803000pt}%
\definecolor{currentstroke}{rgb}{0.000000,0.000000,0.000000}%
\pgfsetstrokecolor{currentstroke}%
\pgfsetdash{}{0pt}%
\pgfsys@defobject{currentmarker}{\pgfqpoint{0.000000in}{-0.048611in}}{\pgfqpoint{0.000000in}{0.000000in}}{%
\pgfpathmoveto{\pgfqpoint{0.000000in}{0.000000in}}%
\pgfpathlineto{\pgfqpoint{0.000000in}{-0.048611in}}%
\pgfusepath{stroke,fill}%
}%
\begin{pgfscope}%
\pgfsys@transformshift{1.269061in}{0.375000in}%
\pgfsys@useobject{currentmarker}{}%
\end{pgfscope}%
\end{pgfscope}%
\begin{pgfscope}%
\definecolor{textcolor}{rgb}{0.000000,0.000000,0.000000}%
\pgfsetstrokecolor{textcolor}%
\pgfsetfillcolor{textcolor}%
\pgftext[x=1.269061in,y=0.277778in,,top]{\color{textcolor}\rmfamily\fontsize{10.000000}{12.000000}\selectfont 100}%
\end{pgfscope}%
\begin{pgfscope}%
\pgfpathrectangle{\pgfqpoint{0.362500in}{0.375000in}}{\pgfqpoint{2.247500in}{2.265000in}}%
\pgfusepath{clip}%
\pgfsetrectcap%
\pgfsetroundjoin%
\pgfsetlinewidth{0.803000pt}%
\definecolor{currentstroke}{rgb}{0.690196,0.690196,0.690196}%
\pgfsetstrokecolor{currentstroke}%
\pgfsetdash{}{0pt}%
\pgfpathmoveto{\pgfqpoint{2.073464in}{0.375000in}}%
\pgfpathlineto{\pgfqpoint{2.073464in}{2.640000in}}%
\pgfusepath{stroke}%
\end{pgfscope}%
\begin{pgfscope}%
\pgfsetbuttcap%
\pgfsetroundjoin%
\definecolor{currentfill}{rgb}{0.000000,0.000000,0.000000}%
\pgfsetfillcolor{currentfill}%
\pgfsetlinewidth{0.803000pt}%
\definecolor{currentstroke}{rgb}{0.000000,0.000000,0.000000}%
\pgfsetstrokecolor{currentstroke}%
\pgfsetdash{}{0pt}%
\pgfsys@defobject{currentmarker}{\pgfqpoint{0.000000in}{-0.048611in}}{\pgfqpoint{0.000000in}{0.000000in}}{%
\pgfpathmoveto{\pgfqpoint{0.000000in}{0.000000in}}%
\pgfpathlineto{\pgfqpoint{0.000000in}{-0.048611in}}%
\pgfusepath{stroke,fill}%
}%
\begin{pgfscope}%
\pgfsys@transformshift{2.073464in}{0.375000in}%
\pgfsys@useobject{currentmarker}{}%
\end{pgfscope}%
\end{pgfscope}%
\begin{pgfscope}%
\definecolor{textcolor}{rgb}{0.000000,0.000000,0.000000}%
\pgfsetstrokecolor{textcolor}%
\pgfsetfillcolor{textcolor}%
\pgftext[x=2.073464in,y=0.277778in,,top]{\color{textcolor}\rmfamily\fontsize{10.000000}{12.000000}\selectfont 200}%
\end{pgfscope}%
\begin{pgfscope}%
\pgfpathrectangle{\pgfqpoint{0.362500in}{0.375000in}}{\pgfqpoint{2.247500in}{2.265000in}}%
\pgfusepath{clip}%
\pgfsetrectcap%
\pgfsetroundjoin%
\pgfsetlinewidth{0.803000pt}%
\definecolor{currentstroke}{rgb}{0.690196,0.690196,0.690196}%
\pgfsetstrokecolor{currentstroke}%
\pgfsetdash{}{0pt}%
\pgfpathmoveto{\pgfqpoint{0.362500in}{0.477955in}}%
\pgfpathlineto{\pgfqpoint{2.610000in}{0.477955in}}%
\pgfusepath{stroke}%
\end{pgfscope}%
\begin{pgfscope}%
\pgfsetbuttcap%
\pgfsetroundjoin%
\definecolor{currentfill}{rgb}{0.000000,0.000000,0.000000}%
\pgfsetfillcolor{currentfill}%
\pgfsetlinewidth{0.803000pt}%
\definecolor{currentstroke}{rgb}{0.000000,0.000000,0.000000}%
\pgfsetstrokecolor{currentstroke}%
\pgfsetdash{}{0pt}%
\pgfsys@defobject{currentmarker}{\pgfqpoint{-0.048611in}{0.000000in}}{\pgfqpoint{0.000000in}{0.000000in}}{%
\pgfpathmoveto{\pgfqpoint{0.000000in}{0.000000in}}%
\pgfpathlineto{\pgfqpoint{-0.048611in}{0.000000in}}%
\pgfusepath{stroke,fill}%
}%
\begin{pgfscope}%
\pgfsys@transformshift{0.362500in}{0.477955in}%
\pgfsys@useobject{currentmarker}{}%
\end{pgfscope}%
\end{pgfscope}%
\begin{pgfscope}%
\definecolor{textcolor}{rgb}{0.000000,0.000000,0.000000}%
\pgfsetstrokecolor{textcolor}%
\pgfsetfillcolor{textcolor}%
\pgftext[x=0.018333in,y=0.429760in,left,base]{\color{textcolor}\rmfamily\fontsize{10.000000}{12.000000}\selectfont 0.00}%
\end{pgfscope}%
\begin{pgfscope}%
\pgfpathrectangle{\pgfqpoint{0.362500in}{0.375000in}}{\pgfqpoint{2.247500in}{2.265000in}}%
\pgfusepath{clip}%
\pgfsetrectcap%
\pgfsetroundjoin%
\pgfsetlinewidth{0.803000pt}%
\definecolor{currentstroke}{rgb}{0.690196,0.690196,0.690196}%
\pgfsetstrokecolor{currentstroke}%
\pgfsetdash{}{0pt}%
\pgfpathmoveto{\pgfqpoint{0.362500in}{0.917232in}}%
\pgfpathlineto{\pgfqpoint{2.610000in}{0.917232in}}%
\pgfusepath{stroke}%
\end{pgfscope}%
\begin{pgfscope}%
\pgfsetbuttcap%
\pgfsetroundjoin%
\definecolor{currentfill}{rgb}{0.000000,0.000000,0.000000}%
\pgfsetfillcolor{currentfill}%
\pgfsetlinewidth{0.803000pt}%
\definecolor{currentstroke}{rgb}{0.000000,0.000000,0.000000}%
\pgfsetstrokecolor{currentstroke}%
\pgfsetdash{}{0pt}%
\pgfsys@defobject{currentmarker}{\pgfqpoint{-0.048611in}{0.000000in}}{\pgfqpoint{0.000000in}{0.000000in}}{%
\pgfpathmoveto{\pgfqpoint{0.000000in}{0.000000in}}%
\pgfpathlineto{\pgfqpoint{-0.048611in}{0.000000in}}%
\pgfusepath{stroke,fill}%
}%
\begin{pgfscope}%
\pgfsys@transformshift{0.362500in}{0.917232in}%
\pgfsys@useobject{currentmarker}{}%
\end{pgfscope}%
\end{pgfscope}%
\begin{pgfscope}%
\definecolor{textcolor}{rgb}{0.000000,0.000000,0.000000}%
\pgfsetstrokecolor{textcolor}%
\pgfsetfillcolor{textcolor}%
\pgftext[x=0.018333in,y=0.869037in,left,base]{\color{textcolor}\rmfamily\fontsize{10.000000}{12.000000}\selectfont 0.02}%
\end{pgfscope}%
\begin{pgfscope}%
\pgfpathrectangle{\pgfqpoint{0.362500in}{0.375000in}}{\pgfqpoint{2.247500in}{2.265000in}}%
\pgfusepath{clip}%
\pgfsetrectcap%
\pgfsetroundjoin%
\pgfsetlinewidth{0.803000pt}%
\definecolor{currentstroke}{rgb}{0.690196,0.690196,0.690196}%
\pgfsetstrokecolor{currentstroke}%
\pgfsetdash{}{0pt}%
\pgfpathmoveto{\pgfqpoint{0.362500in}{1.356509in}}%
\pgfpathlineto{\pgfqpoint{2.610000in}{1.356509in}}%
\pgfusepath{stroke}%
\end{pgfscope}%
\begin{pgfscope}%
\pgfsetbuttcap%
\pgfsetroundjoin%
\definecolor{currentfill}{rgb}{0.000000,0.000000,0.000000}%
\pgfsetfillcolor{currentfill}%
\pgfsetlinewidth{0.803000pt}%
\definecolor{currentstroke}{rgb}{0.000000,0.000000,0.000000}%
\pgfsetstrokecolor{currentstroke}%
\pgfsetdash{}{0pt}%
\pgfsys@defobject{currentmarker}{\pgfqpoint{-0.048611in}{0.000000in}}{\pgfqpoint{0.000000in}{0.000000in}}{%
\pgfpathmoveto{\pgfqpoint{0.000000in}{0.000000in}}%
\pgfpathlineto{\pgfqpoint{-0.048611in}{0.000000in}}%
\pgfusepath{stroke,fill}%
}%
\begin{pgfscope}%
\pgfsys@transformshift{0.362500in}{1.356509in}%
\pgfsys@useobject{currentmarker}{}%
\end{pgfscope}%
\end{pgfscope}%
\begin{pgfscope}%
\definecolor{textcolor}{rgb}{0.000000,0.000000,0.000000}%
\pgfsetstrokecolor{textcolor}%
\pgfsetfillcolor{textcolor}%
\pgftext[x=0.018333in,y=1.308315in,left,base]{\color{textcolor}\rmfamily\fontsize{10.000000}{12.000000}\selectfont 0.04}%
\end{pgfscope}%
\begin{pgfscope}%
\pgfpathrectangle{\pgfqpoint{0.362500in}{0.375000in}}{\pgfqpoint{2.247500in}{2.265000in}}%
\pgfusepath{clip}%
\pgfsetrectcap%
\pgfsetroundjoin%
\pgfsetlinewidth{0.803000pt}%
\definecolor{currentstroke}{rgb}{0.690196,0.690196,0.690196}%
\pgfsetstrokecolor{currentstroke}%
\pgfsetdash{}{0pt}%
\pgfpathmoveto{\pgfqpoint{0.362500in}{1.795786in}}%
\pgfpathlineto{\pgfqpoint{2.610000in}{1.795786in}}%
\pgfusepath{stroke}%
\end{pgfscope}%
\begin{pgfscope}%
\pgfsetbuttcap%
\pgfsetroundjoin%
\definecolor{currentfill}{rgb}{0.000000,0.000000,0.000000}%
\pgfsetfillcolor{currentfill}%
\pgfsetlinewidth{0.803000pt}%
\definecolor{currentstroke}{rgb}{0.000000,0.000000,0.000000}%
\pgfsetstrokecolor{currentstroke}%
\pgfsetdash{}{0pt}%
\pgfsys@defobject{currentmarker}{\pgfqpoint{-0.048611in}{0.000000in}}{\pgfqpoint{0.000000in}{0.000000in}}{%
\pgfpathmoveto{\pgfqpoint{0.000000in}{0.000000in}}%
\pgfpathlineto{\pgfqpoint{-0.048611in}{0.000000in}}%
\pgfusepath{stroke,fill}%
}%
\begin{pgfscope}%
\pgfsys@transformshift{0.362500in}{1.795786in}%
\pgfsys@useobject{currentmarker}{}%
\end{pgfscope}%
\end{pgfscope}%
\begin{pgfscope}%
\definecolor{textcolor}{rgb}{0.000000,0.000000,0.000000}%
\pgfsetstrokecolor{textcolor}%
\pgfsetfillcolor{textcolor}%
\pgftext[x=0.018333in,y=1.747592in,left,base]{\color{textcolor}\rmfamily\fontsize{10.000000}{12.000000}\selectfont 0.06}%
\end{pgfscope}%
\begin{pgfscope}%
\pgfpathrectangle{\pgfqpoint{0.362500in}{0.375000in}}{\pgfqpoint{2.247500in}{2.265000in}}%
\pgfusepath{clip}%
\pgfsetrectcap%
\pgfsetroundjoin%
\pgfsetlinewidth{0.803000pt}%
\definecolor{currentstroke}{rgb}{0.690196,0.690196,0.690196}%
\pgfsetstrokecolor{currentstroke}%
\pgfsetdash{}{0pt}%
\pgfpathmoveto{\pgfqpoint{0.362500in}{2.235063in}}%
\pgfpathlineto{\pgfqpoint{2.610000in}{2.235063in}}%
\pgfusepath{stroke}%
\end{pgfscope}%
\begin{pgfscope}%
\pgfsetbuttcap%
\pgfsetroundjoin%
\definecolor{currentfill}{rgb}{0.000000,0.000000,0.000000}%
\pgfsetfillcolor{currentfill}%
\pgfsetlinewidth{0.803000pt}%
\definecolor{currentstroke}{rgb}{0.000000,0.000000,0.000000}%
\pgfsetstrokecolor{currentstroke}%
\pgfsetdash{}{0pt}%
\pgfsys@defobject{currentmarker}{\pgfqpoint{-0.048611in}{0.000000in}}{\pgfqpoint{0.000000in}{0.000000in}}{%
\pgfpathmoveto{\pgfqpoint{0.000000in}{0.000000in}}%
\pgfpathlineto{\pgfqpoint{-0.048611in}{0.000000in}}%
\pgfusepath{stroke,fill}%
}%
\begin{pgfscope}%
\pgfsys@transformshift{0.362500in}{2.235063in}%
\pgfsys@useobject{currentmarker}{}%
\end{pgfscope}%
\end{pgfscope}%
\begin{pgfscope}%
\definecolor{textcolor}{rgb}{0.000000,0.000000,0.000000}%
\pgfsetstrokecolor{textcolor}%
\pgfsetfillcolor{textcolor}%
\pgftext[x=0.018333in,y=2.186869in,left,base]{\color{textcolor}\rmfamily\fontsize{10.000000}{12.000000}\selectfont 0.08}%
\end{pgfscope}%
\begin{pgfscope}%
\pgfpathrectangle{\pgfqpoint{0.362500in}{0.375000in}}{\pgfqpoint{2.247500in}{2.265000in}}%
\pgfusepath{clip}%
\pgfsetrectcap%
\pgfsetroundjoin%
\pgfsetlinewidth{1.505625pt}%
\definecolor{currentstroke}{rgb}{0.121569,0.466667,0.705882}%
\pgfsetstrokecolor{currentstroke}%
\pgfsetdash{}{0pt}%
\pgfpathmoveto{\pgfqpoint{0.464659in}{2.423107in}}%
\pgfpathlineto{\pgfqpoint{0.472703in}{1.283663in}}%
\pgfpathlineto{\pgfqpoint{0.480747in}{1.096196in}}%
\pgfpathlineto{\pgfqpoint{0.488791in}{0.999157in}}%
\pgfpathlineto{\pgfqpoint{0.496835in}{0.937143in}}%
\pgfpathlineto{\pgfqpoint{0.504879in}{0.893092in}}%
\pgfpathlineto{\pgfqpoint{0.512923in}{0.859713in}}%
\pgfpathlineto{\pgfqpoint{0.529011in}{0.811690in}}%
\pgfpathlineto{\pgfqpoint{0.545099in}{0.778183in}}%
\pgfpathlineto{\pgfqpoint{0.561187in}{0.753096in}}%
\pgfpathlineto{\pgfqpoint{0.577275in}{0.733404in}}%
\pgfpathlineto{\pgfqpoint{0.601407in}{0.710468in}}%
\pgfpathlineto{\pgfqpoint{0.625540in}{0.692777in}}%
\pgfpathlineto{\pgfqpoint{0.649672in}{0.678593in}}%
\pgfpathlineto{\pgfqpoint{0.681848in}{0.663425in}}%
\pgfpathlineto{\pgfqpoint{0.722068in}{0.648561in}}%
\pgfpathlineto{\pgfqpoint{0.770332in}{0.634703in}}%
\pgfpathlineto{\pgfqpoint{0.826640in}{0.622141in}}%
\pgfpathlineto{\pgfqpoint{0.890992in}{0.610924in}}%
\pgfpathlineto{\pgfqpoint{0.971433in}{0.600005in}}%
\pgfpathlineto{\pgfqpoint{1.076005in}{0.589152in}}%
\pgfpathlineto{\pgfqpoint{1.204709in}{0.579079in}}%
\pgfpathlineto{\pgfqpoint{1.365590in}{0.569650in}}%
\pgfpathlineto{\pgfqpoint{1.574734in}{0.560596in}}%
\pgfpathlineto{\pgfqpoint{1.848231in}{0.552005in}}%
\pgfpathlineto{\pgfqpoint{2.210212in}{0.543901in}}%
\pgfpathlineto{\pgfqpoint{2.507841in}{0.538919in}}%
\pgfpathlineto{\pgfqpoint{2.507841in}{0.538919in}}%
\pgfusepath{stroke}%
\end{pgfscope}%
\begin{pgfscope}%
\pgfpathrectangle{\pgfqpoint{0.362500in}{0.375000in}}{\pgfqpoint{2.247500in}{2.265000in}}%
\pgfusepath{clip}%
\pgfsetrectcap%
\pgfsetroundjoin%
\pgfsetlinewidth{1.505625pt}%
\definecolor{currentstroke}{rgb}{1.000000,0.498039,0.054902}%
\pgfsetstrokecolor{currentstroke}%
\pgfsetdash{}{0pt}%
\pgfpathmoveto{\pgfqpoint{0.464659in}{0.477955in}}%
\pgfpathlineto{\pgfqpoint{2.507841in}{0.477955in}}%
\pgfpathlineto{\pgfqpoint{2.507841in}{0.477955in}}%
\pgfusepath{stroke}%
\end{pgfscope}%
\begin{pgfscope}%
\pgfpathrectangle{\pgfqpoint{0.362500in}{0.375000in}}{\pgfqpoint{2.247500in}{2.265000in}}%
\pgfusepath{clip}%
\pgfsetrectcap%
\pgfsetroundjoin%
\pgfsetlinewidth{1.505625pt}%
\definecolor{currentstroke}{rgb}{0.172549,0.627451,0.172549}%
\pgfsetstrokecolor{currentstroke}%
\pgfsetdash{}{0pt}%
\pgfpathmoveto{\pgfqpoint{0.464659in}{0.650220in}}%
\pgfpathlineto{\pgfqpoint{2.507841in}{0.650220in}}%
\pgfpathlineto{\pgfqpoint{2.507841in}{0.650220in}}%
\pgfusepath{stroke}%
\end{pgfscope}%
\begin{pgfscope}%
\pgfpathrectangle{\pgfqpoint{0.362500in}{0.375000in}}{\pgfqpoint{2.247500in}{2.265000in}}%
\pgfusepath{clip}%
\pgfsetrectcap%
\pgfsetroundjoin%
\pgfsetlinewidth{1.505625pt}%
\definecolor{currentstroke}{rgb}{0.839216,0.152941,0.156863}%
\pgfsetstrokecolor{currentstroke}%
\pgfsetdash{}{0pt}%
\pgfpathmoveto{\pgfqpoint{0.464659in}{0.479306in}}%
\pgfpathlineto{\pgfqpoint{2.507841in}{1.165666in}}%
\pgfpathlineto{\pgfqpoint{2.507841in}{1.165666in}}%
\pgfusepath{stroke}%
\end{pgfscope}%
\begin{pgfscope}%
\pgfpathrectangle{\pgfqpoint{0.362500in}{0.375000in}}{\pgfqpoint{2.247500in}{2.265000in}}%
\pgfusepath{clip}%
\pgfsetrectcap%
\pgfsetroundjoin%
\pgfsetlinewidth{1.505625pt}%
\definecolor{currentstroke}{rgb}{0.580392,0.403922,0.741176}%
\pgfsetstrokecolor{currentstroke}%
\pgfsetdash{}{0pt}%
\pgfpathmoveto{\pgfqpoint{0.464659in}{0.477965in}}%
\pgfpathlineto{\pgfqpoint{0.520967in}{0.479745in}}%
\pgfpathlineto{\pgfqpoint{0.577275in}{0.484641in}}%
\pgfpathlineto{\pgfqpoint{0.633584in}{0.492652in}}%
\pgfpathlineto{\pgfqpoint{0.689892in}{0.503779in}}%
\pgfpathlineto{\pgfqpoint{0.746200in}{0.518021in}}%
\pgfpathlineto{\pgfqpoint{0.802508in}{0.535379in}}%
\pgfpathlineto{\pgfqpoint{0.858816in}{0.555852in}}%
\pgfpathlineto{\pgfqpoint{0.915124in}{0.579441in}}%
\pgfpathlineto{\pgfqpoint{0.971433in}{0.606145in}}%
\pgfpathlineto{\pgfqpoint{1.027741in}{0.635965in}}%
\pgfpathlineto{\pgfqpoint{1.084049in}{0.668900in}}%
\pgfpathlineto{\pgfqpoint{1.148401in}{0.710355in}}%
\pgfpathlineto{\pgfqpoint{1.212753in}{0.755879in}}%
\pgfpathlineto{\pgfqpoint{1.277105in}{0.805472in}}%
\pgfpathlineto{\pgfqpoint{1.341458in}{0.859135in}}%
\pgfpathlineto{\pgfqpoint{1.405810in}{0.916867in}}%
\pgfpathlineto{\pgfqpoint{1.470162in}{0.978668in}}%
\pgfpathlineto{\pgfqpoint{1.534514in}{1.044538in}}%
\pgfpathlineto{\pgfqpoint{1.598866in}{1.114478in}}%
\pgfpathlineto{\pgfqpoint{1.671263in}{1.198023in}}%
\pgfpathlineto{\pgfqpoint{1.743659in}{1.286719in}}%
\pgfpathlineto{\pgfqpoint{1.816055in}{1.380565in}}%
\pgfpathlineto{\pgfqpoint{1.888451in}{1.479561in}}%
\pgfpathlineto{\pgfqpoint{1.960847in}{1.583707in}}%
\pgfpathlineto{\pgfqpoint{2.033244in}{1.693004in}}%
\pgfpathlineto{\pgfqpoint{2.105640in}{1.807450in}}%
\pgfpathlineto{\pgfqpoint{2.186080in}{1.940653in}}%
\pgfpathlineto{\pgfqpoint{2.266520in}{2.080214in}}%
\pgfpathlineto{\pgfqpoint{2.346960in}{2.226133in}}%
\pgfpathlineto{\pgfqpoint{2.427401in}{2.378410in}}%
\pgfpathlineto{\pgfqpoint{2.507841in}{2.537045in}}%
\pgfpathlineto{\pgfqpoint{2.507841in}{2.537045in}}%
\pgfusepath{stroke}%
\end{pgfscope}%
\begin{pgfscope}%
\pgfsetrectcap%
\pgfsetmiterjoin%
\pgfsetlinewidth{0.803000pt}%
\definecolor{currentstroke}{rgb}{0.000000,0.000000,0.000000}%
\pgfsetstrokecolor{currentstroke}%
\pgfsetdash{}{0pt}%
\pgfpathmoveto{\pgfqpoint{0.362500in}{0.375000in}}%
\pgfpathlineto{\pgfqpoint{0.362500in}{2.640000in}}%
\pgfusepath{stroke}%
\end{pgfscope}%
\begin{pgfscope}%
\pgfsetrectcap%
\pgfsetmiterjoin%
\pgfsetlinewidth{0.803000pt}%
\definecolor{currentstroke}{rgb}{0.000000,0.000000,0.000000}%
\pgfsetstrokecolor{currentstroke}%
\pgfsetdash{}{0pt}%
\pgfpathmoveto{\pgfqpoint{2.610000in}{0.375000in}}%
\pgfpathlineto{\pgfqpoint{2.610000in}{2.640000in}}%
\pgfusepath{stroke}%
\end{pgfscope}%
\begin{pgfscope}%
\pgfsetrectcap%
\pgfsetmiterjoin%
\pgfsetlinewidth{0.803000pt}%
\definecolor{currentstroke}{rgb}{0.000000,0.000000,0.000000}%
\pgfsetstrokecolor{currentstroke}%
\pgfsetdash{}{0pt}%
\pgfpathmoveto{\pgfqpoint{0.362500in}{0.375000in}}%
\pgfpathlineto{\pgfqpoint{2.610000in}{0.375000in}}%
\pgfusepath{stroke}%
\end{pgfscope}%
\begin{pgfscope}%
\pgfsetrectcap%
\pgfsetmiterjoin%
\pgfsetlinewidth{0.803000pt}%
\definecolor{currentstroke}{rgb}{0.000000,0.000000,0.000000}%
\pgfsetstrokecolor{currentstroke}%
\pgfsetdash{}{0pt}%
\pgfpathmoveto{\pgfqpoint{0.362500in}{2.640000in}}%
\pgfpathlineto{\pgfqpoint{2.610000in}{2.640000in}}%
\pgfusepath{stroke}%
\end{pgfscope}%
\begin{pgfscope}%
\pgfsetbuttcap%
\pgfsetmiterjoin%
\definecolor{currentfill}{rgb}{1.000000,1.000000,1.000000}%
\pgfsetfillcolor{currentfill}%
\pgfsetfillopacity{0.800000}%
\pgfsetlinewidth{1.003750pt}%
\definecolor{currentstroke}{rgb}{0.800000,0.800000,0.800000}%
\pgfsetstrokecolor{currentstroke}%
\pgfsetstrokeopacity{0.800000}%
\pgfsetdash{}{0pt}%
\pgfpathmoveto{\pgfqpoint{1.149010in}{1.560834in}}%
\pgfpathlineto{\pgfqpoint{1.823490in}{1.560834in}}%
\pgfpathquadraticcurveto{\pgfqpoint{1.851268in}{1.560834in}}{\pgfqpoint{1.851268in}{1.588612in}}%
\pgfpathlineto{\pgfqpoint{1.851268in}{2.542778in}}%
\pgfpathquadraticcurveto{\pgfqpoint{1.851268in}{2.570556in}}{\pgfqpoint{1.823490in}{2.570556in}}%
\pgfpathlineto{\pgfqpoint{1.149010in}{2.570556in}}%
\pgfpathquadraticcurveto{\pgfqpoint{1.121232in}{2.570556in}}{\pgfqpoint{1.121232in}{2.542778in}}%
\pgfpathlineto{\pgfqpoint{1.121232in}{1.588612in}}%
\pgfpathquadraticcurveto{\pgfqpoint{1.121232in}{1.560834in}}{\pgfqpoint{1.149010in}{1.560834in}}%
\pgfpathclose%
\pgfusepath{stroke,fill}%
\end{pgfscope}%
\begin{pgfscope}%
\pgfsetrectcap%
\pgfsetroundjoin%
\pgfsetlinewidth{1.505625pt}%
\definecolor{currentstroke}{rgb}{0.121569,0.466667,0.705882}%
\pgfsetstrokecolor{currentstroke}%
\pgfsetdash{}{0pt}%
\pgfpathmoveto{\pgfqpoint{1.176788in}{2.466389in}}%
\pgfpathlineto{\pgfqpoint{1.454566in}{2.466389in}}%
\pgfusepath{stroke}%
\end{pgfscope}%
\begin{pgfscope}%
\definecolor{textcolor}{rgb}{0.000000,0.000000,0.000000}%
\pgfsetstrokecolor{textcolor}%
\pgfsetfillcolor{textcolor}%
\pgftext[x=1.565677in,y=2.417778in,left,base]{\color{textcolor}\rmfamily\fontsize{10.000000}{12.000000}\selectfont \(\displaystyle x^{0.5}\)}%
\end{pgfscope}%
\begin{pgfscope}%
\pgfsetrectcap%
\pgfsetroundjoin%
\pgfsetlinewidth{1.505625pt}%
\definecolor{currentstroke}{rgb}{1.000000,0.498039,0.054902}%
\pgfsetstrokecolor{currentstroke}%
\pgfsetdash{}{0pt}%
\pgfpathmoveto{\pgfqpoint{1.176788in}{2.272778in}}%
\pgfpathlineto{\pgfqpoint{1.454566in}{2.272778in}}%
\pgfusepath{stroke}%
\end{pgfscope}%
\begin{pgfscope}%
\definecolor{textcolor}{rgb}{0.000000,0.000000,0.000000}%
\pgfsetstrokecolor{textcolor}%
\pgfsetfillcolor{textcolor}%
\pgftext[x=1.565677in,y=2.224167in,left,base]{\color{textcolor}\rmfamily\fontsize{10.000000}{12.000000}\selectfont \(\displaystyle x^{0}\)}%
\end{pgfscope}%
\begin{pgfscope}%
\pgfsetrectcap%
\pgfsetroundjoin%
\pgfsetlinewidth{1.505625pt}%
\definecolor{currentstroke}{rgb}{0.172549,0.627451,0.172549}%
\pgfsetstrokecolor{currentstroke}%
\pgfsetdash{}{0pt}%
\pgfpathmoveto{\pgfqpoint{1.176788in}{2.079167in}}%
\pgfpathlineto{\pgfqpoint{1.454566in}{2.079167in}}%
\pgfusepath{stroke}%
\end{pgfscope}%
\begin{pgfscope}%
\definecolor{textcolor}{rgb}{0.000000,0.000000,0.000000}%
\pgfsetstrokecolor{textcolor}%
\pgfsetfillcolor{textcolor}%
\pgftext[x=1.565677in,y=2.030556in,left,base]{\color{textcolor}\rmfamily\fontsize{10.000000}{12.000000}\selectfont \(\displaystyle x^{1}\)}%
\end{pgfscope}%
\begin{pgfscope}%
\pgfsetrectcap%
\pgfsetroundjoin%
\pgfsetlinewidth{1.505625pt}%
\definecolor{currentstroke}{rgb}{0.839216,0.152941,0.156863}%
\pgfsetstrokecolor{currentstroke}%
\pgfsetdash{}{0pt}%
\pgfpathmoveto{\pgfqpoint{1.176788in}{1.885556in}}%
\pgfpathlineto{\pgfqpoint{1.454566in}{1.885556in}}%
\pgfusepath{stroke}%
\end{pgfscope}%
\begin{pgfscope}%
\definecolor{textcolor}{rgb}{0.000000,0.000000,0.000000}%
\pgfsetstrokecolor{textcolor}%
\pgfsetfillcolor{textcolor}%
\pgftext[x=1.565677in,y=1.836945in,left,base]{\color{textcolor}\rmfamily\fontsize{10.000000}{12.000000}\selectfont \(\displaystyle x^{2}\)}%
\end{pgfscope}%
\begin{pgfscope}%
\pgfsetrectcap%
\pgfsetroundjoin%
\pgfsetlinewidth{1.505625pt}%
\definecolor{currentstroke}{rgb}{0.580392,0.403922,0.741176}%
\pgfsetstrokecolor{currentstroke}%
\pgfsetdash{}{0pt}%
\pgfpathmoveto{\pgfqpoint{1.176788in}{1.691945in}}%
\pgfpathlineto{\pgfqpoint{1.454566in}{1.691945in}}%
\pgfusepath{stroke}%
\end{pgfscope}%
\begin{pgfscope}%
\definecolor{textcolor}{rgb}{0.000000,0.000000,0.000000}%
\pgfsetstrokecolor{textcolor}%
\pgfsetfillcolor{textcolor}%
\pgftext[x=1.565677in,y=1.643334in,left,base]{\color{textcolor}\rmfamily\fontsize{10.000000}{12.000000}\selectfont \(\displaystyle x^{3}\)}%
\end{pgfscope}%
\end{pgfpicture}%
\makeatother%
\endgroup%

    \caption{Ableitungen der Polynome\label{polynomials:diff}}
\end{figure}

Dieses Verhalten ist zu erwarten wenn man bedenkt, dass das Haar Wavelet
jeweils die Summe und die Differenz zweier benachbarter Sample analysiert.
Siehe dazu auch \autoref{haar:allwavelets:image} im Kapitel zum Haar Wavelet.

Nun stellt sich die Frage was passiert wenn wir Daubechies Wavelets mit mehr
verschwindenden Momenten zur Analyse einsetzen. In \autoref{polynomials:db2_3}
sind jeweils die Detail Koeffizienten der Analyse mit db1 und db2 Wavelet zu
sehen.

\begin{figure}
    \centering
    %% Creator: Matplotlib, PGF backend
%%
%% To include the figure in your LaTeX document, write
%%   \input{<filename>.pgf}
%%
%% Make sure the required packages are loaded in your preamble
%%   \usepackage{pgf}
%%
%% Figures using additional raster images can only be included by \input if
%% they are in the same directory as the main LaTeX file. For loading figures
%% from other directories you can use the `import` package
%%   \usepackage{import}
%% and then include the figures with
%%   \import{<path to file>}{<filename>.pgf}
%%
%% Matplotlib used the following preamble
%%   \usepackage{fontspec}
%%
\begingroup%
\makeatletter%
\begin{pgfpicture}%
\pgfpathrectangle{\pgfpointorigin}{\pgfqpoint{5.800000in}{3.000000in}}%
\pgfusepath{use as bounding box, clip}%
\begin{pgfscope}%
\pgfsetbuttcap%
\pgfsetmiterjoin%
\definecolor{currentfill}{rgb}{1.000000,1.000000,1.000000}%
\pgfsetfillcolor{currentfill}%
\pgfsetlinewidth{0.000000pt}%
\definecolor{currentstroke}{rgb}{1.000000,1.000000,1.000000}%
\pgfsetstrokecolor{currentstroke}%
\pgfsetdash{}{0pt}%
\pgfpathmoveto{\pgfqpoint{0.000000in}{0.000000in}}%
\pgfpathlineto{\pgfqpoint{5.800000in}{0.000000in}}%
\pgfpathlineto{\pgfqpoint{5.800000in}{3.000000in}}%
\pgfpathlineto{\pgfqpoint{0.000000in}{3.000000in}}%
\pgfpathclose%
\pgfusepath{fill}%
\end{pgfscope}%
\begin{pgfscope}%
\pgfsetbuttcap%
\pgfsetmiterjoin%
\definecolor{currentfill}{rgb}{1.000000,1.000000,1.000000}%
\pgfsetfillcolor{currentfill}%
\pgfsetlinewidth{0.000000pt}%
\definecolor{currentstroke}{rgb}{0.000000,0.000000,0.000000}%
\pgfsetstrokecolor{currentstroke}%
\pgfsetstrokeopacity{0.000000}%
\pgfsetdash{}{0pt}%
\pgfpathmoveto{\pgfqpoint{0.368000in}{0.315889in}}%
\pgfpathlineto{\pgfqpoint{2.745500in}{0.315889in}}%
\pgfpathlineto{\pgfqpoint{2.745500in}{2.704133in}}%
\pgfpathlineto{\pgfqpoint{0.368000in}{2.704133in}}%
\pgfpathclose%
\pgfusepath{fill}%
\end{pgfscope}%
\begin{pgfscope}%
\pgfsetbuttcap%
\pgfsetroundjoin%
\definecolor{currentfill}{rgb}{0.000000,0.000000,0.000000}%
\pgfsetfillcolor{currentfill}%
\pgfsetlinewidth{0.803000pt}%
\definecolor{currentstroke}{rgb}{0.000000,0.000000,0.000000}%
\pgfsetstrokecolor{currentstroke}%
\pgfsetdash{}{0pt}%
\pgfsys@defobject{currentmarker}{\pgfqpoint{0.000000in}{-0.048611in}}{\pgfqpoint{0.000000in}{0.000000in}}{%
\pgfpathmoveto{\pgfqpoint{0.000000in}{0.000000in}}%
\pgfpathlineto{\pgfqpoint{0.000000in}{-0.048611in}}%
\pgfusepath{stroke,fill}%
}%
\begin{pgfscope}%
\pgfsys@transformshift{0.476068in}{0.315889in}%
\pgfsys@useobject{currentmarker}{}%
\end{pgfscope}%
\end{pgfscope}%
\begin{pgfscope}%
\definecolor{textcolor}{rgb}{0.000000,0.000000,0.000000}%
\pgfsetstrokecolor{textcolor}%
\pgfsetfillcolor{textcolor}%
\pgftext[x=0.476068in,y=0.218667in,,top]{\color{textcolor}\rmfamily\fontsize{8.000000}{9.600000}\selectfont 0}%
\end{pgfscope}%
\begin{pgfscope}%
\pgfsetbuttcap%
\pgfsetroundjoin%
\definecolor{currentfill}{rgb}{0.000000,0.000000,0.000000}%
\pgfsetfillcolor{currentfill}%
\pgfsetlinewidth{0.803000pt}%
\definecolor{currentstroke}{rgb}{0.000000,0.000000,0.000000}%
\pgfsetstrokecolor{currentstroke}%
\pgfsetdash{}{0pt}%
\pgfsys@defobject{currentmarker}{\pgfqpoint{0.000000in}{-0.048611in}}{\pgfqpoint{0.000000in}{0.000000in}}{%
\pgfpathmoveto{\pgfqpoint{0.000000in}{0.000000in}}%
\pgfpathlineto{\pgfqpoint{0.000000in}{-0.048611in}}%
\pgfusepath{stroke,fill}%
}%
\begin{pgfscope}%
\pgfsys@transformshift{0.819142in}{0.315889in}%
\pgfsys@useobject{currentmarker}{}%
\end{pgfscope}%
\end{pgfscope}%
\begin{pgfscope}%
\definecolor{textcolor}{rgb}{0.000000,0.000000,0.000000}%
\pgfsetstrokecolor{textcolor}%
\pgfsetfillcolor{textcolor}%
\pgftext[x=0.819142in,y=0.218667in,,top]{\color{textcolor}\rmfamily\fontsize{8.000000}{9.600000}\selectfont 20}%
\end{pgfscope}%
\begin{pgfscope}%
\pgfsetbuttcap%
\pgfsetroundjoin%
\definecolor{currentfill}{rgb}{0.000000,0.000000,0.000000}%
\pgfsetfillcolor{currentfill}%
\pgfsetlinewidth{0.803000pt}%
\definecolor{currentstroke}{rgb}{0.000000,0.000000,0.000000}%
\pgfsetstrokecolor{currentstroke}%
\pgfsetdash{}{0pt}%
\pgfsys@defobject{currentmarker}{\pgfqpoint{0.000000in}{-0.048611in}}{\pgfqpoint{0.000000in}{0.000000in}}{%
\pgfpathmoveto{\pgfqpoint{0.000000in}{0.000000in}}%
\pgfpathlineto{\pgfqpoint{0.000000in}{-0.048611in}}%
\pgfusepath{stroke,fill}%
}%
\begin{pgfscope}%
\pgfsys@transformshift{1.162215in}{0.315889in}%
\pgfsys@useobject{currentmarker}{}%
\end{pgfscope}%
\end{pgfscope}%
\begin{pgfscope}%
\definecolor{textcolor}{rgb}{0.000000,0.000000,0.000000}%
\pgfsetstrokecolor{textcolor}%
\pgfsetfillcolor{textcolor}%
\pgftext[x=1.162215in,y=0.218667in,,top]{\color{textcolor}\rmfamily\fontsize{8.000000}{9.600000}\selectfont 40}%
\end{pgfscope}%
\begin{pgfscope}%
\pgfsetbuttcap%
\pgfsetroundjoin%
\definecolor{currentfill}{rgb}{0.000000,0.000000,0.000000}%
\pgfsetfillcolor{currentfill}%
\pgfsetlinewidth{0.803000pt}%
\definecolor{currentstroke}{rgb}{0.000000,0.000000,0.000000}%
\pgfsetstrokecolor{currentstroke}%
\pgfsetdash{}{0pt}%
\pgfsys@defobject{currentmarker}{\pgfqpoint{0.000000in}{-0.048611in}}{\pgfqpoint{0.000000in}{0.000000in}}{%
\pgfpathmoveto{\pgfqpoint{0.000000in}{0.000000in}}%
\pgfpathlineto{\pgfqpoint{0.000000in}{-0.048611in}}%
\pgfusepath{stroke,fill}%
}%
\begin{pgfscope}%
\pgfsys@transformshift{1.505289in}{0.315889in}%
\pgfsys@useobject{currentmarker}{}%
\end{pgfscope}%
\end{pgfscope}%
\begin{pgfscope}%
\definecolor{textcolor}{rgb}{0.000000,0.000000,0.000000}%
\pgfsetstrokecolor{textcolor}%
\pgfsetfillcolor{textcolor}%
\pgftext[x=1.505289in,y=0.218667in,,top]{\color{textcolor}\rmfamily\fontsize{8.000000}{9.600000}\selectfont 60}%
\end{pgfscope}%
\begin{pgfscope}%
\pgfsetbuttcap%
\pgfsetroundjoin%
\definecolor{currentfill}{rgb}{0.000000,0.000000,0.000000}%
\pgfsetfillcolor{currentfill}%
\pgfsetlinewidth{0.803000pt}%
\definecolor{currentstroke}{rgb}{0.000000,0.000000,0.000000}%
\pgfsetstrokecolor{currentstroke}%
\pgfsetdash{}{0pt}%
\pgfsys@defobject{currentmarker}{\pgfqpoint{0.000000in}{-0.048611in}}{\pgfqpoint{0.000000in}{0.000000in}}{%
\pgfpathmoveto{\pgfqpoint{0.000000in}{0.000000in}}%
\pgfpathlineto{\pgfqpoint{0.000000in}{-0.048611in}}%
\pgfusepath{stroke,fill}%
}%
\begin{pgfscope}%
\pgfsys@transformshift{1.848363in}{0.315889in}%
\pgfsys@useobject{currentmarker}{}%
\end{pgfscope}%
\end{pgfscope}%
\begin{pgfscope}%
\definecolor{textcolor}{rgb}{0.000000,0.000000,0.000000}%
\pgfsetstrokecolor{textcolor}%
\pgfsetfillcolor{textcolor}%
\pgftext[x=1.848363in,y=0.218667in,,top]{\color{textcolor}\rmfamily\fontsize{8.000000}{9.600000}\selectfont 80}%
\end{pgfscope}%
\begin{pgfscope}%
\pgfsetbuttcap%
\pgfsetroundjoin%
\definecolor{currentfill}{rgb}{0.000000,0.000000,0.000000}%
\pgfsetfillcolor{currentfill}%
\pgfsetlinewidth{0.803000pt}%
\definecolor{currentstroke}{rgb}{0.000000,0.000000,0.000000}%
\pgfsetstrokecolor{currentstroke}%
\pgfsetdash{}{0pt}%
\pgfsys@defobject{currentmarker}{\pgfqpoint{0.000000in}{-0.048611in}}{\pgfqpoint{0.000000in}{0.000000in}}{%
\pgfpathmoveto{\pgfqpoint{0.000000in}{0.000000in}}%
\pgfpathlineto{\pgfqpoint{0.000000in}{-0.048611in}}%
\pgfusepath{stroke,fill}%
}%
\begin{pgfscope}%
\pgfsys@transformshift{2.191436in}{0.315889in}%
\pgfsys@useobject{currentmarker}{}%
\end{pgfscope}%
\end{pgfscope}%
\begin{pgfscope}%
\definecolor{textcolor}{rgb}{0.000000,0.000000,0.000000}%
\pgfsetstrokecolor{textcolor}%
\pgfsetfillcolor{textcolor}%
\pgftext[x=2.191436in,y=0.218667in,,top]{\color{textcolor}\rmfamily\fontsize{8.000000}{9.600000}\selectfont 100}%
\end{pgfscope}%
\begin{pgfscope}%
\pgfsetbuttcap%
\pgfsetroundjoin%
\definecolor{currentfill}{rgb}{0.000000,0.000000,0.000000}%
\pgfsetfillcolor{currentfill}%
\pgfsetlinewidth{0.803000pt}%
\definecolor{currentstroke}{rgb}{0.000000,0.000000,0.000000}%
\pgfsetstrokecolor{currentstroke}%
\pgfsetdash{}{0pt}%
\pgfsys@defobject{currentmarker}{\pgfqpoint{0.000000in}{-0.048611in}}{\pgfqpoint{0.000000in}{0.000000in}}{%
\pgfpathmoveto{\pgfqpoint{0.000000in}{0.000000in}}%
\pgfpathlineto{\pgfqpoint{0.000000in}{-0.048611in}}%
\pgfusepath{stroke,fill}%
}%
\begin{pgfscope}%
\pgfsys@transformshift{2.534510in}{0.315889in}%
\pgfsys@useobject{currentmarker}{}%
\end{pgfscope}%
\end{pgfscope}%
\begin{pgfscope}%
\definecolor{textcolor}{rgb}{0.000000,0.000000,0.000000}%
\pgfsetstrokecolor{textcolor}%
\pgfsetfillcolor{textcolor}%
\pgftext[x=2.534510in,y=0.218667in,,top]{\color{textcolor}\rmfamily\fontsize{8.000000}{9.600000}\selectfont 120}%
\end{pgfscope}%
\begin{pgfscope}%
\pgfsetbuttcap%
\pgfsetroundjoin%
\definecolor{currentfill}{rgb}{0.000000,0.000000,0.000000}%
\pgfsetfillcolor{currentfill}%
\pgfsetlinewidth{0.803000pt}%
\definecolor{currentstroke}{rgb}{0.000000,0.000000,0.000000}%
\pgfsetstrokecolor{currentstroke}%
\pgfsetdash{}{0pt}%
\pgfsys@defobject{currentmarker}{\pgfqpoint{-0.048611in}{0.000000in}}{\pgfqpoint{0.000000in}{0.000000in}}{%
\pgfpathmoveto{\pgfqpoint{0.000000in}{0.000000in}}%
\pgfpathlineto{\pgfqpoint{-0.048611in}{0.000000in}}%
\pgfusepath{stroke,fill}%
}%
\begin{pgfscope}%
\pgfsys@transformshift{0.368000in}{0.665220in}%
\pgfsys@useobject{currentmarker}{}%
\end{pgfscope}%
\end{pgfscope}%
\begin{pgfscope}%
\definecolor{textcolor}{rgb}{0.000000,0.000000,0.000000}%
\pgfsetstrokecolor{textcolor}%
\pgfsetfillcolor{textcolor}%
\pgftext[x=0.120000in,y=0.626665in,left,base]{\color{textcolor}\rmfamily\fontsize{8.000000}{9.600000}\selectfont −4}%
\end{pgfscope}%
\begin{pgfscope}%
\pgfsetbuttcap%
\pgfsetroundjoin%
\definecolor{currentfill}{rgb}{0.000000,0.000000,0.000000}%
\pgfsetfillcolor{currentfill}%
\pgfsetlinewidth{0.803000pt}%
\definecolor{currentstroke}{rgb}{0.000000,0.000000,0.000000}%
\pgfsetstrokecolor{currentstroke}%
\pgfsetdash{}{0pt}%
\pgfsys@defobject{currentmarker}{\pgfqpoint{-0.048611in}{0.000000in}}{\pgfqpoint{0.000000in}{0.000000in}}{%
\pgfpathmoveto{\pgfqpoint{0.000000in}{0.000000in}}%
\pgfpathlineto{\pgfqpoint{-0.048611in}{0.000000in}}%
\pgfusepath{stroke,fill}%
}%
\begin{pgfscope}%
\pgfsys@transformshift{0.368000in}{1.147809in}%
\pgfsys@useobject{currentmarker}{}%
\end{pgfscope}%
\end{pgfscope}%
\begin{pgfscope}%
\definecolor{textcolor}{rgb}{0.000000,0.000000,0.000000}%
\pgfsetstrokecolor{textcolor}%
\pgfsetfillcolor{textcolor}%
\pgftext[x=0.120000in,y=1.109254in,left,base]{\color{textcolor}\rmfamily\fontsize{8.000000}{9.600000}\selectfont −3}%
\end{pgfscope}%
\begin{pgfscope}%
\pgfsetbuttcap%
\pgfsetroundjoin%
\definecolor{currentfill}{rgb}{0.000000,0.000000,0.000000}%
\pgfsetfillcolor{currentfill}%
\pgfsetlinewidth{0.803000pt}%
\definecolor{currentstroke}{rgb}{0.000000,0.000000,0.000000}%
\pgfsetstrokecolor{currentstroke}%
\pgfsetdash{}{0pt}%
\pgfsys@defobject{currentmarker}{\pgfqpoint{-0.048611in}{0.000000in}}{\pgfqpoint{0.000000in}{0.000000in}}{%
\pgfpathmoveto{\pgfqpoint{0.000000in}{0.000000in}}%
\pgfpathlineto{\pgfqpoint{-0.048611in}{0.000000in}}%
\pgfusepath{stroke,fill}%
}%
\begin{pgfscope}%
\pgfsys@transformshift{0.368000in}{1.630398in}%
\pgfsys@useobject{currentmarker}{}%
\end{pgfscope}%
\end{pgfscope}%
\begin{pgfscope}%
\definecolor{textcolor}{rgb}{0.000000,0.000000,0.000000}%
\pgfsetstrokecolor{textcolor}%
\pgfsetfillcolor{textcolor}%
\pgftext[x=0.120000in,y=1.591843in,left,base]{\color{textcolor}\rmfamily\fontsize{8.000000}{9.600000}\selectfont −2}%
\end{pgfscope}%
\begin{pgfscope}%
\pgfsetbuttcap%
\pgfsetroundjoin%
\definecolor{currentfill}{rgb}{0.000000,0.000000,0.000000}%
\pgfsetfillcolor{currentfill}%
\pgfsetlinewidth{0.803000pt}%
\definecolor{currentstroke}{rgb}{0.000000,0.000000,0.000000}%
\pgfsetstrokecolor{currentstroke}%
\pgfsetdash{}{0pt}%
\pgfsys@defobject{currentmarker}{\pgfqpoint{-0.048611in}{0.000000in}}{\pgfqpoint{0.000000in}{0.000000in}}{%
\pgfpathmoveto{\pgfqpoint{0.000000in}{0.000000in}}%
\pgfpathlineto{\pgfqpoint{-0.048611in}{0.000000in}}%
\pgfusepath{stroke,fill}%
}%
\begin{pgfscope}%
\pgfsys@transformshift{0.368000in}{2.112988in}%
\pgfsys@useobject{currentmarker}{}%
\end{pgfscope}%
\end{pgfscope}%
\begin{pgfscope}%
\definecolor{textcolor}{rgb}{0.000000,0.000000,0.000000}%
\pgfsetstrokecolor{textcolor}%
\pgfsetfillcolor{textcolor}%
\pgftext[x=0.120000in,y=2.074432in,left,base]{\color{textcolor}\rmfamily\fontsize{8.000000}{9.600000}\selectfont −1}%
\end{pgfscope}%
\begin{pgfscope}%
\pgfsetbuttcap%
\pgfsetroundjoin%
\definecolor{currentfill}{rgb}{0.000000,0.000000,0.000000}%
\pgfsetfillcolor{currentfill}%
\pgfsetlinewidth{0.803000pt}%
\definecolor{currentstroke}{rgb}{0.000000,0.000000,0.000000}%
\pgfsetstrokecolor{currentstroke}%
\pgfsetdash{}{0pt}%
\pgfsys@defobject{currentmarker}{\pgfqpoint{-0.048611in}{0.000000in}}{\pgfqpoint{0.000000in}{0.000000in}}{%
\pgfpathmoveto{\pgfqpoint{0.000000in}{0.000000in}}%
\pgfpathlineto{\pgfqpoint{-0.048611in}{0.000000in}}%
\pgfusepath{stroke,fill}%
}%
\begin{pgfscope}%
\pgfsys@transformshift{0.368000in}{2.595577in}%
\pgfsys@useobject{currentmarker}{}%
\end{pgfscope}%
\end{pgfscope}%
\begin{pgfscope}%
\definecolor{textcolor}{rgb}{0.000000,0.000000,0.000000}%
\pgfsetstrokecolor{textcolor}%
\pgfsetfillcolor{textcolor}%
\pgftext[x=0.211778in,y=2.557021in,left,base]{\color{textcolor}\rmfamily\fontsize{8.000000}{9.600000}\selectfont 0}%
\end{pgfscope}%
\begin{pgfscope}%
\definecolor{textcolor}{rgb}{0.000000,0.000000,0.000000}%
\pgfsetstrokecolor{textcolor}%
\pgfsetfillcolor{textcolor}%
\pgftext[x=0.368000in,y=2.745800in,left,base]{\color{textcolor}\rmfamily\fontsize{8.000000}{9.600000}\selectfont 1e−4}%
\end{pgfscope}%
\begin{pgfscope}%
\pgfpathrectangle{\pgfqpoint{0.368000in}{0.315889in}}{\pgfqpoint{2.377500in}{2.388245in}}%
\pgfusepath{clip}%
\pgfsetrectcap%
\pgfsetroundjoin%
\pgfsetlinewidth{1.505625pt}%
\definecolor{currentstroke}{rgb}{0.121569,0.466667,0.705882}%
\pgfsetstrokecolor{currentstroke}%
\pgfsetdash{}{0pt}%
\pgfpathmoveto{\pgfqpoint{0.476068in}{2.595577in}}%
\pgfpathlineto{\pgfqpoint{0.493222in}{2.595577in}}%
\pgfpathlineto{\pgfqpoint{0.510376in}{2.595577in}}%
\pgfpathlineto{\pgfqpoint{0.527529in}{2.595577in}}%
\pgfpathlineto{\pgfqpoint{0.544683in}{2.595577in}}%
\pgfpathlineto{\pgfqpoint{0.561837in}{2.595577in}}%
\pgfpathlineto{\pgfqpoint{0.578990in}{2.595577in}}%
\pgfpathlineto{\pgfqpoint{0.596144in}{2.595577in}}%
\pgfpathlineto{\pgfqpoint{0.613298in}{2.595577in}}%
\pgfpathlineto{\pgfqpoint{0.630451in}{2.595577in}}%
\pgfpathlineto{\pgfqpoint{0.647605in}{2.595577in}}%
\pgfpathlineto{\pgfqpoint{0.664759in}{2.595577in}}%
\pgfpathlineto{\pgfqpoint{0.681912in}{2.595577in}}%
\pgfpathlineto{\pgfqpoint{0.699066in}{2.595577in}}%
\pgfpathlineto{\pgfqpoint{0.716220in}{2.595577in}}%
\pgfpathlineto{\pgfqpoint{0.733373in}{2.595577in}}%
\pgfpathlineto{\pgfqpoint{0.750527in}{2.595577in}}%
\pgfpathlineto{\pgfqpoint{0.767681in}{2.595577in}}%
\pgfpathlineto{\pgfqpoint{0.784834in}{2.595577in}}%
\pgfpathlineto{\pgfqpoint{0.801988in}{2.595577in}}%
\pgfpathlineto{\pgfqpoint{0.819142in}{2.595577in}}%
\pgfpathlineto{\pgfqpoint{0.836295in}{2.595577in}}%
\pgfpathlineto{\pgfqpoint{0.853449in}{2.595577in}}%
\pgfpathlineto{\pgfqpoint{0.870603in}{2.595577in}}%
\pgfpathlineto{\pgfqpoint{0.887756in}{2.595577in}}%
\pgfpathlineto{\pgfqpoint{0.904910in}{2.595577in}}%
\pgfpathlineto{\pgfqpoint{0.922064in}{2.595577in}}%
\pgfpathlineto{\pgfqpoint{0.939218in}{2.595577in}}%
\pgfpathlineto{\pgfqpoint{0.956371in}{2.595577in}}%
\pgfpathlineto{\pgfqpoint{0.973525in}{2.595577in}}%
\pgfpathlineto{\pgfqpoint{0.990679in}{2.595577in}}%
\pgfpathlineto{\pgfqpoint{1.007832in}{2.595577in}}%
\pgfpathlineto{\pgfqpoint{1.024986in}{2.595577in}}%
\pgfpathlineto{\pgfqpoint{1.042140in}{2.595577in}}%
\pgfpathlineto{\pgfqpoint{1.059293in}{2.595577in}}%
\pgfpathlineto{\pgfqpoint{1.076447in}{2.595577in}}%
\pgfpathlineto{\pgfqpoint{1.093601in}{2.595577in}}%
\pgfpathlineto{\pgfqpoint{1.110754in}{2.595577in}}%
\pgfpathlineto{\pgfqpoint{1.127908in}{2.595577in}}%
\pgfpathlineto{\pgfqpoint{1.145062in}{2.595577in}}%
\pgfpathlineto{\pgfqpoint{1.162215in}{2.595577in}}%
\pgfpathlineto{\pgfqpoint{1.179369in}{2.595577in}}%
\pgfpathlineto{\pgfqpoint{1.196523in}{2.595577in}}%
\pgfpathlineto{\pgfqpoint{1.213676in}{2.595577in}}%
\pgfpathlineto{\pgfqpoint{1.230830in}{2.595577in}}%
\pgfpathlineto{\pgfqpoint{1.247984in}{2.595577in}}%
\pgfpathlineto{\pgfqpoint{1.265137in}{2.595577in}}%
\pgfpathlineto{\pgfqpoint{1.282291in}{2.595577in}}%
\pgfpathlineto{\pgfqpoint{1.299445in}{2.595577in}}%
\pgfpathlineto{\pgfqpoint{1.316598in}{2.595577in}}%
\pgfpathlineto{\pgfqpoint{1.333752in}{2.595577in}}%
\pgfpathlineto{\pgfqpoint{1.350906in}{2.595577in}}%
\pgfpathlineto{\pgfqpoint{1.368060in}{2.595577in}}%
\pgfpathlineto{\pgfqpoint{1.385213in}{2.595577in}}%
\pgfpathlineto{\pgfqpoint{1.402367in}{2.595577in}}%
\pgfpathlineto{\pgfqpoint{1.419521in}{2.595577in}}%
\pgfpathlineto{\pgfqpoint{1.436674in}{2.595577in}}%
\pgfpathlineto{\pgfqpoint{1.453828in}{2.595577in}}%
\pgfpathlineto{\pgfqpoint{1.470982in}{2.595577in}}%
\pgfpathlineto{\pgfqpoint{1.488135in}{2.595577in}}%
\pgfpathlineto{\pgfqpoint{1.505289in}{2.595577in}}%
\pgfpathlineto{\pgfqpoint{1.522443in}{2.595577in}}%
\pgfpathlineto{\pgfqpoint{1.539596in}{2.595577in}}%
\pgfpathlineto{\pgfqpoint{1.556750in}{2.595577in}}%
\pgfpathlineto{\pgfqpoint{1.573904in}{2.595577in}}%
\pgfpathlineto{\pgfqpoint{1.591057in}{2.595577in}}%
\pgfpathlineto{\pgfqpoint{1.608211in}{2.595577in}}%
\pgfpathlineto{\pgfqpoint{1.625365in}{2.595577in}}%
\pgfpathlineto{\pgfqpoint{1.642518in}{2.595577in}}%
\pgfpathlineto{\pgfqpoint{1.659672in}{2.595577in}}%
\pgfpathlineto{\pgfqpoint{1.676826in}{2.595577in}}%
\pgfpathlineto{\pgfqpoint{1.693979in}{2.595577in}}%
\pgfpathlineto{\pgfqpoint{1.711133in}{2.595577in}}%
\pgfpathlineto{\pgfqpoint{1.728287in}{2.595577in}}%
\pgfpathlineto{\pgfqpoint{1.745440in}{2.595577in}}%
\pgfpathlineto{\pgfqpoint{1.762594in}{2.595577in}}%
\pgfpathlineto{\pgfqpoint{1.779748in}{2.595577in}}%
\pgfpathlineto{\pgfqpoint{1.796902in}{2.595577in}}%
\pgfpathlineto{\pgfqpoint{1.814055in}{2.595577in}}%
\pgfpathlineto{\pgfqpoint{1.831209in}{2.595577in}}%
\pgfpathlineto{\pgfqpoint{1.848363in}{2.595577in}}%
\pgfpathlineto{\pgfqpoint{1.865516in}{2.595577in}}%
\pgfpathlineto{\pgfqpoint{1.882670in}{2.595577in}}%
\pgfpathlineto{\pgfqpoint{1.899824in}{2.595577in}}%
\pgfpathlineto{\pgfqpoint{1.916977in}{2.595577in}}%
\pgfpathlineto{\pgfqpoint{1.934131in}{2.595577in}}%
\pgfpathlineto{\pgfqpoint{1.951285in}{2.595577in}}%
\pgfpathlineto{\pgfqpoint{1.968438in}{2.595577in}}%
\pgfpathlineto{\pgfqpoint{1.985592in}{2.595577in}}%
\pgfpathlineto{\pgfqpoint{2.002746in}{2.595577in}}%
\pgfpathlineto{\pgfqpoint{2.019899in}{2.595577in}}%
\pgfpathlineto{\pgfqpoint{2.037053in}{2.595577in}}%
\pgfpathlineto{\pgfqpoint{2.054207in}{2.595577in}}%
\pgfpathlineto{\pgfqpoint{2.071360in}{2.595577in}}%
\pgfpathlineto{\pgfqpoint{2.088514in}{2.595577in}}%
\pgfpathlineto{\pgfqpoint{2.105668in}{2.595577in}}%
\pgfpathlineto{\pgfqpoint{2.122821in}{2.595577in}}%
\pgfpathlineto{\pgfqpoint{2.139975in}{2.595577in}}%
\pgfpathlineto{\pgfqpoint{2.157129in}{2.595577in}}%
\pgfpathlineto{\pgfqpoint{2.174282in}{2.595577in}}%
\pgfpathlineto{\pgfqpoint{2.191436in}{2.595577in}}%
\pgfpathlineto{\pgfqpoint{2.208590in}{2.595577in}}%
\pgfpathlineto{\pgfqpoint{2.225744in}{2.595577in}}%
\pgfpathlineto{\pgfqpoint{2.242897in}{2.595577in}}%
\pgfpathlineto{\pgfqpoint{2.260051in}{2.595577in}}%
\pgfpathlineto{\pgfqpoint{2.277205in}{2.595577in}}%
\pgfpathlineto{\pgfqpoint{2.294358in}{2.595577in}}%
\pgfpathlineto{\pgfqpoint{2.311512in}{2.595577in}}%
\pgfpathlineto{\pgfqpoint{2.328666in}{2.595577in}}%
\pgfpathlineto{\pgfqpoint{2.345819in}{2.595577in}}%
\pgfpathlineto{\pgfqpoint{2.362973in}{2.595577in}}%
\pgfpathlineto{\pgfqpoint{2.380127in}{2.595577in}}%
\pgfpathlineto{\pgfqpoint{2.397280in}{2.595577in}}%
\pgfpathlineto{\pgfqpoint{2.414434in}{2.595577in}}%
\pgfpathlineto{\pgfqpoint{2.431588in}{2.595577in}}%
\pgfpathlineto{\pgfqpoint{2.448741in}{2.595577in}}%
\pgfpathlineto{\pgfqpoint{2.465895in}{2.595577in}}%
\pgfpathlineto{\pgfqpoint{2.483049in}{2.595577in}}%
\pgfpathlineto{\pgfqpoint{2.500202in}{2.595577in}}%
\pgfpathlineto{\pgfqpoint{2.517356in}{2.595577in}}%
\pgfpathlineto{\pgfqpoint{2.534510in}{2.595577in}}%
\pgfpathlineto{\pgfqpoint{2.551663in}{2.595577in}}%
\pgfpathlineto{\pgfqpoint{2.568817in}{2.595577in}}%
\pgfpathlineto{\pgfqpoint{2.585971in}{2.595577in}}%
\pgfpathlineto{\pgfqpoint{2.603124in}{2.595577in}}%
\pgfpathlineto{\pgfqpoint{2.620278in}{2.595577in}}%
\pgfpathlineto{\pgfqpoint{2.637432in}{2.595577in}}%
\pgfusepath{stroke}%
\end{pgfscope}%
\begin{pgfscope}%
\pgfpathrectangle{\pgfqpoint{0.368000in}{0.315889in}}{\pgfqpoint{2.377500in}{2.388245in}}%
\pgfusepath{clip}%
\pgfsetrectcap%
\pgfsetroundjoin%
\pgfsetlinewidth{1.505625pt}%
\definecolor{currentstroke}{rgb}{1.000000,0.498039,0.054902}%
\pgfsetstrokecolor{currentstroke}%
\pgfsetdash{}{0pt}%
\pgfpathmoveto{\pgfqpoint{0.476068in}{2.595577in}}%
\pgfpathlineto{\pgfqpoint{0.493222in}{2.595577in}}%
\pgfpathlineto{\pgfqpoint{0.510376in}{2.595577in}}%
\pgfpathlineto{\pgfqpoint{0.527529in}{2.595577in}}%
\pgfpathlineto{\pgfqpoint{0.544683in}{2.595577in}}%
\pgfpathlineto{\pgfqpoint{0.561837in}{2.595577in}}%
\pgfpathlineto{\pgfqpoint{0.578990in}{2.595577in}}%
\pgfpathlineto{\pgfqpoint{0.596144in}{2.595577in}}%
\pgfpathlineto{\pgfqpoint{0.613298in}{2.595577in}}%
\pgfpathlineto{\pgfqpoint{0.630451in}{2.595577in}}%
\pgfpathlineto{\pgfqpoint{0.647605in}{2.595577in}}%
\pgfpathlineto{\pgfqpoint{0.664759in}{2.595577in}}%
\pgfpathlineto{\pgfqpoint{0.681912in}{2.595577in}}%
\pgfpathlineto{\pgfqpoint{0.699066in}{2.595577in}}%
\pgfpathlineto{\pgfqpoint{0.716220in}{2.595577in}}%
\pgfpathlineto{\pgfqpoint{0.733373in}{2.595577in}}%
\pgfpathlineto{\pgfqpoint{0.750527in}{2.595577in}}%
\pgfpathlineto{\pgfqpoint{0.767681in}{2.595577in}}%
\pgfpathlineto{\pgfqpoint{0.784834in}{2.595577in}}%
\pgfpathlineto{\pgfqpoint{0.801988in}{2.595577in}}%
\pgfpathlineto{\pgfqpoint{0.819142in}{2.595577in}}%
\pgfpathlineto{\pgfqpoint{0.836295in}{2.595577in}}%
\pgfpathlineto{\pgfqpoint{0.853449in}{2.595577in}}%
\pgfpathlineto{\pgfqpoint{0.870603in}{2.595577in}}%
\pgfpathlineto{\pgfqpoint{0.887756in}{2.595577in}}%
\pgfpathlineto{\pgfqpoint{0.904910in}{2.595577in}}%
\pgfpathlineto{\pgfqpoint{0.922064in}{2.595577in}}%
\pgfpathlineto{\pgfqpoint{0.939218in}{2.595577in}}%
\pgfpathlineto{\pgfqpoint{0.956371in}{2.595577in}}%
\pgfpathlineto{\pgfqpoint{0.973525in}{2.595577in}}%
\pgfpathlineto{\pgfqpoint{0.990679in}{2.595577in}}%
\pgfpathlineto{\pgfqpoint{1.007832in}{2.595577in}}%
\pgfpathlineto{\pgfqpoint{1.024986in}{2.595577in}}%
\pgfpathlineto{\pgfqpoint{1.042140in}{2.595577in}}%
\pgfpathlineto{\pgfqpoint{1.059293in}{2.595577in}}%
\pgfpathlineto{\pgfqpoint{1.076447in}{2.595577in}}%
\pgfpathlineto{\pgfqpoint{1.093601in}{2.595577in}}%
\pgfpathlineto{\pgfqpoint{1.110754in}{2.595577in}}%
\pgfpathlineto{\pgfqpoint{1.127908in}{2.595577in}}%
\pgfpathlineto{\pgfqpoint{1.145062in}{2.595577in}}%
\pgfpathlineto{\pgfqpoint{1.162215in}{2.595577in}}%
\pgfpathlineto{\pgfqpoint{1.179369in}{2.595577in}}%
\pgfpathlineto{\pgfqpoint{1.196523in}{2.595577in}}%
\pgfpathlineto{\pgfqpoint{1.213676in}{2.595577in}}%
\pgfpathlineto{\pgfqpoint{1.230830in}{2.595577in}}%
\pgfpathlineto{\pgfqpoint{1.247984in}{2.595577in}}%
\pgfpathlineto{\pgfqpoint{1.265137in}{2.595577in}}%
\pgfpathlineto{\pgfqpoint{1.282291in}{2.595577in}}%
\pgfpathlineto{\pgfqpoint{1.299445in}{2.595577in}}%
\pgfpathlineto{\pgfqpoint{1.316598in}{2.595577in}}%
\pgfpathlineto{\pgfqpoint{1.333752in}{2.595577in}}%
\pgfpathlineto{\pgfqpoint{1.350906in}{2.595577in}}%
\pgfpathlineto{\pgfqpoint{1.368060in}{2.595577in}}%
\pgfpathlineto{\pgfqpoint{1.385213in}{2.595577in}}%
\pgfpathlineto{\pgfqpoint{1.402367in}{2.595577in}}%
\pgfpathlineto{\pgfqpoint{1.419521in}{2.595577in}}%
\pgfpathlineto{\pgfqpoint{1.436674in}{2.595577in}}%
\pgfpathlineto{\pgfqpoint{1.453828in}{2.595577in}}%
\pgfpathlineto{\pgfqpoint{1.470982in}{2.595577in}}%
\pgfpathlineto{\pgfqpoint{1.488135in}{2.595577in}}%
\pgfpathlineto{\pgfqpoint{1.505289in}{2.595577in}}%
\pgfpathlineto{\pgfqpoint{1.522443in}{2.595577in}}%
\pgfpathlineto{\pgfqpoint{1.539596in}{2.595577in}}%
\pgfpathlineto{\pgfqpoint{1.556750in}{2.595577in}}%
\pgfpathlineto{\pgfqpoint{1.573904in}{2.595577in}}%
\pgfpathlineto{\pgfqpoint{1.591057in}{2.595577in}}%
\pgfpathlineto{\pgfqpoint{1.608211in}{2.595577in}}%
\pgfpathlineto{\pgfqpoint{1.625365in}{2.595577in}}%
\pgfpathlineto{\pgfqpoint{1.642518in}{2.595577in}}%
\pgfpathlineto{\pgfqpoint{1.659672in}{2.595577in}}%
\pgfpathlineto{\pgfqpoint{1.676826in}{2.595577in}}%
\pgfpathlineto{\pgfqpoint{1.693979in}{2.595577in}}%
\pgfpathlineto{\pgfqpoint{1.711133in}{2.595577in}}%
\pgfpathlineto{\pgfqpoint{1.728287in}{2.595577in}}%
\pgfpathlineto{\pgfqpoint{1.745440in}{2.595577in}}%
\pgfpathlineto{\pgfqpoint{1.762594in}{2.595577in}}%
\pgfpathlineto{\pgfqpoint{1.779748in}{2.595577in}}%
\pgfpathlineto{\pgfqpoint{1.796902in}{2.595577in}}%
\pgfpathlineto{\pgfqpoint{1.814055in}{2.595577in}}%
\pgfpathlineto{\pgfqpoint{1.831209in}{2.595577in}}%
\pgfpathlineto{\pgfqpoint{1.848363in}{2.595577in}}%
\pgfpathlineto{\pgfqpoint{1.865516in}{2.595577in}}%
\pgfpathlineto{\pgfqpoint{1.882670in}{2.595577in}}%
\pgfpathlineto{\pgfqpoint{1.899824in}{2.595577in}}%
\pgfpathlineto{\pgfqpoint{1.916977in}{2.595577in}}%
\pgfpathlineto{\pgfqpoint{1.934131in}{2.595577in}}%
\pgfpathlineto{\pgfqpoint{1.951285in}{2.595577in}}%
\pgfpathlineto{\pgfqpoint{1.968438in}{2.595577in}}%
\pgfpathlineto{\pgfqpoint{1.985592in}{2.595577in}}%
\pgfpathlineto{\pgfqpoint{2.002746in}{2.595577in}}%
\pgfpathlineto{\pgfqpoint{2.019899in}{2.595577in}}%
\pgfpathlineto{\pgfqpoint{2.037053in}{2.595577in}}%
\pgfpathlineto{\pgfqpoint{2.054207in}{2.595577in}}%
\pgfpathlineto{\pgfqpoint{2.071360in}{2.595577in}}%
\pgfpathlineto{\pgfqpoint{2.088514in}{2.595577in}}%
\pgfpathlineto{\pgfqpoint{2.105668in}{2.595577in}}%
\pgfpathlineto{\pgfqpoint{2.122821in}{2.595577in}}%
\pgfpathlineto{\pgfqpoint{2.139975in}{2.595577in}}%
\pgfpathlineto{\pgfqpoint{2.157129in}{2.595577in}}%
\pgfpathlineto{\pgfqpoint{2.174282in}{2.595577in}}%
\pgfpathlineto{\pgfqpoint{2.191436in}{2.595577in}}%
\pgfpathlineto{\pgfqpoint{2.208590in}{2.595577in}}%
\pgfpathlineto{\pgfqpoint{2.225744in}{2.595577in}}%
\pgfpathlineto{\pgfqpoint{2.242897in}{2.595577in}}%
\pgfpathlineto{\pgfqpoint{2.260051in}{2.595577in}}%
\pgfpathlineto{\pgfqpoint{2.277205in}{2.595577in}}%
\pgfpathlineto{\pgfqpoint{2.294358in}{2.595577in}}%
\pgfpathlineto{\pgfqpoint{2.311512in}{2.595577in}}%
\pgfpathlineto{\pgfqpoint{2.328666in}{2.595577in}}%
\pgfpathlineto{\pgfqpoint{2.345819in}{2.595577in}}%
\pgfpathlineto{\pgfqpoint{2.362973in}{2.595577in}}%
\pgfpathlineto{\pgfqpoint{2.380127in}{2.595577in}}%
\pgfpathlineto{\pgfqpoint{2.397280in}{2.595577in}}%
\pgfpathlineto{\pgfqpoint{2.414434in}{2.595577in}}%
\pgfpathlineto{\pgfqpoint{2.431588in}{2.595577in}}%
\pgfpathlineto{\pgfqpoint{2.448741in}{2.595577in}}%
\pgfpathlineto{\pgfqpoint{2.465895in}{2.595577in}}%
\pgfpathlineto{\pgfqpoint{2.483049in}{2.595577in}}%
\pgfpathlineto{\pgfqpoint{2.500202in}{2.595577in}}%
\pgfpathlineto{\pgfqpoint{2.517356in}{2.595577in}}%
\pgfpathlineto{\pgfqpoint{2.534510in}{2.595577in}}%
\pgfpathlineto{\pgfqpoint{2.551663in}{2.595577in}}%
\pgfpathlineto{\pgfqpoint{2.568817in}{2.595577in}}%
\pgfpathlineto{\pgfqpoint{2.585971in}{2.595577in}}%
\pgfpathlineto{\pgfqpoint{2.603124in}{2.595577in}}%
\pgfpathlineto{\pgfqpoint{2.620278in}{2.595577in}}%
\pgfpathlineto{\pgfqpoint{2.637432in}{2.595577in}}%
\pgfusepath{stroke}%
\end{pgfscope}%
\begin{pgfscope}%
\pgfpathrectangle{\pgfqpoint{0.368000in}{0.315889in}}{\pgfqpoint{2.377500in}{2.388245in}}%
\pgfusepath{clip}%
\pgfsetrectcap%
\pgfsetroundjoin%
\pgfsetlinewidth{1.505625pt}%
\definecolor{currentstroke}{rgb}{0.172549,0.627451,0.172549}%
\pgfsetstrokecolor{currentstroke}%
\pgfsetdash{}{0pt}%
\pgfpathmoveto{\pgfqpoint{0.476068in}{2.231994in}}%
\pgfpathlineto{\pgfqpoint{0.493222in}{2.231994in}}%
\pgfpathlineto{\pgfqpoint{0.510376in}{2.231994in}}%
\pgfpathlineto{\pgfqpoint{0.527529in}{2.231994in}}%
\pgfpathlineto{\pgfqpoint{0.544683in}{2.231994in}}%
\pgfpathlineto{\pgfqpoint{0.561837in}{2.231994in}}%
\pgfpathlineto{\pgfqpoint{0.578990in}{2.231994in}}%
\pgfpathlineto{\pgfqpoint{0.596144in}{2.231994in}}%
\pgfpathlineto{\pgfqpoint{0.613298in}{2.231994in}}%
\pgfpathlineto{\pgfqpoint{0.630451in}{2.231994in}}%
\pgfpathlineto{\pgfqpoint{0.647605in}{2.231994in}}%
\pgfpathlineto{\pgfqpoint{0.664759in}{2.231994in}}%
\pgfpathlineto{\pgfqpoint{0.681912in}{2.231994in}}%
\pgfpathlineto{\pgfqpoint{0.699066in}{2.231994in}}%
\pgfpathlineto{\pgfqpoint{0.716220in}{2.231994in}}%
\pgfpathlineto{\pgfqpoint{0.733373in}{2.231994in}}%
\pgfpathlineto{\pgfqpoint{0.750527in}{2.231994in}}%
\pgfpathlineto{\pgfqpoint{0.767681in}{2.231994in}}%
\pgfpathlineto{\pgfqpoint{0.784834in}{2.231994in}}%
\pgfpathlineto{\pgfqpoint{0.801988in}{2.231994in}}%
\pgfpathlineto{\pgfqpoint{0.819142in}{2.231994in}}%
\pgfpathlineto{\pgfqpoint{0.836295in}{2.231994in}}%
\pgfpathlineto{\pgfqpoint{0.853449in}{2.231994in}}%
\pgfpathlineto{\pgfqpoint{0.870603in}{2.231994in}}%
\pgfpathlineto{\pgfqpoint{0.887756in}{2.231994in}}%
\pgfpathlineto{\pgfqpoint{0.904910in}{2.231994in}}%
\pgfpathlineto{\pgfqpoint{0.922064in}{2.231994in}}%
\pgfpathlineto{\pgfqpoint{0.939218in}{2.231994in}}%
\pgfpathlineto{\pgfqpoint{0.956371in}{2.231994in}}%
\pgfpathlineto{\pgfqpoint{0.973525in}{2.231994in}}%
\pgfpathlineto{\pgfqpoint{0.990679in}{2.231994in}}%
\pgfpathlineto{\pgfqpoint{1.007832in}{2.231994in}}%
\pgfpathlineto{\pgfqpoint{1.024986in}{2.231994in}}%
\pgfpathlineto{\pgfqpoint{1.042140in}{2.231994in}}%
\pgfpathlineto{\pgfqpoint{1.059293in}{2.231994in}}%
\pgfpathlineto{\pgfqpoint{1.076447in}{2.231994in}}%
\pgfpathlineto{\pgfqpoint{1.093601in}{2.231994in}}%
\pgfpathlineto{\pgfqpoint{1.110754in}{2.231994in}}%
\pgfpathlineto{\pgfqpoint{1.127908in}{2.231994in}}%
\pgfpathlineto{\pgfqpoint{1.145062in}{2.231994in}}%
\pgfpathlineto{\pgfqpoint{1.162215in}{2.231994in}}%
\pgfpathlineto{\pgfqpoint{1.179369in}{2.231994in}}%
\pgfpathlineto{\pgfqpoint{1.196523in}{2.231994in}}%
\pgfpathlineto{\pgfqpoint{1.213676in}{2.231994in}}%
\pgfpathlineto{\pgfqpoint{1.230830in}{2.231994in}}%
\pgfpathlineto{\pgfqpoint{1.247984in}{2.231994in}}%
\pgfpathlineto{\pgfqpoint{1.265137in}{2.231994in}}%
\pgfpathlineto{\pgfqpoint{1.282291in}{2.231994in}}%
\pgfpathlineto{\pgfqpoint{1.299445in}{2.231994in}}%
\pgfpathlineto{\pgfqpoint{1.316598in}{2.231994in}}%
\pgfpathlineto{\pgfqpoint{1.333752in}{2.231994in}}%
\pgfpathlineto{\pgfqpoint{1.350906in}{2.231994in}}%
\pgfpathlineto{\pgfqpoint{1.368060in}{2.231994in}}%
\pgfpathlineto{\pgfqpoint{1.385213in}{2.231994in}}%
\pgfpathlineto{\pgfqpoint{1.402367in}{2.231994in}}%
\pgfpathlineto{\pgfqpoint{1.419521in}{2.231994in}}%
\pgfpathlineto{\pgfqpoint{1.436674in}{2.231994in}}%
\pgfpathlineto{\pgfqpoint{1.453828in}{2.231994in}}%
\pgfpathlineto{\pgfqpoint{1.470982in}{2.231994in}}%
\pgfpathlineto{\pgfqpoint{1.488135in}{2.231994in}}%
\pgfpathlineto{\pgfqpoint{1.505289in}{2.231994in}}%
\pgfpathlineto{\pgfqpoint{1.522443in}{2.231994in}}%
\pgfpathlineto{\pgfqpoint{1.539596in}{2.231994in}}%
\pgfpathlineto{\pgfqpoint{1.556750in}{2.231994in}}%
\pgfpathlineto{\pgfqpoint{1.573904in}{2.231994in}}%
\pgfpathlineto{\pgfqpoint{1.591057in}{2.231994in}}%
\pgfpathlineto{\pgfqpoint{1.608211in}{2.231994in}}%
\pgfpathlineto{\pgfqpoint{1.625365in}{2.231994in}}%
\pgfpathlineto{\pgfqpoint{1.642518in}{2.231994in}}%
\pgfpathlineto{\pgfqpoint{1.659672in}{2.231994in}}%
\pgfpathlineto{\pgfqpoint{1.676826in}{2.231994in}}%
\pgfpathlineto{\pgfqpoint{1.693979in}{2.231994in}}%
\pgfpathlineto{\pgfqpoint{1.711133in}{2.231994in}}%
\pgfpathlineto{\pgfqpoint{1.728287in}{2.231994in}}%
\pgfpathlineto{\pgfqpoint{1.745440in}{2.231994in}}%
\pgfpathlineto{\pgfqpoint{1.762594in}{2.231994in}}%
\pgfpathlineto{\pgfqpoint{1.779748in}{2.231994in}}%
\pgfpathlineto{\pgfqpoint{1.796902in}{2.231994in}}%
\pgfpathlineto{\pgfqpoint{1.814055in}{2.231994in}}%
\pgfpathlineto{\pgfqpoint{1.831209in}{2.231994in}}%
\pgfpathlineto{\pgfqpoint{1.848363in}{2.231994in}}%
\pgfpathlineto{\pgfqpoint{1.865516in}{2.231994in}}%
\pgfpathlineto{\pgfqpoint{1.882670in}{2.231994in}}%
\pgfpathlineto{\pgfqpoint{1.899824in}{2.231994in}}%
\pgfpathlineto{\pgfqpoint{1.916977in}{2.231994in}}%
\pgfpathlineto{\pgfqpoint{1.934131in}{2.231994in}}%
\pgfpathlineto{\pgfqpoint{1.951285in}{2.231994in}}%
\pgfpathlineto{\pgfqpoint{1.968438in}{2.231994in}}%
\pgfpathlineto{\pgfqpoint{1.985592in}{2.231994in}}%
\pgfpathlineto{\pgfqpoint{2.002746in}{2.231994in}}%
\pgfpathlineto{\pgfqpoint{2.019899in}{2.231994in}}%
\pgfpathlineto{\pgfqpoint{2.037053in}{2.231994in}}%
\pgfpathlineto{\pgfqpoint{2.054207in}{2.231994in}}%
\pgfpathlineto{\pgfqpoint{2.071360in}{2.231994in}}%
\pgfpathlineto{\pgfqpoint{2.088514in}{2.231994in}}%
\pgfpathlineto{\pgfqpoint{2.105668in}{2.231994in}}%
\pgfpathlineto{\pgfqpoint{2.122821in}{2.231994in}}%
\pgfpathlineto{\pgfqpoint{2.139975in}{2.231994in}}%
\pgfpathlineto{\pgfqpoint{2.157129in}{2.231994in}}%
\pgfpathlineto{\pgfqpoint{2.174282in}{2.231994in}}%
\pgfpathlineto{\pgfqpoint{2.191436in}{2.231994in}}%
\pgfpathlineto{\pgfqpoint{2.208590in}{2.231994in}}%
\pgfpathlineto{\pgfqpoint{2.225744in}{2.231994in}}%
\pgfpathlineto{\pgfqpoint{2.242897in}{2.231994in}}%
\pgfpathlineto{\pgfqpoint{2.260051in}{2.231994in}}%
\pgfpathlineto{\pgfqpoint{2.277205in}{2.231994in}}%
\pgfpathlineto{\pgfqpoint{2.294358in}{2.231994in}}%
\pgfpathlineto{\pgfqpoint{2.311512in}{2.231994in}}%
\pgfpathlineto{\pgfqpoint{2.328666in}{2.231994in}}%
\pgfpathlineto{\pgfqpoint{2.345819in}{2.231994in}}%
\pgfpathlineto{\pgfqpoint{2.362973in}{2.231994in}}%
\pgfpathlineto{\pgfqpoint{2.380127in}{2.231994in}}%
\pgfpathlineto{\pgfqpoint{2.397280in}{2.231994in}}%
\pgfpathlineto{\pgfqpoint{2.414434in}{2.231994in}}%
\pgfpathlineto{\pgfqpoint{2.431588in}{2.231994in}}%
\pgfpathlineto{\pgfqpoint{2.448741in}{2.231994in}}%
\pgfpathlineto{\pgfqpoint{2.465895in}{2.231994in}}%
\pgfpathlineto{\pgfqpoint{2.483049in}{2.231994in}}%
\pgfpathlineto{\pgfqpoint{2.500202in}{2.231994in}}%
\pgfpathlineto{\pgfqpoint{2.517356in}{2.231994in}}%
\pgfpathlineto{\pgfqpoint{2.534510in}{2.231994in}}%
\pgfpathlineto{\pgfqpoint{2.551663in}{2.231994in}}%
\pgfpathlineto{\pgfqpoint{2.568817in}{2.231994in}}%
\pgfpathlineto{\pgfqpoint{2.585971in}{2.231994in}}%
\pgfpathlineto{\pgfqpoint{2.603124in}{2.231994in}}%
\pgfpathlineto{\pgfqpoint{2.620278in}{2.231994in}}%
\pgfpathlineto{\pgfqpoint{2.637432in}{2.231994in}}%
\pgfusepath{stroke}%
\end{pgfscope}%
\begin{pgfscope}%
\pgfpathrectangle{\pgfqpoint{0.368000in}{0.315889in}}{\pgfqpoint{2.377500in}{2.388245in}}%
\pgfusepath{clip}%
\pgfsetrectcap%
\pgfsetroundjoin%
\pgfsetlinewidth{1.505625pt}%
\definecolor{currentstroke}{rgb}{0.839216,0.152941,0.156863}%
\pgfsetstrokecolor{currentstroke}%
\pgfsetdash{}{0pt}%
\pgfpathmoveto{\pgfqpoint{0.476068in}{2.580275in}}%
\pgfpathlineto{\pgfqpoint{0.493222in}{2.563165in}}%
\pgfpathlineto{\pgfqpoint{0.510376in}{2.546055in}}%
\pgfpathlineto{\pgfqpoint{0.527529in}{2.528946in}}%
\pgfpathlineto{\pgfqpoint{0.544683in}{2.511836in}}%
\pgfpathlineto{\pgfqpoint{0.561837in}{2.494726in}}%
\pgfpathlineto{\pgfqpoint{0.578990in}{2.477616in}}%
\pgfpathlineto{\pgfqpoint{0.596144in}{2.460507in}}%
\pgfpathlineto{\pgfqpoint{0.613298in}{2.443397in}}%
\pgfpathlineto{\pgfqpoint{0.630451in}{2.426287in}}%
\pgfpathlineto{\pgfqpoint{0.647605in}{2.409177in}}%
\pgfpathlineto{\pgfqpoint{0.664759in}{2.392068in}}%
\pgfpathlineto{\pgfqpoint{0.681912in}{2.374958in}}%
\pgfpathlineto{\pgfqpoint{0.699066in}{2.357848in}}%
\pgfpathlineto{\pgfqpoint{0.716220in}{2.340738in}}%
\pgfpathlineto{\pgfqpoint{0.733373in}{2.323628in}}%
\pgfpathlineto{\pgfqpoint{0.750527in}{2.306519in}}%
\pgfpathlineto{\pgfqpoint{0.767681in}{2.289409in}}%
\pgfpathlineto{\pgfqpoint{0.784834in}{2.272299in}}%
\pgfpathlineto{\pgfqpoint{0.801988in}{2.255189in}}%
\pgfpathlineto{\pgfqpoint{0.819142in}{2.238080in}}%
\pgfpathlineto{\pgfqpoint{0.836295in}{2.220970in}}%
\pgfpathlineto{\pgfqpoint{0.853449in}{2.203860in}}%
\pgfpathlineto{\pgfqpoint{0.870603in}{2.186750in}}%
\pgfpathlineto{\pgfqpoint{0.887756in}{2.169641in}}%
\pgfpathlineto{\pgfqpoint{0.904910in}{2.152531in}}%
\pgfpathlineto{\pgfqpoint{0.922064in}{2.135421in}}%
\pgfpathlineto{\pgfqpoint{0.939218in}{2.118311in}}%
\pgfpathlineto{\pgfqpoint{0.956371in}{2.101202in}}%
\pgfpathlineto{\pgfqpoint{0.973525in}{2.084092in}}%
\pgfpathlineto{\pgfqpoint{0.990679in}{2.066982in}}%
\pgfpathlineto{\pgfqpoint{1.007832in}{2.049872in}}%
\pgfpathlineto{\pgfqpoint{1.024986in}{2.032763in}}%
\pgfpathlineto{\pgfqpoint{1.042140in}{2.015653in}}%
\pgfpathlineto{\pgfqpoint{1.059293in}{1.998543in}}%
\pgfpathlineto{\pgfqpoint{1.076447in}{1.981433in}}%
\pgfpathlineto{\pgfqpoint{1.093601in}{1.964324in}}%
\pgfpathlineto{\pgfqpoint{1.110754in}{1.947214in}}%
\pgfpathlineto{\pgfqpoint{1.127908in}{1.930104in}}%
\pgfpathlineto{\pgfqpoint{1.145062in}{1.912994in}}%
\pgfpathlineto{\pgfqpoint{1.162215in}{1.895885in}}%
\pgfpathlineto{\pgfqpoint{1.179369in}{1.878775in}}%
\pgfpathlineto{\pgfqpoint{1.196523in}{1.861665in}}%
\pgfpathlineto{\pgfqpoint{1.213676in}{1.844555in}}%
\pgfpathlineto{\pgfqpoint{1.230830in}{1.827445in}}%
\pgfpathlineto{\pgfqpoint{1.247984in}{1.810336in}}%
\pgfpathlineto{\pgfqpoint{1.265137in}{1.793226in}}%
\pgfpathlineto{\pgfqpoint{1.282291in}{1.776116in}}%
\pgfpathlineto{\pgfqpoint{1.299445in}{1.759006in}}%
\pgfpathlineto{\pgfqpoint{1.316598in}{1.741897in}}%
\pgfpathlineto{\pgfqpoint{1.333752in}{1.724787in}}%
\pgfpathlineto{\pgfqpoint{1.350906in}{1.707677in}}%
\pgfpathlineto{\pgfqpoint{1.368060in}{1.690567in}}%
\pgfpathlineto{\pgfqpoint{1.385213in}{1.673458in}}%
\pgfpathlineto{\pgfqpoint{1.402367in}{1.656348in}}%
\pgfpathlineto{\pgfqpoint{1.419521in}{1.639238in}}%
\pgfpathlineto{\pgfqpoint{1.436674in}{1.622128in}}%
\pgfpathlineto{\pgfqpoint{1.453828in}{1.605019in}}%
\pgfpathlineto{\pgfqpoint{1.470982in}{1.587909in}}%
\pgfpathlineto{\pgfqpoint{1.488135in}{1.570799in}}%
\pgfpathlineto{\pgfqpoint{1.505289in}{1.553689in}}%
\pgfpathlineto{\pgfqpoint{1.522443in}{1.536580in}}%
\pgfpathlineto{\pgfqpoint{1.539596in}{1.519470in}}%
\pgfpathlineto{\pgfqpoint{1.556750in}{1.502360in}}%
\pgfpathlineto{\pgfqpoint{1.573904in}{1.485250in}}%
\pgfpathlineto{\pgfqpoint{1.591057in}{1.468141in}}%
\pgfpathlineto{\pgfqpoint{1.608211in}{1.451031in}}%
\pgfpathlineto{\pgfqpoint{1.625365in}{1.433921in}}%
\pgfpathlineto{\pgfqpoint{1.642518in}{1.416811in}}%
\pgfpathlineto{\pgfqpoint{1.659672in}{1.399702in}}%
\pgfpathlineto{\pgfqpoint{1.676826in}{1.382592in}}%
\pgfpathlineto{\pgfqpoint{1.693979in}{1.365482in}}%
\pgfpathlineto{\pgfqpoint{1.711133in}{1.348372in}}%
\pgfpathlineto{\pgfqpoint{1.728287in}{1.331263in}}%
\pgfpathlineto{\pgfqpoint{1.745440in}{1.314153in}}%
\pgfpathlineto{\pgfqpoint{1.762594in}{1.297043in}}%
\pgfpathlineto{\pgfqpoint{1.779748in}{1.279933in}}%
\pgfpathlineto{\pgfqpoint{1.796902in}{1.262823in}}%
\pgfpathlineto{\pgfqpoint{1.814055in}{1.245714in}}%
\pgfpathlineto{\pgfqpoint{1.831209in}{1.228604in}}%
\pgfpathlineto{\pgfqpoint{1.848363in}{1.211494in}}%
\pgfpathlineto{\pgfqpoint{1.865516in}{1.194384in}}%
\pgfpathlineto{\pgfqpoint{1.882670in}{1.177275in}}%
\pgfpathlineto{\pgfqpoint{1.899824in}{1.160165in}}%
\pgfpathlineto{\pgfqpoint{1.916977in}{1.143055in}}%
\pgfpathlineto{\pgfqpoint{1.934131in}{1.125945in}}%
\pgfpathlineto{\pgfqpoint{1.951285in}{1.108836in}}%
\pgfpathlineto{\pgfqpoint{1.968438in}{1.091726in}}%
\pgfpathlineto{\pgfqpoint{1.985592in}{1.074616in}}%
\pgfpathlineto{\pgfqpoint{2.002746in}{1.057506in}}%
\pgfpathlineto{\pgfqpoint{2.019899in}{1.040397in}}%
\pgfpathlineto{\pgfqpoint{2.037053in}{1.023287in}}%
\pgfpathlineto{\pgfqpoint{2.054207in}{1.006177in}}%
\pgfpathlineto{\pgfqpoint{2.071360in}{0.989067in}}%
\pgfpathlineto{\pgfqpoint{2.088514in}{0.971958in}}%
\pgfpathlineto{\pgfqpoint{2.105668in}{0.954848in}}%
\pgfpathlineto{\pgfqpoint{2.122821in}{0.937738in}}%
\pgfpathlineto{\pgfqpoint{2.139975in}{0.920628in}}%
\pgfpathlineto{\pgfqpoint{2.157129in}{0.903519in}}%
\pgfpathlineto{\pgfqpoint{2.174282in}{0.886409in}}%
\pgfpathlineto{\pgfqpoint{2.191436in}{0.869299in}}%
\pgfpathlineto{\pgfqpoint{2.208590in}{0.852189in}}%
\pgfpathlineto{\pgfqpoint{2.225744in}{0.835080in}}%
\pgfpathlineto{\pgfqpoint{2.242897in}{0.817970in}}%
\pgfpathlineto{\pgfqpoint{2.260051in}{0.800860in}}%
\pgfpathlineto{\pgfqpoint{2.277205in}{0.783750in}}%
\pgfpathlineto{\pgfqpoint{2.294358in}{0.766640in}}%
\pgfpathlineto{\pgfqpoint{2.311512in}{0.749531in}}%
\pgfpathlineto{\pgfqpoint{2.328666in}{0.732421in}}%
\pgfpathlineto{\pgfqpoint{2.345819in}{0.715311in}}%
\pgfpathlineto{\pgfqpoint{2.362973in}{0.698201in}}%
\pgfpathlineto{\pgfqpoint{2.380127in}{0.681092in}}%
\pgfpathlineto{\pgfqpoint{2.397280in}{0.663982in}}%
\pgfpathlineto{\pgfqpoint{2.414434in}{0.646872in}}%
\pgfpathlineto{\pgfqpoint{2.431588in}{0.629762in}}%
\pgfpathlineto{\pgfqpoint{2.448741in}{0.612653in}}%
\pgfpathlineto{\pgfqpoint{2.465895in}{0.595543in}}%
\pgfpathlineto{\pgfqpoint{2.483049in}{0.578433in}}%
\pgfpathlineto{\pgfqpoint{2.500202in}{0.561323in}}%
\pgfpathlineto{\pgfqpoint{2.517356in}{0.544214in}}%
\pgfpathlineto{\pgfqpoint{2.534510in}{0.527104in}}%
\pgfpathlineto{\pgfqpoint{2.551663in}{0.509994in}}%
\pgfpathlineto{\pgfqpoint{2.568817in}{0.492884in}}%
\pgfpathlineto{\pgfqpoint{2.585971in}{0.475775in}}%
\pgfpathlineto{\pgfqpoint{2.603124in}{0.458665in}}%
\pgfpathlineto{\pgfqpoint{2.620278in}{0.441555in}}%
\pgfpathlineto{\pgfqpoint{2.637432in}{0.424445in}}%
\pgfusepath{stroke}%
\end{pgfscope}%
\begin{pgfscope}%
\pgfsetrectcap%
\pgfsetmiterjoin%
\pgfsetlinewidth{0.803000pt}%
\definecolor{currentstroke}{rgb}{0.000000,0.000000,0.000000}%
\pgfsetstrokecolor{currentstroke}%
\pgfsetdash{}{0pt}%
\pgfpathmoveto{\pgfqpoint{0.368000in}{0.315889in}}%
\pgfpathlineto{\pgfqpoint{0.368000in}{2.704133in}}%
\pgfusepath{stroke}%
\end{pgfscope}%
\begin{pgfscope}%
\pgfsetrectcap%
\pgfsetmiterjoin%
\pgfsetlinewidth{0.803000pt}%
\definecolor{currentstroke}{rgb}{0.000000,0.000000,0.000000}%
\pgfsetstrokecolor{currentstroke}%
\pgfsetdash{}{0pt}%
\pgfpathmoveto{\pgfqpoint{2.745500in}{0.315889in}}%
\pgfpathlineto{\pgfqpoint{2.745500in}{2.704133in}}%
\pgfusepath{stroke}%
\end{pgfscope}%
\begin{pgfscope}%
\pgfsetrectcap%
\pgfsetmiterjoin%
\pgfsetlinewidth{0.803000pt}%
\definecolor{currentstroke}{rgb}{0.000000,0.000000,0.000000}%
\pgfsetstrokecolor{currentstroke}%
\pgfsetdash{}{0pt}%
\pgfpathmoveto{\pgfqpoint{0.368000in}{0.315889in}}%
\pgfpathlineto{\pgfqpoint{2.745500in}{0.315889in}}%
\pgfusepath{stroke}%
\end{pgfscope}%
\begin{pgfscope}%
\pgfsetrectcap%
\pgfsetmiterjoin%
\pgfsetlinewidth{0.803000pt}%
\definecolor{currentstroke}{rgb}{0.000000,0.000000,0.000000}%
\pgfsetstrokecolor{currentstroke}%
\pgfsetdash{}{0pt}%
\pgfpathmoveto{\pgfqpoint{0.368000in}{2.704133in}}%
\pgfpathlineto{\pgfqpoint{2.745500in}{2.704133in}}%
\pgfusepath{stroke}%
\end{pgfscope}%
\begin{pgfscope}%
\definecolor{textcolor}{rgb}{0.000000,0.000000,0.000000}%
\pgfsetstrokecolor{textcolor}%
\pgfsetfillcolor{textcolor}%
\pgftext[x=1.556750in,y=2.787467in,,base]{\color{textcolor}\rmfamily\fontsize{9.600000}{11.520000}\selectfont db2}%
\end{pgfscope}%
\begin{pgfscope}%
\pgfsetbuttcap%
\pgfsetmiterjoin%
\definecolor{currentfill}{rgb}{1.000000,1.000000,1.000000}%
\pgfsetfillcolor{currentfill}%
\pgfsetfillopacity{0.800000}%
\pgfsetlinewidth{1.003750pt}%
\definecolor{currentstroke}{rgb}{0.800000,0.800000,0.800000}%
\pgfsetstrokecolor{currentstroke}%
\pgfsetstrokeopacity{0.800000}%
\pgfsetdash{}{0pt}%
\pgfpathmoveto{\pgfqpoint{0.445778in}{0.371444in}}%
\pgfpathlineto{\pgfqpoint{0.925658in}{0.371444in}}%
\pgfpathquadraticcurveto{\pgfqpoint{0.947880in}{0.371444in}}{\pgfqpoint{0.947880in}{0.393667in}}%
\pgfpathlineto{\pgfqpoint{0.947880in}{1.004222in}}%
\pgfpathquadraticcurveto{\pgfqpoint{0.947880in}{1.026444in}}{\pgfqpoint{0.925658in}{1.026444in}}%
\pgfpathlineto{\pgfqpoint{0.445778in}{1.026444in}}%
\pgfpathquadraticcurveto{\pgfqpoint{0.423556in}{1.026444in}}{\pgfqpoint{0.423556in}{1.004222in}}%
\pgfpathlineto{\pgfqpoint{0.423556in}{0.393667in}}%
\pgfpathquadraticcurveto{\pgfqpoint{0.423556in}{0.371444in}}{\pgfqpoint{0.445778in}{0.371444in}}%
\pgfpathclose%
\pgfusepath{stroke,fill}%
\end{pgfscope}%
\begin{pgfscope}%
\pgfsetrectcap%
\pgfsetroundjoin%
\pgfsetlinewidth{1.505625pt}%
\definecolor{currentstroke}{rgb}{0.121569,0.466667,0.705882}%
\pgfsetstrokecolor{currentstroke}%
\pgfsetdash{}{0pt}%
\pgfpathmoveto{\pgfqpoint{0.468000in}{0.942583in}}%
\pgfpathlineto{\pgfqpoint{0.690222in}{0.942583in}}%
\pgfusepath{stroke}%
\end{pgfscope}%
\begin{pgfscope}%
\definecolor{textcolor}{rgb}{0.000000,0.000000,0.000000}%
\pgfsetstrokecolor{textcolor}%
\pgfsetfillcolor{textcolor}%
\pgftext[x=0.779111in,y=0.903694in,left,base]{\color{textcolor}\rmfamily\fontsize{8.000000}{9.600000}\selectfont \(\displaystyle x^{0}\)}%
\end{pgfscope}%
\begin{pgfscope}%
\pgfsetrectcap%
\pgfsetroundjoin%
\pgfsetlinewidth{1.505625pt}%
\definecolor{currentstroke}{rgb}{1.000000,0.498039,0.054902}%
\pgfsetstrokecolor{currentstroke}%
\pgfsetdash{}{0pt}%
\pgfpathmoveto{\pgfqpoint{0.468000in}{0.787166in}}%
\pgfpathlineto{\pgfqpoint{0.690222in}{0.787166in}}%
\pgfusepath{stroke}%
\end{pgfscope}%
\begin{pgfscope}%
\definecolor{textcolor}{rgb}{0.000000,0.000000,0.000000}%
\pgfsetstrokecolor{textcolor}%
\pgfsetfillcolor{textcolor}%
\pgftext[x=0.779111in,y=0.748277in,left,base]{\color{textcolor}\rmfamily\fontsize{8.000000}{9.600000}\selectfont \(\displaystyle x^{1}\)}%
\end{pgfscope}%
\begin{pgfscope}%
\pgfsetrectcap%
\pgfsetroundjoin%
\pgfsetlinewidth{1.505625pt}%
\definecolor{currentstroke}{rgb}{0.172549,0.627451,0.172549}%
\pgfsetstrokecolor{currentstroke}%
\pgfsetdash{}{0pt}%
\pgfpathmoveto{\pgfqpoint{0.468000in}{0.631750in}}%
\pgfpathlineto{\pgfqpoint{0.690222in}{0.631750in}}%
\pgfusepath{stroke}%
\end{pgfscope}%
\begin{pgfscope}%
\definecolor{textcolor}{rgb}{0.000000,0.000000,0.000000}%
\pgfsetstrokecolor{textcolor}%
\pgfsetfillcolor{textcolor}%
\pgftext[x=0.779111in,y=0.592861in,left,base]{\color{textcolor}\rmfamily\fontsize{8.000000}{9.600000}\selectfont \(\displaystyle x^{2}\)}%
\end{pgfscope}%
\begin{pgfscope}%
\pgfsetrectcap%
\pgfsetroundjoin%
\pgfsetlinewidth{1.505625pt}%
\definecolor{currentstroke}{rgb}{0.839216,0.152941,0.156863}%
\pgfsetstrokecolor{currentstroke}%
\pgfsetdash{}{0pt}%
\pgfpathmoveto{\pgfqpoint{0.468000in}{0.476333in}}%
\pgfpathlineto{\pgfqpoint{0.690222in}{0.476333in}}%
\pgfusepath{stroke}%
\end{pgfscope}%
\begin{pgfscope}%
\definecolor{textcolor}{rgb}{0.000000,0.000000,0.000000}%
\pgfsetstrokecolor{textcolor}%
\pgfsetfillcolor{textcolor}%
\pgftext[x=0.779111in,y=0.437444in,left,base]{\color{textcolor}\rmfamily\fontsize{8.000000}{9.600000}\selectfont \(\displaystyle x^{3}\)}%
\end{pgfscope}%
\begin{pgfscope}%
\pgfsetbuttcap%
\pgfsetmiterjoin%
\definecolor{currentfill}{rgb}{1.000000,1.000000,1.000000}%
\pgfsetfillcolor{currentfill}%
\pgfsetlinewidth{0.000000pt}%
\definecolor{currentstroke}{rgb}{0.000000,0.000000,0.000000}%
\pgfsetstrokecolor{currentstroke}%
\pgfsetstrokeopacity{0.000000}%
\pgfsetdash{}{0pt}%
\pgfpathmoveto{\pgfqpoint{2.962722in}{0.315889in}}%
\pgfpathlineto{\pgfqpoint{5.340222in}{0.315889in}}%
\pgfpathlineto{\pgfqpoint{5.340222in}{2.704133in}}%
\pgfpathlineto{\pgfqpoint{2.962722in}{2.704133in}}%
\pgfpathclose%
\pgfusepath{fill}%
\end{pgfscope}%
\begin{pgfscope}%
\pgfsetbuttcap%
\pgfsetroundjoin%
\definecolor{currentfill}{rgb}{0.000000,0.000000,0.000000}%
\pgfsetfillcolor{currentfill}%
\pgfsetlinewidth{0.803000pt}%
\definecolor{currentstroke}{rgb}{0.000000,0.000000,0.000000}%
\pgfsetstrokecolor{currentstroke}%
\pgfsetdash{}{0pt}%
\pgfsys@defobject{currentmarker}{\pgfqpoint{0.000000in}{-0.048611in}}{\pgfqpoint{0.000000in}{0.000000in}}{%
\pgfpathmoveto{\pgfqpoint{0.000000in}{0.000000in}}%
\pgfpathlineto{\pgfqpoint{0.000000in}{-0.048611in}}%
\pgfusepath{stroke,fill}%
}%
\begin{pgfscope}%
\pgfsys@transformshift{3.070790in}{0.315889in}%
\pgfsys@useobject{currentmarker}{}%
\end{pgfscope}%
\end{pgfscope}%
\begin{pgfscope}%
\definecolor{textcolor}{rgb}{0.000000,0.000000,0.000000}%
\pgfsetstrokecolor{textcolor}%
\pgfsetfillcolor{textcolor}%
\pgftext[x=3.070790in,y=0.218667in,,top]{\color{textcolor}\rmfamily\fontsize{8.000000}{9.600000}\selectfont 0}%
\end{pgfscope}%
\begin{pgfscope}%
\pgfsetbuttcap%
\pgfsetroundjoin%
\definecolor{currentfill}{rgb}{0.000000,0.000000,0.000000}%
\pgfsetfillcolor{currentfill}%
\pgfsetlinewidth{0.803000pt}%
\definecolor{currentstroke}{rgb}{0.000000,0.000000,0.000000}%
\pgfsetstrokecolor{currentstroke}%
\pgfsetdash{}{0pt}%
\pgfsys@defobject{currentmarker}{\pgfqpoint{0.000000in}{-0.048611in}}{\pgfqpoint{0.000000in}{0.000000in}}{%
\pgfpathmoveto{\pgfqpoint{0.000000in}{0.000000in}}%
\pgfpathlineto{\pgfqpoint{0.000000in}{-0.048611in}}%
\pgfusepath{stroke,fill}%
}%
\begin{pgfscope}%
\pgfsys@transformshift{3.413864in}{0.315889in}%
\pgfsys@useobject{currentmarker}{}%
\end{pgfscope}%
\end{pgfscope}%
\begin{pgfscope}%
\definecolor{textcolor}{rgb}{0.000000,0.000000,0.000000}%
\pgfsetstrokecolor{textcolor}%
\pgfsetfillcolor{textcolor}%
\pgftext[x=3.413864in,y=0.218667in,,top]{\color{textcolor}\rmfamily\fontsize{8.000000}{9.600000}\selectfont 20}%
\end{pgfscope}%
\begin{pgfscope}%
\pgfsetbuttcap%
\pgfsetroundjoin%
\definecolor{currentfill}{rgb}{0.000000,0.000000,0.000000}%
\pgfsetfillcolor{currentfill}%
\pgfsetlinewidth{0.803000pt}%
\definecolor{currentstroke}{rgb}{0.000000,0.000000,0.000000}%
\pgfsetstrokecolor{currentstroke}%
\pgfsetdash{}{0pt}%
\pgfsys@defobject{currentmarker}{\pgfqpoint{0.000000in}{-0.048611in}}{\pgfqpoint{0.000000in}{0.000000in}}{%
\pgfpathmoveto{\pgfqpoint{0.000000in}{0.000000in}}%
\pgfpathlineto{\pgfqpoint{0.000000in}{-0.048611in}}%
\pgfusepath{stroke,fill}%
}%
\begin{pgfscope}%
\pgfsys@transformshift{3.756938in}{0.315889in}%
\pgfsys@useobject{currentmarker}{}%
\end{pgfscope}%
\end{pgfscope}%
\begin{pgfscope}%
\definecolor{textcolor}{rgb}{0.000000,0.000000,0.000000}%
\pgfsetstrokecolor{textcolor}%
\pgfsetfillcolor{textcolor}%
\pgftext[x=3.756938in,y=0.218667in,,top]{\color{textcolor}\rmfamily\fontsize{8.000000}{9.600000}\selectfont 40}%
\end{pgfscope}%
\begin{pgfscope}%
\pgfsetbuttcap%
\pgfsetroundjoin%
\definecolor{currentfill}{rgb}{0.000000,0.000000,0.000000}%
\pgfsetfillcolor{currentfill}%
\pgfsetlinewidth{0.803000pt}%
\definecolor{currentstroke}{rgb}{0.000000,0.000000,0.000000}%
\pgfsetstrokecolor{currentstroke}%
\pgfsetdash{}{0pt}%
\pgfsys@defobject{currentmarker}{\pgfqpoint{0.000000in}{-0.048611in}}{\pgfqpoint{0.000000in}{0.000000in}}{%
\pgfpathmoveto{\pgfqpoint{0.000000in}{0.000000in}}%
\pgfpathlineto{\pgfqpoint{0.000000in}{-0.048611in}}%
\pgfusepath{stroke,fill}%
}%
\begin{pgfscope}%
\pgfsys@transformshift{4.100011in}{0.315889in}%
\pgfsys@useobject{currentmarker}{}%
\end{pgfscope}%
\end{pgfscope}%
\begin{pgfscope}%
\definecolor{textcolor}{rgb}{0.000000,0.000000,0.000000}%
\pgfsetstrokecolor{textcolor}%
\pgfsetfillcolor{textcolor}%
\pgftext[x=4.100011in,y=0.218667in,,top]{\color{textcolor}\rmfamily\fontsize{8.000000}{9.600000}\selectfont 60}%
\end{pgfscope}%
\begin{pgfscope}%
\pgfsetbuttcap%
\pgfsetroundjoin%
\definecolor{currentfill}{rgb}{0.000000,0.000000,0.000000}%
\pgfsetfillcolor{currentfill}%
\pgfsetlinewidth{0.803000pt}%
\definecolor{currentstroke}{rgb}{0.000000,0.000000,0.000000}%
\pgfsetstrokecolor{currentstroke}%
\pgfsetdash{}{0pt}%
\pgfsys@defobject{currentmarker}{\pgfqpoint{0.000000in}{-0.048611in}}{\pgfqpoint{0.000000in}{0.000000in}}{%
\pgfpathmoveto{\pgfqpoint{0.000000in}{0.000000in}}%
\pgfpathlineto{\pgfqpoint{0.000000in}{-0.048611in}}%
\pgfusepath{stroke,fill}%
}%
\begin{pgfscope}%
\pgfsys@transformshift{4.443085in}{0.315889in}%
\pgfsys@useobject{currentmarker}{}%
\end{pgfscope}%
\end{pgfscope}%
\begin{pgfscope}%
\definecolor{textcolor}{rgb}{0.000000,0.000000,0.000000}%
\pgfsetstrokecolor{textcolor}%
\pgfsetfillcolor{textcolor}%
\pgftext[x=4.443085in,y=0.218667in,,top]{\color{textcolor}\rmfamily\fontsize{8.000000}{9.600000}\selectfont 80}%
\end{pgfscope}%
\begin{pgfscope}%
\pgfsetbuttcap%
\pgfsetroundjoin%
\definecolor{currentfill}{rgb}{0.000000,0.000000,0.000000}%
\pgfsetfillcolor{currentfill}%
\pgfsetlinewidth{0.803000pt}%
\definecolor{currentstroke}{rgb}{0.000000,0.000000,0.000000}%
\pgfsetstrokecolor{currentstroke}%
\pgfsetdash{}{0pt}%
\pgfsys@defobject{currentmarker}{\pgfqpoint{0.000000in}{-0.048611in}}{\pgfqpoint{0.000000in}{0.000000in}}{%
\pgfpathmoveto{\pgfqpoint{0.000000in}{0.000000in}}%
\pgfpathlineto{\pgfqpoint{0.000000in}{-0.048611in}}%
\pgfusepath{stroke,fill}%
}%
\begin{pgfscope}%
\pgfsys@transformshift{4.786158in}{0.315889in}%
\pgfsys@useobject{currentmarker}{}%
\end{pgfscope}%
\end{pgfscope}%
\begin{pgfscope}%
\definecolor{textcolor}{rgb}{0.000000,0.000000,0.000000}%
\pgfsetstrokecolor{textcolor}%
\pgfsetfillcolor{textcolor}%
\pgftext[x=4.786158in,y=0.218667in,,top]{\color{textcolor}\rmfamily\fontsize{8.000000}{9.600000}\selectfont 100}%
\end{pgfscope}%
\begin{pgfscope}%
\pgfsetbuttcap%
\pgfsetroundjoin%
\definecolor{currentfill}{rgb}{0.000000,0.000000,0.000000}%
\pgfsetfillcolor{currentfill}%
\pgfsetlinewidth{0.803000pt}%
\definecolor{currentstroke}{rgb}{0.000000,0.000000,0.000000}%
\pgfsetstrokecolor{currentstroke}%
\pgfsetdash{}{0pt}%
\pgfsys@defobject{currentmarker}{\pgfqpoint{0.000000in}{-0.048611in}}{\pgfqpoint{0.000000in}{0.000000in}}{%
\pgfpathmoveto{\pgfqpoint{0.000000in}{0.000000in}}%
\pgfpathlineto{\pgfqpoint{0.000000in}{-0.048611in}}%
\pgfusepath{stroke,fill}%
}%
\begin{pgfscope}%
\pgfsys@transformshift{5.129232in}{0.315889in}%
\pgfsys@useobject{currentmarker}{}%
\end{pgfscope}%
\end{pgfscope}%
\begin{pgfscope}%
\definecolor{textcolor}{rgb}{0.000000,0.000000,0.000000}%
\pgfsetstrokecolor{textcolor}%
\pgfsetfillcolor{textcolor}%
\pgftext[x=5.129232in,y=0.218667in,,top]{\color{textcolor}\rmfamily\fontsize{8.000000}{9.600000}\selectfont 120}%
\end{pgfscope}%
\begin{pgfscope}%
\pgfsetbuttcap%
\pgfsetroundjoin%
\definecolor{currentfill}{rgb}{0.000000,0.000000,0.000000}%
\pgfsetfillcolor{currentfill}%
\pgfsetlinewidth{0.803000pt}%
\definecolor{currentstroke}{rgb}{0.000000,0.000000,0.000000}%
\pgfsetstrokecolor{currentstroke}%
\pgfsetdash{}{0pt}%
\pgfsys@defobject{currentmarker}{\pgfqpoint{0.000000in}{0.000000in}}{\pgfqpoint{0.048611in}{0.000000in}}{%
\pgfpathmoveto{\pgfqpoint{0.000000in}{0.000000in}}%
\pgfpathlineto{\pgfqpoint{0.048611in}{0.000000in}}%
\pgfusepath{stroke,fill}%
}%
\begin{pgfscope}%
\pgfsys@transformshift{5.340222in}{0.448931in}%
\pgfsys@useobject{currentmarker}{}%
\end{pgfscope}%
\end{pgfscope}%
\begin{pgfscope}%
\definecolor{textcolor}{rgb}{0.000000,0.000000,0.000000}%
\pgfsetstrokecolor{textcolor}%
\pgfsetfillcolor{textcolor}%
\pgftext[x=5.437444in,y=0.410375in,left,base]{\color{textcolor}\rmfamily\fontsize{8.000000}{9.600000}\selectfont −1.6}%
\end{pgfscope}%
\begin{pgfscope}%
\pgfsetbuttcap%
\pgfsetroundjoin%
\definecolor{currentfill}{rgb}{0.000000,0.000000,0.000000}%
\pgfsetfillcolor{currentfill}%
\pgfsetlinewidth{0.803000pt}%
\definecolor{currentstroke}{rgb}{0.000000,0.000000,0.000000}%
\pgfsetstrokecolor{currentstroke}%
\pgfsetdash{}{0pt}%
\pgfsys@defobject{currentmarker}{\pgfqpoint{0.000000in}{0.000000in}}{\pgfqpoint{0.048611in}{0.000000in}}{%
\pgfpathmoveto{\pgfqpoint{0.000000in}{0.000000in}}%
\pgfpathlineto{\pgfqpoint{0.048611in}{0.000000in}}%
\pgfusepath{stroke,fill}%
}%
\begin{pgfscope}%
\pgfsys@transformshift{5.340222in}{0.717262in}%
\pgfsys@useobject{currentmarker}{}%
\end{pgfscope}%
\end{pgfscope}%
\begin{pgfscope}%
\definecolor{textcolor}{rgb}{0.000000,0.000000,0.000000}%
\pgfsetstrokecolor{textcolor}%
\pgfsetfillcolor{textcolor}%
\pgftext[x=5.437444in,y=0.678706in,left,base]{\color{textcolor}\rmfamily\fontsize{8.000000}{9.600000}\selectfont −1.4}%
\end{pgfscope}%
\begin{pgfscope}%
\pgfsetbuttcap%
\pgfsetroundjoin%
\definecolor{currentfill}{rgb}{0.000000,0.000000,0.000000}%
\pgfsetfillcolor{currentfill}%
\pgfsetlinewidth{0.803000pt}%
\definecolor{currentstroke}{rgb}{0.000000,0.000000,0.000000}%
\pgfsetstrokecolor{currentstroke}%
\pgfsetdash{}{0pt}%
\pgfsys@defobject{currentmarker}{\pgfqpoint{0.000000in}{0.000000in}}{\pgfqpoint{0.048611in}{0.000000in}}{%
\pgfpathmoveto{\pgfqpoint{0.000000in}{0.000000in}}%
\pgfpathlineto{\pgfqpoint{0.048611in}{0.000000in}}%
\pgfusepath{stroke,fill}%
}%
\begin{pgfscope}%
\pgfsys@transformshift{5.340222in}{0.985592in}%
\pgfsys@useobject{currentmarker}{}%
\end{pgfscope}%
\end{pgfscope}%
\begin{pgfscope}%
\definecolor{textcolor}{rgb}{0.000000,0.000000,0.000000}%
\pgfsetstrokecolor{textcolor}%
\pgfsetfillcolor{textcolor}%
\pgftext[x=5.437444in,y=0.947037in,left,base]{\color{textcolor}\rmfamily\fontsize{8.000000}{9.600000}\selectfont −1.2}%
\end{pgfscope}%
\begin{pgfscope}%
\pgfsetbuttcap%
\pgfsetroundjoin%
\definecolor{currentfill}{rgb}{0.000000,0.000000,0.000000}%
\pgfsetfillcolor{currentfill}%
\pgfsetlinewidth{0.803000pt}%
\definecolor{currentstroke}{rgb}{0.000000,0.000000,0.000000}%
\pgfsetstrokecolor{currentstroke}%
\pgfsetdash{}{0pt}%
\pgfsys@defobject{currentmarker}{\pgfqpoint{0.000000in}{0.000000in}}{\pgfqpoint{0.048611in}{0.000000in}}{%
\pgfpathmoveto{\pgfqpoint{0.000000in}{0.000000in}}%
\pgfpathlineto{\pgfqpoint{0.048611in}{0.000000in}}%
\pgfusepath{stroke,fill}%
}%
\begin{pgfscope}%
\pgfsys@transformshift{5.340222in}{1.253923in}%
\pgfsys@useobject{currentmarker}{}%
\end{pgfscope}%
\end{pgfscope}%
\begin{pgfscope}%
\definecolor{textcolor}{rgb}{0.000000,0.000000,0.000000}%
\pgfsetstrokecolor{textcolor}%
\pgfsetfillcolor{textcolor}%
\pgftext[x=5.437444in,y=1.215368in,left,base]{\color{textcolor}\rmfamily\fontsize{8.000000}{9.600000}\selectfont −1.0}%
\end{pgfscope}%
\begin{pgfscope}%
\pgfsetbuttcap%
\pgfsetroundjoin%
\definecolor{currentfill}{rgb}{0.000000,0.000000,0.000000}%
\pgfsetfillcolor{currentfill}%
\pgfsetlinewidth{0.803000pt}%
\definecolor{currentstroke}{rgb}{0.000000,0.000000,0.000000}%
\pgfsetstrokecolor{currentstroke}%
\pgfsetdash{}{0pt}%
\pgfsys@defobject{currentmarker}{\pgfqpoint{0.000000in}{0.000000in}}{\pgfqpoint{0.048611in}{0.000000in}}{%
\pgfpathmoveto{\pgfqpoint{0.000000in}{0.000000in}}%
\pgfpathlineto{\pgfqpoint{0.048611in}{0.000000in}}%
\pgfusepath{stroke,fill}%
}%
\begin{pgfscope}%
\pgfsys@transformshift{5.340222in}{1.522254in}%
\pgfsys@useobject{currentmarker}{}%
\end{pgfscope}%
\end{pgfscope}%
\begin{pgfscope}%
\definecolor{textcolor}{rgb}{0.000000,0.000000,0.000000}%
\pgfsetstrokecolor{textcolor}%
\pgfsetfillcolor{textcolor}%
\pgftext[x=5.437444in,y=1.483698in,left,base]{\color{textcolor}\rmfamily\fontsize{8.000000}{9.600000}\selectfont −0.8}%
\end{pgfscope}%
\begin{pgfscope}%
\pgfsetbuttcap%
\pgfsetroundjoin%
\definecolor{currentfill}{rgb}{0.000000,0.000000,0.000000}%
\pgfsetfillcolor{currentfill}%
\pgfsetlinewidth{0.803000pt}%
\definecolor{currentstroke}{rgb}{0.000000,0.000000,0.000000}%
\pgfsetstrokecolor{currentstroke}%
\pgfsetdash{}{0pt}%
\pgfsys@defobject{currentmarker}{\pgfqpoint{0.000000in}{0.000000in}}{\pgfqpoint{0.048611in}{0.000000in}}{%
\pgfpathmoveto{\pgfqpoint{0.000000in}{0.000000in}}%
\pgfpathlineto{\pgfqpoint{0.048611in}{0.000000in}}%
\pgfusepath{stroke,fill}%
}%
\begin{pgfscope}%
\pgfsys@transformshift{5.340222in}{1.790585in}%
\pgfsys@useobject{currentmarker}{}%
\end{pgfscope}%
\end{pgfscope}%
\begin{pgfscope}%
\definecolor{textcolor}{rgb}{0.000000,0.000000,0.000000}%
\pgfsetstrokecolor{textcolor}%
\pgfsetfillcolor{textcolor}%
\pgftext[x=5.437444in,y=1.752029in,left,base]{\color{textcolor}\rmfamily\fontsize{8.000000}{9.600000}\selectfont −0.6}%
\end{pgfscope}%
\begin{pgfscope}%
\pgfsetbuttcap%
\pgfsetroundjoin%
\definecolor{currentfill}{rgb}{0.000000,0.000000,0.000000}%
\pgfsetfillcolor{currentfill}%
\pgfsetlinewidth{0.803000pt}%
\definecolor{currentstroke}{rgb}{0.000000,0.000000,0.000000}%
\pgfsetstrokecolor{currentstroke}%
\pgfsetdash{}{0pt}%
\pgfsys@defobject{currentmarker}{\pgfqpoint{0.000000in}{0.000000in}}{\pgfqpoint{0.048611in}{0.000000in}}{%
\pgfpathmoveto{\pgfqpoint{0.000000in}{0.000000in}}%
\pgfpathlineto{\pgfqpoint{0.048611in}{0.000000in}}%
\pgfusepath{stroke,fill}%
}%
\begin{pgfscope}%
\pgfsys@transformshift{5.340222in}{2.058915in}%
\pgfsys@useobject{currentmarker}{}%
\end{pgfscope}%
\end{pgfscope}%
\begin{pgfscope}%
\definecolor{textcolor}{rgb}{0.000000,0.000000,0.000000}%
\pgfsetstrokecolor{textcolor}%
\pgfsetfillcolor{textcolor}%
\pgftext[x=5.437444in,y=2.020360in,left,base]{\color{textcolor}\rmfamily\fontsize{8.000000}{9.600000}\selectfont −0.4}%
\end{pgfscope}%
\begin{pgfscope}%
\pgfsetbuttcap%
\pgfsetroundjoin%
\definecolor{currentfill}{rgb}{0.000000,0.000000,0.000000}%
\pgfsetfillcolor{currentfill}%
\pgfsetlinewidth{0.803000pt}%
\definecolor{currentstroke}{rgb}{0.000000,0.000000,0.000000}%
\pgfsetstrokecolor{currentstroke}%
\pgfsetdash{}{0pt}%
\pgfsys@defobject{currentmarker}{\pgfqpoint{0.000000in}{0.000000in}}{\pgfqpoint{0.048611in}{0.000000in}}{%
\pgfpathmoveto{\pgfqpoint{0.000000in}{0.000000in}}%
\pgfpathlineto{\pgfqpoint{0.048611in}{0.000000in}}%
\pgfusepath{stroke,fill}%
}%
\begin{pgfscope}%
\pgfsys@transformshift{5.340222in}{2.327246in}%
\pgfsys@useobject{currentmarker}{}%
\end{pgfscope}%
\end{pgfscope}%
\begin{pgfscope}%
\definecolor{textcolor}{rgb}{0.000000,0.000000,0.000000}%
\pgfsetstrokecolor{textcolor}%
\pgfsetfillcolor{textcolor}%
\pgftext[x=5.437444in,y=2.288690in,left,base]{\color{textcolor}\rmfamily\fontsize{8.000000}{9.600000}\selectfont −0.2}%
\end{pgfscope}%
\begin{pgfscope}%
\pgfsetbuttcap%
\pgfsetroundjoin%
\definecolor{currentfill}{rgb}{0.000000,0.000000,0.000000}%
\pgfsetfillcolor{currentfill}%
\pgfsetlinewidth{0.803000pt}%
\definecolor{currentstroke}{rgb}{0.000000,0.000000,0.000000}%
\pgfsetstrokecolor{currentstroke}%
\pgfsetdash{}{0pt}%
\pgfsys@defobject{currentmarker}{\pgfqpoint{0.000000in}{0.000000in}}{\pgfqpoint{0.048611in}{0.000000in}}{%
\pgfpathmoveto{\pgfqpoint{0.000000in}{0.000000in}}%
\pgfpathlineto{\pgfqpoint{0.048611in}{0.000000in}}%
\pgfusepath{stroke,fill}%
}%
\begin{pgfscope}%
\pgfsys@transformshift{5.340222in}{2.595577in}%
\pgfsys@useobject{currentmarker}{}%
\end{pgfscope}%
\end{pgfscope}%
\begin{pgfscope}%
\definecolor{textcolor}{rgb}{0.000000,0.000000,0.000000}%
\pgfsetstrokecolor{textcolor}%
\pgfsetfillcolor{textcolor}%
\pgftext[x=5.437444in,y=2.557021in,left,base]{\color{textcolor}\rmfamily\fontsize{8.000000}{9.600000}\selectfont 0.0}%
\end{pgfscope}%
\begin{pgfscope}%
\definecolor{textcolor}{rgb}{0.000000,0.000000,0.000000}%
\pgfsetstrokecolor{textcolor}%
\pgfsetfillcolor{textcolor}%
\pgftext[x=5.340222in,y=2.745800in,right,base]{\color{textcolor}\rmfamily\fontsize{8.000000}{9.600000}\selectfont 1e−6}%
\end{pgfscope}%
\begin{pgfscope}%
\pgfpathrectangle{\pgfqpoint{2.962722in}{0.315889in}}{\pgfqpoint{2.377500in}{2.388245in}}%
\pgfusepath{clip}%
\pgfsetrectcap%
\pgfsetroundjoin%
\pgfsetlinewidth{1.505625pt}%
\definecolor{currentstroke}{rgb}{0.121569,0.466667,0.705882}%
\pgfsetstrokecolor{currentstroke}%
\pgfsetdash{}{0pt}%
\pgfpathmoveto{\pgfqpoint{3.070790in}{2.595577in}}%
\pgfpathlineto{\pgfqpoint{3.087944in}{2.595577in}}%
\pgfpathlineto{\pgfqpoint{3.105098in}{2.595577in}}%
\pgfpathlineto{\pgfqpoint{3.122251in}{2.595577in}}%
\pgfpathlineto{\pgfqpoint{3.139405in}{2.595577in}}%
\pgfpathlineto{\pgfqpoint{3.156559in}{2.595577in}}%
\pgfpathlineto{\pgfqpoint{3.173712in}{2.595577in}}%
\pgfpathlineto{\pgfqpoint{3.190866in}{2.595577in}}%
\pgfpathlineto{\pgfqpoint{3.208020in}{2.595577in}}%
\pgfpathlineto{\pgfqpoint{3.225174in}{2.595577in}}%
\pgfpathlineto{\pgfqpoint{3.242327in}{2.595577in}}%
\pgfpathlineto{\pgfqpoint{3.259481in}{2.595577in}}%
\pgfpathlineto{\pgfqpoint{3.276635in}{2.595577in}}%
\pgfpathlineto{\pgfqpoint{3.293788in}{2.595577in}}%
\pgfpathlineto{\pgfqpoint{3.310942in}{2.595577in}}%
\pgfpathlineto{\pgfqpoint{3.328096in}{2.595577in}}%
\pgfpathlineto{\pgfqpoint{3.345249in}{2.595577in}}%
\pgfpathlineto{\pgfqpoint{3.362403in}{2.595577in}}%
\pgfpathlineto{\pgfqpoint{3.379557in}{2.595577in}}%
\pgfpathlineto{\pgfqpoint{3.396710in}{2.595577in}}%
\pgfpathlineto{\pgfqpoint{3.413864in}{2.595577in}}%
\pgfpathlineto{\pgfqpoint{3.431018in}{2.595577in}}%
\pgfpathlineto{\pgfqpoint{3.448171in}{2.595577in}}%
\pgfpathlineto{\pgfqpoint{3.465325in}{2.595577in}}%
\pgfpathlineto{\pgfqpoint{3.482479in}{2.595577in}}%
\pgfpathlineto{\pgfqpoint{3.499632in}{2.595577in}}%
\pgfpathlineto{\pgfqpoint{3.516786in}{2.595577in}}%
\pgfpathlineto{\pgfqpoint{3.533940in}{2.595577in}}%
\pgfpathlineto{\pgfqpoint{3.551093in}{2.595577in}}%
\pgfpathlineto{\pgfqpoint{3.568247in}{2.595577in}}%
\pgfpathlineto{\pgfqpoint{3.585401in}{2.595577in}}%
\pgfpathlineto{\pgfqpoint{3.602554in}{2.595577in}}%
\pgfpathlineto{\pgfqpoint{3.619708in}{2.595577in}}%
\pgfpathlineto{\pgfqpoint{3.636862in}{2.595577in}}%
\pgfpathlineto{\pgfqpoint{3.654016in}{2.595577in}}%
\pgfpathlineto{\pgfqpoint{3.671169in}{2.595577in}}%
\pgfpathlineto{\pgfqpoint{3.688323in}{2.595577in}}%
\pgfpathlineto{\pgfqpoint{3.705477in}{2.595577in}}%
\pgfpathlineto{\pgfqpoint{3.722630in}{2.595577in}}%
\pgfpathlineto{\pgfqpoint{3.739784in}{2.595577in}}%
\pgfpathlineto{\pgfqpoint{3.756938in}{2.595577in}}%
\pgfpathlineto{\pgfqpoint{3.774091in}{2.595577in}}%
\pgfpathlineto{\pgfqpoint{3.791245in}{2.595577in}}%
\pgfpathlineto{\pgfqpoint{3.808399in}{2.595577in}}%
\pgfpathlineto{\pgfqpoint{3.825552in}{2.595577in}}%
\pgfpathlineto{\pgfqpoint{3.842706in}{2.595577in}}%
\pgfpathlineto{\pgfqpoint{3.859860in}{2.595577in}}%
\pgfpathlineto{\pgfqpoint{3.877013in}{2.595577in}}%
\pgfpathlineto{\pgfqpoint{3.894167in}{2.595577in}}%
\pgfpathlineto{\pgfqpoint{3.911321in}{2.595577in}}%
\pgfpathlineto{\pgfqpoint{3.928474in}{2.595577in}}%
\pgfpathlineto{\pgfqpoint{3.945628in}{2.595577in}}%
\pgfpathlineto{\pgfqpoint{3.962782in}{2.595577in}}%
\pgfpathlineto{\pgfqpoint{3.979935in}{2.595577in}}%
\pgfpathlineto{\pgfqpoint{3.997089in}{2.595577in}}%
\pgfpathlineto{\pgfqpoint{4.014243in}{2.595577in}}%
\pgfpathlineto{\pgfqpoint{4.031396in}{2.595577in}}%
\pgfpathlineto{\pgfqpoint{4.048550in}{2.595577in}}%
\pgfpathlineto{\pgfqpoint{4.065704in}{2.595577in}}%
\pgfpathlineto{\pgfqpoint{4.082858in}{2.595577in}}%
\pgfpathlineto{\pgfqpoint{4.100011in}{2.595577in}}%
\pgfpathlineto{\pgfqpoint{4.117165in}{2.595577in}}%
\pgfpathlineto{\pgfqpoint{4.134319in}{2.595577in}}%
\pgfpathlineto{\pgfqpoint{4.151472in}{2.595577in}}%
\pgfpathlineto{\pgfqpoint{4.168626in}{2.595577in}}%
\pgfpathlineto{\pgfqpoint{4.185780in}{2.595577in}}%
\pgfpathlineto{\pgfqpoint{4.202933in}{2.595577in}}%
\pgfpathlineto{\pgfqpoint{4.220087in}{2.595577in}}%
\pgfpathlineto{\pgfqpoint{4.237241in}{2.595577in}}%
\pgfpathlineto{\pgfqpoint{4.254394in}{2.595577in}}%
\pgfpathlineto{\pgfqpoint{4.271548in}{2.595577in}}%
\pgfpathlineto{\pgfqpoint{4.288702in}{2.595577in}}%
\pgfpathlineto{\pgfqpoint{4.305855in}{2.595577in}}%
\pgfpathlineto{\pgfqpoint{4.323009in}{2.595577in}}%
\pgfpathlineto{\pgfqpoint{4.340163in}{2.595577in}}%
\pgfpathlineto{\pgfqpoint{4.357316in}{2.595577in}}%
\pgfpathlineto{\pgfqpoint{4.374470in}{2.595577in}}%
\pgfpathlineto{\pgfqpoint{4.391624in}{2.595577in}}%
\pgfpathlineto{\pgfqpoint{4.408777in}{2.595577in}}%
\pgfpathlineto{\pgfqpoint{4.425931in}{2.595577in}}%
\pgfpathlineto{\pgfqpoint{4.443085in}{2.595577in}}%
\pgfpathlineto{\pgfqpoint{4.460238in}{2.595577in}}%
\pgfpathlineto{\pgfqpoint{4.477392in}{2.595577in}}%
\pgfpathlineto{\pgfqpoint{4.494546in}{2.595577in}}%
\pgfpathlineto{\pgfqpoint{4.511699in}{2.595577in}}%
\pgfpathlineto{\pgfqpoint{4.528853in}{2.595577in}}%
\pgfpathlineto{\pgfqpoint{4.546007in}{2.595577in}}%
\pgfpathlineto{\pgfqpoint{4.563161in}{2.595577in}}%
\pgfpathlineto{\pgfqpoint{4.580314in}{2.595577in}}%
\pgfpathlineto{\pgfqpoint{4.597468in}{2.595577in}}%
\pgfpathlineto{\pgfqpoint{4.614622in}{2.595577in}}%
\pgfpathlineto{\pgfqpoint{4.631775in}{2.595577in}}%
\pgfpathlineto{\pgfqpoint{4.648929in}{2.595577in}}%
\pgfpathlineto{\pgfqpoint{4.666083in}{2.595577in}}%
\pgfpathlineto{\pgfqpoint{4.683236in}{2.595577in}}%
\pgfpathlineto{\pgfqpoint{4.700390in}{2.595577in}}%
\pgfpathlineto{\pgfqpoint{4.717544in}{2.595577in}}%
\pgfpathlineto{\pgfqpoint{4.734697in}{2.595577in}}%
\pgfpathlineto{\pgfqpoint{4.751851in}{2.595577in}}%
\pgfpathlineto{\pgfqpoint{4.769005in}{2.595577in}}%
\pgfpathlineto{\pgfqpoint{4.786158in}{2.595577in}}%
\pgfpathlineto{\pgfqpoint{4.803312in}{2.595577in}}%
\pgfpathlineto{\pgfqpoint{4.820466in}{2.595577in}}%
\pgfpathlineto{\pgfqpoint{4.837619in}{2.595577in}}%
\pgfpathlineto{\pgfqpoint{4.854773in}{2.595577in}}%
\pgfpathlineto{\pgfqpoint{4.871927in}{2.595577in}}%
\pgfpathlineto{\pgfqpoint{4.889080in}{2.595577in}}%
\pgfpathlineto{\pgfqpoint{4.906234in}{2.595577in}}%
\pgfpathlineto{\pgfqpoint{4.923388in}{2.595577in}}%
\pgfpathlineto{\pgfqpoint{4.940541in}{2.595577in}}%
\pgfpathlineto{\pgfqpoint{4.957695in}{2.595577in}}%
\pgfpathlineto{\pgfqpoint{4.974849in}{2.595577in}}%
\pgfpathlineto{\pgfqpoint{4.992003in}{2.595577in}}%
\pgfpathlineto{\pgfqpoint{5.009156in}{2.595577in}}%
\pgfpathlineto{\pgfqpoint{5.026310in}{2.595577in}}%
\pgfpathlineto{\pgfqpoint{5.043464in}{2.595577in}}%
\pgfpathlineto{\pgfqpoint{5.060617in}{2.595577in}}%
\pgfpathlineto{\pgfqpoint{5.077771in}{2.595577in}}%
\pgfpathlineto{\pgfqpoint{5.094925in}{2.595577in}}%
\pgfpathlineto{\pgfqpoint{5.112078in}{2.595577in}}%
\pgfpathlineto{\pgfqpoint{5.129232in}{2.595577in}}%
\pgfpathlineto{\pgfqpoint{5.146386in}{2.595577in}}%
\pgfpathlineto{\pgfqpoint{5.163539in}{2.595577in}}%
\pgfpathlineto{\pgfqpoint{5.180693in}{2.595577in}}%
\pgfpathlineto{\pgfqpoint{5.197847in}{2.595577in}}%
\pgfpathlineto{\pgfqpoint{5.215000in}{2.595577in}}%
\pgfusepath{stroke}%
\end{pgfscope}%
\begin{pgfscope}%
\pgfpathrectangle{\pgfqpoint{2.962722in}{0.315889in}}{\pgfqpoint{2.377500in}{2.388245in}}%
\pgfusepath{clip}%
\pgfsetrectcap%
\pgfsetroundjoin%
\pgfsetlinewidth{1.505625pt}%
\definecolor{currentstroke}{rgb}{1.000000,0.498039,0.054902}%
\pgfsetstrokecolor{currentstroke}%
\pgfsetdash{}{0pt}%
\pgfpathmoveto{\pgfqpoint{3.070790in}{2.595577in}}%
\pgfpathlineto{\pgfqpoint{3.087944in}{2.595577in}}%
\pgfpathlineto{\pgfqpoint{3.105098in}{2.595577in}}%
\pgfpathlineto{\pgfqpoint{3.122251in}{2.595577in}}%
\pgfpathlineto{\pgfqpoint{3.139405in}{2.595577in}}%
\pgfpathlineto{\pgfqpoint{3.156559in}{2.595577in}}%
\pgfpathlineto{\pgfqpoint{3.173712in}{2.595577in}}%
\pgfpathlineto{\pgfqpoint{3.190866in}{2.595577in}}%
\pgfpathlineto{\pgfqpoint{3.208020in}{2.595577in}}%
\pgfpathlineto{\pgfqpoint{3.225174in}{2.595577in}}%
\pgfpathlineto{\pgfqpoint{3.242327in}{2.595577in}}%
\pgfpathlineto{\pgfqpoint{3.259481in}{2.595577in}}%
\pgfpathlineto{\pgfqpoint{3.276635in}{2.595577in}}%
\pgfpathlineto{\pgfqpoint{3.293788in}{2.595577in}}%
\pgfpathlineto{\pgfqpoint{3.310942in}{2.595577in}}%
\pgfpathlineto{\pgfqpoint{3.328096in}{2.595577in}}%
\pgfpathlineto{\pgfqpoint{3.345249in}{2.595577in}}%
\pgfpathlineto{\pgfqpoint{3.362403in}{2.595577in}}%
\pgfpathlineto{\pgfqpoint{3.379557in}{2.595577in}}%
\pgfpathlineto{\pgfqpoint{3.396710in}{2.595577in}}%
\pgfpathlineto{\pgfqpoint{3.413864in}{2.595577in}}%
\pgfpathlineto{\pgfqpoint{3.431018in}{2.595577in}}%
\pgfpathlineto{\pgfqpoint{3.448171in}{2.595577in}}%
\pgfpathlineto{\pgfqpoint{3.465325in}{2.595577in}}%
\pgfpathlineto{\pgfqpoint{3.482479in}{2.595577in}}%
\pgfpathlineto{\pgfqpoint{3.499632in}{2.595577in}}%
\pgfpathlineto{\pgfqpoint{3.516786in}{2.595577in}}%
\pgfpathlineto{\pgfqpoint{3.533940in}{2.595577in}}%
\pgfpathlineto{\pgfqpoint{3.551093in}{2.595577in}}%
\pgfpathlineto{\pgfqpoint{3.568247in}{2.595577in}}%
\pgfpathlineto{\pgfqpoint{3.585401in}{2.595577in}}%
\pgfpathlineto{\pgfqpoint{3.602554in}{2.595577in}}%
\pgfpathlineto{\pgfqpoint{3.619708in}{2.595577in}}%
\pgfpathlineto{\pgfqpoint{3.636862in}{2.595577in}}%
\pgfpathlineto{\pgfqpoint{3.654016in}{2.595577in}}%
\pgfpathlineto{\pgfqpoint{3.671169in}{2.595577in}}%
\pgfpathlineto{\pgfqpoint{3.688323in}{2.595577in}}%
\pgfpathlineto{\pgfqpoint{3.705477in}{2.595577in}}%
\pgfpathlineto{\pgfqpoint{3.722630in}{2.595577in}}%
\pgfpathlineto{\pgfqpoint{3.739784in}{2.595577in}}%
\pgfpathlineto{\pgfqpoint{3.756938in}{2.595577in}}%
\pgfpathlineto{\pgfqpoint{3.774091in}{2.595577in}}%
\pgfpathlineto{\pgfqpoint{3.791245in}{2.595577in}}%
\pgfpathlineto{\pgfqpoint{3.808399in}{2.595577in}}%
\pgfpathlineto{\pgfqpoint{3.825552in}{2.595577in}}%
\pgfpathlineto{\pgfqpoint{3.842706in}{2.595577in}}%
\pgfpathlineto{\pgfqpoint{3.859860in}{2.595577in}}%
\pgfpathlineto{\pgfqpoint{3.877013in}{2.595577in}}%
\pgfpathlineto{\pgfqpoint{3.894167in}{2.595577in}}%
\pgfpathlineto{\pgfqpoint{3.911321in}{2.595577in}}%
\pgfpathlineto{\pgfqpoint{3.928474in}{2.595577in}}%
\pgfpathlineto{\pgfqpoint{3.945628in}{2.595577in}}%
\pgfpathlineto{\pgfqpoint{3.962782in}{2.595577in}}%
\pgfpathlineto{\pgfqpoint{3.979935in}{2.595577in}}%
\pgfpathlineto{\pgfqpoint{3.997089in}{2.595577in}}%
\pgfpathlineto{\pgfqpoint{4.014243in}{2.595577in}}%
\pgfpathlineto{\pgfqpoint{4.031396in}{2.595577in}}%
\pgfpathlineto{\pgfqpoint{4.048550in}{2.595577in}}%
\pgfpathlineto{\pgfqpoint{4.065704in}{2.595577in}}%
\pgfpathlineto{\pgfqpoint{4.082858in}{2.595577in}}%
\pgfpathlineto{\pgfqpoint{4.100011in}{2.595577in}}%
\pgfpathlineto{\pgfqpoint{4.117165in}{2.595577in}}%
\pgfpathlineto{\pgfqpoint{4.134319in}{2.595577in}}%
\pgfpathlineto{\pgfqpoint{4.151472in}{2.595577in}}%
\pgfpathlineto{\pgfqpoint{4.168626in}{2.595577in}}%
\pgfpathlineto{\pgfqpoint{4.185780in}{2.595577in}}%
\pgfpathlineto{\pgfqpoint{4.202933in}{2.595577in}}%
\pgfpathlineto{\pgfqpoint{4.220087in}{2.595577in}}%
\pgfpathlineto{\pgfqpoint{4.237241in}{2.595577in}}%
\pgfpathlineto{\pgfqpoint{4.254394in}{2.595577in}}%
\pgfpathlineto{\pgfqpoint{4.271548in}{2.595577in}}%
\pgfpathlineto{\pgfqpoint{4.288702in}{2.595577in}}%
\pgfpathlineto{\pgfqpoint{4.305855in}{2.595577in}}%
\pgfpathlineto{\pgfqpoint{4.323009in}{2.595577in}}%
\pgfpathlineto{\pgfqpoint{4.340163in}{2.595577in}}%
\pgfpathlineto{\pgfqpoint{4.357316in}{2.595577in}}%
\pgfpathlineto{\pgfqpoint{4.374470in}{2.595577in}}%
\pgfpathlineto{\pgfqpoint{4.391624in}{2.595577in}}%
\pgfpathlineto{\pgfqpoint{4.408777in}{2.595577in}}%
\pgfpathlineto{\pgfqpoint{4.425931in}{2.595577in}}%
\pgfpathlineto{\pgfqpoint{4.443085in}{2.595577in}}%
\pgfpathlineto{\pgfqpoint{4.460238in}{2.595577in}}%
\pgfpathlineto{\pgfqpoint{4.477392in}{2.595577in}}%
\pgfpathlineto{\pgfqpoint{4.494546in}{2.595577in}}%
\pgfpathlineto{\pgfqpoint{4.511699in}{2.595577in}}%
\pgfpathlineto{\pgfqpoint{4.528853in}{2.595577in}}%
\pgfpathlineto{\pgfqpoint{4.546007in}{2.595577in}}%
\pgfpathlineto{\pgfqpoint{4.563161in}{2.595577in}}%
\pgfpathlineto{\pgfqpoint{4.580314in}{2.595577in}}%
\pgfpathlineto{\pgfqpoint{4.597468in}{2.595577in}}%
\pgfpathlineto{\pgfqpoint{4.614622in}{2.595577in}}%
\pgfpathlineto{\pgfqpoint{4.631775in}{2.595577in}}%
\pgfpathlineto{\pgfqpoint{4.648929in}{2.595577in}}%
\pgfpathlineto{\pgfqpoint{4.666083in}{2.595577in}}%
\pgfpathlineto{\pgfqpoint{4.683236in}{2.595577in}}%
\pgfpathlineto{\pgfqpoint{4.700390in}{2.595577in}}%
\pgfpathlineto{\pgfqpoint{4.717544in}{2.595577in}}%
\pgfpathlineto{\pgfqpoint{4.734697in}{2.595577in}}%
\pgfpathlineto{\pgfqpoint{4.751851in}{2.595577in}}%
\pgfpathlineto{\pgfqpoint{4.769005in}{2.595577in}}%
\pgfpathlineto{\pgfqpoint{4.786158in}{2.595577in}}%
\pgfpathlineto{\pgfqpoint{4.803312in}{2.595577in}}%
\pgfpathlineto{\pgfqpoint{4.820466in}{2.595577in}}%
\pgfpathlineto{\pgfqpoint{4.837619in}{2.595577in}}%
\pgfpathlineto{\pgfqpoint{4.854773in}{2.595577in}}%
\pgfpathlineto{\pgfqpoint{4.871927in}{2.595577in}}%
\pgfpathlineto{\pgfqpoint{4.889080in}{2.595577in}}%
\pgfpathlineto{\pgfqpoint{4.906234in}{2.595577in}}%
\pgfpathlineto{\pgfqpoint{4.923388in}{2.595577in}}%
\pgfpathlineto{\pgfqpoint{4.940541in}{2.595577in}}%
\pgfpathlineto{\pgfqpoint{4.957695in}{2.595577in}}%
\pgfpathlineto{\pgfqpoint{4.974849in}{2.595577in}}%
\pgfpathlineto{\pgfqpoint{4.992003in}{2.595577in}}%
\pgfpathlineto{\pgfqpoint{5.009156in}{2.595577in}}%
\pgfpathlineto{\pgfqpoint{5.026310in}{2.595577in}}%
\pgfpathlineto{\pgfqpoint{5.043464in}{2.595577in}}%
\pgfpathlineto{\pgfqpoint{5.060617in}{2.595577in}}%
\pgfpathlineto{\pgfqpoint{5.077771in}{2.595577in}}%
\pgfpathlineto{\pgfqpoint{5.094925in}{2.595577in}}%
\pgfpathlineto{\pgfqpoint{5.112078in}{2.595577in}}%
\pgfpathlineto{\pgfqpoint{5.129232in}{2.595577in}}%
\pgfpathlineto{\pgfqpoint{5.146386in}{2.595577in}}%
\pgfpathlineto{\pgfqpoint{5.163539in}{2.595577in}}%
\pgfpathlineto{\pgfqpoint{5.180693in}{2.595577in}}%
\pgfpathlineto{\pgfqpoint{5.197847in}{2.595577in}}%
\pgfpathlineto{\pgfqpoint{5.215000in}{2.595577in}}%
\pgfusepath{stroke}%
\end{pgfscope}%
\begin{pgfscope}%
\pgfpathrectangle{\pgfqpoint{2.962722in}{0.315889in}}{\pgfqpoint{2.377500in}{2.388245in}}%
\pgfusepath{clip}%
\pgfsetrectcap%
\pgfsetroundjoin%
\pgfsetlinewidth{1.505625pt}%
\definecolor{currentstroke}{rgb}{0.172549,0.627451,0.172549}%
\pgfsetstrokecolor{currentstroke}%
\pgfsetdash{}{0pt}%
\pgfpathmoveto{\pgfqpoint{3.070790in}{2.595577in}}%
\pgfpathlineto{\pgfqpoint{3.087944in}{2.595577in}}%
\pgfpathlineto{\pgfqpoint{3.105098in}{2.595577in}}%
\pgfpathlineto{\pgfqpoint{3.122251in}{2.595577in}}%
\pgfpathlineto{\pgfqpoint{3.139405in}{2.595577in}}%
\pgfpathlineto{\pgfqpoint{3.156559in}{2.595577in}}%
\pgfpathlineto{\pgfqpoint{3.173712in}{2.595577in}}%
\pgfpathlineto{\pgfqpoint{3.190866in}{2.595577in}}%
\pgfpathlineto{\pgfqpoint{3.208020in}{2.595577in}}%
\pgfpathlineto{\pgfqpoint{3.225174in}{2.595577in}}%
\pgfpathlineto{\pgfqpoint{3.242327in}{2.595577in}}%
\pgfpathlineto{\pgfqpoint{3.259481in}{2.595577in}}%
\pgfpathlineto{\pgfqpoint{3.276635in}{2.595577in}}%
\pgfpathlineto{\pgfqpoint{3.293788in}{2.595577in}}%
\pgfpathlineto{\pgfqpoint{3.310942in}{2.595577in}}%
\pgfpathlineto{\pgfqpoint{3.328096in}{2.595577in}}%
\pgfpathlineto{\pgfqpoint{3.345249in}{2.595577in}}%
\pgfpathlineto{\pgfqpoint{3.362403in}{2.595577in}}%
\pgfpathlineto{\pgfqpoint{3.379557in}{2.595577in}}%
\pgfpathlineto{\pgfqpoint{3.396710in}{2.595577in}}%
\pgfpathlineto{\pgfqpoint{3.413864in}{2.595577in}}%
\pgfpathlineto{\pgfqpoint{3.431018in}{2.595577in}}%
\pgfpathlineto{\pgfqpoint{3.448171in}{2.595577in}}%
\pgfpathlineto{\pgfqpoint{3.465325in}{2.595577in}}%
\pgfpathlineto{\pgfqpoint{3.482479in}{2.595577in}}%
\pgfpathlineto{\pgfqpoint{3.499632in}{2.595577in}}%
\pgfpathlineto{\pgfqpoint{3.516786in}{2.595577in}}%
\pgfpathlineto{\pgfqpoint{3.533940in}{2.595577in}}%
\pgfpathlineto{\pgfqpoint{3.551093in}{2.595577in}}%
\pgfpathlineto{\pgfqpoint{3.568247in}{2.595577in}}%
\pgfpathlineto{\pgfqpoint{3.585401in}{2.595577in}}%
\pgfpathlineto{\pgfqpoint{3.602554in}{2.595577in}}%
\pgfpathlineto{\pgfqpoint{3.619708in}{2.595577in}}%
\pgfpathlineto{\pgfqpoint{3.636862in}{2.595577in}}%
\pgfpathlineto{\pgfqpoint{3.654016in}{2.595577in}}%
\pgfpathlineto{\pgfqpoint{3.671169in}{2.595577in}}%
\pgfpathlineto{\pgfqpoint{3.688323in}{2.595577in}}%
\pgfpathlineto{\pgfqpoint{3.705477in}{2.595577in}}%
\pgfpathlineto{\pgfqpoint{3.722630in}{2.595577in}}%
\pgfpathlineto{\pgfqpoint{3.739784in}{2.595577in}}%
\pgfpathlineto{\pgfqpoint{3.756938in}{2.595577in}}%
\pgfpathlineto{\pgfqpoint{3.774091in}{2.595577in}}%
\pgfpathlineto{\pgfqpoint{3.791245in}{2.595577in}}%
\pgfpathlineto{\pgfqpoint{3.808399in}{2.595577in}}%
\pgfpathlineto{\pgfqpoint{3.825552in}{2.595577in}}%
\pgfpathlineto{\pgfqpoint{3.842706in}{2.595577in}}%
\pgfpathlineto{\pgfqpoint{3.859860in}{2.595577in}}%
\pgfpathlineto{\pgfqpoint{3.877013in}{2.595577in}}%
\pgfpathlineto{\pgfqpoint{3.894167in}{2.595577in}}%
\pgfpathlineto{\pgfqpoint{3.911321in}{2.595577in}}%
\pgfpathlineto{\pgfqpoint{3.928474in}{2.595577in}}%
\pgfpathlineto{\pgfqpoint{3.945628in}{2.595577in}}%
\pgfpathlineto{\pgfqpoint{3.962782in}{2.595577in}}%
\pgfpathlineto{\pgfqpoint{3.979935in}{2.595577in}}%
\pgfpathlineto{\pgfqpoint{3.997089in}{2.595577in}}%
\pgfpathlineto{\pgfqpoint{4.014243in}{2.595577in}}%
\pgfpathlineto{\pgfqpoint{4.031396in}{2.595577in}}%
\pgfpathlineto{\pgfqpoint{4.048550in}{2.595577in}}%
\pgfpathlineto{\pgfqpoint{4.065704in}{2.595577in}}%
\pgfpathlineto{\pgfqpoint{4.082858in}{2.595577in}}%
\pgfpathlineto{\pgfqpoint{4.100011in}{2.595577in}}%
\pgfpathlineto{\pgfqpoint{4.117165in}{2.595577in}}%
\pgfpathlineto{\pgfqpoint{4.134319in}{2.595577in}}%
\pgfpathlineto{\pgfqpoint{4.151472in}{2.595577in}}%
\pgfpathlineto{\pgfqpoint{4.168626in}{2.595577in}}%
\pgfpathlineto{\pgfqpoint{4.185780in}{2.595577in}}%
\pgfpathlineto{\pgfqpoint{4.202933in}{2.595577in}}%
\pgfpathlineto{\pgfqpoint{4.220087in}{2.595577in}}%
\pgfpathlineto{\pgfqpoint{4.237241in}{2.595577in}}%
\pgfpathlineto{\pgfqpoint{4.254394in}{2.595577in}}%
\pgfpathlineto{\pgfqpoint{4.271548in}{2.595577in}}%
\pgfpathlineto{\pgfqpoint{4.288702in}{2.595577in}}%
\pgfpathlineto{\pgfqpoint{4.305855in}{2.595577in}}%
\pgfpathlineto{\pgfqpoint{4.323009in}{2.595577in}}%
\pgfpathlineto{\pgfqpoint{4.340163in}{2.595577in}}%
\pgfpathlineto{\pgfqpoint{4.357316in}{2.595577in}}%
\pgfpathlineto{\pgfqpoint{4.374470in}{2.595577in}}%
\pgfpathlineto{\pgfqpoint{4.391624in}{2.595577in}}%
\pgfpathlineto{\pgfqpoint{4.408777in}{2.595577in}}%
\pgfpathlineto{\pgfqpoint{4.425931in}{2.595577in}}%
\pgfpathlineto{\pgfqpoint{4.443085in}{2.595577in}}%
\pgfpathlineto{\pgfqpoint{4.460238in}{2.595577in}}%
\pgfpathlineto{\pgfqpoint{4.477392in}{2.595577in}}%
\pgfpathlineto{\pgfqpoint{4.494546in}{2.595577in}}%
\pgfpathlineto{\pgfqpoint{4.511699in}{2.595577in}}%
\pgfpathlineto{\pgfqpoint{4.528853in}{2.595577in}}%
\pgfpathlineto{\pgfqpoint{4.546007in}{2.595577in}}%
\pgfpathlineto{\pgfqpoint{4.563161in}{2.595577in}}%
\pgfpathlineto{\pgfqpoint{4.580314in}{2.595577in}}%
\pgfpathlineto{\pgfqpoint{4.597468in}{2.595577in}}%
\pgfpathlineto{\pgfqpoint{4.614622in}{2.595577in}}%
\pgfpathlineto{\pgfqpoint{4.631775in}{2.595577in}}%
\pgfpathlineto{\pgfqpoint{4.648929in}{2.595577in}}%
\pgfpathlineto{\pgfqpoint{4.666083in}{2.595577in}}%
\pgfpathlineto{\pgfqpoint{4.683236in}{2.595577in}}%
\pgfpathlineto{\pgfqpoint{4.700390in}{2.595577in}}%
\pgfpathlineto{\pgfqpoint{4.717544in}{2.595577in}}%
\pgfpathlineto{\pgfqpoint{4.734697in}{2.595577in}}%
\pgfpathlineto{\pgfqpoint{4.751851in}{2.595577in}}%
\pgfpathlineto{\pgfqpoint{4.769005in}{2.595577in}}%
\pgfpathlineto{\pgfqpoint{4.786158in}{2.595577in}}%
\pgfpathlineto{\pgfqpoint{4.803312in}{2.595577in}}%
\pgfpathlineto{\pgfqpoint{4.820466in}{2.595577in}}%
\pgfpathlineto{\pgfqpoint{4.837619in}{2.595577in}}%
\pgfpathlineto{\pgfqpoint{4.854773in}{2.595577in}}%
\pgfpathlineto{\pgfqpoint{4.871927in}{2.595577in}}%
\pgfpathlineto{\pgfqpoint{4.889080in}{2.595577in}}%
\pgfpathlineto{\pgfqpoint{4.906234in}{2.595577in}}%
\pgfpathlineto{\pgfqpoint{4.923388in}{2.595577in}}%
\pgfpathlineto{\pgfqpoint{4.940541in}{2.595577in}}%
\pgfpathlineto{\pgfqpoint{4.957695in}{2.595577in}}%
\pgfpathlineto{\pgfqpoint{4.974849in}{2.595577in}}%
\pgfpathlineto{\pgfqpoint{4.992003in}{2.595577in}}%
\pgfpathlineto{\pgfqpoint{5.009156in}{2.595577in}}%
\pgfpathlineto{\pgfqpoint{5.026310in}{2.595577in}}%
\pgfpathlineto{\pgfqpoint{5.043464in}{2.595577in}}%
\pgfpathlineto{\pgfqpoint{5.060617in}{2.595577in}}%
\pgfpathlineto{\pgfqpoint{5.077771in}{2.595577in}}%
\pgfpathlineto{\pgfqpoint{5.094925in}{2.595577in}}%
\pgfpathlineto{\pgfqpoint{5.112078in}{2.595577in}}%
\pgfpathlineto{\pgfqpoint{5.129232in}{2.595577in}}%
\pgfpathlineto{\pgfqpoint{5.146386in}{2.595577in}}%
\pgfpathlineto{\pgfqpoint{5.163539in}{2.595577in}}%
\pgfpathlineto{\pgfqpoint{5.180693in}{2.595577in}}%
\pgfpathlineto{\pgfqpoint{5.197847in}{2.595577in}}%
\pgfpathlineto{\pgfqpoint{5.215000in}{2.595577in}}%
\pgfusepath{stroke}%
\end{pgfscope}%
\begin{pgfscope}%
\pgfpathrectangle{\pgfqpoint{2.962722in}{0.315889in}}{\pgfqpoint{2.377500in}{2.388245in}}%
\pgfusepath{clip}%
\pgfsetrectcap%
\pgfsetroundjoin%
\pgfsetlinewidth{1.505625pt}%
\definecolor{currentstroke}{rgb}{0.839216,0.152941,0.156863}%
\pgfsetstrokecolor{currentstroke}%
\pgfsetdash{}{0pt}%
\pgfpathmoveto{\pgfqpoint{3.070790in}{0.424445in}}%
\pgfpathlineto{\pgfqpoint{3.087944in}{0.424445in}}%
\pgfpathlineto{\pgfqpoint{3.105098in}{0.424445in}}%
\pgfpathlineto{\pgfqpoint{3.122251in}{0.424445in}}%
\pgfpathlineto{\pgfqpoint{3.139405in}{0.424445in}}%
\pgfpathlineto{\pgfqpoint{3.156559in}{0.424445in}}%
\pgfpathlineto{\pgfqpoint{3.173712in}{0.424445in}}%
\pgfpathlineto{\pgfqpoint{3.190866in}{0.424445in}}%
\pgfpathlineto{\pgfqpoint{3.208020in}{0.424445in}}%
\pgfpathlineto{\pgfqpoint{3.225174in}{0.424445in}}%
\pgfpathlineto{\pgfqpoint{3.242327in}{0.424445in}}%
\pgfpathlineto{\pgfqpoint{3.259481in}{0.424445in}}%
\pgfpathlineto{\pgfqpoint{3.276635in}{0.424445in}}%
\pgfpathlineto{\pgfqpoint{3.293788in}{0.424445in}}%
\pgfpathlineto{\pgfqpoint{3.310942in}{0.424445in}}%
\pgfpathlineto{\pgfqpoint{3.328096in}{0.424445in}}%
\pgfpathlineto{\pgfqpoint{3.345249in}{0.424445in}}%
\pgfpathlineto{\pgfqpoint{3.362403in}{0.424445in}}%
\pgfpathlineto{\pgfqpoint{3.379557in}{0.424445in}}%
\pgfpathlineto{\pgfqpoint{3.396710in}{0.424445in}}%
\pgfpathlineto{\pgfqpoint{3.413864in}{0.424445in}}%
\pgfpathlineto{\pgfqpoint{3.431018in}{0.424445in}}%
\pgfpathlineto{\pgfqpoint{3.448171in}{0.424445in}}%
\pgfpathlineto{\pgfqpoint{3.465325in}{0.424445in}}%
\pgfpathlineto{\pgfqpoint{3.482479in}{0.424445in}}%
\pgfpathlineto{\pgfqpoint{3.499632in}{0.424445in}}%
\pgfpathlineto{\pgfqpoint{3.516786in}{0.424445in}}%
\pgfpathlineto{\pgfqpoint{3.533940in}{0.424445in}}%
\pgfpathlineto{\pgfqpoint{3.551093in}{0.424445in}}%
\pgfpathlineto{\pgfqpoint{3.568247in}{0.424445in}}%
\pgfpathlineto{\pgfqpoint{3.585401in}{0.424445in}}%
\pgfpathlineto{\pgfqpoint{3.602554in}{0.424445in}}%
\pgfpathlineto{\pgfqpoint{3.619708in}{0.424445in}}%
\pgfpathlineto{\pgfqpoint{3.636862in}{0.424445in}}%
\pgfpathlineto{\pgfqpoint{3.654016in}{0.424445in}}%
\pgfpathlineto{\pgfqpoint{3.671169in}{0.424445in}}%
\pgfpathlineto{\pgfqpoint{3.688323in}{0.424445in}}%
\pgfpathlineto{\pgfqpoint{3.705477in}{0.424445in}}%
\pgfpathlineto{\pgfqpoint{3.722630in}{0.424445in}}%
\pgfpathlineto{\pgfqpoint{3.739784in}{0.424445in}}%
\pgfpathlineto{\pgfqpoint{3.756938in}{0.424445in}}%
\pgfpathlineto{\pgfqpoint{3.774091in}{0.424445in}}%
\pgfpathlineto{\pgfqpoint{3.791245in}{0.424445in}}%
\pgfpathlineto{\pgfqpoint{3.808399in}{0.424445in}}%
\pgfpathlineto{\pgfqpoint{3.825552in}{0.424445in}}%
\pgfpathlineto{\pgfqpoint{3.842706in}{0.424445in}}%
\pgfpathlineto{\pgfqpoint{3.859860in}{0.424445in}}%
\pgfpathlineto{\pgfqpoint{3.877013in}{0.424445in}}%
\pgfpathlineto{\pgfqpoint{3.894167in}{0.424445in}}%
\pgfpathlineto{\pgfqpoint{3.911321in}{0.424445in}}%
\pgfpathlineto{\pgfqpoint{3.928474in}{0.424445in}}%
\pgfpathlineto{\pgfqpoint{3.945628in}{0.424445in}}%
\pgfpathlineto{\pgfqpoint{3.962782in}{0.424445in}}%
\pgfpathlineto{\pgfqpoint{3.979935in}{0.424445in}}%
\pgfpathlineto{\pgfqpoint{3.997089in}{0.424445in}}%
\pgfpathlineto{\pgfqpoint{4.014243in}{0.424445in}}%
\pgfpathlineto{\pgfqpoint{4.031396in}{0.424445in}}%
\pgfpathlineto{\pgfqpoint{4.048550in}{0.424445in}}%
\pgfpathlineto{\pgfqpoint{4.065704in}{0.424445in}}%
\pgfpathlineto{\pgfqpoint{4.082858in}{0.424445in}}%
\pgfpathlineto{\pgfqpoint{4.100011in}{0.424445in}}%
\pgfpathlineto{\pgfqpoint{4.117165in}{0.424445in}}%
\pgfpathlineto{\pgfqpoint{4.134319in}{0.424445in}}%
\pgfpathlineto{\pgfqpoint{4.151472in}{0.424445in}}%
\pgfpathlineto{\pgfqpoint{4.168626in}{0.424445in}}%
\pgfpathlineto{\pgfqpoint{4.185780in}{0.424445in}}%
\pgfpathlineto{\pgfqpoint{4.202933in}{0.424445in}}%
\pgfpathlineto{\pgfqpoint{4.220087in}{0.424445in}}%
\pgfpathlineto{\pgfqpoint{4.237241in}{0.424445in}}%
\pgfpathlineto{\pgfqpoint{4.254394in}{0.424445in}}%
\pgfpathlineto{\pgfqpoint{4.271548in}{0.424445in}}%
\pgfpathlineto{\pgfqpoint{4.288702in}{0.424445in}}%
\pgfpathlineto{\pgfqpoint{4.305855in}{0.424445in}}%
\pgfpathlineto{\pgfqpoint{4.323009in}{0.424445in}}%
\pgfpathlineto{\pgfqpoint{4.340163in}{0.424445in}}%
\pgfpathlineto{\pgfqpoint{4.357316in}{0.424445in}}%
\pgfpathlineto{\pgfqpoint{4.374470in}{0.424445in}}%
\pgfpathlineto{\pgfqpoint{4.391624in}{0.424445in}}%
\pgfpathlineto{\pgfqpoint{4.408777in}{0.424445in}}%
\pgfpathlineto{\pgfqpoint{4.425931in}{0.424445in}}%
\pgfpathlineto{\pgfqpoint{4.443085in}{0.424445in}}%
\pgfpathlineto{\pgfqpoint{4.460238in}{0.424445in}}%
\pgfpathlineto{\pgfqpoint{4.477392in}{0.424445in}}%
\pgfpathlineto{\pgfqpoint{4.494546in}{0.424445in}}%
\pgfpathlineto{\pgfqpoint{4.511699in}{0.424445in}}%
\pgfpathlineto{\pgfqpoint{4.528853in}{0.424445in}}%
\pgfpathlineto{\pgfqpoint{4.546007in}{0.424445in}}%
\pgfpathlineto{\pgfqpoint{4.563161in}{0.424445in}}%
\pgfpathlineto{\pgfqpoint{4.580314in}{0.424445in}}%
\pgfpathlineto{\pgfqpoint{4.597468in}{0.424445in}}%
\pgfpathlineto{\pgfqpoint{4.614622in}{0.424445in}}%
\pgfpathlineto{\pgfqpoint{4.631775in}{0.424445in}}%
\pgfpathlineto{\pgfqpoint{4.648929in}{0.424445in}}%
\pgfpathlineto{\pgfqpoint{4.666083in}{0.424445in}}%
\pgfpathlineto{\pgfqpoint{4.683236in}{0.424445in}}%
\pgfpathlineto{\pgfqpoint{4.700390in}{0.424445in}}%
\pgfpathlineto{\pgfqpoint{4.717544in}{0.424445in}}%
\pgfpathlineto{\pgfqpoint{4.734697in}{0.424445in}}%
\pgfpathlineto{\pgfqpoint{4.751851in}{0.424445in}}%
\pgfpathlineto{\pgfqpoint{4.769005in}{0.424445in}}%
\pgfpathlineto{\pgfqpoint{4.786158in}{0.424445in}}%
\pgfpathlineto{\pgfqpoint{4.803312in}{0.424445in}}%
\pgfpathlineto{\pgfqpoint{4.820466in}{0.424445in}}%
\pgfpathlineto{\pgfqpoint{4.837619in}{0.424445in}}%
\pgfpathlineto{\pgfqpoint{4.854773in}{0.424445in}}%
\pgfpathlineto{\pgfqpoint{4.871927in}{0.424445in}}%
\pgfpathlineto{\pgfqpoint{4.889080in}{0.424445in}}%
\pgfpathlineto{\pgfqpoint{4.906234in}{0.424445in}}%
\pgfpathlineto{\pgfqpoint{4.923388in}{0.424445in}}%
\pgfpathlineto{\pgfqpoint{4.940541in}{0.424445in}}%
\pgfpathlineto{\pgfqpoint{4.957695in}{0.424445in}}%
\pgfpathlineto{\pgfqpoint{4.974849in}{0.424445in}}%
\pgfpathlineto{\pgfqpoint{4.992003in}{0.424445in}}%
\pgfpathlineto{\pgfqpoint{5.009156in}{0.424445in}}%
\pgfpathlineto{\pgfqpoint{5.026310in}{0.424445in}}%
\pgfpathlineto{\pgfqpoint{5.043464in}{0.424445in}}%
\pgfpathlineto{\pgfqpoint{5.060617in}{0.424445in}}%
\pgfpathlineto{\pgfqpoint{5.077771in}{0.424445in}}%
\pgfpathlineto{\pgfqpoint{5.094925in}{0.424445in}}%
\pgfpathlineto{\pgfqpoint{5.112078in}{0.424445in}}%
\pgfpathlineto{\pgfqpoint{5.129232in}{0.424445in}}%
\pgfpathlineto{\pgfqpoint{5.146386in}{0.424445in}}%
\pgfpathlineto{\pgfqpoint{5.163539in}{0.424445in}}%
\pgfpathlineto{\pgfqpoint{5.180693in}{0.424445in}}%
\pgfpathlineto{\pgfqpoint{5.197847in}{0.424445in}}%
\pgfpathlineto{\pgfqpoint{5.215000in}{0.424445in}}%
\pgfusepath{stroke}%
\end{pgfscope}%
\begin{pgfscope}%
\pgfsetrectcap%
\pgfsetmiterjoin%
\pgfsetlinewidth{0.803000pt}%
\definecolor{currentstroke}{rgb}{0.000000,0.000000,0.000000}%
\pgfsetstrokecolor{currentstroke}%
\pgfsetdash{}{0pt}%
\pgfpathmoveto{\pgfqpoint{2.962722in}{0.315889in}}%
\pgfpathlineto{\pgfqpoint{2.962722in}{2.704133in}}%
\pgfusepath{stroke}%
\end{pgfscope}%
\begin{pgfscope}%
\pgfsetrectcap%
\pgfsetmiterjoin%
\pgfsetlinewidth{0.803000pt}%
\definecolor{currentstroke}{rgb}{0.000000,0.000000,0.000000}%
\pgfsetstrokecolor{currentstroke}%
\pgfsetdash{}{0pt}%
\pgfpathmoveto{\pgfqpoint{5.340222in}{0.315889in}}%
\pgfpathlineto{\pgfqpoint{5.340222in}{2.704133in}}%
\pgfusepath{stroke}%
\end{pgfscope}%
\begin{pgfscope}%
\pgfsetrectcap%
\pgfsetmiterjoin%
\pgfsetlinewidth{0.803000pt}%
\definecolor{currentstroke}{rgb}{0.000000,0.000000,0.000000}%
\pgfsetstrokecolor{currentstroke}%
\pgfsetdash{}{0pt}%
\pgfpathmoveto{\pgfqpoint{2.962722in}{0.315889in}}%
\pgfpathlineto{\pgfqpoint{5.340222in}{0.315889in}}%
\pgfusepath{stroke}%
\end{pgfscope}%
\begin{pgfscope}%
\pgfsetrectcap%
\pgfsetmiterjoin%
\pgfsetlinewidth{0.803000pt}%
\definecolor{currentstroke}{rgb}{0.000000,0.000000,0.000000}%
\pgfsetstrokecolor{currentstroke}%
\pgfsetdash{}{0pt}%
\pgfpathmoveto{\pgfqpoint{2.962722in}{2.704133in}}%
\pgfpathlineto{\pgfqpoint{5.340222in}{2.704133in}}%
\pgfusepath{stroke}%
\end{pgfscope}%
\begin{pgfscope}%
\definecolor{textcolor}{rgb}{0.000000,0.000000,0.000000}%
\pgfsetstrokecolor{textcolor}%
\pgfsetfillcolor{textcolor}%
\pgftext[x=4.151472in,y=2.787467in,,base]{\color{textcolor}\rmfamily\fontsize{9.600000}{11.520000}\selectfont db3}%
\end{pgfscope}%
\end{pgfpicture}%
\makeatother%
\endgroup%

    \caption{Analyse mit db2/3 Wavelet\label{polynomials:db2_3}}
\end{figure}

Wie sich herausstellt liefern diese Wavelets jeweils die zweite und dritte
Ableitung unserer Signale. Daubechies Wavelets mit $A$ verschwindenden Momenten
können uns also direkt die $A$te Ableitung liefern. Dies im Gegensatz zum
Differenzieren welches mehrfach angewendet werden muss.

\section{Anwendung zur rauscharmen Ableitung}
\rhead{Rauscharme Ableitung}

\section{Hochfrequente Anteile in Polynomen}
\rhead{Hochfrequente Anteile in Polynomen}

\section{Schlussfolgerung}
\rhead{Schlussfolgerung}

\printbibliography[heading=subbibliography]
\end{refsection}
