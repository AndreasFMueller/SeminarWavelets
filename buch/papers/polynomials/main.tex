%
% main.tex -- Paper zum Thema Polynome
%
% (c) 2019 Hochschule Rapperswil
%
\chapter{Wavelets und polynomiale Signale\label{chapter:thema}}
\lhead{Wavelets und polynomiale Signale}
\begin{refsection}
\chapterauthor{Raphael Nestler}

In der Literatur zu Wavelets findet man die folgende Aussage zu Daubechies
Wavelets:
\begin{displayquote}[\cite{wikipedia:daubechies}]
For example, $D2$, with one vanishing moment, easily encodes polynomials of one
coefficient, or constant signal components. $D4$ encodes polynomials with two
coefficients, i.e.\ constant and linear signal components; and $D6$ encodes
3-polynomials, i.e.\ constant, linear and quadratic signal components.
\end{displayquote}
Ein Daubechies Wavelet mit $A$ verschwindenden Momenten und Filterlänge $N=2A$
soll also ein Polynom der Ordnung $A-1$ einfach darstellen können. Wir wollen
erörtern, was das nun in der Praxis genau bedeutet und welche Anwendungen es
ermöglicht.

Wir werden im folgenden Daubechies Wavelets mit $A$ verschwindenden Momenten mit
db$A$ bezeichnen.

Die Analysen und Simulationen wurden jeweils mittels Python~\cite{python} und im
speziellem dem PyWavelets~\cite{gregory_r_lee_2019_2634243} Paket durchgeführt.
Der dazu verwendete Code wurde in einem GitHub-Repository%
\footnote{\url{https://github.com/rnestler/mathsem-FS2019/tree/paper}}%
~\cite{polynomials:repo}
abgelegt. Zusätzlich sind auch interaktive Jupyter Notebooks abgelegt um den
Einfluss von verschiedenen Parametern auszuprobieren. Mittels
Binder%
\footnote{\url{https://mybinder.org/v2/gh/rnestler/mathsem-FS2019/paper}}%
~\cite{project_jupyter-proc-scipy-2018}%
kann eine online Umgebung gestartet werden um diese Beispiele auszuführen.

\section{Analyse von polynomialen Signalen}
\rhead{Polynomiale Signale}

Wir werden als erstes die Signale in \autoref{polynomials:signals} mittels des db1
(Haar) Wavelets analysieren.

\begin{figure}
    \centering
    %% Creator: Matplotlib, PGF backend
%%
%% To include the figure in your LaTeX document, write
%%   \input{<filename>.pgf}
%%
%% Make sure the required packages are loaded in your preamble
%%   \usepackage{pgf}
%%
%% Figures using additional raster images can only be included by \input if
%% they are in the same directory as the main LaTeX file. For loading figures
%% from other directories you can use the `import` package
%%   \usepackage{import}
%% and then include the figures with
%%   \import{<path to file>}{<filename>.pgf}
%%
%% Matplotlib used the following preamble
%%   \usepackage{fontspec}
%%
\begingroup%
\makeatletter%
\begin{pgfpicture}%
\pgfpathrectangle{\pgfpointorigin}{\pgfqpoint{5.800000in}{3.300000in}}%
\pgfusepath{use as bounding box, clip}%
\begin{pgfscope}%
\pgfsetbuttcap%
\pgfsetmiterjoin%
\definecolor{currentfill}{rgb}{1.000000,1.000000,1.000000}%
\pgfsetfillcolor{currentfill}%
\pgfsetlinewidth{0.000000pt}%
\definecolor{currentstroke}{rgb}{1.000000,1.000000,1.000000}%
\pgfsetstrokecolor{currentstroke}%
\pgfsetdash{}{0pt}%
\pgfpathmoveto{\pgfqpoint{0.000000in}{0.000000in}}%
\pgfpathlineto{\pgfqpoint{5.800000in}{0.000000in}}%
\pgfpathlineto{\pgfqpoint{5.800000in}{3.300000in}}%
\pgfpathlineto{\pgfqpoint{0.000000in}{3.300000in}}%
\pgfpathclose%
\pgfusepath{fill}%
\end{pgfscope}%
\begin{pgfscope}%
\pgfsetbuttcap%
\pgfsetmiterjoin%
\definecolor{currentfill}{rgb}{1.000000,1.000000,1.000000}%
\pgfsetfillcolor{currentfill}%
\pgfsetlinewidth{0.000000pt}%
\definecolor{currentstroke}{rgb}{0.000000,0.000000,0.000000}%
\pgfsetstrokecolor{currentstroke}%
\pgfsetstrokeopacity{0.000000}%
\pgfsetdash{}{0pt}%
\pgfpathmoveto{\pgfqpoint{0.725000in}{0.363000in}}%
\pgfpathlineto{\pgfqpoint{5.220000in}{0.363000in}}%
\pgfpathlineto{\pgfqpoint{5.220000in}{2.904000in}}%
\pgfpathlineto{\pgfqpoint{0.725000in}{2.904000in}}%
\pgfpathclose%
\pgfusepath{fill}%
\end{pgfscope}%
\begin{pgfscope}%
\pgfsetbuttcap%
\pgfsetroundjoin%
\definecolor{currentfill}{rgb}{0.000000,0.000000,0.000000}%
\pgfsetfillcolor{currentfill}%
\pgfsetlinewidth{0.803000pt}%
\definecolor{currentstroke}{rgb}{0.000000,0.000000,0.000000}%
\pgfsetstrokecolor{currentstroke}%
\pgfsetdash{}{0pt}%
\pgfsys@defobject{currentmarker}{\pgfqpoint{0.000000in}{-0.048611in}}{\pgfqpoint{0.000000in}{0.000000in}}{%
\pgfpathmoveto{\pgfqpoint{0.000000in}{0.000000in}}%
\pgfpathlineto{\pgfqpoint{0.000000in}{-0.048611in}}%
\pgfusepath{stroke,fill}%
}%
\begin{pgfscope}%
\pgfsys@transformshift{0.929318in}{0.363000in}%
\pgfsys@useobject{currentmarker}{}%
\end{pgfscope}%
\end{pgfscope}%
\begin{pgfscope}%
\definecolor{textcolor}{rgb}{0.000000,0.000000,0.000000}%
\pgfsetstrokecolor{textcolor}%
\pgfsetfillcolor{textcolor}%
\pgftext[x=0.929318in,y=0.265778in,,top]{\color{textcolor}\sffamily\fontsize{10.000000}{12.000000}\selectfont 0.00}%
\end{pgfscope}%
\begin{pgfscope}%
\pgfsetbuttcap%
\pgfsetroundjoin%
\definecolor{currentfill}{rgb}{0.000000,0.000000,0.000000}%
\pgfsetfillcolor{currentfill}%
\pgfsetlinewidth{0.803000pt}%
\definecolor{currentstroke}{rgb}{0.000000,0.000000,0.000000}%
\pgfsetstrokecolor{currentstroke}%
\pgfsetdash{}{0pt}%
\pgfsys@defobject{currentmarker}{\pgfqpoint{0.000000in}{-0.048611in}}{\pgfqpoint{0.000000in}{0.000000in}}{%
\pgfpathmoveto{\pgfqpoint{0.000000in}{0.000000in}}%
\pgfpathlineto{\pgfqpoint{0.000000in}{-0.048611in}}%
\pgfusepath{stroke,fill}%
}%
\begin{pgfscope}%
\pgfsys@transformshift{1.440114in}{0.363000in}%
\pgfsys@useobject{currentmarker}{}%
\end{pgfscope}%
\end{pgfscope}%
\begin{pgfscope}%
\definecolor{textcolor}{rgb}{0.000000,0.000000,0.000000}%
\pgfsetstrokecolor{textcolor}%
\pgfsetfillcolor{textcolor}%
\pgftext[x=1.440114in,y=0.265778in,,top]{\color{textcolor}\sffamily\fontsize{10.000000}{12.000000}\selectfont 0.25}%
\end{pgfscope}%
\begin{pgfscope}%
\pgfsetbuttcap%
\pgfsetroundjoin%
\definecolor{currentfill}{rgb}{0.000000,0.000000,0.000000}%
\pgfsetfillcolor{currentfill}%
\pgfsetlinewidth{0.803000pt}%
\definecolor{currentstroke}{rgb}{0.000000,0.000000,0.000000}%
\pgfsetstrokecolor{currentstroke}%
\pgfsetdash{}{0pt}%
\pgfsys@defobject{currentmarker}{\pgfqpoint{0.000000in}{-0.048611in}}{\pgfqpoint{0.000000in}{0.000000in}}{%
\pgfpathmoveto{\pgfqpoint{0.000000in}{0.000000in}}%
\pgfpathlineto{\pgfqpoint{0.000000in}{-0.048611in}}%
\pgfusepath{stroke,fill}%
}%
\begin{pgfscope}%
\pgfsys@transformshift{1.950909in}{0.363000in}%
\pgfsys@useobject{currentmarker}{}%
\end{pgfscope}%
\end{pgfscope}%
\begin{pgfscope}%
\definecolor{textcolor}{rgb}{0.000000,0.000000,0.000000}%
\pgfsetstrokecolor{textcolor}%
\pgfsetfillcolor{textcolor}%
\pgftext[x=1.950909in,y=0.265778in,,top]{\color{textcolor}\sffamily\fontsize{10.000000}{12.000000}\selectfont 0.50}%
\end{pgfscope}%
\begin{pgfscope}%
\pgfsetbuttcap%
\pgfsetroundjoin%
\definecolor{currentfill}{rgb}{0.000000,0.000000,0.000000}%
\pgfsetfillcolor{currentfill}%
\pgfsetlinewidth{0.803000pt}%
\definecolor{currentstroke}{rgb}{0.000000,0.000000,0.000000}%
\pgfsetstrokecolor{currentstroke}%
\pgfsetdash{}{0pt}%
\pgfsys@defobject{currentmarker}{\pgfqpoint{0.000000in}{-0.048611in}}{\pgfqpoint{0.000000in}{0.000000in}}{%
\pgfpathmoveto{\pgfqpoint{0.000000in}{0.000000in}}%
\pgfpathlineto{\pgfqpoint{0.000000in}{-0.048611in}}%
\pgfusepath{stroke,fill}%
}%
\begin{pgfscope}%
\pgfsys@transformshift{2.461705in}{0.363000in}%
\pgfsys@useobject{currentmarker}{}%
\end{pgfscope}%
\end{pgfscope}%
\begin{pgfscope}%
\definecolor{textcolor}{rgb}{0.000000,0.000000,0.000000}%
\pgfsetstrokecolor{textcolor}%
\pgfsetfillcolor{textcolor}%
\pgftext[x=2.461705in,y=0.265778in,,top]{\color{textcolor}\sffamily\fontsize{10.000000}{12.000000}\selectfont 0.75}%
\end{pgfscope}%
\begin{pgfscope}%
\pgfsetbuttcap%
\pgfsetroundjoin%
\definecolor{currentfill}{rgb}{0.000000,0.000000,0.000000}%
\pgfsetfillcolor{currentfill}%
\pgfsetlinewidth{0.803000pt}%
\definecolor{currentstroke}{rgb}{0.000000,0.000000,0.000000}%
\pgfsetstrokecolor{currentstroke}%
\pgfsetdash{}{0pt}%
\pgfsys@defobject{currentmarker}{\pgfqpoint{0.000000in}{-0.048611in}}{\pgfqpoint{0.000000in}{0.000000in}}{%
\pgfpathmoveto{\pgfqpoint{0.000000in}{0.000000in}}%
\pgfpathlineto{\pgfqpoint{0.000000in}{-0.048611in}}%
\pgfusepath{stroke,fill}%
}%
\begin{pgfscope}%
\pgfsys@transformshift{2.972500in}{0.363000in}%
\pgfsys@useobject{currentmarker}{}%
\end{pgfscope}%
\end{pgfscope}%
\begin{pgfscope}%
\definecolor{textcolor}{rgb}{0.000000,0.000000,0.000000}%
\pgfsetstrokecolor{textcolor}%
\pgfsetfillcolor{textcolor}%
\pgftext[x=2.972500in,y=0.265778in,,top]{\color{textcolor}\sffamily\fontsize{10.000000}{12.000000}\selectfont 1.00}%
\end{pgfscope}%
\begin{pgfscope}%
\pgfsetbuttcap%
\pgfsetroundjoin%
\definecolor{currentfill}{rgb}{0.000000,0.000000,0.000000}%
\pgfsetfillcolor{currentfill}%
\pgfsetlinewidth{0.803000pt}%
\definecolor{currentstroke}{rgb}{0.000000,0.000000,0.000000}%
\pgfsetstrokecolor{currentstroke}%
\pgfsetdash{}{0pt}%
\pgfsys@defobject{currentmarker}{\pgfqpoint{0.000000in}{-0.048611in}}{\pgfqpoint{0.000000in}{0.000000in}}{%
\pgfpathmoveto{\pgfqpoint{0.000000in}{0.000000in}}%
\pgfpathlineto{\pgfqpoint{0.000000in}{-0.048611in}}%
\pgfusepath{stroke,fill}%
}%
\begin{pgfscope}%
\pgfsys@transformshift{3.483295in}{0.363000in}%
\pgfsys@useobject{currentmarker}{}%
\end{pgfscope}%
\end{pgfscope}%
\begin{pgfscope}%
\definecolor{textcolor}{rgb}{0.000000,0.000000,0.000000}%
\pgfsetstrokecolor{textcolor}%
\pgfsetfillcolor{textcolor}%
\pgftext[x=3.483295in,y=0.265778in,,top]{\color{textcolor}\sffamily\fontsize{10.000000}{12.000000}\selectfont 1.25}%
\end{pgfscope}%
\begin{pgfscope}%
\pgfsetbuttcap%
\pgfsetroundjoin%
\definecolor{currentfill}{rgb}{0.000000,0.000000,0.000000}%
\pgfsetfillcolor{currentfill}%
\pgfsetlinewidth{0.803000pt}%
\definecolor{currentstroke}{rgb}{0.000000,0.000000,0.000000}%
\pgfsetstrokecolor{currentstroke}%
\pgfsetdash{}{0pt}%
\pgfsys@defobject{currentmarker}{\pgfqpoint{0.000000in}{-0.048611in}}{\pgfqpoint{0.000000in}{0.000000in}}{%
\pgfpathmoveto{\pgfqpoint{0.000000in}{0.000000in}}%
\pgfpathlineto{\pgfqpoint{0.000000in}{-0.048611in}}%
\pgfusepath{stroke,fill}%
}%
\begin{pgfscope}%
\pgfsys@transformshift{3.994091in}{0.363000in}%
\pgfsys@useobject{currentmarker}{}%
\end{pgfscope}%
\end{pgfscope}%
\begin{pgfscope}%
\definecolor{textcolor}{rgb}{0.000000,0.000000,0.000000}%
\pgfsetstrokecolor{textcolor}%
\pgfsetfillcolor{textcolor}%
\pgftext[x=3.994091in,y=0.265778in,,top]{\color{textcolor}\sffamily\fontsize{10.000000}{12.000000}\selectfont 1.50}%
\end{pgfscope}%
\begin{pgfscope}%
\pgfsetbuttcap%
\pgfsetroundjoin%
\definecolor{currentfill}{rgb}{0.000000,0.000000,0.000000}%
\pgfsetfillcolor{currentfill}%
\pgfsetlinewidth{0.803000pt}%
\definecolor{currentstroke}{rgb}{0.000000,0.000000,0.000000}%
\pgfsetstrokecolor{currentstroke}%
\pgfsetdash{}{0pt}%
\pgfsys@defobject{currentmarker}{\pgfqpoint{0.000000in}{-0.048611in}}{\pgfqpoint{0.000000in}{0.000000in}}{%
\pgfpathmoveto{\pgfqpoint{0.000000in}{0.000000in}}%
\pgfpathlineto{\pgfqpoint{0.000000in}{-0.048611in}}%
\pgfusepath{stroke,fill}%
}%
\begin{pgfscope}%
\pgfsys@transformshift{4.504886in}{0.363000in}%
\pgfsys@useobject{currentmarker}{}%
\end{pgfscope}%
\end{pgfscope}%
\begin{pgfscope}%
\definecolor{textcolor}{rgb}{0.000000,0.000000,0.000000}%
\pgfsetstrokecolor{textcolor}%
\pgfsetfillcolor{textcolor}%
\pgftext[x=4.504886in,y=0.265778in,,top]{\color{textcolor}\sffamily\fontsize{10.000000}{12.000000}\selectfont 1.75}%
\end{pgfscope}%
\begin{pgfscope}%
\pgfsetbuttcap%
\pgfsetroundjoin%
\definecolor{currentfill}{rgb}{0.000000,0.000000,0.000000}%
\pgfsetfillcolor{currentfill}%
\pgfsetlinewidth{0.803000pt}%
\definecolor{currentstroke}{rgb}{0.000000,0.000000,0.000000}%
\pgfsetstrokecolor{currentstroke}%
\pgfsetdash{}{0pt}%
\pgfsys@defobject{currentmarker}{\pgfqpoint{0.000000in}{-0.048611in}}{\pgfqpoint{0.000000in}{0.000000in}}{%
\pgfpathmoveto{\pgfqpoint{0.000000in}{0.000000in}}%
\pgfpathlineto{\pgfqpoint{0.000000in}{-0.048611in}}%
\pgfusepath{stroke,fill}%
}%
\begin{pgfscope}%
\pgfsys@transformshift{5.015682in}{0.363000in}%
\pgfsys@useobject{currentmarker}{}%
\end{pgfscope}%
\end{pgfscope}%
\begin{pgfscope}%
\definecolor{textcolor}{rgb}{0.000000,0.000000,0.000000}%
\pgfsetstrokecolor{textcolor}%
\pgfsetfillcolor{textcolor}%
\pgftext[x=5.015682in,y=0.265778in,,top]{\color{textcolor}\sffamily\fontsize{10.000000}{12.000000}\selectfont 2.00}%
\end{pgfscope}%
\begin{pgfscope}%
\pgfsetbuttcap%
\pgfsetroundjoin%
\definecolor{currentfill}{rgb}{0.000000,0.000000,0.000000}%
\pgfsetfillcolor{currentfill}%
\pgfsetlinewidth{0.803000pt}%
\definecolor{currentstroke}{rgb}{0.000000,0.000000,0.000000}%
\pgfsetstrokecolor{currentstroke}%
\pgfsetdash{}{0pt}%
\pgfsys@defobject{currentmarker}{\pgfqpoint{-0.048611in}{0.000000in}}{\pgfqpoint{0.000000in}{0.000000in}}{%
\pgfpathmoveto{\pgfqpoint{0.000000in}{0.000000in}}%
\pgfpathlineto{\pgfqpoint{-0.048611in}{0.000000in}}%
\pgfusepath{stroke,fill}%
}%
\begin{pgfscope}%
\pgfsys@transformshift{0.725000in}{0.478500in}%
\pgfsys@useobject{currentmarker}{}%
\end{pgfscope}%
\end{pgfscope}%
\begin{pgfscope}%
\definecolor{textcolor}{rgb}{0.000000,0.000000,0.000000}%
\pgfsetstrokecolor{textcolor}%
\pgfsetfillcolor{textcolor}%
\pgftext[x=0.558333in,y=0.430306in,left,base]{\color{textcolor}\sffamily\fontsize{10.000000}{12.000000}\selectfont 0}%
\end{pgfscope}%
\begin{pgfscope}%
\pgfsetbuttcap%
\pgfsetroundjoin%
\definecolor{currentfill}{rgb}{0.000000,0.000000,0.000000}%
\pgfsetfillcolor{currentfill}%
\pgfsetlinewidth{0.803000pt}%
\definecolor{currentstroke}{rgb}{0.000000,0.000000,0.000000}%
\pgfsetstrokecolor{currentstroke}%
\pgfsetdash{}{0pt}%
\pgfsys@defobject{currentmarker}{\pgfqpoint{-0.048611in}{0.000000in}}{\pgfqpoint{0.000000in}{0.000000in}}{%
\pgfpathmoveto{\pgfqpoint{0.000000in}{0.000000in}}%
\pgfpathlineto{\pgfqpoint{-0.048611in}{0.000000in}}%
\pgfusepath{stroke,fill}%
}%
\begin{pgfscope}%
\pgfsys@transformshift{0.725000in}{0.767250in}%
\pgfsys@useobject{currentmarker}{}%
\end{pgfscope}%
\end{pgfscope}%
\begin{pgfscope}%
\definecolor{textcolor}{rgb}{0.000000,0.000000,0.000000}%
\pgfsetstrokecolor{textcolor}%
\pgfsetfillcolor{textcolor}%
\pgftext[x=0.558333in,y=0.719056in,left,base]{\color{textcolor}\sffamily\fontsize{10.000000}{12.000000}\selectfont 1}%
\end{pgfscope}%
\begin{pgfscope}%
\pgfsetbuttcap%
\pgfsetroundjoin%
\definecolor{currentfill}{rgb}{0.000000,0.000000,0.000000}%
\pgfsetfillcolor{currentfill}%
\pgfsetlinewidth{0.803000pt}%
\definecolor{currentstroke}{rgb}{0.000000,0.000000,0.000000}%
\pgfsetstrokecolor{currentstroke}%
\pgfsetdash{}{0pt}%
\pgfsys@defobject{currentmarker}{\pgfqpoint{-0.048611in}{0.000000in}}{\pgfqpoint{0.000000in}{0.000000in}}{%
\pgfpathmoveto{\pgfqpoint{0.000000in}{0.000000in}}%
\pgfpathlineto{\pgfqpoint{-0.048611in}{0.000000in}}%
\pgfusepath{stroke,fill}%
}%
\begin{pgfscope}%
\pgfsys@transformshift{0.725000in}{1.056000in}%
\pgfsys@useobject{currentmarker}{}%
\end{pgfscope}%
\end{pgfscope}%
\begin{pgfscope}%
\definecolor{textcolor}{rgb}{0.000000,0.000000,0.000000}%
\pgfsetstrokecolor{textcolor}%
\pgfsetfillcolor{textcolor}%
\pgftext[x=0.558333in,y=1.007806in,left,base]{\color{textcolor}\sffamily\fontsize{10.000000}{12.000000}\selectfont 2}%
\end{pgfscope}%
\begin{pgfscope}%
\pgfsetbuttcap%
\pgfsetroundjoin%
\definecolor{currentfill}{rgb}{0.000000,0.000000,0.000000}%
\pgfsetfillcolor{currentfill}%
\pgfsetlinewidth{0.803000pt}%
\definecolor{currentstroke}{rgb}{0.000000,0.000000,0.000000}%
\pgfsetstrokecolor{currentstroke}%
\pgfsetdash{}{0pt}%
\pgfsys@defobject{currentmarker}{\pgfqpoint{-0.048611in}{0.000000in}}{\pgfqpoint{0.000000in}{0.000000in}}{%
\pgfpathmoveto{\pgfqpoint{0.000000in}{0.000000in}}%
\pgfpathlineto{\pgfqpoint{-0.048611in}{0.000000in}}%
\pgfusepath{stroke,fill}%
}%
\begin{pgfscope}%
\pgfsys@transformshift{0.725000in}{1.344750in}%
\pgfsys@useobject{currentmarker}{}%
\end{pgfscope}%
\end{pgfscope}%
\begin{pgfscope}%
\definecolor{textcolor}{rgb}{0.000000,0.000000,0.000000}%
\pgfsetstrokecolor{textcolor}%
\pgfsetfillcolor{textcolor}%
\pgftext[x=0.558333in,y=1.296556in,left,base]{\color{textcolor}\sffamily\fontsize{10.000000}{12.000000}\selectfont 3}%
\end{pgfscope}%
\begin{pgfscope}%
\pgfsetbuttcap%
\pgfsetroundjoin%
\definecolor{currentfill}{rgb}{0.000000,0.000000,0.000000}%
\pgfsetfillcolor{currentfill}%
\pgfsetlinewidth{0.803000pt}%
\definecolor{currentstroke}{rgb}{0.000000,0.000000,0.000000}%
\pgfsetstrokecolor{currentstroke}%
\pgfsetdash{}{0pt}%
\pgfsys@defobject{currentmarker}{\pgfqpoint{-0.048611in}{0.000000in}}{\pgfqpoint{0.000000in}{0.000000in}}{%
\pgfpathmoveto{\pgfqpoint{0.000000in}{0.000000in}}%
\pgfpathlineto{\pgfqpoint{-0.048611in}{0.000000in}}%
\pgfusepath{stroke,fill}%
}%
\begin{pgfscope}%
\pgfsys@transformshift{0.725000in}{1.633500in}%
\pgfsys@useobject{currentmarker}{}%
\end{pgfscope}%
\end{pgfscope}%
\begin{pgfscope}%
\definecolor{textcolor}{rgb}{0.000000,0.000000,0.000000}%
\pgfsetstrokecolor{textcolor}%
\pgfsetfillcolor{textcolor}%
\pgftext[x=0.558333in,y=1.585306in,left,base]{\color{textcolor}\sffamily\fontsize{10.000000}{12.000000}\selectfont 4}%
\end{pgfscope}%
\begin{pgfscope}%
\pgfsetbuttcap%
\pgfsetroundjoin%
\definecolor{currentfill}{rgb}{0.000000,0.000000,0.000000}%
\pgfsetfillcolor{currentfill}%
\pgfsetlinewidth{0.803000pt}%
\definecolor{currentstroke}{rgb}{0.000000,0.000000,0.000000}%
\pgfsetstrokecolor{currentstroke}%
\pgfsetdash{}{0pt}%
\pgfsys@defobject{currentmarker}{\pgfqpoint{-0.048611in}{0.000000in}}{\pgfqpoint{0.000000in}{0.000000in}}{%
\pgfpathmoveto{\pgfqpoint{0.000000in}{0.000000in}}%
\pgfpathlineto{\pgfqpoint{-0.048611in}{0.000000in}}%
\pgfusepath{stroke,fill}%
}%
\begin{pgfscope}%
\pgfsys@transformshift{0.725000in}{1.922250in}%
\pgfsys@useobject{currentmarker}{}%
\end{pgfscope}%
\end{pgfscope}%
\begin{pgfscope}%
\definecolor{textcolor}{rgb}{0.000000,0.000000,0.000000}%
\pgfsetstrokecolor{textcolor}%
\pgfsetfillcolor{textcolor}%
\pgftext[x=0.558333in,y=1.874056in,left,base]{\color{textcolor}\sffamily\fontsize{10.000000}{12.000000}\selectfont 5}%
\end{pgfscope}%
\begin{pgfscope}%
\pgfsetbuttcap%
\pgfsetroundjoin%
\definecolor{currentfill}{rgb}{0.000000,0.000000,0.000000}%
\pgfsetfillcolor{currentfill}%
\pgfsetlinewidth{0.803000pt}%
\definecolor{currentstroke}{rgb}{0.000000,0.000000,0.000000}%
\pgfsetstrokecolor{currentstroke}%
\pgfsetdash{}{0pt}%
\pgfsys@defobject{currentmarker}{\pgfqpoint{-0.048611in}{0.000000in}}{\pgfqpoint{0.000000in}{0.000000in}}{%
\pgfpathmoveto{\pgfqpoint{0.000000in}{0.000000in}}%
\pgfpathlineto{\pgfqpoint{-0.048611in}{0.000000in}}%
\pgfusepath{stroke,fill}%
}%
\begin{pgfscope}%
\pgfsys@transformshift{0.725000in}{2.211000in}%
\pgfsys@useobject{currentmarker}{}%
\end{pgfscope}%
\end{pgfscope}%
\begin{pgfscope}%
\definecolor{textcolor}{rgb}{0.000000,0.000000,0.000000}%
\pgfsetstrokecolor{textcolor}%
\pgfsetfillcolor{textcolor}%
\pgftext[x=0.558333in,y=2.162806in,left,base]{\color{textcolor}\sffamily\fontsize{10.000000}{12.000000}\selectfont 6}%
\end{pgfscope}%
\begin{pgfscope}%
\pgfsetbuttcap%
\pgfsetroundjoin%
\definecolor{currentfill}{rgb}{0.000000,0.000000,0.000000}%
\pgfsetfillcolor{currentfill}%
\pgfsetlinewidth{0.803000pt}%
\definecolor{currentstroke}{rgb}{0.000000,0.000000,0.000000}%
\pgfsetstrokecolor{currentstroke}%
\pgfsetdash{}{0pt}%
\pgfsys@defobject{currentmarker}{\pgfqpoint{-0.048611in}{0.000000in}}{\pgfqpoint{0.000000in}{0.000000in}}{%
\pgfpathmoveto{\pgfqpoint{0.000000in}{0.000000in}}%
\pgfpathlineto{\pgfqpoint{-0.048611in}{0.000000in}}%
\pgfusepath{stroke,fill}%
}%
\begin{pgfscope}%
\pgfsys@transformshift{0.725000in}{2.499750in}%
\pgfsys@useobject{currentmarker}{}%
\end{pgfscope}%
\end{pgfscope}%
\begin{pgfscope}%
\definecolor{textcolor}{rgb}{0.000000,0.000000,0.000000}%
\pgfsetstrokecolor{textcolor}%
\pgfsetfillcolor{textcolor}%
\pgftext[x=0.558333in,y=2.451556in,left,base]{\color{textcolor}\sffamily\fontsize{10.000000}{12.000000}\selectfont 7}%
\end{pgfscope}%
\begin{pgfscope}%
\pgfsetbuttcap%
\pgfsetroundjoin%
\definecolor{currentfill}{rgb}{0.000000,0.000000,0.000000}%
\pgfsetfillcolor{currentfill}%
\pgfsetlinewidth{0.803000pt}%
\definecolor{currentstroke}{rgb}{0.000000,0.000000,0.000000}%
\pgfsetstrokecolor{currentstroke}%
\pgfsetdash{}{0pt}%
\pgfsys@defobject{currentmarker}{\pgfqpoint{-0.048611in}{0.000000in}}{\pgfqpoint{0.000000in}{0.000000in}}{%
\pgfpathmoveto{\pgfqpoint{0.000000in}{0.000000in}}%
\pgfpathlineto{\pgfqpoint{-0.048611in}{0.000000in}}%
\pgfusepath{stroke,fill}%
}%
\begin{pgfscope}%
\pgfsys@transformshift{0.725000in}{2.788500in}%
\pgfsys@useobject{currentmarker}{}%
\end{pgfscope}%
\end{pgfscope}%
\begin{pgfscope}%
\definecolor{textcolor}{rgb}{0.000000,0.000000,0.000000}%
\pgfsetstrokecolor{textcolor}%
\pgfsetfillcolor{textcolor}%
\pgftext[x=0.558333in,y=2.740306in,left,base]{\color{textcolor}\sffamily\fontsize{10.000000}{12.000000}\selectfont 8}%
\end{pgfscope}%
\begin{pgfscope}%
\pgfpathrectangle{\pgfqpoint{0.725000in}{0.363000in}}{\pgfqpoint{4.495000in}{2.541000in}}%
\pgfusepath{clip}%
\pgfsetrectcap%
\pgfsetroundjoin%
\pgfsetlinewidth{1.505625pt}%
\definecolor{currentstroke}{rgb}{0.121569,0.466667,0.705882}%
\pgfsetstrokecolor{currentstroke}%
\pgfsetdash{}{0pt}%
\pgfpathmoveto{\pgfqpoint{0.929318in}{0.478500in}}%
\pgfpathlineto{\pgfqpoint{0.945343in}{0.504072in}}%
\pgfpathlineto{\pgfqpoint{0.961368in}{0.514664in}}%
\pgfpathlineto{\pgfqpoint{0.977393in}{0.522792in}}%
\pgfpathlineto{\pgfqpoint{1.009443in}{0.535681in}}%
\pgfpathlineto{\pgfqpoint{1.041493in}{0.546157in}}%
\pgfpathlineto{\pgfqpoint{1.089568in}{0.559366in}}%
\pgfpathlineto{\pgfqpoint{1.153668in}{0.574182in}}%
\pgfpathlineto{\pgfqpoint{1.233792in}{0.589966in}}%
\pgfpathlineto{\pgfqpoint{1.329942in}{0.606361in}}%
\pgfpathlineto{\pgfqpoint{1.442117in}{0.623158in}}%
\pgfpathlineto{\pgfqpoint{1.586341in}{0.642242in}}%
\pgfpathlineto{\pgfqpoint{1.746591in}{0.661122in}}%
\pgfpathlineto{\pgfqpoint{1.938890in}{0.681472in}}%
\pgfpathlineto{\pgfqpoint{2.163240in}{0.702895in}}%
\pgfpathlineto{\pgfqpoint{2.419639in}{0.725109in}}%
\pgfpathlineto{\pgfqpoint{2.724113in}{0.749130in}}%
\pgfpathlineto{\pgfqpoint{3.060637in}{0.773412in}}%
\pgfpathlineto{\pgfqpoint{3.445236in}{0.798918in}}%
\pgfpathlineto{\pgfqpoint{3.877910in}{0.825377in}}%
\pgfpathlineto{\pgfqpoint{4.374684in}{0.853461in}}%
\pgfpathlineto{\pgfqpoint{4.919532in}{0.882021in}}%
\pgfpathlineto{\pgfqpoint{5.015682in}{0.886854in}}%
\pgfpathlineto{\pgfqpoint{5.015682in}{0.886854in}}%
\pgfusepath{stroke}%
\end{pgfscope}%
\begin{pgfscope}%
\pgfpathrectangle{\pgfqpoint{0.725000in}{0.363000in}}{\pgfqpoint{4.495000in}{2.541000in}}%
\pgfusepath{clip}%
\pgfsetrectcap%
\pgfsetroundjoin%
\pgfsetlinewidth{1.505625pt}%
\definecolor{currentstroke}{rgb}{1.000000,0.498039,0.054902}%
\pgfsetstrokecolor{currentstroke}%
\pgfsetdash{}{0pt}%
\pgfpathmoveto{\pgfqpoint{0.929318in}{0.767250in}}%
\pgfpathlineto{\pgfqpoint{5.015682in}{0.767250in}}%
\pgfpathlineto{\pgfqpoint{5.015682in}{0.767250in}}%
\pgfusepath{stroke}%
\end{pgfscope}%
\begin{pgfscope}%
\pgfpathrectangle{\pgfqpoint{0.725000in}{0.363000in}}{\pgfqpoint{4.495000in}{2.541000in}}%
\pgfusepath{clip}%
\pgfsetrectcap%
\pgfsetroundjoin%
\pgfsetlinewidth{1.505625pt}%
\definecolor{currentstroke}{rgb}{0.172549,0.627451,0.172549}%
\pgfsetstrokecolor{currentstroke}%
\pgfsetdash{}{0pt}%
\pgfpathmoveto{\pgfqpoint{0.929318in}{0.478500in}}%
\pgfpathlineto{\pgfqpoint{5.015682in}{1.056000in}}%
\pgfpathlineto{\pgfqpoint{5.015682in}{1.056000in}}%
\pgfusepath{stroke}%
\end{pgfscope}%
\begin{pgfscope}%
\pgfpathrectangle{\pgfqpoint{0.725000in}{0.363000in}}{\pgfqpoint{4.495000in}{2.541000in}}%
\pgfusepath{clip}%
\pgfsetrectcap%
\pgfsetroundjoin%
\pgfsetlinewidth{1.505625pt}%
\definecolor{currentstroke}{rgb}{0.839216,0.152941,0.156863}%
\pgfsetstrokecolor{currentstroke}%
\pgfsetdash{}{0pt}%
\pgfpathmoveto{\pgfqpoint{0.929318in}{0.478500in}}%
\pgfpathlineto{\pgfqpoint{1.057518in}{0.479637in}}%
\pgfpathlineto{\pgfqpoint{1.185717in}{0.483047in}}%
\pgfpathlineto{\pgfqpoint{1.313917in}{0.488731in}}%
\pgfpathlineto{\pgfqpoint{1.442117in}{0.496689in}}%
\pgfpathlineto{\pgfqpoint{1.570316in}{0.506920in}}%
\pgfpathlineto{\pgfqpoint{1.698516in}{0.519425in}}%
\pgfpathlineto{\pgfqpoint{1.826716in}{0.534203in}}%
\pgfpathlineto{\pgfqpoint{1.954915in}{0.551255in}}%
\pgfpathlineto{\pgfqpoint{2.083115in}{0.570580in}}%
\pgfpathlineto{\pgfqpoint{2.211315in}{0.592179in}}%
\pgfpathlineto{\pgfqpoint{2.339514in}{0.616052in}}%
\pgfpathlineto{\pgfqpoint{2.467714in}{0.642198in}}%
\pgfpathlineto{\pgfqpoint{2.595914in}{0.670618in}}%
\pgfpathlineto{\pgfqpoint{2.724113in}{0.701312in}}%
\pgfpathlineto{\pgfqpoint{2.852313in}{0.734279in}}%
\pgfpathlineto{\pgfqpoint{2.980512in}{0.769519in}}%
\pgfpathlineto{\pgfqpoint{3.108712in}{0.807033in}}%
\pgfpathlineto{\pgfqpoint{3.236912in}{0.846821in}}%
\pgfpathlineto{\pgfqpoint{3.365111in}{0.888882in}}%
\pgfpathlineto{\pgfqpoint{3.493311in}{0.933217in}}%
\pgfpathlineto{\pgfqpoint{3.621511in}{0.979826in}}%
\pgfpathlineto{\pgfqpoint{3.749710in}{1.028708in}}%
\pgfpathlineto{\pgfqpoint{3.877910in}{1.079864in}}%
\pgfpathlineto{\pgfqpoint{4.006110in}{1.133293in}}%
\pgfpathlineto{\pgfqpoint{4.134309in}{1.188996in}}%
\pgfpathlineto{\pgfqpoint{4.262509in}{1.246972in}}%
\pgfpathlineto{\pgfqpoint{4.390709in}{1.307222in}}%
\pgfpathlineto{\pgfqpoint{4.518908in}{1.369746in}}%
\pgfpathlineto{\pgfqpoint{4.647108in}{1.434543in}}%
\pgfpathlineto{\pgfqpoint{4.775307in}{1.501614in}}%
\pgfpathlineto{\pgfqpoint{4.903507in}{1.570959in}}%
\pgfpathlineto{\pgfqpoint{5.015682in}{1.633500in}}%
\pgfpathlineto{\pgfqpoint{5.015682in}{1.633500in}}%
\pgfusepath{stroke}%
\end{pgfscope}%
\begin{pgfscope}%
\pgfpathrectangle{\pgfqpoint{0.725000in}{0.363000in}}{\pgfqpoint{4.495000in}{2.541000in}}%
\pgfusepath{clip}%
\pgfsetrectcap%
\pgfsetroundjoin%
\pgfsetlinewidth{1.505625pt}%
\definecolor{currentstroke}{rgb}{0.580392,0.403922,0.741176}%
\pgfsetstrokecolor{currentstroke}%
\pgfsetdash{}{0pt}%
\pgfpathmoveto{\pgfqpoint{0.929318in}{0.478500in}}%
\pgfpathlineto{\pgfqpoint{1.233792in}{0.479456in}}%
\pgfpathlineto{\pgfqpoint{1.410067in}{0.482261in}}%
\pgfpathlineto{\pgfqpoint{1.554291in}{0.486764in}}%
\pgfpathlineto{\pgfqpoint{1.682491in}{0.492964in}}%
\pgfpathlineto{\pgfqpoint{1.794666in}{0.500437in}}%
\pgfpathlineto{\pgfqpoint{1.906840in}{0.510121in}}%
\pgfpathlineto{\pgfqpoint{2.002990in}{0.520400in}}%
\pgfpathlineto{\pgfqpoint{2.099140in}{0.532695in}}%
\pgfpathlineto{\pgfqpoint{2.195290in}{0.547187in}}%
\pgfpathlineto{\pgfqpoint{2.291439in}{0.564056in}}%
\pgfpathlineto{\pgfqpoint{2.387589in}{0.583482in}}%
\pgfpathlineto{\pgfqpoint{2.467714in}{0.601755in}}%
\pgfpathlineto{\pgfqpoint{2.547839in}{0.622034in}}%
\pgfpathlineto{\pgfqpoint{2.627963in}{0.644424in}}%
\pgfpathlineto{\pgfqpoint{2.708088in}{0.669029in}}%
\pgfpathlineto{\pgfqpoint{2.788213in}{0.695953in}}%
\pgfpathlineto{\pgfqpoint{2.868338in}{0.725301in}}%
\pgfpathlineto{\pgfqpoint{2.948463in}{0.757178in}}%
\pgfpathlineto{\pgfqpoint{3.028587in}{0.791688in}}%
\pgfpathlineto{\pgfqpoint{3.108712in}{0.828936in}}%
\pgfpathlineto{\pgfqpoint{3.188837in}{0.869025in}}%
\pgfpathlineto{\pgfqpoint{3.268962in}{0.912061in}}%
\pgfpathlineto{\pgfqpoint{3.349086in}{0.958148in}}%
\pgfpathlineto{\pgfqpoint{3.429211in}{1.007390in}}%
\pgfpathlineto{\pgfqpoint{3.493311in}{1.049126in}}%
\pgfpathlineto{\pgfqpoint{3.557411in}{1.093002in}}%
\pgfpathlineto{\pgfqpoint{3.621511in}{1.139071in}}%
\pgfpathlineto{\pgfqpoint{3.685611in}{1.187387in}}%
\pgfpathlineto{\pgfqpoint{3.749710in}{1.238003in}}%
\pgfpathlineto{\pgfqpoint{3.813810in}{1.290973in}}%
\pgfpathlineto{\pgfqpoint{3.877910in}{1.346350in}}%
\pgfpathlineto{\pgfqpoint{3.942010in}{1.404189in}}%
\pgfpathlineto{\pgfqpoint{4.006110in}{1.464541in}}%
\pgfpathlineto{\pgfqpoint{4.070209in}{1.527462in}}%
\pgfpathlineto{\pgfqpoint{4.134309in}{1.593003in}}%
\pgfpathlineto{\pgfqpoint{4.198409in}{1.661220in}}%
\pgfpathlineto{\pgfqpoint{4.262509in}{1.732165in}}%
\pgfpathlineto{\pgfqpoint{4.326609in}{1.805891in}}%
\pgfpathlineto{\pgfqpoint{4.390709in}{1.882453in}}%
\pgfpathlineto{\pgfqpoint{4.454808in}{1.961904in}}%
\pgfpathlineto{\pgfqpoint{4.518908in}{2.044297in}}%
\pgfpathlineto{\pgfqpoint{4.583008in}{2.129686in}}%
\pgfpathlineto{\pgfqpoint{4.647108in}{2.218124in}}%
\pgfpathlineto{\pgfqpoint{4.711208in}{2.309665in}}%
\pgfpathlineto{\pgfqpoint{4.775307in}{2.404362in}}%
\pgfpathlineto{\pgfqpoint{4.839407in}{2.502269in}}%
\pgfpathlineto{\pgfqpoint{4.919532in}{2.629248in}}%
\pgfpathlineto{\pgfqpoint{4.999657in}{2.761430in}}%
\pgfpathlineto{\pgfqpoint{5.015682in}{2.788500in}}%
\pgfpathlineto{\pgfqpoint{5.015682in}{2.788500in}}%
\pgfusepath{stroke}%
\end{pgfscope}%
\begin{pgfscope}%
\pgfsetrectcap%
\pgfsetmiterjoin%
\pgfsetlinewidth{0.803000pt}%
\definecolor{currentstroke}{rgb}{0.000000,0.000000,0.000000}%
\pgfsetstrokecolor{currentstroke}%
\pgfsetdash{}{0pt}%
\pgfpathmoveto{\pgfqpoint{0.725000in}{0.363000in}}%
\pgfpathlineto{\pgfqpoint{0.725000in}{2.904000in}}%
\pgfusepath{stroke}%
\end{pgfscope}%
\begin{pgfscope}%
\pgfsetrectcap%
\pgfsetmiterjoin%
\pgfsetlinewidth{0.803000pt}%
\definecolor{currentstroke}{rgb}{0.000000,0.000000,0.000000}%
\pgfsetstrokecolor{currentstroke}%
\pgfsetdash{}{0pt}%
\pgfpathmoveto{\pgfqpoint{5.220000in}{0.363000in}}%
\pgfpathlineto{\pgfqpoint{5.220000in}{2.904000in}}%
\pgfusepath{stroke}%
\end{pgfscope}%
\begin{pgfscope}%
\pgfsetrectcap%
\pgfsetmiterjoin%
\pgfsetlinewidth{0.803000pt}%
\definecolor{currentstroke}{rgb}{0.000000,0.000000,0.000000}%
\pgfsetstrokecolor{currentstroke}%
\pgfsetdash{}{0pt}%
\pgfpathmoveto{\pgfqpoint{0.725000in}{0.363000in}}%
\pgfpathlineto{\pgfqpoint{5.220000in}{0.363000in}}%
\pgfusepath{stroke}%
\end{pgfscope}%
\begin{pgfscope}%
\pgfsetrectcap%
\pgfsetmiterjoin%
\pgfsetlinewidth{0.803000pt}%
\definecolor{currentstroke}{rgb}{0.000000,0.000000,0.000000}%
\pgfsetstrokecolor{currentstroke}%
\pgfsetdash{}{0pt}%
\pgfpathmoveto{\pgfqpoint{0.725000in}{2.904000in}}%
\pgfpathlineto{\pgfqpoint{5.220000in}{2.904000in}}%
\pgfusepath{stroke}%
\end{pgfscope}%
\begin{pgfscope}%
\pgfsetbuttcap%
\pgfsetmiterjoin%
\definecolor{currentfill}{rgb}{1.000000,1.000000,1.000000}%
\pgfsetfillcolor{currentfill}%
\pgfsetfillopacity{0.800000}%
\pgfsetlinewidth{1.003750pt}%
\definecolor{currentstroke}{rgb}{0.800000,0.800000,0.800000}%
\pgfsetstrokecolor{currentstroke}%
\pgfsetstrokeopacity{0.800000}%
\pgfsetdash{}{0pt}%
\pgfpathmoveto{\pgfqpoint{0.822222in}{1.824834in}}%
\pgfpathlineto{\pgfqpoint{1.496702in}{1.824834in}}%
\pgfpathquadraticcurveto{\pgfqpoint{1.524480in}{1.824834in}}{\pgfqpoint{1.524480in}{1.852612in}}%
\pgfpathlineto{\pgfqpoint{1.524480in}{2.806778in}}%
\pgfpathquadraticcurveto{\pgfqpoint{1.524480in}{2.834556in}}{\pgfqpoint{1.496702in}{2.834556in}}%
\pgfpathlineto{\pgfqpoint{0.822222in}{2.834556in}}%
\pgfpathquadraticcurveto{\pgfqpoint{0.794444in}{2.834556in}}{\pgfqpoint{0.794444in}{2.806778in}}%
\pgfpathlineto{\pgfqpoint{0.794444in}{1.852612in}}%
\pgfpathquadraticcurveto{\pgfqpoint{0.794444in}{1.824834in}}{\pgfqpoint{0.822222in}{1.824834in}}%
\pgfpathclose%
\pgfusepath{stroke,fill}%
\end{pgfscope}%
\begin{pgfscope}%
\pgfsetrectcap%
\pgfsetroundjoin%
\pgfsetlinewidth{1.505625pt}%
\definecolor{currentstroke}{rgb}{0.121569,0.466667,0.705882}%
\pgfsetstrokecolor{currentstroke}%
\pgfsetdash{}{0pt}%
\pgfpathmoveto{\pgfqpoint{0.850000in}{2.730389in}}%
\pgfpathlineto{\pgfqpoint{1.127778in}{2.730389in}}%
\pgfusepath{stroke}%
\end{pgfscope}%
\begin{pgfscope}%
\definecolor{textcolor}{rgb}{0.000000,0.000000,0.000000}%
\pgfsetstrokecolor{textcolor}%
\pgfsetfillcolor{textcolor}%
\pgftext[x=1.238889in,y=2.681778in,left,base]{\color{textcolor}\sffamily\fontsize{10.000000}{12.000000}\selectfont \(\displaystyle x^{0.5}\)}%
\end{pgfscope}%
\begin{pgfscope}%
\pgfsetrectcap%
\pgfsetroundjoin%
\pgfsetlinewidth{1.505625pt}%
\definecolor{currentstroke}{rgb}{1.000000,0.498039,0.054902}%
\pgfsetstrokecolor{currentstroke}%
\pgfsetdash{}{0pt}%
\pgfpathmoveto{\pgfqpoint{0.850000in}{2.536778in}}%
\pgfpathlineto{\pgfqpoint{1.127778in}{2.536778in}}%
\pgfusepath{stroke}%
\end{pgfscope}%
\begin{pgfscope}%
\definecolor{textcolor}{rgb}{0.000000,0.000000,0.000000}%
\pgfsetstrokecolor{textcolor}%
\pgfsetfillcolor{textcolor}%
\pgftext[x=1.238889in,y=2.488167in,left,base]{\color{textcolor}\sffamily\fontsize{10.000000}{12.000000}\selectfont \(\displaystyle x^{0}\)}%
\end{pgfscope}%
\begin{pgfscope}%
\pgfsetrectcap%
\pgfsetroundjoin%
\pgfsetlinewidth{1.505625pt}%
\definecolor{currentstroke}{rgb}{0.172549,0.627451,0.172549}%
\pgfsetstrokecolor{currentstroke}%
\pgfsetdash{}{0pt}%
\pgfpathmoveto{\pgfqpoint{0.850000in}{2.343167in}}%
\pgfpathlineto{\pgfqpoint{1.127778in}{2.343167in}}%
\pgfusepath{stroke}%
\end{pgfscope}%
\begin{pgfscope}%
\definecolor{textcolor}{rgb}{0.000000,0.000000,0.000000}%
\pgfsetstrokecolor{textcolor}%
\pgfsetfillcolor{textcolor}%
\pgftext[x=1.238889in,y=2.294556in,left,base]{\color{textcolor}\sffamily\fontsize{10.000000}{12.000000}\selectfont \(\displaystyle x^{1}\)}%
\end{pgfscope}%
\begin{pgfscope}%
\pgfsetrectcap%
\pgfsetroundjoin%
\pgfsetlinewidth{1.505625pt}%
\definecolor{currentstroke}{rgb}{0.839216,0.152941,0.156863}%
\pgfsetstrokecolor{currentstroke}%
\pgfsetdash{}{0pt}%
\pgfpathmoveto{\pgfqpoint{0.850000in}{2.149556in}}%
\pgfpathlineto{\pgfqpoint{1.127778in}{2.149556in}}%
\pgfusepath{stroke}%
\end{pgfscope}%
\begin{pgfscope}%
\definecolor{textcolor}{rgb}{0.000000,0.000000,0.000000}%
\pgfsetstrokecolor{textcolor}%
\pgfsetfillcolor{textcolor}%
\pgftext[x=1.238889in,y=2.100945in,left,base]{\color{textcolor}\sffamily\fontsize{10.000000}{12.000000}\selectfont \(\displaystyle x^{2}\)}%
\end{pgfscope}%
\begin{pgfscope}%
\pgfsetrectcap%
\pgfsetroundjoin%
\pgfsetlinewidth{1.505625pt}%
\definecolor{currentstroke}{rgb}{0.580392,0.403922,0.741176}%
\pgfsetstrokecolor{currentstroke}%
\pgfsetdash{}{0pt}%
\pgfpathmoveto{\pgfqpoint{0.850000in}{1.955945in}}%
\pgfpathlineto{\pgfqpoint{1.127778in}{1.955945in}}%
\pgfusepath{stroke}%
\end{pgfscope}%
\begin{pgfscope}%
\definecolor{textcolor}{rgb}{0.000000,0.000000,0.000000}%
\pgfsetstrokecolor{textcolor}%
\pgfsetfillcolor{textcolor}%
\pgftext[x=1.238889in,y=1.907334in,left,base]{\color{textcolor}\sffamily\fontsize{10.000000}{12.000000}\selectfont \(\displaystyle x^{3}\)}%
\end{pgfscope}%
\end{pgfpicture}%
\makeatother%
\endgroup%

    \caption{Die verschiedenen zu analysierenden polynomialen Signale\label{polynomials:signals}}
\end{figure}

In~\autoref{polynomials:haar} sind die Approximations- und die Detailkoeffizienten
der Transformation zu sehen. Wir sehen, dass die Approximationskoeffizienten
uns das grobe Signal und im Fall von $x^0 = 1$ sogar das exakte Signal liefern. Die
Detailkoeffizienten scheinen uns etwas zu liefern was proportional zur ersten
Ableitung des Signals ist.

\begin{figure}
    \centering
    %% Creator: Matplotlib, PGF backend
%%
%% To include the figure in your LaTeX document, write
%%   \input{<filename>.pgf}
%%
%% Make sure the required packages are loaded in your preamble
%%   \usepackage{pgf}
%%
%% Figures using additional raster images can only be included by \input if
%% they are in the same directory as the main LaTeX file. For loading figures
%% from other directories you can use the `import` package
%%   \usepackage{import}
%% and then include the figures with
%%   \import{<path to file>}{<filename>.pgf}
%%
%% Matplotlib used the following preamble
%%   \usepackage{fontspec}
%%
\begingroup%
\makeatletter%
\begin{pgfpicture}%
\pgfpathrectangle{\pgfpointorigin}{\pgfqpoint{5.800000in}{3.300000in}}%
\pgfusepath{use as bounding box, clip}%
\begin{pgfscope}%
\pgfsetbuttcap%
\pgfsetmiterjoin%
\definecolor{currentfill}{rgb}{1.000000,1.000000,1.000000}%
\pgfsetfillcolor{currentfill}%
\pgfsetlinewidth{0.000000pt}%
\definecolor{currentstroke}{rgb}{1.000000,1.000000,1.000000}%
\pgfsetstrokecolor{currentstroke}%
\pgfsetdash{}{0pt}%
\pgfpathmoveto{\pgfqpoint{0.000000in}{0.000000in}}%
\pgfpathlineto{\pgfqpoint{5.800000in}{0.000000in}}%
\pgfpathlineto{\pgfqpoint{5.800000in}{3.300000in}}%
\pgfpathlineto{\pgfqpoint{0.000000in}{3.300000in}}%
\pgfpathclose%
\pgfusepath{fill}%
\end{pgfscope}%
\begin{pgfscope}%
\pgfsetbuttcap%
\pgfsetmiterjoin%
\definecolor{currentfill}{rgb}{1.000000,1.000000,1.000000}%
\pgfsetfillcolor{currentfill}%
\pgfsetlinewidth{0.000000pt}%
\definecolor{currentstroke}{rgb}{0.000000,0.000000,0.000000}%
\pgfsetstrokecolor{currentstroke}%
\pgfsetstrokeopacity{0.000000}%
\pgfsetdash{}{0pt}%
\pgfpathmoveto{\pgfqpoint{0.670972in}{1.861111in}}%
\pgfpathlineto{\pgfqpoint{4.930000in}{1.861111in}}%
\pgfpathlineto{\pgfqpoint{4.930000in}{2.926667in}}%
\pgfpathlineto{\pgfqpoint{0.670972in}{2.926667in}}%
\pgfpathclose%
\pgfusepath{fill}%
\end{pgfscope}%
\begin{pgfscope}%
\pgfsetbuttcap%
\pgfsetroundjoin%
\definecolor{currentfill}{rgb}{0.000000,0.000000,0.000000}%
\pgfsetfillcolor{currentfill}%
\pgfsetlinewidth{0.803000pt}%
\definecolor{currentstroke}{rgb}{0.000000,0.000000,0.000000}%
\pgfsetstrokecolor{currentstroke}%
\pgfsetdash{}{0pt}%
\pgfsys@defobject{currentmarker}{\pgfqpoint{0.000000in}{-0.048611in}}{\pgfqpoint{0.000000in}{0.000000in}}{%
\pgfpathmoveto{\pgfqpoint{0.000000in}{0.000000in}}%
\pgfpathlineto{\pgfqpoint{0.000000in}{-0.048611in}}%
\pgfusepath{stroke,fill}%
}%
\begin{pgfscope}%
\pgfsys@transformshift{0.864564in}{1.861111in}%
\pgfsys@useobject{currentmarker}{}%
\end{pgfscope}%
\end{pgfscope}%
\begin{pgfscope}%
\pgfsetbuttcap%
\pgfsetroundjoin%
\definecolor{currentfill}{rgb}{0.000000,0.000000,0.000000}%
\pgfsetfillcolor{currentfill}%
\pgfsetlinewidth{0.803000pt}%
\definecolor{currentstroke}{rgb}{0.000000,0.000000,0.000000}%
\pgfsetstrokecolor{currentstroke}%
\pgfsetdash{}{0pt}%
\pgfsys@defobject{currentmarker}{\pgfqpoint{0.000000in}{-0.048611in}}{\pgfqpoint{0.000000in}{0.000000in}}{%
\pgfpathmoveto{\pgfqpoint{0.000000in}{0.000000in}}%
\pgfpathlineto{\pgfqpoint{0.000000in}{-0.048611in}}%
\pgfusepath{stroke,fill}%
}%
\begin{pgfscope}%
\pgfsys@transformshift{1.474304in}{1.861111in}%
\pgfsys@useobject{currentmarker}{}%
\end{pgfscope}%
\end{pgfscope}%
\begin{pgfscope}%
\pgfsetbuttcap%
\pgfsetroundjoin%
\definecolor{currentfill}{rgb}{0.000000,0.000000,0.000000}%
\pgfsetfillcolor{currentfill}%
\pgfsetlinewidth{0.803000pt}%
\definecolor{currentstroke}{rgb}{0.000000,0.000000,0.000000}%
\pgfsetstrokecolor{currentstroke}%
\pgfsetdash{}{0pt}%
\pgfsys@defobject{currentmarker}{\pgfqpoint{0.000000in}{-0.048611in}}{\pgfqpoint{0.000000in}{0.000000in}}{%
\pgfpathmoveto{\pgfqpoint{0.000000in}{0.000000in}}%
\pgfpathlineto{\pgfqpoint{0.000000in}{-0.048611in}}%
\pgfusepath{stroke,fill}%
}%
\begin{pgfscope}%
\pgfsys@transformshift{2.084043in}{1.861111in}%
\pgfsys@useobject{currentmarker}{}%
\end{pgfscope}%
\end{pgfscope}%
\begin{pgfscope}%
\pgfsetbuttcap%
\pgfsetroundjoin%
\definecolor{currentfill}{rgb}{0.000000,0.000000,0.000000}%
\pgfsetfillcolor{currentfill}%
\pgfsetlinewidth{0.803000pt}%
\definecolor{currentstroke}{rgb}{0.000000,0.000000,0.000000}%
\pgfsetstrokecolor{currentstroke}%
\pgfsetdash{}{0pt}%
\pgfsys@defobject{currentmarker}{\pgfqpoint{0.000000in}{-0.048611in}}{\pgfqpoint{0.000000in}{0.000000in}}{%
\pgfpathmoveto{\pgfqpoint{0.000000in}{0.000000in}}%
\pgfpathlineto{\pgfqpoint{0.000000in}{-0.048611in}}%
\pgfusepath{stroke,fill}%
}%
\begin{pgfscope}%
\pgfsys@transformshift{2.693782in}{1.861111in}%
\pgfsys@useobject{currentmarker}{}%
\end{pgfscope}%
\end{pgfscope}%
\begin{pgfscope}%
\pgfsetbuttcap%
\pgfsetroundjoin%
\definecolor{currentfill}{rgb}{0.000000,0.000000,0.000000}%
\pgfsetfillcolor{currentfill}%
\pgfsetlinewidth{0.803000pt}%
\definecolor{currentstroke}{rgb}{0.000000,0.000000,0.000000}%
\pgfsetstrokecolor{currentstroke}%
\pgfsetdash{}{0pt}%
\pgfsys@defobject{currentmarker}{\pgfqpoint{0.000000in}{-0.048611in}}{\pgfqpoint{0.000000in}{0.000000in}}{%
\pgfpathmoveto{\pgfqpoint{0.000000in}{0.000000in}}%
\pgfpathlineto{\pgfqpoint{0.000000in}{-0.048611in}}%
\pgfusepath{stroke,fill}%
}%
\begin{pgfscope}%
\pgfsys@transformshift{3.303521in}{1.861111in}%
\pgfsys@useobject{currentmarker}{}%
\end{pgfscope}%
\end{pgfscope}%
\begin{pgfscope}%
\pgfsetbuttcap%
\pgfsetroundjoin%
\definecolor{currentfill}{rgb}{0.000000,0.000000,0.000000}%
\pgfsetfillcolor{currentfill}%
\pgfsetlinewidth{0.803000pt}%
\definecolor{currentstroke}{rgb}{0.000000,0.000000,0.000000}%
\pgfsetstrokecolor{currentstroke}%
\pgfsetdash{}{0pt}%
\pgfsys@defobject{currentmarker}{\pgfqpoint{0.000000in}{-0.048611in}}{\pgfqpoint{0.000000in}{0.000000in}}{%
\pgfpathmoveto{\pgfqpoint{0.000000in}{0.000000in}}%
\pgfpathlineto{\pgfqpoint{0.000000in}{-0.048611in}}%
\pgfusepath{stroke,fill}%
}%
\begin{pgfscope}%
\pgfsys@transformshift{3.913260in}{1.861111in}%
\pgfsys@useobject{currentmarker}{}%
\end{pgfscope}%
\end{pgfscope}%
\begin{pgfscope}%
\pgfsetbuttcap%
\pgfsetroundjoin%
\definecolor{currentfill}{rgb}{0.000000,0.000000,0.000000}%
\pgfsetfillcolor{currentfill}%
\pgfsetlinewidth{0.803000pt}%
\definecolor{currentstroke}{rgb}{0.000000,0.000000,0.000000}%
\pgfsetstrokecolor{currentstroke}%
\pgfsetdash{}{0pt}%
\pgfsys@defobject{currentmarker}{\pgfqpoint{0.000000in}{-0.048611in}}{\pgfqpoint{0.000000in}{0.000000in}}{%
\pgfpathmoveto{\pgfqpoint{0.000000in}{0.000000in}}%
\pgfpathlineto{\pgfqpoint{0.000000in}{-0.048611in}}%
\pgfusepath{stroke,fill}%
}%
\begin{pgfscope}%
\pgfsys@transformshift{4.522999in}{1.861111in}%
\pgfsys@useobject{currentmarker}{}%
\end{pgfscope}%
\end{pgfscope}%
\begin{pgfscope}%
\pgfsetbuttcap%
\pgfsetroundjoin%
\definecolor{currentfill}{rgb}{0.000000,0.000000,0.000000}%
\pgfsetfillcolor{currentfill}%
\pgfsetlinewidth{0.803000pt}%
\definecolor{currentstroke}{rgb}{0.000000,0.000000,0.000000}%
\pgfsetstrokecolor{currentstroke}%
\pgfsetdash{}{0pt}%
\pgfsys@defobject{currentmarker}{\pgfqpoint{-0.048611in}{0.000000in}}{\pgfqpoint{0.000000in}{0.000000in}}{%
\pgfpathmoveto{\pgfqpoint{0.000000in}{0.000000in}}%
\pgfpathlineto{\pgfqpoint{-0.048611in}{0.000000in}}%
\pgfusepath{stroke,fill}%
}%
\begin{pgfscope}%
\pgfsys@transformshift{0.670972in}{1.909545in}%
\pgfsys@useobject{currentmarker}{}%
\end{pgfscope}%
\end{pgfscope}%
\begin{pgfscope}%
\definecolor{textcolor}{rgb}{0.000000,0.000000,0.000000}%
\pgfsetstrokecolor{textcolor}%
\pgfsetfillcolor{textcolor}%
\pgftext[x=0.504306in,y=1.861351in,left,base]{\color{textcolor}\sffamily\fontsize{10.000000}{12.000000}\selectfont 0}%
\end{pgfscope}%
\begin{pgfscope}%
\pgfsetbuttcap%
\pgfsetroundjoin%
\definecolor{currentfill}{rgb}{0.000000,0.000000,0.000000}%
\pgfsetfillcolor{currentfill}%
\pgfsetlinewidth{0.803000pt}%
\definecolor{currentstroke}{rgb}{0.000000,0.000000,0.000000}%
\pgfsetstrokecolor{currentstroke}%
\pgfsetdash{}{0pt}%
\pgfsys@defobject{currentmarker}{\pgfqpoint{-0.048611in}{0.000000in}}{\pgfqpoint{0.000000in}{0.000000in}}{%
\pgfpathmoveto{\pgfqpoint{0.000000in}{0.000000in}}%
\pgfpathlineto{\pgfqpoint{-0.048611in}{0.000000in}}%
\pgfusepath{stroke,fill}%
}%
\begin{pgfscope}%
\pgfsys@transformshift{0.670972in}{2.340172in}%
\pgfsys@useobject{currentmarker}{}%
\end{pgfscope}%
\end{pgfscope}%
\begin{pgfscope}%
\definecolor{textcolor}{rgb}{0.000000,0.000000,0.000000}%
\pgfsetstrokecolor{textcolor}%
\pgfsetfillcolor{textcolor}%
\pgftext[x=0.504306in,y=2.291977in,left,base]{\color{textcolor}\sffamily\fontsize{10.000000}{12.000000}\selectfont 5}%
\end{pgfscope}%
\begin{pgfscope}%
\pgfsetbuttcap%
\pgfsetroundjoin%
\definecolor{currentfill}{rgb}{0.000000,0.000000,0.000000}%
\pgfsetfillcolor{currentfill}%
\pgfsetlinewidth{0.803000pt}%
\definecolor{currentstroke}{rgb}{0.000000,0.000000,0.000000}%
\pgfsetstrokecolor{currentstroke}%
\pgfsetdash{}{0pt}%
\pgfsys@defobject{currentmarker}{\pgfqpoint{-0.048611in}{0.000000in}}{\pgfqpoint{0.000000in}{0.000000in}}{%
\pgfpathmoveto{\pgfqpoint{0.000000in}{0.000000in}}%
\pgfpathlineto{\pgfqpoint{-0.048611in}{0.000000in}}%
\pgfusepath{stroke,fill}%
}%
\begin{pgfscope}%
\pgfsys@transformshift{0.670972in}{2.770798in}%
\pgfsys@useobject{currentmarker}{}%
\end{pgfscope}%
\end{pgfscope}%
\begin{pgfscope}%
\definecolor{textcolor}{rgb}{0.000000,0.000000,0.000000}%
\pgfsetstrokecolor{textcolor}%
\pgfsetfillcolor{textcolor}%
\pgftext[x=0.434861in,y=2.722604in,left,base]{\color{textcolor}\sffamily\fontsize{10.000000}{12.000000}\selectfont 10}%
\end{pgfscope}%
\begin{pgfscope}%
\pgfpathrectangle{\pgfqpoint{0.670972in}{1.861111in}}{\pgfqpoint{4.259028in}{1.065556in}}%
\pgfusepath{clip}%
\pgfsetrectcap%
\pgfsetroundjoin%
\pgfsetlinewidth{1.505625pt}%
\definecolor{currentstroke}{rgb}{0.121569,0.466667,0.705882}%
\pgfsetstrokecolor{currentstroke}%
\pgfsetdash{}{0pt}%
\pgfpathmoveto{\pgfqpoint{0.864564in}{1.914939in}}%
\pgfpathlineto{\pgfqpoint{0.895051in}{1.926514in}}%
\pgfpathlineto{\pgfqpoint{0.925538in}{1.932392in}}%
\pgfpathlineto{\pgfqpoint{0.986512in}{1.940980in}}%
\pgfpathlineto{\pgfqpoint{1.077973in}{1.950614in}}%
\pgfpathlineto{\pgfqpoint{1.199921in}{1.960708in}}%
\pgfpathlineto{\pgfqpoint{1.352356in}{1.971038in}}%
\pgfpathlineto{\pgfqpoint{1.565764in}{1.983100in}}%
\pgfpathlineto{\pgfqpoint{1.840147in}{1.996175in}}%
\pgfpathlineto{\pgfqpoint{2.175504in}{2.009868in}}%
\pgfpathlineto{\pgfqpoint{2.602321in}{2.024968in}}%
\pgfpathlineto{\pgfqpoint{3.120599in}{2.040993in}}%
\pgfpathlineto{\pgfqpoint{3.760825in}{2.058426in}}%
\pgfpathlineto{\pgfqpoint{4.522999in}{2.076827in}}%
\pgfpathlineto{\pgfqpoint{4.736408in}{2.081627in}}%
\pgfpathlineto{\pgfqpoint{4.736408in}{2.081627in}}%
\pgfusepath{stroke}%
\end{pgfscope}%
\begin{pgfscope}%
\pgfpathrectangle{\pgfqpoint{0.670972in}{1.861111in}}{\pgfqpoint{4.259028in}{1.065556in}}%
\pgfusepath{clip}%
\pgfsetrectcap%
\pgfsetroundjoin%
\pgfsetlinewidth{1.505625pt}%
\definecolor{currentstroke}{rgb}{1.000000,0.498039,0.054902}%
\pgfsetstrokecolor{currentstroke}%
\pgfsetdash{}{0pt}%
\pgfpathmoveto{\pgfqpoint{0.864564in}{2.031345in}}%
\pgfpathlineto{\pgfqpoint{4.736408in}{2.031345in}}%
\pgfpathlineto{\pgfqpoint{4.736408in}{2.031345in}}%
\pgfusepath{stroke}%
\end{pgfscope}%
\begin{pgfscope}%
\pgfpathrectangle{\pgfqpoint{0.670972in}{1.861111in}}{\pgfqpoint{4.259028in}{1.065556in}}%
\pgfusepath{clip}%
\pgfsetrectcap%
\pgfsetroundjoin%
\pgfsetlinewidth{1.505625pt}%
\definecolor{currentstroke}{rgb}{0.172549,0.627451,0.172549}%
\pgfsetstrokecolor{currentstroke}%
\pgfsetdash{}{0pt}%
\pgfpathmoveto{\pgfqpoint{0.864564in}{1.910023in}}%
\pgfpathlineto{\pgfqpoint{4.736408in}{2.152667in}}%
\pgfpathlineto{\pgfqpoint{4.736408in}{2.152667in}}%
\pgfusepath{stroke}%
\end{pgfscope}%
\begin{pgfscope}%
\pgfpathrectangle{\pgfqpoint{0.670972in}{1.861111in}}{\pgfqpoint{4.259028in}{1.065556in}}%
\pgfusepath{clip}%
\pgfsetrectcap%
\pgfsetroundjoin%
\pgfsetlinewidth{1.505625pt}%
\definecolor{currentstroke}{rgb}{0.839216,0.152941,0.156863}%
\pgfsetstrokecolor{currentstroke}%
\pgfsetdash{}{0pt}%
\pgfpathmoveto{\pgfqpoint{0.864564in}{1.909549in}}%
\pgfpathlineto{\pgfqpoint{1.047486in}{1.910718in}}%
\pgfpathlineto{\pgfqpoint{1.230408in}{1.914045in}}%
\pgfpathlineto{\pgfqpoint{1.413330in}{1.919529in}}%
\pgfpathlineto{\pgfqpoint{1.596251in}{1.927171in}}%
\pgfpathlineto{\pgfqpoint{1.779173in}{1.936972in}}%
\pgfpathlineto{\pgfqpoint{1.962095in}{1.948930in}}%
\pgfpathlineto{\pgfqpoint{2.145017in}{1.963045in}}%
\pgfpathlineto{\pgfqpoint{2.327938in}{1.979319in}}%
\pgfpathlineto{\pgfqpoint{2.510860in}{1.997751in}}%
\pgfpathlineto{\pgfqpoint{2.693782in}{2.018340in}}%
\pgfpathlineto{\pgfqpoint{2.876704in}{2.041087in}}%
\pgfpathlineto{\pgfqpoint{3.059625in}{2.065992in}}%
\pgfpathlineto{\pgfqpoint{3.242547in}{2.093055in}}%
\pgfpathlineto{\pgfqpoint{3.425469in}{2.122275in}}%
\pgfpathlineto{\pgfqpoint{3.608390in}{2.153654in}}%
\pgfpathlineto{\pgfqpoint{3.791312in}{2.187190in}}%
\pgfpathlineto{\pgfqpoint{3.974234in}{2.222884in}}%
\pgfpathlineto{\pgfqpoint{4.157156in}{2.260736in}}%
\pgfpathlineto{\pgfqpoint{4.340077in}{2.300746in}}%
\pgfpathlineto{\pgfqpoint{4.522999in}{2.342914in}}%
\pgfpathlineto{\pgfqpoint{4.705921in}{2.387239in}}%
\pgfpathlineto{\pgfqpoint{4.736408in}{2.394837in}}%
\pgfpathlineto{\pgfqpoint{4.736408in}{2.394837in}}%
\pgfusepath{stroke}%
\end{pgfscope}%
\begin{pgfscope}%
\pgfpathrectangle{\pgfqpoint{0.670972in}{1.861111in}}{\pgfqpoint{4.259028in}{1.065556in}}%
\pgfusepath{clip}%
\pgfsetrectcap%
\pgfsetroundjoin%
\pgfsetlinewidth{1.505625pt}%
\definecolor{currentstroke}{rgb}{0.580392,0.403922,0.741176}%
\pgfsetstrokecolor{currentstroke}%
\pgfsetdash{}{0pt}%
\pgfpathmoveto{\pgfqpoint{0.864564in}{1.909545in}}%
\pgfpathlineto{\pgfqpoint{1.260895in}{1.910640in}}%
\pgfpathlineto{\pgfqpoint{1.474304in}{1.913451in}}%
\pgfpathlineto{\pgfqpoint{1.657225in}{1.918051in}}%
\pgfpathlineto{\pgfqpoint{1.809660in}{1.923895in}}%
\pgfpathlineto{\pgfqpoint{1.962095in}{1.931942in}}%
\pgfpathlineto{\pgfqpoint{2.114530in}{1.942546in}}%
\pgfpathlineto{\pgfqpoint{2.236477in}{1.953107in}}%
\pgfpathlineto{\pgfqpoint{2.358425in}{1.965709in}}%
\pgfpathlineto{\pgfqpoint{2.480373in}{1.980535in}}%
\pgfpathlineto{\pgfqpoint{2.602321in}{1.997763in}}%
\pgfpathlineto{\pgfqpoint{2.724269in}{2.017576in}}%
\pgfpathlineto{\pgfqpoint{2.846217in}{2.040152in}}%
\pgfpathlineto{\pgfqpoint{2.968164in}{2.065674in}}%
\pgfpathlineto{\pgfqpoint{3.059625in}{2.086856in}}%
\pgfpathlineto{\pgfqpoint{3.151086in}{2.109872in}}%
\pgfpathlineto{\pgfqpoint{3.242547in}{2.134799in}}%
\pgfpathlineto{\pgfqpoint{3.334008in}{2.161712in}}%
\pgfpathlineto{\pgfqpoint{3.425469in}{2.190688in}}%
\pgfpathlineto{\pgfqpoint{3.516930in}{2.221802in}}%
\pgfpathlineto{\pgfqpoint{3.608390in}{2.255132in}}%
\pgfpathlineto{\pgfqpoint{3.699851in}{2.290752in}}%
\pgfpathlineto{\pgfqpoint{3.791312in}{2.328740in}}%
\pgfpathlineto{\pgfqpoint{3.882773in}{2.369172in}}%
\pgfpathlineto{\pgfqpoint{3.974234in}{2.412123in}}%
\pgfpathlineto{\pgfqpoint{4.065695in}{2.457669in}}%
\pgfpathlineto{\pgfqpoint{4.157156in}{2.505888in}}%
\pgfpathlineto{\pgfqpoint{4.248617in}{2.556854in}}%
\pgfpathlineto{\pgfqpoint{4.340077in}{2.610645in}}%
\pgfpathlineto{\pgfqpoint{4.431538in}{2.667337in}}%
\pgfpathlineto{\pgfqpoint{4.522999in}{2.727004in}}%
\pgfpathlineto{\pgfqpoint{4.614460in}{2.789725in}}%
\pgfpathlineto{\pgfqpoint{4.705921in}{2.855574in}}%
\pgfpathlineto{\pgfqpoint{4.736408in}{2.878232in}}%
\pgfpathlineto{\pgfqpoint{4.736408in}{2.878232in}}%
\pgfusepath{stroke}%
\end{pgfscope}%
\begin{pgfscope}%
\pgfsetrectcap%
\pgfsetmiterjoin%
\pgfsetlinewidth{0.803000pt}%
\definecolor{currentstroke}{rgb}{0.000000,0.000000,0.000000}%
\pgfsetstrokecolor{currentstroke}%
\pgfsetdash{}{0pt}%
\pgfpathmoveto{\pgfqpoint{0.670972in}{1.861111in}}%
\pgfpathlineto{\pgfqpoint{0.670972in}{2.926667in}}%
\pgfusepath{stroke}%
\end{pgfscope}%
\begin{pgfscope}%
\pgfsetrectcap%
\pgfsetmiterjoin%
\pgfsetlinewidth{0.803000pt}%
\definecolor{currentstroke}{rgb}{0.000000,0.000000,0.000000}%
\pgfsetstrokecolor{currentstroke}%
\pgfsetdash{}{0pt}%
\pgfpathmoveto{\pgfqpoint{4.930000in}{1.861111in}}%
\pgfpathlineto{\pgfqpoint{4.930000in}{2.926667in}}%
\pgfusepath{stroke}%
\end{pgfscope}%
\begin{pgfscope}%
\pgfsetrectcap%
\pgfsetmiterjoin%
\pgfsetlinewidth{0.803000pt}%
\definecolor{currentstroke}{rgb}{0.000000,0.000000,0.000000}%
\pgfsetstrokecolor{currentstroke}%
\pgfsetdash{}{0pt}%
\pgfpathmoveto{\pgfqpoint{0.670972in}{1.861111in}}%
\pgfpathlineto{\pgfqpoint{4.930000in}{1.861111in}}%
\pgfusepath{stroke}%
\end{pgfscope}%
\begin{pgfscope}%
\pgfsetrectcap%
\pgfsetmiterjoin%
\pgfsetlinewidth{0.803000pt}%
\definecolor{currentstroke}{rgb}{0.000000,0.000000,0.000000}%
\pgfsetstrokecolor{currentstroke}%
\pgfsetdash{}{0pt}%
\pgfpathmoveto{\pgfqpoint{0.670972in}{2.926667in}}%
\pgfpathlineto{\pgfqpoint{4.930000in}{2.926667in}}%
\pgfusepath{stroke}%
\end{pgfscope}%
\begin{pgfscope}%
\definecolor{textcolor}{rgb}{0.000000,0.000000,0.000000}%
\pgfsetstrokecolor{textcolor}%
\pgfsetfillcolor{textcolor}%
\pgftext[x=2.800486in,y=3.010000in,,base]{\color{textcolor}\sffamily\fontsize{12.000000}{14.400000}\selectfont Approximation Koeffizienten}%
\end{pgfscope}%
\begin{pgfscope}%
\pgfsetbuttcap%
\pgfsetmiterjoin%
\definecolor{currentfill}{rgb}{1.000000,1.000000,1.000000}%
\pgfsetfillcolor{currentfill}%
\pgfsetlinewidth{0.000000pt}%
\definecolor{currentstroke}{rgb}{0.000000,0.000000,0.000000}%
\pgfsetstrokecolor{currentstroke}%
\pgfsetstrokeopacity{0.000000}%
\pgfsetdash{}{0pt}%
\pgfpathmoveto{\pgfqpoint{0.670972in}{0.387222in}}%
\pgfpathlineto{\pgfqpoint{4.930000in}{0.387222in}}%
\pgfpathlineto{\pgfqpoint{4.930000in}{1.452778in}}%
\pgfpathlineto{\pgfqpoint{0.670972in}{1.452778in}}%
\pgfpathclose%
\pgfusepath{fill}%
\end{pgfscope}%
\begin{pgfscope}%
\pgfsetbuttcap%
\pgfsetroundjoin%
\definecolor{currentfill}{rgb}{0.000000,0.000000,0.000000}%
\pgfsetfillcolor{currentfill}%
\pgfsetlinewidth{0.803000pt}%
\definecolor{currentstroke}{rgb}{0.000000,0.000000,0.000000}%
\pgfsetstrokecolor{currentstroke}%
\pgfsetdash{}{0pt}%
\pgfsys@defobject{currentmarker}{\pgfqpoint{0.000000in}{-0.048611in}}{\pgfqpoint{0.000000in}{0.000000in}}{%
\pgfpathmoveto{\pgfqpoint{0.000000in}{0.000000in}}%
\pgfpathlineto{\pgfqpoint{0.000000in}{-0.048611in}}%
\pgfusepath{stroke,fill}%
}%
\begin{pgfscope}%
\pgfsys@transformshift{0.864564in}{0.387222in}%
\pgfsys@useobject{currentmarker}{}%
\end{pgfscope}%
\end{pgfscope}%
\begin{pgfscope}%
\definecolor{textcolor}{rgb}{0.000000,0.000000,0.000000}%
\pgfsetstrokecolor{textcolor}%
\pgfsetfillcolor{textcolor}%
\pgftext[x=0.864564in,y=0.290000in,,top]{\color{textcolor}\sffamily\fontsize{10.000000}{12.000000}\selectfont 0}%
\end{pgfscope}%
\begin{pgfscope}%
\pgfsetbuttcap%
\pgfsetroundjoin%
\definecolor{currentfill}{rgb}{0.000000,0.000000,0.000000}%
\pgfsetfillcolor{currentfill}%
\pgfsetlinewidth{0.803000pt}%
\definecolor{currentstroke}{rgb}{0.000000,0.000000,0.000000}%
\pgfsetstrokecolor{currentstroke}%
\pgfsetdash{}{0pt}%
\pgfsys@defobject{currentmarker}{\pgfqpoint{0.000000in}{-0.048611in}}{\pgfqpoint{0.000000in}{0.000000in}}{%
\pgfpathmoveto{\pgfqpoint{0.000000in}{0.000000in}}%
\pgfpathlineto{\pgfqpoint{0.000000in}{-0.048611in}}%
\pgfusepath{stroke,fill}%
}%
\begin{pgfscope}%
\pgfsys@transformshift{1.474304in}{0.387222in}%
\pgfsys@useobject{currentmarker}{}%
\end{pgfscope}%
\end{pgfscope}%
\begin{pgfscope}%
\definecolor{textcolor}{rgb}{0.000000,0.000000,0.000000}%
\pgfsetstrokecolor{textcolor}%
\pgfsetfillcolor{textcolor}%
\pgftext[x=1.474304in,y=0.290000in,,top]{\color{textcolor}\sffamily\fontsize{10.000000}{12.000000}\selectfont 20}%
\end{pgfscope}%
\begin{pgfscope}%
\pgfsetbuttcap%
\pgfsetroundjoin%
\definecolor{currentfill}{rgb}{0.000000,0.000000,0.000000}%
\pgfsetfillcolor{currentfill}%
\pgfsetlinewidth{0.803000pt}%
\definecolor{currentstroke}{rgb}{0.000000,0.000000,0.000000}%
\pgfsetstrokecolor{currentstroke}%
\pgfsetdash{}{0pt}%
\pgfsys@defobject{currentmarker}{\pgfqpoint{0.000000in}{-0.048611in}}{\pgfqpoint{0.000000in}{0.000000in}}{%
\pgfpathmoveto{\pgfqpoint{0.000000in}{0.000000in}}%
\pgfpathlineto{\pgfqpoint{0.000000in}{-0.048611in}}%
\pgfusepath{stroke,fill}%
}%
\begin{pgfscope}%
\pgfsys@transformshift{2.084043in}{0.387222in}%
\pgfsys@useobject{currentmarker}{}%
\end{pgfscope}%
\end{pgfscope}%
\begin{pgfscope}%
\definecolor{textcolor}{rgb}{0.000000,0.000000,0.000000}%
\pgfsetstrokecolor{textcolor}%
\pgfsetfillcolor{textcolor}%
\pgftext[x=2.084043in,y=0.290000in,,top]{\color{textcolor}\sffamily\fontsize{10.000000}{12.000000}\selectfont 40}%
\end{pgfscope}%
\begin{pgfscope}%
\pgfsetbuttcap%
\pgfsetroundjoin%
\definecolor{currentfill}{rgb}{0.000000,0.000000,0.000000}%
\pgfsetfillcolor{currentfill}%
\pgfsetlinewidth{0.803000pt}%
\definecolor{currentstroke}{rgb}{0.000000,0.000000,0.000000}%
\pgfsetstrokecolor{currentstroke}%
\pgfsetdash{}{0pt}%
\pgfsys@defobject{currentmarker}{\pgfqpoint{0.000000in}{-0.048611in}}{\pgfqpoint{0.000000in}{0.000000in}}{%
\pgfpathmoveto{\pgfqpoint{0.000000in}{0.000000in}}%
\pgfpathlineto{\pgfqpoint{0.000000in}{-0.048611in}}%
\pgfusepath{stroke,fill}%
}%
\begin{pgfscope}%
\pgfsys@transformshift{2.693782in}{0.387222in}%
\pgfsys@useobject{currentmarker}{}%
\end{pgfscope}%
\end{pgfscope}%
\begin{pgfscope}%
\definecolor{textcolor}{rgb}{0.000000,0.000000,0.000000}%
\pgfsetstrokecolor{textcolor}%
\pgfsetfillcolor{textcolor}%
\pgftext[x=2.693782in,y=0.290000in,,top]{\color{textcolor}\sffamily\fontsize{10.000000}{12.000000}\selectfont 60}%
\end{pgfscope}%
\begin{pgfscope}%
\pgfsetbuttcap%
\pgfsetroundjoin%
\definecolor{currentfill}{rgb}{0.000000,0.000000,0.000000}%
\pgfsetfillcolor{currentfill}%
\pgfsetlinewidth{0.803000pt}%
\definecolor{currentstroke}{rgb}{0.000000,0.000000,0.000000}%
\pgfsetstrokecolor{currentstroke}%
\pgfsetdash{}{0pt}%
\pgfsys@defobject{currentmarker}{\pgfqpoint{0.000000in}{-0.048611in}}{\pgfqpoint{0.000000in}{0.000000in}}{%
\pgfpathmoveto{\pgfqpoint{0.000000in}{0.000000in}}%
\pgfpathlineto{\pgfqpoint{0.000000in}{-0.048611in}}%
\pgfusepath{stroke,fill}%
}%
\begin{pgfscope}%
\pgfsys@transformshift{3.303521in}{0.387222in}%
\pgfsys@useobject{currentmarker}{}%
\end{pgfscope}%
\end{pgfscope}%
\begin{pgfscope}%
\definecolor{textcolor}{rgb}{0.000000,0.000000,0.000000}%
\pgfsetstrokecolor{textcolor}%
\pgfsetfillcolor{textcolor}%
\pgftext[x=3.303521in,y=0.290000in,,top]{\color{textcolor}\sffamily\fontsize{10.000000}{12.000000}\selectfont 80}%
\end{pgfscope}%
\begin{pgfscope}%
\pgfsetbuttcap%
\pgfsetroundjoin%
\definecolor{currentfill}{rgb}{0.000000,0.000000,0.000000}%
\pgfsetfillcolor{currentfill}%
\pgfsetlinewidth{0.803000pt}%
\definecolor{currentstroke}{rgb}{0.000000,0.000000,0.000000}%
\pgfsetstrokecolor{currentstroke}%
\pgfsetdash{}{0pt}%
\pgfsys@defobject{currentmarker}{\pgfqpoint{0.000000in}{-0.048611in}}{\pgfqpoint{0.000000in}{0.000000in}}{%
\pgfpathmoveto{\pgfqpoint{0.000000in}{0.000000in}}%
\pgfpathlineto{\pgfqpoint{0.000000in}{-0.048611in}}%
\pgfusepath{stroke,fill}%
}%
\begin{pgfscope}%
\pgfsys@transformshift{3.913260in}{0.387222in}%
\pgfsys@useobject{currentmarker}{}%
\end{pgfscope}%
\end{pgfscope}%
\begin{pgfscope}%
\definecolor{textcolor}{rgb}{0.000000,0.000000,0.000000}%
\pgfsetstrokecolor{textcolor}%
\pgfsetfillcolor{textcolor}%
\pgftext[x=3.913260in,y=0.290000in,,top]{\color{textcolor}\sffamily\fontsize{10.000000}{12.000000}\selectfont 100}%
\end{pgfscope}%
\begin{pgfscope}%
\pgfsetbuttcap%
\pgfsetroundjoin%
\definecolor{currentfill}{rgb}{0.000000,0.000000,0.000000}%
\pgfsetfillcolor{currentfill}%
\pgfsetlinewidth{0.803000pt}%
\definecolor{currentstroke}{rgb}{0.000000,0.000000,0.000000}%
\pgfsetstrokecolor{currentstroke}%
\pgfsetdash{}{0pt}%
\pgfsys@defobject{currentmarker}{\pgfqpoint{0.000000in}{-0.048611in}}{\pgfqpoint{0.000000in}{0.000000in}}{%
\pgfpathmoveto{\pgfqpoint{0.000000in}{0.000000in}}%
\pgfpathlineto{\pgfqpoint{0.000000in}{-0.048611in}}%
\pgfusepath{stroke,fill}%
}%
\begin{pgfscope}%
\pgfsys@transformshift{4.522999in}{0.387222in}%
\pgfsys@useobject{currentmarker}{}%
\end{pgfscope}%
\end{pgfscope}%
\begin{pgfscope}%
\definecolor{textcolor}{rgb}{0.000000,0.000000,0.000000}%
\pgfsetstrokecolor{textcolor}%
\pgfsetfillcolor{textcolor}%
\pgftext[x=4.522999in,y=0.290000in,,top]{\color{textcolor}\sffamily\fontsize{10.000000}{12.000000}\selectfont 120}%
\end{pgfscope}%
\begin{pgfscope}%
\pgfsetbuttcap%
\pgfsetroundjoin%
\definecolor{currentfill}{rgb}{0.000000,0.000000,0.000000}%
\pgfsetfillcolor{currentfill}%
\pgfsetlinewidth{0.803000pt}%
\definecolor{currentstroke}{rgb}{0.000000,0.000000,0.000000}%
\pgfsetstrokecolor{currentstroke}%
\pgfsetdash{}{0pt}%
\pgfsys@defobject{currentmarker}{\pgfqpoint{-0.048611in}{0.000000in}}{\pgfqpoint{0.000000in}{0.000000in}}{%
\pgfpathmoveto{\pgfqpoint{0.000000in}{0.000000in}}%
\pgfpathlineto{\pgfqpoint{-0.048611in}{0.000000in}}%
\pgfusepath{stroke,fill}%
}%
\begin{pgfscope}%
\pgfsys@transformshift{0.670972in}{0.673707in}%
\pgfsys@useobject{currentmarker}{}%
\end{pgfscope}%
\end{pgfscope}%
\begin{pgfscope}%
\definecolor{textcolor}{rgb}{0.000000,0.000000,0.000000}%
\pgfsetstrokecolor{textcolor}%
\pgfsetfillcolor{textcolor}%
\pgftext[x=0.149306in,y=0.625512in,left,base]{\color{textcolor}\sffamily\fontsize{10.000000}{12.000000}\selectfont −0.050}%
\end{pgfscope}%
\begin{pgfscope}%
\pgfsetbuttcap%
\pgfsetroundjoin%
\definecolor{currentfill}{rgb}{0.000000,0.000000,0.000000}%
\pgfsetfillcolor{currentfill}%
\pgfsetlinewidth{0.803000pt}%
\definecolor{currentstroke}{rgb}{0.000000,0.000000,0.000000}%
\pgfsetstrokecolor{currentstroke}%
\pgfsetdash{}{0pt}%
\pgfsys@defobject{currentmarker}{\pgfqpoint{-0.048611in}{0.000000in}}{\pgfqpoint{0.000000in}{0.000000in}}{%
\pgfpathmoveto{\pgfqpoint{0.000000in}{0.000000in}}%
\pgfpathlineto{\pgfqpoint{-0.048611in}{0.000000in}}%
\pgfusepath{stroke,fill}%
}%
\begin{pgfscope}%
\pgfsys@transformshift{0.670972in}{1.039025in}%
\pgfsys@useobject{currentmarker}{}%
\end{pgfscope}%
\end{pgfscope}%
\begin{pgfscope}%
\definecolor{textcolor}{rgb}{0.000000,0.000000,0.000000}%
\pgfsetstrokecolor{textcolor}%
\pgfsetfillcolor{textcolor}%
\pgftext[x=0.149306in,y=0.990831in,left,base]{\color{textcolor}\sffamily\fontsize{10.000000}{12.000000}\selectfont −0.025}%
\end{pgfscope}%
\begin{pgfscope}%
\pgfsetbuttcap%
\pgfsetroundjoin%
\definecolor{currentfill}{rgb}{0.000000,0.000000,0.000000}%
\pgfsetfillcolor{currentfill}%
\pgfsetlinewidth{0.803000pt}%
\definecolor{currentstroke}{rgb}{0.000000,0.000000,0.000000}%
\pgfsetstrokecolor{currentstroke}%
\pgfsetdash{}{0pt}%
\pgfsys@defobject{currentmarker}{\pgfqpoint{-0.048611in}{0.000000in}}{\pgfqpoint{0.000000in}{0.000000in}}{%
\pgfpathmoveto{\pgfqpoint{0.000000in}{0.000000in}}%
\pgfpathlineto{\pgfqpoint{-0.048611in}{0.000000in}}%
\pgfusepath{stroke,fill}%
}%
\begin{pgfscope}%
\pgfsys@transformshift{0.670972in}{1.404343in}%
\pgfsys@useobject{currentmarker}{}%
\end{pgfscope}%
\end{pgfscope}%
\begin{pgfscope}%
\definecolor{textcolor}{rgb}{0.000000,0.000000,0.000000}%
\pgfsetstrokecolor{textcolor}%
\pgfsetfillcolor{textcolor}%
\pgftext[x=0.257361in,y=1.356149in,left,base]{\color{textcolor}\sffamily\fontsize{10.000000}{12.000000}\selectfont 0.000}%
\end{pgfscope}%
\begin{pgfscope}%
\pgfpathrectangle{\pgfqpoint{0.670972in}{0.387222in}}{\pgfqpoint{4.259028in}{1.065556in}}%
\pgfusepath{clip}%
\pgfsetrectcap%
\pgfsetroundjoin%
\pgfsetlinewidth{1.505625pt}%
\definecolor{currentstroke}{rgb}{0.121569,0.466667,0.705882}%
\pgfsetstrokecolor{currentstroke}%
\pgfsetdash{}{0pt}%
\pgfpathmoveto{\pgfqpoint{0.864564in}{0.489258in}}%
\pgfpathlineto{\pgfqpoint{0.895051in}{1.113495in}}%
\pgfpathlineto{\pgfqpoint{0.925538in}{1.188321in}}%
\pgfpathlineto{\pgfqpoint{0.956025in}{1.224747in}}%
\pgfpathlineto{\pgfqpoint{0.986512in}{1.247340in}}%
\pgfpathlineto{\pgfqpoint{1.016999in}{1.263103in}}%
\pgfpathlineto{\pgfqpoint{1.047486in}{1.274905in}}%
\pgfpathlineto{\pgfqpoint{1.077973in}{1.284169in}}%
\pgfpathlineto{\pgfqpoint{1.108460in}{1.291691in}}%
\pgfpathlineto{\pgfqpoint{1.169434in}{1.303282in}}%
\pgfpathlineto{\pgfqpoint{1.230408in}{1.311901in}}%
\pgfpathlineto{\pgfqpoint{1.321869in}{1.321493in}}%
\pgfpathlineto{\pgfqpoint{1.413330in}{1.328609in}}%
\pgfpathlineto{\pgfqpoint{1.535277in}{1.335754in}}%
\pgfpathlineto{\pgfqpoint{1.687712in}{1.342365in}}%
\pgfpathlineto{\pgfqpoint{1.901121in}{1.349061in}}%
\pgfpathlineto{\pgfqpoint{2.175504in}{1.355148in}}%
\pgfpathlineto{\pgfqpoint{2.541347in}{1.360817in}}%
\pgfpathlineto{\pgfqpoint{3.029138in}{1.366015in}}%
\pgfpathlineto{\pgfqpoint{3.699851in}{1.370840in}}%
\pgfpathlineto{\pgfqpoint{4.675434in}{1.375435in}}%
\pgfpathlineto{\pgfqpoint{4.736408in}{1.375663in}}%
\pgfpathlineto{\pgfqpoint{4.736408in}{1.375663in}}%
\pgfusepath{stroke}%
\end{pgfscope}%
\begin{pgfscope}%
\pgfpathrectangle{\pgfqpoint{0.670972in}{0.387222in}}{\pgfqpoint{4.259028in}{1.065556in}}%
\pgfusepath{clip}%
\pgfsetrectcap%
\pgfsetroundjoin%
\pgfsetlinewidth{1.505625pt}%
\definecolor{currentstroke}{rgb}{1.000000,0.498039,0.054902}%
\pgfsetstrokecolor{currentstroke}%
\pgfsetdash{}{0pt}%
\pgfpathmoveto{\pgfqpoint{0.864564in}{1.404343in}}%
\pgfpathlineto{\pgfqpoint{4.736408in}{1.404343in}}%
\pgfpathlineto{\pgfqpoint{4.736408in}{1.404343in}}%
\pgfusepath{stroke}%
\end{pgfscope}%
\begin{pgfscope}%
\pgfpathrectangle{\pgfqpoint{0.670972in}{0.387222in}}{\pgfqpoint{4.259028in}{1.065556in}}%
\pgfusepath{clip}%
\pgfsetrectcap%
\pgfsetroundjoin%
\pgfsetlinewidth{1.505625pt}%
\definecolor{currentstroke}{rgb}{0.172549,0.627451,0.172549}%
\pgfsetstrokecolor{currentstroke}%
\pgfsetdash{}{0pt}%
\pgfpathmoveto{\pgfqpoint{0.864564in}{1.323302in}}%
\pgfpathlineto{\pgfqpoint{4.736408in}{1.323302in}}%
\pgfpathlineto{\pgfqpoint{4.736408in}{1.323302in}}%
\pgfusepath{stroke}%
\end{pgfscope}%
\begin{pgfscope}%
\pgfpathrectangle{\pgfqpoint{0.670972in}{0.387222in}}{\pgfqpoint{4.259028in}{1.065556in}}%
\pgfusepath{clip}%
\pgfsetrectcap%
\pgfsetroundjoin%
\pgfsetlinewidth{1.505625pt}%
\definecolor{currentstroke}{rgb}{0.839216,0.152941,0.156863}%
\pgfsetstrokecolor{currentstroke}%
\pgfsetdash{}{0pt}%
\pgfpathmoveto{\pgfqpoint{0.864564in}{1.403708in}}%
\pgfpathlineto{\pgfqpoint{4.736408in}{1.080814in}}%
\pgfpathlineto{\pgfqpoint{4.736408in}{1.080814in}}%
\pgfusepath{stroke}%
\end{pgfscope}%
\begin{pgfscope}%
\pgfpathrectangle{\pgfqpoint{0.670972in}{0.387222in}}{\pgfqpoint{4.259028in}{1.065556in}}%
\pgfusepath{clip}%
\pgfsetrectcap%
\pgfsetroundjoin%
\pgfsetlinewidth{1.505625pt}%
\definecolor{currentstroke}{rgb}{0.580392,0.403922,0.741176}%
\pgfsetstrokecolor{currentstroke}%
\pgfsetdash{}{0pt}%
\pgfpathmoveto{\pgfqpoint{0.864564in}{1.404338in}}%
\pgfpathlineto{\pgfqpoint{0.986512in}{1.403262in}}%
\pgfpathlineto{\pgfqpoint{1.108460in}{1.400270in}}%
\pgfpathlineto{\pgfqpoint{1.230408in}{1.395365in}}%
\pgfpathlineto{\pgfqpoint{1.352356in}{1.388545in}}%
\pgfpathlineto{\pgfqpoint{1.474304in}{1.379811in}}%
\pgfpathlineto{\pgfqpoint{1.596251in}{1.369163in}}%
\pgfpathlineto{\pgfqpoint{1.718199in}{1.356600in}}%
\pgfpathlineto{\pgfqpoint{1.840147in}{1.342123in}}%
\pgfpathlineto{\pgfqpoint{1.962095in}{1.325731in}}%
\pgfpathlineto{\pgfqpoint{2.084043in}{1.307425in}}%
\pgfpathlineto{\pgfqpoint{2.205990in}{1.287205in}}%
\pgfpathlineto{\pgfqpoint{2.327938in}{1.265071in}}%
\pgfpathlineto{\pgfqpoint{2.449886in}{1.241022in}}%
\pgfpathlineto{\pgfqpoint{2.571834in}{1.215059in}}%
\pgfpathlineto{\pgfqpoint{2.693782in}{1.187181in}}%
\pgfpathlineto{\pgfqpoint{2.815730in}{1.157390in}}%
\pgfpathlineto{\pgfqpoint{2.937677in}{1.125684in}}%
\pgfpathlineto{\pgfqpoint{3.059625in}{1.092063in}}%
\pgfpathlineto{\pgfqpoint{3.181573in}{1.056528in}}%
\pgfpathlineto{\pgfqpoint{3.303521in}{1.019079in}}%
\pgfpathlineto{\pgfqpoint{3.425469in}{0.979716in}}%
\pgfpathlineto{\pgfqpoint{3.547417in}{0.938438in}}%
\pgfpathlineto{\pgfqpoint{3.669364in}{0.895246in}}%
\pgfpathlineto{\pgfqpoint{3.791312in}{0.850139in}}%
\pgfpathlineto{\pgfqpoint{3.913260in}{0.803119in}}%
\pgfpathlineto{\pgfqpoint{4.035208in}{0.754183in}}%
\pgfpathlineto{\pgfqpoint{4.187643in}{0.690323in}}%
\pgfpathlineto{\pgfqpoint{4.340077in}{0.623470in}}%
\pgfpathlineto{\pgfqpoint{4.492512in}{0.553627in}}%
\pgfpathlineto{\pgfqpoint{4.644947in}{0.480793in}}%
\pgfpathlineto{\pgfqpoint{4.736408in}{0.435657in}}%
\pgfpathlineto{\pgfqpoint{4.736408in}{0.435657in}}%
\pgfusepath{stroke}%
\end{pgfscope}%
\begin{pgfscope}%
\pgfsetrectcap%
\pgfsetmiterjoin%
\pgfsetlinewidth{0.803000pt}%
\definecolor{currentstroke}{rgb}{0.000000,0.000000,0.000000}%
\pgfsetstrokecolor{currentstroke}%
\pgfsetdash{}{0pt}%
\pgfpathmoveto{\pgfqpoint{0.670972in}{0.387222in}}%
\pgfpathlineto{\pgfqpoint{0.670972in}{1.452778in}}%
\pgfusepath{stroke}%
\end{pgfscope}%
\begin{pgfscope}%
\pgfsetrectcap%
\pgfsetmiterjoin%
\pgfsetlinewidth{0.803000pt}%
\definecolor{currentstroke}{rgb}{0.000000,0.000000,0.000000}%
\pgfsetstrokecolor{currentstroke}%
\pgfsetdash{}{0pt}%
\pgfpathmoveto{\pgfqpoint{4.930000in}{0.387222in}}%
\pgfpathlineto{\pgfqpoint{4.930000in}{1.452778in}}%
\pgfusepath{stroke}%
\end{pgfscope}%
\begin{pgfscope}%
\pgfsetrectcap%
\pgfsetmiterjoin%
\pgfsetlinewidth{0.803000pt}%
\definecolor{currentstroke}{rgb}{0.000000,0.000000,0.000000}%
\pgfsetstrokecolor{currentstroke}%
\pgfsetdash{}{0pt}%
\pgfpathmoveto{\pgfqpoint{0.670972in}{0.387222in}}%
\pgfpathlineto{\pgfqpoint{4.930000in}{0.387222in}}%
\pgfusepath{stroke}%
\end{pgfscope}%
\begin{pgfscope}%
\pgfsetrectcap%
\pgfsetmiterjoin%
\pgfsetlinewidth{0.803000pt}%
\definecolor{currentstroke}{rgb}{0.000000,0.000000,0.000000}%
\pgfsetstrokecolor{currentstroke}%
\pgfsetdash{}{0pt}%
\pgfpathmoveto{\pgfqpoint{0.670972in}{1.452778in}}%
\pgfpathlineto{\pgfqpoint{4.930000in}{1.452778in}}%
\pgfusepath{stroke}%
\end{pgfscope}%
\begin{pgfscope}%
\definecolor{textcolor}{rgb}{0.000000,0.000000,0.000000}%
\pgfsetstrokecolor{textcolor}%
\pgfsetfillcolor{textcolor}%
\pgftext[x=2.800486in,y=1.536111in,,base]{\color{textcolor}\sffamily\fontsize{12.000000}{14.400000}\selectfont Detail Koeffizienten}%
\end{pgfscope}%
\begin{pgfscope}%
\pgfsetbuttcap%
\pgfsetmiterjoin%
\definecolor{currentfill}{rgb}{1.000000,1.000000,1.000000}%
\pgfsetfillcolor{currentfill}%
\pgfsetfillopacity{0.800000}%
\pgfsetlinewidth{1.003750pt}%
\definecolor{currentstroke}{rgb}{0.800000,0.800000,0.800000}%
\pgfsetstrokecolor{currentstroke}%
\pgfsetstrokeopacity{0.800000}%
\pgfsetdash{}{0pt}%
\pgfpathmoveto{\pgfqpoint{5.083854in}{1.145139in}}%
\pgfpathlineto{\pgfqpoint{5.758333in}{1.145139in}}%
\pgfpathquadraticcurveto{\pgfqpoint{5.786111in}{1.145139in}}{\pgfqpoint{5.786111in}{1.172917in}}%
\pgfpathlineto{\pgfqpoint{5.786111in}{2.127083in}}%
\pgfpathquadraticcurveto{\pgfqpoint{5.786111in}{2.154861in}}{\pgfqpoint{5.758333in}{2.154861in}}%
\pgfpathlineto{\pgfqpoint{5.083854in}{2.154861in}}%
\pgfpathquadraticcurveto{\pgfqpoint{5.056076in}{2.154861in}}{\pgfqpoint{5.056076in}{2.127083in}}%
\pgfpathlineto{\pgfqpoint{5.056076in}{1.172917in}}%
\pgfpathquadraticcurveto{\pgfqpoint{5.056076in}{1.145139in}}{\pgfqpoint{5.083854in}{1.145139in}}%
\pgfpathclose%
\pgfusepath{stroke,fill}%
\end{pgfscope}%
\begin{pgfscope}%
\pgfsetrectcap%
\pgfsetroundjoin%
\pgfsetlinewidth{1.505625pt}%
\definecolor{currentstroke}{rgb}{0.121569,0.466667,0.705882}%
\pgfsetstrokecolor{currentstroke}%
\pgfsetdash{}{0pt}%
\pgfpathmoveto{\pgfqpoint{5.111632in}{2.050694in}}%
\pgfpathlineto{\pgfqpoint{5.389409in}{2.050694in}}%
\pgfusepath{stroke}%
\end{pgfscope}%
\begin{pgfscope}%
\definecolor{textcolor}{rgb}{0.000000,0.000000,0.000000}%
\pgfsetstrokecolor{textcolor}%
\pgfsetfillcolor{textcolor}%
\pgftext[x=5.500520in,y=2.002083in,left,base]{\color{textcolor}\sffamily\fontsize{10.000000}{12.000000}\selectfont \(\displaystyle x^{0.5}\)}%
\end{pgfscope}%
\begin{pgfscope}%
\pgfsetrectcap%
\pgfsetroundjoin%
\pgfsetlinewidth{1.505625pt}%
\definecolor{currentstroke}{rgb}{1.000000,0.498039,0.054902}%
\pgfsetstrokecolor{currentstroke}%
\pgfsetdash{}{0pt}%
\pgfpathmoveto{\pgfqpoint{5.111632in}{1.857083in}}%
\pgfpathlineto{\pgfqpoint{5.389409in}{1.857083in}}%
\pgfusepath{stroke}%
\end{pgfscope}%
\begin{pgfscope}%
\definecolor{textcolor}{rgb}{0.000000,0.000000,0.000000}%
\pgfsetstrokecolor{textcolor}%
\pgfsetfillcolor{textcolor}%
\pgftext[x=5.500520in,y=1.808472in,left,base]{\color{textcolor}\sffamily\fontsize{10.000000}{12.000000}\selectfont \(\displaystyle x^{0}\)}%
\end{pgfscope}%
\begin{pgfscope}%
\pgfsetrectcap%
\pgfsetroundjoin%
\pgfsetlinewidth{1.505625pt}%
\definecolor{currentstroke}{rgb}{0.172549,0.627451,0.172549}%
\pgfsetstrokecolor{currentstroke}%
\pgfsetdash{}{0pt}%
\pgfpathmoveto{\pgfqpoint{5.111632in}{1.663472in}}%
\pgfpathlineto{\pgfqpoint{5.389409in}{1.663472in}}%
\pgfusepath{stroke}%
\end{pgfscope}%
\begin{pgfscope}%
\definecolor{textcolor}{rgb}{0.000000,0.000000,0.000000}%
\pgfsetstrokecolor{textcolor}%
\pgfsetfillcolor{textcolor}%
\pgftext[x=5.500520in,y=1.614861in,left,base]{\color{textcolor}\sffamily\fontsize{10.000000}{12.000000}\selectfont \(\displaystyle x^{1}\)}%
\end{pgfscope}%
\begin{pgfscope}%
\pgfsetrectcap%
\pgfsetroundjoin%
\pgfsetlinewidth{1.505625pt}%
\definecolor{currentstroke}{rgb}{0.839216,0.152941,0.156863}%
\pgfsetstrokecolor{currentstroke}%
\pgfsetdash{}{0pt}%
\pgfpathmoveto{\pgfqpoint{5.111632in}{1.469861in}}%
\pgfpathlineto{\pgfqpoint{5.389409in}{1.469861in}}%
\pgfusepath{stroke}%
\end{pgfscope}%
\begin{pgfscope}%
\definecolor{textcolor}{rgb}{0.000000,0.000000,0.000000}%
\pgfsetstrokecolor{textcolor}%
\pgfsetfillcolor{textcolor}%
\pgftext[x=5.500520in,y=1.421250in,left,base]{\color{textcolor}\sffamily\fontsize{10.000000}{12.000000}\selectfont \(\displaystyle x^{2}\)}%
\end{pgfscope}%
\begin{pgfscope}%
\pgfsetrectcap%
\pgfsetroundjoin%
\pgfsetlinewidth{1.505625pt}%
\definecolor{currentstroke}{rgb}{0.580392,0.403922,0.741176}%
\pgfsetstrokecolor{currentstroke}%
\pgfsetdash{}{0pt}%
\pgfpathmoveto{\pgfqpoint{5.111632in}{1.276250in}}%
\pgfpathlineto{\pgfqpoint{5.389409in}{1.276250in}}%
\pgfusepath{stroke}%
\end{pgfscope}%
\begin{pgfscope}%
\definecolor{textcolor}{rgb}{0.000000,0.000000,0.000000}%
\pgfsetstrokecolor{textcolor}%
\pgfsetfillcolor{textcolor}%
\pgftext[x=5.500520in,y=1.227639in,left,base]{\color{textcolor}\sffamily\fontsize{10.000000}{12.000000}\selectfont \(\displaystyle x^{3}\)}%
\end{pgfscope}%
\end{pgfpicture}%
\makeatother%
\endgroup%

    \caption{Analyse mit db1 (Haar) Wavelet\label{polynomials:haar}}
\end{figure}

In \autoref{polynomials:diff} sind zum Vergleich die Ableitungen der
verschiedenen Signale geben.

\begin{figure}
    \centering
    %% Creator: Matplotlib, PGF backend
%%
%% To include the figure in your LaTeX document, write
%%   \input{<filename>.pgf}
%%
%% Make sure the required packages are loaded in your preamble
%%   \usepackage{pgf}
%%
%% Figures using additional raster images can only be included by \input if
%% they are in the same directory as the main LaTeX file. For loading figures
%% from other directories you can use the `import` package
%%   \usepackage{import}
%% and then include the figures with
%%   \import{<path to file>}{<filename>.pgf}
%%
%% Matplotlib used the following preamble
%%   \usepackage{fontspec}
%%
\begingroup%
\makeatletter%
\begin{pgfpicture}%
\pgfpathrectangle{\pgfpointorigin}{\pgfqpoint{2.900000in}{3.000000in}}%
\pgfusepath{use as bounding box, clip}%
\begin{pgfscope}%
\pgfsetbuttcap%
\pgfsetmiterjoin%
\definecolor{currentfill}{rgb}{1.000000,1.000000,1.000000}%
\pgfsetfillcolor{currentfill}%
\pgfsetlinewidth{0.000000pt}%
\definecolor{currentstroke}{rgb}{1.000000,1.000000,1.000000}%
\pgfsetstrokecolor{currentstroke}%
\pgfsetdash{}{0pt}%
\pgfpathmoveto{\pgfqpoint{0.000000in}{0.000000in}}%
\pgfpathlineto{\pgfqpoint{2.900000in}{0.000000in}}%
\pgfpathlineto{\pgfqpoint{2.900000in}{3.000000in}}%
\pgfpathlineto{\pgfqpoint{0.000000in}{3.000000in}}%
\pgfpathclose%
\pgfusepath{fill}%
\end{pgfscope}%
\begin{pgfscope}%
\pgfsetbuttcap%
\pgfsetmiterjoin%
\definecolor{currentfill}{rgb}{1.000000,1.000000,1.000000}%
\pgfsetfillcolor{currentfill}%
\pgfsetlinewidth{0.000000pt}%
\definecolor{currentstroke}{rgb}{0.000000,0.000000,0.000000}%
\pgfsetstrokecolor{currentstroke}%
\pgfsetstrokeopacity{0.000000}%
\pgfsetdash{}{0pt}%
\pgfpathmoveto{\pgfqpoint{0.362500in}{0.375000in}}%
\pgfpathlineto{\pgfqpoint{2.610000in}{0.375000in}}%
\pgfpathlineto{\pgfqpoint{2.610000in}{2.640000in}}%
\pgfpathlineto{\pgfqpoint{0.362500in}{2.640000in}}%
\pgfpathclose%
\pgfusepath{fill}%
\end{pgfscope}%
\begin{pgfscope}%
\pgfpathrectangle{\pgfqpoint{0.362500in}{0.375000in}}{\pgfqpoint{2.247500in}{2.265000in}}%
\pgfusepath{clip}%
\pgfsetrectcap%
\pgfsetroundjoin%
\pgfsetlinewidth{0.803000pt}%
\definecolor{currentstroke}{rgb}{0.690196,0.690196,0.690196}%
\pgfsetstrokecolor{currentstroke}%
\pgfsetdash{}{0pt}%
\pgfpathmoveto{\pgfqpoint{0.464659in}{0.375000in}}%
\pgfpathlineto{\pgfqpoint{0.464659in}{2.640000in}}%
\pgfusepath{stroke}%
\end{pgfscope}%
\begin{pgfscope}%
\pgfsetbuttcap%
\pgfsetroundjoin%
\definecolor{currentfill}{rgb}{0.000000,0.000000,0.000000}%
\pgfsetfillcolor{currentfill}%
\pgfsetlinewidth{0.803000pt}%
\definecolor{currentstroke}{rgb}{0.000000,0.000000,0.000000}%
\pgfsetstrokecolor{currentstroke}%
\pgfsetdash{}{0pt}%
\pgfsys@defobject{currentmarker}{\pgfqpoint{0.000000in}{-0.048611in}}{\pgfqpoint{0.000000in}{0.000000in}}{%
\pgfpathmoveto{\pgfqpoint{0.000000in}{0.000000in}}%
\pgfpathlineto{\pgfqpoint{0.000000in}{-0.048611in}}%
\pgfusepath{stroke,fill}%
}%
\begin{pgfscope}%
\pgfsys@transformshift{0.464659in}{0.375000in}%
\pgfsys@useobject{currentmarker}{}%
\end{pgfscope}%
\end{pgfscope}%
\begin{pgfscope}%
\definecolor{textcolor}{rgb}{0.000000,0.000000,0.000000}%
\pgfsetstrokecolor{textcolor}%
\pgfsetfillcolor{textcolor}%
\pgftext[x=0.464659in,y=0.277778in,,top]{\color{textcolor}\rmfamily\fontsize{10.000000}{12.000000}\selectfont 0}%
\end{pgfscope}%
\begin{pgfscope}%
\pgfpathrectangle{\pgfqpoint{0.362500in}{0.375000in}}{\pgfqpoint{2.247500in}{2.265000in}}%
\pgfusepath{clip}%
\pgfsetrectcap%
\pgfsetroundjoin%
\pgfsetlinewidth{0.803000pt}%
\definecolor{currentstroke}{rgb}{0.690196,0.690196,0.690196}%
\pgfsetstrokecolor{currentstroke}%
\pgfsetdash{}{0pt}%
\pgfpathmoveto{\pgfqpoint{1.269061in}{0.375000in}}%
\pgfpathlineto{\pgfqpoint{1.269061in}{2.640000in}}%
\pgfusepath{stroke}%
\end{pgfscope}%
\begin{pgfscope}%
\pgfsetbuttcap%
\pgfsetroundjoin%
\definecolor{currentfill}{rgb}{0.000000,0.000000,0.000000}%
\pgfsetfillcolor{currentfill}%
\pgfsetlinewidth{0.803000pt}%
\definecolor{currentstroke}{rgb}{0.000000,0.000000,0.000000}%
\pgfsetstrokecolor{currentstroke}%
\pgfsetdash{}{0pt}%
\pgfsys@defobject{currentmarker}{\pgfqpoint{0.000000in}{-0.048611in}}{\pgfqpoint{0.000000in}{0.000000in}}{%
\pgfpathmoveto{\pgfqpoint{0.000000in}{0.000000in}}%
\pgfpathlineto{\pgfqpoint{0.000000in}{-0.048611in}}%
\pgfusepath{stroke,fill}%
}%
\begin{pgfscope}%
\pgfsys@transformshift{1.269061in}{0.375000in}%
\pgfsys@useobject{currentmarker}{}%
\end{pgfscope}%
\end{pgfscope}%
\begin{pgfscope}%
\definecolor{textcolor}{rgb}{0.000000,0.000000,0.000000}%
\pgfsetstrokecolor{textcolor}%
\pgfsetfillcolor{textcolor}%
\pgftext[x=1.269061in,y=0.277778in,,top]{\color{textcolor}\rmfamily\fontsize{10.000000}{12.000000}\selectfont 100}%
\end{pgfscope}%
\begin{pgfscope}%
\pgfpathrectangle{\pgfqpoint{0.362500in}{0.375000in}}{\pgfqpoint{2.247500in}{2.265000in}}%
\pgfusepath{clip}%
\pgfsetrectcap%
\pgfsetroundjoin%
\pgfsetlinewidth{0.803000pt}%
\definecolor{currentstroke}{rgb}{0.690196,0.690196,0.690196}%
\pgfsetstrokecolor{currentstroke}%
\pgfsetdash{}{0pt}%
\pgfpathmoveto{\pgfqpoint{2.073464in}{0.375000in}}%
\pgfpathlineto{\pgfqpoint{2.073464in}{2.640000in}}%
\pgfusepath{stroke}%
\end{pgfscope}%
\begin{pgfscope}%
\pgfsetbuttcap%
\pgfsetroundjoin%
\definecolor{currentfill}{rgb}{0.000000,0.000000,0.000000}%
\pgfsetfillcolor{currentfill}%
\pgfsetlinewidth{0.803000pt}%
\definecolor{currentstroke}{rgb}{0.000000,0.000000,0.000000}%
\pgfsetstrokecolor{currentstroke}%
\pgfsetdash{}{0pt}%
\pgfsys@defobject{currentmarker}{\pgfqpoint{0.000000in}{-0.048611in}}{\pgfqpoint{0.000000in}{0.000000in}}{%
\pgfpathmoveto{\pgfqpoint{0.000000in}{0.000000in}}%
\pgfpathlineto{\pgfqpoint{0.000000in}{-0.048611in}}%
\pgfusepath{stroke,fill}%
}%
\begin{pgfscope}%
\pgfsys@transformshift{2.073464in}{0.375000in}%
\pgfsys@useobject{currentmarker}{}%
\end{pgfscope}%
\end{pgfscope}%
\begin{pgfscope}%
\definecolor{textcolor}{rgb}{0.000000,0.000000,0.000000}%
\pgfsetstrokecolor{textcolor}%
\pgfsetfillcolor{textcolor}%
\pgftext[x=2.073464in,y=0.277778in,,top]{\color{textcolor}\rmfamily\fontsize{10.000000}{12.000000}\selectfont 200}%
\end{pgfscope}%
\begin{pgfscope}%
\pgfpathrectangle{\pgfqpoint{0.362500in}{0.375000in}}{\pgfqpoint{2.247500in}{2.265000in}}%
\pgfusepath{clip}%
\pgfsetrectcap%
\pgfsetroundjoin%
\pgfsetlinewidth{0.803000pt}%
\definecolor{currentstroke}{rgb}{0.690196,0.690196,0.690196}%
\pgfsetstrokecolor{currentstroke}%
\pgfsetdash{}{0pt}%
\pgfpathmoveto{\pgfqpoint{0.362500in}{0.477955in}}%
\pgfpathlineto{\pgfqpoint{2.610000in}{0.477955in}}%
\pgfusepath{stroke}%
\end{pgfscope}%
\begin{pgfscope}%
\pgfsetbuttcap%
\pgfsetroundjoin%
\definecolor{currentfill}{rgb}{0.000000,0.000000,0.000000}%
\pgfsetfillcolor{currentfill}%
\pgfsetlinewidth{0.803000pt}%
\definecolor{currentstroke}{rgb}{0.000000,0.000000,0.000000}%
\pgfsetstrokecolor{currentstroke}%
\pgfsetdash{}{0pt}%
\pgfsys@defobject{currentmarker}{\pgfqpoint{-0.048611in}{0.000000in}}{\pgfqpoint{0.000000in}{0.000000in}}{%
\pgfpathmoveto{\pgfqpoint{0.000000in}{0.000000in}}%
\pgfpathlineto{\pgfqpoint{-0.048611in}{0.000000in}}%
\pgfusepath{stroke,fill}%
}%
\begin{pgfscope}%
\pgfsys@transformshift{0.362500in}{0.477955in}%
\pgfsys@useobject{currentmarker}{}%
\end{pgfscope}%
\end{pgfscope}%
\begin{pgfscope}%
\definecolor{textcolor}{rgb}{0.000000,0.000000,0.000000}%
\pgfsetstrokecolor{textcolor}%
\pgfsetfillcolor{textcolor}%
\pgftext[x=0.018333in,y=0.429760in,left,base]{\color{textcolor}\rmfamily\fontsize{10.000000}{12.000000}\selectfont 0.00}%
\end{pgfscope}%
\begin{pgfscope}%
\pgfpathrectangle{\pgfqpoint{0.362500in}{0.375000in}}{\pgfqpoint{2.247500in}{2.265000in}}%
\pgfusepath{clip}%
\pgfsetrectcap%
\pgfsetroundjoin%
\pgfsetlinewidth{0.803000pt}%
\definecolor{currentstroke}{rgb}{0.690196,0.690196,0.690196}%
\pgfsetstrokecolor{currentstroke}%
\pgfsetdash{}{0pt}%
\pgfpathmoveto{\pgfqpoint{0.362500in}{0.917232in}}%
\pgfpathlineto{\pgfqpoint{2.610000in}{0.917232in}}%
\pgfusepath{stroke}%
\end{pgfscope}%
\begin{pgfscope}%
\pgfsetbuttcap%
\pgfsetroundjoin%
\definecolor{currentfill}{rgb}{0.000000,0.000000,0.000000}%
\pgfsetfillcolor{currentfill}%
\pgfsetlinewidth{0.803000pt}%
\definecolor{currentstroke}{rgb}{0.000000,0.000000,0.000000}%
\pgfsetstrokecolor{currentstroke}%
\pgfsetdash{}{0pt}%
\pgfsys@defobject{currentmarker}{\pgfqpoint{-0.048611in}{0.000000in}}{\pgfqpoint{0.000000in}{0.000000in}}{%
\pgfpathmoveto{\pgfqpoint{0.000000in}{0.000000in}}%
\pgfpathlineto{\pgfqpoint{-0.048611in}{0.000000in}}%
\pgfusepath{stroke,fill}%
}%
\begin{pgfscope}%
\pgfsys@transformshift{0.362500in}{0.917232in}%
\pgfsys@useobject{currentmarker}{}%
\end{pgfscope}%
\end{pgfscope}%
\begin{pgfscope}%
\definecolor{textcolor}{rgb}{0.000000,0.000000,0.000000}%
\pgfsetstrokecolor{textcolor}%
\pgfsetfillcolor{textcolor}%
\pgftext[x=0.018333in,y=0.869037in,left,base]{\color{textcolor}\rmfamily\fontsize{10.000000}{12.000000}\selectfont 0.02}%
\end{pgfscope}%
\begin{pgfscope}%
\pgfpathrectangle{\pgfqpoint{0.362500in}{0.375000in}}{\pgfqpoint{2.247500in}{2.265000in}}%
\pgfusepath{clip}%
\pgfsetrectcap%
\pgfsetroundjoin%
\pgfsetlinewidth{0.803000pt}%
\definecolor{currentstroke}{rgb}{0.690196,0.690196,0.690196}%
\pgfsetstrokecolor{currentstroke}%
\pgfsetdash{}{0pt}%
\pgfpathmoveto{\pgfqpoint{0.362500in}{1.356509in}}%
\pgfpathlineto{\pgfqpoint{2.610000in}{1.356509in}}%
\pgfusepath{stroke}%
\end{pgfscope}%
\begin{pgfscope}%
\pgfsetbuttcap%
\pgfsetroundjoin%
\definecolor{currentfill}{rgb}{0.000000,0.000000,0.000000}%
\pgfsetfillcolor{currentfill}%
\pgfsetlinewidth{0.803000pt}%
\definecolor{currentstroke}{rgb}{0.000000,0.000000,0.000000}%
\pgfsetstrokecolor{currentstroke}%
\pgfsetdash{}{0pt}%
\pgfsys@defobject{currentmarker}{\pgfqpoint{-0.048611in}{0.000000in}}{\pgfqpoint{0.000000in}{0.000000in}}{%
\pgfpathmoveto{\pgfqpoint{0.000000in}{0.000000in}}%
\pgfpathlineto{\pgfqpoint{-0.048611in}{0.000000in}}%
\pgfusepath{stroke,fill}%
}%
\begin{pgfscope}%
\pgfsys@transformshift{0.362500in}{1.356509in}%
\pgfsys@useobject{currentmarker}{}%
\end{pgfscope}%
\end{pgfscope}%
\begin{pgfscope}%
\definecolor{textcolor}{rgb}{0.000000,0.000000,0.000000}%
\pgfsetstrokecolor{textcolor}%
\pgfsetfillcolor{textcolor}%
\pgftext[x=0.018333in,y=1.308315in,left,base]{\color{textcolor}\rmfamily\fontsize{10.000000}{12.000000}\selectfont 0.04}%
\end{pgfscope}%
\begin{pgfscope}%
\pgfpathrectangle{\pgfqpoint{0.362500in}{0.375000in}}{\pgfqpoint{2.247500in}{2.265000in}}%
\pgfusepath{clip}%
\pgfsetrectcap%
\pgfsetroundjoin%
\pgfsetlinewidth{0.803000pt}%
\definecolor{currentstroke}{rgb}{0.690196,0.690196,0.690196}%
\pgfsetstrokecolor{currentstroke}%
\pgfsetdash{}{0pt}%
\pgfpathmoveto{\pgfqpoint{0.362500in}{1.795786in}}%
\pgfpathlineto{\pgfqpoint{2.610000in}{1.795786in}}%
\pgfusepath{stroke}%
\end{pgfscope}%
\begin{pgfscope}%
\pgfsetbuttcap%
\pgfsetroundjoin%
\definecolor{currentfill}{rgb}{0.000000,0.000000,0.000000}%
\pgfsetfillcolor{currentfill}%
\pgfsetlinewidth{0.803000pt}%
\definecolor{currentstroke}{rgb}{0.000000,0.000000,0.000000}%
\pgfsetstrokecolor{currentstroke}%
\pgfsetdash{}{0pt}%
\pgfsys@defobject{currentmarker}{\pgfqpoint{-0.048611in}{0.000000in}}{\pgfqpoint{0.000000in}{0.000000in}}{%
\pgfpathmoveto{\pgfqpoint{0.000000in}{0.000000in}}%
\pgfpathlineto{\pgfqpoint{-0.048611in}{0.000000in}}%
\pgfusepath{stroke,fill}%
}%
\begin{pgfscope}%
\pgfsys@transformshift{0.362500in}{1.795786in}%
\pgfsys@useobject{currentmarker}{}%
\end{pgfscope}%
\end{pgfscope}%
\begin{pgfscope}%
\definecolor{textcolor}{rgb}{0.000000,0.000000,0.000000}%
\pgfsetstrokecolor{textcolor}%
\pgfsetfillcolor{textcolor}%
\pgftext[x=0.018333in,y=1.747592in,left,base]{\color{textcolor}\rmfamily\fontsize{10.000000}{12.000000}\selectfont 0.06}%
\end{pgfscope}%
\begin{pgfscope}%
\pgfpathrectangle{\pgfqpoint{0.362500in}{0.375000in}}{\pgfqpoint{2.247500in}{2.265000in}}%
\pgfusepath{clip}%
\pgfsetrectcap%
\pgfsetroundjoin%
\pgfsetlinewidth{0.803000pt}%
\definecolor{currentstroke}{rgb}{0.690196,0.690196,0.690196}%
\pgfsetstrokecolor{currentstroke}%
\pgfsetdash{}{0pt}%
\pgfpathmoveto{\pgfqpoint{0.362500in}{2.235063in}}%
\pgfpathlineto{\pgfqpoint{2.610000in}{2.235063in}}%
\pgfusepath{stroke}%
\end{pgfscope}%
\begin{pgfscope}%
\pgfsetbuttcap%
\pgfsetroundjoin%
\definecolor{currentfill}{rgb}{0.000000,0.000000,0.000000}%
\pgfsetfillcolor{currentfill}%
\pgfsetlinewidth{0.803000pt}%
\definecolor{currentstroke}{rgb}{0.000000,0.000000,0.000000}%
\pgfsetstrokecolor{currentstroke}%
\pgfsetdash{}{0pt}%
\pgfsys@defobject{currentmarker}{\pgfqpoint{-0.048611in}{0.000000in}}{\pgfqpoint{0.000000in}{0.000000in}}{%
\pgfpathmoveto{\pgfqpoint{0.000000in}{0.000000in}}%
\pgfpathlineto{\pgfqpoint{-0.048611in}{0.000000in}}%
\pgfusepath{stroke,fill}%
}%
\begin{pgfscope}%
\pgfsys@transformshift{0.362500in}{2.235063in}%
\pgfsys@useobject{currentmarker}{}%
\end{pgfscope}%
\end{pgfscope}%
\begin{pgfscope}%
\definecolor{textcolor}{rgb}{0.000000,0.000000,0.000000}%
\pgfsetstrokecolor{textcolor}%
\pgfsetfillcolor{textcolor}%
\pgftext[x=0.018333in,y=2.186869in,left,base]{\color{textcolor}\rmfamily\fontsize{10.000000}{12.000000}\selectfont 0.08}%
\end{pgfscope}%
\begin{pgfscope}%
\pgfpathrectangle{\pgfqpoint{0.362500in}{0.375000in}}{\pgfqpoint{2.247500in}{2.265000in}}%
\pgfusepath{clip}%
\pgfsetrectcap%
\pgfsetroundjoin%
\pgfsetlinewidth{1.505625pt}%
\definecolor{currentstroke}{rgb}{0.121569,0.466667,0.705882}%
\pgfsetstrokecolor{currentstroke}%
\pgfsetdash{}{0pt}%
\pgfpathmoveto{\pgfqpoint{0.464659in}{2.423107in}}%
\pgfpathlineto{\pgfqpoint{0.472703in}{1.283663in}}%
\pgfpathlineto{\pgfqpoint{0.480747in}{1.096196in}}%
\pgfpathlineto{\pgfqpoint{0.488791in}{0.999157in}}%
\pgfpathlineto{\pgfqpoint{0.496835in}{0.937143in}}%
\pgfpathlineto{\pgfqpoint{0.504879in}{0.893092in}}%
\pgfpathlineto{\pgfqpoint{0.512923in}{0.859713in}}%
\pgfpathlineto{\pgfqpoint{0.529011in}{0.811690in}}%
\pgfpathlineto{\pgfqpoint{0.545099in}{0.778183in}}%
\pgfpathlineto{\pgfqpoint{0.561187in}{0.753096in}}%
\pgfpathlineto{\pgfqpoint{0.577275in}{0.733404in}}%
\pgfpathlineto{\pgfqpoint{0.601407in}{0.710468in}}%
\pgfpathlineto{\pgfqpoint{0.625540in}{0.692777in}}%
\pgfpathlineto{\pgfqpoint{0.649672in}{0.678593in}}%
\pgfpathlineto{\pgfqpoint{0.681848in}{0.663425in}}%
\pgfpathlineto{\pgfqpoint{0.722068in}{0.648561in}}%
\pgfpathlineto{\pgfqpoint{0.770332in}{0.634703in}}%
\pgfpathlineto{\pgfqpoint{0.826640in}{0.622141in}}%
\pgfpathlineto{\pgfqpoint{0.890992in}{0.610924in}}%
\pgfpathlineto{\pgfqpoint{0.971433in}{0.600005in}}%
\pgfpathlineto{\pgfqpoint{1.076005in}{0.589152in}}%
\pgfpathlineto{\pgfqpoint{1.204709in}{0.579079in}}%
\pgfpathlineto{\pgfqpoint{1.365590in}{0.569650in}}%
\pgfpathlineto{\pgfqpoint{1.574734in}{0.560596in}}%
\pgfpathlineto{\pgfqpoint{1.848231in}{0.552005in}}%
\pgfpathlineto{\pgfqpoint{2.210212in}{0.543901in}}%
\pgfpathlineto{\pgfqpoint{2.507841in}{0.538919in}}%
\pgfpathlineto{\pgfqpoint{2.507841in}{0.538919in}}%
\pgfusepath{stroke}%
\end{pgfscope}%
\begin{pgfscope}%
\pgfpathrectangle{\pgfqpoint{0.362500in}{0.375000in}}{\pgfqpoint{2.247500in}{2.265000in}}%
\pgfusepath{clip}%
\pgfsetrectcap%
\pgfsetroundjoin%
\pgfsetlinewidth{1.505625pt}%
\definecolor{currentstroke}{rgb}{1.000000,0.498039,0.054902}%
\pgfsetstrokecolor{currentstroke}%
\pgfsetdash{}{0pt}%
\pgfpathmoveto{\pgfqpoint{0.464659in}{0.477955in}}%
\pgfpathlineto{\pgfqpoint{2.507841in}{0.477955in}}%
\pgfpathlineto{\pgfqpoint{2.507841in}{0.477955in}}%
\pgfusepath{stroke}%
\end{pgfscope}%
\begin{pgfscope}%
\pgfpathrectangle{\pgfqpoint{0.362500in}{0.375000in}}{\pgfqpoint{2.247500in}{2.265000in}}%
\pgfusepath{clip}%
\pgfsetrectcap%
\pgfsetroundjoin%
\pgfsetlinewidth{1.505625pt}%
\definecolor{currentstroke}{rgb}{0.172549,0.627451,0.172549}%
\pgfsetstrokecolor{currentstroke}%
\pgfsetdash{}{0pt}%
\pgfpathmoveto{\pgfqpoint{0.464659in}{0.650220in}}%
\pgfpathlineto{\pgfqpoint{2.507841in}{0.650220in}}%
\pgfpathlineto{\pgfqpoint{2.507841in}{0.650220in}}%
\pgfusepath{stroke}%
\end{pgfscope}%
\begin{pgfscope}%
\pgfpathrectangle{\pgfqpoint{0.362500in}{0.375000in}}{\pgfqpoint{2.247500in}{2.265000in}}%
\pgfusepath{clip}%
\pgfsetrectcap%
\pgfsetroundjoin%
\pgfsetlinewidth{1.505625pt}%
\definecolor{currentstroke}{rgb}{0.839216,0.152941,0.156863}%
\pgfsetstrokecolor{currentstroke}%
\pgfsetdash{}{0pt}%
\pgfpathmoveto{\pgfqpoint{0.464659in}{0.479306in}}%
\pgfpathlineto{\pgfqpoint{2.507841in}{1.165666in}}%
\pgfpathlineto{\pgfqpoint{2.507841in}{1.165666in}}%
\pgfusepath{stroke}%
\end{pgfscope}%
\begin{pgfscope}%
\pgfpathrectangle{\pgfqpoint{0.362500in}{0.375000in}}{\pgfqpoint{2.247500in}{2.265000in}}%
\pgfusepath{clip}%
\pgfsetrectcap%
\pgfsetroundjoin%
\pgfsetlinewidth{1.505625pt}%
\definecolor{currentstroke}{rgb}{0.580392,0.403922,0.741176}%
\pgfsetstrokecolor{currentstroke}%
\pgfsetdash{}{0pt}%
\pgfpathmoveto{\pgfqpoint{0.464659in}{0.477965in}}%
\pgfpathlineto{\pgfqpoint{0.520967in}{0.479745in}}%
\pgfpathlineto{\pgfqpoint{0.577275in}{0.484641in}}%
\pgfpathlineto{\pgfqpoint{0.633584in}{0.492652in}}%
\pgfpathlineto{\pgfqpoint{0.689892in}{0.503779in}}%
\pgfpathlineto{\pgfqpoint{0.746200in}{0.518021in}}%
\pgfpathlineto{\pgfqpoint{0.802508in}{0.535379in}}%
\pgfpathlineto{\pgfqpoint{0.858816in}{0.555852in}}%
\pgfpathlineto{\pgfqpoint{0.915124in}{0.579441in}}%
\pgfpathlineto{\pgfqpoint{0.971433in}{0.606145in}}%
\pgfpathlineto{\pgfqpoint{1.027741in}{0.635965in}}%
\pgfpathlineto{\pgfqpoint{1.084049in}{0.668900in}}%
\pgfpathlineto{\pgfqpoint{1.148401in}{0.710355in}}%
\pgfpathlineto{\pgfqpoint{1.212753in}{0.755879in}}%
\pgfpathlineto{\pgfqpoint{1.277105in}{0.805472in}}%
\pgfpathlineto{\pgfqpoint{1.341458in}{0.859135in}}%
\pgfpathlineto{\pgfqpoint{1.405810in}{0.916867in}}%
\pgfpathlineto{\pgfqpoint{1.470162in}{0.978668in}}%
\pgfpathlineto{\pgfqpoint{1.534514in}{1.044538in}}%
\pgfpathlineto{\pgfqpoint{1.598866in}{1.114478in}}%
\pgfpathlineto{\pgfqpoint{1.671263in}{1.198023in}}%
\pgfpathlineto{\pgfqpoint{1.743659in}{1.286719in}}%
\pgfpathlineto{\pgfqpoint{1.816055in}{1.380565in}}%
\pgfpathlineto{\pgfqpoint{1.888451in}{1.479561in}}%
\pgfpathlineto{\pgfqpoint{1.960847in}{1.583707in}}%
\pgfpathlineto{\pgfqpoint{2.033244in}{1.693004in}}%
\pgfpathlineto{\pgfqpoint{2.105640in}{1.807450in}}%
\pgfpathlineto{\pgfqpoint{2.186080in}{1.940653in}}%
\pgfpathlineto{\pgfqpoint{2.266520in}{2.080214in}}%
\pgfpathlineto{\pgfqpoint{2.346960in}{2.226133in}}%
\pgfpathlineto{\pgfqpoint{2.427401in}{2.378410in}}%
\pgfpathlineto{\pgfqpoint{2.507841in}{2.537045in}}%
\pgfpathlineto{\pgfqpoint{2.507841in}{2.537045in}}%
\pgfusepath{stroke}%
\end{pgfscope}%
\begin{pgfscope}%
\pgfsetrectcap%
\pgfsetmiterjoin%
\pgfsetlinewidth{0.803000pt}%
\definecolor{currentstroke}{rgb}{0.000000,0.000000,0.000000}%
\pgfsetstrokecolor{currentstroke}%
\pgfsetdash{}{0pt}%
\pgfpathmoveto{\pgfqpoint{0.362500in}{0.375000in}}%
\pgfpathlineto{\pgfqpoint{0.362500in}{2.640000in}}%
\pgfusepath{stroke}%
\end{pgfscope}%
\begin{pgfscope}%
\pgfsetrectcap%
\pgfsetmiterjoin%
\pgfsetlinewidth{0.803000pt}%
\definecolor{currentstroke}{rgb}{0.000000,0.000000,0.000000}%
\pgfsetstrokecolor{currentstroke}%
\pgfsetdash{}{0pt}%
\pgfpathmoveto{\pgfqpoint{2.610000in}{0.375000in}}%
\pgfpathlineto{\pgfqpoint{2.610000in}{2.640000in}}%
\pgfusepath{stroke}%
\end{pgfscope}%
\begin{pgfscope}%
\pgfsetrectcap%
\pgfsetmiterjoin%
\pgfsetlinewidth{0.803000pt}%
\definecolor{currentstroke}{rgb}{0.000000,0.000000,0.000000}%
\pgfsetstrokecolor{currentstroke}%
\pgfsetdash{}{0pt}%
\pgfpathmoveto{\pgfqpoint{0.362500in}{0.375000in}}%
\pgfpathlineto{\pgfqpoint{2.610000in}{0.375000in}}%
\pgfusepath{stroke}%
\end{pgfscope}%
\begin{pgfscope}%
\pgfsetrectcap%
\pgfsetmiterjoin%
\pgfsetlinewidth{0.803000pt}%
\definecolor{currentstroke}{rgb}{0.000000,0.000000,0.000000}%
\pgfsetstrokecolor{currentstroke}%
\pgfsetdash{}{0pt}%
\pgfpathmoveto{\pgfqpoint{0.362500in}{2.640000in}}%
\pgfpathlineto{\pgfqpoint{2.610000in}{2.640000in}}%
\pgfusepath{stroke}%
\end{pgfscope}%
\begin{pgfscope}%
\pgfsetbuttcap%
\pgfsetmiterjoin%
\definecolor{currentfill}{rgb}{1.000000,1.000000,1.000000}%
\pgfsetfillcolor{currentfill}%
\pgfsetfillopacity{0.800000}%
\pgfsetlinewidth{1.003750pt}%
\definecolor{currentstroke}{rgb}{0.800000,0.800000,0.800000}%
\pgfsetstrokecolor{currentstroke}%
\pgfsetstrokeopacity{0.800000}%
\pgfsetdash{}{0pt}%
\pgfpathmoveto{\pgfqpoint{1.149010in}{1.560834in}}%
\pgfpathlineto{\pgfqpoint{1.823490in}{1.560834in}}%
\pgfpathquadraticcurveto{\pgfqpoint{1.851268in}{1.560834in}}{\pgfqpoint{1.851268in}{1.588612in}}%
\pgfpathlineto{\pgfqpoint{1.851268in}{2.542778in}}%
\pgfpathquadraticcurveto{\pgfqpoint{1.851268in}{2.570556in}}{\pgfqpoint{1.823490in}{2.570556in}}%
\pgfpathlineto{\pgfqpoint{1.149010in}{2.570556in}}%
\pgfpathquadraticcurveto{\pgfqpoint{1.121232in}{2.570556in}}{\pgfqpoint{1.121232in}{2.542778in}}%
\pgfpathlineto{\pgfqpoint{1.121232in}{1.588612in}}%
\pgfpathquadraticcurveto{\pgfqpoint{1.121232in}{1.560834in}}{\pgfqpoint{1.149010in}{1.560834in}}%
\pgfpathclose%
\pgfusepath{stroke,fill}%
\end{pgfscope}%
\begin{pgfscope}%
\pgfsetrectcap%
\pgfsetroundjoin%
\pgfsetlinewidth{1.505625pt}%
\definecolor{currentstroke}{rgb}{0.121569,0.466667,0.705882}%
\pgfsetstrokecolor{currentstroke}%
\pgfsetdash{}{0pt}%
\pgfpathmoveto{\pgfqpoint{1.176788in}{2.466389in}}%
\pgfpathlineto{\pgfqpoint{1.454566in}{2.466389in}}%
\pgfusepath{stroke}%
\end{pgfscope}%
\begin{pgfscope}%
\definecolor{textcolor}{rgb}{0.000000,0.000000,0.000000}%
\pgfsetstrokecolor{textcolor}%
\pgfsetfillcolor{textcolor}%
\pgftext[x=1.565677in,y=2.417778in,left,base]{\color{textcolor}\rmfamily\fontsize{10.000000}{12.000000}\selectfont \(\displaystyle x^{0.5}\)}%
\end{pgfscope}%
\begin{pgfscope}%
\pgfsetrectcap%
\pgfsetroundjoin%
\pgfsetlinewidth{1.505625pt}%
\definecolor{currentstroke}{rgb}{1.000000,0.498039,0.054902}%
\pgfsetstrokecolor{currentstroke}%
\pgfsetdash{}{0pt}%
\pgfpathmoveto{\pgfqpoint{1.176788in}{2.272778in}}%
\pgfpathlineto{\pgfqpoint{1.454566in}{2.272778in}}%
\pgfusepath{stroke}%
\end{pgfscope}%
\begin{pgfscope}%
\definecolor{textcolor}{rgb}{0.000000,0.000000,0.000000}%
\pgfsetstrokecolor{textcolor}%
\pgfsetfillcolor{textcolor}%
\pgftext[x=1.565677in,y=2.224167in,left,base]{\color{textcolor}\rmfamily\fontsize{10.000000}{12.000000}\selectfont \(\displaystyle x^{0}\)}%
\end{pgfscope}%
\begin{pgfscope}%
\pgfsetrectcap%
\pgfsetroundjoin%
\pgfsetlinewidth{1.505625pt}%
\definecolor{currentstroke}{rgb}{0.172549,0.627451,0.172549}%
\pgfsetstrokecolor{currentstroke}%
\pgfsetdash{}{0pt}%
\pgfpathmoveto{\pgfqpoint{1.176788in}{2.079167in}}%
\pgfpathlineto{\pgfqpoint{1.454566in}{2.079167in}}%
\pgfusepath{stroke}%
\end{pgfscope}%
\begin{pgfscope}%
\definecolor{textcolor}{rgb}{0.000000,0.000000,0.000000}%
\pgfsetstrokecolor{textcolor}%
\pgfsetfillcolor{textcolor}%
\pgftext[x=1.565677in,y=2.030556in,left,base]{\color{textcolor}\rmfamily\fontsize{10.000000}{12.000000}\selectfont \(\displaystyle x^{1}\)}%
\end{pgfscope}%
\begin{pgfscope}%
\pgfsetrectcap%
\pgfsetroundjoin%
\pgfsetlinewidth{1.505625pt}%
\definecolor{currentstroke}{rgb}{0.839216,0.152941,0.156863}%
\pgfsetstrokecolor{currentstroke}%
\pgfsetdash{}{0pt}%
\pgfpathmoveto{\pgfqpoint{1.176788in}{1.885556in}}%
\pgfpathlineto{\pgfqpoint{1.454566in}{1.885556in}}%
\pgfusepath{stroke}%
\end{pgfscope}%
\begin{pgfscope}%
\definecolor{textcolor}{rgb}{0.000000,0.000000,0.000000}%
\pgfsetstrokecolor{textcolor}%
\pgfsetfillcolor{textcolor}%
\pgftext[x=1.565677in,y=1.836945in,left,base]{\color{textcolor}\rmfamily\fontsize{10.000000}{12.000000}\selectfont \(\displaystyle x^{2}\)}%
\end{pgfscope}%
\begin{pgfscope}%
\pgfsetrectcap%
\pgfsetroundjoin%
\pgfsetlinewidth{1.505625pt}%
\definecolor{currentstroke}{rgb}{0.580392,0.403922,0.741176}%
\pgfsetstrokecolor{currentstroke}%
\pgfsetdash{}{0pt}%
\pgfpathmoveto{\pgfqpoint{1.176788in}{1.691945in}}%
\pgfpathlineto{\pgfqpoint{1.454566in}{1.691945in}}%
\pgfusepath{stroke}%
\end{pgfscope}%
\begin{pgfscope}%
\definecolor{textcolor}{rgb}{0.000000,0.000000,0.000000}%
\pgfsetstrokecolor{textcolor}%
\pgfsetfillcolor{textcolor}%
\pgftext[x=1.565677in,y=1.643334in,left,base]{\color{textcolor}\rmfamily\fontsize{10.000000}{12.000000}\selectfont \(\displaystyle x^{3}\)}%
\end{pgfscope}%
\end{pgfpicture}%
\makeatother%
\endgroup%

    \caption{Ableitungen der Polynome\label{polynomials:diff}}
\end{figure}

Dieses Verhalten ist zu erwarten wenn man bedenkt, dass das Haar Wavelet
jeweils die Summe und die Differenz zweier benachbarter Samples analysiert.
Siehe dazu auch \autoref{haar:allwavelets:image} im Kapitel zum Haar Wavelet.

Nun stellt sich die Frage was passiert wenn wir Daubechies Wavelets mit mehr
verschwindenden Momenten zur Analyse einsetzen. In \autoref{polynomials:db2_3}
sind jeweils die Detailkoeffizienten der Analyse mit db1 und db2 Wavelet zu
sehen.

\begin{figure}
    \centering
    %% Creator: Matplotlib, PGF backend
%%
%% To include the figure in your LaTeX document, write
%%   \input{<filename>.pgf}
%%
%% Make sure the required packages are loaded in your preamble
%%   \usepackage{pgf}
%%
%% Figures using additional raster images can only be included by \input if
%% they are in the same directory as the main LaTeX file. For loading figures
%% from other directories you can use the `import` package
%%   \usepackage{import}
%% and then include the figures with
%%   \import{<path to file>}{<filename>.pgf}
%%
%% Matplotlib used the following preamble
%%   \usepackage{fontspec}
%%
\begingroup%
\makeatletter%
\begin{pgfpicture}%
\pgfpathrectangle{\pgfpointorigin}{\pgfqpoint{5.800000in}{3.000000in}}%
\pgfusepath{use as bounding box, clip}%
\begin{pgfscope}%
\pgfsetbuttcap%
\pgfsetmiterjoin%
\definecolor{currentfill}{rgb}{1.000000,1.000000,1.000000}%
\pgfsetfillcolor{currentfill}%
\pgfsetlinewidth{0.000000pt}%
\definecolor{currentstroke}{rgb}{1.000000,1.000000,1.000000}%
\pgfsetstrokecolor{currentstroke}%
\pgfsetdash{}{0pt}%
\pgfpathmoveto{\pgfqpoint{0.000000in}{0.000000in}}%
\pgfpathlineto{\pgfqpoint{5.800000in}{0.000000in}}%
\pgfpathlineto{\pgfqpoint{5.800000in}{3.000000in}}%
\pgfpathlineto{\pgfqpoint{0.000000in}{3.000000in}}%
\pgfpathclose%
\pgfusepath{fill}%
\end{pgfscope}%
\begin{pgfscope}%
\pgfsetbuttcap%
\pgfsetmiterjoin%
\definecolor{currentfill}{rgb}{1.000000,1.000000,1.000000}%
\pgfsetfillcolor{currentfill}%
\pgfsetlinewidth{0.000000pt}%
\definecolor{currentstroke}{rgb}{0.000000,0.000000,0.000000}%
\pgfsetstrokecolor{currentstroke}%
\pgfsetstrokeopacity{0.000000}%
\pgfsetdash{}{0pt}%
\pgfpathmoveto{\pgfqpoint{0.368000in}{0.315889in}}%
\pgfpathlineto{\pgfqpoint{2.745500in}{0.315889in}}%
\pgfpathlineto{\pgfqpoint{2.745500in}{2.704133in}}%
\pgfpathlineto{\pgfqpoint{0.368000in}{2.704133in}}%
\pgfpathclose%
\pgfusepath{fill}%
\end{pgfscope}%
\begin{pgfscope}%
\pgfsetbuttcap%
\pgfsetroundjoin%
\definecolor{currentfill}{rgb}{0.000000,0.000000,0.000000}%
\pgfsetfillcolor{currentfill}%
\pgfsetlinewidth{0.803000pt}%
\definecolor{currentstroke}{rgb}{0.000000,0.000000,0.000000}%
\pgfsetstrokecolor{currentstroke}%
\pgfsetdash{}{0pt}%
\pgfsys@defobject{currentmarker}{\pgfqpoint{0.000000in}{-0.048611in}}{\pgfqpoint{0.000000in}{0.000000in}}{%
\pgfpathmoveto{\pgfqpoint{0.000000in}{0.000000in}}%
\pgfpathlineto{\pgfqpoint{0.000000in}{-0.048611in}}%
\pgfusepath{stroke,fill}%
}%
\begin{pgfscope}%
\pgfsys@transformshift{0.476068in}{0.315889in}%
\pgfsys@useobject{currentmarker}{}%
\end{pgfscope}%
\end{pgfscope}%
\begin{pgfscope}%
\definecolor{textcolor}{rgb}{0.000000,0.000000,0.000000}%
\pgfsetstrokecolor{textcolor}%
\pgfsetfillcolor{textcolor}%
\pgftext[x=0.476068in,y=0.218667in,,top]{\color{textcolor}\rmfamily\fontsize{8.000000}{9.600000}\selectfont 0}%
\end{pgfscope}%
\begin{pgfscope}%
\pgfsetbuttcap%
\pgfsetroundjoin%
\definecolor{currentfill}{rgb}{0.000000,0.000000,0.000000}%
\pgfsetfillcolor{currentfill}%
\pgfsetlinewidth{0.803000pt}%
\definecolor{currentstroke}{rgb}{0.000000,0.000000,0.000000}%
\pgfsetstrokecolor{currentstroke}%
\pgfsetdash{}{0pt}%
\pgfsys@defobject{currentmarker}{\pgfqpoint{0.000000in}{-0.048611in}}{\pgfqpoint{0.000000in}{0.000000in}}{%
\pgfpathmoveto{\pgfqpoint{0.000000in}{0.000000in}}%
\pgfpathlineto{\pgfqpoint{0.000000in}{-0.048611in}}%
\pgfusepath{stroke,fill}%
}%
\begin{pgfscope}%
\pgfsys@transformshift{0.819142in}{0.315889in}%
\pgfsys@useobject{currentmarker}{}%
\end{pgfscope}%
\end{pgfscope}%
\begin{pgfscope}%
\definecolor{textcolor}{rgb}{0.000000,0.000000,0.000000}%
\pgfsetstrokecolor{textcolor}%
\pgfsetfillcolor{textcolor}%
\pgftext[x=0.819142in,y=0.218667in,,top]{\color{textcolor}\rmfamily\fontsize{8.000000}{9.600000}\selectfont 20}%
\end{pgfscope}%
\begin{pgfscope}%
\pgfsetbuttcap%
\pgfsetroundjoin%
\definecolor{currentfill}{rgb}{0.000000,0.000000,0.000000}%
\pgfsetfillcolor{currentfill}%
\pgfsetlinewidth{0.803000pt}%
\definecolor{currentstroke}{rgb}{0.000000,0.000000,0.000000}%
\pgfsetstrokecolor{currentstroke}%
\pgfsetdash{}{0pt}%
\pgfsys@defobject{currentmarker}{\pgfqpoint{0.000000in}{-0.048611in}}{\pgfqpoint{0.000000in}{0.000000in}}{%
\pgfpathmoveto{\pgfqpoint{0.000000in}{0.000000in}}%
\pgfpathlineto{\pgfqpoint{0.000000in}{-0.048611in}}%
\pgfusepath{stroke,fill}%
}%
\begin{pgfscope}%
\pgfsys@transformshift{1.162215in}{0.315889in}%
\pgfsys@useobject{currentmarker}{}%
\end{pgfscope}%
\end{pgfscope}%
\begin{pgfscope}%
\definecolor{textcolor}{rgb}{0.000000,0.000000,0.000000}%
\pgfsetstrokecolor{textcolor}%
\pgfsetfillcolor{textcolor}%
\pgftext[x=1.162215in,y=0.218667in,,top]{\color{textcolor}\rmfamily\fontsize{8.000000}{9.600000}\selectfont 40}%
\end{pgfscope}%
\begin{pgfscope}%
\pgfsetbuttcap%
\pgfsetroundjoin%
\definecolor{currentfill}{rgb}{0.000000,0.000000,0.000000}%
\pgfsetfillcolor{currentfill}%
\pgfsetlinewidth{0.803000pt}%
\definecolor{currentstroke}{rgb}{0.000000,0.000000,0.000000}%
\pgfsetstrokecolor{currentstroke}%
\pgfsetdash{}{0pt}%
\pgfsys@defobject{currentmarker}{\pgfqpoint{0.000000in}{-0.048611in}}{\pgfqpoint{0.000000in}{0.000000in}}{%
\pgfpathmoveto{\pgfqpoint{0.000000in}{0.000000in}}%
\pgfpathlineto{\pgfqpoint{0.000000in}{-0.048611in}}%
\pgfusepath{stroke,fill}%
}%
\begin{pgfscope}%
\pgfsys@transformshift{1.505289in}{0.315889in}%
\pgfsys@useobject{currentmarker}{}%
\end{pgfscope}%
\end{pgfscope}%
\begin{pgfscope}%
\definecolor{textcolor}{rgb}{0.000000,0.000000,0.000000}%
\pgfsetstrokecolor{textcolor}%
\pgfsetfillcolor{textcolor}%
\pgftext[x=1.505289in,y=0.218667in,,top]{\color{textcolor}\rmfamily\fontsize{8.000000}{9.600000}\selectfont 60}%
\end{pgfscope}%
\begin{pgfscope}%
\pgfsetbuttcap%
\pgfsetroundjoin%
\definecolor{currentfill}{rgb}{0.000000,0.000000,0.000000}%
\pgfsetfillcolor{currentfill}%
\pgfsetlinewidth{0.803000pt}%
\definecolor{currentstroke}{rgb}{0.000000,0.000000,0.000000}%
\pgfsetstrokecolor{currentstroke}%
\pgfsetdash{}{0pt}%
\pgfsys@defobject{currentmarker}{\pgfqpoint{0.000000in}{-0.048611in}}{\pgfqpoint{0.000000in}{0.000000in}}{%
\pgfpathmoveto{\pgfqpoint{0.000000in}{0.000000in}}%
\pgfpathlineto{\pgfqpoint{0.000000in}{-0.048611in}}%
\pgfusepath{stroke,fill}%
}%
\begin{pgfscope}%
\pgfsys@transformshift{1.848363in}{0.315889in}%
\pgfsys@useobject{currentmarker}{}%
\end{pgfscope}%
\end{pgfscope}%
\begin{pgfscope}%
\definecolor{textcolor}{rgb}{0.000000,0.000000,0.000000}%
\pgfsetstrokecolor{textcolor}%
\pgfsetfillcolor{textcolor}%
\pgftext[x=1.848363in,y=0.218667in,,top]{\color{textcolor}\rmfamily\fontsize{8.000000}{9.600000}\selectfont 80}%
\end{pgfscope}%
\begin{pgfscope}%
\pgfsetbuttcap%
\pgfsetroundjoin%
\definecolor{currentfill}{rgb}{0.000000,0.000000,0.000000}%
\pgfsetfillcolor{currentfill}%
\pgfsetlinewidth{0.803000pt}%
\definecolor{currentstroke}{rgb}{0.000000,0.000000,0.000000}%
\pgfsetstrokecolor{currentstroke}%
\pgfsetdash{}{0pt}%
\pgfsys@defobject{currentmarker}{\pgfqpoint{0.000000in}{-0.048611in}}{\pgfqpoint{0.000000in}{0.000000in}}{%
\pgfpathmoveto{\pgfqpoint{0.000000in}{0.000000in}}%
\pgfpathlineto{\pgfqpoint{0.000000in}{-0.048611in}}%
\pgfusepath{stroke,fill}%
}%
\begin{pgfscope}%
\pgfsys@transformshift{2.191436in}{0.315889in}%
\pgfsys@useobject{currentmarker}{}%
\end{pgfscope}%
\end{pgfscope}%
\begin{pgfscope}%
\definecolor{textcolor}{rgb}{0.000000,0.000000,0.000000}%
\pgfsetstrokecolor{textcolor}%
\pgfsetfillcolor{textcolor}%
\pgftext[x=2.191436in,y=0.218667in,,top]{\color{textcolor}\rmfamily\fontsize{8.000000}{9.600000}\selectfont 100}%
\end{pgfscope}%
\begin{pgfscope}%
\pgfsetbuttcap%
\pgfsetroundjoin%
\definecolor{currentfill}{rgb}{0.000000,0.000000,0.000000}%
\pgfsetfillcolor{currentfill}%
\pgfsetlinewidth{0.803000pt}%
\definecolor{currentstroke}{rgb}{0.000000,0.000000,0.000000}%
\pgfsetstrokecolor{currentstroke}%
\pgfsetdash{}{0pt}%
\pgfsys@defobject{currentmarker}{\pgfqpoint{0.000000in}{-0.048611in}}{\pgfqpoint{0.000000in}{0.000000in}}{%
\pgfpathmoveto{\pgfqpoint{0.000000in}{0.000000in}}%
\pgfpathlineto{\pgfqpoint{0.000000in}{-0.048611in}}%
\pgfusepath{stroke,fill}%
}%
\begin{pgfscope}%
\pgfsys@transformshift{2.534510in}{0.315889in}%
\pgfsys@useobject{currentmarker}{}%
\end{pgfscope}%
\end{pgfscope}%
\begin{pgfscope}%
\definecolor{textcolor}{rgb}{0.000000,0.000000,0.000000}%
\pgfsetstrokecolor{textcolor}%
\pgfsetfillcolor{textcolor}%
\pgftext[x=2.534510in,y=0.218667in,,top]{\color{textcolor}\rmfamily\fontsize{8.000000}{9.600000}\selectfont 120}%
\end{pgfscope}%
\begin{pgfscope}%
\pgfsetbuttcap%
\pgfsetroundjoin%
\definecolor{currentfill}{rgb}{0.000000,0.000000,0.000000}%
\pgfsetfillcolor{currentfill}%
\pgfsetlinewidth{0.803000pt}%
\definecolor{currentstroke}{rgb}{0.000000,0.000000,0.000000}%
\pgfsetstrokecolor{currentstroke}%
\pgfsetdash{}{0pt}%
\pgfsys@defobject{currentmarker}{\pgfqpoint{-0.048611in}{0.000000in}}{\pgfqpoint{0.000000in}{0.000000in}}{%
\pgfpathmoveto{\pgfqpoint{0.000000in}{0.000000in}}%
\pgfpathlineto{\pgfqpoint{-0.048611in}{0.000000in}}%
\pgfusepath{stroke,fill}%
}%
\begin{pgfscope}%
\pgfsys@transformshift{0.368000in}{0.665220in}%
\pgfsys@useobject{currentmarker}{}%
\end{pgfscope}%
\end{pgfscope}%
\begin{pgfscope}%
\definecolor{textcolor}{rgb}{0.000000,0.000000,0.000000}%
\pgfsetstrokecolor{textcolor}%
\pgfsetfillcolor{textcolor}%
\pgftext[x=0.120000in,y=0.626665in,left,base]{\color{textcolor}\rmfamily\fontsize{8.000000}{9.600000}\selectfont −4}%
\end{pgfscope}%
\begin{pgfscope}%
\pgfsetbuttcap%
\pgfsetroundjoin%
\definecolor{currentfill}{rgb}{0.000000,0.000000,0.000000}%
\pgfsetfillcolor{currentfill}%
\pgfsetlinewidth{0.803000pt}%
\definecolor{currentstroke}{rgb}{0.000000,0.000000,0.000000}%
\pgfsetstrokecolor{currentstroke}%
\pgfsetdash{}{0pt}%
\pgfsys@defobject{currentmarker}{\pgfqpoint{-0.048611in}{0.000000in}}{\pgfqpoint{0.000000in}{0.000000in}}{%
\pgfpathmoveto{\pgfqpoint{0.000000in}{0.000000in}}%
\pgfpathlineto{\pgfqpoint{-0.048611in}{0.000000in}}%
\pgfusepath{stroke,fill}%
}%
\begin{pgfscope}%
\pgfsys@transformshift{0.368000in}{1.147809in}%
\pgfsys@useobject{currentmarker}{}%
\end{pgfscope}%
\end{pgfscope}%
\begin{pgfscope}%
\definecolor{textcolor}{rgb}{0.000000,0.000000,0.000000}%
\pgfsetstrokecolor{textcolor}%
\pgfsetfillcolor{textcolor}%
\pgftext[x=0.120000in,y=1.109254in,left,base]{\color{textcolor}\rmfamily\fontsize{8.000000}{9.600000}\selectfont −3}%
\end{pgfscope}%
\begin{pgfscope}%
\pgfsetbuttcap%
\pgfsetroundjoin%
\definecolor{currentfill}{rgb}{0.000000,0.000000,0.000000}%
\pgfsetfillcolor{currentfill}%
\pgfsetlinewidth{0.803000pt}%
\definecolor{currentstroke}{rgb}{0.000000,0.000000,0.000000}%
\pgfsetstrokecolor{currentstroke}%
\pgfsetdash{}{0pt}%
\pgfsys@defobject{currentmarker}{\pgfqpoint{-0.048611in}{0.000000in}}{\pgfqpoint{0.000000in}{0.000000in}}{%
\pgfpathmoveto{\pgfqpoint{0.000000in}{0.000000in}}%
\pgfpathlineto{\pgfqpoint{-0.048611in}{0.000000in}}%
\pgfusepath{stroke,fill}%
}%
\begin{pgfscope}%
\pgfsys@transformshift{0.368000in}{1.630398in}%
\pgfsys@useobject{currentmarker}{}%
\end{pgfscope}%
\end{pgfscope}%
\begin{pgfscope}%
\definecolor{textcolor}{rgb}{0.000000,0.000000,0.000000}%
\pgfsetstrokecolor{textcolor}%
\pgfsetfillcolor{textcolor}%
\pgftext[x=0.120000in,y=1.591843in,left,base]{\color{textcolor}\rmfamily\fontsize{8.000000}{9.600000}\selectfont −2}%
\end{pgfscope}%
\begin{pgfscope}%
\pgfsetbuttcap%
\pgfsetroundjoin%
\definecolor{currentfill}{rgb}{0.000000,0.000000,0.000000}%
\pgfsetfillcolor{currentfill}%
\pgfsetlinewidth{0.803000pt}%
\definecolor{currentstroke}{rgb}{0.000000,0.000000,0.000000}%
\pgfsetstrokecolor{currentstroke}%
\pgfsetdash{}{0pt}%
\pgfsys@defobject{currentmarker}{\pgfqpoint{-0.048611in}{0.000000in}}{\pgfqpoint{0.000000in}{0.000000in}}{%
\pgfpathmoveto{\pgfqpoint{0.000000in}{0.000000in}}%
\pgfpathlineto{\pgfqpoint{-0.048611in}{0.000000in}}%
\pgfusepath{stroke,fill}%
}%
\begin{pgfscope}%
\pgfsys@transformshift{0.368000in}{2.112988in}%
\pgfsys@useobject{currentmarker}{}%
\end{pgfscope}%
\end{pgfscope}%
\begin{pgfscope}%
\definecolor{textcolor}{rgb}{0.000000,0.000000,0.000000}%
\pgfsetstrokecolor{textcolor}%
\pgfsetfillcolor{textcolor}%
\pgftext[x=0.120000in,y=2.074432in,left,base]{\color{textcolor}\rmfamily\fontsize{8.000000}{9.600000}\selectfont −1}%
\end{pgfscope}%
\begin{pgfscope}%
\pgfsetbuttcap%
\pgfsetroundjoin%
\definecolor{currentfill}{rgb}{0.000000,0.000000,0.000000}%
\pgfsetfillcolor{currentfill}%
\pgfsetlinewidth{0.803000pt}%
\definecolor{currentstroke}{rgb}{0.000000,0.000000,0.000000}%
\pgfsetstrokecolor{currentstroke}%
\pgfsetdash{}{0pt}%
\pgfsys@defobject{currentmarker}{\pgfqpoint{-0.048611in}{0.000000in}}{\pgfqpoint{0.000000in}{0.000000in}}{%
\pgfpathmoveto{\pgfqpoint{0.000000in}{0.000000in}}%
\pgfpathlineto{\pgfqpoint{-0.048611in}{0.000000in}}%
\pgfusepath{stroke,fill}%
}%
\begin{pgfscope}%
\pgfsys@transformshift{0.368000in}{2.595577in}%
\pgfsys@useobject{currentmarker}{}%
\end{pgfscope}%
\end{pgfscope}%
\begin{pgfscope}%
\definecolor{textcolor}{rgb}{0.000000,0.000000,0.000000}%
\pgfsetstrokecolor{textcolor}%
\pgfsetfillcolor{textcolor}%
\pgftext[x=0.211778in,y=2.557021in,left,base]{\color{textcolor}\rmfamily\fontsize{8.000000}{9.600000}\selectfont 0}%
\end{pgfscope}%
\begin{pgfscope}%
\definecolor{textcolor}{rgb}{0.000000,0.000000,0.000000}%
\pgfsetstrokecolor{textcolor}%
\pgfsetfillcolor{textcolor}%
\pgftext[x=0.368000in,y=2.745800in,left,base]{\color{textcolor}\rmfamily\fontsize{8.000000}{9.600000}\selectfont 1e−4}%
\end{pgfscope}%
\begin{pgfscope}%
\pgfpathrectangle{\pgfqpoint{0.368000in}{0.315889in}}{\pgfqpoint{2.377500in}{2.388245in}}%
\pgfusepath{clip}%
\pgfsetrectcap%
\pgfsetroundjoin%
\pgfsetlinewidth{1.505625pt}%
\definecolor{currentstroke}{rgb}{0.121569,0.466667,0.705882}%
\pgfsetstrokecolor{currentstroke}%
\pgfsetdash{}{0pt}%
\pgfpathmoveto{\pgfqpoint{0.476068in}{2.595577in}}%
\pgfpathlineto{\pgfqpoint{0.493222in}{2.595577in}}%
\pgfpathlineto{\pgfqpoint{0.510376in}{2.595577in}}%
\pgfpathlineto{\pgfqpoint{0.527529in}{2.595577in}}%
\pgfpathlineto{\pgfqpoint{0.544683in}{2.595577in}}%
\pgfpathlineto{\pgfqpoint{0.561837in}{2.595577in}}%
\pgfpathlineto{\pgfqpoint{0.578990in}{2.595577in}}%
\pgfpathlineto{\pgfqpoint{0.596144in}{2.595577in}}%
\pgfpathlineto{\pgfqpoint{0.613298in}{2.595577in}}%
\pgfpathlineto{\pgfqpoint{0.630451in}{2.595577in}}%
\pgfpathlineto{\pgfqpoint{0.647605in}{2.595577in}}%
\pgfpathlineto{\pgfqpoint{0.664759in}{2.595577in}}%
\pgfpathlineto{\pgfqpoint{0.681912in}{2.595577in}}%
\pgfpathlineto{\pgfqpoint{0.699066in}{2.595577in}}%
\pgfpathlineto{\pgfqpoint{0.716220in}{2.595577in}}%
\pgfpathlineto{\pgfqpoint{0.733373in}{2.595577in}}%
\pgfpathlineto{\pgfqpoint{0.750527in}{2.595577in}}%
\pgfpathlineto{\pgfqpoint{0.767681in}{2.595577in}}%
\pgfpathlineto{\pgfqpoint{0.784834in}{2.595577in}}%
\pgfpathlineto{\pgfqpoint{0.801988in}{2.595577in}}%
\pgfpathlineto{\pgfqpoint{0.819142in}{2.595577in}}%
\pgfpathlineto{\pgfqpoint{0.836295in}{2.595577in}}%
\pgfpathlineto{\pgfqpoint{0.853449in}{2.595577in}}%
\pgfpathlineto{\pgfqpoint{0.870603in}{2.595577in}}%
\pgfpathlineto{\pgfqpoint{0.887756in}{2.595577in}}%
\pgfpathlineto{\pgfqpoint{0.904910in}{2.595577in}}%
\pgfpathlineto{\pgfqpoint{0.922064in}{2.595577in}}%
\pgfpathlineto{\pgfqpoint{0.939218in}{2.595577in}}%
\pgfpathlineto{\pgfqpoint{0.956371in}{2.595577in}}%
\pgfpathlineto{\pgfqpoint{0.973525in}{2.595577in}}%
\pgfpathlineto{\pgfqpoint{0.990679in}{2.595577in}}%
\pgfpathlineto{\pgfqpoint{1.007832in}{2.595577in}}%
\pgfpathlineto{\pgfqpoint{1.024986in}{2.595577in}}%
\pgfpathlineto{\pgfqpoint{1.042140in}{2.595577in}}%
\pgfpathlineto{\pgfqpoint{1.059293in}{2.595577in}}%
\pgfpathlineto{\pgfqpoint{1.076447in}{2.595577in}}%
\pgfpathlineto{\pgfqpoint{1.093601in}{2.595577in}}%
\pgfpathlineto{\pgfqpoint{1.110754in}{2.595577in}}%
\pgfpathlineto{\pgfqpoint{1.127908in}{2.595577in}}%
\pgfpathlineto{\pgfqpoint{1.145062in}{2.595577in}}%
\pgfpathlineto{\pgfqpoint{1.162215in}{2.595577in}}%
\pgfpathlineto{\pgfqpoint{1.179369in}{2.595577in}}%
\pgfpathlineto{\pgfqpoint{1.196523in}{2.595577in}}%
\pgfpathlineto{\pgfqpoint{1.213676in}{2.595577in}}%
\pgfpathlineto{\pgfqpoint{1.230830in}{2.595577in}}%
\pgfpathlineto{\pgfqpoint{1.247984in}{2.595577in}}%
\pgfpathlineto{\pgfqpoint{1.265137in}{2.595577in}}%
\pgfpathlineto{\pgfqpoint{1.282291in}{2.595577in}}%
\pgfpathlineto{\pgfqpoint{1.299445in}{2.595577in}}%
\pgfpathlineto{\pgfqpoint{1.316598in}{2.595577in}}%
\pgfpathlineto{\pgfqpoint{1.333752in}{2.595577in}}%
\pgfpathlineto{\pgfqpoint{1.350906in}{2.595577in}}%
\pgfpathlineto{\pgfqpoint{1.368060in}{2.595577in}}%
\pgfpathlineto{\pgfqpoint{1.385213in}{2.595577in}}%
\pgfpathlineto{\pgfqpoint{1.402367in}{2.595577in}}%
\pgfpathlineto{\pgfqpoint{1.419521in}{2.595577in}}%
\pgfpathlineto{\pgfqpoint{1.436674in}{2.595577in}}%
\pgfpathlineto{\pgfqpoint{1.453828in}{2.595577in}}%
\pgfpathlineto{\pgfqpoint{1.470982in}{2.595577in}}%
\pgfpathlineto{\pgfqpoint{1.488135in}{2.595577in}}%
\pgfpathlineto{\pgfqpoint{1.505289in}{2.595577in}}%
\pgfpathlineto{\pgfqpoint{1.522443in}{2.595577in}}%
\pgfpathlineto{\pgfqpoint{1.539596in}{2.595577in}}%
\pgfpathlineto{\pgfqpoint{1.556750in}{2.595577in}}%
\pgfpathlineto{\pgfqpoint{1.573904in}{2.595577in}}%
\pgfpathlineto{\pgfqpoint{1.591057in}{2.595577in}}%
\pgfpathlineto{\pgfqpoint{1.608211in}{2.595577in}}%
\pgfpathlineto{\pgfqpoint{1.625365in}{2.595577in}}%
\pgfpathlineto{\pgfqpoint{1.642518in}{2.595577in}}%
\pgfpathlineto{\pgfqpoint{1.659672in}{2.595577in}}%
\pgfpathlineto{\pgfqpoint{1.676826in}{2.595577in}}%
\pgfpathlineto{\pgfqpoint{1.693979in}{2.595577in}}%
\pgfpathlineto{\pgfqpoint{1.711133in}{2.595577in}}%
\pgfpathlineto{\pgfqpoint{1.728287in}{2.595577in}}%
\pgfpathlineto{\pgfqpoint{1.745440in}{2.595577in}}%
\pgfpathlineto{\pgfqpoint{1.762594in}{2.595577in}}%
\pgfpathlineto{\pgfqpoint{1.779748in}{2.595577in}}%
\pgfpathlineto{\pgfqpoint{1.796902in}{2.595577in}}%
\pgfpathlineto{\pgfqpoint{1.814055in}{2.595577in}}%
\pgfpathlineto{\pgfqpoint{1.831209in}{2.595577in}}%
\pgfpathlineto{\pgfqpoint{1.848363in}{2.595577in}}%
\pgfpathlineto{\pgfqpoint{1.865516in}{2.595577in}}%
\pgfpathlineto{\pgfqpoint{1.882670in}{2.595577in}}%
\pgfpathlineto{\pgfqpoint{1.899824in}{2.595577in}}%
\pgfpathlineto{\pgfqpoint{1.916977in}{2.595577in}}%
\pgfpathlineto{\pgfqpoint{1.934131in}{2.595577in}}%
\pgfpathlineto{\pgfqpoint{1.951285in}{2.595577in}}%
\pgfpathlineto{\pgfqpoint{1.968438in}{2.595577in}}%
\pgfpathlineto{\pgfqpoint{1.985592in}{2.595577in}}%
\pgfpathlineto{\pgfqpoint{2.002746in}{2.595577in}}%
\pgfpathlineto{\pgfqpoint{2.019899in}{2.595577in}}%
\pgfpathlineto{\pgfqpoint{2.037053in}{2.595577in}}%
\pgfpathlineto{\pgfqpoint{2.054207in}{2.595577in}}%
\pgfpathlineto{\pgfqpoint{2.071360in}{2.595577in}}%
\pgfpathlineto{\pgfqpoint{2.088514in}{2.595577in}}%
\pgfpathlineto{\pgfqpoint{2.105668in}{2.595577in}}%
\pgfpathlineto{\pgfqpoint{2.122821in}{2.595577in}}%
\pgfpathlineto{\pgfqpoint{2.139975in}{2.595577in}}%
\pgfpathlineto{\pgfqpoint{2.157129in}{2.595577in}}%
\pgfpathlineto{\pgfqpoint{2.174282in}{2.595577in}}%
\pgfpathlineto{\pgfqpoint{2.191436in}{2.595577in}}%
\pgfpathlineto{\pgfqpoint{2.208590in}{2.595577in}}%
\pgfpathlineto{\pgfqpoint{2.225744in}{2.595577in}}%
\pgfpathlineto{\pgfqpoint{2.242897in}{2.595577in}}%
\pgfpathlineto{\pgfqpoint{2.260051in}{2.595577in}}%
\pgfpathlineto{\pgfqpoint{2.277205in}{2.595577in}}%
\pgfpathlineto{\pgfqpoint{2.294358in}{2.595577in}}%
\pgfpathlineto{\pgfqpoint{2.311512in}{2.595577in}}%
\pgfpathlineto{\pgfqpoint{2.328666in}{2.595577in}}%
\pgfpathlineto{\pgfqpoint{2.345819in}{2.595577in}}%
\pgfpathlineto{\pgfqpoint{2.362973in}{2.595577in}}%
\pgfpathlineto{\pgfqpoint{2.380127in}{2.595577in}}%
\pgfpathlineto{\pgfqpoint{2.397280in}{2.595577in}}%
\pgfpathlineto{\pgfqpoint{2.414434in}{2.595577in}}%
\pgfpathlineto{\pgfqpoint{2.431588in}{2.595577in}}%
\pgfpathlineto{\pgfqpoint{2.448741in}{2.595577in}}%
\pgfpathlineto{\pgfqpoint{2.465895in}{2.595577in}}%
\pgfpathlineto{\pgfqpoint{2.483049in}{2.595577in}}%
\pgfpathlineto{\pgfqpoint{2.500202in}{2.595577in}}%
\pgfpathlineto{\pgfqpoint{2.517356in}{2.595577in}}%
\pgfpathlineto{\pgfqpoint{2.534510in}{2.595577in}}%
\pgfpathlineto{\pgfqpoint{2.551663in}{2.595577in}}%
\pgfpathlineto{\pgfqpoint{2.568817in}{2.595577in}}%
\pgfpathlineto{\pgfqpoint{2.585971in}{2.595577in}}%
\pgfpathlineto{\pgfqpoint{2.603124in}{2.595577in}}%
\pgfpathlineto{\pgfqpoint{2.620278in}{2.595577in}}%
\pgfpathlineto{\pgfqpoint{2.637432in}{2.595577in}}%
\pgfusepath{stroke}%
\end{pgfscope}%
\begin{pgfscope}%
\pgfpathrectangle{\pgfqpoint{0.368000in}{0.315889in}}{\pgfqpoint{2.377500in}{2.388245in}}%
\pgfusepath{clip}%
\pgfsetrectcap%
\pgfsetroundjoin%
\pgfsetlinewidth{1.505625pt}%
\definecolor{currentstroke}{rgb}{1.000000,0.498039,0.054902}%
\pgfsetstrokecolor{currentstroke}%
\pgfsetdash{}{0pt}%
\pgfpathmoveto{\pgfqpoint{0.476068in}{2.595577in}}%
\pgfpathlineto{\pgfqpoint{0.493222in}{2.595577in}}%
\pgfpathlineto{\pgfqpoint{0.510376in}{2.595577in}}%
\pgfpathlineto{\pgfqpoint{0.527529in}{2.595577in}}%
\pgfpathlineto{\pgfqpoint{0.544683in}{2.595577in}}%
\pgfpathlineto{\pgfqpoint{0.561837in}{2.595577in}}%
\pgfpathlineto{\pgfqpoint{0.578990in}{2.595577in}}%
\pgfpathlineto{\pgfqpoint{0.596144in}{2.595577in}}%
\pgfpathlineto{\pgfqpoint{0.613298in}{2.595577in}}%
\pgfpathlineto{\pgfqpoint{0.630451in}{2.595577in}}%
\pgfpathlineto{\pgfqpoint{0.647605in}{2.595577in}}%
\pgfpathlineto{\pgfqpoint{0.664759in}{2.595577in}}%
\pgfpathlineto{\pgfqpoint{0.681912in}{2.595577in}}%
\pgfpathlineto{\pgfqpoint{0.699066in}{2.595577in}}%
\pgfpathlineto{\pgfqpoint{0.716220in}{2.595577in}}%
\pgfpathlineto{\pgfqpoint{0.733373in}{2.595577in}}%
\pgfpathlineto{\pgfqpoint{0.750527in}{2.595577in}}%
\pgfpathlineto{\pgfqpoint{0.767681in}{2.595577in}}%
\pgfpathlineto{\pgfqpoint{0.784834in}{2.595577in}}%
\pgfpathlineto{\pgfqpoint{0.801988in}{2.595577in}}%
\pgfpathlineto{\pgfqpoint{0.819142in}{2.595577in}}%
\pgfpathlineto{\pgfqpoint{0.836295in}{2.595577in}}%
\pgfpathlineto{\pgfqpoint{0.853449in}{2.595577in}}%
\pgfpathlineto{\pgfqpoint{0.870603in}{2.595577in}}%
\pgfpathlineto{\pgfqpoint{0.887756in}{2.595577in}}%
\pgfpathlineto{\pgfqpoint{0.904910in}{2.595577in}}%
\pgfpathlineto{\pgfqpoint{0.922064in}{2.595577in}}%
\pgfpathlineto{\pgfqpoint{0.939218in}{2.595577in}}%
\pgfpathlineto{\pgfqpoint{0.956371in}{2.595577in}}%
\pgfpathlineto{\pgfqpoint{0.973525in}{2.595577in}}%
\pgfpathlineto{\pgfqpoint{0.990679in}{2.595577in}}%
\pgfpathlineto{\pgfqpoint{1.007832in}{2.595577in}}%
\pgfpathlineto{\pgfqpoint{1.024986in}{2.595577in}}%
\pgfpathlineto{\pgfqpoint{1.042140in}{2.595577in}}%
\pgfpathlineto{\pgfqpoint{1.059293in}{2.595577in}}%
\pgfpathlineto{\pgfqpoint{1.076447in}{2.595577in}}%
\pgfpathlineto{\pgfqpoint{1.093601in}{2.595577in}}%
\pgfpathlineto{\pgfqpoint{1.110754in}{2.595577in}}%
\pgfpathlineto{\pgfqpoint{1.127908in}{2.595577in}}%
\pgfpathlineto{\pgfqpoint{1.145062in}{2.595577in}}%
\pgfpathlineto{\pgfqpoint{1.162215in}{2.595577in}}%
\pgfpathlineto{\pgfqpoint{1.179369in}{2.595577in}}%
\pgfpathlineto{\pgfqpoint{1.196523in}{2.595577in}}%
\pgfpathlineto{\pgfqpoint{1.213676in}{2.595577in}}%
\pgfpathlineto{\pgfqpoint{1.230830in}{2.595577in}}%
\pgfpathlineto{\pgfqpoint{1.247984in}{2.595577in}}%
\pgfpathlineto{\pgfqpoint{1.265137in}{2.595577in}}%
\pgfpathlineto{\pgfqpoint{1.282291in}{2.595577in}}%
\pgfpathlineto{\pgfqpoint{1.299445in}{2.595577in}}%
\pgfpathlineto{\pgfqpoint{1.316598in}{2.595577in}}%
\pgfpathlineto{\pgfqpoint{1.333752in}{2.595577in}}%
\pgfpathlineto{\pgfqpoint{1.350906in}{2.595577in}}%
\pgfpathlineto{\pgfqpoint{1.368060in}{2.595577in}}%
\pgfpathlineto{\pgfqpoint{1.385213in}{2.595577in}}%
\pgfpathlineto{\pgfqpoint{1.402367in}{2.595577in}}%
\pgfpathlineto{\pgfqpoint{1.419521in}{2.595577in}}%
\pgfpathlineto{\pgfqpoint{1.436674in}{2.595577in}}%
\pgfpathlineto{\pgfqpoint{1.453828in}{2.595577in}}%
\pgfpathlineto{\pgfqpoint{1.470982in}{2.595577in}}%
\pgfpathlineto{\pgfqpoint{1.488135in}{2.595577in}}%
\pgfpathlineto{\pgfqpoint{1.505289in}{2.595577in}}%
\pgfpathlineto{\pgfqpoint{1.522443in}{2.595577in}}%
\pgfpathlineto{\pgfqpoint{1.539596in}{2.595577in}}%
\pgfpathlineto{\pgfqpoint{1.556750in}{2.595577in}}%
\pgfpathlineto{\pgfqpoint{1.573904in}{2.595577in}}%
\pgfpathlineto{\pgfqpoint{1.591057in}{2.595577in}}%
\pgfpathlineto{\pgfqpoint{1.608211in}{2.595577in}}%
\pgfpathlineto{\pgfqpoint{1.625365in}{2.595577in}}%
\pgfpathlineto{\pgfqpoint{1.642518in}{2.595577in}}%
\pgfpathlineto{\pgfqpoint{1.659672in}{2.595577in}}%
\pgfpathlineto{\pgfqpoint{1.676826in}{2.595577in}}%
\pgfpathlineto{\pgfqpoint{1.693979in}{2.595577in}}%
\pgfpathlineto{\pgfqpoint{1.711133in}{2.595577in}}%
\pgfpathlineto{\pgfqpoint{1.728287in}{2.595577in}}%
\pgfpathlineto{\pgfqpoint{1.745440in}{2.595577in}}%
\pgfpathlineto{\pgfqpoint{1.762594in}{2.595577in}}%
\pgfpathlineto{\pgfqpoint{1.779748in}{2.595577in}}%
\pgfpathlineto{\pgfqpoint{1.796902in}{2.595577in}}%
\pgfpathlineto{\pgfqpoint{1.814055in}{2.595577in}}%
\pgfpathlineto{\pgfqpoint{1.831209in}{2.595577in}}%
\pgfpathlineto{\pgfqpoint{1.848363in}{2.595577in}}%
\pgfpathlineto{\pgfqpoint{1.865516in}{2.595577in}}%
\pgfpathlineto{\pgfqpoint{1.882670in}{2.595577in}}%
\pgfpathlineto{\pgfqpoint{1.899824in}{2.595577in}}%
\pgfpathlineto{\pgfqpoint{1.916977in}{2.595577in}}%
\pgfpathlineto{\pgfqpoint{1.934131in}{2.595577in}}%
\pgfpathlineto{\pgfqpoint{1.951285in}{2.595577in}}%
\pgfpathlineto{\pgfqpoint{1.968438in}{2.595577in}}%
\pgfpathlineto{\pgfqpoint{1.985592in}{2.595577in}}%
\pgfpathlineto{\pgfqpoint{2.002746in}{2.595577in}}%
\pgfpathlineto{\pgfqpoint{2.019899in}{2.595577in}}%
\pgfpathlineto{\pgfqpoint{2.037053in}{2.595577in}}%
\pgfpathlineto{\pgfqpoint{2.054207in}{2.595577in}}%
\pgfpathlineto{\pgfqpoint{2.071360in}{2.595577in}}%
\pgfpathlineto{\pgfqpoint{2.088514in}{2.595577in}}%
\pgfpathlineto{\pgfqpoint{2.105668in}{2.595577in}}%
\pgfpathlineto{\pgfqpoint{2.122821in}{2.595577in}}%
\pgfpathlineto{\pgfqpoint{2.139975in}{2.595577in}}%
\pgfpathlineto{\pgfqpoint{2.157129in}{2.595577in}}%
\pgfpathlineto{\pgfqpoint{2.174282in}{2.595577in}}%
\pgfpathlineto{\pgfqpoint{2.191436in}{2.595577in}}%
\pgfpathlineto{\pgfqpoint{2.208590in}{2.595577in}}%
\pgfpathlineto{\pgfqpoint{2.225744in}{2.595577in}}%
\pgfpathlineto{\pgfqpoint{2.242897in}{2.595577in}}%
\pgfpathlineto{\pgfqpoint{2.260051in}{2.595577in}}%
\pgfpathlineto{\pgfqpoint{2.277205in}{2.595577in}}%
\pgfpathlineto{\pgfqpoint{2.294358in}{2.595577in}}%
\pgfpathlineto{\pgfqpoint{2.311512in}{2.595577in}}%
\pgfpathlineto{\pgfqpoint{2.328666in}{2.595577in}}%
\pgfpathlineto{\pgfqpoint{2.345819in}{2.595577in}}%
\pgfpathlineto{\pgfqpoint{2.362973in}{2.595577in}}%
\pgfpathlineto{\pgfqpoint{2.380127in}{2.595577in}}%
\pgfpathlineto{\pgfqpoint{2.397280in}{2.595577in}}%
\pgfpathlineto{\pgfqpoint{2.414434in}{2.595577in}}%
\pgfpathlineto{\pgfqpoint{2.431588in}{2.595577in}}%
\pgfpathlineto{\pgfqpoint{2.448741in}{2.595577in}}%
\pgfpathlineto{\pgfqpoint{2.465895in}{2.595577in}}%
\pgfpathlineto{\pgfqpoint{2.483049in}{2.595577in}}%
\pgfpathlineto{\pgfqpoint{2.500202in}{2.595577in}}%
\pgfpathlineto{\pgfqpoint{2.517356in}{2.595577in}}%
\pgfpathlineto{\pgfqpoint{2.534510in}{2.595577in}}%
\pgfpathlineto{\pgfqpoint{2.551663in}{2.595577in}}%
\pgfpathlineto{\pgfqpoint{2.568817in}{2.595577in}}%
\pgfpathlineto{\pgfqpoint{2.585971in}{2.595577in}}%
\pgfpathlineto{\pgfqpoint{2.603124in}{2.595577in}}%
\pgfpathlineto{\pgfqpoint{2.620278in}{2.595577in}}%
\pgfpathlineto{\pgfqpoint{2.637432in}{2.595577in}}%
\pgfusepath{stroke}%
\end{pgfscope}%
\begin{pgfscope}%
\pgfpathrectangle{\pgfqpoint{0.368000in}{0.315889in}}{\pgfqpoint{2.377500in}{2.388245in}}%
\pgfusepath{clip}%
\pgfsetrectcap%
\pgfsetroundjoin%
\pgfsetlinewidth{1.505625pt}%
\definecolor{currentstroke}{rgb}{0.172549,0.627451,0.172549}%
\pgfsetstrokecolor{currentstroke}%
\pgfsetdash{}{0pt}%
\pgfpathmoveto{\pgfqpoint{0.476068in}{2.231994in}}%
\pgfpathlineto{\pgfqpoint{0.493222in}{2.231994in}}%
\pgfpathlineto{\pgfqpoint{0.510376in}{2.231994in}}%
\pgfpathlineto{\pgfqpoint{0.527529in}{2.231994in}}%
\pgfpathlineto{\pgfqpoint{0.544683in}{2.231994in}}%
\pgfpathlineto{\pgfqpoint{0.561837in}{2.231994in}}%
\pgfpathlineto{\pgfqpoint{0.578990in}{2.231994in}}%
\pgfpathlineto{\pgfqpoint{0.596144in}{2.231994in}}%
\pgfpathlineto{\pgfqpoint{0.613298in}{2.231994in}}%
\pgfpathlineto{\pgfqpoint{0.630451in}{2.231994in}}%
\pgfpathlineto{\pgfqpoint{0.647605in}{2.231994in}}%
\pgfpathlineto{\pgfqpoint{0.664759in}{2.231994in}}%
\pgfpathlineto{\pgfqpoint{0.681912in}{2.231994in}}%
\pgfpathlineto{\pgfqpoint{0.699066in}{2.231994in}}%
\pgfpathlineto{\pgfqpoint{0.716220in}{2.231994in}}%
\pgfpathlineto{\pgfqpoint{0.733373in}{2.231994in}}%
\pgfpathlineto{\pgfqpoint{0.750527in}{2.231994in}}%
\pgfpathlineto{\pgfqpoint{0.767681in}{2.231994in}}%
\pgfpathlineto{\pgfqpoint{0.784834in}{2.231994in}}%
\pgfpathlineto{\pgfqpoint{0.801988in}{2.231994in}}%
\pgfpathlineto{\pgfqpoint{0.819142in}{2.231994in}}%
\pgfpathlineto{\pgfqpoint{0.836295in}{2.231994in}}%
\pgfpathlineto{\pgfqpoint{0.853449in}{2.231994in}}%
\pgfpathlineto{\pgfqpoint{0.870603in}{2.231994in}}%
\pgfpathlineto{\pgfqpoint{0.887756in}{2.231994in}}%
\pgfpathlineto{\pgfqpoint{0.904910in}{2.231994in}}%
\pgfpathlineto{\pgfqpoint{0.922064in}{2.231994in}}%
\pgfpathlineto{\pgfqpoint{0.939218in}{2.231994in}}%
\pgfpathlineto{\pgfqpoint{0.956371in}{2.231994in}}%
\pgfpathlineto{\pgfqpoint{0.973525in}{2.231994in}}%
\pgfpathlineto{\pgfqpoint{0.990679in}{2.231994in}}%
\pgfpathlineto{\pgfqpoint{1.007832in}{2.231994in}}%
\pgfpathlineto{\pgfqpoint{1.024986in}{2.231994in}}%
\pgfpathlineto{\pgfqpoint{1.042140in}{2.231994in}}%
\pgfpathlineto{\pgfqpoint{1.059293in}{2.231994in}}%
\pgfpathlineto{\pgfqpoint{1.076447in}{2.231994in}}%
\pgfpathlineto{\pgfqpoint{1.093601in}{2.231994in}}%
\pgfpathlineto{\pgfqpoint{1.110754in}{2.231994in}}%
\pgfpathlineto{\pgfqpoint{1.127908in}{2.231994in}}%
\pgfpathlineto{\pgfqpoint{1.145062in}{2.231994in}}%
\pgfpathlineto{\pgfqpoint{1.162215in}{2.231994in}}%
\pgfpathlineto{\pgfqpoint{1.179369in}{2.231994in}}%
\pgfpathlineto{\pgfqpoint{1.196523in}{2.231994in}}%
\pgfpathlineto{\pgfqpoint{1.213676in}{2.231994in}}%
\pgfpathlineto{\pgfqpoint{1.230830in}{2.231994in}}%
\pgfpathlineto{\pgfqpoint{1.247984in}{2.231994in}}%
\pgfpathlineto{\pgfqpoint{1.265137in}{2.231994in}}%
\pgfpathlineto{\pgfqpoint{1.282291in}{2.231994in}}%
\pgfpathlineto{\pgfqpoint{1.299445in}{2.231994in}}%
\pgfpathlineto{\pgfqpoint{1.316598in}{2.231994in}}%
\pgfpathlineto{\pgfqpoint{1.333752in}{2.231994in}}%
\pgfpathlineto{\pgfqpoint{1.350906in}{2.231994in}}%
\pgfpathlineto{\pgfqpoint{1.368060in}{2.231994in}}%
\pgfpathlineto{\pgfqpoint{1.385213in}{2.231994in}}%
\pgfpathlineto{\pgfqpoint{1.402367in}{2.231994in}}%
\pgfpathlineto{\pgfqpoint{1.419521in}{2.231994in}}%
\pgfpathlineto{\pgfqpoint{1.436674in}{2.231994in}}%
\pgfpathlineto{\pgfqpoint{1.453828in}{2.231994in}}%
\pgfpathlineto{\pgfqpoint{1.470982in}{2.231994in}}%
\pgfpathlineto{\pgfqpoint{1.488135in}{2.231994in}}%
\pgfpathlineto{\pgfqpoint{1.505289in}{2.231994in}}%
\pgfpathlineto{\pgfqpoint{1.522443in}{2.231994in}}%
\pgfpathlineto{\pgfqpoint{1.539596in}{2.231994in}}%
\pgfpathlineto{\pgfqpoint{1.556750in}{2.231994in}}%
\pgfpathlineto{\pgfqpoint{1.573904in}{2.231994in}}%
\pgfpathlineto{\pgfqpoint{1.591057in}{2.231994in}}%
\pgfpathlineto{\pgfqpoint{1.608211in}{2.231994in}}%
\pgfpathlineto{\pgfqpoint{1.625365in}{2.231994in}}%
\pgfpathlineto{\pgfqpoint{1.642518in}{2.231994in}}%
\pgfpathlineto{\pgfqpoint{1.659672in}{2.231994in}}%
\pgfpathlineto{\pgfqpoint{1.676826in}{2.231994in}}%
\pgfpathlineto{\pgfqpoint{1.693979in}{2.231994in}}%
\pgfpathlineto{\pgfqpoint{1.711133in}{2.231994in}}%
\pgfpathlineto{\pgfqpoint{1.728287in}{2.231994in}}%
\pgfpathlineto{\pgfqpoint{1.745440in}{2.231994in}}%
\pgfpathlineto{\pgfqpoint{1.762594in}{2.231994in}}%
\pgfpathlineto{\pgfqpoint{1.779748in}{2.231994in}}%
\pgfpathlineto{\pgfqpoint{1.796902in}{2.231994in}}%
\pgfpathlineto{\pgfqpoint{1.814055in}{2.231994in}}%
\pgfpathlineto{\pgfqpoint{1.831209in}{2.231994in}}%
\pgfpathlineto{\pgfqpoint{1.848363in}{2.231994in}}%
\pgfpathlineto{\pgfqpoint{1.865516in}{2.231994in}}%
\pgfpathlineto{\pgfqpoint{1.882670in}{2.231994in}}%
\pgfpathlineto{\pgfqpoint{1.899824in}{2.231994in}}%
\pgfpathlineto{\pgfqpoint{1.916977in}{2.231994in}}%
\pgfpathlineto{\pgfqpoint{1.934131in}{2.231994in}}%
\pgfpathlineto{\pgfqpoint{1.951285in}{2.231994in}}%
\pgfpathlineto{\pgfqpoint{1.968438in}{2.231994in}}%
\pgfpathlineto{\pgfqpoint{1.985592in}{2.231994in}}%
\pgfpathlineto{\pgfqpoint{2.002746in}{2.231994in}}%
\pgfpathlineto{\pgfqpoint{2.019899in}{2.231994in}}%
\pgfpathlineto{\pgfqpoint{2.037053in}{2.231994in}}%
\pgfpathlineto{\pgfqpoint{2.054207in}{2.231994in}}%
\pgfpathlineto{\pgfqpoint{2.071360in}{2.231994in}}%
\pgfpathlineto{\pgfqpoint{2.088514in}{2.231994in}}%
\pgfpathlineto{\pgfqpoint{2.105668in}{2.231994in}}%
\pgfpathlineto{\pgfqpoint{2.122821in}{2.231994in}}%
\pgfpathlineto{\pgfqpoint{2.139975in}{2.231994in}}%
\pgfpathlineto{\pgfqpoint{2.157129in}{2.231994in}}%
\pgfpathlineto{\pgfqpoint{2.174282in}{2.231994in}}%
\pgfpathlineto{\pgfqpoint{2.191436in}{2.231994in}}%
\pgfpathlineto{\pgfqpoint{2.208590in}{2.231994in}}%
\pgfpathlineto{\pgfqpoint{2.225744in}{2.231994in}}%
\pgfpathlineto{\pgfqpoint{2.242897in}{2.231994in}}%
\pgfpathlineto{\pgfqpoint{2.260051in}{2.231994in}}%
\pgfpathlineto{\pgfqpoint{2.277205in}{2.231994in}}%
\pgfpathlineto{\pgfqpoint{2.294358in}{2.231994in}}%
\pgfpathlineto{\pgfqpoint{2.311512in}{2.231994in}}%
\pgfpathlineto{\pgfqpoint{2.328666in}{2.231994in}}%
\pgfpathlineto{\pgfqpoint{2.345819in}{2.231994in}}%
\pgfpathlineto{\pgfqpoint{2.362973in}{2.231994in}}%
\pgfpathlineto{\pgfqpoint{2.380127in}{2.231994in}}%
\pgfpathlineto{\pgfqpoint{2.397280in}{2.231994in}}%
\pgfpathlineto{\pgfqpoint{2.414434in}{2.231994in}}%
\pgfpathlineto{\pgfqpoint{2.431588in}{2.231994in}}%
\pgfpathlineto{\pgfqpoint{2.448741in}{2.231994in}}%
\pgfpathlineto{\pgfqpoint{2.465895in}{2.231994in}}%
\pgfpathlineto{\pgfqpoint{2.483049in}{2.231994in}}%
\pgfpathlineto{\pgfqpoint{2.500202in}{2.231994in}}%
\pgfpathlineto{\pgfqpoint{2.517356in}{2.231994in}}%
\pgfpathlineto{\pgfqpoint{2.534510in}{2.231994in}}%
\pgfpathlineto{\pgfqpoint{2.551663in}{2.231994in}}%
\pgfpathlineto{\pgfqpoint{2.568817in}{2.231994in}}%
\pgfpathlineto{\pgfqpoint{2.585971in}{2.231994in}}%
\pgfpathlineto{\pgfqpoint{2.603124in}{2.231994in}}%
\pgfpathlineto{\pgfqpoint{2.620278in}{2.231994in}}%
\pgfpathlineto{\pgfqpoint{2.637432in}{2.231994in}}%
\pgfusepath{stroke}%
\end{pgfscope}%
\begin{pgfscope}%
\pgfpathrectangle{\pgfqpoint{0.368000in}{0.315889in}}{\pgfqpoint{2.377500in}{2.388245in}}%
\pgfusepath{clip}%
\pgfsetrectcap%
\pgfsetroundjoin%
\pgfsetlinewidth{1.505625pt}%
\definecolor{currentstroke}{rgb}{0.839216,0.152941,0.156863}%
\pgfsetstrokecolor{currentstroke}%
\pgfsetdash{}{0pt}%
\pgfpathmoveto{\pgfqpoint{0.476068in}{2.580275in}}%
\pgfpathlineto{\pgfqpoint{0.493222in}{2.563165in}}%
\pgfpathlineto{\pgfqpoint{0.510376in}{2.546055in}}%
\pgfpathlineto{\pgfqpoint{0.527529in}{2.528946in}}%
\pgfpathlineto{\pgfqpoint{0.544683in}{2.511836in}}%
\pgfpathlineto{\pgfqpoint{0.561837in}{2.494726in}}%
\pgfpathlineto{\pgfqpoint{0.578990in}{2.477616in}}%
\pgfpathlineto{\pgfqpoint{0.596144in}{2.460507in}}%
\pgfpathlineto{\pgfqpoint{0.613298in}{2.443397in}}%
\pgfpathlineto{\pgfqpoint{0.630451in}{2.426287in}}%
\pgfpathlineto{\pgfqpoint{0.647605in}{2.409177in}}%
\pgfpathlineto{\pgfqpoint{0.664759in}{2.392068in}}%
\pgfpathlineto{\pgfqpoint{0.681912in}{2.374958in}}%
\pgfpathlineto{\pgfqpoint{0.699066in}{2.357848in}}%
\pgfpathlineto{\pgfqpoint{0.716220in}{2.340738in}}%
\pgfpathlineto{\pgfqpoint{0.733373in}{2.323628in}}%
\pgfpathlineto{\pgfqpoint{0.750527in}{2.306519in}}%
\pgfpathlineto{\pgfqpoint{0.767681in}{2.289409in}}%
\pgfpathlineto{\pgfqpoint{0.784834in}{2.272299in}}%
\pgfpathlineto{\pgfqpoint{0.801988in}{2.255189in}}%
\pgfpathlineto{\pgfqpoint{0.819142in}{2.238080in}}%
\pgfpathlineto{\pgfqpoint{0.836295in}{2.220970in}}%
\pgfpathlineto{\pgfqpoint{0.853449in}{2.203860in}}%
\pgfpathlineto{\pgfqpoint{0.870603in}{2.186750in}}%
\pgfpathlineto{\pgfqpoint{0.887756in}{2.169641in}}%
\pgfpathlineto{\pgfqpoint{0.904910in}{2.152531in}}%
\pgfpathlineto{\pgfqpoint{0.922064in}{2.135421in}}%
\pgfpathlineto{\pgfqpoint{0.939218in}{2.118311in}}%
\pgfpathlineto{\pgfqpoint{0.956371in}{2.101202in}}%
\pgfpathlineto{\pgfqpoint{0.973525in}{2.084092in}}%
\pgfpathlineto{\pgfqpoint{0.990679in}{2.066982in}}%
\pgfpathlineto{\pgfqpoint{1.007832in}{2.049872in}}%
\pgfpathlineto{\pgfqpoint{1.024986in}{2.032763in}}%
\pgfpathlineto{\pgfqpoint{1.042140in}{2.015653in}}%
\pgfpathlineto{\pgfqpoint{1.059293in}{1.998543in}}%
\pgfpathlineto{\pgfqpoint{1.076447in}{1.981433in}}%
\pgfpathlineto{\pgfqpoint{1.093601in}{1.964324in}}%
\pgfpathlineto{\pgfqpoint{1.110754in}{1.947214in}}%
\pgfpathlineto{\pgfqpoint{1.127908in}{1.930104in}}%
\pgfpathlineto{\pgfqpoint{1.145062in}{1.912994in}}%
\pgfpathlineto{\pgfqpoint{1.162215in}{1.895885in}}%
\pgfpathlineto{\pgfqpoint{1.179369in}{1.878775in}}%
\pgfpathlineto{\pgfqpoint{1.196523in}{1.861665in}}%
\pgfpathlineto{\pgfqpoint{1.213676in}{1.844555in}}%
\pgfpathlineto{\pgfqpoint{1.230830in}{1.827445in}}%
\pgfpathlineto{\pgfqpoint{1.247984in}{1.810336in}}%
\pgfpathlineto{\pgfqpoint{1.265137in}{1.793226in}}%
\pgfpathlineto{\pgfqpoint{1.282291in}{1.776116in}}%
\pgfpathlineto{\pgfqpoint{1.299445in}{1.759006in}}%
\pgfpathlineto{\pgfqpoint{1.316598in}{1.741897in}}%
\pgfpathlineto{\pgfqpoint{1.333752in}{1.724787in}}%
\pgfpathlineto{\pgfqpoint{1.350906in}{1.707677in}}%
\pgfpathlineto{\pgfqpoint{1.368060in}{1.690567in}}%
\pgfpathlineto{\pgfqpoint{1.385213in}{1.673458in}}%
\pgfpathlineto{\pgfqpoint{1.402367in}{1.656348in}}%
\pgfpathlineto{\pgfqpoint{1.419521in}{1.639238in}}%
\pgfpathlineto{\pgfqpoint{1.436674in}{1.622128in}}%
\pgfpathlineto{\pgfqpoint{1.453828in}{1.605019in}}%
\pgfpathlineto{\pgfqpoint{1.470982in}{1.587909in}}%
\pgfpathlineto{\pgfqpoint{1.488135in}{1.570799in}}%
\pgfpathlineto{\pgfqpoint{1.505289in}{1.553689in}}%
\pgfpathlineto{\pgfqpoint{1.522443in}{1.536580in}}%
\pgfpathlineto{\pgfqpoint{1.539596in}{1.519470in}}%
\pgfpathlineto{\pgfqpoint{1.556750in}{1.502360in}}%
\pgfpathlineto{\pgfqpoint{1.573904in}{1.485250in}}%
\pgfpathlineto{\pgfqpoint{1.591057in}{1.468141in}}%
\pgfpathlineto{\pgfqpoint{1.608211in}{1.451031in}}%
\pgfpathlineto{\pgfqpoint{1.625365in}{1.433921in}}%
\pgfpathlineto{\pgfqpoint{1.642518in}{1.416811in}}%
\pgfpathlineto{\pgfqpoint{1.659672in}{1.399702in}}%
\pgfpathlineto{\pgfqpoint{1.676826in}{1.382592in}}%
\pgfpathlineto{\pgfqpoint{1.693979in}{1.365482in}}%
\pgfpathlineto{\pgfqpoint{1.711133in}{1.348372in}}%
\pgfpathlineto{\pgfqpoint{1.728287in}{1.331263in}}%
\pgfpathlineto{\pgfqpoint{1.745440in}{1.314153in}}%
\pgfpathlineto{\pgfqpoint{1.762594in}{1.297043in}}%
\pgfpathlineto{\pgfqpoint{1.779748in}{1.279933in}}%
\pgfpathlineto{\pgfqpoint{1.796902in}{1.262823in}}%
\pgfpathlineto{\pgfqpoint{1.814055in}{1.245714in}}%
\pgfpathlineto{\pgfqpoint{1.831209in}{1.228604in}}%
\pgfpathlineto{\pgfqpoint{1.848363in}{1.211494in}}%
\pgfpathlineto{\pgfqpoint{1.865516in}{1.194384in}}%
\pgfpathlineto{\pgfqpoint{1.882670in}{1.177275in}}%
\pgfpathlineto{\pgfqpoint{1.899824in}{1.160165in}}%
\pgfpathlineto{\pgfqpoint{1.916977in}{1.143055in}}%
\pgfpathlineto{\pgfqpoint{1.934131in}{1.125945in}}%
\pgfpathlineto{\pgfqpoint{1.951285in}{1.108836in}}%
\pgfpathlineto{\pgfqpoint{1.968438in}{1.091726in}}%
\pgfpathlineto{\pgfqpoint{1.985592in}{1.074616in}}%
\pgfpathlineto{\pgfqpoint{2.002746in}{1.057506in}}%
\pgfpathlineto{\pgfqpoint{2.019899in}{1.040397in}}%
\pgfpathlineto{\pgfqpoint{2.037053in}{1.023287in}}%
\pgfpathlineto{\pgfqpoint{2.054207in}{1.006177in}}%
\pgfpathlineto{\pgfqpoint{2.071360in}{0.989067in}}%
\pgfpathlineto{\pgfqpoint{2.088514in}{0.971958in}}%
\pgfpathlineto{\pgfqpoint{2.105668in}{0.954848in}}%
\pgfpathlineto{\pgfqpoint{2.122821in}{0.937738in}}%
\pgfpathlineto{\pgfqpoint{2.139975in}{0.920628in}}%
\pgfpathlineto{\pgfqpoint{2.157129in}{0.903519in}}%
\pgfpathlineto{\pgfqpoint{2.174282in}{0.886409in}}%
\pgfpathlineto{\pgfqpoint{2.191436in}{0.869299in}}%
\pgfpathlineto{\pgfqpoint{2.208590in}{0.852189in}}%
\pgfpathlineto{\pgfqpoint{2.225744in}{0.835080in}}%
\pgfpathlineto{\pgfqpoint{2.242897in}{0.817970in}}%
\pgfpathlineto{\pgfqpoint{2.260051in}{0.800860in}}%
\pgfpathlineto{\pgfqpoint{2.277205in}{0.783750in}}%
\pgfpathlineto{\pgfqpoint{2.294358in}{0.766640in}}%
\pgfpathlineto{\pgfqpoint{2.311512in}{0.749531in}}%
\pgfpathlineto{\pgfqpoint{2.328666in}{0.732421in}}%
\pgfpathlineto{\pgfqpoint{2.345819in}{0.715311in}}%
\pgfpathlineto{\pgfqpoint{2.362973in}{0.698201in}}%
\pgfpathlineto{\pgfqpoint{2.380127in}{0.681092in}}%
\pgfpathlineto{\pgfqpoint{2.397280in}{0.663982in}}%
\pgfpathlineto{\pgfqpoint{2.414434in}{0.646872in}}%
\pgfpathlineto{\pgfqpoint{2.431588in}{0.629762in}}%
\pgfpathlineto{\pgfqpoint{2.448741in}{0.612653in}}%
\pgfpathlineto{\pgfqpoint{2.465895in}{0.595543in}}%
\pgfpathlineto{\pgfqpoint{2.483049in}{0.578433in}}%
\pgfpathlineto{\pgfqpoint{2.500202in}{0.561323in}}%
\pgfpathlineto{\pgfqpoint{2.517356in}{0.544214in}}%
\pgfpathlineto{\pgfqpoint{2.534510in}{0.527104in}}%
\pgfpathlineto{\pgfqpoint{2.551663in}{0.509994in}}%
\pgfpathlineto{\pgfqpoint{2.568817in}{0.492884in}}%
\pgfpathlineto{\pgfqpoint{2.585971in}{0.475775in}}%
\pgfpathlineto{\pgfqpoint{2.603124in}{0.458665in}}%
\pgfpathlineto{\pgfqpoint{2.620278in}{0.441555in}}%
\pgfpathlineto{\pgfqpoint{2.637432in}{0.424445in}}%
\pgfusepath{stroke}%
\end{pgfscope}%
\begin{pgfscope}%
\pgfsetrectcap%
\pgfsetmiterjoin%
\pgfsetlinewidth{0.803000pt}%
\definecolor{currentstroke}{rgb}{0.000000,0.000000,0.000000}%
\pgfsetstrokecolor{currentstroke}%
\pgfsetdash{}{0pt}%
\pgfpathmoveto{\pgfqpoint{0.368000in}{0.315889in}}%
\pgfpathlineto{\pgfqpoint{0.368000in}{2.704133in}}%
\pgfusepath{stroke}%
\end{pgfscope}%
\begin{pgfscope}%
\pgfsetrectcap%
\pgfsetmiterjoin%
\pgfsetlinewidth{0.803000pt}%
\definecolor{currentstroke}{rgb}{0.000000,0.000000,0.000000}%
\pgfsetstrokecolor{currentstroke}%
\pgfsetdash{}{0pt}%
\pgfpathmoveto{\pgfqpoint{2.745500in}{0.315889in}}%
\pgfpathlineto{\pgfqpoint{2.745500in}{2.704133in}}%
\pgfusepath{stroke}%
\end{pgfscope}%
\begin{pgfscope}%
\pgfsetrectcap%
\pgfsetmiterjoin%
\pgfsetlinewidth{0.803000pt}%
\definecolor{currentstroke}{rgb}{0.000000,0.000000,0.000000}%
\pgfsetstrokecolor{currentstroke}%
\pgfsetdash{}{0pt}%
\pgfpathmoveto{\pgfqpoint{0.368000in}{0.315889in}}%
\pgfpathlineto{\pgfqpoint{2.745500in}{0.315889in}}%
\pgfusepath{stroke}%
\end{pgfscope}%
\begin{pgfscope}%
\pgfsetrectcap%
\pgfsetmiterjoin%
\pgfsetlinewidth{0.803000pt}%
\definecolor{currentstroke}{rgb}{0.000000,0.000000,0.000000}%
\pgfsetstrokecolor{currentstroke}%
\pgfsetdash{}{0pt}%
\pgfpathmoveto{\pgfqpoint{0.368000in}{2.704133in}}%
\pgfpathlineto{\pgfqpoint{2.745500in}{2.704133in}}%
\pgfusepath{stroke}%
\end{pgfscope}%
\begin{pgfscope}%
\definecolor{textcolor}{rgb}{0.000000,0.000000,0.000000}%
\pgfsetstrokecolor{textcolor}%
\pgfsetfillcolor{textcolor}%
\pgftext[x=1.556750in,y=2.787467in,,base]{\color{textcolor}\rmfamily\fontsize{9.600000}{11.520000}\selectfont db2}%
\end{pgfscope}%
\begin{pgfscope}%
\pgfsetbuttcap%
\pgfsetmiterjoin%
\definecolor{currentfill}{rgb}{1.000000,1.000000,1.000000}%
\pgfsetfillcolor{currentfill}%
\pgfsetfillopacity{0.800000}%
\pgfsetlinewidth{1.003750pt}%
\definecolor{currentstroke}{rgb}{0.800000,0.800000,0.800000}%
\pgfsetstrokecolor{currentstroke}%
\pgfsetstrokeopacity{0.800000}%
\pgfsetdash{}{0pt}%
\pgfpathmoveto{\pgfqpoint{0.445778in}{0.371444in}}%
\pgfpathlineto{\pgfqpoint{0.925658in}{0.371444in}}%
\pgfpathquadraticcurveto{\pgfqpoint{0.947880in}{0.371444in}}{\pgfqpoint{0.947880in}{0.393667in}}%
\pgfpathlineto{\pgfqpoint{0.947880in}{1.004222in}}%
\pgfpathquadraticcurveto{\pgfqpoint{0.947880in}{1.026444in}}{\pgfqpoint{0.925658in}{1.026444in}}%
\pgfpathlineto{\pgfqpoint{0.445778in}{1.026444in}}%
\pgfpathquadraticcurveto{\pgfqpoint{0.423556in}{1.026444in}}{\pgfqpoint{0.423556in}{1.004222in}}%
\pgfpathlineto{\pgfqpoint{0.423556in}{0.393667in}}%
\pgfpathquadraticcurveto{\pgfqpoint{0.423556in}{0.371444in}}{\pgfqpoint{0.445778in}{0.371444in}}%
\pgfpathclose%
\pgfusepath{stroke,fill}%
\end{pgfscope}%
\begin{pgfscope}%
\pgfsetrectcap%
\pgfsetroundjoin%
\pgfsetlinewidth{1.505625pt}%
\definecolor{currentstroke}{rgb}{0.121569,0.466667,0.705882}%
\pgfsetstrokecolor{currentstroke}%
\pgfsetdash{}{0pt}%
\pgfpathmoveto{\pgfqpoint{0.468000in}{0.942583in}}%
\pgfpathlineto{\pgfqpoint{0.690222in}{0.942583in}}%
\pgfusepath{stroke}%
\end{pgfscope}%
\begin{pgfscope}%
\definecolor{textcolor}{rgb}{0.000000,0.000000,0.000000}%
\pgfsetstrokecolor{textcolor}%
\pgfsetfillcolor{textcolor}%
\pgftext[x=0.779111in,y=0.903694in,left,base]{\color{textcolor}\rmfamily\fontsize{8.000000}{9.600000}\selectfont \(\displaystyle x^{0}\)}%
\end{pgfscope}%
\begin{pgfscope}%
\pgfsetrectcap%
\pgfsetroundjoin%
\pgfsetlinewidth{1.505625pt}%
\definecolor{currentstroke}{rgb}{1.000000,0.498039,0.054902}%
\pgfsetstrokecolor{currentstroke}%
\pgfsetdash{}{0pt}%
\pgfpathmoveto{\pgfqpoint{0.468000in}{0.787166in}}%
\pgfpathlineto{\pgfqpoint{0.690222in}{0.787166in}}%
\pgfusepath{stroke}%
\end{pgfscope}%
\begin{pgfscope}%
\definecolor{textcolor}{rgb}{0.000000,0.000000,0.000000}%
\pgfsetstrokecolor{textcolor}%
\pgfsetfillcolor{textcolor}%
\pgftext[x=0.779111in,y=0.748277in,left,base]{\color{textcolor}\rmfamily\fontsize{8.000000}{9.600000}\selectfont \(\displaystyle x^{1}\)}%
\end{pgfscope}%
\begin{pgfscope}%
\pgfsetrectcap%
\pgfsetroundjoin%
\pgfsetlinewidth{1.505625pt}%
\definecolor{currentstroke}{rgb}{0.172549,0.627451,0.172549}%
\pgfsetstrokecolor{currentstroke}%
\pgfsetdash{}{0pt}%
\pgfpathmoveto{\pgfqpoint{0.468000in}{0.631750in}}%
\pgfpathlineto{\pgfqpoint{0.690222in}{0.631750in}}%
\pgfusepath{stroke}%
\end{pgfscope}%
\begin{pgfscope}%
\definecolor{textcolor}{rgb}{0.000000,0.000000,0.000000}%
\pgfsetstrokecolor{textcolor}%
\pgfsetfillcolor{textcolor}%
\pgftext[x=0.779111in,y=0.592861in,left,base]{\color{textcolor}\rmfamily\fontsize{8.000000}{9.600000}\selectfont \(\displaystyle x^{2}\)}%
\end{pgfscope}%
\begin{pgfscope}%
\pgfsetrectcap%
\pgfsetroundjoin%
\pgfsetlinewidth{1.505625pt}%
\definecolor{currentstroke}{rgb}{0.839216,0.152941,0.156863}%
\pgfsetstrokecolor{currentstroke}%
\pgfsetdash{}{0pt}%
\pgfpathmoveto{\pgfqpoint{0.468000in}{0.476333in}}%
\pgfpathlineto{\pgfqpoint{0.690222in}{0.476333in}}%
\pgfusepath{stroke}%
\end{pgfscope}%
\begin{pgfscope}%
\definecolor{textcolor}{rgb}{0.000000,0.000000,0.000000}%
\pgfsetstrokecolor{textcolor}%
\pgfsetfillcolor{textcolor}%
\pgftext[x=0.779111in,y=0.437444in,left,base]{\color{textcolor}\rmfamily\fontsize{8.000000}{9.600000}\selectfont \(\displaystyle x^{3}\)}%
\end{pgfscope}%
\begin{pgfscope}%
\pgfsetbuttcap%
\pgfsetmiterjoin%
\definecolor{currentfill}{rgb}{1.000000,1.000000,1.000000}%
\pgfsetfillcolor{currentfill}%
\pgfsetlinewidth{0.000000pt}%
\definecolor{currentstroke}{rgb}{0.000000,0.000000,0.000000}%
\pgfsetstrokecolor{currentstroke}%
\pgfsetstrokeopacity{0.000000}%
\pgfsetdash{}{0pt}%
\pgfpathmoveto{\pgfqpoint{2.962722in}{0.315889in}}%
\pgfpathlineto{\pgfqpoint{5.340222in}{0.315889in}}%
\pgfpathlineto{\pgfqpoint{5.340222in}{2.704133in}}%
\pgfpathlineto{\pgfqpoint{2.962722in}{2.704133in}}%
\pgfpathclose%
\pgfusepath{fill}%
\end{pgfscope}%
\begin{pgfscope}%
\pgfsetbuttcap%
\pgfsetroundjoin%
\definecolor{currentfill}{rgb}{0.000000,0.000000,0.000000}%
\pgfsetfillcolor{currentfill}%
\pgfsetlinewidth{0.803000pt}%
\definecolor{currentstroke}{rgb}{0.000000,0.000000,0.000000}%
\pgfsetstrokecolor{currentstroke}%
\pgfsetdash{}{0pt}%
\pgfsys@defobject{currentmarker}{\pgfqpoint{0.000000in}{-0.048611in}}{\pgfqpoint{0.000000in}{0.000000in}}{%
\pgfpathmoveto{\pgfqpoint{0.000000in}{0.000000in}}%
\pgfpathlineto{\pgfqpoint{0.000000in}{-0.048611in}}%
\pgfusepath{stroke,fill}%
}%
\begin{pgfscope}%
\pgfsys@transformshift{3.070790in}{0.315889in}%
\pgfsys@useobject{currentmarker}{}%
\end{pgfscope}%
\end{pgfscope}%
\begin{pgfscope}%
\definecolor{textcolor}{rgb}{0.000000,0.000000,0.000000}%
\pgfsetstrokecolor{textcolor}%
\pgfsetfillcolor{textcolor}%
\pgftext[x=3.070790in,y=0.218667in,,top]{\color{textcolor}\rmfamily\fontsize{8.000000}{9.600000}\selectfont 0}%
\end{pgfscope}%
\begin{pgfscope}%
\pgfsetbuttcap%
\pgfsetroundjoin%
\definecolor{currentfill}{rgb}{0.000000,0.000000,0.000000}%
\pgfsetfillcolor{currentfill}%
\pgfsetlinewidth{0.803000pt}%
\definecolor{currentstroke}{rgb}{0.000000,0.000000,0.000000}%
\pgfsetstrokecolor{currentstroke}%
\pgfsetdash{}{0pt}%
\pgfsys@defobject{currentmarker}{\pgfqpoint{0.000000in}{-0.048611in}}{\pgfqpoint{0.000000in}{0.000000in}}{%
\pgfpathmoveto{\pgfqpoint{0.000000in}{0.000000in}}%
\pgfpathlineto{\pgfqpoint{0.000000in}{-0.048611in}}%
\pgfusepath{stroke,fill}%
}%
\begin{pgfscope}%
\pgfsys@transformshift{3.413864in}{0.315889in}%
\pgfsys@useobject{currentmarker}{}%
\end{pgfscope}%
\end{pgfscope}%
\begin{pgfscope}%
\definecolor{textcolor}{rgb}{0.000000,0.000000,0.000000}%
\pgfsetstrokecolor{textcolor}%
\pgfsetfillcolor{textcolor}%
\pgftext[x=3.413864in,y=0.218667in,,top]{\color{textcolor}\rmfamily\fontsize{8.000000}{9.600000}\selectfont 20}%
\end{pgfscope}%
\begin{pgfscope}%
\pgfsetbuttcap%
\pgfsetroundjoin%
\definecolor{currentfill}{rgb}{0.000000,0.000000,0.000000}%
\pgfsetfillcolor{currentfill}%
\pgfsetlinewidth{0.803000pt}%
\definecolor{currentstroke}{rgb}{0.000000,0.000000,0.000000}%
\pgfsetstrokecolor{currentstroke}%
\pgfsetdash{}{0pt}%
\pgfsys@defobject{currentmarker}{\pgfqpoint{0.000000in}{-0.048611in}}{\pgfqpoint{0.000000in}{0.000000in}}{%
\pgfpathmoveto{\pgfqpoint{0.000000in}{0.000000in}}%
\pgfpathlineto{\pgfqpoint{0.000000in}{-0.048611in}}%
\pgfusepath{stroke,fill}%
}%
\begin{pgfscope}%
\pgfsys@transformshift{3.756938in}{0.315889in}%
\pgfsys@useobject{currentmarker}{}%
\end{pgfscope}%
\end{pgfscope}%
\begin{pgfscope}%
\definecolor{textcolor}{rgb}{0.000000,0.000000,0.000000}%
\pgfsetstrokecolor{textcolor}%
\pgfsetfillcolor{textcolor}%
\pgftext[x=3.756938in,y=0.218667in,,top]{\color{textcolor}\rmfamily\fontsize{8.000000}{9.600000}\selectfont 40}%
\end{pgfscope}%
\begin{pgfscope}%
\pgfsetbuttcap%
\pgfsetroundjoin%
\definecolor{currentfill}{rgb}{0.000000,0.000000,0.000000}%
\pgfsetfillcolor{currentfill}%
\pgfsetlinewidth{0.803000pt}%
\definecolor{currentstroke}{rgb}{0.000000,0.000000,0.000000}%
\pgfsetstrokecolor{currentstroke}%
\pgfsetdash{}{0pt}%
\pgfsys@defobject{currentmarker}{\pgfqpoint{0.000000in}{-0.048611in}}{\pgfqpoint{0.000000in}{0.000000in}}{%
\pgfpathmoveto{\pgfqpoint{0.000000in}{0.000000in}}%
\pgfpathlineto{\pgfqpoint{0.000000in}{-0.048611in}}%
\pgfusepath{stroke,fill}%
}%
\begin{pgfscope}%
\pgfsys@transformshift{4.100011in}{0.315889in}%
\pgfsys@useobject{currentmarker}{}%
\end{pgfscope}%
\end{pgfscope}%
\begin{pgfscope}%
\definecolor{textcolor}{rgb}{0.000000,0.000000,0.000000}%
\pgfsetstrokecolor{textcolor}%
\pgfsetfillcolor{textcolor}%
\pgftext[x=4.100011in,y=0.218667in,,top]{\color{textcolor}\rmfamily\fontsize{8.000000}{9.600000}\selectfont 60}%
\end{pgfscope}%
\begin{pgfscope}%
\pgfsetbuttcap%
\pgfsetroundjoin%
\definecolor{currentfill}{rgb}{0.000000,0.000000,0.000000}%
\pgfsetfillcolor{currentfill}%
\pgfsetlinewidth{0.803000pt}%
\definecolor{currentstroke}{rgb}{0.000000,0.000000,0.000000}%
\pgfsetstrokecolor{currentstroke}%
\pgfsetdash{}{0pt}%
\pgfsys@defobject{currentmarker}{\pgfqpoint{0.000000in}{-0.048611in}}{\pgfqpoint{0.000000in}{0.000000in}}{%
\pgfpathmoveto{\pgfqpoint{0.000000in}{0.000000in}}%
\pgfpathlineto{\pgfqpoint{0.000000in}{-0.048611in}}%
\pgfusepath{stroke,fill}%
}%
\begin{pgfscope}%
\pgfsys@transformshift{4.443085in}{0.315889in}%
\pgfsys@useobject{currentmarker}{}%
\end{pgfscope}%
\end{pgfscope}%
\begin{pgfscope}%
\definecolor{textcolor}{rgb}{0.000000,0.000000,0.000000}%
\pgfsetstrokecolor{textcolor}%
\pgfsetfillcolor{textcolor}%
\pgftext[x=4.443085in,y=0.218667in,,top]{\color{textcolor}\rmfamily\fontsize{8.000000}{9.600000}\selectfont 80}%
\end{pgfscope}%
\begin{pgfscope}%
\pgfsetbuttcap%
\pgfsetroundjoin%
\definecolor{currentfill}{rgb}{0.000000,0.000000,0.000000}%
\pgfsetfillcolor{currentfill}%
\pgfsetlinewidth{0.803000pt}%
\definecolor{currentstroke}{rgb}{0.000000,0.000000,0.000000}%
\pgfsetstrokecolor{currentstroke}%
\pgfsetdash{}{0pt}%
\pgfsys@defobject{currentmarker}{\pgfqpoint{0.000000in}{-0.048611in}}{\pgfqpoint{0.000000in}{0.000000in}}{%
\pgfpathmoveto{\pgfqpoint{0.000000in}{0.000000in}}%
\pgfpathlineto{\pgfqpoint{0.000000in}{-0.048611in}}%
\pgfusepath{stroke,fill}%
}%
\begin{pgfscope}%
\pgfsys@transformshift{4.786158in}{0.315889in}%
\pgfsys@useobject{currentmarker}{}%
\end{pgfscope}%
\end{pgfscope}%
\begin{pgfscope}%
\definecolor{textcolor}{rgb}{0.000000,0.000000,0.000000}%
\pgfsetstrokecolor{textcolor}%
\pgfsetfillcolor{textcolor}%
\pgftext[x=4.786158in,y=0.218667in,,top]{\color{textcolor}\rmfamily\fontsize{8.000000}{9.600000}\selectfont 100}%
\end{pgfscope}%
\begin{pgfscope}%
\pgfsetbuttcap%
\pgfsetroundjoin%
\definecolor{currentfill}{rgb}{0.000000,0.000000,0.000000}%
\pgfsetfillcolor{currentfill}%
\pgfsetlinewidth{0.803000pt}%
\definecolor{currentstroke}{rgb}{0.000000,0.000000,0.000000}%
\pgfsetstrokecolor{currentstroke}%
\pgfsetdash{}{0pt}%
\pgfsys@defobject{currentmarker}{\pgfqpoint{0.000000in}{-0.048611in}}{\pgfqpoint{0.000000in}{0.000000in}}{%
\pgfpathmoveto{\pgfqpoint{0.000000in}{0.000000in}}%
\pgfpathlineto{\pgfqpoint{0.000000in}{-0.048611in}}%
\pgfusepath{stroke,fill}%
}%
\begin{pgfscope}%
\pgfsys@transformshift{5.129232in}{0.315889in}%
\pgfsys@useobject{currentmarker}{}%
\end{pgfscope}%
\end{pgfscope}%
\begin{pgfscope}%
\definecolor{textcolor}{rgb}{0.000000,0.000000,0.000000}%
\pgfsetstrokecolor{textcolor}%
\pgfsetfillcolor{textcolor}%
\pgftext[x=5.129232in,y=0.218667in,,top]{\color{textcolor}\rmfamily\fontsize{8.000000}{9.600000}\selectfont 120}%
\end{pgfscope}%
\begin{pgfscope}%
\pgfsetbuttcap%
\pgfsetroundjoin%
\definecolor{currentfill}{rgb}{0.000000,0.000000,0.000000}%
\pgfsetfillcolor{currentfill}%
\pgfsetlinewidth{0.803000pt}%
\definecolor{currentstroke}{rgb}{0.000000,0.000000,0.000000}%
\pgfsetstrokecolor{currentstroke}%
\pgfsetdash{}{0pt}%
\pgfsys@defobject{currentmarker}{\pgfqpoint{0.000000in}{0.000000in}}{\pgfqpoint{0.048611in}{0.000000in}}{%
\pgfpathmoveto{\pgfqpoint{0.000000in}{0.000000in}}%
\pgfpathlineto{\pgfqpoint{0.048611in}{0.000000in}}%
\pgfusepath{stroke,fill}%
}%
\begin{pgfscope}%
\pgfsys@transformshift{5.340222in}{0.448931in}%
\pgfsys@useobject{currentmarker}{}%
\end{pgfscope}%
\end{pgfscope}%
\begin{pgfscope}%
\definecolor{textcolor}{rgb}{0.000000,0.000000,0.000000}%
\pgfsetstrokecolor{textcolor}%
\pgfsetfillcolor{textcolor}%
\pgftext[x=5.437444in,y=0.410375in,left,base]{\color{textcolor}\rmfamily\fontsize{8.000000}{9.600000}\selectfont −1.6}%
\end{pgfscope}%
\begin{pgfscope}%
\pgfsetbuttcap%
\pgfsetroundjoin%
\definecolor{currentfill}{rgb}{0.000000,0.000000,0.000000}%
\pgfsetfillcolor{currentfill}%
\pgfsetlinewidth{0.803000pt}%
\definecolor{currentstroke}{rgb}{0.000000,0.000000,0.000000}%
\pgfsetstrokecolor{currentstroke}%
\pgfsetdash{}{0pt}%
\pgfsys@defobject{currentmarker}{\pgfqpoint{0.000000in}{0.000000in}}{\pgfqpoint{0.048611in}{0.000000in}}{%
\pgfpathmoveto{\pgfqpoint{0.000000in}{0.000000in}}%
\pgfpathlineto{\pgfqpoint{0.048611in}{0.000000in}}%
\pgfusepath{stroke,fill}%
}%
\begin{pgfscope}%
\pgfsys@transformshift{5.340222in}{0.717262in}%
\pgfsys@useobject{currentmarker}{}%
\end{pgfscope}%
\end{pgfscope}%
\begin{pgfscope}%
\definecolor{textcolor}{rgb}{0.000000,0.000000,0.000000}%
\pgfsetstrokecolor{textcolor}%
\pgfsetfillcolor{textcolor}%
\pgftext[x=5.437444in,y=0.678706in,left,base]{\color{textcolor}\rmfamily\fontsize{8.000000}{9.600000}\selectfont −1.4}%
\end{pgfscope}%
\begin{pgfscope}%
\pgfsetbuttcap%
\pgfsetroundjoin%
\definecolor{currentfill}{rgb}{0.000000,0.000000,0.000000}%
\pgfsetfillcolor{currentfill}%
\pgfsetlinewidth{0.803000pt}%
\definecolor{currentstroke}{rgb}{0.000000,0.000000,0.000000}%
\pgfsetstrokecolor{currentstroke}%
\pgfsetdash{}{0pt}%
\pgfsys@defobject{currentmarker}{\pgfqpoint{0.000000in}{0.000000in}}{\pgfqpoint{0.048611in}{0.000000in}}{%
\pgfpathmoveto{\pgfqpoint{0.000000in}{0.000000in}}%
\pgfpathlineto{\pgfqpoint{0.048611in}{0.000000in}}%
\pgfusepath{stroke,fill}%
}%
\begin{pgfscope}%
\pgfsys@transformshift{5.340222in}{0.985592in}%
\pgfsys@useobject{currentmarker}{}%
\end{pgfscope}%
\end{pgfscope}%
\begin{pgfscope}%
\definecolor{textcolor}{rgb}{0.000000,0.000000,0.000000}%
\pgfsetstrokecolor{textcolor}%
\pgfsetfillcolor{textcolor}%
\pgftext[x=5.437444in,y=0.947037in,left,base]{\color{textcolor}\rmfamily\fontsize{8.000000}{9.600000}\selectfont −1.2}%
\end{pgfscope}%
\begin{pgfscope}%
\pgfsetbuttcap%
\pgfsetroundjoin%
\definecolor{currentfill}{rgb}{0.000000,0.000000,0.000000}%
\pgfsetfillcolor{currentfill}%
\pgfsetlinewidth{0.803000pt}%
\definecolor{currentstroke}{rgb}{0.000000,0.000000,0.000000}%
\pgfsetstrokecolor{currentstroke}%
\pgfsetdash{}{0pt}%
\pgfsys@defobject{currentmarker}{\pgfqpoint{0.000000in}{0.000000in}}{\pgfqpoint{0.048611in}{0.000000in}}{%
\pgfpathmoveto{\pgfqpoint{0.000000in}{0.000000in}}%
\pgfpathlineto{\pgfqpoint{0.048611in}{0.000000in}}%
\pgfusepath{stroke,fill}%
}%
\begin{pgfscope}%
\pgfsys@transformshift{5.340222in}{1.253923in}%
\pgfsys@useobject{currentmarker}{}%
\end{pgfscope}%
\end{pgfscope}%
\begin{pgfscope}%
\definecolor{textcolor}{rgb}{0.000000,0.000000,0.000000}%
\pgfsetstrokecolor{textcolor}%
\pgfsetfillcolor{textcolor}%
\pgftext[x=5.437444in,y=1.215368in,left,base]{\color{textcolor}\rmfamily\fontsize{8.000000}{9.600000}\selectfont −1.0}%
\end{pgfscope}%
\begin{pgfscope}%
\pgfsetbuttcap%
\pgfsetroundjoin%
\definecolor{currentfill}{rgb}{0.000000,0.000000,0.000000}%
\pgfsetfillcolor{currentfill}%
\pgfsetlinewidth{0.803000pt}%
\definecolor{currentstroke}{rgb}{0.000000,0.000000,0.000000}%
\pgfsetstrokecolor{currentstroke}%
\pgfsetdash{}{0pt}%
\pgfsys@defobject{currentmarker}{\pgfqpoint{0.000000in}{0.000000in}}{\pgfqpoint{0.048611in}{0.000000in}}{%
\pgfpathmoveto{\pgfqpoint{0.000000in}{0.000000in}}%
\pgfpathlineto{\pgfqpoint{0.048611in}{0.000000in}}%
\pgfusepath{stroke,fill}%
}%
\begin{pgfscope}%
\pgfsys@transformshift{5.340222in}{1.522254in}%
\pgfsys@useobject{currentmarker}{}%
\end{pgfscope}%
\end{pgfscope}%
\begin{pgfscope}%
\definecolor{textcolor}{rgb}{0.000000,0.000000,0.000000}%
\pgfsetstrokecolor{textcolor}%
\pgfsetfillcolor{textcolor}%
\pgftext[x=5.437444in,y=1.483698in,left,base]{\color{textcolor}\rmfamily\fontsize{8.000000}{9.600000}\selectfont −0.8}%
\end{pgfscope}%
\begin{pgfscope}%
\pgfsetbuttcap%
\pgfsetroundjoin%
\definecolor{currentfill}{rgb}{0.000000,0.000000,0.000000}%
\pgfsetfillcolor{currentfill}%
\pgfsetlinewidth{0.803000pt}%
\definecolor{currentstroke}{rgb}{0.000000,0.000000,0.000000}%
\pgfsetstrokecolor{currentstroke}%
\pgfsetdash{}{0pt}%
\pgfsys@defobject{currentmarker}{\pgfqpoint{0.000000in}{0.000000in}}{\pgfqpoint{0.048611in}{0.000000in}}{%
\pgfpathmoveto{\pgfqpoint{0.000000in}{0.000000in}}%
\pgfpathlineto{\pgfqpoint{0.048611in}{0.000000in}}%
\pgfusepath{stroke,fill}%
}%
\begin{pgfscope}%
\pgfsys@transformshift{5.340222in}{1.790585in}%
\pgfsys@useobject{currentmarker}{}%
\end{pgfscope}%
\end{pgfscope}%
\begin{pgfscope}%
\definecolor{textcolor}{rgb}{0.000000,0.000000,0.000000}%
\pgfsetstrokecolor{textcolor}%
\pgfsetfillcolor{textcolor}%
\pgftext[x=5.437444in,y=1.752029in,left,base]{\color{textcolor}\rmfamily\fontsize{8.000000}{9.600000}\selectfont −0.6}%
\end{pgfscope}%
\begin{pgfscope}%
\pgfsetbuttcap%
\pgfsetroundjoin%
\definecolor{currentfill}{rgb}{0.000000,0.000000,0.000000}%
\pgfsetfillcolor{currentfill}%
\pgfsetlinewidth{0.803000pt}%
\definecolor{currentstroke}{rgb}{0.000000,0.000000,0.000000}%
\pgfsetstrokecolor{currentstroke}%
\pgfsetdash{}{0pt}%
\pgfsys@defobject{currentmarker}{\pgfqpoint{0.000000in}{0.000000in}}{\pgfqpoint{0.048611in}{0.000000in}}{%
\pgfpathmoveto{\pgfqpoint{0.000000in}{0.000000in}}%
\pgfpathlineto{\pgfqpoint{0.048611in}{0.000000in}}%
\pgfusepath{stroke,fill}%
}%
\begin{pgfscope}%
\pgfsys@transformshift{5.340222in}{2.058915in}%
\pgfsys@useobject{currentmarker}{}%
\end{pgfscope}%
\end{pgfscope}%
\begin{pgfscope}%
\definecolor{textcolor}{rgb}{0.000000,0.000000,0.000000}%
\pgfsetstrokecolor{textcolor}%
\pgfsetfillcolor{textcolor}%
\pgftext[x=5.437444in,y=2.020360in,left,base]{\color{textcolor}\rmfamily\fontsize{8.000000}{9.600000}\selectfont −0.4}%
\end{pgfscope}%
\begin{pgfscope}%
\pgfsetbuttcap%
\pgfsetroundjoin%
\definecolor{currentfill}{rgb}{0.000000,0.000000,0.000000}%
\pgfsetfillcolor{currentfill}%
\pgfsetlinewidth{0.803000pt}%
\definecolor{currentstroke}{rgb}{0.000000,0.000000,0.000000}%
\pgfsetstrokecolor{currentstroke}%
\pgfsetdash{}{0pt}%
\pgfsys@defobject{currentmarker}{\pgfqpoint{0.000000in}{0.000000in}}{\pgfqpoint{0.048611in}{0.000000in}}{%
\pgfpathmoveto{\pgfqpoint{0.000000in}{0.000000in}}%
\pgfpathlineto{\pgfqpoint{0.048611in}{0.000000in}}%
\pgfusepath{stroke,fill}%
}%
\begin{pgfscope}%
\pgfsys@transformshift{5.340222in}{2.327246in}%
\pgfsys@useobject{currentmarker}{}%
\end{pgfscope}%
\end{pgfscope}%
\begin{pgfscope}%
\definecolor{textcolor}{rgb}{0.000000,0.000000,0.000000}%
\pgfsetstrokecolor{textcolor}%
\pgfsetfillcolor{textcolor}%
\pgftext[x=5.437444in,y=2.288690in,left,base]{\color{textcolor}\rmfamily\fontsize{8.000000}{9.600000}\selectfont −0.2}%
\end{pgfscope}%
\begin{pgfscope}%
\pgfsetbuttcap%
\pgfsetroundjoin%
\definecolor{currentfill}{rgb}{0.000000,0.000000,0.000000}%
\pgfsetfillcolor{currentfill}%
\pgfsetlinewidth{0.803000pt}%
\definecolor{currentstroke}{rgb}{0.000000,0.000000,0.000000}%
\pgfsetstrokecolor{currentstroke}%
\pgfsetdash{}{0pt}%
\pgfsys@defobject{currentmarker}{\pgfqpoint{0.000000in}{0.000000in}}{\pgfqpoint{0.048611in}{0.000000in}}{%
\pgfpathmoveto{\pgfqpoint{0.000000in}{0.000000in}}%
\pgfpathlineto{\pgfqpoint{0.048611in}{0.000000in}}%
\pgfusepath{stroke,fill}%
}%
\begin{pgfscope}%
\pgfsys@transformshift{5.340222in}{2.595577in}%
\pgfsys@useobject{currentmarker}{}%
\end{pgfscope}%
\end{pgfscope}%
\begin{pgfscope}%
\definecolor{textcolor}{rgb}{0.000000,0.000000,0.000000}%
\pgfsetstrokecolor{textcolor}%
\pgfsetfillcolor{textcolor}%
\pgftext[x=5.437444in,y=2.557021in,left,base]{\color{textcolor}\rmfamily\fontsize{8.000000}{9.600000}\selectfont 0.0}%
\end{pgfscope}%
\begin{pgfscope}%
\definecolor{textcolor}{rgb}{0.000000,0.000000,0.000000}%
\pgfsetstrokecolor{textcolor}%
\pgfsetfillcolor{textcolor}%
\pgftext[x=5.340222in,y=2.745800in,right,base]{\color{textcolor}\rmfamily\fontsize{8.000000}{9.600000}\selectfont 1e−6}%
\end{pgfscope}%
\begin{pgfscope}%
\pgfpathrectangle{\pgfqpoint{2.962722in}{0.315889in}}{\pgfqpoint{2.377500in}{2.388245in}}%
\pgfusepath{clip}%
\pgfsetrectcap%
\pgfsetroundjoin%
\pgfsetlinewidth{1.505625pt}%
\definecolor{currentstroke}{rgb}{0.121569,0.466667,0.705882}%
\pgfsetstrokecolor{currentstroke}%
\pgfsetdash{}{0pt}%
\pgfpathmoveto{\pgfqpoint{3.070790in}{2.595577in}}%
\pgfpathlineto{\pgfqpoint{3.087944in}{2.595577in}}%
\pgfpathlineto{\pgfqpoint{3.105098in}{2.595577in}}%
\pgfpathlineto{\pgfqpoint{3.122251in}{2.595577in}}%
\pgfpathlineto{\pgfqpoint{3.139405in}{2.595577in}}%
\pgfpathlineto{\pgfqpoint{3.156559in}{2.595577in}}%
\pgfpathlineto{\pgfqpoint{3.173712in}{2.595577in}}%
\pgfpathlineto{\pgfqpoint{3.190866in}{2.595577in}}%
\pgfpathlineto{\pgfqpoint{3.208020in}{2.595577in}}%
\pgfpathlineto{\pgfqpoint{3.225174in}{2.595577in}}%
\pgfpathlineto{\pgfqpoint{3.242327in}{2.595577in}}%
\pgfpathlineto{\pgfqpoint{3.259481in}{2.595577in}}%
\pgfpathlineto{\pgfqpoint{3.276635in}{2.595577in}}%
\pgfpathlineto{\pgfqpoint{3.293788in}{2.595577in}}%
\pgfpathlineto{\pgfqpoint{3.310942in}{2.595577in}}%
\pgfpathlineto{\pgfqpoint{3.328096in}{2.595577in}}%
\pgfpathlineto{\pgfqpoint{3.345249in}{2.595577in}}%
\pgfpathlineto{\pgfqpoint{3.362403in}{2.595577in}}%
\pgfpathlineto{\pgfqpoint{3.379557in}{2.595577in}}%
\pgfpathlineto{\pgfqpoint{3.396710in}{2.595577in}}%
\pgfpathlineto{\pgfqpoint{3.413864in}{2.595577in}}%
\pgfpathlineto{\pgfqpoint{3.431018in}{2.595577in}}%
\pgfpathlineto{\pgfqpoint{3.448171in}{2.595577in}}%
\pgfpathlineto{\pgfqpoint{3.465325in}{2.595577in}}%
\pgfpathlineto{\pgfqpoint{3.482479in}{2.595577in}}%
\pgfpathlineto{\pgfqpoint{3.499632in}{2.595577in}}%
\pgfpathlineto{\pgfqpoint{3.516786in}{2.595577in}}%
\pgfpathlineto{\pgfqpoint{3.533940in}{2.595577in}}%
\pgfpathlineto{\pgfqpoint{3.551093in}{2.595577in}}%
\pgfpathlineto{\pgfqpoint{3.568247in}{2.595577in}}%
\pgfpathlineto{\pgfqpoint{3.585401in}{2.595577in}}%
\pgfpathlineto{\pgfqpoint{3.602554in}{2.595577in}}%
\pgfpathlineto{\pgfqpoint{3.619708in}{2.595577in}}%
\pgfpathlineto{\pgfqpoint{3.636862in}{2.595577in}}%
\pgfpathlineto{\pgfqpoint{3.654016in}{2.595577in}}%
\pgfpathlineto{\pgfqpoint{3.671169in}{2.595577in}}%
\pgfpathlineto{\pgfqpoint{3.688323in}{2.595577in}}%
\pgfpathlineto{\pgfqpoint{3.705477in}{2.595577in}}%
\pgfpathlineto{\pgfqpoint{3.722630in}{2.595577in}}%
\pgfpathlineto{\pgfqpoint{3.739784in}{2.595577in}}%
\pgfpathlineto{\pgfqpoint{3.756938in}{2.595577in}}%
\pgfpathlineto{\pgfqpoint{3.774091in}{2.595577in}}%
\pgfpathlineto{\pgfqpoint{3.791245in}{2.595577in}}%
\pgfpathlineto{\pgfqpoint{3.808399in}{2.595577in}}%
\pgfpathlineto{\pgfqpoint{3.825552in}{2.595577in}}%
\pgfpathlineto{\pgfqpoint{3.842706in}{2.595577in}}%
\pgfpathlineto{\pgfqpoint{3.859860in}{2.595577in}}%
\pgfpathlineto{\pgfqpoint{3.877013in}{2.595577in}}%
\pgfpathlineto{\pgfqpoint{3.894167in}{2.595577in}}%
\pgfpathlineto{\pgfqpoint{3.911321in}{2.595577in}}%
\pgfpathlineto{\pgfqpoint{3.928474in}{2.595577in}}%
\pgfpathlineto{\pgfqpoint{3.945628in}{2.595577in}}%
\pgfpathlineto{\pgfqpoint{3.962782in}{2.595577in}}%
\pgfpathlineto{\pgfqpoint{3.979935in}{2.595577in}}%
\pgfpathlineto{\pgfqpoint{3.997089in}{2.595577in}}%
\pgfpathlineto{\pgfqpoint{4.014243in}{2.595577in}}%
\pgfpathlineto{\pgfqpoint{4.031396in}{2.595577in}}%
\pgfpathlineto{\pgfqpoint{4.048550in}{2.595577in}}%
\pgfpathlineto{\pgfqpoint{4.065704in}{2.595577in}}%
\pgfpathlineto{\pgfqpoint{4.082858in}{2.595577in}}%
\pgfpathlineto{\pgfqpoint{4.100011in}{2.595577in}}%
\pgfpathlineto{\pgfqpoint{4.117165in}{2.595577in}}%
\pgfpathlineto{\pgfqpoint{4.134319in}{2.595577in}}%
\pgfpathlineto{\pgfqpoint{4.151472in}{2.595577in}}%
\pgfpathlineto{\pgfqpoint{4.168626in}{2.595577in}}%
\pgfpathlineto{\pgfqpoint{4.185780in}{2.595577in}}%
\pgfpathlineto{\pgfqpoint{4.202933in}{2.595577in}}%
\pgfpathlineto{\pgfqpoint{4.220087in}{2.595577in}}%
\pgfpathlineto{\pgfqpoint{4.237241in}{2.595577in}}%
\pgfpathlineto{\pgfqpoint{4.254394in}{2.595577in}}%
\pgfpathlineto{\pgfqpoint{4.271548in}{2.595577in}}%
\pgfpathlineto{\pgfqpoint{4.288702in}{2.595577in}}%
\pgfpathlineto{\pgfqpoint{4.305855in}{2.595577in}}%
\pgfpathlineto{\pgfqpoint{4.323009in}{2.595577in}}%
\pgfpathlineto{\pgfqpoint{4.340163in}{2.595577in}}%
\pgfpathlineto{\pgfqpoint{4.357316in}{2.595577in}}%
\pgfpathlineto{\pgfqpoint{4.374470in}{2.595577in}}%
\pgfpathlineto{\pgfqpoint{4.391624in}{2.595577in}}%
\pgfpathlineto{\pgfqpoint{4.408777in}{2.595577in}}%
\pgfpathlineto{\pgfqpoint{4.425931in}{2.595577in}}%
\pgfpathlineto{\pgfqpoint{4.443085in}{2.595577in}}%
\pgfpathlineto{\pgfqpoint{4.460238in}{2.595577in}}%
\pgfpathlineto{\pgfqpoint{4.477392in}{2.595577in}}%
\pgfpathlineto{\pgfqpoint{4.494546in}{2.595577in}}%
\pgfpathlineto{\pgfqpoint{4.511699in}{2.595577in}}%
\pgfpathlineto{\pgfqpoint{4.528853in}{2.595577in}}%
\pgfpathlineto{\pgfqpoint{4.546007in}{2.595577in}}%
\pgfpathlineto{\pgfqpoint{4.563161in}{2.595577in}}%
\pgfpathlineto{\pgfqpoint{4.580314in}{2.595577in}}%
\pgfpathlineto{\pgfqpoint{4.597468in}{2.595577in}}%
\pgfpathlineto{\pgfqpoint{4.614622in}{2.595577in}}%
\pgfpathlineto{\pgfqpoint{4.631775in}{2.595577in}}%
\pgfpathlineto{\pgfqpoint{4.648929in}{2.595577in}}%
\pgfpathlineto{\pgfqpoint{4.666083in}{2.595577in}}%
\pgfpathlineto{\pgfqpoint{4.683236in}{2.595577in}}%
\pgfpathlineto{\pgfqpoint{4.700390in}{2.595577in}}%
\pgfpathlineto{\pgfqpoint{4.717544in}{2.595577in}}%
\pgfpathlineto{\pgfqpoint{4.734697in}{2.595577in}}%
\pgfpathlineto{\pgfqpoint{4.751851in}{2.595577in}}%
\pgfpathlineto{\pgfqpoint{4.769005in}{2.595577in}}%
\pgfpathlineto{\pgfqpoint{4.786158in}{2.595577in}}%
\pgfpathlineto{\pgfqpoint{4.803312in}{2.595577in}}%
\pgfpathlineto{\pgfqpoint{4.820466in}{2.595577in}}%
\pgfpathlineto{\pgfqpoint{4.837619in}{2.595577in}}%
\pgfpathlineto{\pgfqpoint{4.854773in}{2.595577in}}%
\pgfpathlineto{\pgfqpoint{4.871927in}{2.595577in}}%
\pgfpathlineto{\pgfqpoint{4.889080in}{2.595577in}}%
\pgfpathlineto{\pgfqpoint{4.906234in}{2.595577in}}%
\pgfpathlineto{\pgfqpoint{4.923388in}{2.595577in}}%
\pgfpathlineto{\pgfqpoint{4.940541in}{2.595577in}}%
\pgfpathlineto{\pgfqpoint{4.957695in}{2.595577in}}%
\pgfpathlineto{\pgfqpoint{4.974849in}{2.595577in}}%
\pgfpathlineto{\pgfqpoint{4.992003in}{2.595577in}}%
\pgfpathlineto{\pgfqpoint{5.009156in}{2.595577in}}%
\pgfpathlineto{\pgfqpoint{5.026310in}{2.595577in}}%
\pgfpathlineto{\pgfqpoint{5.043464in}{2.595577in}}%
\pgfpathlineto{\pgfqpoint{5.060617in}{2.595577in}}%
\pgfpathlineto{\pgfqpoint{5.077771in}{2.595577in}}%
\pgfpathlineto{\pgfqpoint{5.094925in}{2.595577in}}%
\pgfpathlineto{\pgfqpoint{5.112078in}{2.595577in}}%
\pgfpathlineto{\pgfqpoint{5.129232in}{2.595577in}}%
\pgfpathlineto{\pgfqpoint{5.146386in}{2.595577in}}%
\pgfpathlineto{\pgfqpoint{5.163539in}{2.595577in}}%
\pgfpathlineto{\pgfqpoint{5.180693in}{2.595577in}}%
\pgfpathlineto{\pgfqpoint{5.197847in}{2.595577in}}%
\pgfpathlineto{\pgfqpoint{5.215000in}{2.595577in}}%
\pgfusepath{stroke}%
\end{pgfscope}%
\begin{pgfscope}%
\pgfpathrectangle{\pgfqpoint{2.962722in}{0.315889in}}{\pgfqpoint{2.377500in}{2.388245in}}%
\pgfusepath{clip}%
\pgfsetrectcap%
\pgfsetroundjoin%
\pgfsetlinewidth{1.505625pt}%
\definecolor{currentstroke}{rgb}{1.000000,0.498039,0.054902}%
\pgfsetstrokecolor{currentstroke}%
\pgfsetdash{}{0pt}%
\pgfpathmoveto{\pgfqpoint{3.070790in}{2.595577in}}%
\pgfpathlineto{\pgfqpoint{3.087944in}{2.595577in}}%
\pgfpathlineto{\pgfqpoint{3.105098in}{2.595577in}}%
\pgfpathlineto{\pgfqpoint{3.122251in}{2.595577in}}%
\pgfpathlineto{\pgfqpoint{3.139405in}{2.595577in}}%
\pgfpathlineto{\pgfqpoint{3.156559in}{2.595577in}}%
\pgfpathlineto{\pgfqpoint{3.173712in}{2.595577in}}%
\pgfpathlineto{\pgfqpoint{3.190866in}{2.595577in}}%
\pgfpathlineto{\pgfqpoint{3.208020in}{2.595577in}}%
\pgfpathlineto{\pgfqpoint{3.225174in}{2.595577in}}%
\pgfpathlineto{\pgfqpoint{3.242327in}{2.595577in}}%
\pgfpathlineto{\pgfqpoint{3.259481in}{2.595577in}}%
\pgfpathlineto{\pgfqpoint{3.276635in}{2.595577in}}%
\pgfpathlineto{\pgfqpoint{3.293788in}{2.595577in}}%
\pgfpathlineto{\pgfqpoint{3.310942in}{2.595577in}}%
\pgfpathlineto{\pgfqpoint{3.328096in}{2.595577in}}%
\pgfpathlineto{\pgfqpoint{3.345249in}{2.595577in}}%
\pgfpathlineto{\pgfqpoint{3.362403in}{2.595577in}}%
\pgfpathlineto{\pgfqpoint{3.379557in}{2.595577in}}%
\pgfpathlineto{\pgfqpoint{3.396710in}{2.595577in}}%
\pgfpathlineto{\pgfqpoint{3.413864in}{2.595577in}}%
\pgfpathlineto{\pgfqpoint{3.431018in}{2.595577in}}%
\pgfpathlineto{\pgfqpoint{3.448171in}{2.595577in}}%
\pgfpathlineto{\pgfqpoint{3.465325in}{2.595577in}}%
\pgfpathlineto{\pgfqpoint{3.482479in}{2.595577in}}%
\pgfpathlineto{\pgfqpoint{3.499632in}{2.595577in}}%
\pgfpathlineto{\pgfqpoint{3.516786in}{2.595577in}}%
\pgfpathlineto{\pgfqpoint{3.533940in}{2.595577in}}%
\pgfpathlineto{\pgfqpoint{3.551093in}{2.595577in}}%
\pgfpathlineto{\pgfqpoint{3.568247in}{2.595577in}}%
\pgfpathlineto{\pgfqpoint{3.585401in}{2.595577in}}%
\pgfpathlineto{\pgfqpoint{3.602554in}{2.595577in}}%
\pgfpathlineto{\pgfqpoint{3.619708in}{2.595577in}}%
\pgfpathlineto{\pgfqpoint{3.636862in}{2.595577in}}%
\pgfpathlineto{\pgfqpoint{3.654016in}{2.595577in}}%
\pgfpathlineto{\pgfqpoint{3.671169in}{2.595577in}}%
\pgfpathlineto{\pgfqpoint{3.688323in}{2.595577in}}%
\pgfpathlineto{\pgfqpoint{3.705477in}{2.595577in}}%
\pgfpathlineto{\pgfqpoint{3.722630in}{2.595577in}}%
\pgfpathlineto{\pgfqpoint{3.739784in}{2.595577in}}%
\pgfpathlineto{\pgfqpoint{3.756938in}{2.595577in}}%
\pgfpathlineto{\pgfqpoint{3.774091in}{2.595577in}}%
\pgfpathlineto{\pgfqpoint{3.791245in}{2.595577in}}%
\pgfpathlineto{\pgfqpoint{3.808399in}{2.595577in}}%
\pgfpathlineto{\pgfqpoint{3.825552in}{2.595577in}}%
\pgfpathlineto{\pgfqpoint{3.842706in}{2.595577in}}%
\pgfpathlineto{\pgfqpoint{3.859860in}{2.595577in}}%
\pgfpathlineto{\pgfqpoint{3.877013in}{2.595577in}}%
\pgfpathlineto{\pgfqpoint{3.894167in}{2.595577in}}%
\pgfpathlineto{\pgfqpoint{3.911321in}{2.595577in}}%
\pgfpathlineto{\pgfqpoint{3.928474in}{2.595577in}}%
\pgfpathlineto{\pgfqpoint{3.945628in}{2.595577in}}%
\pgfpathlineto{\pgfqpoint{3.962782in}{2.595577in}}%
\pgfpathlineto{\pgfqpoint{3.979935in}{2.595577in}}%
\pgfpathlineto{\pgfqpoint{3.997089in}{2.595577in}}%
\pgfpathlineto{\pgfqpoint{4.014243in}{2.595577in}}%
\pgfpathlineto{\pgfqpoint{4.031396in}{2.595577in}}%
\pgfpathlineto{\pgfqpoint{4.048550in}{2.595577in}}%
\pgfpathlineto{\pgfqpoint{4.065704in}{2.595577in}}%
\pgfpathlineto{\pgfqpoint{4.082858in}{2.595577in}}%
\pgfpathlineto{\pgfqpoint{4.100011in}{2.595577in}}%
\pgfpathlineto{\pgfqpoint{4.117165in}{2.595577in}}%
\pgfpathlineto{\pgfqpoint{4.134319in}{2.595577in}}%
\pgfpathlineto{\pgfqpoint{4.151472in}{2.595577in}}%
\pgfpathlineto{\pgfqpoint{4.168626in}{2.595577in}}%
\pgfpathlineto{\pgfqpoint{4.185780in}{2.595577in}}%
\pgfpathlineto{\pgfqpoint{4.202933in}{2.595577in}}%
\pgfpathlineto{\pgfqpoint{4.220087in}{2.595577in}}%
\pgfpathlineto{\pgfqpoint{4.237241in}{2.595577in}}%
\pgfpathlineto{\pgfqpoint{4.254394in}{2.595577in}}%
\pgfpathlineto{\pgfqpoint{4.271548in}{2.595577in}}%
\pgfpathlineto{\pgfqpoint{4.288702in}{2.595577in}}%
\pgfpathlineto{\pgfqpoint{4.305855in}{2.595577in}}%
\pgfpathlineto{\pgfqpoint{4.323009in}{2.595577in}}%
\pgfpathlineto{\pgfqpoint{4.340163in}{2.595577in}}%
\pgfpathlineto{\pgfqpoint{4.357316in}{2.595577in}}%
\pgfpathlineto{\pgfqpoint{4.374470in}{2.595577in}}%
\pgfpathlineto{\pgfqpoint{4.391624in}{2.595577in}}%
\pgfpathlineto{\pgfqpoint{4.408777in}{2.595577in}}%
\pgfpathlineto{\pgfqpoint{4.425931in}{2.595577in}}%
\pgfpathlineto{\pgfqpoint{4.443085in}{2.595577in}}%
\pgfpathlineto{\pgfqpoint{4.460238in}{2.595577in}}%
\pgfpathlineto{\pgfqpoint{4.477392in}{2.595577in}}%
\pgfpathlineto{\pgfqpoint{4.494546in}{2.595577in}}%
\pgfpathlineto{\pgfqpoint{4.511699in}{2.595577in}}%
\pgfpathlineto{\pgfqpoint{4.528853in}{2.595577in}}%
\pgfpathlineto{\pgfqpoint{4.546007in}{2.595577in}}%
\pgfpathlineto{\pgfqpoint{4.563161in}{2.595577in}}%
\pgfpathlineto{\pgfqpoint{4.580314in}{2.595577in}}%
\pgfpathlineto{\pgfqpoint{4.597468in}{2.595577in}}%
\pgfpathlineto{\pgfqpoint{4.614622in}{2.595577in}}%
\pgfpathlineto{\pgfqpoint{4.631775in}{2.595577in}}%
\pgfpathlineto{\pgfqpoint{4.648929in}{2.595577in}}%
\pgfpathlineto{\pgfqpoint{4.666083in}{2.595577in}}%
\pgfpathlineto{\pgfqpoint{4.683236in}{2.595577in}}%
\pgfpathlineto{\pgfqpoint{4.700390in}{2.595577in}}%
\pgfpathlineto{\pgfqpoint{4.717544in}{2.595577in}}%
\pgfpathlineto{\pgfqpoint{4.734697in}{2.595577in}}%
\pgfpathlineto{\pgfqpoint{4.751851in}{2.595577in}}%
\pgfpathlineto{\pgfqpoint{4.769005in}{2.595577in}}%
\pgfpathlineto{\pgfqpoint{4.786158in}{2.595577in}}%
\pgfpathlineto{\pgfqpoint{4.803312in}{2.595577in}}%
\pgfpathlineto{\pgfqpoint{4.820466in}{2.595577in}}%
\pgfpathlineto{\pgfqpoint{4.837619in}{2.595577in}}%
\pgfpathlineto{\pgfqpoint{4.854773in}{2.595577in}}%
\pgfpathlineto{\pgfqpoint{4.871927in}{2.595577in}}%
\pgfpathlineto{\pgfqpoint{4.889080in}{2.595577in}}%
\pgfpathlineto{\pgfqpoint{4.906234in}{2.595577in}}%
\pgfpathlineto{\pgfqpoint{4.923388in}{2.595577in}}%
\pgfpathlineto{\pgfqpoint{4.940541in}{2.595577in}}%
\pgfpathlineto{\pgfqpoint{4.957695in}{2.595577in}}%
\pgfpathlineto{\pgfqpoint{4.974849in}{2.595577in}}%
\pgfpathlineto{\pgfqpoint{4.992003in}{2.595577in}}%
\pgfpathlineto{\pgfqpoint{5.009156in}{2.595577in}}%
\pgfpathlineto{\pgfqpoint{5.026310in}{2.595577in}}%
\pgfpathlineto{\pgfqpoint{5.043464in}{2.595577in}}%
\pgfpathlineto{\pgfqpoint{5.060617in}{2.595577in}}%
\pgfpathlineto{\pgfqpoint{5.077771in}{2.595577in}}%
\pgfpathlineto{\pgfqpoint{5.094925in}{2.595577in}}%
\pgfpathlineto{\pgfqpoint{5.112078in}{2.595577in}}%
\pgfpathlineto{\pgfqpoint{5.129232in}{2.595577in}}%
\pgfpathlineto{\pgfqpoint{5.146386in}{2.595577in}}%
\pgfpathlineto{\pgfqpoint{5.163539in}{2.595577in}}%
\pgfpathlineto{\pgfqpoint{5.180693in}{2.595577in}}%
\pgfpathlineto{\pgfqpoint{5.197847in}{2.595577in}}%
\pgfpathlineto{\pgfqpoint{5.215000in}{2.595577in}}%
\pgfusepath{stroke}%
\end{pgfscope}%
\begin{pgfscope}%
\pgfpathrectangle{\pgfqpoint{2.962722in}{0.315889in}}{\pgfqpoint{2.377500in}{2.388245in}}%
\pgfusepath{clip}%
\pgfsetrectcap%
\pgfsetroundjoin%
\pgfsetlinewidth{1.505625pt}%
\definecolor{currentstroke}{rgb}{0.172549,0.627451,0.172549}%
\pgfsetstrokecolor{currentstroke}%
\pgfsetdash{}{0pt}%
\pgfpathmoveto{\pgfqpoint{3.070790in}{2.595577in}}%
\pgfpathlineto{\pgfqpoint{3.087944in}{2.595577in}}%
\pgfpathlineto{\pgfqpoint{3.105098in}{2.595577in}}%
\pgfpathlineto{\pgfqpoint{3.122251in}{2.595577in}}%
\pgfpathlineto{\pgfqpoint{3.139405in}{2.595577in}}%
\pgfpathlineto{\pgfqpoint{3.156559in}{2.595577in}}%
\pgfpathlineto{\pgfqpoint{3.173712in}{2.595577in}}%
\pgfpathlineto{\pgfqpoint{3.190866in}{2.595577in}}%
\pgfpathlineto{\pgfqpoint{3.208020in}{2.595577in}}%
\pgfpathlineto{\pgfqpoint{3.225174in}{2.595577in}}%
\pgfpathlineto{\pgfqpoint{3.242327in}{2.595577in}}%
\pgfpathlineto{\pgfqpoint{3.259481in}{2.595577in}}%
\pgfpathlineto{\pgfqpoint{3.276635in}{2.595577in}}%
\pgfpathlineto{\pgfqpoint{3.293788in}{2.595577in}}%
\pgfpathlineto{\pgfqpoint{3.310942in}{2.595577in}}%
\pgfpathlineto{\pgfqpoint{3.328096in}{2.595577in}}%
\pgfpathlineto{\pgfqpoint{3.345249in}{2.595577in}}%
\pgfpathlineto{\pgfqpoint{3.362403in}{2.595577in}}%
\pgfpathlineto{\pgfqpoint{3.379557in}{2.595577in}}%
\pgfpathlineto{\pgfqpoint{3.396710in}{2.595577in}}%
\pgfpathlineto{\pgfqpoint{3.413864in}{2.595577in}}%
\pgfpathlineto{\pgfqpoint{3.431018in}{2.595577in}}%
\pgfpathlineto{\pgfqpoint{3.448171in}{2.595577in}}%
\pgfpathlineto{\pgfqpoint{3.465325in}{2.595577in}}%
\pgfpathlineto{\pgfqpoint{3.482479in}{2.595577in}}%
\pgfpathlineto{\pgfqpoint{3.499632in}{2.595577in}}%
\pgfpathlineto{\pgfqpoint{3.516786in}{2.595577in}}%
\pgfpathlineto{\pgfqpoint{3.533940in}{2.595577in}}%
\pgfpathlineto{\pgfqpoint{3.551093in}{2.595577in}}%
\pgfpathlineto{\pgfqpoint{3.568247in}{2.595577in}}%
\pgfpathlineto{\pgfqpoint{3.585401in}{2.595577in}}%
\pgfpathlineto{\pgfqpoint{3.602554in}{2.595577in}}%
\pgfpathlineto{\pgfqpoint{3.619708in}{2.595577in}}%
\pgfpathlineto{\pgfqpoint{3.636862in}{2.595577in}}%
\pgfpathlineto{\pgfqpoint{3.654016in}{2.595577in}}%
\pgfpathlineto{\pgfqpoint{3.671169in}{2.595577in}}%
\pgfpathlineto{\pgfqpoint{3.688323in}{2.595577in}}%
\pgfpathlineto{\pgfqpoint{3.705477in}{2.595577in}}%
\pgfpathlineto{\pgfqpoint{3.722630in}{2.595577in}}%
\pgfpathlineto{\pgfqpoint{3.739784in}{2.595577in}}%
\pgfpathlineto{\pgfqpoint{3.756938in}{2.595577in}}%
\pgfpathlineto{\pgfqpoint{3.774091in}{2.595577in}}%
\pgfpathlineto{\pgfqpoint{3.791245in}{2.595577in}}%
\pgfpathlineto{\pgfqpoint{3.808399in}{2.595577in}}%
\pgfpathlineto{\pgfqpoint{3.825552in}{2.595577in}}%
\pgfpathlineto{\pgfqpoint{3.842706in}{2.595577in}}%
\pgfpathlineto{\pgfqpoint{3.859860in}{2.595577in}}%
\pgfpathlineto{\pgfqpoint{3.877013in}{2.595577in}}%
\pgfpathlineto{\pgfqpoint{3.894167in}{2.595577in}}%
\pgfpathlineto{\pgfqpoint{3.911321in}{2.595577in}}%
\pgfpathlineto{\pgfqpoint{3.928474in}{2.595577in}}%
\pgfpathlineto{\pgfqpoint{3.945628in}{2.595577in}}%
\pgfpathlineto{\pgfqpoint{3.962782in}{2.595577in}}%
\pgfpathlineto{\pgfqpoint{3.979935in}{2.595577in}}%
\pgfpathlineto{\pgfqpoint{3.997089in}{2.595577in}}%
\pgfpathlineto{\pgfqpoint{4.014243in}{2.595577in}}%
\pgfpathlineto{\pgfqpoint{4.031396in}{2.595577in}}%
\pgfpathlineto{\pgfqpoint{4.048550in}{2.595577in}}%
\pgfpathlineto{\pgfqpoint{4.065704in}{2.595577in}}%
\pgfpathlineto{\pgfqpoint{4.082858in}{2.595577in}}%
\pgfpathlineto{\pgfqpoint{4.100011in}{2.595577in}}%
\pgfpathlineto{\pgfqpoint{4.117165in}{2.595577in}}%
\pgfpathlineto{\pgfqpoint{4.134319in}{2.595577in}}%
\pgfpathlineto{\pgfqpoint{4.151472in}{2.595577in}}%
\pgfpathlineto{\pgfqpoint{4.168626in}{2.595577in}}%
\pgfpathlineto{\pgfqpoint{4.185780in}{2.595577in}}%
\pgfpathlineto{\pgfqpoint{4.202933in}{2.595577in}}%
\pgfpathlineto{\pgfqpoint{4.220087in}{2.595577in}}%
\pgfpathlineto{\pgfqpoint{4.237241in}{2.595577in}}%
\pgfpathlineto{\pgfqpoint{4.254394in}{2.595577in}}%
\pgfpathlineto{\pgfqpoint{4.271548in}{2.595577in}}%
\pgfpathlineto{\pgfqpoint{4.288702in}{2.595577in}}%
\pgfpathlineto{\pgfqpoint{4.305855in}{2.595577in}}%
\pgfpathlineto{\pgfqpoint{4.323009in}{2.595577in}}%
\pgfpathlineto{\pgfqpoint{4.340163in}{2.595577in}}%
\pgfpathlineto{\pgfqpoint{4.357316in}{2.595577in}}%
\pgfpathlineto{\pgfqpoint{4.374470in}{2.595577in}}%
\pgfpathlineto{\pgfqpoint{4.391624in}{2.595577in}}%
\pgfpathlineto{\pgfqpoint{4.408777in}{2.595577in}}%
\pgfpathlineto{\pgfqpoint{4.425931in}{2.595577in}}%
\pgfpathlineto{\pgfqpoint{4.443085in}{2.595577in}}%
\pgfpathlineto{\pgfqpoint{4.460238in}{2.595577in}}%
\pgfpathlineto{\pgfqpoint{4.477392in}{2.595577in}}%
\pgfpathlineto{\pgfqpoint{4.494546in}{2.595577in}}%
\pgfpathlineto{\pgfqpoint{4.511699in}{2.595577in}}%
\pgfpathlineto{\pgfqpoint{4.528853in}{2.595577in}}%
\pgfpathlineto{\pgfqpoint{4.546007in}{2.595577in}}%
\pgfpathlineto{\pgfqpoint{4.563161in}{2.595577in}}%
\pgfpathlineto{\pgfqpoint{4.580314in}{2.595577in}}%
\pgfpathlineto{\pgfqpoint{4.597468in}{2.595577in}}%
\pgfpathlineto{\pgfqpoint{4.614622in}{2.595577in}}%
\pgfpathlineto{\pgfqpoint{4.631775in}{2.595577in}}%
\pgfpathlineto{\pgfqpoint{4.648929in}{2.595577in}}%
\pgfpathlineto{\pgfqpoint{4.666083in}{2.595577in}}%
\pgfpathlineto{\pgfqpoint{4.683236in}{2.595577in}}%
\pgfpathlineto{\pgfqpoint{4.700390in}{2.595577in}}%
\pgfpathlineto{\pgfqpoint{4.717544in}{2.595577in}}%
\pgfpathlineto{\pgfqpoint{4.734697in}{2.595577in}}%
\pgfpathlineto{\pgfqpoint{4.751851in}{2.595577in}}%
\pgfpathlineto{\pgfqpoint{4.769005in}{2.595577in}}%
\pgfpathlineto{\pgfqpoint{4.786158in}{2.595577in}}%
\pgfpathlineto{\pgfqpoint{4.803312in}{2.595577in}}%
\pgfpathlineto{\pgfqpoint{4.820466in}{2.595577in}}%
\pgfpathlineto{\pgfqpoint{4.837619in}{2.595577in}}%
\pgfpathlineto{\pgfqpoint{4.854773in}{2.595577in}}%
\pgfpathlineto{\pgfqpoint{4.871927in}{2.595577in}}%
\pgfpathlineto{\pgfqpoint{4.889080in}{2.595577in}}%
\pgfpathlineto{\pgfqpoint{4.906234in}{2.595577in}}%
\pgfpathlineto{\pgfqpoint{4.923388in}{2.595577in}}%
\pgfpathlineto{\pgfqpoint{4.940541in}{2.595577in}}%
\pgfpathlineto{\pgfqpoint{4.957695in}{2.595577in}}%
\pgfpathlineto{\pgfqpoint{4.974849in}{2.595577in}}%
\pgfpathlineto{\pgfqpoint{4.992003in}{2.595577in}}%
\pgfpathlineto{\pgfqpoint{5.009156in}{2.595577in}}%
\pgfpathlineto{\pgfqpoint{5.026310in}{2.595577in}}%
\pgfpathlineto{\pgfqpoint{5.043464in}{2.595577in}}%
\pgfpathlineto{\pgfqpoint{5.060617in}{2.595577in}}%
\pgfpathlineto{\pgfqpoint{5.077771in}{2.595577in}}%
\pgfpathlineto{\pgfqpoint{5.094925in}{2.595577in}}%
\pgfpathlineto{\pgfqpoint{5.112078in}{2.595577in}}%
\pgfpathlineto{\pgfqpoint{5.129232in}{2.595577in}}%
\pgfpathlineto{\pgfqpoint{5.146386in}{2.595577in}}%
\pgfpathlineto{\pgfqpoint{5.163539in}{2.595577in}}%
\pgfpathlineto{\pgfqpoint{5.180693in}{2.595577in}}%
\pgfpathlineto{\pgfqpoint{5.197847in}{2.595577in}}%
\pgfpathlineto{\pgfqpoint{5.215000in}{2.595577in}}%
\pgfusepath{stroke}%
\end{pgfscope}%
\begin{pgfscope}%
\pgfpathrectangle{\pgfqpoint{2.962722in}{0.315889in}}{\pgfqpoint{2.377500in}{2.388245in}}%
\pgfusepath{clip}%
\pgfsetrectcap%
\pgfsetroundjoin%
\pgfsetlinewidth{1.505625pt}%
\definecolor{currentstroke}{rgb}{0.839216,0.152941,0.156863}%
\pgfsetstrokecolor{currentstroke}%
\pgfsetdash{}{0pt}%
\pgfpathmoveto{\pgfqpoint{3.070790in}{0.424445in}}%
\pgfpathlineto{\pgfqpoint{3.087944in}{0.424445in}}%
\pgfpathlineto{\pgfqpoint{3.105098in}{0.424445in}}%
\pgfpathlineto{\pgfqpoint{3.122251in}{0.424445in}}%
\pgfpathlineto{\pgfqpoint{3.139405in}{0.424445in}}%
\pgfpathlineto{\pgfqpoint{3.156559in}{0.424445in}}%
\pgfpathlineto{\pgfqpoint{3.173712in}{0.424445in}}%
\pgfpathlineto{\pgfqpoint{3.190866in}{0.424445in}}%
\pgfpathlineto{\pgfqpoint{3.208020in}{0.424445in}}%
\pgfpathlineto{\pgfqpoint{3.225174in}{0.424445in}}%
\pgfpathlineto{\pgfqpoint{3.242327in}{0.424445in}}%
\pgfpathlineto{\pgfqpoint{3.259481in}{0.424445in}}%
\pgfpathlineto{\pgfqpoint{3.276635in}{0.424445in}}%
\pgfpathlineto{\pgfqpoint{3.293788in}{0.424445in}}%
\pgfpathlineto{\pgfqpoint{3.310942in}{0.424445in}}%
\pgfpathlineto{\pgfqpoint{3.328096in}{0.424445in}}%
\pgfpathlineto{\pgfqpoint{3.345249in}{0.424445in}}%
\pgfpathlineto{\pgfqpoint{3.362403in}{0.424445in}}%
\pgfpathlineto{\pgfqpoint{3.379557in}{0.424445in}}%
\pgfpathlineto{\pgfqpoint{3.396710in}{0.424445in}}%
\pgfpathlineto{\pgfqpoint{3.413864in}{0.424445in}}%
\pgfpathlineto{\pgfqpoint{3.431018in}{0.424445in}}%
\pgfpathlineto{\pgfqpoint{3.448171in}{0.424445in}}%
\pgfpathlineto{\pgfqpoint{3.465325in}{0.424445in}}%
\pgfpathlineto{\pgfqpoint{3.482479in}{0.424445in}}%
\pgfpathlineto{\pgfqpoint{3.499632in}{0.424445in}}%
\pgfpathlineto{\pgfqpoint{3.516786in}{0.424445in}}%
\pgfpathlineto{\pgfqpoint{3.533940in}{0.424445in}}%
\pgfpathlineto{\pgfqpoint{3.551093in}{0.424445in}}%
\pgfpathlineto{\pgfqpoint{3.568247in}{0.424445in}}%
\pgfpathlineto{\pgfqpoint{3.585401in}{0.424445in}}%
\pgfpathlineto{\pgfqpoint{3.602554in}{0.424445in}}%
\pgfpathlineto{\pgfqpoint{3.619708in}{0.424445in}}%
\pgfpathlineto{\pgfqpoint{3.636862in}{0.424445in}}%
\pgfpathlineto{\pgfqpoint{3.654016in}{0.424445in}}%
\pgfpathlineto{\pgfqpoint{3.671169in}{0.424445in}}%
\pgfpathlineto{\pgfqpoint{3.688323in}{0.424445in}}%
\pgfpathlineto{\pgfqpoint{3.705477in}{0.424445in}}%
\pgfpathlineto{\pgfqpoint{3.722630in}{0.424445in}}%
\pgfpathlineto{\pgfqpoint{3.739784in}{0.424445in}}%
\pgfpathlineto{\pgfqpoint{3.756938in}{0.424445in}}%
\pgfpathlineto{\pgfqpoint{3.774091in}{0.424445in}}%
\pgfpathlineto{\pgfqpoint{3.791245in}{0.424445in}}%
\pgfpathlineto{\pgfqpoint{3.808399in}{0.424445in}}%
\pgfpathlineto{\pgfqpoint{3.825552in}{0.424445in}}%
\pgfpathlineto{\pgfqpoint{3.842706in}{0.424445in}}%
\pgfpathlineto{\pgfqpoint{3.859860in}{0.424445in}}%
\pgfpathlineto{\pgfqpoint{3.877013in}{0.424445in}}%
\pgfpathlineto{\pgfqpoint{3.894167in}{0.424445in}}%
\pgfpathlineto{\pgfqpoint{3.911321in}{0.424445in}}%
\pgfpathlineto{\pgfqpoint{3.928474in}{0.424445in}}%
\pgfpathlineto{\pgfqpoint{3.945628in}{0.424445in}}%
\pgfpathlineto{\pgfqpoint{3.962782in}{0.424445in}}%
\pgfpathlineto{\pgfqpoint{3.979935in}{0.424445in}}%
\pgfpathlineto{\pgfqpoint{3.997089in}{0.424445in}}%
\pgfpathlineto{\pgfqpoint{4.014243in}{0.424445in}}%
\pgfpathlineto{\pgfqpoint{4.031396in}{0.424445in}}%
\pgfpathlineto{\pgfqpoint{4.048550in}{0.424445in}}%
\pgfpathlineto{\pgfqpoint{4.065704in}{0.424445in}}%
\pgfpathlineto{\pgfqpoint{4.082858in}{0.424445in}}%
\pgfpathlineto{\pgfqpoint{4.100011in}{0.424445in}}%
\pgfpathlineto{\pgfqpoint{4.117165in}{0.424445in}}%
\pgfpathlineto{\pgfqpoint{4.134319in}{0.424445in}}%
\pgfpathlineto{\pgfqpoint{4.151472in}{0.424445in}}%
\pgfpathlineto{\pgfqpoint{4.168626in}{0.424445in}}%
\pgfpathlineto{\pgfqpoint{4.185780in}{0.424445in}}%
\pgfpathlineto{\pgfqpoint{4.202933in}{0.424445in}}%
\pgfpathlineto{\pgfqpoint{4.220087in}{0.424445in}}%
\pgfpathlineto{\pgfqpoint{4.237241in}{0.424445in}}%
\pgfpathlineto{\pgfqpoint{4.254394in}{0.424445in}}%
\pgfpathlineto{\pgfqpoint{4.271548in}{0.424445in}}%
\pgfpathlineto{\pgfqpoint{4.288702in}{0.424445in}}%
\pgfpathlineto{\pgfqpoint{4.305855in}{0.424445in}}%
\pgfpathlineto{\pgfqpoint{4.323009in}{0.424445in}}%
\pgfpathlineto{\pgfqpoint{4.340163in}{0.424445in}}%
\pgfpathlineto{\pgfqpoint{4.357316in}{0.424445in}}%
\pgfpathlineto{\pgfqpoint{4.374470in}{0.424445in}}%
\pgfpathlineto{\pgfqpoint{4.391624in}{0.424445in}}%
\pgfpathlineto{\pgfqpoint{4.408777in}{0.424445in}}%
\pgfpathlineto{\pgfqpoint{4.425931in}{0.424445in}}%
\pgfpathlineto{\pgfqpoint{4.443085in}{0.424445in}}%
\pgfpathlineto{\pgfqpoint{4.460238in}{0.424445in}}%
\pgfpathlineto{\pgfqpoint{4.477392in}{0.424445in}}%
\pgfpathlineto{\pgfqpoint{4.494546in}{0.424445in}}%
\pgfpathlineto{\pgfqpoint{4.511699in}{0.424445in}}%
\pgfpathlineto{\pgfqpoint{4.528853in}{0.424445in}}%
\pgfpathlineto{\pgfqpoint{4.546007in}{0.424445in}}%
\pgfpathlineto{\pgfqpoint{4.563161in}{0.424445in}}%
\pgfpathlineto{\pgfqpoint{4.580314in}{0.424445in}}%
\pgfpathlineto{\pgfqpoint{4.597468in}{0.424445in}}%
\pgfpathlineto{\pgfqpoint{4.614622in}{0.424445in}}%
\pgfpathlineto{\pgfqpoint{4.631775in}{0.424445in}}%
\pgfpathlineto{\pgfqpoint{4.648929in}{0.424445in}}%
\pgfpathlineto{\pgfqpoint{4.666083in}{0.424445in}}%
\pgfpathlineto{\pgfqpoint{4.683236in}{0.424445in}}%
\pgfpathlineto{\pgfqpoint{4.700390in}{0.424445in}}%
\pgfpathlineto{\pgfqpoint{4.717544in}{0.424445in}}%
\pgfpathlineto{\pgfqpoint{4.734697in}{0.424445in}}%
\pgfpathlineto{\pgfqpoint{4.751851in}{0.424445in}}%
\pgfpathlineto{\pgfqpoint{4.769005in}{0.424445in}}%
\pgfpathlineto{\pgfqpoint{4.786158in}{0.424445in}}%
\pgfpathlineto{\pgfqpoint{4.803312in}{0.424445in}}%
\pgfpathlineto{\pgfqpoint{4.820466in}{0.424445in}}%
\pgfpathlineto{\pgfqpoint{4.837619in}{0.424445in}}%
\pgfpathlineto{\pgfqpoint{4.854773in}{0.424445in}}%
\pgfpathlineto{\pgfqpoint{4.871927in}{0.424445in}}%
\pgfpathlineto{\pgfqpoint{4.889080in}{0.424445in}}%
\pgfpathlineto{\pgfqpoint{4.906234in}{0.424445in}}%
\pgfpathlineto{\pgfqpoint{4.923388in}{0.424445in}}%
\pgfpathlineto{\pgfqpoint{4.940541in}{0.424445in}}%
\pgfpathlineto{\pgfqpoint{4.957695in}{0.424445in}}%
\pgfpathlineto{\pgfqpoint{4.974849in}{0.424445in}}%
\pgfpathlineto{\pgfqpoint{4.992003in}{0.424445in}}%
\pgfpathlineto{\pgfqpoint{5.009156in}{0.424445in}}%
\pgfpathlineto{\pgfqpoint{5.026310in}{0.424445in}}%
\pgfpathlineto{\pgfqpoint{5.043464in}{0.424445in}}%
\pgfpathlineto{\pgfqpoint{5.060617in}{0.424445in}}%
\pgfpathlineto{\pgfqpoint{5.077771in}{0.424445in}}%
\pgfpathlineto{\pgfqpoint{5.094925in}{0.424445in}}%
\pgfpathlineto{\pgfqpoint{5.112078in}{0.424445in}}%
\pgfpathlineto{\pgfqpoint{5.129232in}{0.424445in}}%
\pgfpathlineto{\pgfqpoint{5.146386in}{0.424445in}}%
\pgfpathlineto{\pgfqpoint{5.163539in}{0.424445in}}%
\pgfpathlineto{\pgfqpoint{5.180693in}{0.424445in}}%
\pgfpathlineto{\pgfqpoint{5.197847in}{0.424445in}}%
\pgfpathlineto{\pgfqpoint{5.215000in}{0.424445in}}%
\pgfusepath{stroke}%
\end{pgfscope}%
\begin{pgfscope}%
\pgfsetrectcap%
\pgfsetmiterjoin%
\pgfsetlinewidth{0.803000pt}%
\definecolor{currentstroke}{rgb}{0.000000,0.000000,0.000000}%
\pgfsetstrokecolor{currentstroke}%
\pgfsetdash{}{0pt}%
\pgfpathmoveto{\pgfqpoint{2.962722in}{0.315889in}}%
\pgfpathlineto{\pgfqpoint{2.962722in}{2.704133in}}%
\pgfusepath{stroke}%
\end{pgfscope}%
\begin{pgfscope}%
\pgfsetrectcap%
\pgfsetmiterjoin%
\pgfsetlinewidth{0.803000pt}%
\definecolor{currentstroke}{rgb}{0.000000,0.000000,0.000000}%
\pgfsetstrokecolor{currentstroke}%
\pgfsetdash{}{0pt}%
\pgfpathmoveto{\pgfqpoint{5.340222in}{0.315889in}}%
\pgfpathlineto{\pgfqpoint{5.340222in}{2.704133in}}%
\pgfusepath{stroke}%
\end{pgfscope}%
\begin{pgfscope}%
\pgfsetrectcap%
\pgfsetmiterjoin%
\pgfsetlinewidth{0.803000pt}%
\definecolor{currentstroke}{rgb}{0.000000,0.000000,0.000000}%
\pgfsetstrokecolor{currentstroke}%
\pgfsetdash{}{0pt}%
\pgfpathmoveto{\pgfqpoint{2.962722in}{0.315889in}}%
\pgfpathlineto{\pgfqpoint{5.340222in}{0.315889in}}%
\pgfusepath{stroke}%
\end{pgfscope}%
\begin{pgfscope}%
\pgfsetrectcap%
\pgfsetmiterjoin%
\pgfsetlinewidth{0.803000pt}%
\definecolor{currentstroke}{rgb}{0.000000,0.000000,0.000000}%
\pgfsetstrokecolor{currentstroke}%
\pgfsetdash{}{0pt}%
\pgfpathmoveto{\pgfqpoint{2.962722in}{2.704133in}}%
\pgfpathlineto{\pgfqpoint{5.340222in}{2.704133in}}%
\pgfusepath{stroke}%
\end{pgfscope}%
\begin{pgfscope}%
\definecolor{textcolor}{rgb}{0.000000,0.000000,0.000000}%
\pgfsetstrokecolor{textcolor}%
\pgfsetfillcolor{textcolor}%
\pgftext[x=4.151472in,y=2.787467in,,base]{\color{textcolor}\rmfamily\fontsize{9.600000}{11.520000}\selectfont db3}%
\end{pgfscope}%
\end{pgfpicture}%
\makeatother%
\endgroup%

    \caption{Analyse mit db2/3 Wavelet\label{polynomials:db2_3}}
\end{figure}

Wie sich herausstellt liefern diese Wavelets jeweils die zweite und dritte
Ableitung unserer Signale. Daubechies Wavelets mit $A$ verschwindenden Momenten
können uns also direkt die $A$te Ableitung liefern. Dies im Gegensatz zum
Differenzieren welches mehrfach angewendet werden muss.

\section{Anwendung zur rauscharmen Ableitung}
\rhead{Rauscharme Ableitung}

Beim diskreten Ableiten (Differenzen) von Signalen stellt sich oftmals das
Problem, das Rauschen in dem Signal verstärkt wird. Anstelle der Ableitung des
interessanten Signals sehen wir also nur die zufällige Ableitung des Rauschens.
Dies wird nochmals verstärkt wenn ein Signal mehrfach abgeleitet werden soll.

\autoref{polynomials:noise:signals} zeigt unsere ursprünglichen polynomialen
Signale mit leichtem Rauschen versehen und die Ableitung davon. Wie erwartet
ist die Ableitung vom Rauschen dominiert und die Steigung des ursprünglichen
Polynoms ist kaum noch erkennbar.

\begin{figure}
    \centering
    %% Creator: Matplotlib, PGF backend
%%
%% To include the figure in your LaTeX document, write
%%   \input{<filename>.pgf}
%%
%% Make sure the required packages are loaded in your preamble
%%   \usepackage{pgf}
%%
%% Figures using additional raster images can only be included by \input if
%% they are in the same directory as the main LaTeX file. For loading figures
%% from other directories you can use the `import` package
%%   \usepackage{import}
%% and then include the figures with
%%   \import{<path to file>}{<filename>.pgf}
%%
%% Matplotlib used the following preamble
%%   \usepackage{fontspec}
%%
\begingroup%
\makeatletter%
\begin{pgfpicture}%
\pgfpathrectangle{\pgfpointorigin}{\pgfqpoint{5.800000in}{3.000000in}}%
\pgfusepath{use as bounding box, clip}%
\begin{pgfscope}%
\pgfsetbuttcap%
\pgfsetmiterjoin%
\definecolor{currentfill}{rgb}{1.000000,1.000000,1.000000}%
\pgfsetfillcolor{currentfill}%
\pgfsetlinewidth{0.000000pt}%
\definecolor{currentstroke}{rgb}{1.000000,1.000000,1.000000}%
\pgfsetstrokecolor{currentstroke}%
\pgfsetdash{}{0pt}%
\pgfpathmoveto{\pgfqpoint{0.000000in}{0.000000in}}%
\pgfpathlineto{\pgfqpoint{5.800000in}{0.000000in}}%
\pgfpathlineto{\pgfqpoint{5.800000in}{3.000000in}}%
\pgfpathlineto{\pgfqpoint{0.000000in}{3.000000in}}%
\pgfpathclose%
\pgfusepath{fill}%
\end{pgfscope}%
\begin{pgfscope}%
\pgfsetbuttcap%
\pgfsetmiterjoin%
\definecolor{currentfill}{rgb}{1.000000,1.000000,1.000000}%
\pgfsetfillcolor{currentfill}%
\pgfsetlinewidth{0.000000pt}%
\definecolor{currentstroke}{rgb}{0.000000,0.000000,0.000000}%
\pgfsetstrokecolor{currentstroke}%
\pgfsetstrokeopacity{0.000000}%
\pgfsetdash{}{0pt}%
\pgfpathmoveto{\pgfqpoint{0.335764in}{0.386111in}}%
\pgfpathlineto{\pgfqpoint{2.564583in}{0.386111in}}%
\pgfpathlineto{\pgfqpoint{2.564583in}{2.801389in}}%
\pgfpathlineto{\pgfqpoint{0.335764in}{2.801389in}}%
\pgfpathclose%
\pgfusepath{fill}%
\end{pgfscope}%
\begin{pgfscope}%
\pgfsetbuttcap%
\pgfsetroundjoin%
\definecolor{currentfill}{rgb}{0.000000,0.000000,0.000000}%
\pgfsetfillcolor{currentfill}%
\pgfsetlinewidth{0.803000pt}%
\definecolor{currentstroke}{rgb}{0.000000,0.000000,0.000000}%
\pgfsetstrokecolor{currentstroke}%
\pgfsetdash{}{0pt}%
\pgfsys@defobject{currentmarker}{\pgfqpoint{0.000000in}{-0.048611in}}{\pgfqpoint{0.000000in}{0.000000in}}{%
\pgfpathmoveto{\pgfqpoint{0.000000in}{0.000000in}}%
\pgfpathlineto{\pgfqpoint{0.000000in}{-0.048611in}}%
\pgfusepath{stroke,fill}%
}%
\begin{pgfscope}%
\pgfsys@transformshift{0.437074in}{0.386111in}%
\pgfsys@useobject{currentmarker}{}%
\end{pgfscope}%
\end{pgfscope}%
\begin{pgfscope}%
\definecolor{textcolor}{rgb}{0.000000,0.000000,0.000000}%
\pgfsetstrokecolor{textcolor}%
\pgfsetfillcolor{textcolor}%
\pgftext[x=0.437074in,y=0.288889in,,top]{\color{textcolor}\rmfamily\fontsize{10.000000}{12.000000}\selectfont 0}%
\end{pgfscope}%
\begin{pgfscope}%
\pgfsetbuttcap%
\pgfsetroundjoin%
\definecolor{currentfill}{rgb}{0.000000,0.000000,0.000000}%
\pgfsetfillcolor{currentfill}%
\pgfsetlinewidth{0.803000pt}%
\definecolor{currentstroke}{rgb}{0.000000,0.000000,0.000000}%
\pgfsetstrokecolor{currentstroke}%
\pgfsetdash{}{0pt}%
\pgfsys@defobject{currentmarker}{\pgfqpoint{0.000000in}{-0.048611in}}{\pgfqpoint{0.000000in}{0.000000in}}{%
\pgfpathmoveto{\pgfqpoint{0.000000in}{0.000000in}}%
\pgfpathlineto{\pgfqpoint{0.000000in}{-0.048611in}}%
\pgfusepath{stroke,fill}%
}%
\begin{pgfscope}%
\pgfsys@transformshift{1.231662in}{0.386111in}%
\pgfsys@useobject{currentmarker}{}%
\end{pgfscope}%
\end{pgfscope}%
\begin{pgfscope}%
\definecolor{textcolor}{rgb}{0.000000,0.000000,0.000000}%
\pgfsetstrokecolor{textcolor}%
\pgfsetfillcolor{textcolor}%
\pgftext[x=1.231662in,y=0.288889in,,top]{\color{textcolor}\rmfamily\fontsize{10.000000}{12.000000}\selectfont 100}%
\end{pgfscope}%
\begin{pgfscope}%
\pgfsetbuttcap%
\pgfsetroundjoin%
\definecolor{currentfill}{rgb}{0.000000,0.000000,0.000000}%
\pgfsetfillcolor{currentfill}%
\pgfsetlinewidth{0.803000pt}%
\definecolor{currentstroke}{rgb}{0.000000,0.000000,0.000000}%
\pgfsetstrokecolor{currentstroke}%
\pgfsetdash{}{0pt}%
\pgfsys@defobject{currentmarker}{\pgfqpoint{0.000000in}{-0.048611in}}{\pgfqpoint{0.000000in}{0.000000in}}{%
\pgfpathmoveto{\pgfqpoint{0.000000in}{0.000000in}}%
\pgfpathlineto{\pgfqpoint{0.000000in}{-0.048611in}}%
\pgfusepath{stroke,fill}%
}%
\begin{pgfscope}%
\pgfsys@transformshift{2.026250in}{0.386111in}%
\pgfsys@useobject{currentmarker}{}%
\end{pgfscope}%
\end{pgfscope}%
\begin{pgfscope}%
\definecolor{textcolor}{rgb}{0.000000,0.000000,0.000000}%
\pgfsetstrokecolor{textcolor}%
\pgfsetfillcolor{textcolor}%
\pgftext[x=2.026250in,y=0.288889in,,top]{\color{textcolor}\rmfamily\fontsize{10.000000}{12.000000}\selectfont 200}%
\end{pgfscope}%
\begin{pgfscope}%
\pgfsetbuttcap%
\pgfsetroundjoin%
\definecolor{currentfill}{rgb}{0.000000,0.000000,0.000000}%
\pgfsetfillcolor{currentfill}%
\pgfsetlinewidth{0.803000pt}%
\definecolor{currentstroke}{rgb}{0.000000,0.000000,0.000000}%
\pgfsetstrokecolor{currentstroke}%
\pgfsetdash{}{0pt}%
\pgfsys@defobject{currentmarker}{\pgfqpoint{-0.048611in}{0.000000in}}{\pgfqpoint{0.000000in}{0.000000in}}{%
\pgfpathmoveto{\pgfqpoint{0.000000in}{0.000000in}}%
\pgfpathlineto{\pgfqpoint{-0.048611in}{0.000000in}}%
\pgfusepath{stroke,fill}%
}%
\begin{pgfscope}%
\pgfsys@transformshift{0.335764in}{0.494597in}%
\pgfsys@useobject{currentmarker}{}%
\end{pgfscope}%
\end{pgfscope}%
\begin{pgfscope}%
\definecolor{textcolor}{rgb}{0.000000,0.000000,0.000000}%
\pgfsetstrokecolor{textcolor}%
\pgfsetfillcolor{textcolor}%
\pgftext[x=0.169097in,y=0.446402in,left,base]{\color{textcolor}\rmfamily\fontsize{10.000000}{12.000000}\selectfont 0}%
\end{pgfscope}%
\begin{pgfscope}%
\pgfsetbuttcap%
\pgfsetroundjoin%
\definecolor{currentfill}{rgb}{0.000000,0.000000,0.000000}%
\pgfsetfillcolor{currentfill}%
\pgfsetlinewidth{0.803000pt}%
\definecolor{currentstroke}{rgb}{0.000000,0.000000,0.000000}%
\pgfsetstrokecolor{currentstroke}%
\pgfsetdash{}{0pt}%
\pgfsys@defobject{currentmarker}{\pgfqpoint{-0.048611in}{0.000000in}}{\pgfqpoint{0.000000in}{0.000000in}}{%
\pgfpathmoveto{\pgfqpoint{0.000000in}{0.000000in}}%
\pgfpathlineto{\pgfqpoint{-0.048611in}{0.000000in}}%
\pgfusepath{stroke,fill}%
}%
\begin{pgfscope}%
\pgfsys@transformshift{0.335764in}{1.041521in}%
\pgfsys@useobject{currentmarker}{}%
\end{pgfscope}%
\end{pgfscope}%
\begin{pgfscope}%
\definecolor{textcolor}{rgb}{0.000000,0.000000,0.000000}%
\pgfsetstrokecolor{textcolor}%
\pgfsetfillcolor{textcolor}%
\pgftext[x=0.169097in,y=0.993326in,left,base]{\color{textcolor}\rmfamily\fontsize{10.000000}{12.000000}\selectfont 2}%
\end{pgfscope}%
\begin{pgfscope}%
\pgfsetbuttcap%
\pgfsetroundjoin%
\definecolor{currentfill}{rgb}{0.000000,0.000000,0.000000}%
\pgfsetfillcolor{currentfill}%
\pgfsetlinewidth{0.803000pt}%
\definecolor{currentstroke}{rgb}{0.000000,0.000000,0.000000}%
\pgfsetstrokecolor{currentstroke}%
\pgfsetdash{}{0pt}%
\pgfsys@defobject{currentmarker}{\pgfqpoint{-0.048611in}{0.000000in}}{\pgfqpoint{0.000000in}{0.000000in}}{%
\pgfpathmoveto{\pgfqpoint{0.000000in}{0.000000in}}%
\pgfpathlineto{\pgfqpoint{-0.048611in}{0.000000in}}%
\pgfusepath{stroke,fill}%
}%
\begin{pgfscope}%
\pgfsys@transformshift{0.335764in}{1.588445in}%
\pgfsys@useobject{currentmarker}{}%
\end{pgfscope}%
\end{pgfscope}%
\begin{pgfscope}%
\definecolor{textcolor}{rgb}{0.000000,0.000000,0.000000}%
\pgfsetstrokecolor{textcolor}%
\pgfsetfillcolor{textcolor}%
\pgftext[x=0.169097in,y=1.540251in,left,base]{\color{textcolor}\rmfamily\fontsize{10.000000}{12.000000}\selectfont 4}%
\end{pgfscope}%
\begin{pgfscope}%
\pgfsetbuttcap%
\pgfsetroundjoin%
\definecolor{currentfill}{rgb}{0.000000,0.000000,0.000000}%
\pgfsetfillcolor{currentfill}%
\pgfsetlinewidth{0.803000pt}%
\definecolor{currentstroke}{rgb}{0.000000,0.000000,0.000000}%
\pgfsetstrokecolor{currentstroke}%
\pgfsetdash{}{0pt}%
\pgfsys@defobject{currentmarker}{\pgfqpoint{-0.048611in}{0.000000in}}{\pgfqpoint{0.000000in}{0.000000in}}{%
\pgfpathmoveto{\pgfqpoint{0.000000in}{0.000000in}}%
\pgfpathlineto{\pgfqpoint{-0.048611in}{0.000000in}}%
\pgfusepath{stroke,fill}%
}%
\begin{pgfscope}%
\pgfsys@transformshift{0.335764in}{2.135369in}%
\pgfsys@useobject{currentmarker}{}%
\end{pgfscope}%
\end{pgfscope}%
\begin{pgfscope}%
\definecolor{textcolor}{rgb}{0.000000,0.000000,0.000000}%
\pgfsetstrokecolor{textcolor}%
\pgfsetfillcolor{textcolor}%
\pgftext[x=0.169097in,y=2.087175in,left,base]{\color{textcolor}\rmfamily\fontsize{10.000000}{12.000000}\selectfont 6}%
\end{pgfscope}%
\begin{pgfscope}%
\pgfsetbuttcap%
\pgfsetroundjoin%
\definecolor{currentfill}{rgb}{0.000000,0.000000,0.000000}%
\pgfsetfillcolor{currentfill}%
\pgfsetlinewidth{0.803000pt}%
\definecolor{currentstroke}{rgb}{0.000000,0.000000,0.000000}%
\pgfsetstrokecolor{currentstroke}%
\pgfsetdash{}{0pt}%
\pgfsys@defobject{currentmarker}{\pgfqpoint{-0.048611in}{0.000000in}}{\pgfqpoint{0.000000in}{0.000000in}}{%
\pgfpathmoveto{\pgfqpoint{0.000000in}{0.000000in}}%
\pgfpathlineto{\pgfqpoint{-0.048611in}{0.000000in}}%
\pgfusepath{stroke,fill}%
}%
\begin{pgfscope}%
\pgfsys@transformshift{0.335764in}{2.682294in}%
\pgfsys@useobject{currentmarker}{}%
\end{pgfscope}%
\end{pgfscope}%
\begin{pgfscope}%
\definecolor{textcolor}{rgb}{0.000000,0.000000,0.000000}%
\pgfsetstrokecolor{textcolor}%
\pgfsetfillcolor{textcolor}%
\pgftext[x=0.169097in,y=2.634099in,left,base]{\color{textcolor}\rmfamily\fontsize{10.000000}{12.000000}\selectfont 8}%
\end{pgfscope}%
\begin{pgfscope}%
\pgfpathrectangle{\pgfqpoint{0.335764in}{0.386111in}}{\pgfqpoint{2.228819in}{2.415278in}}%
\pgfusepath{clip}%
\pgfsetrectcap%
\pgfsetroundjoin%
\pgfsetlinewidth{1.505625pt}%
\definecolor{currentstroke}{rgb}{0.121569,0.466667,0.705882}%
\pgfsetstrokecolor{currentstroke}%
\pgfsetdash{}{0pt}%
\pgfpathmoveto{\pgfqpoint{0.437074in}{0.768863in}}%
\pgfpathlineto{\pgfqpoint{0.445020in}{0.780350in}}%
\pgfpathlineto{\pgfqpoint{0.452966in}{0.769356in}}%
\pgfpathlineto{\pgfqpoint{0.460912in}{0.779350in}}%
\pgfpathlineto{\pgfqpoint{0.468857in}{0.780307in}}%
\pgfpathlineto{\pgfqpoint{0.476803in}{0.778771in}}%
\pgfpathlineto{\pgfqpoint{0.484749in}{0.769810in}}%
\pgfpathlineto{\pgfqpoint{0.500641in}{0.773172in}}%
\pgfpathlineto{\pgfqpoint{0.508587in}{0.780085in}}%
\pgfpathlineto{\pgfqpoint{0.516533in}{0.780133in}}%
\pgfpathlineto{\pgfqpoint{0.524479in}{0.769800in}}%
\pgfpathlineto{\pgfqpoint{0.532424in}{0.781621in}}%
\pgfpathlineto{\pgfqpoint{0.540370in}{0.770754in}}%
\pgfpathlineto{\pgfqpoint{0.548316in}{0.778316in}}%
\pgfpathlineto{\pgfqpoint{0.556262in}{0.768718in}}%
\pgfpathlineto{\pgfqpoint{0.564208in}{0.769613in}}%
\pgfpathlineto{\pgfqpoint{0.572154in}{0.778770in}}%
\pgfpathlineto{\pgfqpoint{0.580100in}{0.780683in}}%
\pgfpathlineto{\pgfqpoint{0.588046in}{0.778070in}}%
\pgfpathlineto{\pgfqpoint{0.595991in}{0.768908in}}%
\pgfpathlineto{\pgfqpoint{0.603937in}{0.769362in}}%
\pgfpathlineto{\pgfqpoint{0.611883in}{0.780824in}}%
\pgfpathlineto{\pgfqpoint{0.627775in}{0.772964in}}%
\pgfpathlineto{\pgfqpoint{0.635721in}{0.771185in}}%
\pgfpathlineto{\pgfqpoint{0.643667in}{0.780832in}}%
\pgfpathlineto{\pgfqpoint{0.651613in}{0.768209in}}%
\pgfpathlineto{\pgfqpoint{0.659559in}{0.768876in}}%
\pgfpathlineto{\pgfqpoint{0.667504in}{0.771814in}}%
\pgfpathlineto{\pgfqpoint{0.675450in}{0.772788in}}%
\pgfpathlineto{\pgfqpoint{0.683396in}{0.777177in}}%
\pgfpathlineto{\pgfqpoint{0.707234in}{0.773931in}}%
\pgfpathlineto{\pgfqpoint{0.715180in}{0.771441in}}%
\pgfpathlineto{\pgfqpoint{0.723126in}{0.771521in}}%
\pgfpathlineto{\pgfqpoint{0.731071in}{0.776220in}}%
\pgfpathlineto{\pgfqpoint{0.746963in}{0.769665in}}%
\pgfpathlineto{\pgfqpoint{0.754909in}{0.775959in}}%
\pgfpathlineto{\pgfqpoint{0.762855in}{0.774822in}}%
\pgfpathlineto{\pgfqpoint{0.770801in}{0.770784in}}%
\pgfpathlineto{\pgfqpoint{0.778747in}{0.769414in}}%
\pgfpathlineto{\pgfqpoint{0.786693in}{0.778370in}}%
\pgfpathlineto{\pgfqpoint{0.794638in}{0.771526in}}%
\pgfpathlineto{\pgfqpoint{0.802584in}{0.774766in}}%
\pgfpathlineto{\pgfqpoint{0.810530in}{0.768153in}}%
\pgfpathlineto{\pgfqpoint{0.818476in}{0.771459in}}%
\pgfpathlineto{\pgfqpoint{0.826422in}{0.768855in}}%
\pgfpathlineto{\pgfqpoint{0.834368in}{0.776647in}}%
\pgfpathlineto{\pgfqpoint{0.842314in}{0.774133in}}%
\pgfpathlineto{\pgfqpoint{0.850260in}{0.776105in}}%
\pgfpathlineto{\pgfqpoint{0.858206in}{0.769413in}}%
\pgfpathlineto{\pgfqpoint{0.866151in}{0.771794in}}%
\pgfpathlineto{\pgfqpoint{0.882043in}{0.772344in}}%
\pgfpathlineto{\pgfqpoint{0.889989in}{0.779085in}}%
\pgfpathlineto{\pgfqpoint{0.897935in}{0.780895in}}%
\pgfpathlineto{\pgfqpoint{0.921773in}{0.770321in}}%
\pgfpathlineto{\pgfqpoint{0.929718in}{0.777114in}}%
\pgfpathlineto{\pgfqpoint{0.937664in}{0.774587in}}%
\pgfpathlineto{\pgfqpoint{0.945610in}{0.778201in}}%
\pgfpathlineto{\pgfqpoint{0.961502in}{0.770672in}}%
\pgfpathlineto{\pgfqpoint{0.969448in}{0.780652in}}%
\pgfpathlineto{\pgfqpoint{0.977394in}{0.773762in}}%
\pgfpathlineto{\pgfqpoint{0.985340in}{0.771160in}}%
\pgfpathlineto{\pgfqpoint{0.993285in}{0.780300in}}%
\pgfpathlineto{\pgfqpoint{1.001231in}{0.779703in}}%
\pgfpathlineto{\pgfqpoint{1.009177in}{0.777369in}}%
\pgfpathlineto{\pgfqpoint{1.017123in}{0.781529in}}%
\pgfpathlineto{\pgfqpoint{1.033015in}{0.778651in}}%
\pgfpathlineto{\pgfqpoint{1.040961in}{0.772633in}}%
\pgfpathlineto{\pgfqpoint{1.048907in}{0.779118in}}%
\pgfpathlineto{\pgfqpoint{1.064798in}{0.775212in}}%
\pgfpathlineto{\pgfqpoint{1.080690in}{0.772247in}}%
\pgfpathlineto{\pgfqpoint{1.088636in}{0.780515in}}%
\pgfpathlineto{\pgfqpoint{1.096582in}{0.777412in}}%
\pgfpathlineto{\pgfqpoint{1.104528in}{0.768560in}}%
\pgfpathlineto{\pgfqpoint{1.112474in}{0.777480in}}%
\pgfpathlineto{\pgfqpoint{1.120420in}{0.777047in}}%
\pgfpathlineto{\pgfqpoint{1.128365in}{0.771291in}}%
\pgfpathlineto{\pgfqpoint{1.136311in}{0.779905in}}%
\pgfpathlineto{\pgfqpoint{1.144257in}{0.775600in}}%
\pgfpathlineto{\pgfqpoint{1.152203in}{0.769091in}}%
\pgfpathlineto{\pgfqpoint{1.160149in}{0.777470in}}%
\pgfpathlineto{\pgfqpoint{1.176041in}{0.773817in}}%
\pgfpathlineto{\pgfqpoint{1.183987in}{0.774536in}}%
\pgfpathlineto{\pgfqpoint{1.191932in}{0.768605in}}%
\pgfpathlineto{\pgfqpoint{1.199878in}{0.768799in}}%
\pgfpathlineto{\pgfqpoint{1.207824in}{0.781698in}}%
\pgfpathlineto{\pgfqpoint{1.215770in}{0.771385in}}%
\pgfpathlineto{\pgfqpoint{1.223716in}{0.770916in}}%
\pgfpathlineto{\pgfqpoint{1.231662in}{0.777357in}}%
\pgfpathlineto{\pgfqpoint{1.239608in}{0.768407in}}%
\pgfpathlineto{\pgfqpoint{1.255500in}{0.777513in}}%
\pgfpathlineto{\pgfqpoint{1.263445in}{0.775425in}}%
\pgfpathlineto{\pgfqpoint{1.271391in}{0.777338in}}%
\pgfpathlineto{\pgfqpoint{1.287283in}{0.770375in}}%
\pgfpathlineto{\pgfqpoint{1.303175in}{0.772760in}}%
\pgfpathlineto{\pgfqpoint{1.311121in}{0.779919in}}%
\pgfpathlineto{\pgfqpoint{1.327012in}{0.768981in}}%
\pgfpathlineto{\pgfqpoint{1.334958in}{0.780225in}}%
\pgfpathlineto{\pgfqpoint{1.342904in}{0.774102in}}%
\pgfpathlineto{\pgfqpoint{1.350850in}{0.775475in}}%
\pgfpathlineto{\pgfqpoint{1.358796in}{0.770037in}}%
\pgfpathlineto{\pgfqpoint{1.366742in}{0.771712in}}%
\pgfpathlineto{\pgfqpoint{1.374688in}{0.776192in}}%
\pgfpathlineto{\pgfqpoint{1.382634in}{0.775474in}}%
\pgfpathlineto{\pgfqpoint{1.390580in}{0.776827in}}%
\pgfpathlineto{\pgfqpoint{1.398525in}{0.769370in}}%
\pgfpathlineto{\pgfqpoint{1.406471in}{0.769281in}}%
\pgfpathlineto{\pgfqpoint{1.414417in}{0.778992in}}%
\pgfpathlineto{\pgfqpoint{1.422363in}{0.778332in}}%
\pgfpathlineto{\pgfqpoint{1.430309in}{0.772071in}}%
\pgfpathlineto{\pgfqpoint{1.438255in}{0.780509in}}%
\pgfpathlineto{\pgfqpoint{1.446201in}{0.780285in}}%
\pgfpathlineto{\pgfqpoint{1.454147in}{0.771153in}}%
\pgfpathlineto{\pgfqpoint{1.462092in}{0.778993in}}%
\pgfpathlineto{\pgfqpoint{1.470038in}{0.781265in}}%
\pgfpathlineto{\pgfqpoint{1.477984in}{0.773656in}}%
\pgfpathlineto{\pgfqpoint{1.485930in}{0.771942in}}%
\pgfpathlineto{\pgfqpoint{1.493876in}{0.777950in}}%
\pgfpathlineto{\pgfqpoint{1.501822in}{0.775969in}}%
\pgfpathlineto{\pgfqpoint{1.509768in}{0.768895in}}%
\pgfpathlineto{\pgfqpoint{1.517714in}{0.776007in}}%
\pgfpathlineto{\pgfqpoint{1.525659in}{0.768887in}}%
\pgfpathlineto{\pgfqpoint{1.549497in}{0.772319in}}%
\pgfpathlineto{\pgfqpoint{1.557443in}{0.780825in}}%
\pgfpathlineto{\pgfqpoint{1.565389in}{0.773473in}}%
\pgfpathlineto{\pgfqpoint{1.573335in}{0.769813in}}%
\pgfpathlineto{\pgfqpoint{1.581281in}{0.772735in}}%
\pgfpathlineto{\pgfqpoint{1.589227in}{0.769331in}}%
\pgfpathlineto{\pgfqpoint{1.605118in}{0.775201in}}%
\pgfpathlineto{\pgfqpoint{1.613064in}{0.773954in}}%
\pgfpathlineto{\pgfqpoint{1.621010in}{0.778300in}}%
\pgfpathlineto{\pgfqpoint{1.628956in}{0.775973in}}%
\pgfpathlineto{\pgfqpoint{1.636902in}{0.780927in}}%
\pgfpathlineto{\pgfqpoint{1.644848in}{0.775267in}}%
\pgfpathlineto{\pgfqpoint{1.652794in}{0.772600in}}%
\pgfpathlineto{\pgfqpoint{1.660739in}{0.780124in}}%
\pgfpathlineto{\pgfqpoint{1.668685in}{0.779119in}}%
\pgfpathlineto{\pgfqpoint{1.676631in}{0.773186in}}%
\pgfpathlineto{\pgfqpoint{1.684577in}{0.780216in}}%
\pgfpathlineto{\pgfqpoint{1.692523in}{0.778463in}}%
\pgfpathlineto{\pgfqpoint{1.700469in}{0.779771in}}%
\pgfpathlineto{\pgfqpoint{1.708415in}{0.770766in}}%
\pgfpathlineto{\pgfqpoint{1.716361in}{0.780446in}}%
\pgfpathlineto{\pgfqpoint{1.724306in}{0.774949in}}%
\pgfpathlineto{\pgfqpoint{1.732252in}{0.777287in}}%
\pgfpathlineto{\pgfqpoint{1.740198in}{0.769270in}}%
\pgfpathlineto{\pgfqpoint{1.748144in}{0.775052in}}%
\pgfpathlineto{\pgfqpoint{1.756090in}{0.768209in}}%
\pgfpathlineto{\pgfqpoint{1.764036in}{0.775843in}}%
\pgfpathlineto{\pgfqpoint{1.771982in}{0.777597in}}%
\pgfpathlineto{\pgfqpoint{1.779928in}{0.775397in}}%
\pgfpathlineto{\pgfqpoint{1.787874in}{0.771588in}}%
\pgfpathlineto{\pgfqpoint{1.819657in}{0.769536in}}%
\pgfpathlineto{\pgfqpoint{1.827603in}{0.780605in}}%
\pgfpathlineto{\pgfqpoint{1.835549in}{0.773265in}}%
\pgfpathlineto{\pgfqpoint{1.843495in}{0.769855in}}%
\pgfpathlineto{\pgfqpoint{1.851441in}{0.769128in}}%
\pgfpathlineto{\pgfqpoint{1.859386in}{0.771685in}}%
\pgfpathlineto{\pgfqpoint{1.875278in}{0.772391in}}%
\pgfpathlineto{\pgfqpoint{1.883224in}{0.777735in}}%
\pgfpathlineto{\pgfqpoint{1.891170in}{0.776198in}}%
\pgfpathlineto{\pgfqpoint{1.899116in}{0.771586in}}%
\pgfpathlineto{\pgfqpoint{1.907062in}{0.773732in}}%
\pgfpathlineto{\pgfqpoint{1.915008in}{0.771687in}}%
\pgfpathlineto{\pgfqpoint{1.922953in}{0.778583in}}%
\pgfpathlineto{\pgfqpoint{1.930899in}{0.780794in}}%
\pgfpathlineto{\pgfqpoint{1.938845in}{0.772991in}}%
\pgfpathlineto{\pgfqpoint{1.946791in}{0.779400in}}%
\pgfpathlineto{\pgfqpoint{1.954737in}{0.781355in}}%
\pgfpathlineto{\pgfqpoint{1.962683in}{0.773141in}}%
\pgfpathlineto{\pgfqpoint{1.970629in}{0.781178in}}%
\pgfpathlineto{\pgfqpoint{1.978575in}{0.771767in}}%
\pgfpathlineto{\pgfqpoint{1.986521in}{0.779812in}}%
\pgfpathlineto{\pgfqpoint{1.994466in}{0.780135in}}%
\pgfpathlineto{\pgfqpoint{2.002412in}{0.771510in}}%
\pgfpathlineto{\pgfqpoint{2.010358in}{0.778715in}}%
\pgfpathlineto{\pgfqpoint{2.018304in}{0.780698in}}%
\pgfpathlineto{\pgfqpoint{2.026250in}{0.774536in}}%
\pgfpathlineto{\pgfqpoint{2.042142in}{0.774887in}}%
\pgfpathlineto{\pgfqpoint{2.050088in}{0.768869in}}%
\pgfpathlineto{\pgfqpoint{2.058033in}{0.771600in}}%
\pgfpathlineto{\pgfqpoint{2.073925in}{0.774134in}}%
\pgfpathlineto{\pgfqpoint{2.081871in}{0.772494in}}%
\pgfpathlineto{\pgfqpoint{2.097763in}{0.778893in}}%
\pgfpathlineto{\pgfqpoint{2.105709in}{0.771394in}}%
\pgfpathlineto{\pgfqpoint{2.113655in}{0.770268in}}%
\pgfpathlineto{\pgfqpoint{2.121601in}{0.779067in}}%
\pgfpathlineto{\pgfqpoint{2.129546in}{0.781362in}}%
\pgfpathlineto{\pgfqpoint{2.137492in}{0.774364in}}%
\pgfpathlineto{\pgfqpoint{2.145438in}{0.781000in}}%
\pgfpathlineto{\pgfqpoint{2.153384in}{0.768589in}}%
\pgfpathlineto{\pgfqpoint{2.161330in}{0.781686in}}%
\pgfpathlineto{\pgfqpoint{2.169276in}{0.770874in}}%
\pgfpathlineto{\pgfqpoint{2.177222in}{0.768257in}}%
\pgfpathlineto{\pgfqpoint{2.193113in}{0.776812in}}%
\pgfpathlineto{\pgfqpoint{2.201059in}{0.771977in}}%
\pgfpathlineto{\pgfqpoint{2.209005in}{0.772521in}}%
\pgfpathlineto{\pgfqpoint{2.216951in}{0.770015in}}%
\pgfpathlineto{\pgfqpoint{2.224897in}{0.771314in}}%
\pgfpathlineto{\pgfqpoint{2.232843in}{0.769783in}}%
\pgfpathlineto{\pgfqpoint{2.240789in}{0.773929in}}%
\pgfpathlineto{\pgfqpoint{2.248735in}{0.770224in}}%
\pgfpathlineto{\pgfqpoint{2.256680in}{0.769628in}}%
\pgfpathlineto{\pgfqpoint{2.264626in}{0.777994in}}%
\pgfpathlineto{\pgfqpoint{2.272572in}{0.769313in}}%
\pgfpathlineto{\pgfqpoint{2.280518in}{0.771205in}}%
\pgfpathlineto{\pgfqpoint{2.288464in}{0.778269in}}%
\pgfpathlineto{\pgfqpoint{2.296410in}{0.772718in}}%
\pgfpathlineto{\pgfqpoint{2.304356in}{0.775703in}}%
\pgfpathlineto{\pgfqpoint{2.312302in}{0.772974in}}%
\pgfpathlineto{\pgfqpoint{2.320248in}{0.774392in}}%
\pgfpathlineto{\pgfqpoint{2.328193in}{0.777408in}}%
\pgfpathlineto{\pgfqpoint{2.336139in}{0.773174in}}%
\pgfpathlineto{\pgfqpoint{2.344085in}{0.770790in}}%
\pgfpathlineto{\pgfqpoint{2.352031in}{0.781056in}}%
\pgfpathlineto{\pgfqpoint{2.359977in}{0.779605in}}%
\pgfpathlineto{\pgfqpoint{2.367923in}{0.771458in}}%
\pgfpathlineto{\pgfqpoint{2.375869in}{0.778287in}}%
\pgfpathlineto{\pgfqpoint{2.383815in}{0.774797in}}%
\pgfpathlineto{\pgfqpoint{2.399706in}{0.773975in}}%
\pgfpathlineto{\pgfqpoint{2.407652in}{0.779606in}}%
\pgfpathlineto{\pgfqpoint{2.415598in}{0.772629in}}%
\pgfpathlineto{\pgfqpoint{2.423544in}{0.769872in}}%
\pgfpathlineto{\pgfqpoint{2.431490in}{0.781149in}}%
\pgfpathlineto{\pgfqpoint{2.439436in}{0.780548in}}%
\pgfpathlineto{\pgfqpoint{2.447382in}{0.776370in}}%
\pgfpathlineto{\pgfqpoint{2.455327in}{0.776345in}}%
\pgfpathlineto{\pgfqpoint{2.463273in}{0.778417in}}%
\pgfpathlineto{\pgfqpoint{2.463273in}{0.778417in}}%
\pgfusepath{stroke}%
\end{pgfscope}%
\begin{pgfscope}%
\pgfpathrectangle{\pgfqpoint{0.335764in}{0.386111in}}{\pgfqpoint{2.228819in}{2.415278in}}%
\pgfusepath{clip}%
\pgfsetrectcap%
\pgfsetroundjoin%
\pgfsetlinewidth{1.505625pt}%
\definecolor{currentstroke}{rgb}{1.000000,0.498039,0.054902}%
\pgfsetstrokecolor{currentstroke}%
\pgfsetdash{}{0pt}%
\pgfpathmoveto{\pgfqpoint{0.437074in}{0.498315in}}%
\pgfpathlineto{\pgfqpoint{0.445020in}{0.499794in}}%
\pgfpathlineto{\pgfqpoint{0.452966in}{0.507711in}}%
\pgfpathlineto{\pgfqpoint{0.460912in}{0.512841in}}%
\pgfpathlineto{\pgfqpoint{0.476803in}{0.517382in}}%
\pgfpathlineto{\pgfqpoint{0.484749in}{0.519777in}}%
\pgfpathlineto{\pgfqpoint{0.492695in}{0.511550in}}%
\pgfpathlineto{\pgfqpoint{0.500641in}{0.523612in}}%
\pgfpathlineto{\pgfqpoint{0.508587in}{0.526969in}}%
\pgfpathlineto{\pgfqpoint{0.516533in}{0.517852in}}%
\pgfpathlineto{\pgfqpoint{0.524479in}{0.530893in}}%
\pgfpathlineto{\pgfqpoint{0.532424in}{0.520548in}}%
\pgfpathlineto{\pgfqpoint{0.540370in}{0.524606in}}%
\pgfpathlineto{\pgfqpoint{0.548316in}{0.536651in}}%
\pgfpathlineto{\pgfqpoint{0.556262in}{0.531134in}}%
\pgfpathlineto{\pgfqpoint{0.564208in}{0.538357in}}%
\pgfpathlineto{\pgfqpoint{0.572154in}{0.542513in}}%
\pgfpathlineto{\pgfqpoint{0.580100in}{0.538508in}}%
\pgfpathlineto{\pgfqpoint{0.588046in}{0.538358in}}%
\pgfpathlineto{\pgfqpoint{0.603937in}{0.549110in}}%
\pgfpathlineto{\pgfqpoint{0.611883in}{0.545461in}}%
\pgfpathlineto{\pgfqpoint{0.619829in}{0.550035in}}%
\pgfpathlineto{\pgfqpoint{0.627775in}{0.556895in}}%
\pgfpathlineto{\pgfqpoint{0.635721in}{0.560919in}}%
\pgfpathlineto{\pgfqpoint{0.643667in}{0.562933in}}%
\pgfpathlineto{\pgfqpoint{0.651613in}{0.557581in}}%
\pgfpathlineto{\pgfqpoint{0.659559in}{0.561823in}}%
\pgfpathlineto{\pgfqpoint{0.667504in}{0.556847in}}%
\pgfpathlineto{\pgfqpoint{0.675450in}{0.559278in}}%
\pgfpathlineto{\pgfqpoint{0.683396in}{0.570179in}}%
\pgfpathlineto{\pgfqpoint{0.691342in}{0.573555in}}%
\pgfpathlineto{\pgfqpoint{0.699288in}{0.568081in}}%
\pgfpathlineto{\pgfqpoint{0.707234in}{0.576448in}}%
\pgfpathlineto{\pgfqpoint{0.715180in}{0.576401in}}%
\pgfpathlineto{\pgfqpoint{0.723126in}{0.582354in}}%
\pgfpathlineto{\pgfqpoint{0.731071in}{0.577336in}}%
\pgfpathlineto{\pgfqpoint{0.746963in}{0.584561in}}%
\pgfpathlineto{\pgfqpoint{0.754909in}{0.582662in}}%
\pgfpathlineto{\pgfqpoint{0.762855in}{0.585529in}}%
\pgfpathlineto{\pgfqpoint{0.770801in}{0.592593in}}%
\pgfpathlineto{\pgfqpoint{0.778747in}{0.590060in}}%
\pgfpathlineto{\pgfqpoint{0.794638in}{0.596161in}}%
\pgfpathlineto{\pgfqpoint{0.802584in}{0.606127in}}%
\pgfpathlineto{\pgfqpoint{0.810530in}{0.599070in}}%
\pgfpathlineto{\pgfqpoint{0.818476in}{0.610750in}}%
\pgfpathlineto{\pgfqpoint{0.826422in}{0.603963in}}%
\pgfpathlineto{\pgfqpoint{0.834368in}{0.614444in}}%
\pgfpathlineto{\pgfqpoint{0.842314in}{0.605123in}}%
\pgfpathlineto{\pgfqpoint{0.850260in}{0.615371in}}%
\pgfpathlineto{\pgfqpoint{0.858206in}{0.616861in}}%
\pgfpathlineto{\pgfqpoint{0.866151in}{0.614807in}}%
\pgfpathlineto{\pgfqpoint{0.882043in}{0.626959in}}%
\pgfpathlineto{\pgfqpoint{0.897935in}{0.622321in}}%
\pgfpathlineto{\pgfqpoint{0.905881in}{0.623062in}}%
\pgfpathlineto{\pgfqpoint{0.913827in}{0.625673in}}%
\pgfpathlineto{\pgfqpoint{0.921773in}{0.630376in}}%
\pgfpathlineto{\pgfqpoint{0.929718in}{0.631598in}}%
\pgfpathlineto{\pgfqpoint{0.937664in}{0.642229in}}%
\pgfpathlineto{\pgfqpoint{0.945610in}{0.633037in}}%
\pgfpathlineto{\pgfqpoint{0.953556in}{0.646695in}}%
\pgfpathlineto{\pgfqpoint{0.969448in}{0.647298in}}%
\pgfpathlineto{\pgfqpoint{0.977394in}{0.640869in}}%
\pgfpathlineto{\pgfqpoint{0.993285in}{0.648838in}}%
\pgfpathlineto{\pgfqpoint{1.001231in}{0.657752in}}%
\pgfpathlineto{\pgfqpoint{1.009177in}{0.651269in}}%
\pgfpathlineto{\pgfqpoint{1.017123in}{0.651814in}}%
\pgfpathlineto{\pgfqpoint{1.025069in}{0.661444in}}%
\pgfpathlineto{\pgfqpoint{1.033015in}{0.656358in}}%
\pgfpathlineto{\pgfqpoint{1.040961in}{0.665462in}}%
\pgfpathlineto{\pgfqpoint{1.048907in}{0.667608in}}%
\pgfpathlineto{\pgfqpoint{1.064798in}{0.667349in}}%
\pgfpathlineto{\pgfqpoint{1.072744in}{0.673910in}}%
\pgfpathlineto{\pgfqpoint{1.080690in}{0.675767in}}%
\pgfpathlineto{\pgfqpoint{1.088636in}{0.682191in}}%
\pgfpathlineto{\pgfqpoint{1.096582in}{0.674247in}}%
\pgfpathlineto{\pgfqpoint{1.104528in}{0.677571in}}%
\pgfpathlineto{\pgfqpoint{1.120420in}{0.689346in}}%
\pgfpathlineto{\pgfqpoint{1.144257in}{0.692406in}}%
\pgfpathlineto{\pgfqpoint{1.152203in}{0.691404in}}%
\pgfpathlineto{\pgfqpoint{1.160149in}{0.699050in}}%
\pgfpathlineto{\pgfqpoint{1.168095in}{0.697933in}}%
\pgfpathlineto{\pgfqpoint{1.176041in}{0.694590in}}%
\pgfpathlineto{\pgfqpoint{1.183987in}{0.704289in}}%
\pgfpathlineto{\pgfqpoint{1.191932in}{0.700273in}}%
\pgfpathlineto{\pgfqpoint{1.199878in}{0.709085in}}%
\pgfpathlineto{\pgfqpoint{1.207824in}{0.709517in}}%
\pgfpathlineto{\pgfqpoint{1.215770in}{0.705370in}}%
\pgfpathlineto{\pgfqpoint{1.223716in}{0.717343in}}%
\pgfpathlineto{\pgfqpoint{1.231662in}{0.709091in}}%
\pgfpathlineto{\pgfqpoint{1.239608in}{0.711633in}}%
\pgfpathlineto{\pgfqpoint{1.247554in}{0.718143in}}%
\pgfpathlineto{\pgfqpoint{1.255500in}{0.719409in}}%
\pgfpathlineto{\pgfqpoint{1.263445in}{0.730529in}}%
\pgfpathlineto{\pgfqpoint{1.271391in}{0.723358in}}%
\pgfpathlineto{\pgfqpoint{1.287283in}{0.726519in}}%
\pgfpathlineto{\pgfqpoint{1.295229in}{0.726847in}}%
\pgfpathlineto{\pgfqpoint{1.303175in}{0.729154in}}%
\pgfpathlineto{\pgfqpoint{1.311121in}{0.741044in}}%
\pgfpathlineto{\pgfqpoint{1.319067in}{0.739234in}}%
\pgfpathlineto{\pgfqpoint{1.334958in}{0.742611in}}%
\pgfpathlineto{\pgfqpoint{1.342904in}{0.740091in}}%
\pgfpathlineto{\pgfqpoint{1.350850in}{0.745226in}}%
\pgfpathlineto{\pgfqpoint{1.358796in}{0.744324in}}%
\pgfpathlineto{\pgfqpoint{1.366742in}{0.752804in}}%
\pgfpathlineto{\pgfqpoint{1.382634in}{0.753285in}}%
\pgfpathlineto{\pgfqpoint{1.390580in}{0.764468in}}%
\pgfpathlineto{\pgfqpoint{1.398525in}{0.765453in}}%
\pgfpathlineto{\pgfqpoint{1.406471in}{0.761228in}}%
\pgfpathlineto{\pgfqpoint{1.414417in}{0.767134in}}%
\pgfpathlineto{\pgfqpoint{1.422363in}{0.762079in}}%
\pgfpathlineto{\pgfqpoint{1.430309in}{0.766872in}}%
\pgfpathlineto{\pgfqpoint{1.438255in}{0.769159in}}%
\pgfpathlineto{\pgfqpoint{1.446201in}{0.773114in}}%
\pgfpathlineto{\pgfqpoint{1.462092in}{0.773114in}}%
\pgfpathlineto{\pgfqpoint{1.470038in}{0.786329in}}%
\pgfpathlineto{\pgfqpoint{1.477984in}{0.777760in}}%
\pgfpathlineto{\pgfqpoint{1.485930in}{0.790082in}}%
\pgfpathlineto{\pgfqpoint{1.493876in}{0.790530in}}%
\pgfpathlineto{\pgfqpoint{1.501822in}{0.786064in}}%
\pgfpathlineto{\pgfqpoint{1.509768in}{0.793167in}}%
\pgfpathlineto{\pgfqpoint{1.517714in}{0.787797in}}%
\pgfpathlineto{\pgfqpoint{1.525659in}{0.793476in}}%
\pgfpathlineto{\pgfqpoint{1.533605in}{0.801233in}}%
\pgfpathlineto{\pgfqpoint{1.541551in}{0.797463in}}%
\pgfpathlineto{\pgfqpoint{1.557443in}{0.799100in}}%
\pgfpathlineto{\pgfqpoint{1.565389in}{0.801628in}}%
\pgfpathlineto{\pgfqpoint{1.573335in}{0.813567in}}%
\pgfpathlineto{\pgfqpoint{1.581281in}{0.815114in}}%
\pgfpathlineto{\pgfqpoint{1.589227in}{0.814201in}}%
\pgfpathlineto{\pgfqpoint{1.597172in}{0.816972in}}%
\pgfpathlineto{\pgfqpoint{1.605118in}{0.810679in}}%
\pgfpathlineto{\pgfqpoint{1.613064in}{0.819783in}}%
\pgfpathlineto{\pgfqpoint{1.621010in}{0.814850in}}%
\pgfpathlineto{\pgfqpoint{1.628956in}{0.826721in}}%
\pgfpathlineto{\pgfqpoint{1.636902in}{0.820763in}}%
\pgfpathlineto{\pgfqpoint{1.644848in}{0.823946in}}%
\pgfpathlineto{\pgfqpoint{1.652794in}{0.835111in}}%
\pgfpathlineto{\pgfqpoint{1.660739in}{0.833528in}}%
\pgfpathlineto{\pgfqpoint{1.668685in}{0.840339in}}%
\pgfpathlineto{\pgfqpoint{1.676631in}{0.839391in}}%
\pgfpathlineto{\pgfqpoint{1.684577in}{0.843824in}}%
\pgfpathlineto{\pgfqpoint{1.692523in}{0.835010in}}%
\pgfpathlineto{\pgfqpoint{1.700469in}{0.846419in}}%
\pgfpathlineto{\pgfqpoint{1.708415in}{0.838297in}}%
\pgfpathlineto{\pgfqpoint{1.716361in}{0.843213in}}%
\pgfpathlineto{\pgfqpoint{1.724306in}{0.846149in}}%
\pgfpathlineto{\pgfqpoint{1.732252in}{0.851041in}}%
\pgfpathlineto{\pgfqpoint{1.748144in}{0.851260in}}%
\pgfpathlineto{\pgfqpoint{1.764036in}{0.852813in}}%
\pgfpathlineto{\pgfqpoint{1.771982in}{0.863897in}}%
\pgfpathlineto{\pgfqpoint{1.779928in}{0.870072in}}%
\pgfpathlineto{\pgfqpoint{1.787874in}{0.863042in}}%
\pgfpathlineto{\pgfqpoint{1.795819in}{0.873951in}}%
\pgfpathlineto{\pgfqpoint{1.803765in}{0.870926in}}%
\pgfpathlineto{\pgfqpoint{1.811711in}{0.869929in}}%
\pgfpathlineto{\pgfqpoint{1.819657in}{0.879573in}}%
\pgfpathlineto{\pgfqpoint{1.827603in}{0.877691in}}%
\pgfpathlineto{\pgfqpoint{1.843495in}{0.881187in}}%
\pgfpathlineto{\pgfqpoint{1.851441in}{0.885148in}}%
\pgfpathlineto{\pgfqpoint{1.867332in}{0.882416in}}%
\pgfpathlineto{\pgfqpoint{1.875278in}{0.890288in}}%
\pgfpathlineto{\pgfqpoint{1.883224in}{0.885106in}}%
\pgfpathlineto{\pgfqpoint{1.891170in}{0.896290in}}%
\pgfpathlineto{\pgfqpoint{1.899116in}{0.902369in}}%
\pgfpathlineto{\pgfqpoint{1.907062in}{0.903740in}}%
\pgfpathlineto{\pgfqpoint{1.915008in}{0.895275in}}%
\pgfpathlineto{\pgfqpoint{1.938845in}{0.904804in}}%
\pgfpathlineto{\pgfqpoint{1.946791in}{0.913237in}}%
\pgfpathlineto{\pgfqpoint{1.954737in}{0.913169in}}%
\pgfpathlineto{\pgfqpoint{1.962683in}{0.917179in}}%
\pgfpathlineto{\pgfqpoint{1.970629in}{0.916301in}}%
\pgfpathlineto{\pgfqpoint{1.978575in}{0.922434in}}%
\pgfpathlineto{\pgfqpoint{2.010358in}{0.930114in}}%
\pgfpathlineto{\pgfqpoint{2.018304in}{0.935054in}}%
\pgfpathlineto{\pgfqpoint{2.026250in}{0.935655in}}%
\pgfpathlineto{\pgfqpoint{2.034196in}{0.927036in}}%
\pgfpathlineto{\pgfqpoint{2.050088in}{0.930056in}}%
\pgfpathlineto{\pgfqpoint{2.058033in}{0.944709in}}%
\pgfpathlineto{\pgfqpoint{2.065979in}{0.943308in}}%
\pgfpathlineto{\pgfqpoint{2.081871in}{0.947024in}}%
\pgfpathlineto{\pgfqpoint{2.089817in}{0.952884in}}%
\pgfpathlineto{\pgfqpoint{2.097763in}{0.943770in}}%
\pgfpathlineto{\pgfqpoint{2.105709in}{0.956225in}}%
\pgfpathlineto{\pgfqpoint{2.113655in}{0.952853in}}%
\pgfpathlineto{\pgfqpoint{2.121601in}{0.956434in}}%
\pgfpathlineto{\pgfqpoint{2.129546in}{0.954024in}}%
\pgfpathlineto{\pgfqpoint{2.137492in}{0.964980in}}%
\pgfpathlineto{\pgfqpoint{2.145438in}{0.968489in}}%
\pgfpathlineto{\pgfqpoint{2.161330in}{0.962818in}}%
\pgfpathlineto{\pgfqpoint{2.169276in}{0.966900in}}%
\pgfpathlineto{\pgfqpoint{2.177222in}{0.975811in}}%
\pgfpathlineto{\pgfqpoint{2.185168in}{0.970529in}}%
\pgfpathlineto{\pgfqpoint{2.193113in}{0.970190in}}%
\pgfpathlineto{\pgfqpoint{2.201059in}{0.975928in}}%
\pgfpathlineto{\pgfqpoint{2.209005in}{0.985508in}}%
\pgfpathlineto{\pgfqpoint{2.216951in}{0.976879in}}%
\pgfpathlineto{\pgfqpoint{2.240789in}{0.992595in}}%
\pgfpathlineto{\pgfqpoint{2.248735in}{0.986531in}}%
\pgfpathlineto{\pgfqpoint{2.256680in}{0.995190in}}%
\pgfpathlineto{\pgfqpoint{2.264626in}{0.997989in}}%
\pgfpathlineto{\pgfqpoint{2.272572in}{0.991773in}}%
\pgfpathlineto{\pgfqpoint{2.280518in}{0.996226in}}%
\pgfpathlineto{\pgfqpoint{2.288464in}{1.003210in}}%
\pgfpathlineto{\pgfqpoint{2.296410in}{1.007217in}}%
\pgfpathlineto{\pgfqpoint{2.304356in}{1.003166in}}%
\pgfpathlineto{\pgfqpoint{2.320248in}{1.003447in}}%
\pgfpathlineto{\pgfqpoint{2.328193in}{1.005362in}}%
\pgfpathlineto{\pgfqpoint{2.336139in}{1.017050in}}%
\pgfpathlineto{\pgfqpoint{2.344085in}{1.012657in}}%
\pgfpathlineto{\pgfqpoint{2.352031in}{1.023798in}}%
\pgfpathlineto{\pgfqpoint{2.359977in}{1.017349in}}%
\pgfpathlineto{\pgfqpoint{2.367923in}{1.017737in}}%
\pgfpathlineto{\pgfqpoint{2.375869in}{1.029808in}}%
\pgfpathlineto{\pgfqpoint{2.383815in}{1.021433in}}%
\pgfpathlineto{\pgfqpoint{2.391760in}{1.034094in}}%
\pgfpathlineto{\pgfqpoint{2.399706in}{1.037768in}}%
\pgfpathlineto{\pgfqpoint{2.407652in}{1.026528in}}%
\pgfpathlineto{\pgfqpoint{2.415598in}{1.038213in}}%
\pgfpathlineto{\pgfqpoint{2.423544in}{1.033241in}}%
\pgfpathlineto{\pgfqpoint{2.431490in}{1.040415in}}%
\pgfpathlineto{\pgfqpoint{2.439436in}{1.037305in}}%
\pgfpathlineto{\pgfqpoint{2.447382in}{1.042851in}}%
\pgfpathlineto{\pgfqpoint{2.455327in}{1.042542in}}%
\pgfpathlineto{\pgfqpoint{2.463273in}{1.051379in}}%
\pgfpathlineto{\pgfqpoint{2.463273in}{1.051379in}}%
\pgfusepath{stroke}%
\end{pgfscope}%
\begin{pgfscope}%
\pgfpathrectangle{\pgfqpoint{0.335764in}{0.386111in}}{\pgfqpoint{2.228819in}{2.415278in}}%
\pgfusepath{clip}%
\pgfsetrectcap%
\pgfsetroundjoin%
\pgfsetlinewidth{1.505625pt}%
\definecolor{currentstroke}{rgb}{0.172549,0.627451,0.172549}%
\pgfsetstrokecolor{currentstroke}%
\pgfsetdash{}{0pt}%
\pgfpathmoveto{\pgfqpoint{0.437074in}{0.495909in}}%
\pgfpathlineto{\pgfqpoint{0.452966in}{0.498279in}}%
\pgfpathlineto{\pgfqpoint{0.460912in}{0.504061in}}%
\pgfpathlineto{\pgfqpoint{0.468857in}{0.500515in}}%
\pgfpathlineto{\pgfqpoint{0.476803in}{0.500375in}}%
\pgfpathlineto{\pgfqpoint{0.484749in}{0.506793in}}%
\pgfpathlineto{\pgfqpoint{0.492695in}{0.507523in}}%
\pgfpathlineto{\pgfqpoint{0.500641in}{0.496952in}}%
\pgfpathlineto{\pgfqpoint{0.508587in}{0.500209in}}%
\pgfpathlineto{\pgfqpoint{0.516533in}{0.497855in}}%
\pgfpathlineto{\pgfqpoint{0.524479in}{0.509735in}}%
\pgfpathlineto{\pgfqpoint{0.532424in}{0.500747in}}%
\pgfpathlineto{\pgfqpoint{0.540370in}{0.508952in}}%
\pgfpathlineto{\pgfqpoint{0.548316in}{0.498444in}}%
\pgfpathlineto{\pgfqpoint{0.556262in}{0.504772in}}%
\pgfpathlineto{\pgfqpoint{0.564208in}{0.505002in}}%
\pgfpathlineto{\pgfqpoint{0.572154in}{0.499507in}}%
\pgfpathlineto{\pgfqpoint{0.580100in}{0.501939in}}%
\pgfpathlineto{\pgfqpoint{0.588046in}{0.506654in}}%
\pgfpathlineto{\pgfqpoint{0.595991in}{0.503121in}}%
\pgfpathlineto{\pgfqpoint{0.603937in}{0.502098in}}%
\pgfpathlineto{\pgfqpoint{0.611883in}{0.511296in}}%
\pgfpathlineto{\pgfqpoint{0.619829in}{0.516544in}}%
\pgfpathlineto{\pgfqpoint{0.627775in}{0.505789in}}%
\pgfpathlineto{\pgfqpoint{0.651613in}{0.512196in}}%
\pgfpathlineto{\pgfqpoint{0.659559in}{0.510653in}}%
\pgfpathlineto{\pgfqpoint{0.667504in}{0.517649in}}%
\pgfpathlineto{\pgfqpoint{0.675450in}{0.513197in}}%
\pgfpathlineto{\pgfqpoint{0.683396in}{0.513533in}}%
\pgfpathlineto{\pgfqpoint{0.691342in}{0.522050in}}%
\pgfpathlineto{\pgfqpoint{0.699288in}{0.516124in}}%
\pgfpathlineto{\pgfqpoint{0.707234in}{0.527419in}}%
\pgfpathlineto{\pgfqpoint{0.715180in}{0.520652in}}%
\pgfpathlineto{\pgfqpoint{0.723126in}{0.529895in}}%
\pgfpathlineto{\pgfqpoint{0.731071in}{0.527126in}}%
\pgfpathlineto{\pgfqpoint{0.746963in}{0.527000in}}%
\pgfpathlineto{\pgfqpoint{0.754909in}{0.533492in}}%
\pgfpathlineto{\pgfqpoint{0.770801in}{0.536377in}}%
\pgfpathlineto{\pgfqpoint{0.778747in}{0.528716in}}%
\pgfpathlineto{\pgfqpoint{0.786693in}{0.536743in}}%
\pgfpathlineto{\pgfqpoint{0.794638in}{0.535846in}}%
\pgfpathlineto{\pgfqpoint{0.802584in}{0.531966in}}%
\pgfpathlineto{\pgfqpoint{0.810530in}{0.544692in}}%
\pgfpathlineto{\pgfqpoint{0.818476in}{0.540371in}}%
\pgfpathlineto{\pgfqpoint{0.826422in}{0.548572in}}%
\pgfpathlineto{\pgfqpoint{0.834368in}{0.544777in}}%
\pgfpathlineto{\pgfqpoint{0.842314in}{0.548242in}}%
\pgfpathlineto{\pgfqpoint{0.850260in}{0.542770in}}%
\pgfpathlineto{\pgfqpoint{0.858206in}{0.549460in}}%
\pgfpathlineto{\pgfqpoint{0.866151in}{0.544448in}}%
\pgfpathlineto{\pgfqpoint{0.882043in}{0.548202in}}%
\pgfpathlineto{\pgfqpoint{0.889989in}{0.557412in}}%
\pgfpathlineto{\pgfqpoint{0.897935in}{0.551621in}}%
\pgfpathlineto{\pgfqpoint{0.905881in}{0.559924in}}%
\pgfpathlineto{\pgfqpoint{0.913827in}{0.564546in}}%
\pgfpathlineto{\pgfqpoint{0.921773in}{0.560600in}}%
\pgfpathlineto{\pgfqpoint{0.929718in}{0.567739in}}%
\pgfpathlineto{\pgfqpoint{0.937664in}{0.569851in}}%
\pgfpathlineto{\pgfqpoint{0.945610in}{0.565450in}}%
\pgfpathlineto{\pgfqpoint{0.953556in}{0.578466in}}%
\pgfpathlineto{\pgfqpoint{0.961502in}{0.571654in}}%
\pgfpathlineto{\pgfqpoint{0.969448in}{0.580744in}}%
\pgfpathlineto{\pgfqpoint{0.985340in}{0.581245in}}%
\pgfpathlineto{\pgfqpoint{0.993285in}{0.584878in}}%
\pgfpathlineto{\pgfqpoint{1.001231in}{0.583676in}}%
\pgfpathlineto{\pgfqpoint{1.009177in}{0.592036in}}%
\pgfpathlineto{\pgfqpoint{1.025069in}{0.587688in}}%
\pgfpathlineto{\pgfqpoint{1.033015in}{0.601439in}}%
\pgfpathlineto{\pgfqpoint{1.040961in}{0.600234in}}%
\pgfpathlineto{\pgfqpoint{1.048907in}{0.607680in}}%
\pgfpathlineto{\pgfqpoint{1.056853in}{0.608156in}}%
\pgfpathlineto{\pgfqpoint{1.064798in}{0.603246in}}%
\pgfpathlineto{\pgfqpoint{1.072744in}{0.613087in}}%
\pgfpathlineto{\pgfqpoint{1.080690in}{0.605140in}}%
\pgfpathlineto{\pgfqpoint{1.104528in}{0.626584in}}%
\pgfpathlineto{\pgfqpoint{1.112474in}{0.622355in}}%
\pgfpathlineto{\pgfqpoint{1.120420in}{0.631404in}}%
\pgfpathlineto{\pgfqpoint{1.128365in}{0.622976in}}%
\pgfpathlineto{\pgfqpoint{1.136311in}{0.630264in}}%
\pgfpathlineto{\pgfqpoint{1.144257in}{0.641232in}}%
\pgfpathlineto{\pgfqpoint{1.152203in}{0.633315in}}%
\pgfpathlineto{\pgfqpoint{1.160149in}{0.638659in}}%
\pgfpathlineto{\pgfqpoint{1.168095in}{0.646012in}}%
\pgfpathlineto{\pgfqpoint{1.176041in}{0.644187in}}%
\pgfpathlineto{\pgfqpoint{1.183987in}{0.650663in}}%
\pgfpathlineto{\pgfqpoint{1.199878in}{0.650923in}}%
\pgfpathlineto{\pgfqpoint{1.207824in}{0.663092in}}%
\pgfpathlineto{\pgfqpoint{1.215770in}{0.659629in}}%
\pgfpathlineto{\pgfqpoint{1.223716in}{0.670008in}}%
\pgfpathlineto{\pgfqpoint{1.231662in}{0.667438in}}%
\pgfpathlineto{\pgfqpoint{1.239608in}{0.678130in}}%
\pgfpathlineto{\pgfqpoint{1.247554in}{0.671402in}}%
\pgfpathlineto{\pgfqpoint{1.255500in}{0.676427in}}%
\pgfpathlineto{\pgfqpoint{1.263445in}{0.687911in}}%
\pgfpathlineto{\pgfqpoint{1.271391in}{0.693281in}}%
\pgfpathlineto{\pgfqpoint{1.279337in}{0.693011in}}%
\pgfpathlineto{\pgfqpoint{1.287283in}{0.689867in}}%
\pgfpathlineto{\pgfqpoint{1.303175in}{0.706814in}}%
\pgfpathlineto{\pgfqpoint{1.311121in}{0.698996in}}%
\pgfpathlineto{\pgfqpoint{1.319067in}{0.709695in}}%
\pgfpathlineto{\pgfqpoint{1.327012in}{0.714376in}}%
\pgfpathlineto{\pgfqpoint{1.334958in}{0.716502in}}%
\pgfpathlineto{\pgfqpoint{1.342904in}{0.713487in}}%
\pgfpathlineto{\pgfqpoint{1.350850in}{0.723052in}}%
\pgfpathlineto{\pgfqpoint{1.358796in}{0.724546in}}%
\pgfpathlineto{\pgfqpoint{1.374688in}{0.731395in}}%
\pgfpathlineto{\pgfqpoint{1.382634in}{0.740999in}}%
\pgfpathlineto{\pgfqpoint{1.390580in}{0.743539in}}%
\pgfpathlineto{\pgfqpoint{1.398525in}{0.743689in}}%
\pgfpathlineto{\pgfqpoint{1.406471in}{0.756526in}}%
\pgfpathlineto{\pgfqpoint{1.414417in}{0.753095in}}%
\pgfpathlineto{\pgfqpoint{1.422363in}{0.757447in}}%
\pgfpathlineto{\pgfqpoint{1.430309in}{0.759302in}}%
\pgfpathlineto{\pgfqpoint{1.438255in}{0.766011in}}%
\pgfpathlineto{\pgfqpoint{1.446201in}{0.767883in}}%
\pgfpathlineto{\pgfqpoint{1.454147in}{0.782655in}}%
\pgfpathlineto{\pgfqpoint{1.462092in}{0.783267in}}%
\pgfpathlineto{\pgfqpoint{1.470038in}{0.789827in}}%
\pgfpathlineto{\pgfqpoint{1.477984in}{0.784400in}}%
\pgfpathlineto{\pgfqpoint{1.485930in}{0.790589in}}%
\pgfpathlineto{\pgfqpoint{1.493876in}{0.803856in}}%
\pgfpathlineto{\pgfqpoint{1.501822in}{0.797131in}}%
\pgfpathlineto{\pgfqpoint{1.509768in}{0.811295in}}%
\pgfpathlineto{\pgfqpoint{1.525659in}{0.817977in}}%
\pgfpathlineto{\pgfqpoint{1.533605in}{0.822669in}}%
\pgfpathlineto{\pgfqpoint{1.541551in}{0.823964in}}%
\pgfpathlineto{\pgfqpoint{1.557443in}{0.842516in}}%
\pgfpathlineto{\pgfqpoint{1.573335in}{0.851442in}}%
\pgfpathlineto{\pgfqpoint{1.581281in}{0.846203in}}%
\pgfpathlineto{\pgfqpoint{1.589227in}{0.856375in}}%
\pgfpathlineto{\pgfqpoint{1.597172in}{0.862706in}}%
\pgfpathlineto{\pgfqpoint{1.605118in}{0.859637in}}%
\pgfpathlineto{\pgfqpoint{1.613064in}{0.872319in}}%
\pgfpathlineto{\pgfqpoint{1.621010in}{0.868793in}}%
\pgfpathlineto{\pgfqpoint{1.628956in}{0.878285in}}%
\pgfpathlineto{\pgfqpoint{1.636902in}{0.881288in}}%
\pgfpathlineto{\pgfqpoint{1.644848in}{0.896089in}}%
\pgfpathlineto{\pgfqpoint{1.652794in}{0.895230in}}%
\pgfpathlineto{\pgfqpoint{1.668685in}{0.908375in}}%
\pgfpathlineto{\pgfqpoint{1.676631in}{0.909449in}}%
\pgfpathlineto{\pgfqpoint{1.684577in}{0.917455in}}%
\pgfpathlineto{\pgfqpoint{1.692523in}{0.914652in}}%
\pgfpathlineto{\pgfqpoint{1.700469in}{0.932045in}}%
\pgfpathlineto{\pgfqpoint{1.708415in}{0.938222in}}%
\pgfpathlineto{\pgfqpoint{1.716361in}{0.930799in}}%
\pgfpathlineto{\pgfqpoint{1.732252in}{0.949883in}}%
\pgfpathlineto{\pgfqpoint{1.748144in}{0.956482in}}%
\pgfpathlineto{\pgfqpoint{1.756090in}{0.969947in}}%
\pgfpathlineto{\pgfqpoint{1.764036in}{0.964502in}}%
\pgfpathlineto{\pgfqpoint{1.771982in}{0.971560in}}%
\pgfpathlineto{\pgfqpoint{1.779928in}{0.981369in}}%
\pgfpathlineto{\pgfqpoint{1.803765in}{0.992960in}}%
\pgfpathlineto{\pgfqpoint{1.827603in}{1.015164in}}%
\pgfpathlineto{\pgfqpoint{1.835549in}{1.017094in}}%
\pgfpathlineto{\pgfqpoint{1.843495in}{1.022127in}}%
\pgfpathlineto{\pgfqpoint{1.851441in}{1.035934in}}%
\pgfpathlineto{\pgfqpoint{1.859386in}{1.037183in}}%
\pgfpathlineto{\pgfqpoint{1.867332in}{1.050083in}}%
\pgfpathlineto{\pgfqpoint{1.875278in}{1.054685in}}%
\pgfpathlineto{\pgfqpoint{1.883224in}{1.055611in}}%
\pgfpathlineto{\pgfqpoint{1.891170in}{1.059230in}}%
\pgfpathlineto{\pgfqpoint{1.899116in}{1.073503in}}%
\pgfpathlineto{\pgfqpoint{1.907062in}{1.074233in}}%
\pgfpathlineto{\pgfqpoint{1.915008in}{1.077657in}}%
\pgfpathlineto{\pgfqpoint{1.930899in}{1.100187in}}%
\pgfpathlineto{\pgfqpoint{1.946791in}{1.103535in}}%
\pgfpathlineto{\pgfqpoint{1.954737in}{1.111641in}}%
\pgfpathlineto{\pgfqpoint{1.962683in}{1.115648in}}%
\pgfpathlineto{\pgfqpoint{1.978575in}{1.137574in}}%
\pgfpathlineto{\pgfqpoint{1.986521in}{1.144513in}}%
\pgfpathlineto{\pgfqpoint{1.994466in}{1.153617in}}%
\pgfpathlineto{\pgfqpoint{2.002412in}{1.153512in}}%
\pgfpathlineto{\pgfqpoint{2.010358in}{1.167689in}}%
\pgfpathlineto{\pgfqpoint{2.018304in}{1.167584in}}%
\pgfpathlineto{\pgfqpoint{2.026250in}{1.181012in}}%
\pgfpathlineto{\pgfqpoint{2.034196in}{1.181700in}}%
\pgfpathlineto{\pgfqpoint{2.042142in}{1.188615in}}%
\pgfpathlineto{\pgfqpoint{2.050088in}{1.201309in}}%
\pgfpathlineto{\pgfqpoint{2.058033in}{1.199360in}}%
\pgfpathlineto{\pgfqpoint{2.065979in}{1.214707in}}%
\pgfpathlineto{\pgfqpoint{2.073925in}{1.209369in}}%
\pgfpathlineto{\pgfqpoint{2.081871in}{1.226752in}}%
\pgfpathlineto{\pgfqpoint{2.089817in}{1.232424in}}%
\pgfpathlineto{\pgfqpoint{2.097763in}{1.229873in}}%
\pgfpathlineto{\pgfqpoint{2.113655in}{1.251019in}}%
\pgfpathlineto{\pgfqpoint{2.121601in}{1.257477in}}%
\pgfpathlineto{\pgfqpoint{2.129546in}{1.267286in}}%
\pgfpathlineto{\pgfqpoint{2.137492in}{1.269497in}}%
\pgfpathlineto{\pgfqpoint{2.145438in}{1.276764in}}%
\pgfpathlineto{\pgfqpoint{2.153384in}{1.290340in}}%
\pgfpathlineto{\pgfqpoint{2.161330in}{1.290587in}}%
\pgfpathlineto{\pgfqpoint{2.169276in}{1.304318in}}%
\pgfpathlineto{\pgfqpoint{2.177222in}{1.309658in}}%
\pgfpathlineto{\pgfqpoint{2.185168in}{1.317946in}}%
\pgfpathlineto{\pgfqpoint{2.201059in}{1.325733in}}%
\pgfpathlineto{\pgfqpoint{2.209005in}{1.336050in}}%
\pgfpathlineto{\pgfqpoint{2.224897in}{1.347278in}}%
\pgfpathlineto{\pgfqpoint{2.232843in}{1.366755in}}%
\pgfpathlineto{\pgfqpoint{2.240789in}{1.374191in}}%
\pgfpathlineto{\pgfqpoint{2.248735in}{1.369342in}}%
\pgfpathlineto{\pgfqpoint{2.256680in}{1.388238in}}%
\pgfpathlineto{\pgfqpoint{2.264626in}{1.397560in}}%
\pgfpathlineto{\pgfqpoint{2.272572in}{1.401386in}}%
\pgfpathlineto{\pgfqpoint{2.280518in}{1.402582in}}%
\pgfpathlineto{\pgfqpoint{2.288464in}{1.414046in}}%
\pgfpathlineto{\pgfqpoint{2.296410in}{1.418516in}}%
\pgfpathlineto{\pgfqpoint{2.312302in}{1.442979in}}%
\pgfpathlineto{\pgfqpoint{2.320248in}{1.439598in}}%
\pgfpathlineto{\pgfqpoint{2.328193in}{1.458360in}}%
\pgfpathlineto{\pgfqpoint{2.344085in}{1.467214in}}%
\pgfpathlineto{\pgfqpoint{2.352031in}{1.476598in}}%
\pgfpathlineto{\pgfqpoint{2.359977in}{1.489828in}}%
\pgfpathlineto{\pgfqpoint{2.367923in}{1.494907in}}%
\pgfpathlineto{\pgfqpoint{2.375869in}{1.509347in}}%
\pgfpathlineto{\pgfqpoint{2.383815in}{1.517570in}}%
\pgfpathlineto{\pgfqpoint{2.391760in}{1.514771in}}%
\pgfpathlineto{\pgfqpoint{2.399706in}{1.533897in}}%
\pgfpathlineto{\pgfqpoint{2.407652in}{1.537547in}}%
\pgfpathlineto{\pgfqpoint{2.415598in}{1.550582in}}%
\pgfpathlineto{\pgfqpoint{2.423544in}{1.551669in}}%
\pgfpathlineto{\pgfqpoint{2.431490in}{1.564170in}}%
\pgfpathlineto{\pgfqpoint{2.447382in}{1.574480in}}%
\pgfpathlineto{\pgfqpoint{2.455327in}{1.592337in}}%
\pgfpathlineto{\pgfqpoint{2.463273in}{1.591262in}}%
\pgfpathlineto{\pgfqpoint{2.463273in}{1.591262in}}%
\pgfusepath{stroke}%
\end{pgfscope}%
\begin{pgfscope}%
\pgfpathrectangle{\pgfqpoint{0.335764in}{0.386111in}}{\pgfqpoint{2.228819in}{2.415278in}}%
\pgfusepath{clip}%
\pgfsetrectcap%
\pgfsetroundjoin%
\pgfsetlinewidth{1.505625pt}%
\definecolor{currentstroke}{rgb}{0.839216,0.152941,0.156863}%
\pgfsetstrokecolor{currentstroke}%
\pgfsetdash{}{0pt}%
\pgfpathmoveto{\pgfqpoint{0.437074in}{0.500791in}}%
\pgfpathlineto{\pgfqpoint{0.445020in}{0.507047in}}%
\pgfpathlineto{\pgfqpoint{0.468857in}{0.507341in}}%
\pgfpathlineto{\pgfqpoint{0.476803in}{0.499625in}}%
\pgfpathlineto{\pgfqpoint{0.484749in}{0.496599in}}%
\pgfpathlineto{\pgfqpoint{0.492695in}{0.502284in}}%
\pgfpathlineto{\pgfqpoint{0.508587in}{0.505503in}}%
\pgfpathlineto{\pgfqpoint{0.516533in}{0.506287in}}%
\pgfpathlineto{\pgfqpoint{0.524479in}{0.496698in}}%
\pgfpathlineto{\pgfqpoint{0.540370in}{0.507134in}}%
\pgfpathlineto{\pgfqpoint{0.548316in}{0.504138in}}%
\pgfpathlineto{\pgfqpoint{0.556262in}{0.495896in}}%
\pgfpathlineto{\pgfqpoint{0.564208in}{0.504768in}}%
\pgfpathlineto{\pgfqpoint{0.572154in}{0.506262in}}%
\pgfpathlineto{\pgfqpoint{0.580100in}{0.501046in}}%
\pgfpathlineto{\pgfqpoint{0.588046in}{0.506021in}}%
\pgfpathlineto{\pgfqpoint{0.595991in}{0.498523in}}%
\pgfpathlineto{\pgfqpoint{0.603937in}{0.499918in}}%
\pgfpathlineto{\pgfqpoint{0.611883in}{0.496157in}}%
\pgfpathlineto{\pgfqpoint{0.619829in}{0.505563in}}%
\pgfpathlineto{\pgfqpoint{0.627775in}{0.497309in}}%
\pgfpathlineto{\pgfqpoint{0.635721in}{0.506268in}}%
\pgfpathlineto{\pgfqpoint{0.643667in}{0.507375in}}%
\pgfpathlineto{\pgfqpoint{0.651613in}{0.501402in}}%
\pgfpathlineto{\pgfqpoint{0.659559in}{0.510510in}}%
\pgfpathlineto{\pgfqpoint{0.667504in}{0.505359in}}%
\pgfpathlineto{\pgfqpoint{0.683396in}{0.510724in}}%
\pgfpathlineto{\pgfqpoint{0.691342in}{0.503439in}}%
\pgfpathlineto{\pgfqpoint{0.699288in}{0.510802in}}%
\pgfpathlineto{\pgfqpoint{0.707234in}{0.500642in}}%
\pgfpathlineto{\pgfqpoint{0.715180in}{0.512391in}}%
\pgfpathlineto{\pgfqpoint{0.723126in}{0.506465in}}%
\pgfpathlineto{\pgfqpoint{0.731071in}{0.509830in}}%
\pgfpathlineto{\pgfqpoint{0.739017in}{0.509780in}}%
\pgfpathlineto{\pgfqpoint{0.746963in}{0.514344in}}%
\pgfpathlineto{\pgfqpoint{0.754909in}{0.503814in}}%
\pgfpathlineto{\pgfqpoint{0.762855in}{0.505768in}}%
\pgfpathlineto{\pgfqpoint{0.770801in}{0.505249in}}%
\pgfpathlineto{\pgfqpoint{0.778747in}{0.508583in}}%
\pgfpathlineto{\pgfqpoint{0.786693in}{0.517472in}}%
\pgfpathlineto{\pgfqpoint{0.794638in}{0.508803in}}%
\pgfpathlineto{\pgfqpoint{0.810530in}{0.515449in}}%
\pgfpathlineto{\pgfqpoint{0.818476in}{0.517761in}}%
\pgfpathlineto{\pgfqpoint{0.826422in}{0.510369in}}%
\pgfpathlineto{\pgfqpoint{0.834368in}{0.522988in}}%
\pgfpathlineto{\pgfqpoint{0.842314in}{0.522185in}}%
\pgfpathlineto{\pgfqpoint{0.850260in}{0.513981in}}%
\pgfpathlineto{\pgfqpoint{0.858206in}{0.524718in}}%
\pgfpathlineto{\pgfqpoint{0.866151in}{0.518941in}}%
\pgfpathlineto{\pgfqpoint{0.882043in}{0.518279in}}%
\pgfpathlineto{\pgfqpoint{0.889989in}{0.529237in}}%
\pgfpathlineto{\pgfqpoint{0.897935in}{0.532177in}}%
\pgfpathlineto{\pgfqpoint{0.905881in}{0.522141in}}%
\pgfpathlineto{\pgfqpoint{0.913827in}{0.536449in}}%
\pgfpathlineto{\pgfqpoint{0.921773in}{0.526776in}}%
\pgfpathlineto{\pgfqpoint{0.929718in}{0.528071in}}%
\pgfpathlineto{\pgfqpoint{0.937664in}{0.527791in}}%
\pgfpathlineto{\pgfqpoint{0.945610in}{0.541819in}}%
\pgfpathlineto{\pgfqpoint{0.953556in}{0.531292in}}%
\pgfpathlineto{\pgfqpoint{0.961502in}{0.536414in}}%
\pgfpathlineto{\pgfqpoint{0.969448in}{0.536679in}}%
\pgfpathlineto{\pgfqpoint{0.977394in}{0.547488in}}%
\pgfpathlineto{\pgfqpoint{0.985340in}{0.543629in}}%
\pgfpathlineto{\pgfqpoint{0.993285in}{0.544891in}}%
\pgfpathlineto{\pgfqpoint{1.001231in}{0.552812in}}%
\pgfpathlineto{\pgfqpoint{1.009177in}{0.544549in}}%
\pgfpathlineto{\pgfqpoint{1.017123in}{0.547131in}}%
\pgfpathlineto{\pgfqpoint{1.025069in}{0.559807in}}%
\pgfpathlineto{\pgfqpoint{1.040961in}{0.554905in}}%
\pgfpathlineto{\pgfqpoint{1.048907in}{0.558788in}}%
\pgfpathlineto{\pgfqpoint{1.056853in}{0.557566in}}%
\pgfpathlineto{\pgfqpoint{1.064798in}{0.561087in}}%
\pgfpathlineto{\pgfqpoint{1.072744in}{0.566899in}}%
\pgfpathlineto{\pgfqpoint{1.080690in}{0.570769in}}%
\pgfpathlineto{\pgfqpoint{1.088636in}{0.579320in}}%
\pgfpathlineto{\pgfqpoint{1.096582in}{0.581329in}}%
\pgfpathlineto{\pgfqpoint{1.104528in}{0.574713in}}%
\pgfpathlineto{\pgfqpoint{1.112474in}{0.584794in}}%
\pgfpathlineto{\pgfqpoint{1.120420in}{0.581423in}}%
\pgfpathlineto{\pgfqpoint{1.144257in}{0.590470in}}%
\pgfpathlineto{\pgfqpoint{1.152203in}{0.601763in}}%
\pgfpathlineto{\pgfqpoint{1.160149in}{0.598565in}}%
\pgfpathlineto{\pgfqpoint{1.168095in}{0.598196in}}%
\pgfpathlineto{\pgfqpoint{1.176041in}{0.610119in}}%
\pgfpathlineto{\pgfqpoint{1.183987in}{0.607543in}}%
\pgfpathlineto{\pgfqpoint{1.191932in}{0.611256in}}%
\pgfpathlineto{\pgfqpoint{1.199878in}{0.623298in}}%
\pgfpathlineto{\pgfqpoint{1.207824in}{0.615215in}}%
\pgfpathlineto{\pgfqpoint{1.215770in}{0.622161in}}%
\pgfpathlineto{\pgfqpoint{1.223716in}{0.632600in}}%
\pgfpathlineto{\pgfqpoint{1.231662in}{0.628788in}}%
\pgfpathlineto{\pgfqpoint{1.239608in}{0.638809in}}%
\pgfpathlineto{\pgfqpoint{1.247554in}{0.640354in}}%
\pgfpathlineto{\pgfqpoint{1.255500in}{0.644251in}}%
\pgfpathlineto{\pgfqpoint{1.271391in}{0.659647in}}%
\pgfpathlineto{\pgfqpoint{1.279337in}{0.661891in}}%
\pgfpathlineto{\pgfqpoint{1.287283in}{0.669636in}}%
\pgfpathlineto{\pgfqpoint{1.295229in}{0.666518in}}%
\pgfpathlineto{\pgfqpoint{1.303175in}{0.674880in}}%
\pgfpathlineto{\pgfqpoint{1.311121in}{0.677415in}}%
\pgfpathlineto{\pgfqpoint{1.319067in}{0.687047in}}%
\pgfpathlineto{\pgfqpoint{1.327012in}{0.693563in}}%
\pgfpathlineto{\pgfqpoint{1.334958in}{0.693725in}}%
\pgfpathlineto{\pgfqpoint{1.342904in}{0.696965in}}%
\pgfpathlineto{\pgfqpoint{1.350850in}{0.695685in}}%
\pgfpathlineto{\pgfqpoint{1.358796in}{0.709123in}}%
\pgfpathlineto{\pgfqpoint{1.366742in}{0.712941in}}%
\pgfpathlineto{\pgfqpoint{1.374688in}{0.720422in}}%
\pgfpathlineto{\pgfqpoint{1.382634in}{0.730559in}}%
\pgfpathlineto{\pgfqpoint{1.390580in}{0.735345in}}%
\pgfpathlineto{\pgfqpoint{1.398525in}{0.735425in}}%
\pgfpathlineto{\pgfqpoint{1.414417in}{0.745452in}}%
\pgfpathlineto{\pgfqpoint{1.422363in}{0.756769in}}%
\pgfpathlineto{\pgfqpoint{1.430309in}{0.759351in}}%
\pgfpathlineto{\pgfqpoint{1.438255in}{0.771879in}}%
\pgfpathlineto{\pgfqpoint{1.446201in}{0.767795in}}%
\pgfpathlineto{\pgfqpoint{1.462092in}{0.781243in}}%
\pgfpathlineto{\pgfqpoint{1.470038in}{0.795970in}}%
\pgfpathlineto{\pgfqpoint{1.477984in}{0.804755in}}%
\pgfpathlineto{\pgfqpoint{1.485930in}{0.805374in}}%
\pgfpathlineto{\pgfqpoint{1.493876in}{0.808593in}}%
\pgfpathlineto{\pgfqpoint{1.501822in}{0.824846in}}%
\pgfpathlineto{\pgfqpoint{1.509768in}{0.824029in}}%
\pgfpathlineto{\pgfqpoint{1.517714in}{0.828544in}}%
\pgfpathlineto{\pgfqpoint{1.525659in}{0.844203in}}%
\pgfpathlineto{\pgfqpoint{1.533605in}{0.851898in}}%
\pgfpathlineto{\pgfqpoint{1.541551in}{0.861956in}}%
\pgfpathlineto{\pgfqpoint{1.549497in}{0.858210in}}%
\pgfpathlineto{\pgfqpoint{1.557443in}{0.877627in}}%
\pgfpathlineto{\pgfqpoint{1.589227in}{0.901313in}}%
\pgfpathlineto{\pgfqpoint{1.597172in}{0.906244in}}%
\pgfpathlineto{\pgfqpoint{1.605118in}{0.922263in}}%
\pgfpathlineto{\pgfqpoint{1.613064in}{0.933647in}}%
\pgfpathlineto{\pgfqpoint{1.621010in}{0.935268in}}%
\pgfpathlineto{\pgfqpoint{1.628956in}{0.951570in}}%
\pgfpathlineto{\pgfqpoint{1.636902in}{0.949615in}}%
\pgfpathlineto{\pgfqpoint{1.644848in}{0.959175in}}%
\pgfpathlineto{\pgfqpoint{1.660739in}{0.988863in}}%
\pgfpathlineto{\pgfqpoint{1.668685in}{0.987722in}}%
\pgfpathlineto{\pgfqpoint{1.676631in}{1.005073in}}%
\pgfpathlineto{\pgfqpoint{1.684577in}{1.018449in}}%
\pgfpathlineto{\pgfqpoint{1.692523in}{1.023977in}}%
\pgfpathlineto{\pgfqpoint{1.700469in}{1.025895in}}%
\pgfpathlineto{\pgfqpoint{1.708415in}{1.041814in}}%
\pgfpathlineto{\pgfqpoint{1.716361in}{1.053940in}}%
\pgfpathlineto{\pgfqpoint{1.724306in}{1.055532in}}%
\pgfpathlineto{\pgfqpoint{1.732252in}{1.070192in}}%
\pgfpathlineto{\pgfqpoint{1.740198in}{1.088745in}}%
\pgfpathlineto{\pgfqpoint{1.748144in}{1.090162in}}%
\pgfpathlineto{\pgfqpoint{1.756090in}{1.111111in}}%
\pgfpathlineto{\pgfqpoint{1.771982in}{1.130823in}}%
\pgfpathlineto{\pgfqpoint{1.779928in}{1.132532in}}%
\pgfpathlineto{\pgfqpoint{1.787874in}{1.155598in}}%
\pgfpathlineto{\pgfqpoint{1.795819in}{1.155052in}}%
\pgfpathlineto{\pgfqpoint{1.803765in}{1.168608in}}%
\pgfpathlineto{\pgfqpoint{1.811711in}{1.186610in}}%
\pgfpathlineto{\pgfqpoint{1.819657in}{1.190862in}}%
\pgfpathlineto{\pgfqpoint{1.827603in}{1.205393in}}%
\pgfpathlineto{\pgfqpoint{1.835549in}{1.214532in}}%
\pgfpathlineto{\pgfqpoint{1.843495in}{1.238197in}}%
\pgfpathlineto{\pgfqpoint{1.851441in}{1.251912in}}%
\pgfpathlineto{\pgfqpoint{1.859386in}{1.257769in}}%
\pgfpathlineto{\pgfqpoint{1.867332in}{1.276426in}}%
\pgfpathlineto{\pgfqpoint{1.875278in}{1.283799in}}%
\pgfpathlineto{\pgfqpoint{1.883224in}{1.301334in}}%
\pgfpathlineto{\pgfqpoint{1.891170in}{1.303269in}}%
\pgfpathlineto{\pgfqpoint{1.899116in}{1.316992in}}%
\pgfpathlineto{\pgfqpoint{1.907062in}{1.339069in}}%
\pgfpathlineto{\pgfqpoint{1.915008in}{1.354644in}}%
\pgfpathlineto{\pgfqpoint{1.930899in}{1.372713in}}%
\pgfpathlineto{\pgfqpoint{1.938845in}{1.391854in}}%
\pgfpathlineto{\pgfqpoint{1.946791in}{1.401170in}}%
\pgfpathlineto{\pgfqpoint{1.954737in}{1.414397in}}%
\pgfpathlineto{\pgfqpoint{1.962683in}{1.439114in}}%
\pgfpathlineto{\pgfqpoint{1.970629in}{1.456488in}}%
\pgfpathlineto{\pgfqpoint{1.978575in}{1.458450in}}%
\pgfpathlineto{\pgfqpoint{1.986521in}{1.473148in}}%
\pgfpathlineto{\pgfqpoint{1.994466in}{1.501436in}}%
\pgfpathlineto{\pgfqpoint{2.002412in}{1.504554in}}%
\pgfpathlineto{\pgfqpoint{2.026250in}{1.560043in}}%
\pgfpathlineto{\pgfqpoint{2.034196in}{1.570654in}}%
\pgfpathlineto{\pgfqpoint{2.058033in}{1.618845in}}%
\pgfpathlineto{\pgfqpoint{2.065979in}{1.634053in}}%
\pgfpathlineto{\pgfqpoint{2.073925in}{1.656534in}}%
\pgfpathlineto{\pgfqpoint{2.105709in}{1.723680in}}%
\pgfpathlineto{\pgfqpoint{2.113655in}{1.743254in}}%
\pgfpathlineto{\pgfqpoint{2.121601in}{1.753349in}}%
\pgfpathlineto{\pgfqpoint{2.137492in}{1.801234in}}%
\pgfpathlineto{\pgfqpoint{2.145438in}{1.808179in}}%
\pgfpathlineto{\pgfqpoint{2.161330in}{1.848876in}}%
\pgfpathlineto{\pgfqpoint{2.169276in}{1.867247in}}%
\pgfpathlineto{\pgfqpoint{2.177222in}{1.881395in}}%
\pgfpathlineto{\pgfqpoint{2.193113in}{1.928370in}}%
\pgfpathlineto{\pgfqpoint{2.209005in}{1.964210in}}%
\pgfpathlineto{\pgfqpoint{2.224897in}{2.010844in}}%
\pgfpathlineto{\pgfqpoint{2.232843in}{2.022782in}}%
\pgfpathlineto{\pgfqpoint{2.240789in}{2.045254in}}%
\pgfpathlineto{\pgfqpoint{2.248735in}{2.060634in}}%
\pgfpathlineto{\pgfqpoint{2.256680in}{2.080621in}}%
\pgfpathlineto{\pgfqpoint{2.264626in}{2.111832in}}%
\pgfpathlineto{\pgfqpoint{2.280518in}{2.155090in}}%
\pgfpathlineto{\pgfqpoint{2.288464in}{2.168096in}}%
\pgfpathlineto{\pgfqpoint{2.304356in}{2.215899in}}%
\pgfpathlineto{\pgfqpoint{2.320248in}{2.256342in}}%
\pgfpathlineto{\pgfqpoint{2.328193in}{2.285483in}}%
\pgfpathlineto{\pgfqpoint{2.336139in}{2.297859in}}%
\pgfpathlineto{\pgfqpoint{2.344085in}{2.329609in}}%
\pgfpathlineto{\pgfqpoint{2.352031in}{2.350250in}}%
\pgfpathlineto{\pgfqpoint{2.359977in}{2.365405in}}%
\pgfpathlineto{\pgfqpoint{2.375869in}{2.419994in}}%
\pgfpathlineto{\pgfqpoint{2.383815in}{2.438729in}}%
\pgfpathlineto{\pgfqpoint{2.399706in}{2.493018in}}%
\pgfpathlineto{\pgfqpoint{2.407652in}{2.509333in}}%
\pgfpathlineto{\pgfqpoint{2.415598in}{2.543844in}}%
\pgfpathlineto{\pgfqpoint{2.431490in}{2.592896in}}%
\pgfpathlineto{\pgfqpoint{2.439436in}{2.609475in}}%
\pgfpathlineto{\pgfqpoint{2.447382in}{2.641963in}}%
\pgfpathlineto{\pgfqpoint{2.463273in}{2.691604in}}%
\pgfpathlineto{\pgfqpoint{2.463273in}{2.691604in}}%
\pgfusepath{stroke}%
\end{pgfscope}%
\begin{pgfscope}%
\pgfsetrectcap%
\pgfsetmiterjoin%
\pgfsetlinewidth{0.803000pt}%
\definecolor{currentstroke}{rgb}{0.000000,0.000000,0.000000}%
\pgfsetstrokecolor{currentstroke}%
\pgfsetdash{}{0pt}%
\pgfpathmoveto{\pgfqpoint{0.335764in}{0.386111in}}%
\pgfpathlineto{\pgfqpoint{0.335764in}{2.801389in}}%
\pgfusepath{stroke}%
\end{pgfscope}%
\begin{pgfscope}%
\pgfsetrectcap%
\pgfsetmiterjoin%
\pgfsetlinewidth{0.803000pt}%
\definecolor{currentstroke}{rgb}{0.000000,0.000000,0.000000}%
\pgfsetstrokecolor{currentstroke}%
\pgfsetdash{}{0pt}%
\pgfpathmoveto{\pgfqpoint{2.564583in}{0.386111in}}%
\pgfpathlineto{\pgfqpoint{2.564583in}{2.801389in}}%
\pgfusepath{stroke}%
\end{pgfscope}%
\begin{pgfscope}%
\pgfsetrectcap%
\pgfsetmiterjoin%
\pgfsetlinewidth{0.803000pt}%
\definecolor{currentstroke}{rgb}{0.000000,0.000000,0.000000}%
\pgfsetstrokecolor{currentstroke}%
\pgfsetdash{}{0pt}%
\pgfpathmoveto{\pgfqpoint{0.335764in}{0.386111in}}%
\pgfpathlineto{\pgfqpoint{2.564583in}{0.386111in}}%
\pgfusepath{stroke}%
\end{pgfscope}%
\begin{pgfscope}%
\pgfsetrectcap%
\pgfsetmiterjoin%
\pgfsetlinewidth{0.803000pt}%
\definecolor{currentstroke}{rgb}{0.000000,0.000000,0.000000}%
\pgfsetstrokecolor{currentstroke}%
\pgfsetdash{}{0pt}%
\pgfpathmoveto{\pgfqpoint{0.335764in}{2.801389in}}%
\pgfpathlineto{\pgfqpoint{2.564583in}{2.801389in}}%
\pgfusepath{stroke}%
\end{pgfscope}%
\begin{pgfscope}%
\pgfsetbuttcap%
\pgfsetmiterjoin%
\definecolor{currentfill}{rgb}{1.000000,1.000000,1.000000}%
\pgfsetfillcolor{currentfill}%
\pgfsetfillopacity{0.800000}%
\pgfsetlinewidth{1.003750pt}%
\definecolor{currentstroke}{rgb}{0.800000,0.800000,0.800000}%
\pgfsetstrokecolor{currentstroke}%
\pgfsetstrokeopacity{0.800000}%
\pgfsetdash{}{0pt}%
\pgfpathmoveto{\pgfqpoint{0.432986in}{1.886202in}}%
\pgfpathlineto{\pgfqpoint{1.255387in}{1.886202in}}%
\pgfpathquadraticcurveto{\pgfqpoint{1.283165in}{1.886202in}}{\pgfqpoint{1.283165in}{1.913980in}}%
\pgfpathlineto{\pgfqpoint{1.283165in}{2.704167in}}%
\pgfpathquadraticcurveto{\pgfqpoint{1.283165in}{2.731944in}}{\pgfqpoint{1.255387in}{2.731944in}}%
\pgfpathlineto{\pgfqpoint{0.432986in}{2.731944in}}%
\pgfpathquadraticcurveto{\pgfqpoint{0.405208in}{2.731944in}}{\pgfqpoint{0.405208in}{2.704167in}}%
\pgfpathlineto{\pgfqpoint{0.405208in}{1.913980in}}%
\pgfpathquadraticcurveto{\pgfqpoint{0.405208in}{1.886202in}}{\pgfqpoint{0.432986in}{1.886202in}}%
\pgfpathclose%
\pgfusepath{stroke,fill}%
\end{pgfscope}%
\begin{pgfscope}%
\pgfsetrectcap%
\pgfsetroundjoin%
\pgfsetlinewidth{1.505625pt}%
\definecolor{currentstroke}{rgb}{0.121569,0.466667,0.705882}%
\pgfsetstrokecolor{currentstroke}%
\pgfsetdash{}{0pt}%
\pgfpathmoveto{\pgfqpoint{0.460764in}{2.620370in}}%
\pgfpathlineto{\pgfqpoint{0.738542in}{2.620370in}}%
\pgfusepath{stroke}%
\end{pgfscope}%
\begin{pgfscope}%
\definecolor{textcolor}{rgb}{0.000000,0.000000,0.000000}%
\pgfsetstrokecolor{textcolor}%
\pgfsetfillcolor{textcolor}%
\pgftext[x=0.849653in,y=2.571759in,left,base]{\color{textcolor}\rmfamily\fontsize{10.000000}{12.000000}\selectfont \(\displaystyle x^{0} + r\)}%
\end{pgfscope}%
\begin{pgfscope}%
\pgfsetrectcap%
\pgfsetroundjoin%
\pgfsetlinewidth{1.505625pt}%
\definecolor{currentstroke}{rgb}{1.000000,0.498039,0.054902}%
\pgfsetstrokecolor{currentstroke}%
\pgfsetdash{}{0pt}%
\pgfpathmoveto{\pgfqpoint{0.460764in}{2.419351in}}%
\pgfpathlineto{\pgfqpoint{0.738542in}{2.419351in}}%
\pgfusepath{stroke}%
\end{pgfscope}%
\begin{pgfscope}%
\definecolor{textcolor}{rgb}{0.000000,0.000000,0.000000}%
\pgfsetstrokecolor{textcolor}%
\pgfsetfillcolor{textcolor}%
\pgftext[x=0.849653in,y=2.370740in,left,base]{\color{textcolor}\rmfamily\fontsize{10.000000}{12.000000}\selectfont \(\displaystyle x^{1} + r\)}%
\end{pgfscope}%
\begin{pgfscope}%
\pgfsetrectcap%
\pgfsetroundjoin%
\pgfsetlinewidth{1.505625pt}%
\definecolor{currentstroke}{rgb}{0.172549,0.627451,0.172549}%
\pgfsetstrokecolor{currentstroke}%
\pgfsetdash{}{0pt}%
\pgfpathmoveto{\pgfqpoint{0.460764in}{2.218332in}}%
\pgfpathlineto{\pgfqpoint{0.738542in}{2.218332in}}%
\pgfusepath{stroke}%
\end{pgfscope}%
\begin{pgfscope}%
\definecolor{textcolor}{rgb}{0.000000,0.000000,0.000000}%
\pgfsetstrokecolor{textcolor}%
\pgfsetfillcolor{textcolor}%
\pgftext[x=0.849653in,y=2.169721in,left,base]{\color{textcolor}\rmfamily\fontsize{10.000000}{12.000000}\selectfont \(\displaystyle x^{2} + r\)}%
\end{pgfscope}%
\begin{pgfscope}%
\pgfsetrectcap%
\pgfsetroundjoin%
\pgfsetlinewidth{1.505625pt}%
\definecolor{currentstroke}{rgb}{0.839216,0.152941,0.156863}%
\pgfsetstrokecolor{currentstroke}%
\pgfsetdash{}{0pt}%
\pgfpathmoveto{\pgfqpoint{0.460764in}{2.017313in}}%
\pgfpathlineto{\pgfqpoint{0.738542in}{2.017313in}}%
\pgfusepath{stroke}%
\end{pgfscope}%
\begin{pgfscope}%
\definecolor{textcolor}{rgb}{0.000000,0.000000,0.000000}%
\pgfsetstrokecolor{textcolor}%
\pgfsetfillcolor{textcolor}%
\pgftext[x=0.849653in,y=1.968702in,left,base]{\color{textcolor}\rmfamily\fontsize{10.000000}{12.000000}\selectfont \(\displaystyle x^{3} + r\)}%
\end{pgfscope}%
\begin{pgfscope}%
\pgfsetbuttcap%
\pgfsetmiterjoin%
\definecolor{currentfill}{rgb}{1.000000,1.000000,1.000000}%
\pgfsetfillcolor{currentfill}%
\pgfsetlinewidth{0.000000pt}%
\definecolor{currentstroke}{rgb}{0.000000,0.000000,0.000000}%
\pgfsetstrokecolor{currentstroke}%
\pgfsetstrokeopacity{0.000000}%
\pgfsetdash{}{0pt}%
\pgfpathmoveto{\pgfqpoint{2.811806in}{0.386111in}}%
\pgfpathlineto{\pgfqpoint{5.040625in}{0.386111in}}%
\pgfpathlineto{\pgfqpoint{5.040625in}{2.801389in}}%
\pgfpathlineto{\pgfqpoint{2.811806in}{2.801389in}}%
\pgfpathclose%
\pgfusepath{fill}%
\end{pgfscope}%
\begin{pgfscope}%
\pgfsetbuttcap%
\pgfsetroundjoin%
\definecolor{currentfill}{rgb}{0.000000,0.000000,0.000000}%
\pgfsetfillcolor{currentfill}%
\pgfsetlinewidth{0.803000pt}%
\definecolor{currentstroke}{rgb}{0.000000,0.000000,0.000000}%
\pgfsetstrokecolor{currentstroke}%
\pgfsetdash{}{0pt}%
\pgfsys@defobject{currentmarker}{\pgfqpoint{0.000000in}{-0.048611in}}{\pgfqpoint{0.000000in}{0.000000in}}{%
\pgfpathmoveto{\pgfqpoint{0.000000in}{0.000000in}}%
\pgfpathlineto{\pgfqpoint{0.000000in}{-0.048611in}}%
\pgfusepath{stroke,fill}%
}%
\begin{pgfscope}%
\pgfsys@transformshift{2.913116in}{0.386111in}%
\pgfsys@useobject{currentmarker}{}%
\end{pgfscope}%
\end{pgfscope}%
\begin{pgfscope}%
\definecolor{textcolor}{rgb}{0.000000,0.000000,0.000000}%
\pgfsetstrokecolor{textcolor}%
\pgfsetfillcolor{textcolor}%
\pgftext[x=2.913116in,y=0.288889in,,top]{\color{textcolor}\rmfamily\fontsize{10.000000}{12.000000}\selectfont 0}%
\end{pgfscope}%
\begin{pgfscope}%
\pgfsetbuttcap%
\pgfsetroundjoin%
\definecolor{currentfill}{rgb}{0.000000,0.000000,0.000000}%
\pgfsetfillcolor{currentfill}%
\pgfsetlinewidth{0.803000pt}%
\definecolor{currentstroke}{rgb}{0.000000,0.000000,0.000000}%
\pgfsetstrokecolor{currentstroke}%
\pgfsetdash{}{0pt}%
\pgfsys@defobject{currentmarker}{\pgfqpoint{0.000000in}{-0.048611in}}{\pgfqpoint{0.000000in}{0.000000in}}{%
\pgfpathmoveto{\pgfqpoint{0.000000in}{0.000000in}}%
\pgfpathlineto{\pgfqpoint{0.000000in}{-0.048611in}}%
\pgfusepath{stroke,fill}%
}%
\begin{pgfscope}%
\pgfsys@transformshift{3.710832in}{0.386111in}%
\pgfsys@useobject{currentmarker}{}%
\end{pgfscope}%
\end{pgfscope}%
\begin{pgfscope}%
\definecolor{textcolor}{rgb}{0.000000,0.000000,0.000000}%
\pgfsetstrokecolor{textcolor}%
\pgfsetfillcolor{textcolor}%
\pgftext[x=3.710832in,y=0.288889in,,top]{\color{textcolor}\rmfamily\fontsize{10.000000}{12.000000}\selectfont 100}%
\end{pgfscope}%
\begin{pgfscope}%
\pgfsetbuttcap%
\pgfsetroundjoin%
\definecolor{currentfill}{rgb}{0.000000,0.000000,0.000000}%
\pgfsetfillcolor{currentfill}%
\pgfsetlinewidth{0.803000pt}%
\definecolor{currentstroke}{rgb}{0.000000,0.000000,0.000000}%
\pgfsetstrokecolor{currentstroke}%
\pgfsetdash{}{0pt}%
\pgfsys@defobject{currentmarker}{\pgfqpoint{0.000000in}{-0.048611in}}{\pgfqpoint{0.000000in}{0.000000in}}{%
\pgfpathmoveto{\pgfqpoint{0.000000in}{0.000000in}}%
\pgfpathlineto{\pgfqpoint{0.000000in}{-0.048611in}}%
\pgfusepath{stroke,fill}%
}%
\begin{pgfscope}%
\pgfsys@transformshift{4.508548in}{0.386111in}%
\pgfsys@useobject{currentmarker}{}%
\end{pgfscope}%
\end{pgfscope}%
\begin{pgfscope}%
\definecolor{textcolor}{rgb}{0.000000,0.000000,0.000000}%
\pgfsetstrokecolor{textcolor}%
\pgfsetfillcolor{textcolor}%
\pgftext[x=4.508548in,y=0.288889in,,top]{\color{textcolor}\rmfamily\fontsize{10.000000}{12.000000}\selectfont 200}%
\end{pgfscope}%
\begin{pgfscope}%
\pgfsetbuttcap%
\pgfsetroundjoin%
\definecolor{currentfill}{rgb}{0.000000,0.000000,0.000000}%
\pgfsetfillcolor{currentfill}%
\pgfsetlinewidth{0.803000pt}%
\definecolor{currentstroke}{rgb}{0.000000,0.000000,0.000000}%
\pgfsetstrokecolor{currentstroke}%
\pgfsetdash{}{0pt}%
\pgfsys@defobject{currentmarker}{\pgfqpoint{0.000000in}{0.000000in}}{\pgfqpoint{0.048611in}{0.000000in}}{%
\pgfpathmoveto{\pgfqpoint{0.000000in}{0.000000in}}%
\pgfpathlineto{\pgfqpoint{0.048611in}{0.000000in}}%
\pgfusepath{stroke,fill}%
}%
\begin{pgfscope}%
\pgfsys@transformshift{5.040625in}{0.446990in}%
\pgfsys@useobject{currentmarker}{}%
\end{pgfscope}%
\end{pgfscope}%
\begin{pgfscope}%
\definecolor{textcolor}{rgb}{0.000000,0.000000,0.000000}%
\pgfsetstrokecolor{textcolor}%
\pgfsetfillcolor{textcolor}%
\pgftext[x=5.137847in,y=0.398796in,left,base]{\color{textcolor}\rmfamily\fontsize{10.000000}{12.000000}\selectfont −0.050}%
\end{pgfscope}%
\begin{pgfscope}%
\pgfsetbuttcap%
\pgfsetroundjoin%
\definecolor{currentfill}{rgb}{0.000000,0.000000,0.000000}%
\pgfsetfillcolor{currentfill}%
\pgfsetlinewidth{0.803000pt}%
\definecolor{currentstroke}{rgb}{0.000000,0.000000,0.000000}%
\pgfsetstrokecolor{currentstroke}%
\pgfsetdash{}{0pt}%
\pgfsys@defobject{currentmarker}{\pgfqpoint{0.000000in}{0.000000in}}{\pgfqpoint{0.048611in}{0.000000in}}{%
\pgfpathmoveto{\pgfqpoint{0.000000in}{0.000000in}}%
\pgfpathlineto{\pgfqpoint{0.048611in}{0.000000in}}%
\pgfusepath{stroke,fill}%
}%
\begin{pgfscope}%
\pgfsys@transformshift{5.040625in}{0.765462in}%
\pgfsys@useobject{currentmarker}{}%
\end{pgfscope}%
\end{pgfscope}%
\begin{pgfscope}%
\definecolor{textcolor}{rgb}{0.000000,0.000000,0.000000}%
\pgfsetstrokecolor{textcolor}%
\pgfsetfillcolor{textcolor}%
\pgftext[x=5.137847in,y=0.717268in,left,base]{\color{textcolor}\rmfamily\fontsize{10.000000}{12.000000}\selectfont −0.025}%
\end{pgfscope}%
\begin{pgfscope}%
\pgfsetbuttcap%
\pgfsetroundjoin%
\definecolor{currentfill}{rgb}{0.000000,0.000000,0.000000}%
\pgfsetfillcolor{currentfill}%
\pgfsetlinewidth{0.803000pt}%
\definecolor{currentstroke}{rgb}{0.000000,0.000000,0.000000}%
\pgfsetstrokecolor{currentstroke}%
\pgfsetdash{}{0pt}%
\pgfsys@defobject{currentmarker}{\pgfqpoint{0.000000in}{0.000000in}}{\pgfqpoint{0.048611in}{0.000000in}}{%
\pgfpathmoveto{\pgfqpoint{0.000000in}{0.000000in}}%
\pgfpathlineto{\pgfqpoint{0.048611in}{0.000000in}}%
\pgfusepath{stroke,fill}%
}%
\begin{pgfscope}%
\pgfsys@transformshift{5.040625in}{1.083934in}%
\pgfsys@useobject{currentmarker}{}%
\end{pgfscope}%
\end{pgfscope}%
\begin{pgfscope}%
\definecolor{textcolor}{rgb}{0.000000,0.000000,0.000000}%
\pgfsetstrokecolor{textcolor}%
\pgfsetfillcolor{textcolor}%
\pgftext[x=5.137847in,y=1.035740in,left,base]{\color{textcolor}\rmfamily\fontsize{10.000000}{12.000000}\selectfont 0.000}%
\end{pgfscope}%
\begin{pgfscope}%
\pgfsetbuttcap%
\pgfsetroundjoin%
\definecolor{currentfill}{rgb}{0.000000,0.000000,0.000000}%
\pgfsetfillcolor{currentfill}%
\pgfsetlinewidth{0.803000pt}%
\definecolor{currentstroke}{rgb}{0.000000,0.000000,0.000000}%
\pgfsetstrokecolor{currentstroke}%
\pgfsetdash{}{0pt}%
\pgfsys@defobject{currentmarker}{\pgfqpoint{0.000000in}{0.000000in}}{\pgfqpoint{0.048611in}{0.000000in}}{%
\pgfpathmoveto{\pgfqpoint{0.000000in}{0.000000in}}%
\pgfpathlineto{\pgfqpoint{0.048611in}{0.000000in}}%
\pgfusepath{stroke,fill}%
}%
\begin{pgfscope}%
\pgfsys@transformshift{5.040625in}{1.402406in}%
\pgfsys@useobject{currentmarker}{}%
\end{pgfscope}%
\end{pgfscope}%
\begin{pgfscope}%
\definecolor{textcolor}{rgb}{0.000000,0.000000,0.000000}%
\pgfsetstrokecolor{textcolor}%
\pgfsetfillcolor{textcolor}%
\pgftext[x=5.137847in,y=1.354212in,left,base]{\color{textcolor}\rmfamily\fontsize{10.000000}{12.000000}\selectfont 0.025}%
\end{pgfscope}%
\begin{pgfscope}%
\pgfsetbuttcap%
\pgfsetroundjoin%
\definecolor{currentfill}{rgb}{0.000000,0.000000,0.000000}%
\pgfsetfillcolor{currentfill}%
\pgfsetlinewidth{0.803000pt}%
\definecolor{currentstroke}{rgb}{0.000000,0.000000,0.000000}%
\pgfsetstrokecolor{currentstroke}%
\pgfsetdash{}{0pt}%
\pgfsys@defobject{currentmarker}{\pgfqpoint{0.000000in}{0.000000in}}{\pgfqpoint{0.048611in}{0.000000in}}{%
\pgfpathmoveto{\pgfqpoint{0.000000in}{0.000000in}}%
\pgfpathlineto{\pgfqpoint{0.048611in}{0.000000in}}%
\pgfusepath{stroke,fill}%
}%
\begin{pgfscope}%
\pgfsys@transformshift{5.040625in}{1.720879in}%
\pgfsys@useobject{currentmarker}{}%
\end{pgfscope}%
\end{pgfscope}%
\begin{pgfscope}%
\definecolor{textcolor}{rgb}{0.000000,0.000000,0.000000}%
\pgfsetstrokecolor{textcolor}%
\pgfsetfillcolor{textcolor}%
\pgftext[x=5.137847in,y=1.672684in,left,base]{\color{textcolor}\rmfamily\fontsize{10.000000}{12.000000}\selectfont 0.050}%
\end{pgfscope}%
\begin{pgfscope}%
\pgfsetbuttcap%
\pgfsetroundjoin%
\definecolor{currentfill}{rgb}{0.000000,0.000000,0.000000}%
\pgfsetfillcolor{currentfill}%
\pgfsetlinewidth{0.803000pt}%
\definecolor{currentstroke}{rgb}{0.000000,0.000000,0.000000}%
\pgfsetstrokecolor{currentstroke}%
\pgfsetdash{}{0pt}%
\pgfsys@defobject{currentmarker}{\pgfqpoint{0.000000in}{0.000000in}}{\pgfqpoint{0.048611in}{0.000000in}}{%
\pgfpathmoveto{\pgfqpoint{0.000000in}{0.000000in}}%
\pgfpathlineto{\pgfqpoint{0.048611in}{0.000000in}}%
\pgfusepath{stroke,fill}%
}%
\begin{pgfscope}%
\pgfsys@transformshift{5.040625in}{2.039351in}%
\pgfsys@useobject{currentmarker}{}%
\end{pgfscope}%
\end{pgfscope}%
\begin{pgfscope}%
\definecolor{textcolor}{rgb}{0.000000,0.000000,0.000000}%
\pgfsetstrokecolor{textcolor}%
\pgfsetfillcolor{textcolor}%
\pgftext[x=5.137847in,y=1.991156in,left,base]{\color{textcolor}\rmfamily\fontsize{10.000000}{12.000000}\selectfont 0.075}%
\end{pgfscope}%
\begin{pgfscope}%
\pgfsetbuttcap%
\pgfsetroundjoin%
\definecolor{currentfill}{rgb}{0.000000,0.000000,0.000000}%
\pgfsetfillcolor{currentfill}%
\pgfsetlinewidth{0.803000pt}%
\definecolor{currentstroke}{rgb}{0.000000,0.000000,0.000000}%
\pgfsetstrokecolor{currentstroke}%
\pgfsetdash{}{0pt}%
\pgfsys@defobject{currentmarker}{\pgfqpoint{0.000000in}{0.000000in}}{\pgfqpoint{0.048611in}{0.000000in}}{%
\pgfpathmoveto{\pgfqpoint{0.000000in}{0.000000in}}%
\pgfpathlineto{\pgfqpoint{0.048611in}{0.000000in}}%
\pgfusepath{stroke,fill}%
}%
\begin{pgfscope}%
\pgfsys@transformshift{5.040625in}{2.357823in}%
\pgfsys@useobject{currentmarker}{}%
\end{pgfscope}%
\end{pgfscope}%
\begin{pgfscope}%
\definecolor{textcolor}{rgb}{0.000000,0.000000,0.000000}%
\pgfsetstrokecolor{textcolor}%
\pgfsetfillcolor{textcolor}%
\pgftext[x=5.137847in,y=2.309628in,left,base]{\color{textcolor}\rmfamily\fontsize{10.000000}{12.000000}\selectfont 0.100}%
\end{pgfscope}%
\begin{pgfscope}%
\pgfsetbuttcap%
\pgfsetroundjoin%
\definecolor{currentfill}{rgb}{0.000000,0.000000,0.000000}%
\pgfsetfillcolor{currentfill}%
\pgfsetlinewidth{0.803000pt}%
\definecolor{currentstroke}{rgb}{0.000000,0.000000,0.000000}%
\pgfsetstrokecolor{currentstroke}%
\pgfsetdash{}{0pt}%
\pgfsys@defobject{currentmarker}{\pgfqpoint{0.000000in}{0.000000in}}{\pgfqpoint{0.048611in}{0.000000in}}{%
\pgfpathmoveto{\pgfqpoint{0.000000in}{0.000000in}}%
\pgfpathlineto{\pgfqpoint{0.048611in}{0.000000in}}%
\pgfusepath{stroke,fill}%
}%
\begin{pgfscope}%
\pgfsys@transformshift{5.040625in}{2.676295in}%
\pgfsys@useobject{currentmarker}{}%
\end{pgfscope}%
\end{pgfscope}%
\begin{pgfscope}%
\definecolor{textcolor}{rgb}{0.000000,0.000000,0.000000}%
\pgfsetstrokecolor{textcolor}%
\pgfsetfillcolor{textcolor}%
\pgftext[x=5.137847in,y=2.628100in,left,base]{\color{textcolor}\rmfamily\fontsize{10.000000}{12.000000}\selectfont 0.125}%
\end{pgfscope}%
\begin{pgfscope}%
\pgfpathrectangle{\pgfqpoint{2.811806in}{0.386111in}}{\pgfqpoint{2.228819in}{2.415278in}}%
\pgfusepath{clip}%
\pgfsetrectcap%
\pgfsetroundjoin%
\pgfsetlinewidth{1.505625pt}%
\definecolor{currentstroke}{rgb}{0.121569,0.466667,0.705882}%
\pgfsetstrokecolor{currentstroke}%
\pgfsetdash{}{0pt}%
\pgfpathmoveto{\pgfqpoint{2.913116in}{1.619072in}}%
\pgfpathlineto{\pgfqpoint{2.921093in}{0.571781in}}%
\pgfpathlineto{\pgfqpoint{2.929070in}{1.549500in}}%
\pgfpathlineto{\pgfqpoint{2.937047in}{1.128494in}}%
\pgfpathlineto{\pgfqpoint{2.945024in}{1.012398in}}%
\pgfpathlineto{\pgfqpoint{2.953001in}{0.666471in}}%
\pgfpathlineto{\pgfqpoint{2.960979in}{1.162211in}}%
\pgfpathlineto{\pgfqpoint{2.968956in}{1.162273in}}%
\pgfpathlineto{\pgfqpoint{2.976933in}{1.405963in}}%
\pgfpathlineto{\pgfqpoint{2.984910in}{1.086190in}}%
\pgfpathlineto{\pgfqpoint{2.992887in}{0.602570in}}%
\pgfpathlineto{\pgfqpoint{3.000864in}{1.634602in}}%
\pgfpathlineto{\pgfqpoint{3.008841in}{0.577726in}}%
\pgfpathlineto{\pgfqpoint{3.016819in}{1.436194in}}%
\pgfpathlineto{\pgfqpoint{3.024796in}{0.636822in}}%
\pgfpathlineto{\pgfqpoint{3.032773in}{1.125647in}}%
\pgfpathlineto{\pgfqpoint{3.040750in}{1.510476in}}%
\pgfpathlineto{\pgfqpoint{3.048727in}{1.173048in}}%
\pgfpathlineto{\pgfqpoint{3.056704in}{0.962205in}}%
\pgfpathlineto{\pgfqpoint{3.064682in}{0.657129in}}%
\pgfpathlineto{\pgfqpoint{3.080636in}{1.617888in}}%
\pgfpathlineto{\pgfqpoint{3.088613in}{0.899921in}}%
\pgfpathlineto{\pgfqpoint{3.096590in}{0.901773in}}%
\pgfpathlineto{\pgfqpoint{3.104567in}{1.001047in}}%
\pgfpathlineto{\pgfqpoint{3.112545in}{1.533347in}}%
\pgfpathlineto{\pgfqpoint{3.120522in}{0.495896in}}%
\pgfpathlineto{\pgfqpoint{3.128499in}{1.115001in}}%
\pgfpathlineto{\pgfqpoint{3.136476in}{1.220809in}}%
\pgfpathlineto{\pgfqpoint{3.144453in}{1.129291in}}%
\pgfpathlineto{\pgfqpoint{3.152430in}{1.288416in}}%
\pgfpathlineto{\pgfqpoint{3.160408in}{1.041823in}}%
\pgfpathlineto{\pgfqpoint{3.168385in}{1.045163in}}%
\pgfpathlineto{\pgfqpoint{3.176362in}{1.013616in}}%
\pgfpathlineto{\pgfqpoint{3.184339in}{0.967938in}}%
\pgfpathlineto{\pgfqpoint{3.192316in}{1.087652in}}%
\pgfpathlineto{\pgfqpoint{3.200293in}{1.302815in}}%
\pgfpathlineto{\pgfqpoint{3.208271in}{0.957813in}}%
\pgfpathlineto{\pgfqpoint{3.216248in}{0.904711in}}%
\pgfpathlineto{\pgfqpoint{3.224225in}{1.377152in}}%
\pgfpathlineto{\pgfqpoint{3.232202in}{1.030969in}}%
\pgfpathlineto{\pgfqpoint{3.240179in}{0.895827in}}%
\pgfpathlineto{\pgfqpoint{3.248156in}{1.020074in}}%
\pgfpathlineto{\pgfqpoint{3.256134in}{1.501142in}}%
\pgfpathlineto{\pgfqpoint{3.264111in}{0.765126in}}%
\pgfpathlineto{\pgfqpoint{3.272088in}{1.234854in}}%
\pgfpathlineto{\pgfqpoint{3.280065in}{0.775914in}}%
\pgfpathlineto{\pgfqpoint{3.288042in}{1.237926in}}%
\pgfpathlineto{\pgfqpoint{3.296019in}{0.962643in}}%
\pgfpathlineto{\pgfqpoint{3.303997in}{1.446901in}}%
\pgfpathlineto{\pgfqpoint{3.311974in}{0.966830in}}%
\pgfpathlineto{\pgfqpoint{3.319951in}{1.175801in}}%
\pgfpathlineto{\pgfqpoint{3.327928in}{0.772203in}}%
\pgfpathlineto{\pgfqpoint{3.335905in}{1.194844in}}%
\pgfpathlineto{\pgfqpoint{3.343882in}{1.087420in}}%
\pgfpathlineto{\pgfqpoint{3.351860in}{1.106073in}}%
\pgfpathlineto{\pgfqpoint{3.359837in}{1.397938in}}%
\pgfpathlineto{\pgfqpoint{3.375791in}{0.929263in}}%
\pgfpathlineto{\pgfqpoint{3.383768in}{0.944343in}}%
\pgfpathlineto{\pgfqpoint{3.391745in}{0.885626in}}%
\pgfpathlineto{\pgfqpoint{3.399722in}{1.400357in}}%
\pgfpathlineto{\pgfqpoint{3.407700in}{0.966216in}}%
\pgfpathlineto{\pgfqpoint{3.415677in}{1.252279in}}%
\pgfpathlineto{\pgfqpoint{3.423654in}{0.900296in}}%
\pgfpathlineto{\pgfqpoint{3.431631in}{0.916846in}}%
\pgfpathlineto{\pgfqpoint{3.439608in}{1.548879in}}%
\pgfpathlineto{\pgfqpoint{3.447585in}{0.762935in}}%
\pgfpathlineto{\pgfqpoint{3.455563in}{0.962758in}}%
\pgfpathlineto{\pgfqpoint{3.463540in}{1.509679in}}%
\pgfpathlineto{\pgfqpoint{3.471517in}{1.056120in}}%
\pgfpathlineto{\pgfqpoint{3.479494in}{0.975206in}}%
\pgfpathlineto{\pgfqpoint{3.487471in}{1.277760in}}%
\pgfpathlineto{\pgfqpoint{3.495448in}{1.040365in}}%
\pgfpathlineto{\pgfqpoint{3.503426in}{0.993396in}}%
\pgfpathlineto{\pgfqpoint{3.511403in}{0.803619in}}%
\pgfpathlineto{\pgfqpoint{3.519380in}{1.386041in}}%
\pgfpathlineto{\pgfqpoint{3.527357in}{1.028142in}}%
\pgfpathlineto{\pgfqpoint{3.535334in}{0.957736in}}%
\pgfpathlineto{\pgfqpoint{3.543311in}{1.001045in}}%
\pgfpathlineto{\pgfqpoint{3.551289in}{1.028720in}}%
\pgfpathlineto{\pgfqpoint{3.559266in}{1.469084in}}%
\pgfpathlineto{\pgfqpoint{3.567243in}{0.939401in}}%
\pgfpathlineto{\pgfqpoint{3.575220in}{0.671566in}}%
\pgfpathlineto{\pgfqpoint{3.583197in}{1.499438in}}%
\pgfpathlineto{\pgfqpoint{3.591174in}{1.063793in}}%
\pgfpathlineto{\pgfqpoint{3.599152in}{0.815782in}}%
\pgfpathlineto{\pgfqpoint{3.607129in}{1.485236in}}%
\pgfpathlineto{\pgfqpoint{3.615106in}{0.883370in}}%
\pgfpathlineto{\pgfqpoint{3.623083in}{0.780733in}}%
\pgfpathlineto{\pgfqpoint{3.631060in}{1.474255in}}%
\pgfpathlineto{\pgfqpoint{3.639037in}{0.997272in}}%
\pgfpathlineto{\pgfqpoint{3.647015in}{1.000436in}}%
\pgfpathlineto{\pgfqpoint{3.654992in}{1.117414in}}%
\pgfpathlineto{\pgfqpoint{3.662969in}{0.807622in}}%
\pgfpathlineto{\pgfqpoint{3.670946in}{1.093011in}}%
\pgfpathlineto{\pgfqpoint{3.678923in}{1.684800in}}%
\pgfpathlineto{\pgfqpoint{3.686900in}{0.603502in}}%
\pgfpathlineto{\pgfqpoint{3.694878in}{1.062104in}}%
\pgfpathlineto{\pgfqpoint{3.702855in}{1.383972in}}%
\pgfpathlineto{\pgfqpoint{3.710832in}{0.667000in}}%
\pgfpathlineto{\pgfqpoint{3.718809in}{1.326245in}}%
\pgfpathlineto{\pgfqpoint{3.726786in}{1.265854in}}%
\pgfpathlineto{\pgfqpoint{3.734763in}{0.986665in}}%
\pgfpathlineto{\pgfqpoint{3.742741in}{1.173048in}}%
\pgfpathlineto{\pgfqpoint{3.750718in}{0.956859in}}%
\pgfpathlineto{\pgfqpoint{3.758695in}{0.886647in}}%
\pgfpathlineto{\pgfqpoint{3.766672in}{1.151377in}}%
\pgfpathlineto{\pgfqpoint{3.774649in}{1.127556in}}%
\pgfpathlineto{\pgfqpoint{3.782626in}{1.417464in}}%
\pgfpathlineto{\pgfqpoint{3.790604in}{0.869354in}}%
\pgfpathlineto{\pgfqpoint{3.798581in}{0.788986in}}%
\pgfpathlineto{\pgfqpoint{3.806558in}{1.607715in}}%
\pgfpathlineto{\pgfqpoint{3.814535in}{0.798690in}}%
\pgfpathlineto{\pgfqpoint{3.822512in}{1.147904in}}%
\pgfpathlineto{\pgfqpoint{3.830489in}{0.830574in}}%
\pgfpathlineto{\pgfqpoint{3.838466in}{1.161987in}}%
\pgfpathlineto{\pgfqpoint{3.846444in}{1.292612in}}%
\pgfpathlineto{\pgfqpoint{3.854421in}{1.050501in}}%
\pgfpathlineto{\pgfqpoint{3.862398in}{1.146947in}}%
\pgfpathlineto{\pgfqpoint{3.870375in}{0.736558in}}%
\pgfpathlineto{\pgfqpoint{3.878352in}{1.079799in}}%
\pgfpathlineto{\pgfqpoint{3.886329in}{1.536294in}}%
\pgfpathlineto{\pgfqpoint{3.894307in}{1.053208in}}%
\pgfpathlineto{\pgfqpoint{3.902284in}{0.792272in}}%
\pgfpathlineto{\pgfqpoint{3.910261in}{1.477023in}}%
\pgfpathlineto{\pgfqpoint{3.926215in}{0.658507in}}%
\pgfpathlineto{\pgfqpoint{3.934192in}{1.449158in}}%
\pgfpathlineto{\pgfqpoint{3.942170in}{1.189795in}}%
\pgfpathlineto{\pgfqpoint{3.950147in}{0.729453in}}%
\pgfpathlineto{\pgfqpoint{3.958124in}{1.004094in}}%
\pgfpathlineto{\pgfqpoint{3.966101in}{1.363799in}}%
\pgfpathlineto{\pgfqpoint{3.974078in}{0.991680in}}%
\pgfpathlineto{\pgfqpoint{3.982055in}{0.754390in}}%
\pgfpathlineto{\pgfqpoint{3.990033in}{1.415235in}}%
\pgfpathlineto{\pgfqpoint{3.998010in}{0.752250in}}%
\pgfpathlineto{\pgfqpoint{4.005987in}{1.126482in}}%
\pgfpathlineto{\pgfqpoint{4.013964in}{1.178553in}}%
\pgfpathlineto{\pgfqpoint{4.021941in}{1.106625in}}%
\pgfpathlineto{\pgfqpoint{4.029918in}{1.480201in}}%
\pgfpathlineto{\pgfqpoint{4.037896in}{0.741452in}}%
\pgfpathlineto{\pgfqpoint{4.045873in}{0.913437in}}%
\pgfpathlineto{\pgfqpoint{4.053850in}{1.220057in}}%
\pgfpathlineto{\pgfqpoint{4.061827in}{0.925346in}}%
\pgfpathlineto{\pgfqpoint{4.069804in}{1.219978in}}%
\pgfpathlineto{\pgfqpoint{4.077781in}{1.221363in}}%
\pgfpathlineto{\pgfqpoint{4.085759in}{1.025840in}}%
\pgfpathlineto{\pgfqpoint{4.093736in}{1.286360in}}%
\pgfpathlineto{\pgfqpoint{4.101713in}{0.975537in}}%
\pgfpathlineto{\pgfqpoint{4.109690in}{1.314735in}}%
\pgfpathlineto{\pgfqpoint{4.117667in}{0.820272in}}%
\pgfpathlineto{\pgfqpoint{4.125644in}{0.959684in}}%
\pgfpathlineto{\pgfqpoint{4.133622in}{1.434427in}}%
\pgfpathlineto{\pgfqpoint{4.141599in}{1.037125in}}%
\pgfpathlineto{\pgfqpoint{4.149576in}{0.807529in}}%
\pgfpathlineto{\pgfqpoint{4.157553in}{1.411438in}}%
\pgfpathlineto{\pgfqpoint{4.165530in}{1.002272in}}%
\pgfpathlineto{\pgfqpoint{4.173507in}{1.144848in}}%
\pgfpathlineto{\pgfqpoint{4.181485in}{0.664457in}}%
\pgfpathlineto{\pgfqpoint{4.189462in}{1.534848in}}%
\pgfpathlineto{\pgfqpoint{4.197439in}{0.827906in}}%
\pgfpathlineto{\pgfqpoint{4.205416in}{1.192835in}}%
\pgfpathlineto{\pgfqpoint{4.213393in}{0.710454in}}%
\pgfpathlineto{\pgfqpoint{4.221370in}{1.353281in}}%
\pgfpathlineto{\pgfqpoint{4.229347in}{0.765187in}}%
\pgfpathlineto{\pgfqpoint{4.237325in}{1.439543in}}%
\pgfpathlineto{\pgfqpoint{4.245302in}{1.165643in}}%
\pgfpathlineto{\pgfqpoint{4.253279in}{0.981463in}}%
\pgfpathlineto{\pgfqpoint{4.261256in}{0.906485in}}%
\pgfpathlineto{\pgfqpoint{4.269233in}{1.049128in}}%
\pgfpathlineto{\pgfqpoint{4.277210in}{1.087883in}}%
\pgfpathlineto{\pgfqpoint{4.285188in}{0.973614in}}%
\pgfpathlineto{\pgfqpoint{4.293165in}{1.129493in}}%
\pgfpathlineto{\pgfqpoint{4.301142in}{1.599572in}}%
\pgfpathlineto{\pgfqpoint{4.309119in}{0.742048in}}%
\pgfpathlineto{\pgfqpoint{4.317096in}{0.925042in}}%
\pgfpathlineto{\pgfqpoint{4.325073in}{1.050081in}}%
\pgfpathlineto{\pgfqpoint{4.333051in}{1.203054in}}%
\pgfpathlineto{\pgfqpoint{4.341028in}{1.095369in}}%
\pgfpathlineto{\pgfqpoint{4.349005in}{1.105378in}}%
\pgfpathlineto{\pgfqpoint{4.356982in}{1.332903in}}%
\pgfpathlineto{\pgfqpoint{4.364959in}{1.012345in}}%
\pgfpathlineto{\pgfqpoint{4.372936in}{0.869091in}}%
\pgfpathlineto{\pgfqpoint{4.380914in}{1.183894in}}%
\pgfpathlineto{\pgfqpoint{4.388891in}{0.988679in}}%
\pgfpathlineto{\pgfqpoint{4.396868in}{1.405163in}}%
\pgfpathlineto{\pgfqpoint{4.404845in}{1.186910in}}%
\pgfpathlineto{\pgfqpoint{4.412822in}{0.720460in}}%
\pgfpathlineto{\pgfqpoint{4.420799in}{1.382468in}}%
\pgfpathlineto{\pgfqpoint{4.428777in}{1.175008in}}%
\pgfpathlineto{\pgfqpoint{4.436754in}{0.701322in}}%
\pgfpathlineto{\pgfqpoint{4.444731in}{1.458328in}}%
\pgfpathlineto{\pgfqpoint{4.452708in}{0.645531in}}%
\pgfpathlineto{\pgfqpoint{4.460685in}{1.458691in}}%
\pgfpathlineto{\pgfqpoint{4.476640in}{0.682164in}}%
\pgfpathlineto{\pgfqpoint{4.484617in}{1.419556in}}%
\pgfpathlineto{\pgfqpoint{4.492594in}{1.176325in}}%
\pgfpathlineto{\pgfqpoint{4.500571in}{0.796910in}}%
\pgfpathlineto{\pgfqpoint{4.508548in}{1.080692in}}%
\pgfpathlineto{\pgfqpoint{4.516525in}{1.103522in}}%
\pgfpathlineto{\pgfqpoint{4.524503in}{0.803593in}}%
\pgfpathlineto{\pgfqpoint{4.532480in}{1.211127in}}%
\pgfpathlineto{\pgfqpoint{4.540457in}{1.132343in}}%
\pgfpathlineto{\pgfqpoint{4.548434in}{1.153587in}}%
\pgfpathlineto{\pgfqpoint{4.556411in}{1.007523in}}%
\pgfpathlineto{\pgfqpoint{4.564388in}{1.207658in}}%
\pgfpathlineto{\pgfqpoint{4.572366in}{1.258301in}}%
\pgfpathlineto{\pgfqpoint{4.580343in}{0.734595in}}%
\pgfpathlineto{\pgfqpoint{4.588320in}{1.031486in}}%
\pgfpathlineto{\pgfqpoint{4.596297in}{1.493837in}}%
\pgfpathlineto{\pgfqpoint{4.604274in}{1.190834in}}%
\pgfpathlineto{\pgfqpoint{4.612251in}{0.757965in}}%
\pgfpathlineto{\pgfqpoint{4.620228in}{1.393035in}}%
\pgfpathlineto{\pgfqpoint{4.628206in}{0.505787in}}%
\pgfpathlineto{\pgfqpoint{4.636183in}{1.694073in}}%
\pgfpathlineto{\pgfqpoint{4.644160in}{0.580262in}}%
\pgfpathlineto{\pgfqpoint{4.652137in}{0.962034in}}%
\pgfpathlineto{\pgfqpoint{4.660114in}{1.266586in}}%
\pgfpathlineto{\pgfqpoint{4.668091in}{1.299794in}}%
\pgfpathlineto{\pgfqpoint{4.676069in}{0.858691in}}%
\pgfpathlineto{\pgfqpoint{4.684046in}{1.109259in}}%
\pgfpathlineto{\pgfqpoint{4.692023in}{0.967201in}}%
\pgfpathlineto{\pgfqpoint{4.700000in}{1.144473in}}%
\pgfpathlineto{\pgfqpoint{4.707977in}{1.012626in}}%
\pgfpathlineto{\pgfqpoint{4.715954in}{1.277050in}}%
\pgfpathlineto{\pgfqpoint{4.723932in}{0.911334in}}%
\pgfpathlineto{\pgfqpoint{4.731909in}{1.056172in}}%
\pgfpathlineto{\pgfqpoint{4.739886in}{1.473671in}}%
\pgfpathlineto{\pgfqpoint{4.747863in}{0.679510in}}%
\pgfpathlineto{\pgfqpoint{4.755840in}{1.172096in}}%
\pgfpathlineto{\pgfqpoint{4.763817in}{1.412979in}}%
\pgfpathlineto{\pgfqpoint{4.771795in}{0.825340in}}%
\pgfpathlineto{\pgfqpoint{4.779772in}{1.223004in}}%
\pgfpathlineto{\pgfqpoint{4.787749in}{0.956832in}}%
\pgfpathlineto{\pgfqpoint{4.795726in}{1.149994in}}%
\pgfpathlineto{\pgfqpoint{4.803703in}{1.224412in}}%
\pgfpathlineto{\pgfqpoint{4.811680in}{0.886691in}}%
\pgfpathlineto{\pgfqpoint{4.819658in}{0.972879in}}%
\pgfpathlineto{\pgfqpoint{4.827635in}{1.562147in}}%
\pgfpathlineto{\pgfqpoint{4.835612in}{1.016372in}}%
\pgfpathlineto{\pgfqpoint{4.843589in}{0.704407in}}%
\pgfpathlineto{\pgfqpoint{4.851566in}{1.402074in}}%
\pgfpathlineto{\pgfqpoint{4.859543in}{0.921351in}}%
\pgfpathlineto{\pgfqpoint{4.867521in}{1.099416in}}%
\pgfpathlineto{\pgfqpoint{4.875498in}{1.030141in}}%
\pgfpathlineto{\pgfqpoint{4.883475in}{1.346269in}}%
\pgfpathlineto{\pgfqpoint{4.891452in}{0.758891in}}%
\pgfpathlineto{\pgfqpoint{4.899429in}{0.955509in}}%
\pgfpathlineto{\pgfqpoint{4.907406in}{1.609251in}}%
\pgfpathlineto{\pgfqpoint{4.915384in}{1.055933in}}%
\pgfpathlineto{\pgfqpoint{4.923361in}{0.889343in}}%
\pgfpathlineto{\pgfqpoint{4.931338in}{1.082736in}}%
\pgfpathlineto{\pgfqpoint{4.939315in}{1.180489in}}%
\pgfpathlineto{\pgfqpoint{4.939315in}{1.180489in}}%
\pgfusepath{stroke}%
\end{pgfscope}%
\begin{pgfscope}%
\pgfpathrectangle{\pgfqpoint{2.811806in}{0.386111in}}{\pgfqpoint{2.228819in}{2.415278in}}%
\pgfusepath{clip}%
\pgfsetrectcap%
\pgfsetroundjoin%
\pgfsetlinewidth{1.505625pt}%
\definecolor{currentstroke}{rgb}{1.000000,0.498039,0.054902}%
\pgfsetstrokecolor{currentstroke}%
\pgfsetdash{}{0pt}%
\pgfpathmoveto{\pgfqpoint{2.913116in}{1.152827in}}%
\pgfpathlineto{\pgfqpoint{2.921093in}{1.452750in}}%
\pgfpathlineto{\pgfqpoint{2.929070in}{1.322918in}}%
\pgfpathlineto{\pgfqpoint{2.937047in}{1.222611in}}%
\pgfpathlineto{\pgfqpoint{2.945024in}{1.156786in}}%
\pgfpathlineto{\pgfqpoint{2.953001in}{1.195489in}}%
\pgfpathlineto{\pgfqpoint{2.960979in}{0.700699in}}%
\pgfpathlineto{\pgfqpoint{2.968956in}{1.645845in}}%
\pgfpathlineto{\pgfqpoint{2.976933in}{1.240290in}}%
\pgfpathlineto{\pgfqpoint{2.984910in}{0.659218in}}%
\pgfpathlineto{\pgfqpoint{2.992887in}{1.691452in}}%
\pgfpathlineto{\pgfqpoint{3.000864in}{0.602046in}}%
\pgfpathlineto{\pgfqpoint{3.008841in}{1.272942in}}%
\pgfpathlineto{\pgfqpoint{3.016819in}{1.645053in}}%
\pgfpathlineto{\pgfqpoint{3.024796in}{0.826944in}}%
\pgfpathlineto{\pgfqpoint{3.032773in}{1.420375in}}%
\pgfpathlineto{\pgfqpoint{3.040750in}{1.277535in}}%
\pgfpathlineto{\pgfqpoint{3.048727in}{0.897368in}}%
\pgfpathlineto{\pgfqpoint{3.056704in}{1.076941in}}%
\pgfpathlineto{\pgfqpoint{3.064682in}{1.340507in}}%
\pgfpathlineto{\pgfqpoint{3.072659in}{1.328254in}}%
\pgfpathlineto{\pgfqpoint{3.080636in}{0.913938in}}%
\pgfpathlineto{\pgfqpoint{3.088613in}{1.297027in}}%
\pgfpathlineto{\pgfqpoint{3.096590in}{1.403472in}}%
\pgfpathlineto{\pgfqpoint{3.104567in}{1.271381in}}%
\pgfpathlineto{\pgfqpoint{3.112545in}{1.177776in}}%
\pgfpathlineto{\pgfqpoint{3.120522in}{0.834605in}}%
\pgfpathlineto{\pgfqpoint{3.128499in}{1.281533in}}%
\pgfpathlineto{\pgfqpoint{3.136476in}{0.852162in}}%
\pgfpathlineto{\pgfqpoint{3.152430in}{1.591704in}}%
\pgfpathlineto{\pgfqpoint{3.168385in}{0.828926in}}%
\pgfpathlineto{\pgfqpoint{3.176362in}{1.473702in}}%
\pgfpathlineto{\pgfqpoint{3.184339in}{1.081742in}}%
\pgfpathlineto{\pgfqpoint{3.192316in}{1.361289in}}%
\pgfpathlineto{\pgfqpoint{3.200293in}{0.850154in}}%
\pgfpathlineto{\pgfqpoint{3.208271in}{1.235414in}}%
\pgfpathlineto{\pgfqpoint{3.216248in}{1.269035in}}%
\pgfpathlineto{\pgfqpoint{3.224225in}{0.995460in}}%
\pgfpathlineto{\pgfqpoint{3.240179in}{1.413038in}}%
\pgfpathlineto{\pgfqpoint{3.248156in}{0.965923in}}%
\pgfpathlineto{\pgfqpoint{3.256134in}{1.231343in}}%
\pgfpathlineto{\pgfqpoint{3.264111in}{1.220702in}}%
\pgfpathlineto{\pgfqpoint{3.272088in}{1.548200in}}%
\pgfpathlineto{\pgfqpoint{3.280065in}{0.755223in}}%
\pgfpathlineto{\pgfqpoint{3.288042in}{1.628006in}}%
\pgfpathlineto{\pgfqpoint{3.296019in}{0.767769in}}%
\pgfpathlineto{\pgfqpoint{3.303997in}{1.572194in}}%
\pgfpathlineto{\pgfqpoint{3.311974in}{0.649723in}}%
\pgfpathlineto{\pgfqpoint{3.319951in}{1.561316in}}%
\pgfpathlineto{\pgfqpoint{3.327928in}{1.153335in}}%
\pgfpathlineto{\pgfqpoint{3.335905in}{0.988264in}}%
\pgfpathlineto{\pgfqpoint{3.343882in}{1.338545in}}%
\pgfpathlineto{\pgfqpoint{3.351860in}{1.395411in}}%
\pgfpathlineto{\pgfqpoint{3.359837in}{0.941050in}}%
\pgfpathlineto{\pgfqpoint{3.367814in}{1.010774in}}%
\pgfpathlineto{\pgfqpoint{3.383768in}{1.205536in}}%
\pgfpathlineto{\pgfqpoint{3.391745in}{1.303042in}}%
\pgfpathlineto{\pgfqpoint{3.399722in}{1.140858in}}%
\pgfpathlineto{\pgfqpoint{3.407700in}{1.579129in}}%
\pgfpathlineto{\pgfqpoint{3.415677in}{0.655737in}}%
\pgfpathlineto{\pgfqpoint{3.423654in}{1.720211in}}%
\pgfpathlineto{\pgfqpoint{3.431631in}{1.079010in}}%
\pgfpathlineto{\pgfqpoint{3.439608in}{1.116948in}}%
\pgfpathlineto{\pgfqpoint{3.447585in}{0.784442in}}%
\pgfpathlineto{\pgfqpoint{3.455563in}{1.299545in}}%
\pgfpathlineto{\pgfqpoint{3.463540in}{1.239556in}}%
\pgfpathlineto{\pgfqpoint{3.471517in}{1.499178in}}%
\pgfpathlineto{\pgfqpoint{3.479494in}{0.781916in}}%
\pgfpathlineto{\pgfqpoint{3.487471in}{1.109328in}}%
\pgfpathlineto{\pgfqpoint{3.495448in}{1.532547in}}%
\pgfpathlineto{\pgfqpoint{3.503426in}{0.847006in}}%
\pgfpathlineto{\pgfqpoint{3.511403in}{1.508002in}}%
\pgfpathlineto{\pgfqpoint{3.519380in}{1.183912in}}%
\pgfpathlineto{\pgfqpoint{3.527357in}{1.069811in}}%
\pgfpathlineto{\pgfqpoint{3.535334in}{1.086021in}}%
\pgfpathlineto{\pgfqpoint{3.543311in}{1.389547in}}%
\pgfpathlineto{\pgfqpoint{3.551289in}{1.170452in}}%
\pgfpathlineto{\pgfqpoint{3.559266in}{1.383167in}}%
\pgfpathlineto{\pgfqpoint{3.567243in}{0.713898in}}%
\pgfpathlineto{\pgfqpoint{3.575220in}{1.238780in}}%
\pgfpathlineto{\pgfqpoint{3.583197in}{1.353252in}}%
\pgfpathlineto{\pgfqpoint{3.591174in}{1.363142in}}%
\pgfpathlineto{\pgfqpoint{3.599152in}{1.134805in}}%
\pgfpathlineto{\pgfqpoint{3.607129in}{1.106199in}}%
\pgfpathlineto{\pgfqpoint{3.615106in}{1.153350in}}%
\pgfpathlineto{\pgfqpoint{3.623083in}{1.037257in}}%
\pgfpathlineto{\pgfqpoint{3.631060in}{1.440085in}}%
\pgfpathlineto{\pgfqpoint{3.639037in}{1.031908in}}%
\pgfpathlineto{\pgfqpoint{3.647015in}{0.928212in}}%
\pgfpathlineto{\pgfqpoint{3.654992in}{1.535743in}}%
\pgfpathlineto{\pgfqpoint{3.662969in}{0.896855in}}%
\pgfpathlineto{\pgfqpoint{3.670946in}{1.494422in}}%
\pgfpathlineto{\pgfqpoint{3.678923in}{1.104058in}}%
\pgfpathlineto{\pgfqpoint{3.686900in}{0.890751in}}%
\pgfpathlineto{\pgfqpoint{3.694878in}{1.641692in}}%
\pgfpathlineto{\pgfqpoint{3.702855in}{0.699533in}}%
\pgfpathlineto{\pgfqpoint{3.710832in}{1.202337in}}%
\pgfpathlineto{\pgfqpoint{3.718809in}{1.387221in}}%
\pgfpathlineto{\pgfqpoint{3.726786in}{1.142883in}}%
\pgfpathlineto{\pgfqpoint{3.734763in}{1.601970in}}%
\pgfpathlineto{\pgfqpoint{3.742741in}{0.749887in}}%
\pgfpathlineto{\pgfqpoint{3.750718in}{1.167740in}}%
\pgfpathlineto{\pgfqpoint{3.758695in}{1.147378in}}%
\pgfpathlineto{\pgfqpoint{3.766672in}{1.099212in}}%
\pgfpathlineto{\pgfqpoint{3.774649in}{1.191381in}}%
\pgfpathlineto{\pgfqpoint{3.782626in}{1.637843in}}%
\pgfpathlineto{\pgfqpoint{3.790604in}{0.999573in}}%
\pgfpathlineto{\pgfqpoint{3.798581in}{1.192131in}}%
\pgfpathlineto{\pgfqpoint{3.806558in}{1.133071in}}%
\pgfpathlineto{\pgfqpoint{3.814535in}{0.966527in}}%
\pgfpathlineto{\pgfqpoint{3.822512in}{1.323140in}}%
\pgfpathlineto{\pgfqpoint{3.830489in}{1.041950in}}%
\pgfpathlineto{\pgfqpoint{3.838466in}{1.478936in}}%
\pgfpathlineto{\pgfqpoint{3.846444in}{1.120498in}}%
\pgfpathlineto{\pgfqpoint{3.854421in}{1.069800in}}%
\pgfpathlineto{\pgfqpoint{3.862398in}{1.604860in}}%
\pgfpathlineto{\pgfqpoint{3.870375in}{1.129825in}}%
\pgfpathlineto{\pgfqpoint{3.878352in}{0.887100in}}%
\pgfpathlineto{\pgfqpoint{3.886329in}{1.359103in}}%
\pgfpathlineto{\pgfqpoint{3.894307in}{0.848428in}}%
\pgfpathlineto{\pgfqpoint{3.902284in}{1.307214in}}%
\pgfpathlineto{\pgfqpoint{3.910261in}{1.190460in}}%
\pgfpathlineto{\pgfqpoint{3.918238in}{1.268205in}}%
\pgfpathlineto{\pgfqpoint{3.926215in}{1.083575in}}%
\pgfpathlineto{\pgfqpoint{3.934192in}{1.084255in}}%
\pgfpathlineto{\pgfqpoint{3.942170in}{1.699559in}}%
\pgfpathlineto{\pgfqpoint{3.950147in}{0.684750in}}%
\pgfpathlineto{\pgfqpoint{3.958124in}{1.657959in}}%
\pgfpathlineto{\pgfqpoint{3.966101in}{1.104789in}}%
\pgfpathlineto{\pgfqpoint{3.974078in}{0.875901in}}%
\pgfpathlineto{\pgfqpoint{3.982055in}{1.414787in}}%
\pgfpathlineto{\pgfqpoint{3.990033in}{0.833824in}}%
\pgfpathlineto{\pgfqpoint{3.998010in}{1.348461in}}%
\pgfpathlineto{\pgfqpoint{4.005987in}{1.445287in}}%
\pgfpathlineto{\pgfqpoint{4.013964in}{0.908315in}}%
\pgfpathlineto{\pgfqpoint{4.021941in}{1.108706in}}%
\pgfpathlineto{\pgfqpoint{4.029918in}{1.135401in}}%
\pgfpathlineto{\pgfqpoint{4.037896in}{1.201737in}}%
\pgfpathlineto{\pgfqpoint{4.045873in}{1.640079in}}%
\pgfpathlineto{\pgfqpoint{4.053850in}{1.155997in}}%
\pgfpathlineto{\pgfqpoint{4.061827in}{1.041394in}}%
\pgfpathlineto{\pgfqpoint{4.069804in}{1.213025in}}%
\pgfpathlineto{\pgfqpoint{4.077781in}{0.790788in}}%
\pgfpathlineto{\pgfqpoint{4.085759in}{1.508047in}}%
\pgfpathlineto{\pgfqpoint{4.093736in}{0.854116in}}%
\pgfpathlineto{\pgfqpoint{4.101713in}{1.636949in}}%
\pgfpathlineto{\pgfqpoint{4.109690in}{0.806380in}}%
\pgfpathlineto{\pgfqpoint{4.125644in}{1.604044in}}%
\pgfpathlineto{\pgfqpoint{4.133622in}{1.010197in}}%
\pgfpathlineto{\pgfqpoint{4.141599in}{1.401227in}}%
\pgfpathlineto{\pgfqpoint{4.149576in}{1.039793in}}%
\pgfpathlineto{\pgfqpoint{4.157553in}{1.290414in}}%
\pgfpathlineto{\pgfqpoint{4.165530in}{0.673362in}}%
\pgfpathlineto{\pgfqpoint{4.173507in}{1.615375in}}%
\pgfpathlineto{\pgfqpoint{4.181485in}{0.705591in}}%
\pgfpathlineto{\pgfqpoint{4.189462in}{1.312951in}}%
\pgfpathlineto{\pgfqpoint{4.197439in}{1.220694in}}%
\pgfpathlineto{\pgfqpoint{4.205416in}{1.311831in}}%
\pgfpathlineto{\pgfqpoint{4.213393in}{1.076014in}}%
\pgfpathlineto{\pgfqpoint{4.221370in}{1.102049in}}%
\pgfpathlineto{\pgfqpoint{4.229347in}{1.154291in}}%
\pgfpathlineto{\pgfqpoint{4.237325in}{1.085931in}}%
\pgfpathlineto{\pgfqpoint{4.245302in}{1.600289in}}%
\pgfpathlineto{\pgfqpoint{4.253279in}{1.371553in}}%
\pgfpathlineto{\pgfqpoint{4.261256in}{0.756482in}}%
\pgfpathlineto{\pgfqpoint{4.269233in}{1.592102in}}%
\pgfpathlineto{\pgfqpoint{4.277210in}{0.942993in}}%
\pgfpathlineto{\pgfqpoint{4.285188in}{1.037494in}}%
\pgfpathlineto{\pgfqpoint{4.293165in}{1.533200in}}%
\pgfpathlineto{\pgfqpoint{4.301142in}{0.996279in}}%
\pgfpathlineto{\pgfqpoint{4.309119in}{1.144638in}}%
\pgfpathlineto{\pgfqpoint{4.317096in}{1.186065in}}%
\pgfpathlineto{\pgfqpoint{4.325073in}{1.268460in}}%
\pgfpathlineto{\pgfqpoint{4.333051in}{1.023064in}}%
\pgfpathlineto{\pgfqpoint{4.341028in}{1.017558in}}%
\pgfpathlineto{\pgfqpoint{4.349005in}{1.450640in}}%
\pgfpathlineto{\pgfqpoint{4.356982in}{0.842534in}}%
\pgfpathlineto{\pgfqpoint{4.364959in}{1.604916in}}%
\pgfpathlineto{\pgfqpoint{4.380914in}{1.147805in}}%
\pgfpathlineto{\pgfqpoint{4.388891in}{0.689601in}}%
\pgfpathlineto{\pgfqpoint{4.396868in}{1.238463in}}%
\pgfpathlineto{\pgfqpoint{4.404845in}{1.261527in}}%
\pgfpathlineto{\pgfqpoint{4.412822in}{1.195747in}}%
\pgfpathlineto{\pgfqpoint{4.420799in}{1.476742in}}%
\pgfpathlineto{\pgfqpoint{4.428777in}{1.080782in}}%
\pgfpathlineto{\pgfqpoint{4.436754in}{1.270756in}}%
\pgfpathlineto{\pgfqpoint{4.444731in}{1.043024in}}%
\pgfpathlineto{\pgfqpoint{4.452708in}{1.369611in}}%
\pgfpathlineto{\pgfqpoint{4.460685in}{1.185651in}}%
\pgfpathlineto{\pgfqpoint{4.468662in}{1.216523in}}%
\pgfpathlineto{\pgfqpoint{4.476640in}{1.108161in}}%
\pgfpathlineto{\pgfqpoint{4.484617in}{1.183161in}}%
\pgfpathlineto{\pgfqpoint{4.492594in}{1.314088in}}%
\pgfpathlineto{\pgfqpoint{4.500571in}{1.111919in}}%
\pgfpathlineto{\pgfqpoint{4.508548in}{0.682409in}}%
\pgfpathlineto{\pgfqpoint{4.516525in}{1.184953in}}%
\pgfpathlineto{\pgfqpoint{4.524503in}{1.123598in}}%
\pgfpathlineto{\pgfqpoint{4.532480in}{1.766558in}}%
\pgfpathlineto{\pgfqpoint{4.540457in}{1.018643in}}%
\pgfpathlineto{\pgfqpoint{4.548434in}{1.199949in}}%
\pgfpathlineto{\pgfqpoint{4.556411in}{1.141035in}}%
\pgfpathlineto{\pgfqpoint{4.564388in}{1.356920in}}%
\pgfpathlineto{\pgfqpoint{4.572366in}{0.659370in}}%
\pgfpathlineto{\pgfqpoint{4.580343in}{1.664108in}}%
\pgfpathlineto{\pgfqpoint{4.588320in}{0.926860in}}%
\pgfpathlineto{\pgfqpoint{4.596297in}{1.250749in}}%
\pgfpathlineto{\pgfqpoint{4.604274in}{0.971690in}}%
\pgfpathlineto{\pgfqpoint{4.612251in}{1.594308in}}%
\pgfpathlineto{\pgfqpoint{4.628206in}{0.966552in}}%
\pgfpathlineto{\pgfqpoint{4.636183in}{0.937142in}}%
\pgfpathlineto{\pgfqpoint{4.644160in}{1.274070in}}%
\pgfpathlineto{\pgfqpoint{4.652137in}{1.499070in}}%
\pgfpathlineto{\pgfqpoint{4.660114in}{0.837853in}}%
\pgfpathlineto{\pgfqpoint{4.668091in}{1.068140in}}%
\pgfpathlineto{\pgfqpoint{4.676069in}{1.351266in}}%
\pgfpathlineto{\pgfqpoint{4.684046in}{1.530195in}}%
\pgfpathlineto{\pgfqpoint{4.692023in}{0.681979in}}%
\pgfpathlineto{\pgfqpoint{4.700000in}{1.353762in}}%
\pgfpathlineto{\pgfqpoint{4.707977in}{1.284434in}}%
\pgfpathlineto{\pgfqpoint{4.715954in}{1.345706in}}%
\pgfpathlineto{\pgfqpoint{4.723932in}{0.801453in}}%
\pgfpathlineto{\pgfqpoint{4.731909in}{1.487311in}}%
\pgfpathlineto{\pgfqpoint{4.739886in}{1.214294in}}%
\pgfpathlineto{\pgfqpoint{4.747863in}{0.794365in}}%
\pgfpathlineto{\pgfqpoint{4.755840in}{1.291388in}}%
\pgfpathlineto{\pgfqpoint{4.763817in}{1.409258in}}%
\pgfpathlineto{\pgfqpoint{4.771795in}{1.270604in}}%
\pgfpathlineto{\pgfqpoint{4.779772in}{0.895222in}}%
\pgfpathlineto{\pgfqpoint{4.787749in}{1.091769in}}%
\pgfpathlineto{\pgfqpoint{4.795726in}{1.089184in}}%
\pgfpathlineto{\pgfqpoint{4.803703in}{1.173139in}}%
\pgfpathlineto{\pgfqpoint{4.811680in}{1.628419in}}%
\pgfpathlineto{\pgfqpoint{4.819658in}{0.879309in}}%
\pgfpathlineto{\pgfqpoint{4.827635in}{1.602917in}}%
\pgfpathlineto{\pgfqpoint{4.835612in}{0.783510in}}%
\pgfpathlineto{\pgfqpoint{4.843589in}{1.102013in}}%
\pgfpathlineto{\pgfqpoint{4.851566in}{1.646224in}}%
\pgfpathlineto{\pgfqpoint{4.859543in}{0.693821in}}%
\pgfpathlineto{\pgfqpoint{4.867521in}{1.673702in}}%
\pgfpathlineto{\pgfqpoint{4.875498in}{1.255083in}}%
\pgfpathlineto{\pgfqpoint{4.883475in}{0.560336in}}%
\pgfpathlineto{\pgfqpoint{4.891452in}{1.628279in}}%
\pgfpathlineto{\pgfqpoint{4.899429in}{0.852323in}}%
\pgfpathlineto{\pgfqpoint{4.907406in}{1.418103in}}%
\pgfpathlineto{\pgfqpoint{4.915384in}{0.939061in}}%
\pgfpathlineto{\pgfqpoint{4.923361in}{1.342307in}}%
\pgfpathlineto{\pgfqpoint{4.931338in}{1.069546in}}%
\pgfpathlineto{\pgfqpoint{4.939315in}{1.495568in}}%
\pgfpathlineto{\pgfqpoint{4.939315in}{1.495568in}}%
\pgfusepath{stroke}%
\end{pgfscope}%
\begin{pgfscope}%
\pgfpathrectangle{\pgfqpoint{2.811806in}{0.386111in}}{\pgfqpoint{2.228819in}{2.415278in}}%
\pgfusepath{clip}%
\pgfsetrectcap%
\pgfsetroundjoin%
\pgfsetlinewidth{1.505625pt}%
\definecolor{currentstroke}{rgb}{0.172549,0.627451,0.172549}%
\pgfsetstrokecolor{currentstroke}%
\pgfsetdash{}{0pt}%
\pgfpathmoveto{\pgfqpoint{2.913116in}{1.173592in}}%
\pgfpathlineto{\pgfqpoint{2.921093in}{1.104697in}}%
\pgfpathlineto{\pgfqpoint{2.929070in}{1.353290in}}%
\pgfpathlineto{\pgfqpoint{2.937047in}{0.918743in}}%
\pgfpathlineto{\pgfqpoint{2.945024in}{1.077404in}}%
\pgfpathlineto{\pgfqpoint{2.953001in}{1.382895in}}%
\pgfpathlineto{\pgfqpoint{2.960979in}{1.117949in}}%
\pgfpathlineto{\pgfqpoint{2.968956in}{0.591508in}}%
\pgfpathlineto{\pgfqpoint{2.976933in}{1.235667in}}%
\pgfpathlineto{\pgfqpoint{2.984910in}{0.974258in}}%
\pgfpathlineto{\pgfqpoint{2.992887in}{1.637366in}}%
\pgfpathlineto{\pgfqpoint{3.000864in}{0.665221in}}%
\pgfpathlineto{\pgfqpoint{3.008841in}{1.466145in}}%
\pgfpathlineto{\pgfqpoint{3.016819in}{0.594425in}}%
\pgfpathlineto{\pgfqpoint{3.024796in}{1.378754in}}%
\pgfpathlineto{\pgfqpoint{3.040750in}{0.827971in}}%
\pgfpathlineto{\pgfqpoint{3.048727in}{1.197229in}}%
\pgfpathlineto{\pgfqpoint{3.056704in}{1.303545in}}%
\pgfpathlineto{\pgfqpoint{3.064682in}{0.919367in}}%
\pgfpathlineto{\pgfqpoint{3.072659in}{1.036274in}}%
\pgfpathlineto{\pgfqpoint{3.080636in}{1.512437in}}%
\pgfpathlineto{\pgfqpoint{3.088613in}{1.328406in}}%
\pgfpathlineto{\pgfqpoint{3.096590in}{0.582894in}}%
\pgfpathlineto{\pgfqpoint{3.104567in}{1.181138in}}%
\pgfpathlineto{\pgfqpoint{3.112545in}{1.123645in}}%
\pgfpathlineto{\pgfqpoint{3.120522in}{1.245489in}}%
\pgfpathlineto{\pgfqpoint{3.128499in}{1.012051in}}%
\pgfpathlineto{\pgfqpoint{3.136476in}{1.409866in}}%
\pgfpathlineto{\pgfqpoint{3.144453in}{0.876534in}}%
\pgfpathlineto{\pgfqpoint{3.152430in}{1.099568in}}%
\pgfpathlineto{\pgfqpoint{3.160408in}{1.480705in}}%
\pgfpathlineto{\pgfqpoint{3.168385in}{0.807855in}}%
\pgfpathlineto{\pgfqpoint{3.176362in}{1.610137in}}%
\pgfpathlineto{\pgfqpoint{3.184339in}{0.768677in}}%
\pgfpathlineto{\pgfqpoint{3.192316in}{1.514503in}}%
\pgfpathlineto{\pgfqpoint{3.200293in}{0.954946in}}%
\pgfpathlineto{\pgfqpoint{3.208271in}{1.095057in}}%
\pgfpathlineto{\pgfqpoint{3.216248in}{1.066946in}}%
\pgfpathlineto{\pgfqpoint{3.224225in}{1.386343in}}%
\pgfpathlineto{\pgfqpoint{3.232202in}{1.173700in}}%
\pgfpathlineto{\pgfqpoint{3.240179in}{1.128601in}}%
\pgfpathlineto{\pgfqpoint{3.248156in}{0.727039in}}%
\pgfpathlineto{\pgfqpoint{3.256134in}{1.457841in}}%
\pgfpathlineto{\pgfqpoint{3.264111in}{1.042186in}}%
\pgfpathlineto{\pgfqpoint{3.272088in}{0.903153in}}%
\pgfpathlineto{\pgfqpoint{3.280065in}{1.676785in}}%
\pgfpathlineto{\pgfqpoint{3.288042in}{0.882629in}}%
\pgfpathlineto{\pgfqpoint{3.296019in}{1.465988in}}%
\pgfpathlineto{\pgfqpoint{3.303997in}{0.907116in}}%
\pgfpathlineto{\pgfqpoint{3.311974in}{1.245389in}}%
\pgfpathlineto{\pgfqpoint{3.319951in}{0.829021in}}%
\pgfpathlineto{\pgfqpoint{3.327928in}{1.395586in}}%
\pgfpathlineto{\pgfqpoint{3.335905in}{0.850432in}}%
\pgfpathlineto{\pgfqpoint{3.343882in}{1.175198in}}%
\pgfpathlineto{\pgfqpoint{3.351860in}{1.167548in}}%
\pgfpathlineto{\pgfqpoint{3.359837in}{1.512967in}}%
\pgfpathlineto{\pgfqpoint{3.367814in}{0.814176in}}%
\pgfpathlineto{\pgfqpoint{3.375791in}{1.470715in}}%
\pgfpathlineto{\pgfqpoint{3.383768in}{1.299253in}}%
\pgfpathlineto{\pgfqpoint{3.391745in}{0.900106in}}%
\pgfpathlineto{\pgfqpoint{3.399722in}{1.416491in}}%
\pgfpathlineto{\pgfqpoint{3.407700in}{1.182324in}}%
\pgfpathlineto{\pgfqpoint{3.415677in}{0.878907in}}%
\pgfpathlineto{\pgfqpoint{3.423654in}{1.690304in}}%
\pgfpathlineto{\pgfqpoint{3.431631in}{0.766603in}}%
\pgfpathlineto{\pgfqpoint{3.439608in}{1.507385in}}%
\pgfpathlineto{\pgfqpoint{3.447585in}{1.115306in}}%
\pgfpathlineto{\pgfqpoint{3.455563in}{1.075869in}}%
\pgfpathlineto{\pgfqpoint{3.463540in}{1.253172in}}%
\pgfpathlineto{\pgfqpoint{3.471517in}{1.027944in}}%
\pgfpathlineto{\pgfqpoint{3.479494in}{1.473384in}}%
\pgfpathlineto{\pgfqpoint{3.487471in}{0.951453in}}%
\pgfpathlineto{\pgfqpoint{3.495448in}{1.013869in}}%
\pgfpathlineto{\pgfqpoint{3.503426in}{1.724505in}}%
\pgfpathlineto{\pgfqpoint{3.511403in}{1.027821in}}%
\pgfpathlineto{\pgfqpoint{3.519380in}{1.430791in}}%
\pgfpathlineto{\pgfqpoint{3.527357in}{1.106081in}}%
\pgfpathlineto{\pgfqpoint{3.535334in}{0.855228in}}%
\pgfpathlineto{\pgfqpoint{3.543311in}{1.542346in}}%
\pgfpathlineto{\pgfqpoint{3.551289in}{0.713751in}}%
\pgfpathlineto{\pgfqpoint{3.559266in}{1.402418in}}%
\pgfpathlineto{\pgfqpoint{3.567243in}{1.358635in}}%
\pgfpathlineto{\pgfqpoint{3.575220in}{1.489684in}}%
\pgfpathlineto{\pgfqpoint{3.583197in}{0.886921in}}%
\pgfpathlineto{\pgfqpoint{3.591174in}{1.505466in}}%
\pgfpathlineto{\pgfqpoint{3.599152in}{0.691359in}}%
\pgfpathlineto{\pgfqpoint{3.607129in}{1.423428in}}%
\pgfpathlineto{\pgfqpoint{3.615106in}{1.594873in}}%
\pgfpathlineto{\pgfqpoint{3.623083in}{0.715132in}}%
\pgfpathlineto{\pgfqpoint{3.631060in}{1.332876in}}%
\pgfpathlineto{\pgfqpoint{3.639037in}{1.426435in}}%
\pgfpathlineto{\pgfqpoint{3.647015in}{0.998939in}}%
\pgfpathlineto{\pgfqpoint{3.654992in}{1.385616in}}%
\pgfpathlineto{\pgfqpoint{3.662969in}{1.095083in}}%
\pgfpathlineto{\pgfqpoint{3.670946in}{1.084875in}}%
\pgfpathlineto{\pgfqpoint{3.678923in}{1.650844in}}%
\pgfpathlineto{\pgfqpoint{3.686900in}{0.922576in}}%
\pgfpathlineto{\pgfqpoint{3.694878in}{1.567446in}}%
\pgfpathlineto{\pgfqpoint{3.702855in}{0.964232in}}%
\pgfpathlineto{\pgfqpoint{3.710832in}{1.581995in}}%
\pgfpathlineto{\pgfqpoint{3.718809in}{0.770525in}}%
\pgfpathlineto{\pgfqpoint{3.726786in}{1.318010in}}%
\pgfpathlineto{\pgfqpoint{3.734763in}{1.618889in}}%
\pgfpathlineto{\pgfqpoint{3.750718in}{1.071334in}}%
\pgfpathlineto{\pgfqpoint{3.758695in}{0.937513in}}%
\pgfpathlineto{\pgfqpoint{3.766672in}{1.457636in}}%
\pgfpathlineto{\pgfqpoint{3.774649in}{1.499646in}}%
\pgfpathlineto{\pgfqpoint{3.782626in}{0.719757in}}%
\pgfpathlineto{\pgfqpoint{3.790604in}{1.582355in}}%
\pgfpathlineto{\pgfqpoint{3.798581in}{1.301998in}}%
\pgfpathlineto{\pgfqpoint{3.806558in}{1.182966in}}%
\pgfpathlineto{\pgfqpoint{3.814535in}{0.943452in}}%
\pgfpathlineto{\pgfqpoint{3.822512in}{1.529543in}}%
\pgfpathlineto{\pgfqpoint{3.830489in}{1.153495in}}%
\pgfpathlineto{\pgfqpoint{3.838466in}{1.262887in}}%
\pgfpathlineto{\pgfqpoint{3.846444in}{1.224048in}}%
\pgfpathlineto{\pgfqpoint{3.854421in}{1.531339in}}%
\pgfpathlineto{\pgfqpoint{3.862398in}{1.202254in}}%
\pgfpathlineto{\pgfqpoint{3.870375in}{1.090932in}}%
\pgfpathlineto{\pgfqpoint{3.878352in}{1.681890in}}%
\pgfpathlineto{\pgfqpoint{3.886329in}{0.924129in}}%
\pgfpathlineto{\pgfqpoint{3.894307in}{1.286674in}}%
\pgfpathlineto{\pgfqpoint{3.902284in}{1.170361in}}%
\pgfpathlineto{\pgfqpoint{3.910261in}{1.396425in}}%
\pgfpathlineto{\pgfqpoint{3.918238in}{1.171149in}}%
\pgfpathlineto{\pgfqpoint{3.926215in}{1.772057in}}%
\pgfpathlineto{\pgfqpoint{3.934192in}{1.112464in}}%
\pgfpathlineto{\pgfqpoint{3.942170in}{1.389536in}}%
\pgfpathlineto{\pgfqpoint{3.950147in}{0.831117in}}%
\pgfpathlineto{\pgfqpoint{3.958124in}{1.372237in}}%
\pgfpathlineto{\pgfqpoint{3.966101in}{1.701960in}}%
\pgfpathlineto{\pgfqpoint{3.974078in}{0.770674in}}%
\pgfpathlineto{\pgfqpoint{3.982055in}{1.743744in}}%
\pgfpathlineto{\pgfqpoint{3.990033in}{1.230725in}}%
\pgfpathlineto{\pgfqpoint{3.998010in}{1.248392in}}%
\pgfpathlineto{\pgfqpoint{4.005987in}{1.302500in}}%
\pgfpathlineto{\pgfqpoint{4.013964in}{1.144282in}}%
\pgfpathlineto{\pgfqpoint{4.021941in}{1.563445in}}%
\pgfpathlineto{\pgfqpoint{4.029918in}{1.468619in}}%
\pgfpathlineto{\pgfqpoint{4.037896in}{1.314547in}}%
\pgfpathlineto{\pgfqpoint{4.045873in}{1.269142in}}%
\pgfpathlineto{\pgfqpoint{4.053850in}{0.839888in}}%
\pgfpathlineto{\pgfqpoint{4.061827in}{1.557780in}}%
\pgfpathlineto{\pgfqpoint{4.069804in}{1.378851in}}%
\pgfpathlineto{\pgfqpoint{4.077781in}{0.940957in}}%
\pgfpathlineto{\pgfqpoint{4.085759in}{1.674743in}}%
\pgfpathlineto{\pgfqpoint{4.093736in}{0.919685in}}%
\pgfpathlineto{\pgfqpoint{4.101713in}{1.526067in}}%
\pgfpathlineto{\pgfqpoint{4.109690in}{1.223854in}}%
\pgfpathlineto{\pgfqpoint{4.117667in}{1.773399in}}%
\pgfpathlineto{\pgfqpoint{4.125644in}{1.043933in}}%
\pgfpathlineto{\pgfqpoint{4.133622in}{1.417567in}}%
\pgfpathlineto{\pgfqpoint{4.141599in}{1.362637in}}%
\pgfpathlineto{\pgfqpoint{4.149576in}{1.133973in}}%
\pgfpathlineto{\pgfqpoint{4.157553in}{1.456905in}}%
\pgfpathlineto{\pgfqpoint{4.165530in}{0.953338in}}%
\pgfpathlineto{\pgfqpoint{4.173507in}{1.894167in}}%
\pgfpathlineto{\pgfqpoint{4.181485in}{1.371668in}}%
\pgfpathlineto{\pgfqpoint{4.189462in}{0.738168in}}%
\pgfpathlineto{\pgfqpoint{4.197439in}{1.565748in}}%
\pgfpathlineto{\pgfqpoint{4.205416in}{1.491123in}}%
\pgfpathlineto{\pgfqpoint{4.213393in}{1.224251in}}%
\pgfpathlineto{\pgfqpoint{4.221370in}{1.251005in}}%
\pgfpathlineto{\pgfqpoint{4.229347in}{1.711187in}}%
\pgfpathlineto{\pgfqpoint{4.237325in}{0.830291in}}%
\pgfpathlineto{\pgfqpoint{4.245302in}{1.412710in}}%
\pgfpathlineto{\pgfqpoint{4.253279in}{1.540907in}}%
\pgfpathlineto{\pgfqpoint{4.261256in}{1.268235in}}%
\pgfpathlineto{\pgfqpoint{4.269233in}{1.190795in}}%
\pgfpathlineto{\pgfqpoint{4.277210in}{1.332720in}}%
\pgfpathlineto{\pgfqpoint{4.285188in}{1.419555in}}%
\pgfpathlineto{\pgfqpoint{4.293165in}{1.325549in}}%
\pgfpathlineto{\pgfqpoint{4.301142in}{1.541050in}}%
\pgfpathlineto{\pgfqpoint{4.309119in}{1.173810in}}%
\pgfpathlineto{\pgfqpoint{4.317096in}{1.318399in}}%
\pgfpathlineto{\pgfqpoint{4.325073in}{1.727136in}}%
\pgfpathlineto{\pgfqpoint{4.333051in}{1.142108in}}%
\pgfpathlineto{\pgfqpoint{4.341028in}{1.684851in}}%
\pgfpathlineto{\pgfqpoint{4.349005in}{1.298332in}}%
\pgfpathlineto{\pgfqpoint{4.356982in}{1.127060in}}%
\pgfpathlineto{\pgfqpoint{4.364959in}{1.252525in}}%
\pgfpathlineto{\pgfqpoint{4.372936in}{1.748838in}}%
\pgfpathlineto{\pgfqpoint{4.380914in}{1.117919in}}%
\pgfpathlineto{\pgfqpoint{4.388891in}{1.243417in}}%
\pgfpathlineto{\pgfqpoint{4.396868in}{1.613792in}}%
\pgfpathlineto{\pgfqpoint{4.404845in}{1.603630in}}%
\pgfpathlineto{\pgfqpoint{4.412822in}{1.195755in}}%
\pgfpathlineto{\pgfqpoint{4.420799in}{1.128082in}}%
\pgfpathlineto{\pgfqpoint{4.428777in}{1.461523in}}%
\pgfpathlineto{\pgfqpoint{4.436754in}{1.270622in}}%
\pgfpathlineto{\pgfqpoint{4.444731in}{1.638547in}}%
\pgfpathlineto{\pgfqpoint{4.452708in}{1.550698in}}%
\pgfpathlineto{\pgfqpoint{4.460685in}{1.407185in}}%
\pgfpathlineto{\pgfqpoint{4.468662in}{1.508016in}}%
\pgfpathlineto{\pgfqpoint{4.476640in}{1.079040in}}%
\pgfpathlineto{\pgfqpoint{4.484617in}{1.744387in}}%
\pgfpathlineto{\pgfqpoint{4.492594in}{1.079047in}}%
\pgfpathlineto{\pgfqpoint{4.500571in}{1.709454in}}%
\pgfpathlineto{\pgfqpoint{4.508548in}{1.115980in}}%
\pgfpathlineto{\pgfqpoint{4.524503in}{1.675284in}}%
\pgfpathlineto{\pgfqpoint{4.532480in}{0.993121in}}%
\pgfpathlineto{\pgfqpoint{4.540457in}{1.798885in}}%
\pgfpathlineto{\pgfqpoint{4.548434in}{0.835250in}}%
\pgfpathlineto{\pgfqpoint{4.556411in}{1.893694in}}%
\pgfpathlineto{\pgfqpoint{4.564388in}{1.348182in}}%
\pgfpathlineto{\pgfqpoint{4.572366in}{0.965084in}}%
\pgfpathlineto{\pgfqpoint{4.580343in}{1.558329in}}%
\pgfpathlineto{\pgfqpoint{4.588320in}{1.594591in}}%
\pgfpathlineto{\pgfqpoint{4.596297in}{1.384802in}}%
\pgfpathlineto{\pgfqpoint{4.604274in}{1.540866in}}%
\pgfpathlineto{\pgfqpoint{4.612251in}{1.186903in}}%
\pgfpathlineto{\pgfqpoint{4.620228in}{1.422453in}}%
\pgfpathlineto{\pgfqpoint{4.628206in}{1.716395in}}%
\pgfpathlineto{\pgfqpoint{4.636183in}{1.095404in}}%
\pgfpathlineto{\pgfqpoint{4.644160in}{1.723580in}}%
\pgfpathlineto{\pgfqpoint{4.652137in}{1.332685in}}%
\pgfpathlineto{\pgfqpoint{4.660114in}{1.470046in}}%
\pgfpathlineto{\pgfqpoint{4.668091in}{1.259029in}}%
\pgfpathlineto{\pgfqpoint{4.676069in}{1.271557in}}%
\pgfpathlineto{\pgfqpoint{4.684046in}{1.564573in}}%
\pgfpathlineto{\pgfqpoint{4.692023in}{1.347447in}}%
\pgfpathlineto{\pgfqpoint{4.700000in}{1.343447in}}%
\pgfpathlineto{\pgfqpoint{4.707977in}{1.991234in}}%
\pgfpathlineto{\pgfqpoint{4.723932in}{0.858065in}}%
\pgfpathlineto{\pgfqpoint{4.731909in}{1.964195in}}%
\pgfpathlineto{\pgfqpoint{4.739886in}{1.518172in}}%
\pgfpathlineto{\pgfqpoint{4.747863in}{1.262169in}}%
\pgfpathlineto{\pgfqpoint{4.755840in}{1.139634in}}%
\pgfpathlineto{\pgfqpoint{4.763817in}{1.617984in}}%
\pgfpathlineto{\pgfqpoint{4.771795in}{1.292159in}}%
\pgfpathlineto{\pgfqpoint{4.779772in}{1.695670in}}%
\pgfpathlineto{\pgfqpoint{4.787749in}{1.611753in}}%
\pgfpathlineto{\pgfqpoint{4.795726in}{0.926476in}}%
\pgfpathlineto{\pgfqpoint{4.803703in}{1.957914in}}%
\pgfpathlineto{\pgfqpoint{4.811680in}{1.270856in}}%
\pgfpathlineto{\pgfqpoint{4.819658in}{1.309482in}}%
\pgfpathlineto{\pgfqpoint{4.835612in}{1.700269in}}%
\pgfpathlineto{\pgfqpoint{4.843589in}{1.320527in}}%
\pgfpathlineto{\pgfqpoint{4.851566in}{1.756592in}}%
\pgfpathlineto{\pgfqpoint{4.859543in}{1.467000in}}%
\pgfpathlineto{\pgfqpoint{4.867521in}{0.953525in}}%
\pgfpathlineto{\pgfqpoint{4.875498in}{1.974923in}}%
\pgfpathlineto{\pgfqpoint{4.883475in}{1.253933in}}%
\pgfpathlineto{\pgfqpoint{4.891452in}{1.691176in}}%
\pgfpathlineto{\pgfqpoint{4.899429in}{1.134543in}}%
\pgfpathlineto{\pgfqpoint{4.907406in}{1.666286in}}%
\pgfpathlineto{\pgfqpoint{4.915384in}{1.322064in}}%
\pgfpathlineto{\pgfqpoint{4.923361in}{1.326099in}}%
\pgfpathlineto{\pgfqpoint{4.931338in}{1.915780in}}%
\pgfpathlineto{\pgfqpoint{4.939315in}{1.033862in}}%
\pgfpathlineto{\pgfqpoint{4.939315in}{1.033862in}}%
\pgfusepath{stroke}%
\end{pgfscope}%
\begin{pgfscope}%
\pgfpathrectangle{\pgfqpoint{2.811806in}{0.386111in}}{\pgfqpoint{2.228819in}{2.415278in}}%
\pgfusepath{clip}%
\pgfsetrectcap%
\pgfsetroundjoin%
\pgfsetlinewidth{1.505625pt}%
\definecolor{currentstroke}{rgb}{0.839216,0.152941,0.156863}%
\pgfsetstrokecolor{currentstroke}%
\pgfsetdash{}{0pt}%
\pgfpathmoveto{\pgfqpoint{2.913116in}{1.375381in}}%
\pgfpathlineto{\pgfqpoint{2.921093in}{1.108797in}}%
\pgfpathlineto{\pgfqpoint{2.929070in}{1.069610in}}%
\pgfpathlineto{\pgfqpoint{2.937047in}{1.087072in}}%
\pgfpathlineto{\pgfqpoint{2.945024in}{0.724480in}}%
\pgfpathlineto{\pgfqpoint{2.953001in}{0.942983in}}%
\pgfpathlineto{\pgfqpoint{2.960979in}{1.348780in}}%
\pgfpathlineto{\pgfqpoint{2.968956in}{1.182308in}}%
\pgfpathlineto{\pgfqpoint{2.976933in}{1.135483in}}%
\pgfpathlineto{\pgfqpoint{2.984910in}{1.120497in}}%
\pgfpathlineto{\pgfqpoint{2.992887in}{0.637218in}}%
\pgfpathlineto{\pgfqpoint{3.000864in}{1.299944in}}%
\pgfpathlineto{\pgfqpoint{3.008841in}{1.354074in}}%
\pgfpathlineto{\pgfqpoint{3.016819in}{0.944370in}}%
\pgfpathlineto{\pgfqpoint{3.024796in}{0.700012in}}%
\pgfpathlineto{\pgfqpoint{3.032773in}{1.497203in}}%
\pgfpathlineto{\pgfqpoint{3.048727in}{0.840957in}}%
\pgfpathlineto{\pgfqpoint{3.056704in}{1.315688in}}%
\pgfpathlineto{\pgfqpoint{3.064682in}{0.734688in}}%
\pgfpathlineto{\pgfqpoint{3.072659in}{1.148887in}}%
\pgfpathlineto{\pgfqpoint{3.080636in}{0.908733in}}%
\pgfpathlineto{\pgfqpoint{3.088613in}{1.522118in}}%
\pgfpathlineto{\pgfqpoint{3.096590in}{0.699442in}}%
\pgfpathlineto{\pgfqpoint{3.104567in}{1.501258in}}%
\pgfpathlineto{\pgfqpoint{3.120522in}{0.805713in}}%
\pgfpathlineto{\pgfqpoint{3.128499in}{1.508219in}}%
\pgfpathlineto{\pgfqpoint{3.136476in}{0.843976in}}%
\pgfpathlineto{\pgfqpoint{3.144453in}{1.185406in}}%
\pgfpathlineto{\pgfqpoint{3.152430in}{1.232382in}}%
\pgfpathlineto{\pgfqpoint{3.160408in}{0.744552in}}%
\pgfpathlineto{\pgfqpoint{3.168385in}{1.426966in}}%
\pgfpathlineto{\pgfqpoint{3.176362in}{0.610627in}}%
\pgfpathlineto{\pgfqpoint{3.184339in}{1.631263in}}%
\pgfpathlineto{\pgfqpoint{3.192316in}{0.807870in}}%
\pgfpathlineto{\pgfqpoint{3.200293in}{1.240690in}}%
\pgfpathlineto{\pgfqpoint{3.208271in}{1.081607in}}%
\pgfpathlineto{\pgfqpoint{3.216248in}{1.296543in}}%
\pgfpathlineto{\pgfqpoint{3.224225in}{0.593387in}}%
\pgfpathlineto{\pgfqpoint{3.232202in}{1.174972in}}%
\pgfpathlineto{\pgfqpoint{3.240179in}{1.059765in}}%
\pgfpathlineto{\pgfqpoint{3.248156in}{1.239218in}}%
\pgfpathlineto{\pgfqpoint{3.256134in}{1.498046in}}%
\pgfpathlineto{\pgfqpoint{3.264111in}{0.680066in}}%
\pgfpathlineto{\pgfqpoint{3.272088in}{1.266972in}}%
\pgfpathlineto{\pgfqpoint{3.280065in}{1.210505in}}%
\pgfpathlineto{\pgfqpoint{3.288042in}{1.191637in}}%
\pgfpathlineto{\pgfqpoint{3.296019in}{0.739608in}}%
\pgfpathlineto{\pgfqpoint{3.303997in}{1.671767in}}%
\pgfpathlineto{\pgfqpoint{3.311974in}{1.046519in}}%
\pgfpathlineto{\pgfqpoint{3.319951in}{0.701745in}}%
\pgfpathlineto{\pgfqpoint{3.327928in}{1.584117in}}%
\pgfpathlineto{\pgfqpoint{3.335905in}{0.814832in}}%
\pgfpathlineto{\pgfqpoint{3.343882in}{1.074537in}}%
\pgfpathlineto{\pgfqpoint{3.351860in}{1.062505in}}%
\pgfpathlineto{\pgfqpoint{3.359837in}{1.594375in}}%
\pgfpathlineto{\pgfqpoint{3.367814in}{1.220883in}}%
\pgfpathlineto{\pgfqpoint{3.375791in}{0.616437in}}%
\pgfpathlineto{\pgfqpoint{3.383768in}{1.750447in}}%
\pgfpathlineto{\pgfqpoint{3.391745in}{0.633313in}}%
\pgfpathlineto{\pgfqpoint{3.399722in}{1.144274in}}%
\pgfpathlineto{\pgfqpoint{3.407700in}{1.070901in}}%
\pgfpathlineto{\pgfqpoint{3.415677in}{1.737393in}}%
\pgfpathlineto{\pgfqpoint{3.423654in}{0.593555in}}%
\pgfpathlineto{\pgfqpoint{3.431631in}{1.322544in}}%
\pgfpathlineto{\pgfqpoint{3.439608in}{1.096294in}}%
\pgfpathlineto{\pgfqpoint{3.447585in}{1.587452in}}%
\pgfpathlineto{\pgfqpoint{3.455563in}{0.904149in}}%
\pgfpathlineto{\pgfqpoint{3.463540in}{1.142712in}}%
\pgfpathlineto{\pgfqpoint{3.471517in}{1.452920in}}%
\pgfpathlineto{\pgfqpoint{3.479494in}{0.699055in}}%
\pgfpathlineto{\pgfqpoint{3.495448in}{1.674430in}}%
\pgfpathlineto{\pgfqpoint{3.503426in}{0.954487in}}%
\pgfpathlineto{\pgfqpoint{3.511403in}{0.985031in}}%
\pgfpathlineto{\pgfqpoint{3.519380in}{1.264818in}}%
\pgfpathlineto{\pgfqpoint{3.527357in}{1.027005in}}%
\pgfpathlineto{\pgfqpoint{3.535334in}{1.247988in}}%
\pgfpathlineto{\pgfqpoint{3.543311in}{1.354669in}}%
\pgfpathlineto{\pgfqpoint{3.551289in}{1.264191in}}%
\pgfpathlineto{\pgfqpoint{3.559266in}{1.482294in}}%
\pgfpathlineto{\pgfqpoint{3.567243in}{1.177513in}}%
\pgfpathlineto{\pgfqpoint{3.575220in}{0.775742in}}%
\pgfpathlineto{\pgfqpoint{3.583197in}{1.553530in}}%
\pgfpathlineto{\pgfqpoint{3.591174in}{0.926906in}}%
\pgfpathlineto{\pgfqpoint{3.599152in}{1.233615in}}%
\pgfpathlineto{\pgfqpoint{3.607129in}{1.169292in}}%
\pgfpathlineto{\pgfqpoint{3.615106in}{1.270330in}}%
\pgfpathlineto{\pgfqpoint{3.623083in}{1.610035in}}%
\pgfpathlineto{\pgfqpoint{3.631060in}{0.934966in}}%
\pgfpathlineto{\pgfqpoint{3.639037in}{1.066719in}}%
\pgfpathlineto{\pgfqpoint{3.647015in}{1.639361in}}%
\pgfpathlineto{\pgfqpoint{3.654992in}{0.963950in}}%
\pgfpathlineto{\pgfqpoint{3.662969in}{1.256883in}}%
\pgfpathlineto{\pgfqpoint{3.670946in}{1.644901in}}%
\pgfpathlineto{\pgfqpoint{3.678923in}{0.707378in}}%
\pgfpathlineto{\pgfqpoint{3.686900in}{1.407533in}}%
\pgfpathlineto{\pgfqpoint{3.694878in}{1.570195in}}%
\pgfpathlineto{\pgfqpoint{3.702855in}{0.906387in}}%
\pgfpathlineto{\pgfqpoint{3.710832in}{1.550744in}}%
\pgfpathlineto{\pgfqpoint{3.718809in}{1.155891in}}%
\pgfpathlineto{\pgfqpoint{3.726786in}{1.265497in}}%
\pgfpathlineto{\pgfqpoint{3.734763in}{1.405642in}}%
\pgfpathlineto{\pgfqpoint{3.742741in}{1.479429in}}%
\pgfpathlineto{\pgfqpoint{3.750718in}{1.188468in}}%
\pgfpathlineto{\pgfqpoint{3.758695in}{1.444696in}}%
\pgfpathlineto{\pgfqpoint{3.766672in}{0.938711in}}%
\pgfpathlineto{\pgfqpoint{3.774649in}{1.473451in}}%
\pgfpathlineto{\pgfqpoint{3.782626in}{1.202006in}}%
\pgfpathlineto{\pgfqpoint{3.790604in}{1.532649in}}%
\pgfpathlineto{\pgfqpoint{3.798581in}{1.387455in}}%
\pgfpathlineto{\pgfqpoint{3.806558in}{1.091507in}}%
\pgfpathlineto{\pgfqpoint{3.814535in}{1.234862in}}%
\pgfpathlineto{\pgfqpoint{3.822512in}{1.024281in}}%
\pgfpathlineto{\pgfqpoint{3.830489in}{1.709958in}}%
\pgfpathlineto{\pgfqpoint{3.838466in}{1.261759in}}%
\pgfpathlineto{\pgfqpoint{3.846444in}{1.432469in}}%
\pgfpathlineto{\pgfqpoint{3.854421in}{1.556144in}}%
\pgfpathlineto{\pgfqpoint{3.870375in}{1.087643in}}%
\pgfpathlineto{\pgfqpoint{3.878352in}{1.319698in}}%
\pgfpathlineto{\pgfqpoint{3.886329in}{1.315304in}}%
\pgfpathlineto{\pgfqpoint{3.894307in}{1.611110in}}%
\pgfpathlineto{\pgfqpoint{3.902284in}{1.204213in}}%
\pgfpathlineto{\pgfqpoint{3.910261in}{1.667526in}}%
\pgfpathlineto{\pgfqpoint{3.918238in}{0.893669in}}%
\pgfpathlineto{\pgfqpoint{3.926215in}{1.367853in}}%
\pgfpathlineto{\pgfqpoint{3.934192in}{1.426469in}}%
\pgfpathlineto{\pgfqpoint{3.942170in}{1.769992in}}%
\pgfpathlineto{\pgfqpoint{3.950147in}{1.493175in}}%
\pgfpathlineto{\pgfqpoint{3.958124in}{1.112760in}}%
\pgfpathlineto{\pgfqpoint{3.966101in}{1.233885in}}%
\pgfpathlineto{\pgfqpoint{3.974078in}{1.841059in}}%
\pgfpathlineto{\pgfqpoint{3.982055in}{1.045895in}}%
\pgfpathlineto{\pgfqpoint{3.990033in}{1.294250in}}%
\pgfpathlineto{\pgfqpoint{3.998010in}{1.813387in}}%
\pgfpathlineto{\pgfqpoint{4.005987in}{1.442406in}}%
\pgfpathlineto{\pgfqpoint{4.013964in}{1.552480in}}%
\pgfpathlineto{\pgfqpoint{4.021941in}{0.909425in}}%
\pgfpathlineto{\pgfqpoint{4.029918in}{1.988430in}}%
\pgfpathlineto{\pgfqpoint{4.037896in}{1.370625in}}%
\pgfpathlineto{\pgfqpoint{4.045873in}{1.406669in}}%
\pgfpathlineto{\pgfqpoint{4.053850in}{1.340285in}}%
\pgfpathlineto{\pgfqpoint{4.061827in}{1.321555in}}%
\pgfpathlineto{\pgfqpoint{4.069804in}{1.313618in}}%
\pgfpathlineto{\pgfqpoint{4.077781in}{1.830198in}}%
\pgfpathlineto{\pgfqpoint{4.085759in}{1.614241in}}%
\pgfpathlineto{\pgfqpoint{4.093736in}{1.159411in}}%
\pgfpathlineto{\pgfqpoint{4.101713in}{1.843355in}}%
\pgfpathlineto{\pgfqpoint{4.109690in}{0.992852in}}%
\pgfpathlineto{\pgfqpoint{4.117667in}{1.529296in}}%
\pgfpathlineto{\pgfqpoint{4.125644in}{1.798367in}}%
\pgfpathlineto{\pgfqpoint{4.133622in}{1.752461in}}%
\pgfpathlineto{\pgfqpoint{4.141599in}{1.030808in}}%
\pgfpathlineto{\pgfqpoint{4.149576in}{1.892182in}}%
\pgfpathlineto{\pgfqpoint{4.157553in}{1.707076in}}%
\pgfpathlineto{\pgfqpoint{4.165530in}{1.341432in}}%
\pgfpathlineto{\pgfqpoint{4.173507in}{1.173289in}}%
\pgfpathlineto{\pgfqpoint{4.181485in}{1.825478in}}%
\pgfpathlineto{\pgfqpoint{4.189462in}{1.648817in}}%
\pgfpathlineto{\pgfqpoint{4.197439in}{1.158086in}}%
\pgfpathlineto{\pgfqpoint{4.205416in}{1.766874in}}%
\pgfpathlineto{\pgfqpoint{4.213393in}{1.948172in}}%
\pgfpathlineto{\pgfqpoint{4.221370in}{1.149945in}}%
\pgfpathlineto{\pgfqpoint{4.229347in}{2.059860in}}%
\pgfpathlineto{\pgfqpoint{4.237325in}{1.537667in}}%
\pgfpathlineto{\pgfqpoint{4.245302in}{1.548453in}}%
\pgfpathlineto{\pgfqpoint{4.253279in}{1.163536in}}%
\pgfpathlineto{\pgfqpoint{4.261256in}{2.158409in}}%
\pgfpathlineto{\pgfqpoint{4.269233in}{1.058499in}}%
\pgfpathlineto{\pgfqpoint{4.277210in}{1.715429in}}%
\pgfpathlineto{\pgfqpoint{4.285188in}{1.922559in}}%
\pgfpathlineto{\pgfqpoint{4.293165in}{1.282011in}}%
\pgfpathlineto{\pgfqpoint{4.301142in}{1.760812in}}%
\pgfpathlineto{\pgfqpoint{4.309119in}{1.509666in}}%
\pgfpathlineto{\pgfqpoint{4.317096in}{2.186335in}}%
\pgfpathlineto{\pgfqpoint{4.325073in}{1.722831in}}%
\pgfpathlineto{\pgfqpoint{4.333051in}{1.356814in}}%
\pgfpathlineto{\pgfqpoint{4.341028in}{1.953013in}}%
\pgfpathlineto{\pgfqpoint{4.349005in}{1.427394in}}%
\pgfpathlineto{\pgfqpoint{4.356982in}{1.900805in}}%
\pgfpathlineto{\pgfqpoint{4.364959in}{1.174055in}}%
\pgfpathlineto{\pgfqpoint{4.372936in}{1.723228in}}%
\pgfpathlineto{\pgfqpoint{4.380914in}{2.112353in}}%
\pgfpathlineto{\pgfqpoint{4.396868in}{1.484815in}}%
\pgfpathlineto{\pgfqpoint{4.404845in}{1.524782in}}%
\pgfpathlineto{\pgfqpoint{4.412822in}{1.975566in}}%
\pgfpathlineto{\pgfqpoint{4.420799in}{1.517897in}}%
\pgfpathlineto{\pgfqpoint{4.428777in}{1.700100in}}%
\pgfpathlineto{\pgfqpoint{4.436754in}{2.235350in}}%
\pgfpathlineto{\pgfqpoint{4.444731in}{1.893306in}}%
\pgfpathlineto{\pgfqpoint{4.452708in}{1.175318in}}%
\pgfpathlineto{\pgfqpoint{4.468662in}{2.401713in}}%
\pgfpathlineto{\pgfqpoint{4.476640in}{1.229163in}}%
\pgfpathlineto{\pgfqpoint{4.484617in}{1.931480in}}%
\pgfpathlineto{\pgfqpoint{4.492594in}{2.044749in}}%
\pgfpathlineto{\pgfqpoint{4.500571in}{1.860470in}}%
\pgfpathlineto{\pgfqpoint{4.508548in}{1.578250in}}%
\pgfpathlineto{\pgfqpoint{4.516525in}{1.869626in}}%
\pgfpathlineto{\pgfqpoint{4.524503in}{1.764304in}}%
\pgfpathlineto{\pgfqpoint{4.532480in}{1.862772in}}%
\pgfpathlineto{\pgfqpoint{4.540457in}{1.792394in}}%
\pgfpathlineto{\pgfqpoint{4.548434in}{2.131188in}}%
\pgfpathlineto{\pgfqpoint{4.556411in}{1.840053in}}%
\pgfpathlineto{\pgfqpoint{4.564388in}{1.832267in}}%
\pgfpathlineto{\pgfqpoint{4.572366in}{1.900211in}}%
\pgfpathlineto{\pgfqpoint{4.580343in}{1.891115in}}%
\pgfpathlineto{\pgfqpoint{4.588320in}{1.995753in}}%
\pgfpathlineto{\pgfqpoint{4.596297in}{1.554213in}}%
\pgfpathlineto{\pgfqpoint{4.604274in}{2.256867in}}%
\pgfpathlineto{\pgfqpoint{4.612251in}{2.141664in}}%
\pgfpathlineto{\pgfqpoint{4.620228in}{1.407426in}}%
\pgfpathlineto{\pgfqpoint{4.628206in}{2.111042in}}%
\pgfpathlineto{\pgfqpoint{4.636183in}{1.952670in}}%
\pgfpathlineto{\pgfqpoint{4.644160in}{1.939718in}}%
\pgfpathlineto{\pgfqpoint{4.652137in}{1.743012in}}%
\pgfpathlineto{\pgfqpoint{4.660114in}{2.250066in}}%
\pgfpathlineto{\pgfqpoint{4.676069in}{1.944032in}}%
\pgfpathlineto{\pgfqpoint{4.684046in}{1.893379in}}%
\pgfpathlineto{\pgfqpoint{4.692023in}{2.188013in}}%
\pgfpathlineto{\pgfqpoint{4.700000in}{2.152241in}}%
\pgfpathlineto{\pgfqpoint{4.707977in}{1.640066in}}%
\pgfpathlineto{\pgfqpoint{4.715954in}{2.130756in}}%
\pgfpathlineto{\pgfqpoint{4.723932in}{1.800389in}}%
\pgfpathlineto{\pgfqpoint{4.731909in}{2.015018in}}%
\pgfpathlineto{\pgfqpoint{4.739886in}{2.537847in}}%
\pgfpathlineto{\pgfqpoint{4.747863in}{2.093477in}}%
\pgfpathlineto{\pgfqpoint{4.755840in}{2.089525in}}%
\pgfpathlineto{\pgfqpoint{4.763817in}{1.689814in}}%
\pgfpathlineto{\pgfqpoint{4.771795in}{2.226625in}}%
\pgfpathlineto{\pgfqpoint{4.779772in}{2.168053in}}%
\pgfpathlineto{\pgfqpoint{4.787749in}{1.966774in}}%
\pgfpathlineto{\pgfqpoint{4.795726in}{2.085074in}}%
\pgfpathlineto{\pgfqpoint{4.803703in}{2.441464in}}%
\pgfpathlineto{\pgfqpoint{4.811680in}{1.660420in}}%
\pgfpathlineto{\pgfqpoint{4.819658in}{2.563008in}}%
\pgfpathlineto{\pgfqpoint{4.827635in}{2.045471in}}%
\pgfpathlineto{\pgfqpoint{4.835612in}{1.789906in}}%
\pgfpathlineto{\pgfqpoint{4.843589in}{2.473924in}}%
\pgfpathlineto{\pgfqpoint{4.859543in}{1.956683in}}%
\pgfpathlineto{\pgfqpoint{4.867521in}{2.439404in}}%
\pgfpathlineto{\pgfqpoint{4.875498in}{2.257480in}}%
\pgfpathlineto{\pgfqpoint{4.883475in}{1.843928in}}%
\pgfpathlineto{\pgfqpoint{4.891452in}{2.691604in}}%
\pgfpathlineto{\pgfqpoint{4.899429in}{2.196583in}}%
\pgfpathlineto{\pgfqpoint{4.907406in}{2.256310in}}%
\pgfpathlineto{\pgfqpoint{4.915384in}{1.856212in}}%
\pgfpathlineto{\pgfqpoint{4.923361in}{2.597359in}}%
\pgfpathlineto{\pgfqpoint{4.931338in}{2.285561in}}%
\pgfpathlineto{\pgfqpoint{4.939315in}{2.194758in}}%
\pgfpathlineto{\pgfqpoint{4.939315in}{2.194758in}}%
\pgfusepath{stroke}%
\end{pgfscope}%
\begin{pgfscope}%
\pgfsetrectcap%
\pgfsetmiterjoin%
\pgfsetlinewidth{0.803000pt}%
\definecolor{currentstroke}{rgb}{0.000000,0.000000,0.000000}%
\pgfsetstrokecolor{currentstroke}%
\pgfsetdash{}{0pt}%
\pgfpathmoveto{\pgfqpoint{2.811806in}{0.386111in}}%
\pgfpathlineto{\pgfqpoint{2.811806in}{2.801389in}}%
\pgfusepath{stroke}%
\end{pgfscope}%
\begin{pgfscope}%
\pgfsetrectcap%
\pgfsetmiterjoin%
\pgfsetlinewidth{0.803000pt}%
\definecolor{currentstroke}{rgb}{0.000000,0.000000,0.000000}%
\pgfsetstrokecolor{currentstroke}%
\pgfsetdash{}{0pt}%
\pgfpathmoveto{\pgfqpoint{5.040625in}{0.386111in}}%
\pgfpathlineto{\pgfqpoint{5.040625in}{2.801389in}}%
\pgfusepath{stroke}%
\end{pgfscope}%
\begin{pgfscope}%
\pgfsetrectcap%
\pgfsetmiterjoin%
\pgfsetlinewidth{0.803000pt}%
\definecolor{currentstroke}{rgb}{0.000000,0.000000,0.000000}%
\pgfsetstrokecolor{currentstroke}%
\pgfsetdash{}{0pt}%
\pgfpathmoveto{\pgfqpoint{2.811806in}{0.386111in}}%
\pgfpathlineto{\pgfqpoint{5.040625in}{0.386111in}}%
\pgfusepath{stroke}%
\end{pgfscope}%
\begin{pgfscope}%
\pgfsetrectcap%
\pgfsetmiterjoin%
\pgfsetlinewidth{0.803000pt}%
\definecolor{currentstroke}{rgb}{0.000000,0.000000,0.000000}%
\pgfsetstrokecolor{currentstroke}%
\pgfsetdash{}{0pt}%
\pgfpathmoveto{\pgfqpoint{2.811806in}{2.801389in}}%
\pgfpathlineto{\pgfqpoint{5.040625in}{2.801389in}}%
\pgfusepath{stroke}%
\end{pgfscope}%
\end{pgfpicture}%
\makeatother%
\endgroup%

    \caption{Verrauschte Signale und deren Ableitung\label{polynomials:noise:signals}}
\end{figure}

Was passiert nun wenn wir das verrauschte Signale mit einem Haar Wavelet
analysieren?  Wie in \autoref{polynomials:noise:db1} zu sehen bringt das so
noch keine wesentliche Verbesserung zum direkten Ableiten.

\begin{figure}
    \centering
    %% Creator: Matplotlib, PGF backend
%%
%% To include the figure in your LaTeX document, write
%%   \input{<filename>.pgf}
%%
%% Make sure the required packages are loaded in your preamble
%%   \usepackage{pgf}
%%
%% Figures using additional raster images can only be included by \input if
%% they are in the same directory as the main LaTeX file. For loading figures
%% from other directories you can use the `import` package
%%   \usepackage{import}
%% and then include the figures with
%%   \import{<path to file>}{<filename>.pgf}
%%
%% Matplotlib used the following preamble
%%   \usepackage{fontspec}
%%
\begingroup%
\makeatletter%
\begin{pgfpicture}%
\pgfpathrectangle{\pgfpointorigin}{\pgfqpoint{5.800000in}{3.000000in}}%
\pgfusepath{use as bounding box, clip}%
\begin{pgfscope}%
\pgfsetbuttcap%
\pgfsetmiterjoin%
\definecolor{currentfill}{rgb}{1.000000,1.000000,1.000000}%
\pgfsetfillcolor{currentfill}%
\pgfsetlinewidth{0.000000pt}%
\definecolor{currentstroke}{rgb}{1.000000,1.000000,1.000000}%
\pgfsetstrokecolor{currentstroke}%
\pgfsetdash{}{0pt}%
\pgfpathmoveto{\pgfqpoint{0.000000in}{0.000000in}}%
\pgfpathlineto{\pgfqpoint{5.800000in}{0.000000in}}%
\pgfpathlineto{\pgfqpoint{5.800000in}{3.000000in}}%
\pgfpathlineto{\pgfqpoint{0.000000in}{3.000000in}}%
\pgfpathclose%
\pgfusepath{fill}%
\end{pgfscope}%
\begin{pgfscope}%
\pgfsetbuttcap%
\pgfsetmiterjoin%
\definecolor{currentfill}{rgb}{1.000000,1.000000,1.000000}%
\pgfsetfillcolor{currentfill}%
\pgfsetlinewidth{0.000000pt}%
\definecolor{currentstroke}{rgb}{0.000000,0.000000,0.000000}%
\pgfsetstrokecolor{currentstroke}%
\pgfsetstrokeopacity{0.000000}%
\pgfsetdash{}{0pt}%
\pgfpathmoveto{\pgfqpoint{0.335222in}{0.315889in}}%
\pgfpathlineto{\pgfqpoint{2.699611in}{0.315889in}}%
\pgfpathlineto{\pgfqpoint{2.699611in}{2.701200in}}%
\pgfpathlineto{\pgfqpoint{0.335222in}{2.701200in}}%
\pgfpathclose%
\pgfusepath{fill}%
\end{pgfscope}%
\begin{pgfscope}%
\pgfsetbuttcap%
\pgfsetroundjoin%
\definecolor{currentfill}{rgb}{0.000000,0.000000,0.000000}%
\pgfsetfillcolor{currentfill}%
\pgfsetlinewidth{0.803000pt}%
\definecolor{currentstroke}{rgb}{0.000000,0.000000,0.000000}%
\pgfsetstrokecolor{currentstroke}%
\pgfsetdash{}{0pt}%
\pgfsys@defobject{currentmarker}{\pgfqpoint{0.000000in}{-0.048611in}}{\pgfqpoint{0.000000in}{0.000000in}}{%
\pgfpathmoveto{\pgfqpoint{0.000000in}{0.000000in}}%
\pgfpathlineto{\pgfqpoint{0.000000in}{-0.048611in}}%
\pgfusepath{stroke,fill}%
}%
\begin{pgfscope}%
\pgfsys@transformshift{0.442694in}{0.315889in}%
\pgfsys@useobject{currentmarker}{}%
\end{pgfscope}%
\end{pgfscope}%
\begin{pgfscope}%
\definecolor{textcolor}{rgb}{0.000000,0.000000,0.000000}%
\pgfsetstrokecolor{textcolor}%
\pgfsetfillcolor{textcolor}%
\pgftext[x=0.442694in,y=0.218667in,,top]{\color{textcolor}\rmfamily\fontsize{8.000000}{9.600000}\selectfont 0}%
\end{pgfscope}%
\begin{pgfscope}%
\pgfsetbuttcap%
\pgfsetroundjoin%
\definecolor{currentfill}{rgb}{0.000000,0.000000,0.000000}%
\pgfsetfillcolor{currentfill}%
\pgfsetlinewidth{0.803000pt}%
\definecolor{currentstroke}{rgb}{0.000000,0.000000,0.000000}%
\pgfsetstrokecolor{currentstroke}%
\pgfsetdash{}{0pt}%
\pgfsys@defobject{currentmarker}{\pgfqpoint{0.000000in}{-0.048611in}}{\pgfqpoint{0.000000in}{0.000000in}}{%
\pgfpathmoveto{\pgfqpoint{0.000000in}{0.000000in}}%
\pgfpathlineto{\pgfqpoint{0.000000in}{-0.048611in}}%
\pgfusepath{stroke,fill}%
}%
\begin{pgfscope}%
\pgfsys@transformshift{0.781190in}{0.315889in}%
\pgfsys@useobject{currentmarker}{}%
\end{pgfscope}%
\end{pgfscope}%
\begin{pgfscope}%
\definecolor{textcolor}{rgb}{0.000000,0.000000,0.000000}%
\pgfsetstrokecolor{textcolor}%
\pgfsetfillcolor{textcolor}%
\pgftext[x=0.781190in,y=0.218667in,,top]{\color{textcolor}\rmfamily\fontsize{8.000000}{9.600000}\selectfont 20}%
\end{pgfscope}%
\begin{pgfscope}%
\pgfsetbuttcap%
\pgfsetroundjoin%
\definecolor{currentfill}{rgb}{0.000000,0.000000,0.000000}%
\pgfsetfillcolor{currentfill}%
\pgfsetlinewidth{0.803000pt}%
\definecolor{currentstroke}{rgb}{0.000000,0.000000,0.000000}%
\pgfsetstrokecolor{currentstroke}%
\pgfsetdash{}{0pt}%
\pgfsys@defobject{currentmarker}{\pgfqpoint{0.000000in}{-0.048611in}}{\pgfqpoint{0.000000in}{0.000000in}}{%
\pgfpathmoveto{\pgfqpoint{0.000000in}{0.000000in}}%
\pgfpathlineto{\pgfqpoint{0.000000in}{-0.048611in}}%
\pgfusepath{stroke,fill}%
}%
\begin{pgfscope}%
\pgfsys@transformshift{1.119685in}{0.315889in}%
\pgfsys@useobject{currentmarker}{}%
\end{pgfscope}%
\end{pgfscope}%
\begin{pgfscope}%
\definecolor{textcolor}{rgb}{0.000000,0.000000,0.000000}%
\pgfsetstrokecolor{textcolor}%
\pgfsetfillcolor{textcolor}%
\pgftext[x=1.119685in,y=0.218667in,,top]{\color{textcolor}\rmfamily\fontsize{8.000000}{9.600000}\selectfont 40}%
\end{pgfscope}%
\begin{pgfscope}%
\pgfsetbuttcap%
\pgfsetroundjoin%
\definecolor{currentfill}{rgb}{0.000000,0.000000,0.000000}%
\pgfsetfillcolor{currentfill}%
\pgfsetlinewidth{0.803000pt}%
\definecolor{currentstroke}{rgb}{0.000000,0.000000,0.000000}%
\pgfsetstrokecolor{currentstroke}%
\pgfsetdash{}{0pt}%
\pgfsys@defobject{currentmarker}{\pgfqpoint{0.000000in}{-0.048611in}}{\pgfqpoint{0.000000in}{0.000000in}}{%
\pgfpathmoveto{\pgfqpoint{0.000000in}{0.000000in}}%
\pgfpathlineto{\pgfqpoint{0.000000in}{-0.048611in}}%
\pgfusepath{stroke,fill}%
}%
\begin{pgfscope}%
\pgfsys@transformshift{1.458180in}{0.315889in}%
\pgfsys@useobject{currentmarker}{}%
\end{pgfscope}%
\end{pgfscope}%
\begin{pgfscope}%
\definecolor{textcolor}{rgb}{0.000000,0.000000,0.000000}%
\pgfsetstrokecolor{textcolor}%
\pgfsetfillcolor{textcolor}%
\pgftext[x=1.458180in,y=0.218667in,,top]{\color{textcolor}\rmfamily\fontsize{8.000000}{9.600000}\selectfont 60}%
\end{pgfscope}%
\begin{pgfscope}%
\pgfsetbuttcap%
\pgfsetroundjoin%
\definecolor{currentfill}{rgb}{0.000000,0.000000,0.000000}%
\pgfsetfillcolor{currentfill}%
\pgfsetlinewidth{0.803000pt}%
\definecolor{currentstroke}{rgb}{0.000000,0.000000,0.000000}%
\pgfsetstrokecolor{currentstroke}%
\pgfsetdash{}{0pt}%
\pgfsys@defobject{currentmarker}{\pgfqpoint{0.000000in}{-0.048611in}}{\pgfqpoint{0.000000in}{0.000000in}}{%
\pgfpathmoveto{\pgfqpoint{0.000000in}{0.000000in}}%
\pgfpathlineto{\pgfqpoint{0.000000in}{-0.048611in}}%
\pgfusepath{stroke,fill}%
}%
\begin{pgfscope}%
\pgfsys@transformshift{1.796675in}{0.315889in}%
\pgfsys@useobject{currentmarker}{}%
\end{pgfscope}%
\end{pgfscope}%
\begin{pgfscope}%
\definecolor{textcolor}{rgb}{0.000000,0.000000,0.000000}%
\pgfsetstrokecolor{textcolor}%
\pgfsetfillcolor{textcolor}%
\pgftext[x=1.796675in,y=0.218667in,,top]{\color{textcolor}\rmfamily\fontsize{8.000000}{9.600000}\selectfont 80}%
\end{pgfscope}%
\begin{pgfscope}%
\pgfsetbuttcap%
\pgfsetroundjoin%
\definecolor{currentfill}{rgb}{0.000000,0.000000,0.000000}%
\pgfsetfillcolor{currentfill}%
\pgfsetlinewidth{0.803000pt}%
\definecolor{currentstroke}{rgb}{0.000000,0.000000,0.000000}%
\pgfsetstrokecolor{currentstroke}%
\pgfsetdash{}{0pt}%
\pgfsys@defobject{currentmarker}{\pgfqpoint{0.000000in}{-0.048611in}}{\pgfqpoint{0.000000in}{0.000000in}}{%
\pgfpathmoveto{\pgfqpoint{0.000000in}{0.000000in}}%
\pgfpathlineto{\pgfqpoint{0.000000in}{-0.048611in}}%
\pgfusepath{stroke,fill}%
}%
\begin{pgfscope}%
\pgfsys@transformshift{2.135170in}{0.315889in}%
\pgfsys@useobject{currentmarker}{}%
\end{pgfscope}%
\end{pgfscope}%
\begin{pgfscope}%
\definecolor{textcolor}{rgb}{0.000000,0.000000,0.000000}%
\pgfsetstrokecolor{textcolor}%
\pgfsetfillcolor{textcolor}%
\pgftext[x=2.135170in,y=0.218667in,,top]{\color{textcolor}\rmfamily\fontsize{8.000000}{9.600000}\selectfont 100}%
\end{pgfscope}%
\begin{pgfscope}%
\pgfsetbuttcap%
\pgfsetroundjoin%
\definecolor{currentfill}{rgb}{0.000000,0.000000,0.000000}%
\pgfsetfillcolor{currentfill}%
\pgfsetlinewidth{0.803000pt}%
\definecolor{currentstroke}{rgb}{0.000000,0.000000,0.000000}%
\pgfsetstrokecolor{currentstroke}%
\pgfsetdash{}{0pt}%
\pgfsys@defobject{currentmarker}{\pgfqpoint{0.000000in}{-0.048611in}}{\pgfqpoint{0.000000in}{0.000000in}}{%
\pgfpathmoveto{\pgfqpoint{0.000000in}{0.000000in}}%
\pgfpathlineto{\pgfqpoint{0.000000in}{-0.048611in}}%
\pgfusepath{stroke,fill}%
}%
\begin{pgfscope}%
\pgfsys@transformshift{2.473666in}{0.315889in}%
\pgfsys@useobject{currentmarker}{}%
\end{pgfscope}%
\end{pgfscope}%
\begin{pgfscope}%
\definecolor{textcolor}{rgb}{0.000000,0.000000,0.000000}%
\pgfsetstrokecolor{textcolor}%
\pgfsetfillcolor{textcolor}%
\pgftext[x=2.473666in,y=0.218667in,,top]{\color{textcolor}\rmfamily\fontsize{8.000000}{9.600000}\selectfont 120}%
\end{pgfscope}%
\begin{pgfscope}%
\pgfsetbuttcap%
\pgfsetroundjoin%
\definecolor{currentfill}{rgb}{0.000000,0.000000,0.000000}%
\pgfsetfillcolor{currentfill}%
\pgfsetlinewidth{0.803000pt}%
\definecolor{currentstroke}{rgb}{0.000000,0.000000,0.000000}%
\pgfsetstrokecolor{currentstroke}%
\pgfsetdash{}{0pt}%
\pgfsys@defobject{currentmarker}{\pgfqpoint{-0.048611in}{0.000000in}}{\pgfqpoint{0.000000in}{0.000000in}}{%
\pgfpathmoveto{\pgfqpoint{0.000000in}{0.000000in}}%
\pgfpathlineto{\pgfqpoint{-0.048611in}{0.000000in}}%
\pgfusepath{stroke,fill}%
}%
\begin{pgfscope}%
\pgfsys@transformshift{0.335222in}{0.421633in}%
\pgfsys@useobject{currentmarker}{}%
\end{pgfscope}%
\end{pgfscope}%
\begin{pgfscope}%
\definecolor{textcolor}{rgb}{0.000000,0.000000,0.000000}%
\pgfsetstrokecolor{textcolor}%
\pgfsetfillcolor{textcolor}%
\pgftext[x=0.179000in,y=0.383077in,left,base]{\color{textcolor}\rmfamily\fontsize{8.000000}{9.600000}\selectfont 0}%
\end{pgfscope}%
\begin{pgfscope}%
\pgfsetbuttcap%
\pgfsetroundjoin%
\definecolor{currentfill}{rgb}{0.000000,0.000000,0.000000}%
\pgfsetfillcolor{currentfill}%
\pgfsetlinewidth{0.803000pt}%
\definecolor{currentstroke}{rgb}{0.000000,0.000000,0.000000}%
\pgfsetstrokecolor{currentstroke}%
\pgfsetdash{}{0pt}%
\pgfsys@defobject{currentmarker}{\pgfqpoint{-0.048611in}{0.000000in}}{\pgfqpoint{0.000000in}{0.000000in}}{%
\pgfpathmoveto{\pgfqpoint{0.000000in}{0.000000in}}%
\pgfpathlineto{\pgfqpoint{-0.048611in}{0.000000in}}%
\pgfusepath{stroke,fill}%
}%
\begin{pgfscope}%
\pgfsys@transformshift{0.335222in}{0.807240in}%
\pgfsys@useobject{currentmarker}{}%
\end{pgfscope}%
\end{pgfscope}%
\begin{pgfscope}%
\definecolor{textcolor}{rgb}{0.000000,0.000000,0.000000}%
\pgfsetstrokecolor{textcolor}%
\pgfsetfillcolor{textcolor}%
\pgftext[x=0.179000in,y=0.768685in,left,base]{\color{textcolor}\rmfamily\fontsize{8.000000}{9.600000}\selectfont 2}%
\end{pgfscope}%
\begin{pgfscope}%
\pgfsetbuttcap%
\pgfsetroundjoin%
\definecolor{currentfill}{rgb}{0.000000,0.000000,0.000000}%
\pgfsetfillcolor{currentfill}%
\pgfsetlinewidth{0.803000pt}%
\definecolor{currentstroke}{rgb}{0.000000,0.000000,0.000000}%
\pgfsetstrokecolor{currentstroke}%
\pgfsetdash{}{0pt}%
\pgfsys@defobject{currentmarker}{\pgfqpoint{-0.048611in}{0.000000in}}{\pgfqpoint{0.000000in}{0.000000in}}{%
\pgfpathmoveto{\pgfqpoint{0.000000in}{0.000000in}}%
\pgfpathlineto{\pgfqpoint{-0.048611in}{0.000000in}}%
\pgfusepath{stroke,fill}%
}%
\begin{pgfscope}%
\pgfsys@transformshift{0.335222in}{1.192847in}%
\pgfsys@useobject{currentmarker}{}%
\end{pgfscope}%
\end{pgfscope}%
\begin{pgfscope}%
\definecolor{textcolor}{rgb}{0.000000,0.000000,0.000000}%
\pgfsetstrokecolor{textcolor}%
\pgfsetfillcolor{textcolor}%
\pgftext[x=0.179000in,y=1.154292in,left,base]{\color{textcolor}\rmfamily\fontsize{8.000000}{9.600000}\selectfont 4}%
\end{pgfscope}%
\begin{pgfscope}%
\pgfsetbuttcap%
\pgfsetroundjoin%
\definecolor{currentfill}{rgb}{0.000000,0.000000,0.000000}%
\pgfsetfillcolor{currentfill}%
\pgfsetlinewidth{0.803000pt}%
\definecolor{currentstroke}{rgb}{0.000000,0.000000,0.000000}%
\pgfsetstrokecolor{currentstroke}%
\pgfsetdash{}{0pt}%
\pgfsys@defobject{currentmarker}{\pgfqpoint{-0.048611in}{0.000000in}}{\pgfqpoint{0.000000in}{0.000000in}}{%
\pgfpathmoveto{\pgfqpoint{0.000000in}{0.000000in}}%
\pgfpathlineto{\pgfqpoint{-0.048611in}{0.000000in}}%
\pgfusepath{stroke,fill}%
}%
\begin{pgfscope}%
\pgfsys@transformshift{0.335222in}{1.578454in}%
\pgfsys@useobject{currentmarker}{}%
\end{pgfscope}%
\end{pgfscope}%
\begin{pgfscope}%
\definecolor{textcolor}{rgb}{0.000000,0.000000,0.000000}%
\pgfsetstrokecolor{textcolor}%
\pgfsetfillcolor{textcolor}%
\pgftext[x=0.179000in,y=1.539899in,left,base]{\color{textcolor}\rmfamily\fontsize{8.000000}{9.600000}\selectfont 6}%
\end{pgfscope}%
\begin{pgfscope}%
\pgfsetbuttcap%
\pgfsetroundjoin%
\definecolor{currentfill}{rgb}{0.000000,0.000000,0.000000}%
\pgfsetfillcolor{currentfill}%
\pgfsetlinewidth{0.803000pt}%
\definecolor{currentstroke}{rgb}{0.000000,0.000000,0.000000}%
\pgfsetstrokecolor{currentstroke}%
\pgfsetdash{}{0pt}%
\pgfsys@defobject{currentmarker}{\pgfqpoint{-0.048611in}{0.000000in}}{\pgfqpoint{0.000000in}{0.000000in}}{%
\pgfpathmoveto{\pgfqpoint{0.000000in}{0.000000in}}%
\pgfpathlineto{\pgfqpoint{-0.048611in}{0.000000in}}%
\pgfusepath{stroke,fill}%
}%
\begin{pgfscope}%
\pgfsys@transformshift{0.335222in}{1.964062in}%
\pgfsys@useobject{currentmarker}{}%
\end{pgfscope}%
\end{pgfscope}%
\begin{pgfscope}%
\definecolor{textcolor}{rgb}{0.000000,0.000000,0.000000}%
\pgfsetstrokecolor{textcolor}%
\pgfsetfillcolor{textcolor}%
\pgftext[x=0.179000in,y=1.925506in,left,base]{\color{textcolor}\rmfamily\fontsize{8.000000}{9.600000}\selectfont 8}%
\end{pgfscope}%
\begin{pgfscope}%
\pgfsetbuttcap%
\pgfsetroundjoin%
\definecolor{currentfill}{rgb}{0.000000,0.000000,0.000000}%
\pgfsetfillcolor{currentfill}%
\pgfsetlinewidth{0.803000pt}%
\definecolor{currentstroke}{rgb}{0.000000,0.000000,0.000000}%
\pgfsetstrokecolor{currentstroke}%
\pgfsetdash{}{0pt}%
\pgfsys@defobject{currentmarker}{\pgfqpoint{-0.048611in}{0.000000in}}{\pgfqpoint{0.000000in}{0.000000in}}{%
\pgfpathmoveto{\pgfqpoint{0.000000in}{0.000000in}}%
\pgfpathlineto{\pgfqpoint{-0.048611in}{0.000000in}}%
\pgfusepath{stroke,fill}%
}%
\begin{pgfscope}%
\pgfsys@transformshift{0.335222in}{2.349669in}%
\pgfsys@useobject{currentmarker}{}%
\end{pgfscope}%
\end{pgfscope}%
\begin{pgfscope}%
\definecolor{textcolor}{rgb}{0.000000,0.000000,0.000000}%
\pgfsetstrokecolor{textcolor}%
\pgfsetfillcolor{textcolor}%
\pgftext[x=0.120000in,y=2.311113in,left,base]{\color{textcolor}\rmfamily\fontsize{8.000000}{9.600000}\selectfont 10}%
\end{pgfscope}%
\begin{pgfscope}%
\pgfpathrectangle{\pgfqpoint{0.335222in}{0.315889in}}{\pgfqpoint{2.364389in}{2.385311in}}%
\pgfusepath{clip}%
\pgfsetrectcap%
\pgfsetroundjoin%
\pgfsetlinewidth{1.505625pt}%
\definecolor{currentstroke}{rgb}{0.121569,0.466667,0.705882}%
\pgfsetstrokecolor{currentstroke}%
\pgfsetdash{}{0pt}%
\pgfpathmoveto{\pgfqpoint{0.442694in}{0.697324in}}%
\pgfpathlineto{\pgfqpoint{0.459619in}{0.703324in}}%
\pgfpathlineto{\pgfqpoint{0.476544in}{0.700487in}}%
\pgfpathlineto{\pgfqpoint{0.493469in}{0.705508in}}%
\pgfpathlineto{\pgfqpoint{0.510393in}{0.701330in}}%
\pgfpathlineto{\pgfqpoint{0.527318in}{0.699829in}}%
\pgfpathlineto{\pgfqpoint{0.544243in}{0.700025in}}%
\pgfpathlineto{\pgfqpoint{0.561168in}{0.702195in}}%
\pgfpathlineto{\pgfqpoint{0.578093in}{0.697393in}}%
\pgfpathlineto{\pgfqpoint{0.628867in}{0.703054in}}%
\pgfpathlineto{\pgfqpoint{0.645792in}{0.699022in}}%
\pgfpathlineto{\pgfqpoint{0.679641in}{0.694676in}}%
\pgfpathlineto{\pgfqpoint{0.696566in}{0.703423in}}%
\pgfpathlineto{\pgfqpoint{0.713491in}{0.699020in}}%
\pgfpathlineto{\pgfqpoint{0.747340in}{0.703790in}}%
\pgfpathlineto{\pgfqpoint{0.764265in}{0.697514in}}%
\pgfpathlineto{\pgfqpoint{0.781190in}{0.702557in}}%
\pgfpathlineto{\pgfqpoint{0.815039in}{0.698897in}}%
\pgfpathlineto{\pgfqpoint{0.831964in}{0.698637in}}%
\pgfpathlineto{\pgfqpoint{0.848889in}{0.702751in}}%
\pgfpathlineto{\pgfqpoint{0.916588in}{0.699178in}}%
\pgfpathlineto{\pgfqpoint{0.933512in}{0.706744in}}%
\pgfpathlineto{\pgfqpoint{0.950437in}{0.700399in}}%
\pgfpathlineto{\pgfqpoint{0.967362in}{0.703020in}}%
\pgfpathlineto{\pgfqpoint{1.001212in}{0.701147in}}%
\pgfpathlineto{\pgfqpoint{1.018136in}{0.706110in}}%
\pgfpathlineto{\pgfqpoint{1.035061in}{0.701126in}}%
\pgfpathlineto{\pgfqpoint{1.051986in}{0.702496in}}%
\pgfpathlineto{\pgfqpoint{1.068911in}{0.696012in}}%
\pgfpathlineto{\pgfqpoint{1.102760in}{0.694959in}}%
\pgfpathlineto{\pgfqpoint{1.119685in}{0.702290in}}%
\pgfpathlineto{\pgfqpoint{1.136610in}{0.699335in}}%
\pgfpathlineto{\pgfqpoint{1.153534in}{0.704362in}}%
\pgfpathlineto{\pgfqpoint{1.170459in}{0.703828in}}%
\pgfpathlineto{\pgfqpoint{1.187384in}{0.698067in}}%
\pgfpathlineto{\pgfqpoint{1.204309in}{0.705614in}}%
\pgfpathlineto{\pgfqpoint{1.221233in}{0.700365in}}%
\pgfpathlineto{\pgfqpoint{1.255083in}{0.705130in}}%
\pgfpathlineto{\pgfqpoint{1.272008in}{0.698962in}}%
\pgfpathlineto{\pgfqpoint{1.288932in}{0.695802in}}%
\pgfpathlineto{\pgfqpoint{1.322782in}{0.699850in}}%
\pgfpathlineto{\pgfqpoint{1.339707in}{0.696734in}}%
\pgfpathlineto{\pgfqpoint{1.356631in}{0.701281in}}%
\pgfpathlineto{\pgfqpoint{1.373556in}{0.698165in}}%
\pgfpathlineto{\pgfqpoint{1.390481in}{0.705341in}}%
\pgfpathlineto{\pgfqpoint{1.407406in}{0.705352in}}%
\pgfpathlineto{\pgfqpoint{1.424330in}{0.697652in}}%
\pgfpathlineto{\pgfqpoint{1.441255in}{0.704159in}}%
\pgfpathlineto{\pgfqpoint{1.458180in}{0.702093in}}%
\pgfpathlineto{\pgfqpoint{1.475105in}{0.696668in}}%
\pgfpathlineto{\pgfqpoint{1.492030in}{0.703322in}}%
\pgfpathlineto{\pgfqpoint{1.508954in}{0.701227in}}%
\pgfpathlineto{\pgfqpoint{1.559729in}{0.700735in}}%
\pgfpathlineto{\pgfqpoint{1.576653in}{0.697712in}}%
\pgfpathlineto{\pgfqpoint{1.593578in}{0.700160in}}%
\pgfpathlineto{\pgfqpoint{1.610503in}{0.698516in}}%
\pgfpathlineto{\pgfqpoint{1.627428in}{0.703097in}}%
\pgfpathlineto{\pgfqpoint{1.644352in}{0.701109in}}%
\pgfpathlineto{\pgfqpoint{1.661277in}{0.703409in}}%
\pgfpathlineto{\pgfqpoint{1.678202in}{0.701925in}}%
\pgfpathlineto{\pgfqpoint{1.695127in}{0.705253in}}%
\pgfpathlineto{\pgfqpoint{1.712051in}{0.704590in}}%
\pgfpathlineto{\pgfqpoint{1.728976in}{0.700140in}}%
\pgfpathlineto{\pgfqpoint{1.745901in}{0.699812in}}%
\pgfpathlineto{\pgfqpoint{1.762826in}{0.701160in}}%
\pgfpathlineto{\pgfqpoint{1.779750in}{0.700538in}}%
\pgfpathlineto{\pgfqpoint{1.796675in}{0.704591in}}%
\pgfpathlineto{\pgfqpoint{1.813600in}{0.696434in}}%
\pgfpathlineto{\pgfqpoint{1.830525in}{0.700560in}}%
\pgfpathlineto{\pgfqpoint{1.847449in}{0.702271in}}%
\pgfpathlineto{\pgfqpoint{1.864374in}{0.698788in}}%
\pgfpathlineto{\pgfqpoint{1.881299in}{0.702302in}}%
\pgfpathlineto{\pgfqpoint{1.898224in}{0.702754in}}%
\pgfpathlineto{\pgfqpoint{1.915149in}{0.696825in}}%
\pgfpathlineto{\pgfqpoint{1.932073in}{0.703808in}}%
\pgfpathlineto{\pgfqpoint{1.948998in}{0.698839in}}%
\pgfpathlineto{\pgfqpoint{1.965923in}{0.699415in}}%
\pgfpathlineto{\pgfqpoint{1.982848in}{0.697758in}}%
\pgfpathlineto{\pgfqpoint{1.999772in}{0.703112in}}%
\pgfpathlineto{\pgfqpoint{2.016697in}{0.699975in}}%
\pgfpathlineto{\pgfqpoint{2.033622in}{0.698526in}}%
\pgfpathlineto{\pgfqpoint{2.050547in}{0.706152in}}%
\pgfpathlineto{\pgfqpoint{2.067471in}{0.700065in}}%
\pgfpathlineto{\pgfqpoint{2.084396in}{0.697947in}}%
\pgfpathlineto{\pgfqpoint{2.101321in}{0.703224in}}%
\pgfpathlineto{\pgfqpoint{2.135170in}{0.703836in}}%
\pgfpathlineto{\pgfqpoint{2.152095in}{0.700524in}}%
\pgfpathlineto{\pgfqpoint{2.202869in}{0.705372in}}%
\pgfpathlineto{\pgfqpoint{2.253644in}{0.704037in}}%
\pgfpathlineto{\pgfqpoint{2.270568in}{0.701959in}}%
\pgfpathlineto{\pgfqpoint{2.287493in}{0.697356in}}%
\pgfpathlineto{\pgfqpoint{2.304418in}{0.697003in}}%
\pgfpathlineto{\pgfqpoint{2.321343in}{0.700329in}}%
\pgfpathlineto{\pgfqpoint{2.355192in}{0.703460in}}%
\pgfpathlineto{\pgfqpoint{2.372117in}{0.700775in}}%
\pgfpathlineto{\pgfqpoint{2.405967in}{0.699618in}}%
\pgfpathlineto{\pgfqpoint{2.422891in}{0.701615in}}%
\pgfpathlineto{\pgfqpoint{2.439816in}{0.700807in}}%
\pgfpathlineto{\pgfqpoint{2.456741in}{0.702141in}}%
\pgfpathlineto{\pgfqpoint{2.507515in}{0.699305in}}%
\pgfpathlineto{\pgfqpoint{2.524440in}{0.701742in}}%
\pgfpathlineto{\pgfqpoint{2.541365in}{0.697381in}}%
\pgfpathlineto{\pgfqpoint{2.558289in}{0.704284in}}%
\pgfpathlineto{\pgfqpoint{2.592139in}{0.696995in}}%
\pgfpathlineto{\pgfqpoint{2.592139in}{0.696995in}}%
\pgfusepath{stroke}%
\end{pgfscope}%
\begin{pgfscope}%
\pgfpathrectangle{\pgfqpoint{0.335222in}{0.315889in}}{\pgfqpoint{2.364389in}{2.385311in}}%
\pgfusepath{clip}%
\pgfsetrectcap%
\pgfsetroundjoin%
\pgfsetlinewidth{1.505625pt}%
\definecolor{currentstroke}{rgb}{1.000000,0.498039,0.054902}%
\pgfsetstrokecolor{currentstroke}%
\pgfsetdash{}{0pt}%
\pgfpathmoveto{\pgfqpoint{0.442694in}{0.428388in}}%
\pgfpathlineto{\pgfqpoint{0.476544in}{0.439999in}}%
\pgfpathlineto{\pgfqpoint{0.510393in}{0.445481in}}%
\pgfpathlineto{\pgfqpoint{0.527318in}{0.450326in}}%
\pgfpathlineto{\pgfqpoint{0.544243in}{0.453332in}}%
\pgfpathlineto{\pgfqpoint{0.561168in}{0.459752in}}%
\pgfpathlineto{\pgfqpoint{0.578093in}{0.459421in}}%
\pgfpathlineto{\pgfqpoint{0.595017in}{0.467321in}}%
\pgfpathlineto{\pgfqpoint{0.628867in}{0.477702in}}%
\pgfpathlineto{\pgfqpoint{0.645792in}{0.477576in}}%
\pgfpathlineto{\pgfqpoint{0.662716in}{0.488916in}}%
\pgfpathlineto{\pgfqpoint{0.679641in}{0.489014in}}%
\pgfpathlineto{\pgfqpoint{0.696566in}{0.495500in}}%
\pgfpathlineto{\pgfqpoint{0.713491in}{0.493810in}}%
\pgfpathlineto{\pgfqpoint{0.730415in}{0.498497in}}%
\pgfpathlineto{\pgfqpoint{0.747340in}{0.505048in}}%
\pgfpathlineto{\pgfqpoint{0.764265in}{0.506099in}}%
\pgfpathlineto{\pgfqpoint{0.781190in}{0.517904in}}%
\pgfpathlineto{\pgfqpoint{0.815039in}{0.525088in}}%
\pgfpathlineto{\pgfqpoint{0.831964in}{0.526076in}}%
\pgfpathlineto{\pgfqpoint{0.848889in}{0.537760in}}%
\pgfpathlineto{\pgfqpoint{0.865813in}{0.538542in}}%
\pgfpathlineto{\pgfqpoint{0.882738in}{0.544671in}}%
\pgfpathlineto{\pgfqpoint{0.899663in}{0.544673in}}%
\pgfpathlineto{\pgfqpoint{0.916588in}{0.548108in}}%
\pgfpathlineto{\pgfqpoint{0.933512in}{0.549028in}}%
\pgfpathlineto{\pgfqpoint{0.967362in}{0.564313in}}%
\pgfpathlineto{\pgfqpoint{0.984287in}{0.561169in}}%
\pgfpathlineto{\pgfqpoint{1.001212in}{0.574884in}}%
\pgfpathlineto{\pgfqpoint{1.035061in}{0.574323in}}%
\pgfpathlineto{\pgfqpoint{1.085835in}{0.598421in}}%
\pgfpathlineto{\pgfqpoint{1.102760in}{0.591407in}}%
\pgfpathlineto{\pgfqpoint{1.119685in}{0.598944in}}%
\pgfpathlineto{\pgfqpoint{1.153534in}{0.611503in}}%
\pgfpathlineto{\pgfqpoint{1.170459in}{0.613236in}}%
\pgfpathlineto{\pgfqpoint{1.204309in}{0.624111in}}%
\pgfpathlineto{\pgfqpoint{1.221233in}{0.621514in}}%
\pgfpathlineto{\pgfqpoint{1.255083in}{0.636425in}}%
\pgfpathlineto{\pgfqpoint{1.272008in}{0.640119in}}%
\pgfpathlineto{\pgfqpoint{1.288932in}{0.645778in}}%
\pgfpathlineto{\pgfqpoint{1.305857in}{0.642590in}}%
\pgfpathlineto{\pgfqpoint{1.322782in}{0.650109in}}%
\pgfpathlineto{\pgfqpoint{1.356631in}{0.660513in}}%
\pgfpathlineto{\pgfqpoint{1.373556in}{0.661152in}}%
\pgfpathlineto{\pgfqpoint{1.390481in}{0.673777in}}%
\pgfpathlineto{\pgfqpoint{1.407406in}{0.673811in}}%
\pgfpathlineto{\pgfqpoint{1.441255in}{0.686167in}}%
\pgfpathlineto{\pgfqpoint{1.458180in}{0.683057in}}%
\pgfpathlineto{\pgfqpoint{1.475105in}{0.692271in}}%
\pgfpathlineto{\pgfqpoint{1.492030in}{0.693774in}}%
\pgfpathlineto{\pgfqpoint{1.508954in}{0.699070in}}%
\pgfpathlineto{\pgfqpoint{1.525879in}{0.701497in}}%
\pgfpathlineto{\pgfqpoint{1.542804in}{0.706642in}}%
\pgfpathlineto{\pgfqpoint{1.559729in}{0.708782in}}%
\pgfpathlineto{\pgfqpoint{1.576653in}{0.716221in}}%
\pgfpathlineto{\pgfqpoint{1.593578in}{0.720173in}}%
\pgfpathlineto{\pgfqpoint{1.610503in}{0.729151in}}%
\pgfpathlineto{\pgfqpoint{1.627428in}{0.724436in}}%
\pgfpathlineto{\pgfqpoint{1.644352in}{0.730925in}}%
\pgfpathlineto{\pgfqpoint{1.661277in}{0.733330in}}%
\pgfpathlineto{\pgfqpoint{1.695127in}{0.745271in}}%
\pgfpathlineto{\pgfqpoint{1.712051in}{0.753563in}}%
\pgfpathlineto{\pgfqpoint{1.728976in}{0.752445in}}%
\pgfpathlineto{\pgfqpoint{1.745901in}{0.756686in}}%
\pgfpathlineto{\pgfqpoint{1.762826in}{0.766985in}}%
\pgfpathlineto{\pgfqpoint{1.779750in}{0.767224in}}%
\pgfpathlineto{\pgfqpoint{1.796675in}{0.771249in}}%
\pgfpathlineto{\pgfqpoint{1.813600in}{0.779373in}}%
\pgfpathlineto{\pgfqpoint{1.830525in}{0.778221in}}%
\pgfpathlineto{\pgfqpoint{1.847449in}{0.779803in}}%
\pgfpathlineto{\pgfqpoint{1.864374in}{0.793903in}}%
\pgfpathlineto{\pgfqpoint{1.898224in}{0.796835in}}%
\pgfpathlineto{\pgfqpoint{1.915149in}{0.800919in}}%
\pgfpathlineto{\pgfqpoint{1.932073in}{0.808890in}}%
\pgfpathlineto{\pgfqpoint{1.948998in}{0.803740in}}%
\pgfpathlineto{\pgfqpoint{1.965923in}{0.812122in}}%
\pgfpathlineto{\pgfqpoint{1.999772in}{0.822511in}}%
\pgfpathlineto{\pgfqpoint{2.016697in}{0.824986in}}%
\pgfpathlineto{\pgfqpoint{2.050547in}{0.837734in}}%
\pgfpathlineto{\pgfqpoint{2.067471in}{0.836423in}}%
\pgfpathlineto{\pgfqpoint{2.084396in}{0.846366in}}%
\pgfpathlineto{\pgfqpoint{2.101321in}{0.845344in}}%
\pgfpathlineto{\pgfqpoint{2.118246in}{0.855023in}}%
\pgfpathlineto{\pgfqpoint{2.135170in}{0.853810in}}%
\pgfpathlineto{\pgfqpoint{2.152095in}{0.864531in}}%
\pgfpathlineto{\pgfqpoint{2.169020in}{0.864799in}}%
\pgfpathlineto{\pgfqpoint{2.219794in}{0.875728in}}%
\pgfpathlineto{\pgfqpoint{2.236719in}{0.883640in}}%
\pgfpathlineto{\pgfqpoint{2.253644in}{0.889227in}}%
\pgfpathlineto{\pgfqpoint{2.287493in}{0.889483in}}%
\pgfpathlineto{\pgfqpoint{2.304418in}{0.899954in}}%
\pgfpathlineto{\pgfqpoint{2.338267in}{0.904655in}}%
\pgfpathlineto{\pgfqpoint{2.355192in}{0.916171in}}%
\pgfpathlineto{\pgfqpoint{2.389042in}{0.923279in}}%
\pgfpathlineto{\pgfqpoint{2.405967in}{0.923208in}}%
\pgfpathlineto{\pgfqpoint{2.422891in}{0.930252in}}%
\pgfpathlineto{\pgfqpoint{2.439816in}{0.929217in}}%
\pgfpathlineto{\pgfqpoint{2.456741in}{0.942346in}}%
\pgfpathlineto{\pgfqpoint{2.490590in}{0.944445in}}%
\pgfpathlineto{\pgfqpoint{2.507515in}{0.953176in}}%
\pgfpathlineto{\pgfqpoint{2.524440in}{0.955372in}}%
\pgfpathlineto{\pgfqpoint{2.541365in}{0.963494in}}%
\pgfpathlineto{\pgfqpoint{2.558289in}{0.961584in}}%
\pgfpathlineto{\pgfqpoint{2.575214in}{0.970325in}}%
\pgfpathlineto{\pgfqpoint{2.592139in}{0.971560in}}%
\pgfpathlineto{\pgfqpoint{2.592139in}{0.971560in}}%
\pgfusepath{stroke}%
\end{pgfscope}%
\begin{pgfscope}%
\pgfpathrectangle{\pgfqpoint{0.335222in}{0.315889in}}{\pgfqpoint{2.364389in}{2.385311in}}%
\pgfusepath{clip}%
\pgfsetrectcap%
\pgfsetroundjoin%
\pgfsetlinewidth{1.505625pt}%
\definecolor{currentstroke}{rgb}{0.172549,0.627451,0.172549}%
\pgfsetstrokecolor{currentstroke}%
\pgfsetdash{}{0pt}%
\pgfpathmoveto{\pgfqpoint{0.442694in}{0.428181in}}%
\pgfpathlineto{\pgfqpoint{0.476544in}{0.428053in}}%
\pgfpathlineto{\pgfqpoint{0.493469in}{0.426396in}}%
\pgfpathlineto{\pgfqpoint{0.510393in}{0.427106in}}%
\pgfpathlineto{\pgfqpoint{0.527318in}{0.424452in}}%
\pgfpathlineto{\pgfqpoint{0.561168in}{0.429271in}}%
\pgfpathlineto{\pgfqpoint{0.578093in}{0.436489in}}%
\pgfpathlineto{\pgfqpoint{0.611942in}{0.433856in}}%
\pgfpathlineto{\pgfqpoint{0.628867in}{0.435357in}}%
\pgfpathlineto{\pgfqpoint{0.645792in}{0.440536in}}%
\pgfpathlineto{\pgfqpoint{0.662716in}{0.438383in}}%
\pgfpathlineto{\pgfqpoint{0.679641in}{0.441181in}}%
\pgfpathlineto{\pgfqpoint{0.696566in}{0.448982in}}%
\pgfpathlineto{\pgfqpoint{0.730415in}{0.449835in}}%
\pgfpathlineto{\pgfqpoint{0.747340in}{0.449326in}}%
\pgfpathlineto{\pgfqpoint{0.764265in}{0.456664in}}%
\pgfpathlineto{\pgfqpoint{0.798114in}{0.455840in}}%
\pgfpathlineto{\pgfqpoint{0.831964in}{0.465658in}}%
\pgfpathlineto{\pgfqpoint{0.848889in}{0.466269in}}%
\pgfpathlineto{\pgfqpoint{0.865813in}{0.470201in}}%
\pgfpathlineto{\pgfqpoint{0.882738in}{0.479297in}}%
\pgfpathlineto{\pgfqpoint{0.899663in}{0.474557in}}%
\pgfpathlineto{\pgfqpoint{0.916588in}{0.485689in}}%
\pgfpathlineto{\pgfqpoint{0.933512in}{0.482059in}}%
\pgfpathlineto{\pgfqpoint{0.950437in}{0.490697in}}%
\pgfpathlineto{\pgfqpoint{0.967362in}{0.496389in}}%
\pgfpathlineto{\pgfqpoint{0.984287in}{0.499140in}}%
\pgfpathlineto{\pgfqpoint{1.001212in}{0.499644in}}%
\pgfpathlineto{\pgfqpoint{1.035061in}{0.514032in}}%
\pgfpathlineto{\pgfqpoint{1.068911in}{0.520685in}}%
\pgfpathlineto{\pgfqpoint{1.085835in}{0.529231in}}%
\pgfpathlineto{\pgfqpoint{1.102760in}{0.534591in}}%
\pgfpathlineto{\pgfqpoint{1.119685in}{0.536554in}}%
\pgfpathlineto{\pgfqpoint{1.136610in}{0.544396in}}%
\pgfpathlineto{\pgfqpoint{1.153534in}{0.545035in}}%
\pgfpathlineto{\pgfqpoint{1.170459in}{0.553588in}}%
\pgfpathlineto{\pgfqpoint{1.204309in}{0.562924in}}%
\pgfpathlineto{\pgfqpoint{1.221233in}{0.572491in}}%
\pgfpathlineto{\pgfqpoint{1.238158in}{0.574268in}}%
\pgfpathlineto{\pgfqpoint{1.272008in}{0.592329in}}%
\pgfpathlineto{\pgfqpoint{1.322782in}{0.608199in}}%
\pgfpathlineto{\pgfqpoint{1.339707in}{0.617473in}}%
\pgfpathlineto{\pgfqpoint{1.356631in}{0.622758in}}%
\pgfpathlineto{\pgfqpoint{1.373556in}{0.636722in}}%
\pgfpathlineto{\pgfqpoint{1.390481in}{0.641184in}}%
\pgfpathlineto{\pgfqpoint{1.424330in}{0.659296in}}%
\pgfpathlineto{\pgfqpoint{1.441255in}{0.664039in}}%
\pgfpathlineto{\pgfqpoint{1.475105in}{0.679594in}}%
\pgfpathlineto{\pgfqpoint{1.492030in}{0.694999in}}%
\pgfpathlineto{\pgfqpoint{1.525879in}{0.704104in}}%
\pgfpathlineto{\pgfqpoint{1.542804in}{0.718122in}}%
\pgfpathlineto{\pgfqpoint{1.559729in}{0.723476in}}%
\pgfpathlineto{\pgfqpoint{1.576653in}{0.725434in}}%
\pgfpathlineto{\pgfqpoint{1.593578in}{0.746795in}}%
\pgfpathlineto{\pgfqpoint{1.610503in}{0.748552in}}%
\pgfpathlineto{\pgfqpoint{1.627428in}{0.755647in}}%
\pgfpathlineto{\pgfqpoint{1.644352in}{0.766371in}}%
\pgfpathlineto{\pgfqpoint{1.661277in}{0.782429in}}%
\pgfpathlineto{\pgfqpoint{1.695127in}{0.799264in}}%
\pgfpathlineto{\pgfqpoint{1.712051in}{0.805147in}}%
\pgfpathlineto{\pgfqpoint{1.728976in}{0.821454in}}%
\pgfpathlineto{\pgfqpoint{1.745901in}{0.826736in}}%
\pgfpathlineto{\pgfqpoint{1.762826in}{0.844893in}}%
\pgfpathlineto{\pgfqpoint{1.796675in}{0.863452in}}%
\pgfpathlineto{\pgfqpoint{1.813600in}{0.874442in}}%
\pgfpathlineto{\pgfqpoint{1.830525in}{0.881483in}}%
\pgfpathlineto{\pgfqpoint{1.847449in}{0.893071in}}%
\pgfpathlineto{\pgfqpoint{1.864374in}{0.908963in}}%
\pgfpathlineto{\pgfqpoint{1.881299in}{0.914880in}}%
\pgfpathlineto{\pgfqpoint{1.898224in}{0.933117in}}%
\pgfpathlineto{\pgfqpoint{1.932073in}{0.948332in}}%
\pgfpathlineto{\pgfqpoint{1.948998in}{0.969317in}}%
\pgfpathlineto{\pgfqpoint{1.982848in}{0.990238in}}%
\pgfpathlineto{\pgfqpoint{1.999772in}{1.000191in}}%
\pgfpathlineto{\pgfqpoint{2.016697in}{1.007571in}}%
\pgfpathlineto{\pgfqpoint{2.033622in}{1.029847in}}%
\pgfpathlineto{\pgfqpoint{2.050547in}{1.033725in}}%
\pgfpathlineto{\pgfqpoint{2.067471in}{1.047369in}}%
\pgfpathlineto{\pgfqpoint{2.084396in}{1.063456in}}%
\pgfpathlineto{\pgfqpoint{2.101321in}{1.072442in}}%
\pgfpathlineto{\pgfqpoint{2.118246in}{1.090129in}}%
\pgfpathlineto{\pgfqpoint{2.135170in}{1.096736in}}%
\pgfpathlineto{\pgfqpoint{2.152095in}{1.112003in}}%
\pgfpathlineto{\pgfqpoint{2.169020in}{1.124811in}}%
\pgfpathlineto{\pgfqpoint{2.202869in}{1.161381in}}%
\pgfpathlineto{\pgfqpoint{2.219794in}{1.174595in}}%
\pgfpathlineto{\pgfqpoint{2.236719in}{1.185043in}}%
\pgfpathlineto{\pgfqpoint{2.270568in}{1.210185in}}%
\pgfpathlineto{\pgfqpoint{2.287493in}{1.225945in}}%
\pgfpathlineto{\pgfqpoint{2.321343in}{1.262891in}}%
\pgfpathlineto{\pgfqpoint{2.338267in}{1.273147in}}%
\pgfpathlineto{\pgfqpoint{2.355192in}{1.285954in}}%
\pgfpathlineto{\pgfqpoint{2.372117in}{1.304640in}}%
\pgfpathlineto{\pgfqpoint{2.389042in}{1.316393in}}%
\pgfpathlineto{\pgfqpoint{2.405967in}{1.338931in}}%
\pgfpathlineto{\pgfqpoint{2.422891in}{1.357240in}}%
\pgfpathlineto{\pgfqpoint{2.456741in}{1.380072in}}%
\pgfpathlineto{\pgfqpoint{2.490590in}{1.419052in}}%
\pgfpathlineto{\pgfqpoint{2.507515in}{1.433562in}}%
\pgfpathlineto{\pgfqpoint{2.558289in}{1.486894in}}%
\pgfpathlineto{\pgfqpoint{2.575214in}{1.497036in}}%
\pgfpathlineto{\pgfqpoint{2.592139in}{1.510127in}}%
\pgfpathlineto{\pgfqpoint{2.592139in}{1.510127in}}%
\pgfusepath{stroke}%
\end{pgfscope}%
\begin{pgfscope}%
\pgfpathrectangle{\pgfqpoint{0.335222in}{0.315889in}}{\pgfqpoint{2.364389in}{2.385311in}}%
\pgfusepath{clip}%
\pgfsetrectcap%
\pgfsetroundjoin%
\pgfsetlinewidth{1.505625pt}%
\definecolor{currentstroke}{rgb}{0.839216,0.152941,0.156863}%
\pgfsetstrokecolor{currentstroke}%
\pgfsetdash{}{0pt}%
\pgfpathmoveto{\pgfqpoint{0.442694in}{0.427710in}}%
\pgfpathlineto{\pgfqpoint{0.459619in}{0.428302in}}%
\pgfpathlineto{\pgfqpoint{0.476544in}{0.424312in}}%
\pgfpathlineto{\pgfqpoint{0.493469in}{0.426455in}}%
\pgfpathlineto{\pgfqpoint{0.510393in}{0.431584in}}%
\pgfpathlineto{\pgfqpoint{0.527318in}{0.429351in}}%
\pgfpathlineto{\pgfqpoint{0.544243in}{0.425390in}}%
\pgfpathlineto{\pgfqpoint{0.561168in}{0.429691in}}%
\pgfpathlineto{\pgfqpoint{0.595017in}{0.429159in}}%
\pgfpathlineto{\pgfqpoint{0.611942in}{0.431792in}}%
\pgfpathlineto{\pgfqpoint{0.628867in}{0.426166in}}%
\pgfpathlineto{\pgfqpoint{0.645792in}{0.433257in}}%
\pgfpathlineto{\pgfqpoint{0.662716in}{0.429806in}}%
\pgfpathlineto{\pgfqpoint{0.679641in}{0.430759in}}%
\pgfpathlineto{\pgfqpoint{0.696566in}{0.435144in}}%
\pgfpathlineto{\pgfqpoint{0.730415in}{0.430714in}}%
\pgfpathlineto{\pgfqpoint{0.747340in}{0.432139in}}%
\pgfpathlineto{\pgfqpoint{0.764265in}{0.440062in}}%
\pgfpathlineto{\pgfqpoint{0.781190in}{0.432427in}}%
\pgfpathlineto{\pgfqpoint{0.798114in}{0.438584in}}%
\pgfpathlineto{\pgfqpoint{0.815039in}{0.442214in}}%
\pgfpathlineto{\pgfqpoint{0.848889in}{0.442577in}}%
\pgfpathlineto{\pgfqpoint{0.865813in}{0.446698in}}%
\pgfpathlineto{\pgfqpoint{0.882738in}{0.446674in}}%
\pgfpathlineto{\pgfqpoint{0.899663in}{0.456091in}}%
\pgfpathlineto{\pgfqpoint{0.916588in}{0.454733in}}%
\pgfpathlineto{\pgfqpoint{0.933512in}{0.450702in}}%
\pgfpathlineto{\pgfqpoint{0.950437in}{0.458096in}}%
\pgfpathlineto{\pgfqpoint{1.018136in}{0.473008in}}%
\pgfpathlineto{\pgfqpoint{1.035061in}{0.480390in}}%
\pgfpathlineto{\pgfqpoint{1.051986in}{0.474021in}}%
\pgfpathlineto{\pgfqpoint{1.085835in}{0.486927in}}%
\pgfpathlineto{\pgfqpoint{1.102760in}{0.489422in}}%
\pgfpathlineto{\pgfqpoint{1.119685in}{0.495874in}}%
\pgfpathlineto{\pgfqpoint{1.136610in}{0.505795in}}%
\pgfpathlineto{\pgfqpoint{1.153534in}{0.508651in}}%
\pgfpathlineto{\pgfqpoint{1.170459in}{0.513735in}}%
\pgfpathlineto{\pgfqpoint{1.187384in}{0.524989in}}%
\pgfpathlineto{\pgfqpoint{1.204309in}{0.522429in}}%
\pgfpathlineto{\pgfqpoint{1.221233in}{0.528420in}}%
\pgfpathlineto{\pgfqpoint{1.238158in}{0.541913in}}%
\pgfpathlineto{\pgfqpoint{1.272008in}{0.554309in}}%
\pgfpathlineto{\pgfqpoint{1.288932in}{0.568177in}}%
\pgfpathlineto{\pgfqpoint{1.305857in}{0.570629in}}%
\pgfpathlineto{\pgfqpoint{1.322782in}{0.580827in}}%
\pgfpathlineto{\pgfqpoint{1.339707in}{0.589151in}}%
\pgfpathlineto{\pgfqpoint{1.356631in}{0.592711in}}%
\pgfpathlineto{\pgfqpoint{1.373556in}{0.610041in}}%
\pgfpathlineto{\pgfqpoint{1.390481in}{0.622346in}}%
\pgfpathlineto{\pgfqpoint{1.407406in}{0.625848in}}%
\pgfpathlineto{\pgfqpoint{1.424330in}{0.637381in}}%
\pgfpathlineto{\pgfqpoint{1.441255in}{0.643180in}}%
\pgfpathlineto{\pgfqpoint{1.458180in}{0.656247in}}%
\pgfpathlineto{\pgfqpoint{1.475105in}{0.667019in}}%
\pgfpathlineto{\pgfqpoint{1.492030in}{0.684337in}}%
\pgfpathlineto{\pgfqpoint{1.525879in}{0.710216in}}%
\pgfpathlineto{\pgfqpoint{1.542804in}{0.721288in}}%
\pgfpathlineto{\pgfqpoint{1.559729in}{0.729858in}}%
\pgfpathlineto{\pgfqpoint{1.576653in}{0.748173in}}%
\pgfpathlineto{\pgfqpoint{1.593578in}{0.761937in}}%
\pgfpathlineto{\pgfqpoint{1.627428in}{0.794486in}}%
\pgfpathlineto{\pgfqpoint{1.644352in}{0.804793in}}%
\pgfpathlineto{\pgfqpoint{1.661277in}{0.822099in}}%
\pgfpathlineto{\pgfqpoint{1.678202in}{0.836834in}}%
\pgfpathlineto{\pgfqpoint{1.712051in}{0.877886in}}%
\pgfpathlineto{\pgfqpoint{1.728976in}{0.894720in}}%
\pgfpathlineto{\pgfqpoint{1.745901in}{0.915750in}}%
\pgfpathlineto{\pgfqpoint{1.762826in}{0.928527in}}%
\pgfpathlineto{\pgfqpoint{1.796675in}{0.975668in}}%
\pgfpathlineto{\pgfqpoint{1.813600in}{0.992842in}}%
\pgfpathlineto{\pgfqpoint{1.830525in}{1.018036in}}%
\pgfpathlineto{\pgfqpoint{1.847449in}{1.032740in}}%
\pgfpathlineto{\pgfqpoint{1.898224in}{1.104847in}}%
\pgfpathlineto{\pgfqpoint{1.915149in}{1.128952in}}%
\pgfpathlineto{\pgfqpoint{1.932073in}{1.149345in}}%
\pgfpathlineto{\pgfqpoint{1.948998in}{1.178387in}}%
\pgfpathlineto{\pgfqpoint{1.982848in}{1.228042in}}%
\pgfpathlineto{\pgfqpoint{1.999772in}{1.259162in}}%
\pgfpathlineto{\pgfqpoint{2.033622in}{1.307513in}}%
\pgfpathlineto{\pgfqpoint{2.050547in}{1.335692in}}%
\pgfpathlineto{\pgfqpoint{2.067471in}{1.367477in}}%
\pgfpathlineto{\pgfqpoint{2.084396in}{1.395584in}}%
\pgfpathlineto{\pgfqpoint{2.101321in}{1.428782in}}%
\pgfpathlineto{\pgfqpoint{2.118246in}{1.458242in}}%
\pgfpathlineto{\pgfqpoint{2.135170in}{1.491053in}}%
\pgfpathlineto{\pgfqpoint{2.152095in}{1.520295in}}%
\pgfpathlineto{\pgfqpoint{2.169020in}{1.556797in}}%
\pgfpathlineto{\pgfqpoint{2.185945in}{1.583117in}}%
\pgfpathlineto{\pgfqpoint{2.202869in}{1.617029in}}%
\pgfpathlineto{\pgfqpoint{2.219794in}{1.655957in}}%
\pgfpathlineto{\pgfqpoint{2.236719in}{1.687160in}}%
\pgfpathlineto{\pgfqpoint{2.270568in}{1.764398in}}%
\pgfpathlineto{\pgfqpoint{2.287493in}{1.798364in}}%
\pgfpathlineto{\pgfqpoint{2.355192in}{1.952526in}}%
\pgfpathlineto{\pgfqpoint{2.372117in}{1.999107in}}%
\pgfpathlineto{\pgfqpoint{2.405967in}{2.080749in}}%
\pgfpathlineto{\pgfqpoint{2.456741in}{2.217365in}}%
\pgfpathlineto{\pgfqpoint{2.507515in}{2.347045in}}%
\pgfpathlineto{\pgfqpoint{2.524440in}{2.400434in}}%
\pgfpathlineto{\pgfqpoint{2.541365in}{2.444180in}}%
\pgfpathlineto{\pgfqpoint{2.558289in}{2.493577in}}%
\pgfpathlineto{\pgfqpoint{2.575214in}{2.551072in}}%
\pgfpathlineto{\pgfqpoint{2.592139in}{2.592777in}}%
\pgfpathlineto{\pgfqpoint{2.592139in}{2.592777in}}%
\pgfusepath{stroke}%
\end{pgfscope}%
\begin{pgfscope}%
\pgfsetrectcap%
\pgfsetmiterjoin%
\pgfsetlinewidth{0.803000pt}%
\definecolor{currentstroke}{rgb}{0.000000,0.000000,0.000000}%
\pgfsetstrokecolor{currentstroke}%
\pgfsetdash{}{0pt}%
\pgfpathmoveto{\pgfqpoint{0.335222in}{0.315889in}}%
\pgfpathlineto{\pgfqpoint{0.335222in}{2.701200in}}%
\pgfusepath{stroke}%
\end{pgfscope}%
\begin{pgfscope}%
\pgfsetrectcap%
\pgfsetmiterjoin%
\pgfsetlinewidth{0.803000pt}%
\definecolor{currentstroke}{rgb}{0.000000,0.000000,0.000000}%
\pgfsetstrokecolor{currentstroke}%
\pgfsetdash{}{0pt}%
\pgfpathmoveto{\pgfqpoint{2.699611in}{0.315889in}}%
\pgfpathlineto{\pgfqpoint{2.699611in}{2.701200in}}%
\pgfusepath{stroke}%
\end{pgfscope}%
\begin{pgfscope}%
\pgfsetrectcap%
\pgfsetmiterjoin%
\pgfsetlinewidth{0.803000pt}%
\definecolor{currentstroke}{rgb}{0.000000,0.000000,0.000000}%
\pgfsetstrokecolor{currentstroke}%
\pgfsetdash{}{0pt}%
\pgfpathmoveto{\pgfqpoint{0.335222in}{0.315889in}}%
\pgfpathlineto{\pgfqpoint{2.699611in}{0.315889in}}%
\pgfusepath{stroke}%
\end{pgfscope}%
\begin{pgfscope}%
\pgfsetrectcap%
\pgfsetmiterjoin%
\pgfsetlinewidth{0.803000pt}%
\definecolor{currentstroke}{rgb}{0.000000,0.000000,0.000000}%
\pgfsetstrokecolor{currentstroke}%
\pgfsetdash{}{0pt}%
\pgfpathmoveto{\pgfqpoint{0.335222in}{2.701200in}}%
\pgfpathlineto{\pgfqpoint{2.699611in}{2.701200in}}%
\pgfusepath{stroke}%
\end{pgfscope}%
\begin{pgfscope}%
\definecolor{textcolor}{rgb}{0.000000,0.000000,0.000000}%
\pgfsetstrokecolor{textcolor}%
\pgfsetfillcolor{textcolor}%
\pgftext[x=1.517417in,y=2.784533in,,base]{\color{textcolor}\rmfamily\fontsize{9.600000}{11.520000}\selectfont Approximationskoeffizienten}%
\end{pgfscope}%
\begin{pgfscope}%
\pgfsetbuttcap%
\pgfsetmiterjoin%
\definecolor{currentfill}{rgb}{1.000000,1.000000,1.000000}%
\pgfsetfillcolor{currentfill}%
\pgfsetfillopacity{0.800000}%
\pgfsetlinewidth{1.003750pt}%
\definecolor{currentstroke}{rgb}{0.800000,0.800000,0.800000}%
\pgfsetstrokecolor{currentstroke}%
\pgfsetstrokeopacity{0.800000}%
\pgfsetdash{}{0pt}%
\pgfpathmoveto{\pgfqpoint{0.413000in}{1.944347in}}%
\pgfpathlineto{\pgfqpoint{1.093725in}{1.944347in}}%
\pgfpathquadraticcurveto{\pgfqpoint{1.115948in}{1.944347in}}{\pgfqpoint{1.115948in}{1.966569in}}%
\pgfpathlineto{\pgfqpoint{1.115948in}{2.623422in}}%
\pgfpathquadraticcurveto{\pgfqpoint{1.115948in}{2.645644in}}{\pgfqpoint{1.093725in}{2.645644in}}%
\pgfpathlineto{\pgfqpoint{0.413000in}{2.645644in}}%
\pgfpathquadraticcurveto{\pgfqpoint{0.390778in}{2.645644in}}{\pgfqpoint{0.390778in}{2.623422in}}%
\pgfpathlineto{\pgfqpoint{0.390778in}{1.966569in}}%
\pgfpathquadraticcurveto{\pgfqpoint{0.390778in}{1.944347in}}{\pgfqpoint{0.413000in}{1.944347in}}%
\pgfpathclose%
\pgfusepath{stroke,fill}%
\end{pgfscope}%
\begin{pgfscope}%
\pgfsetrectcap%
\pgfsetroundjoin%
\pgfsetlinewidth{1.505625pt}%
\definecolor{currentstroke}{rgb}{0.121569,0.466667,0.705882}%
\pgfsetstrokecolor{currentstroke}%
\pgfsetdash{}{0pt}%
\pgfpathmoveto{\pgfqpoint{0.435222in}{2.550209in}}%
\pgfpathlineto{\pgfqpoint{0.657444in}{2.550209in}}%
\pgfusepath{stroke}%
\end{pgfscope}%
\begin{pgfscope}%
\definecolor{textcolor}{rgb}{0.000000,0.000000,0.000000}%
\pgfsetstrokecolor{textcolor}%
\pgfsetfillcolor{textcolor}%
\pgftext[x=0.746333in,y=2.511320in,left,base]{\color{textcolor}\rmfamily\fontsize{8.000000}{9.600000}\selectfont \(\displaystyle x^{0} + r\)}%
\end{pgfscope}%
\begin{pgfscope}%
\pgfsetrectcap%
\pgfsetroundjoin%
\pgfsetlinewidth{1.505625pt}%
\definecolor{currentstroke}{rgb}{1.000000,0.498039,0.054902}%
\pgfsetstrokecolor{currentstroke}%
\pgfsetdash{}{0pt}%
\pgfpathmoveto{\pgfqpoint{0.435222in}{2.383217in}}%
\pgfpathlineto{\pgfqpoint{0.657444in}{2.383217in}}%
\pgfusepath{stroke}%
\end{pgfscope}%
\begin{pgfscope}%
\definecolor{textcolor}{rgb}{0.000000,0.000000,0.000000}%
\pgfsetstrokecolor{textcolor}%
\pgfsetfillcolor{textcolor}%
\pgftext[x=0.746333in,y=2.344329in,left,base]{\color{textcolor}\rmfamily\fontsize{8.000000}{9.600000}\selectfont \(\displaystyle x^{1} + r\)}%
\end{pgfscope}%
\begin{pgfscope}%
\pgfsetrectcap%
\pgfsetroundjoin%
\pgfsetlinewidth{1.505625pt}%
\definecolor{currentstroke}{rgb}{0.172549,0.627451,0.172549}%
\pgfsetstrokecolor{currentstroke}%
\pgfsetdash{}{0pt}%
\pgfpathmoveto{\pgfqpoint{0.435222in}{2.216226in}}%
\pgfpathlineto{\pgfqpoint{0.657444in}{2.216226in}}%
\pgfusepath{stroke}%
\end{pgfscope}%
\begin{pgfscope}%
\definecolor{textcolor}{rgb}{0.000000,0.000000,0.000000}%
\pgfsetstrokecolor{textcolor}%
\pgfsetfillcolor{textcolor}%
\pgftext[x=0.746333in,y=2.177337in,left,base]{\color{textcolor}\rmfamily\fontsize{8.000000}{9.600000}\selectfont \(\displaystyle x^{2} + r\)}%
\end{pgfscope}%
\begin{pgfscope}%
\pgfsetrectcap%
\pgfsetroundjoin%
\pgfsetlinewidth{1.505625pt}%
\definecolor{currentstroke}{rgb}{0.839216,0.152941,0.156863}%
\pgfsetstrokecolor{currentstroke}%
\pgfsetdash{}{0pt}%
\pgfpathmoveto{\pgfqpoint{0.435222in}{2.049235in}}%
\pgfpathlineto{\pgfqpoint{0.657444in}{2.049235in}}%
\pgfusepath{stroke}%
\end{pgfscope}%
\begin{pgfscope}%
\definecolor{textcolor}{rgb}{0.000000,0.000000,0.000000}%
\pgfsetstrokecolor{textcolor}%
\pgfsetfillcolor{textcolor}%
\pgftext[x=0.746333in,y=2.010346in,left,base]{\color{textcolor}\rmfamily\fontsize{8.000000}{9.600000}\selectfont \(\displaystyle x^{3} + r\)}%
\end{pgfscope}%
\begin{pgfscope}%
\pgfsetbuttcap%
\pgfsetmiterjoin%
\definecolor{currentfill}{rgb}{1.000000,1.000000,1.000000}%
\pgfsetfillcolor{currentfill}%
\pgfsetlinewidth{0.000000pt}%
\definecolor{currentstroke}{rgb}{0.000000,0.000000,0.000000}%
\pgfsetstrokecolor{currentstroke}%
\pgfsetstrokeopacity{0.000000}%
\pgfsetdash{}{0pt}%
\pgfpathmoveto{\pgfqpoint{2.916833in}{0.315889in}}%
\pgfpathlineto{\pgfqpoint{5.281222in}{0.315889in}}%
\pgfpathlineto{\pgfqpoint{5.281222in}{2.701200in}}%
\pgfpathlineto{\pgfqpoint{2.916833in}{2.701200in}}%
\pgfpathclose%
\pgfusepath{fill}%
\end{pgfscope}%
\begin{pgfscope}%
\pgfsetbuttcap%
\pgfsetroundjoin%
\definecolor{currentfill}{rgb}{0.000000,0.000000,0.000000}%
\pgfsetfillcolor{currentfill}%
\pgfsetlinewidth{0.803000pt}%
\definecolor{currentstroke}{rgb}{0.000000,0.000000,0.000000}%
\pgfsetstrokecolor{currentstroke}%
\pgfsetdash{}{0pt}%
\pgfsys@defobject{currentmarker}{\pgfqpoint{0.000000in}{-0.048611in}}{\pgfqpoint{0.000000in}{0.000000in}}{%
\pgfpathmoveto{\pgfqpoint{0.000000in}{0.000000in}}%
\pgfpathlineto{\pgfqpoint{0.000000in}{-0.048611in}}%
\pgfusepath{stroke,fill}%
}%
\begin{pgfscope}%
\pgfsys@transformshift{3.024306in}{0.315889in}%
\pgfsys@useobject{currentmarker}{}%
\end{pgfscope}%
\end{pgfscope}%
\begin{pgfscope}%
\definecolor{textcolor}{rgb}{0.000000,0.000000,0.000000}%
\pgfsetstrokecolor{textcolor}%
\pgfsetfillcolor{textcolor}%
\pgftext[x=3.024306in,y=0.218667in,,top]{\color{textcolor}\rmfamily\fontsize{8.000000}{9.600000}\selectfont 0}%
\end{pgfscope}%
\begin{pgfscope}%
\pgfsetbuttcap%
\pgfsetroundjoin%
\definecolor{currentfill}{rgb}{0.000000,0.000000,0.000000}%
\pgfsetfillcolor{currentfill}%
\pgfsetlinewidth{0.803000pt}%
\definecolor{currentstroke}{rgb}{0.000000,0.000000,0.000000}%
\pgfsetstrokecolor{currentstroke}%
\pgfsetdash{}{0pt}%
\pgfsys@defobject{currentmarker}{\pgfqpoint{0.000000in}{-0.048611in}}{\pgfqpoint{0.000000in}{0.000000in}}{%
\pgfpathmoveto{\pgfqpoint{0.000000in}{0.000000in}}%
\pgfpathlineto{\pgfqpoint{0.000000in}{-0.048611in}}%
\pgfusepath{stroke,fill}%
}%
\begin{pgfscope}%
\pgfsys@transformshift{3.362801in}{0.315889in}%
\pgfsys@useobject{currentmarker}{}%
\end{pgfscope}%
\end{pgfscope}%
\begin{pgfscope}%
\definecolor{textcolor}{rgb}{0.000000,0.000000,0.000000}%
\pgfsetstrokecolor{textcolor}%
\pgfsetfillcolor{textcolor}%
\pgftext[x=3.362801in,y=0.218667in,,top]{\color{textcolor}\rmfamily\fontsize{8.000000}{9.600000}\selectfont 20}%
\end{pgfscope}%
\begin{pgfscope}%
\pgfsetbuttcap%
\pgfsetroundjoin%
\definecolor{currentfill}{rgb}{0.000000,0.000000,0.000000}%
\pgfsetfillcolor{currentfill}%
\pgfsetlinewidth{0.803000pt}%
\definecolor{currentstroke}{rgb}{0.000000,0.000000,0.000000}%
\pgfsetstrokecolor{currentstroke}%
\pgfsetdash{}{0pt}%
\pgfsys@defobject{currentmarker}{\pgfqpoint{0.000000in}{-0.048611in}}{\pgfqpoint{0.000000in}{0.000000in}}{%
\pgfpathmoveto{\pgfqpoint{0.000000in}{0.000000in}}%
\pgfpathlineto{\pgfqpoint{0.000000in}{-0.048611in}}%
\pgfusepath{stroke,fill}%
}%
\begin{pgfscope}%
\pgfsys@transformshift{3.701296in}{0.315889in}%
\pgfsys@useobject{currentmarker}{}%
\end{pgfscope}%
\end{pgfscope}%
\begin{pgfscope}%
\definecolor{textcolor}{rgb}{0.000000,0.000000,0.000000}%
\pgfsetstrokecolor{textcolor}%
\pgfsetfillcolor{textcolor}%
\pgftext[x=3.701296in,y=0.218667in,,top]{\color{textcolor}\rmfamily\fontsize{8.000000}{9.600000}\selectfont 40}%
\end{pgfscope}%
\begin{pgfscope}%
\pgfsetbuttcap%
\pgfsetroundjoin%
\definecolor{currentfill}{rgb}{0.000000,0.000000,0.000000}%
\pgfsetfillcolor{currentfill}%
\pgfsetlinewidth{0.803000pt}%
\definecolor{currentstroke}{rgb}{0.000000,0.000000,0.000000}%
\pgfsetstrokecolor{currentstroke}%
\pgfsetdash{}{0pt}%
\pgfsys@defobject{currentmarker}{\pgfqpoint{0.000000in}{-0.048611in}}{\pgfqpoint{0.000000in}{0.000000in}}{%
\pgfpathmoveto{\pgfqpoint{0.000000in}{0.000000in}}%
\pgfpathlineto{\pgfqpoint{0.000000in}{-0.048611in}}%
\pgfusepath{stroke,fill}%
}%
\begin{pgfscope}%
\pgfsys@transformshift{4.039791in}{0.315889in}%
\pgfsys@useobject{currentmarker}{}%
\end{pgfscope}%
\end{pgfscope}%
\begin{pgfscope}%
\definecolor{textcolor}{rgb}{0.000000,0.000000,0.000000}%
\pgfsetstrokecolor{textcolor}%
\pgfsetfillcolor{textcolor}%
\pgftext[x=4.039791in,y=0.218667in,,top]{\color{textcolor}\rmfamily\fontsize{8.000000}{9.600000}\selectfont 60}%
\end{pgfscope}%
\begin{pgfscope}%
\pgfsetbuttcap%
\pgfsetroundjoin%
\definecolor{currentfill}{rgb}{0.000000,0.000000,0.000000}%
\pgfsetfillcolor{currentfill}%
\pgfsetlinewidth{0.803000pt}%
\definecolor{currentstroke}{rgb}{0.000000,0.000000,0.000000}%
\pgfsetstrokecolor{currentstroke}%
\pgfsetdash{}{0pt}%
\pgfsys@defobject{currentmarker}{\pgfqpoint{0.000000in}{-0.048611in}}{\pgfqpoint{0.000000in}{0.000000in}}{%
\pgfpathmoveto{\pgfqpoint{0.000000in}{0.000000in}}%
\pgfpathlineto{\pgfqpoint{0.000000in}{-0.048611in}}%
\pgfusepath{stroke,fill}%
}%
\begin{pgfscope}%
\pgfsys@transformshift{4.378286in}{0.315889in}%
\pgfsys@useobject{currentmarker}{}%
\end{pgfscope}%
\end{pgfscope}%
\begin{pgfscope}%
\definecolor{textcolor}{rgb}{0.000000,0.000000,0.000000}%
\pgfsetstrokecolor{textcolor}%
\pgfsetfillcolor{textcolor}%
\pgftext[x=4.378286in,y=0.218667in,,top]{\color{textcolor}\rmfamily\fontsize{8.000000}{9.600000}\selectfont 80}%
\end{pgfscope}%
\begin{pgfscope}%
\pgfsetbuttcap%
\pgfsetroundjoin%
\definecolor{currentfill}{rgb}{0.000000,0.000000,0.000000}%
\pgfsetfillcolor{currentfill}%
\pgfsetlinewidth{0.803000pt}%
\definecolor{currentstroke}{rgb}{0.000000,0.000000,0.000000}%
\pgfsetstrokecolor{currentstroke}%
\pgfsetdash{}{0pt}%
\pgfsys@defobject{currentmarker}{\pgfqpoint{0.000000in}{-0.048611in}}{\pgfqpoint{0.000000in}{0.000000in}}{%
\pgfpathmoveto{\pgfqpoint{0.000000in}{0.000000in}}%
\pgfpathlineto{\pgfqpoint{0.000000in}{-0.048611in}}%
\pgfusepath{stroke,fill}%
}%
\begin{pgfscope}%
\pgfsys@transformshift{4.716781in}{0.315889in}%
\pgfsys@useobject{currentmarker}{}%
\end{pgfscope}%
\end{pgfscope}%
\begin{pgfscope}%
\definecolor{textcolor}{rgb}{0.000000,0.000000,0.000000}%
\pgfsetstrokecolor{textcolor}%
\pgfsetfillcolor{textcolor}%
\pgftext[x=4.716781in,y=0.218667in,,top]{\color{textcolor}\rmfamily\fontsize{8.000000}{9.600000}\selectfont 100}%
\end{pgfscope}%
\begin{pgfscope}%
\pgfsetbuttcap%
\pgfsetroundjoin%
\definecolor{currentfill}{rgb}{0.000000,0.000000,0.000000}%
\pgfsetfillcolor{currentfill}%
\pgfsetlinewidth{0.803000pt}%
\definecolor{currentstroke}{rgb}{0.000000,0.000000,0.000000}%
\pgfsetstrokecolor{currentstroke}%
\pgfsetdash{}{0pt}%
\pgfsys@defobject{currentmarker}{\pgfqpoint{0.000000in}{-0.048611in}}{\pgfqpoint{0.000000in}{0.000000in}}{%
\pgfpathmoveto{\pgfqpoint{0.000000in}{0.000000in}}%
\pgfpathlineto{\pgfqpoint{0.000000in}{-0.048611in}}%
\pgfusepath{stroke,fill}%
}%
\begin{pgfscope}%
\pgfsys@transformshift{5.055277in}{0.315889in}%
\pgfsys@useobject{currentmarker}{}%
\end{pgfscope}%
\end{pgfscope}%
\begin{pgfscope}%
\definecolor{textcolor}{rgb}{0.000000,0.000000,0.000000}%
\pgfsetstrokecolor{textcolor}%
\pgfsetfillcolor{textcolor}%
\pgftext[x=5.055277in,y=0.218667in,,top]{\color{textcolor}\rmfamily\fontsize{8.000000}{9.600000}\selectfont 120}%
\end{pgfscope}%
\begin{pgfscope}%
\pgfsetbuttcap%
\pgfsetroundjoin%
\definecolor{currentfill}{rgb}{0.000000,0.000000,0.000000}%
\pgfsetfillcolor{currentfill}%
\pgfsetlinewidth{0.803000pt}%
\definecolor{currentstroke}{rgb}{0.000000,0.000000,0.000000}%
\pgfsetstrokecolor{currentstroke}%
\pgfsetdash{}{0pt}%
\pgfsys@defobject{currentmarker}{\pgfqpoint{0.000000in}{0.000000in}}{\pgfqpoint{0.048611in}{0.000000in}}{%
\pgfpathmoveto{\pgfqpoint{0.000000in}{0.000000in}}%
\pgfpathlineto{\pgfqpoint{0.048611in}{0.000000in}}%
\pgfusepath{stroke,fill}%
}%
\begin{pgfscope}%
\pgfsys@transformshift{5.281222in}{0.627875in}%
\pgfsys@useobject{currentmarker}{}%
\end{pgfscope}%
\end{pgfscope}%
\begin{pgfscope}%
\definecolor{textcolor}{rgb}{0.000000,0.000000,0.000000}%
\pgfsetstrokecolor{textcolor}%
\pgfsetfillcolor{textcolor}%
\pgftext[x=5.378444in,y=0.589320in,left,base]{\color{textcolor}\rmfamily\fontsize{8.000000}{9.600000}\selectfont −0.08}%
\end{pgfscope}%
\begin{pgfscope}%
\pgfsetbuttcap%
\pgfsetroundjoin%
\definecolor{currentfill}{rgb}{0.000000,0.000000,0.000000}%
\pgfsetfillcolor{currentfill}%
\pgfsetlinewidth{0.803000pt}%
\definecolor{currentstroke}{rgb}{0.000000,0.000000,0.000000}%
\pgfsetstrokecolor{currentstroke}%
\pgfsetdash{}{0pt}%
\pgfsys@defobject{currentmarker}{\pgfqpoint{0.000000in}{0.000000in}}{\pgfqpoint{0.048611in}{0.000000in}}{%
\pgfpathmoveto{\pgfqpoint{0.000000in}{0.000000in}}%
\pgfpathlineto{\pgfqpoint{0.048611in}{0.000000in}}%
\pgfusepath{stroke,fill}%
}%
\begin{pgfscope}%
\pgfsys@transformshift{5.281222in}{0.987773in}%
\pgfsys@useobject{currentmarker}{}%
\end{pgfscope}%
\end{pgfscope}%
\begin{pgfscope}%
\definecolor{textcolor}{rgb}{0.000000,0.000000,0.000000}%
\pgfsetstrokecolor{textcolor}%
\pgfsetfillcolor{textcolor}%
\pgftext[x=5.378444in,y=0.949217in,left,base]{\color{textcolor}\rmfamily\fontsize{8.000000}{9.600000}\selectfont −0.06}%
\end{pgfscope}%
\begin{pgfscope}%
\pgfsetbuttcap%
\pgfsetroundjoin%
\definecolor{currentfill}{rgb}{0.000000,0.000000,0.000000}%
\pgfsetfillcolor{currentfill}%
\pgfsetlinewidth{0.803000pt}%
\definecolor{currentstroke}{rgb}{0.000000,0.000000,0.000000}%
\pgfsetstrokecolor{currentstroke}%
\pgfsetdash{}{0pt}%
\pgfsys@defobject{currentmarker}{\pgfqpoint{0.000000in}{0.000000in}}{\pgfqpoint{0.048611in}{0.000000in}}{%
\pgfpathmoveto{\pgfqpoint{0.000000in}{0.000000in}}%
\pgfpathlineto{\pgfqpoint{0.048611in}{0.000000in}}%
\pgfusepath{stroke,fill}%
}%
\begin{pgfscope}%
\pgfsys@transformshift{5.281222in}{1.347670in}%
\pgfsys@useobject{currentmarker}{}%
\end{pgfscope}%
\end{pgfscope}%
\begin{pgfscope}%
\definecolor{textcolor}{rgb}{0.000000,0.000000,0.000000}%
\pgfsetstrokecolor{textcolor}%
\pgfsetfillcolor{textcolor}%
\pgftext[x=5.378444in,y=1.309114in,left,base]{\color{textcolor}\rmfamily\fontsize{8.000000}{9.600000}\selectfont −0.04}%
\end{pgfscope}%
\begin{pgfscope}%
\pgfsetbuttcap%
\pgfsetroundjoin%
\definecolor{currentfill}{rgb}{0.000000,0.000000,0.000000}%
\pgfsetfillcolor{currentfill}%
\pgfsetlinewidth{0.803000pt}%
\definecolor{currentstroke}{rgb}{0.000000,0.000000,0.000000}%
\pgfsetstrokecolor{currentstroke}%
\pgfsetdash{}{0pt}%
\pgfsys@defobject{currentmarker}{\pgfqpoint{0.000000in}{0.000000in}}{\pgfqpoint{0.048611in}{0.000000in}}{%
\pgfpathmoveto{\pgfqpoint{0.000000in}{0.000000in}}%
\pgfpathlineto{\pgfqpoint{0.048611in}{0.000000in}}%
\pgfusepath{stroke,fill}%
}%
\begin{pgfscope}%
\pgfsys@transformshift{5.281222in}{1.707567in}%
\pgfsys@useobject{currentmarker}{}%
\end{pgfscope}%
\end{pgfscope}%
\begin{pgfscope}%
\definecolor{textcolor}{rgb}{0.000000,0.000000,0.000000}%
\pgfsetstrokecolor{textcolor}%
\pgfsetfillcolor{textcolor}%
\pgftext[x=5.378444in,y=1.669012in,left,base]{\color{textcolor}\rmfamily\fontsize{8.000000}{9.600000}\selectfont −0.02}%
\end{pgfscope}%
\begin{pgfscope}%
\pgfsetbuttcap%
\pgfsetroundjoin%
\definecolor{currentfill}{rgb}{0.000000,0.000000,0.000000}%
\pgfsetfillcolor{currentfill}%
\pgfsetlinewidth{0.803000pt}%
\definecolor{currentstroke}{rgb}{0.000000,0.000000,0.000000}%
\pgfsetstrokecolor{currentstroke}%
\pgfsetdash{}{0pt}%
\pgfsys@defobject{currentmarker}{\pgfqpoint{0.000000in}{0.000000in}}{\pgfqpoint{0.048611in}{0.000000in}}{%
\pgfpathmoveto{\pgfqpoint{0.000000in}{0.000000in}}%
\pgfpathlineto{\pgfqpoint{0.048611in}{0.000000in}}%
\pgfusepath{stroke,fill}%
}%
\begin{pgfscope}%
\pgfsys@transformshift{5.281222in}{2.067465in}%
\pgfsys@useobject{currentmarker}{}%
\end{pgfscope}%
\end{pgfscope}%
\begin{pgfscope}%
\definecolor{textcolor}{rgb}{0.000000,0.000000,0.000000}%
\pgfsetstrokecolor{textcolor}%
\pgfsetfillcolor{textcolor}%
\pgftext[x=5.378444in,y=2.028909in,left,base]{\color{textcolor}\rmfamily\fontsize{8.000000}{9.600000}\selectfont 0.00}%
\end{pgfscope}%
\begin{pgfscope}%
\pgfsetbuttcap%
\pgfsetroundjoin%
\definecolor{currentfill}{rgb}{0.000000,0.000000,0.000000}%
\pgfsetfillcolor{currentfill}%
\pgfsetlinewidth{0.803000pt}%
\definecolor{currentstroke}{rgb}{0.000000,0.000000,0.000000}%
\pgfsetstrokecolor{currentstroke}%
\pgfsetdash{}{0pt}%
\pgfsys@defobject{currentmarker}{\pgfqpoint{0.000000in}{0.000000in}}{\pgfqpoint{0.048611in}{0.000000in}}{%
\pgfpathmoveto{\pgfqpoint{0.000000in}{0.000000in}}%
\pgfpathlineto{\pgfqpoint{0.048611in}{0.000000in}}%
\pgfusepath{stroke,fill}%
}%
\begin{pgfscope}%
\pgfsys@transformshift{5.281222in}{2.427362in}%
\pgfsys@useobject{currentmarker}{}%
\end{pgfscope}%
\end{pgfscope}%
\begin{pgfscope}%
\definecolor{textcolor}{rgb}{0.000000,0.000000,0.000000}%
\pgfsetstrokecolor{textcolor}%
\pgfsetfillcolor{textcolor}%
\pgftext[x=5.378444in,y=2.388806in,left,base]{\color{textcolor}\rmfamily\fontsize{8.000000}{9.600000}\selectfont 0.02}%
\end{pgfscope}%
\begin{pgfscope}%
\pgfpathrectangle{\pgfqpoint{2.916833in}{0.315889in}}{\pgfqpoint{2.364389in}{2.385311in}}%
\pgfusepath{clip}%
\pgfsetrectcap%
\pgfsetroundjoin%
\pgfsetlinewidth{1.505625pt}%
\definecolor{currentstroke}{rgb}{0.121569,0.466667,0.705882}%
\pgfsetstrokecolor{currentstroke}%
\pgfsetdash{}{0pt}%
\pgfpathmoveto{\pgfqpoint{3.024306in}{2.291338in}}%
\pgfpathlineto{\pgfqpoint{3.041230in}{1.904907in}}%
\pgfpathlineto{\pgfqpoint{3.058155in}{2.588732in}}%
\pgfpathlineto{\pgfqpoint{3.075080in}{2.058495in}}%
\pgfpathlineto{\pgfqpoint{3.092005in}{2.372613in}}%
\pgfpathlineto{\pgfqpoint{3.108929in}{1.903421in}}%
\pgfpathlineto{\pgfqpoint{3.125854in}{2.431617in}}%
\pgfpathlineto{\pgfqpoint{3.142779in}{2.341422in}}%
\pgfpathlineto{\pgfqpoint{3.159704in}{1.909945in}}%
\pgfpathlineto{\pgfqpoint{3.176628in}{2.213578in}}%
\pgfpathlineto{\pgfqpoint{3.193553in}{2.215852in}}%
\pgfpathlineto{\pgfqpoint{3.210478in}{2.133137in}}%
\pgfpathlineto{\pgfqpoint{3.227403in}{1.746116in}}%
\pgfpathlineto{\pgfqpoint{3.244327in}{1.922254in}}%
\pgfpathlineto{\pgfqpoint{3.261252in}{2.063522in}}%
\pgfpathlineto{\pgfqpoint{3.278177in}{2.448122in}}%
\pgfpathlineto{\pgfqpoint{3.295102in}{2.135392in}}%
\pgfpathlineto{\pgfqpoint{3.328951in}{1.828518in}}%
\pgfpathlineto{\pgfqpoint{3.345876in}{2.277617in}}%
\pgfpathlineto{\pgfqpoint{3.362801in}{2.385487in}}%
\pgfpathlineto{\pgfqpoint{3.379726in}{2.529958in}}%
\pgfpathlineto{\pgfqpoint{3.396650in}{1.892524in}}%
\pgfpathlineto{\pgfqpoint{3.430500in}{2.162140in}}%
\pgfpathlineto{\pgfqpoint{3.447425in}{2.201704in}}%
\pgfpathlineto{\pgfqpoint{3.464349in}{2.504711in}}%
\pgfpathlineto{\pgfqpoint{3.481274in}{2.104138in}}%
\pgfpathlineto{\pgfqpoint{3.498199in}{2.417020in}}%
\pgfpathlineto{\pgfqpoint{3.515124in}{1.962134in}}%
\pgfpathlineto{\pgfqpoint{3.532048in}{1.579484in}}%
\pgfpathlineto{\pgfqpoint{3.548973in}{2.303796in}}%
\pgfpathlineto{\pgfqpoint{3.565898in}{2.052708in}}%
\pgfpathlineto{\pgfqpoint{3.582823in}{1.695166in}}%
\pgfpathlineto{\pgfqpoint{3.599747in}{2.232382in}}%
\pgfpathlineto{\pgfqpoint{3.616672in}{2.107437in}}%
\pgfpathlineto{\pgfqpoint{3.633597in}{1.942685in}}%
\pgfpathlineto{\pgfqpoint{3.650522in}{2.163406in}}%
\pgfpathlineto{\pgfqpoint{3.667446in}{2.088690in}}%
\pgfpathlineto{\pgfqpoint{3.684371in}{2.118925in}}%
\pgfpathlineto{\pgfqpoint{3.701296in}{2.376022in}}%
\pgfpathlineto{\pgfqpoint{3.718221in}{2.536286in}}%
\pgfpathlineto{\pgfqpoint{3.752070in}{2.038142in}}%
\pgfpathlineto{\pgfqpoint{3.768995in}{1.876837in}}%
\pgfpathlineto{\pgfqpoint{3.785920in}{2.161890in}}%
\pgfpathlineto{\pgfqpoint{3.802844in}{1.589100in}}%
\pgfpathlineto{\pgfqpoint{3.819769in}{2.456397in}}%
\pgfpathlineto{\pgfqpoint{3.836694in}{2.219667in}}%
\pgfpathlineto{\pgfqpoint{3.853619in}{1.918216in}}%
\pgfpathlineto{\pgfqpoint{3.870544in}{2.139109in}}%
\pgfpathlineto{\pgfqpoint{3.887468in}{2.124111in}}%
\pgfpathlineto{\pgfqpoint{3.904393in}{2.059723in}}%
\pgfpathlineto{\pgfqpoint{3.921318in}{2.116808in}}%
\pgfpathlineto{\pgfqpoint{3.938243in}{1.727419in}}%
\pgfpathlineto{\pgfqpoint{3.955167in}{1.864941in}}%
\pgfpathlineto{\pgfqpoint{3.972092in}{1.913659in}}%
\pgfpathlineto{\pgfqpoint{3.989017in}{1.909076in}}%
\pgfpathlineto{\pgfqpoint{4.005942in}{2.135463in}}%
\pgfpathlineto{\pgfqpoint{4.022866in}{2.153814in}}%
\pgfpathlineto{\pgfqpoint{4.039791in}{1.838279in}}%
\pgfpathlineto{\pgfqpoint{4.056716in}{1.873216in}}%
\pgfpathlineto{\pgfqpoint{4.073641in}{1.957599in}}%
\pgfpathlineto{\pgfqpoint{4.090565in}{1.750879in}}%
\pgfpathlineto{\pgfqpoint{4.107490in}{2.015129in}}%
\pgfpathlineto{\pgfqpoint{4.124415in}{2.592777in}}%
\pgfpathlineto{\pgfqpoint{4.141340in}{1.951316in}}%
\pgfpathlineto{\pgfqpoint{4.192114in}{2.292292in}}%
\pgfpathlineto{\pgfqpoint{4.209039in}{2.254302in}}%
\pgfpathlineto{\pgfqpoint{4.225963in}{2.384876in}}%
\pgfpathlineto{\pgfqpoint{4.242888in}{2.473874in}}%
\pgfpathlineto{\pgfqpoint{4.259813in}{1.745032in}}%
\pgfpathlineto{\pgfqpoint{4.276738in}{2.244873in}}%
\pgfpathlineto{\pgfqpoint{4.293663in}{2.021958in}}%
\pgfpathlineto{\pgfqpoint{4.310587in}{2.205617in}}%
\pgfpathlineto{\pgfqpoint{4.327512in}{1.957335in}}%
\pgfpathlineto{\pgfqpoint{4.344437in}{1.926920in}}%
\pgfpathlineto{\pgfqpoint{4.361362in}{2.304395in}}%
\pgfpathlineto{\pgfqpoint{4.378286in}{2.238260in}}%
\pgfpathlineto{\pgfqpoint{4.395211in}{1.895678in}}%
\pgfpathlineto{\pgfqpoint{4.412136in}{2.294464in}}%
\pgfpathlineto{\pgfqpoint{4.429061in}{2.448193in}}%
\pgfpathlineto{\pgfqpoint{4.445985in}{2.450736in}}%
\pgfpathlineto{\pgfqpoint{4.462910in}{1.620827in}}%
\pgfpathlineto{\pgfqpoint{4.479835in}{2.160016in}}%
\pgfpathlineto{\pgfqpoint{4.496760in}{1.929697in}}%
\pgfpathlineto{\pgfqpoint{4.513684in}{1.954733in}}%
\pgfpathlineto{\pgfqpoint{4.530609in}{2.421751in}}%
\pgfpathlineto{\pgfqpoint{4.547534in}{2.081625in}}%
\pgfpathlineto{\pgfqpoint{4.564459in}{2.315117in}}%
\pgfpathlineto{\pgfqpoint{4.581383in}{2.349348in}}%
\pgfpathlineto{\pgfqpoint{4.598308in}{1.998417in}}%
\pgfpathlineto{\pgfqpoint{4.615233in}{1.695602in}}%
\pgfpathlineto{\pgfqpoint{4.632158in}{2.033194in}}%
\pgfpathlineto{\pgfqpoint{4.649082in}{1.857780in}}%
\pgfpathlineto{\pgfqpoint{4.666007in}{2.322695in}}%
\pgfpathlineto{\pgfqpoint{4.682932in}{1.682908in}}%
\pgfpathlineto{\pgfqpoint{4.699857in}{2.356311in}}%
\pgfpathlineto{\pgfqpoint{4.716781in}{2.219015in}}%
\pgfpathlineto{\pgfqpoint{4.733706in}{2.150780in}}%
\pgfpathlineto{\pgfqpoint{4.750631in}{1.938261in}}%
\pgfpathlineto{\pgfqpoint{4.767556in}{1.785672in}}%
\pgfpathlineto{\pgfqpoint{4.784481in}{1.990410in}}%
\pgfpathlineto{\pgfqpoint{4.801405in}{1.944480in}}%
\pgfpathlineto{\pgfqpoint{4.818330in}{1.984625in}}%
\pgfpathlineto{\pgfqpoint{4.835255in}{1.958079in}}%
\pgfpathlineto{\pgfqpoint{4.852180in}{1.578989in}}%
\pgfpathlineto{\pgfqpoint{4.869104in}{2.216241in}}%
\pgfpathlineto{\pgfqpoint{4.886029in}{2.218792in}}%
\pgfpathlineto{\pgfqpoint{4.902954in}{1.952772in}}%
\pgfpathlineto{\pgfqpoint{4.919879in}{2.036881in}}%
\pgfpathlineto{\pgfqpoint{4.936803in}{2.028753in}}%
\pgfpathlineto{\pgfqpoint{4.953728in}{1.701205in}}%
\pgfpathlineto{\pgfqpoint{4.970653in}{1.885561in}}%
\pgfpathlineto{\pgfqpoint{4.987578in}{2.508546in}}%
\pgfpathlineto{\pgfqpoint{5.004502in}{1.871945in}}%
\pgfpathlineto{\pgfqpoint{5.038352in}{2.399564in}}%
\pgfpathlineto{\pgfqpoint{5.055277in}{1.529706in}}%
\pgfpathlineto{\pgfqpoint{5.072201in}{2.261213in}}%
\pgfpathlineto{\pgfqpoint{5.089126in}{1.880665in}}%
\pgfpathlineto{\pgfqpoint{5.106051in}{2.565084in}}%
\pgfpathlineto{\pgfqpoint{5.122976in}{2.337964in}}%
\pgfpathlineto{\pgfqpoint{5.139900in}{2.307518in}}%
\pgfpathlineto{\pgfqpoint{5.156825in}{2.172316in}}%
\pgfpathlineto{\pgfqpoint{5.173750in}{2.263220in}}%
\pgfpathlineto{\pgfqpoint{5.173750in}{2.263220in}}%
\pgfusepath{stroke}%
\end{pgfscope}%
\begin{pgfscope}%
\pgfpathrectangle{\pgfqpoint{2.916833in}{0.315889in}}{\pgfqpoint{2.364389in}{2.385311in}}%
\pgfusepath{clip}%
\pgfsetrectcap%
\pgfsetroundjoin%
\pgfsetlinewidth{1.505625pt}%
\definecolor{currentstroke}{rgb}{1.000000,0.498039,0.054902}%
\pgfsetstrokecolor{currentstroke}%
\pgfsetdash{}{0pt}%
\pgfpathmoveto{\pgfqpoint{3.024306in}{1.468736in}}%
\pgfpathlineto{\pgfqpoint{3.041230in}{1.771269in}}%
\pgfpathlineto{\pgfqpoint{3.058155in}{2.203441in}}%
\pgfpathlineto{\pgfqpoint{3.075080in}{1.534976in}}%
\pgfpathlineto{\pgfqpoint{3.092005in}{2.426244in}}%
\pgfpathlineto{\pgfqpoint{3.108929in}{2.155065in}}%
\pgfpathlineto{\pgfqpoint{3.125854in}{2.138233in}}%
\pgfpathlineto{\pgfqpoint{3.142779in}{1.563914in}}%
\pgfpathlineto{\pgfqpoint{3.159704in}{2.034765in}}%
\pgfpathlineto{\pgfqpoint{3.176628in}{1.768359in}}%
\pgfpathlineto{\pgfqpoint{3.193553in}{1.542769in}}%
\pgfpathlineto{\pgfqpoint{3.210478in}{2.181019in}}%
\pgfpathlineto{\pgfqpoint{3.227403in}{1.795321in}}%
\pgfpathlineto{\pgfqpoint{3.244327in}{1.815165in}}%
\pgfpathlineto{\pgfqpoint{3.261252in}{1.926019in}}%
\pgfpathlineto{\pgfqpoint{3.278177in}{2.226991in}}%
\pgfpathlineto{\pgfqpoint{3.295102in}{1.822826in}}%
\pgfpathlineto{\pgfqpoint{3.312026in}{2.077883in}}%
\pgfpathlineto{\pgfqpoint{3.328951in}{1.478016in}}%
\pgfpathlineto{\pgfqpoint{3.345876in}{1.873169in}}%
\pgfpathlineto{\pgfqpoint{3.362801in}{1.912174in}}%
\pgfpathlineto{\pgfqpoint{3.379726in}{1.616983in}}%
\pgfpathlineto{\pgfqpoint{3.396650in}{2.320545in}}%
\pgfpathlineto{\pgfqpoint{3.413575in}{2.196688in}}%
\pgfpathlineto{\pgfqpoint{3.430500in}{2.034306in}}%
\pgfpathlineto{\pgfqpoint{3.447425in}{2.375912in}}%
\pgfpathlineto{\pgfqpoint{3.464349in}{2.123131in}}%
\pgfpathlineto{\pgfqpoint{3.481274in}{2.193103in}}%
\pgfpathlineto{\pgfqpoint{3.498199in}{1.768849in}}%
\pgfpathlineto{\pgfqpoint{3.515124in}{1.888117in}}%
\pgfpathlineto{\pgfqpoint{3.532048in}{2.161176in}}%
\pgfpathlineto{\pgfqpoint{3.548973in}{1.801889in}}%
\pgfpathlineto{\pgfqpoint{3.565898in}{1.955960in}}%
\pgfpathlineto{\pgfqpoint{3.582823in}{2.155038in}}%
\pgfpathlineto{\pgfqpoint{3.599747in}{1.670099in}}%
\pgfpathlineto{\pgfqpoint{3.616672in}{1.997614in}}%
\pgfpathlineto{\pgfqpoint{3.633597in}{2.159138in}}%
\pgfpathlineto{\pgfqpoint{3.650522in}{1.824517in}}%
\pgfpathlineto{\pgfqpoint{3.667446in}{1.967883in}}%
\pgfpathlineto{\pgfqpoint{3.684371in}{1.927001in}}%
\pgfpathlineto{\pgfqpoint{3.701296in}{1.943447in}}%
\pgfpathlineto{\pgfqpoint{3.718221in}{1.786035in}}%
\pgfpathlineto{\pgfqpoint{3.735145in}{2.254231in}}%
\pgfpathlineto{\pgfqpoint{3.752070in}{2.390271in}}%
\pgfpathlineto{\pgfqpoint{3.768995in}{1.761760in}}%
\pgfpathlineto{\pgfqpoint{3.785920in}{2.150791in}}%
\pgfpathlineto{\pgfqpoint{3.802844in}{1.784102in}}%
\pgfpathlineto{\pgfqpoint{3.819769in}{1.984374in}}%
\pgfpathlineto{\pgfqpoint{3.836694in}{1.586827in}}%
\pgfpathlineto{\pgfqpoint{3.853619in}{2.382846in}}%
\pgfpathlineto{\pgfqpoint{3.870544in}{1.827408in}}%
\pgfpathlineto{\pgfqpoint{3.887468in}{2.027260in}}%
\pgfpathlineto{\pgfqpoint{3.904393in}{1.647726in}}%
\pgfpathlineto{\pgfqpoint{3.921318in}{1.471950in}}%
\pgfpathlineto{\pgfqpoint{3.938243in}{2.198132in}}%
\pgfpathlineto{\pgfqpoint{3.955167in}{1.732559in}}%
\pgfpathlineto{\pgfqpoint{3.972092in}{2.006209in}}%
\pgfpathlineto{\pgfqpoint{3.989017in}{2.306258in}}%
\pgfpathlineto{\pgfqpoint{4.005942in}{1.741098in}}%
\pgfpathlineto{\pgfqpoint{4.022866in}{1.801427in}}%
\pgfpathlineto{\pgfqpoint{4.039791in}{1.889360in}}%
\pgfpathlineto{\pgfqpoint{4.056716in}{2.162001in}}%
\pgfpathlineto{\pgfqpoint{4.073641in}{2.363491in}}%
\pgfpathlineto{\pgfqpoint{4.090565in}{1.818287in}}%
\pgfpathlineto{\pgfqpoint{4.107490in}{1.641776in}}%
\pgfpathlineto{\pgfqpoint{4.124415in}{2.175155in}}%
\pgfpathlineto{\pgfqpoint{4.141340in}{1.874129in}}%
\pgfpathlineto{\pgfqpoint{4.158264in}{2.537594in}}%
\pgfpathlineto{\pgfqpoint{4.175189in}{2.387873in}}%
\pgfpathlineto{\pgfqpoint{4.192114in}{1.834321in}}%
\pgfpathlineto{\pgfqpoint{4.209039in}{2.142273in}}%
\pgfpathlineto{\pgfqpoint{4.225963in}{1.771598in}}%
\pgfpathlineto{\pgfqpoint{4.242888in}{1.942634in}}%
\pgfpathlineto{\pgfqpoint{4.259813in}{1.766869in}}%
\pgfpathlineto{\pgfqpoint{4.276738in}{1.923567in}}%
\pgfpathlineto{\pgfqpoint{4.293663in}{1.680717in}}%
\pgfpathlineto{\pgfqpoint{4.310587in}{1.865984in}}%
\pgfpathlineto{\pgfqpoint{4.327512in}{2.186302in}}%
\pgfpathlineto{\pgfqpoint{4.344437in}{1.727823in}}%
\pgfpathlineto{\pgfqpoint{4.361362in}{1.481793in}}%
\pgfpathlineto{\pgfqpoint{4.378286in}{2.215376in}}%
\pgfpathlineto{\pgfqpoint{4.395211in}{1.785546in}}%
\pgfpathlineto{\pgfqpoint{4.412136in}{2.019506in}}%
\pgfpathlineto{\pgfqpoint{4.429061in}{2.062264in}}%
\pgfpathlineto{\pgfqpoint{4.445985in}{2.046983in}}%
\pgfpathlineto{\pgfqpoint{4.462910in}{1.591527in}}%
\pgfpathlineto{\pgfqpoint{4.479835in}{2.270389in}}%
\pgfpathlineto{\pgfqpoint{4.496760in}{1.954429in}}%
\pgfpathlineto{\pgfqpoint{4.513684in}{1.823069in}}%
\pgfpathlineto{\pgfqpoint{4.530609in}{1.961305in}}%
\pgfpathlineto{\pgfqpoint{4.547534in}{1.786304in}}%
\pgfpathlineto{\pgfqpoint{4.564459in}{2.057479in}}%
\pgfpathlineto{\pgfqpoint{4.581383in}{1.999203in}}%
\pgfpathlineto{\pgfqpoint{4.598308in}{1.627364in}}%
\pgfpathlineto{\pgfqpoint{4.615233in}{1.435802in}}%
\pgfpathlineto{\pgfqpoint{4.632158in}{2.256153in}}%
\pgfpathlineto{\pgfqpoint{4.649082in}{1.899177in}}%
\pgfpathlineto{\pgfqpoint{4.666007in}{2.103873in}}%
\pgfpathlineto{\pgfqpoint{4.682932in}{2.011128in}}%
\pgfpathlineto{\pgfqpoint{4.699857in}{2.204592in}}%
\pgfpathlineto{\pgfqpoint{4.716781in}{1.999814in}}%
\pgfpathlineto{\pgfqpoint{4.733706in}{1.882732in}}%
\pgfpathlineto{\pgfqpoint{4.750631in}{1.734652in}}%
\pgfpathlineto{\pgfqpoint{4.767556in}{1.865619in}}%
\pgfpathlineto{\pgfqpoint{4.784481in}{2.038484in}}%
\pgfpathlineto{\pgfqpoint{4.801405in}{1.842617in}}%
\pgfpathlineto{\pgfqpoint{4.818330in}{2.342602in}}%
\pgfpathlineto{\pgfqpoint{4.835255in}{2.394160in}}%
\pgfpathlineto{\pgfqpoint{4.852180in}{2.360958in}}%
\pgfpathlineto{\pgfqpoint{4.869104in}{1.993580in}}%
\pgfpathlineto{\pgfqpoint{4.886029in}{2.289907in}}%
\pgfpathlineto{\pgfqpoint{4.902954in}{2.212687in}}%
\pgfpathlineto{\pgfqpoint{4.919879in}{1.983373in}}%
\pgfpathlineto{\pgfqpoint{4.936803in}{1.869416in}}%
\pgfpathlineto{\pgfqpoint{4.953728in}{2.096282in}}%
\pgfpathlineto{\pgfqpoint{4.970653in}{2.296113in}}%
\pgfpathlineto{\pgfqpoint{4.987578in}{1.644128in}}%
\pgfpathlineto{\pgfqpoint{5.004502in}{2.222646in}}%
\pgfpathlineto{\pgfqpoint{5.021427in}{2.134224in}}%
\pgfpathlineto{\pgfqpoint{5.038352in}{1.704242in}}%
\pgfpathlineto{\pgfqpoint{5.055277in}{1.581654in}}%
\pgfpathlineto{\pgfqpoint{5.072201in}{2.033107in}}%
\pgfpathlineto{\pgfqpoint{5.089126in}{1.665222in}}%
\pgfpathlineto{\pgfqpoint{5.106051in}{1.462509in}}%
\pgfpathlineto{\pgfqpoint{5.122976in}{1.998295in}}%
\pgfpathlineto{\pgfqpoint{5.139900in}{1.623400in}}%
\pgfpathlineto{\pgfqpoint{5.156825in}{1.894430in}}%
\pgfpathlineto{\pgfqpoint{5.173750in}{1.719198in}}%
\pgfpathlineto{\pgfqpoint{5.173750in}{1.719198in}}%
\pgfusepath{stroke}%
\end{pgfscope}%
\begin{pgfscope}%
\pgfpathrectangle{\pgfqpoint{2.916833in}{0.315889in}}{\pgfqpoint{2.364389in}{2.385311in}}%
\pgfusepath{clip}%
\pgfsetrectcap%
\pgfsetroundjoin%
\pgfsetlinewidth{1.505625pt}%
\definecolor{currentstroke}{rgb}{0.172549,0.627451,0.172549}%
\pgfsetstrokecolor{currentstroke}%
\pgfsetdash{}{0pt}%
\pgfpathmoveto{\pgfqpoint{3.024306in}{1.751697in}}%
\pgfpathlineto{\pgfqpoint{3.041230in}{2.163401in}}%
\pgfpathlineto{\pgfqpoint{3.058155in}{1.777307in}}%
\pgfpathlineto{\pgfqpoint{3.075080in}{2.325270in}}%
\pgfpathlineto{\pgfqpoint{3.092005in}{1.693122in}}%
\pgfpathlineto{\pgfqpoint{3.108929in}{2.139652in}}%
\pgfpathlineto{\pgfqpoint{3.125854in}{2.019145in}}%
\pgfpathlineto{\pgfqpoint{3.142779in}{2.422478in}}%
\pgfpathlineto{\pgfqpoint{3.159704in}{2.097129in}}%
\pgfpathlineto{\pgfqpoint{3.176628in}{2.228523in}}%
\pgfpathlineto{\pgfqpoint{3.193553in}{2.101427in}}%
\pgfpathlineto{\pgfqpoint{3.210478in}{1.789379in}}%
\pgfpathlineto{\pgfqpoint{3.227403in}{1.805483in}}%
\pgfpathlineto{\pgfqpoint{3.244327in}{2.487797in}}%
\pgfpathlineto{\pgfqpoint{3.261252in}{1.573438in}}%
\pgfpathlineto{\pgfqpoint{3.278177in}{2.044465in}}%
\pgfpathlineto{\pgfqpoint{3.295102in}{1.811405in}}%
\pgfpathlineto{\pgfqpoint{3.312026in}{2.254533in}}%
\pgfpathlineto{\pgfqpoint{3.328951in}{2.399252in}}%
\pgfpathlineto{\pgfqpoint{3.345876in}{1.770322in}}%
\pgfpathlineto{\pgfqpoint{3.362801in}{1.770427in}}%
\pgfpathlineto{\pgfqpoint{3.379726in}{2.330249in}}%
\pgfpathlineto{\pgfqpoint{3.396650in}{1.575998in}}%
\pgfpathlineto{\pgfqpoint{3.413575in}{2.383962in}}%
\pgfpathlineto{\pgfqpoint{3.430500in}{2.019316in}}%
\pgfpathlineto{\pgfqpoint{3.447425in}{2.120714in}}%
\pgfpathlineto{\pgfqpoint{3.464349in}{2.014571in}}%
\pgfpathlineto{\pgfqpoint{3.481274in}{2.178860in}}%
\pgfpathlineto{\pgfqpoint{3.498199in}{1.761323in}}%
\pgfpathlineto{\pgfqpoint{3.515124in}{1.716783in}}%
\pgfpathlineto{\pgfqpoint{3.532048in}{1.760593in}}%
\pgfpathlineto{\pgfqpoint{3.548973in}{1.719388in}}%
\pgfpathlineto{\pgfqpoint{3.565898in}{2.397474in}}%
\pgfpathlineto{\pgfqpoint{3.582823in}{2.066720in}}%
\pgfpathlineto{\pgfqpoint{3.599747in}{1.381622in}}%
\pgfpathlineto{\pgfqpoint{3.633597in}{1.922418in}}%
\pgfpathlineto{\pgfqpoint{3.650522in}{1.602648in}}%
\pgfpathlineto{\pgfqpoint{3.667446in}{1.925941in}}%
\pgfpathlineto{\pgfqpoint{3.684371in}{1.637217in}}%
\pgfpathlineto{\pgfqpoint{3.701296in}{2.223768in}}%
\pgfpathlineto{\pgfqpoint{3.718221in}{2.241421in}}%
\pgfpathlineto{\pgfqpoint{3.735145in}{1.901329in}}%
\pgfpathlineto{\pgfqpoint{3.752070in}{1.768575in}}%
\pgfpathlineto{\pgfqpoint{3.768995in}{2.187530in}}%
\pgfpathlineto{\pgfqpoint{3.785920in}{2.113771in}}%
\pgfpathlineto{\pgfqpoint{3.802844in}{2.179319in}}%
\pgfpathlineto{\pgfqpoint{3.819769in}{2.052651in}}%
\pgfpathlineto{\pgfqpoint{3.836694in}{2.274369in}}%
\pgfpathlineto{\pgfqpoint{3.853619in}{2.283033in}}%
\pgfpathlineto{\pgfqpoint{3.870544in}{1.636196in}}%
\pgfpathlineto{\pgfqpoint{3.887468in}{1.704257in}}%
\pgfpathlineto{\pgfqpoint{3.904393in}{1.905053in}}%
\pgfpathlineto{\pgfqpoint{3.921318in}{1.381200in}}%
\pgfpathlineto{\pgfqpoint{3.938243in}{1.605895in}}%
\pgfpathlineto{\pgfqpoint{3.955167in}{2.178829in}}%
\pgfpathlineto{\pgfqpoint{3.972092in}{1.975721in}}%
\pgfpathlineto{\pgfqpoint{3.989017in}{2.087662in}}%
\pgfpathlineto{\pgfqpoint{4.005942in}{1.756185in}}%
\pgfpathlineto{\pgfqpoint{4.022866in}{1.781790in}}%
\pgfpathlineto{\pgfqpoint{4.039791in}{1.371037in}}%
\pgfpathlineto{\pgfqpoint{4.056716in}{1.866171in}}%
\pgfpathlineto{\pgfqpoint{4.073641in}{1.875121in}}%
\pgfpathlineto{\pgfqpoint{4.090565in}{2.158564in}}%
\pgfpathlineto{\pgfqpoint{4.107490in}{1.608339in}}%
\pgfpathlineto{\pgfqpoint{4.124415in}{1.841394in}}%
\pgfpathlineto{\pgfqpoint{4.141340in}{1.660600in}}%
\pgfpathlineto{\pgfqpoint{4.158264in}{1.879281in}}%
\pgfpathlineto{\pgfqpoint{4.175189in}{1.915567in}}%
\pgfpathlineto{\pgfqpoint{4.192114in}{1.543830in}}%
\pgfpathlineto{\pgfqpoint{4.209039in}{1.627158in}}%
\pgfpathlineto{\pgfqpoint{4.225963in}{1.751299in}}%
\pgfpathlineto{\pgfqpoint{4.242888in}{1.735892in}}%
\pgfpathlineto{\pgfqpoint{4.259813in}{1.959816in}}%
\pgfpathlineto{\pgfqpoint{4.276738in}{2.300193in}}%
\pgfpathlineto{\pgfqpoint{4.293663in}{1.618862in}}%
\pgfpathlineto{\pgfqpoint{4.310587in}{2.144299in}}%
\pgfpathlineto{\pgfqpoint{4.327512in}{2.249486in}}%
\pgfpathlineto{\pgfqpoint{4.344437in}{1.858863in}}%
\pgfpathlineto{\pgfqpoint{4.361362in}{2.073743in}}%
\pgfpathlineto{\pgfqpoint{4.378286in}{1.602512in}}%
\pgfpathlineto{\pgfqpoint{4.395211in}{1.641205in}}%
\pgfpathlineto{\pgfqpoint{4.412136in}{1.434586in}}%
\pgfpathlineto{\pgfqpoint{4.429061in}{2.041299in}}%
\pgfpathlineto{\pgfqpoint{4.462910in}{1.634647in}}%
\pgfpathlineto{\pgfqpoint{4.479835in}{1.799538in}}%
\pgfpathlineto{\pgfqpoint{4.496760in}{2.274130in}}%
\pgfpathlineto{\pgfqpoint{4.513684in}{1.999594in}}%
\pgfpathlineto{\pgfqpoint{4.530609in}{1.820481in}}%
\pgfpathlineto{\pgfqpoint{4.547534in}{1.697687in}}%
\pgfpathlineto{\pgfqpoint{4.564459in}{1.748471in}}%
\pgfpathlineto{\pgfqpoint{4.581383in}{2.328138in}}%
\pgfpathlineto{\pgfqpoint{4.598308in}{1.730323in}}%
\pgfpathlineto{\pgfqpoint{4.615233in}{1.884308in}}%
\pgfpathlineto{\pgfqpoint{4.632158in}{1.617092in}}%
\pgfpathlineto{\pgfqpoint{4.649082in}{2.150019in}}%
\pgfpathlineto{\pgfqpoint{4.666007in}{1.951131in}}%
\pgfpathlineto{\pgfqpoint{4.682932in}{1.591454in}}%
\pgfpathlineto{\pgfqpoint{4.699857in}{1.739985in}}%
\pgfpathlineto{\pgfqpoint{4.716781in}{1.679294in}}%
\pgfpathlineto{\pgfqpoint{4.733706in}{1.782005in}}%
\pgfpathlineto{\pgfqpoint{4.750631in}{1.711679in}}%
\pgfpathlineto{\pgfqpoint{4.767556in}{2.017021in}}%
\pgfpathlineto{\pgfqpoint{4.784481in}{1.974109in}}%
\pgfpathlineto{\pgfqpoint{4.801405in}{1.816119in}}%
\pgfpathlineto{\pgfqpoint{4.818330in}{2.209482in}}%
\pgfpathlineto{\pgfqpoint{4.835255in}{1.483074in}}%
\pgfpathlineto{\pgfqpoint{4.852180in}{1.691919in}}%
\pgfpathlineto{\pgfqpoint{4.869104in}{1.448495in}}%
\pgfpathlineto{\pgfqpoint{4.886029in}{2.230565in}}%
\pgfpathlineto{\pgfqpoint{4.902954in}{1.929479in}}%
\pgfpathlineto{\pgfqpoint{4.919879in}{2.282995in}}%
\pgfpathlineto{\pgfqpoint{4.936803in}{1.414027in}}%
\pgfpathlineto{\pgfqpoint{4.953728in}{2.159888in}}%
\pgfpathlineto{\pgfqpoint{4.970653in}{1.704154in}}%
\pgfpathlineto{\pgfqpoint{4.987578in}{1.971912in}}%
\pgfpathlineto{\pgfqpoint{5.004502in}{1.718903in}}%
\pgfpathlineto{\pgfqpoint{5.021427in}{1.528675in}}%
\pgfpathlineto{\pgfqpoint{5.038352in}{1.561392in}}%
\pgfpathlineto{\pgfqpoint{5.055277in}{1.290270in}}%
\pgfpathlineto{\pgfqpoint{5.072201in}{1.448282in}}%
\pgfpathlineto{\pgfqpoint{5.089126in}{1.771726in}}%
\pgfpathlineto{\pgfqpoint{5.106051in}{1.643193in}}%
\pgfpathlineto{\pgfqpoint{5.122976in}{1.197003in}}%
\pgfpathlineto{\pgfqpoint{5.139900in}{1.651830in}}%
\pgfpathlineto{\pgfqpoint{5.156825in}{1.302702in}}%
\pgfpathlineto{\pgfqpoint{5.173750in}{1.579054in}}%
\pgfpathlineto{\pgfqpoint{5.173750in}{1.579054in}}%
\pgfusepath{stroke}%
\end{pgfscope}%
\begin{pgfscope}%
\pgfpathrectangle{\pgfqpoint{2.916833in}{0.315889in}}{\pgfqpoint{2.364389in}{2.385311in}}%
\pgfusepath{clip}%
\pgfsetrectcap%
\pgfsetroundjoin%
\pgfsetlinewidth{1.505625pt}%
\definecolor{currentstroke}{rgb}{0.839216,0.152941,0.156863}%
\pgfsetstrokecolor{currentstroke}%
\pgfsetdash{}{0pt}%
\pgfpathmoveto{\pgfqpoint{3.024306in}{2.342020in}}%
\pgfpathlineto{\pgfqpoint{3.041230in}{1.975662in}}%
\pgfpathlineto{\pgfqpoint{3.058155in}{2.314843in}}%
\pgfpathlineto{\pgfqpoint{3.075080in}{1.799909in}}%
\pgfpathlineto{\pgfqpoint{3.092005in}{2.187385in}}%
\pgfpathlineto{\pgfqpoint{3.108929in}{2.079657in}}%
\pgfpathlineto{\pgfqpoint{3.125854in}{2.341987in}}%
\pgfpathlineto{\pgfqpoint{3.142779in}{2.234268in}}%
\pgfpathlineto{\pgfqpoint{3.159704in}{1.777252in}}%
\pgfpathlineto{\pgfqpoint{3.176628in}{1.946951in}}%
\pgfpathlineto{\pgfqpoint{3.193553in}{2.345760in}}%
\pgfpathlineto{\pgfqpoint{3.210478in}{1.998236in}}%
\pgfpathlineto{\pgfqpoint{3.227403in}{1.781174in}}%
\pgfpathlineto{\pgfqpoint{3.244327in}{1.621123in}}%
\pgfpathlineto{\pgfqpoint{3.278177in}{2.112238in}}%
\pgfpathlineto{\pgfqpoint{3.295102in}{1.933312in}}%
\pgfpathlineto{\pgfqpoint{3.312026in}{2.250318in}}%
\pgfpathlineto{\pgfqpoint{3.328951in}{2.166164in}}%
\pgfpathlineto{\pgfqpoint{3.345876in}{2.170713in}}%
\pgfpathlineto{\pgfqpoint{3.362801in}{1.875261in}}%
\pgfpathlineto{\pgfqpoint{3.379726in}{1.550486in}}%
\pgfpathlineto{\pgfqpoint{3.396650in}{1.881645in}}%
\pgfpathlineto{\pgfqpoint{3.413575in}{1.889926in}}%
\pgfpathlineto{\pgfqpoint{3.430500in}{2.345456in}}%
\pgfpathlineto{\pgfqpoint{3.447425in}{2.513773in}}%
\pgfpathlineto{\pgfqpoint{3.464349in}{2.524702in}}%
\pgfpathlineto{\pgfqpoint{3.481274in}{2.028503in}}%
\pgfpathlineto{\pgfqpoint{3.498199in}{1.941757in}}%
\pgfpathlineto{\pgfqpoint{3.515124in}{1.807707in}}%
\pgfpathlineto{\pgfqpoint{3.532048in}{1.850780in}}%
\pgfpathlineto{\pgfqpoint{3.548973in}{1.977064in}}%
\pgfpathlineto{\pgfqpoint{3.565898in}{1.606397in}}%
\pgfpathlineto{\pgfqpoint{3.582823in}{1.803666in}}%
\pgfpathlineto{\pgfqpoint{3.599747in}{1.592810in}}%
\pgfpathlineto{\pgfqpoint{3.616672in}{1.921137in}}%
\pgfpathlineto{\pgfqpoint{3.650522in}{2.427825in}}%
\pgfpathlineto{\pgfqpoint{3.667446in}{2.313251in}}%
\pgfpathlineto{\pgfqpoint{3.684371in}{1.985566in}}%
\pgfpathlineto{\pgfqpoint{3.701296in}{1.887157in}}%
\pgfpathlineto{\pgfqpoint{3.718221in}{1.744132in}}%
\pgfpathlineto{\pgfqpoint{3.735145in}{2.345553in}}%
\pgfpathlineto{\pgfqpoint{3.752070in}{2.332416in}}%
\pgfpathlineto{\pgfqpoint{3.768995in}{1.790382in}}%
\pgfpathlineto{\pgfqpoint{3.785920in}{2.178471in}}%
\pgfpathlineto{\pgfqpoint{3.802844in}{1.834102in}}%
\pgfpathlineto{\pgfqpoint{3.819769in}{1.766440in}}%
\pgfpathlineto{\pgfqpoint{3.836694in}{1.568762in}}%
\pgfpathlineto{\pgfqpoint{3.853619in}{1.747634in}}%
\pgfpathlineto{\pgfqpoint{3.870544in}{1.871143in}}%
\pgfpathlineto{\pgfqpoint{3.887468in}{2.083219in}}%
\pgfpathlineto{\pgfqpoint{3.904393in}{1.526919in}}%
\pgfpathlineto{\pgfqpoint{3.921318in}{1.852409in}}%
\pgfpathlineto{\pgfqpoint{3.938243in}{1.996916in}}%
\pgfpathlineto{\pgfqpoint{3.955167in}{2.018174in}}%
\pgfpathlineto{\pgfqpoint{3.972092in}{1.845327in}}%
\pgfpathlineto{\pgfqpoint{3.989017in}{1.530985in}}%
\pgfpathlineto{\pgfqpoint{4.005942in}{2.000287in}}%
\pgfpathlineto{\pgfqpoint{4.022866in}{1.595710in}}%
\pgfpathlineto{\pgfqpoint{4.039791in}{1.587117in}}%
\pgfpathlineto{\pgfqpoint{4.056716in}{1.988324in}}%
\pgfpathlineto{\pgfqpoint{4.073641in}{1.442353in}}%
\pgfpathlineto{\pgfqpoint{4.090565in}{1.683249in}}%
\pgfpathlineto{\pgfqpoint{4.107490in}{1.446404in}}%
\pgfpathlineto{\pgfqpoint{4.124415in}{1.525438in}}%
\pgfpathlineto{\pgfqpoint{4.158264in}{2.104044in}}%
\pgfpathlineto{\pgfqpoint{4.175189in}{1.595777in}}%
\pgfpathlineto{\pgfqpoint{4.192114in}{1.338274in}}%
\pgfpathlineto{\pgfqpoint{4.209039in}{1.541650in}}%
\pgfpathlineto{\pgfqpoint{4.225963in}{1.612516in}}%
\pgfpathlineto{\pgfqpoint{4.242888in}{1.910433in}}%
\pgfpathlineto{\pgfqpoint{4.259813in}{1.687349in}}%
\pgfpathlineto{\pgfqpoint{4.276738in}{1.154913in}}%
\pgfpathlineto{\pgfqpoint{4.293663in}{1.723654in}}%
\pgfpathlineto{\pgfqpoint{4.310587in}{1.510726in}}%
\pgfpathlineto{\pgfqpoint{4.327512in}{1.697580in}}%
\pgfpathlineto{\pgfqpoint{4.344437in}{1.680828in}}%
\pgfpathlineto{\pgfqpoint{4.361362in}{1.962579in}}%
\pgfpathlineto{\pgfqpoint{4.378286in}{1.421338in}}%
\pgfpathlineto{\pgfqpoint{4.395211in}{1.267131in}}%
\pgfpathlineto{\pgfqpoint{4.412136in}{1.826970in}}%
\pgfpathlineto{\pgfqpoint{4.429061in}{1.675260in}}%
\pgfpathlineto{\pgfqpoint{4.445985in}{1.695056in}}%
\pgfpathlineto{\pgfqpoint{4.462910in}{1.961604in}}%
\pgfpathlineto{\pgfqpoint{4.479835in}{1.312028in}}%
\pgfpathlineto{\pgfqpoint{4.496760in}{1.049693in}}%
\pgfpathlineto{\pgfqpoint{4.513684in}{1.165053in}}%
\pgfpathlineto{\pgfqpoint{4.530609in}{1.229485in}}%
\pgfpathlineto{\pgfqpoint{4.547534in}{1.782218in}}%
\pgfpathlineto{\pgfqpoint{4.564459in}{1.922018in}}%
\pgfpathlineto{\pgfqpoint{4.581383in}{1.303343in}}%
\pgfpathlineto{\pgfqpoint{4.598308in}{1.543461in}}%
\pgfpathlineto{\pgfqpoint{4.615233in}{1.640841in}}%
\pgfpathlineto{\pgfqpoint{4.632158in}{1.467565in}}%
\pgfpathlineto{\pgfqpoint{4.649082in}{1.865486in}}%
\pgfpathlineto{\pgfqpoint{4.666007in}{1.382736in}}%
\pgfpathlineto{\pgfqpoint{4.682932in}{1.401026in}}%
\pgfpathlineto{\pgfqpoint{4.699857in}{1.618975in}}%
\pgfpathlineto{\pgfqpoint{4.716781in}{1.645058in}}%
\pgfpathlineto{\pgfqpoint{4.733706in}{1.885338in}}%
\pgfpathlineto{\pgfqpoint{4.750631in}{1.506306in}}%
\pgfpathlineto{\pgfqpoint{4.767556in}{1.522825in}}%
\pgfpathlineto{\pgfqpoint{4.784481in}{1.074597in}}%
\pgfpathlineto{\pgfqpoint{4.801405in}{0.949057in}}%
\pgfpathlineto{\pgfqpoint{4.818330in}{1.448944in}}%
\pgfpathlineto{\pgfqpoint{4.835255in}{0.915462in}}%
\pgfpathlineto{\pgfqpoint{4.852180in}{1.627050in}}%
\pgfpathlineto{\pgfqpoint{4.869104in}{1.308563in}}%
\pgfpathlineto{\pgfqpoint{4.886029in}{0.749583in}}%
\pgfpathlineto{\pgfqpoint{4.902954in}{1.332340in}}%
\pgfpathlineto{\pgfqpoint{4.919879in}{1.154560in}}%
\pgfpathlineto{\pgfqpoint{4.936803in}{1.100890in}}%
\pgfpathlineto{\pgfqpoint{4.953728in}{1.610005in}}%
\pgfpathlineto{\pgfqpoint{4.970653in}{0.620635in}}%
\pgfpathlineto{\pgfqpoint{4.987578in}{0.585883in}}%
\pgfpathlineto{\pgfqpoint{5.004502in}{0.720499in}}%
\pgfpathlineto{\pgfqpoint{5.021427in}{0.657776in}}%
\pgfpathlineto{\pgfqpoint{5.038352in}{0.846767in}}%
\pgfpathlineto{\pgfqpoint{5.055277in}{1.327449in}}%
\pgfpathlineto{\pgfqpoint{5.072201in}{0.424312in}}%
\pgfpathlineto{\pgfqpoint{5.089126in}{0.875551in}}%
\pgfpathlineto{\pgfqpoint{5.106051in}{0.543912in}}%
\pgfpathlineto{\pgfqpoint{5.122976in}{1.238272in}}%
\pgfpathlineto{\pgfqpoint{5.139900in}{0.989581in}}%
\pgfpathlineto{\pgfqpoint{5.156825in}{0.850358in}}%
\pgfpathlineto{\pgfqpoint{5.173750in}{1.083119in}}%
\pgfpathlineto{\pgfqpoint{5.173750in}{1.083119in}}%
\pgfusepath{stroke}%
\end{pgfscope}%
\begin{pgfscope}%
\pgfsetrectcap%
\pgfsetmiterjoin%
\pgfsetlinewidth{0.803000pt}%
\definecolor{currentstroke}{rgb}{0.000000,0.000000,0.000000}%
\pgfsetstrokecolor{currentstroke}%
\pgfsetdash{}{0pt}%
\pgfpathmoveto{\pgfqpoint{2.916833in}{0.315889in}}%
\pgfpathlineto{\pgfqpoint{2.916833in}{2.701200in}}%
\pgfusepath{stroke}%
\end{pgfscope}%
\begin{pgfscope}%
\pgfsetrectcap%
\pgfsetmiterjoin%
\pgfsetlinewidth{0.803000pt}%
\definecolor{currentstroke}{rgb}{0.000000,0.000000,0.000000}%
\pgfsetstrokecolor{currentstroke}%
\pgfsetdash{}{0pt}%
\pgfpathmoveto{\pgfqpoint{5.281222in}{0.315889in}}%
\pgfpathlineto{\pgfqpoint{5.281222in}{2.701200in}}%
\pgfusepath{stroke}%
\end{pgfscope}%
\begin{pgfscope}%
\pgfsetrectcap%
\pgfsetmiterjoin%
\pgfsetlinewidth{0.803000pt}%
\definecolor{currentstroke}{rgb}{0.000000,0.000000,0.000000}%
\pgfsetstrokecolor{currentstroke}%
\pgfsetdash{}{0pt}%
\pgfpathmoveto{\pgfqpoint{2.916833in}{0.315889in}}%
\pgfpathlineto{\pgfqpoint{5.281222in}{0.315889in}}%
\pgfusepath{stroke}%
\end{pgfscope}%
\begin{pgfscope}%
\pgfsetrectcap%
\pgfsetmiterjoin%
\pgfsetlinewidth{0.803000pt}%
\definecolor{currentstroke}{rgb}{0.000000,0.000000,0.000000}%
\pgfsetstrokecolor{currentstroke}%
\pgfsetdash{}{0pt}%
\pgfpathmoveto{\pgfqpoint{2.916833in}{2.701200in}}%
\pgfpathlineto{\pgfqpoint{5.281222in}{2.701200in}}%
\pgfusepath{stroke}%
\end{pgfscope}%
\begin{pgfscope}%
\definecolor{textcolor}{rgb}{0.000000,0.000000,0.000000}%
\pgfsetstrokecolor{textcolor}%
\pgfsetfillcolor{textcolor}%
\pgftext[x=4.099028in,y=2.784533in,,base]{\color{textcolor}\rmfamily\fontsize{9.600000}{11.520000}\selectfont Detailkoeffizienten}%
\end{pgfscope}%
\end{pgfpicture}%
\makeatother%
\endgroup%

    \caption{Analyse verrauschter Signale mit Haar Wavelet\label{polynomials:noise:db1}}
\end{figure}

Nun können wir aber mit dem Wavelet eine Multiskalenanalyse durchführen. Im
Fall vom Haar Wavelet ist dies vergleichbar wie wenn zuerst der Mittelwert
gebildet und dann abgeleitet wird. 

In \autoref{polynomials:noise:db1_multi} ist das Signal $x^2 + r$, die
Detailkoeffizienten der verschiedenen Stufen der Multiskalenanalyse mit dem
Haar Wavelet und dann noch die Approximationskoeffizienten der letzten Stufe
abgebildet. Es ist zu sehen, dass die Detailkoeffizienten mit höherer Stufe
eher der Ableitung des originalen Polynoms entsprechen und weniger vom Rauschen
betroffen sind. Jedoch nimmt die zeitliche Auflösung ab.
Die Stufenbreite macht jeweils den zeitlichen Bereich, also die Anzahl Samples,
welcher für die Berechnung des jeweiligen Koeffizienten relevant ist sichtbar.

\begin{figure}
    \centering
    %% Creator: Matplotlib, PGF backend
%%
%% To include the figure in your LaTeX document, write
%%   \input{<filename>.pgf}
%%
%% Make sure the required packages are loaded in your preamble
%%   \usepackage{pgf}
%%
%% Figures using additional raster images can only be included by \input if
%% they are in the same directory as the main LaTeX file. For loading figures
%% from other directories you can use the `import` package
%%   \usepackage{import}
%% and then include the figures with
%%   \import{<path to file>}{<filename>.pgf}
%%
%% Matplotlib used the following preamble
%%   \usepackage{fontspec}
%%
\begingroup%
\makeatletter%
\begin{pgfpicture}%
\pgfpathrectangle{\pgfpointorigin}{\pgfqpoint{5.800000in}{6.000000in}}%
\pgfusepath{use as bounding box, clip}%
\begin{pgfscope}%
\pgfsetbuttcap%
\pgfsetmiterjoin%
\definecolor{currentfill}{rgb}{1.000000,1.000000,1.000000}%
\pgfsetfillcolor{currentfill}%
\pgfsetlinewidth{0.000000pt}%
\definecolor{currentstroke}{rgb}{1.000000,1.000000,1.000000}%
\pgfsetstrokecolor{currentstroke}%
\pgfsetdash{}{0pt}%
\pgfpathmoveto{\pgfqpoint{0.000000in}{0.000000in}}%
\pgfpathlineto{\pgfqpoint{5.800000in}{0.000000in}}%
\pgfpathlineto{\pgfqpoint{5.800000in}{6.000000in}}%
\pgfpathlineto{\pgfqpoint{0.000000in}{6.000000in}}%
\pgfpathclose%
\pgfusepath{fill}%
\end{pgfscope}%
\begin{pgfscope}%
\pgfsetbuttcap%
\pgfsetmiterjoin%
\definecolor{currentfill}{rgb}{1.000000,1.000000,1.000000}%
\pgfsetfillcolor{currentfill}%
\pgfsetlinewidth{0.000000pt}%
\definecolor{currentstroke}{rgb}{0.000000,0.000000,0.000000}%
\pgfsetstrokecolor{currentstroke}%
\pgfsetstrokeopacity{0.000000}%
\pgfsetdash{}{0pt}%
\pgfpathmoveto{\pgfqpoint{0.459778in}{5.093981in}}%
\pgfpathlineto{\pgfqpoint{5.631389in}{5.093981in}}%
\pgfpathlineto{\pgfqpoint{5.631389in}{5.831389in}}%
\pgfpathlineto{\pgfqpoint{0.459778in}{5.831389in}}%
\pgfpathclose%
\pgfusepath{fill}%
\end{pgfscope}%
\begin{pgfscope}%
\pgfsetbuttcap%
\pgfsetroundjoin%
\definecolor{currentfill}{rgb}{0.000000,0.000000,0.000000}%
\pgfsetfillcolor{currentfill}%
\pgfsetlinewidth{0.803000pt}%
\definecolor{currentstroke}{rgb}{0.000000,0.000000,0.000000}%
\pgfsetstrokecolor{currentstroke}%
\pgfsetdash{}{0pt}%
\pgfsys@defobject{currentmarker}{\pgfqpoint{0.000000in}{-0.048611in}}{\pgfqpoint{0.000000in}{0.000000in}}{%
\pgfpathmoveto{\pgfqpoint{0.000000in}{0.000000in}}%
\pgfpathlineto{\pgfqpoint{0.000000in}{-0.048611in}}%
\pgfusepath{stroke,fill}%
}%
\begin{pgfscope}%
\pgfsys@transformshift{0.694851in}{5.093981in}%
\pgfsys@useobject{currentmarker}{}%
\end{pgfscope}%
\end{pgfscope}%
\begin{pgfscope}%
\pgfsetbuttcap%
\pgfsetroundjoin%
\definecolor{currentfill}{rgb}{0.000000,0.000000,0.000000}%
\pgfsetfillcolor{currentfill}%
\pgfsetlinewidth{0.803000pt}%
\definecolor{currentstroke}{rgb}{0.000000,0.000000,0.000000}%
\pgfsetstrokecolor{currentstroke}%
\pgfsetdash{}{0pt}%
\pgfsys@defobject{currentmarker}{\pgfqpoint{0.000000in}{-0.048611in}}{\pgfqpoint{0.000000in}{0.000000in}}{%
\pgfpathmoveto{\pgfqpoint{0.000000in}{0.000000in}}%
\pgfpathlineto{\pgfqpoint{0.000000in}{-0.048611in}}%
\pgfusepath{stroke,fill}%
}%
\begin{pgfscope}%
\pgfsys@transformshift{1.282534in}{5.093981in}%
\pgfsys@useobject{currentmarker}{}%
\end{pgfscope}%
\end{pgfscope}%
\begin{pgfscope}%
\pgfsetbuttcap%
\pgfsetroundjoin%
\definecolor{currentfill}{rgb}{0.000000,0.000000,0.000000}%
\pgfsetfillcolor{currentfill}%
\pgfsetlinewidth{0.803000pt}%
\definecolor{currentstroke}{rgb}{0.000000,0.000000,0.000000}%
\pgfsetstrokecolor{currentstroke}%
\pgfsetdash{}{0pt}%
\pgfsys@defobject{currentmarker}{\pgfqpoint{0.000000in}{-0.048611in}}{\pgfqpoint{0.000000in}{0.000000in}}{%
\pgfpathmoveto{\pgfqpoint{0.000000in}{0.000000in}}%
\pgfpathlineto{\pgfqpoint{0.000000in}{-0.048611in}}%
\pgfusepath{stroke,fill}%
}%
\begin{pgfscope}%
\pgfsys@transformshift{1.870217in}{5.093981in}%
\pgfsys@useobject{currentmarker}{}%
\end{pgfscope}%
\end{pgfscope}%
\begin{pgfscope}%
\pgfsetbuttcap%
\pgfsetroundjoin%
\definecolor{currentfill}{rgb}{0.000000,0.000000,0.000000}%
\pgfsetfillcolor{currentfill}%
\pgfsetlinewidth{0.803000pt}%
\definecolor{currentstroke}{rgb}{0.000000,0.000000,0.000000}%
\pgfsetstrokecolor{currentstroke}%
\pgfsetdash{}{0pt}%
\pgfsys@defobject{currentmarker}{\pgfqpoint{0.000000in}{-0.048611in}}{\pgfqpoint{0.000000in}{0.000000in}}{%
\pgfpathmoveto{\pgfqpoint{0.000000in}{0.000000in}}%
\pgfpathlineto{\pgfqpoint{0.000000in}{-0.048611in}}%
\pgfusepath{stroke,fill}%
}%
\begin{pgfscope}%
\pgfsys@transformshift{2.457900in}{5.093981in}%
\pgfsys@useobject{currentmarker}{}%
\end{pgfscope}%
\end{pgfscope}%
\begin{pgfscope}%
\pgfsetbuttcap%
\pgfsetroundjoin%
\definecolor{currentfill}{rgb}{0.000000,0.000000,0.000000}%
\pgfsetfillcolor{currentfill}%
\pgfsetlinewidth{0.803000pt}%
\definecolor{currentstroke}{rgb}{0.000000,0.000000,0.000000}%
\pgfsetstrokecolor{currentstroke}%
\pgfsetdash{}{0pt}%
\pgfsys@defobject{currentmarker}{\pgfqpoint{0.000000in}{-0.048611in}}{\pgfqpoint{0.000000in}{0.000000in}}{%
\pgfpathmoveto{\pgfqpoint{0.000000in}{0.000000in}}%
\pgfpathlineto{\pgfqpoint{0.000000in}{-0.048611in}}%
\pgfusepath{stroke,fill}%
}%
\begin{pgfscope}%
\pgfsys@transformshift{3.045583in}{5.093981in}%
\pgfsys@useobject{currentmarker}{}%
\end{pgfscope}%
\end{pgfscope}%
\begin{pgfscope}%
\pgfsetbuttcap%
\pgfsetroundjoin%
\definecolor{currentfill}{rgb}{0.000000,0.000000,0.000000}%
\pgfsetfillcolor{currentfill}%
\pgfsetlinewidth{0.803000pt}%
\definecolor{currentstroke}{rgb}{0.000000,0.000000,0.000000}%
\pgfsetstrokecolor{currentstroke}%
\pgfsetdash{}{0pt}%
\pgfsys@defobject{currentmarker}{\pgfqpoint{0.000000in}{-0.048611in}}{\pgfqpoint{0.000000in}{0.000000in}}{%
\pgfpathmoveto{\pgfqpoint{0.000000in}{0.000000in}}%
\pgfpathlineto{\pgfqpoint{0.000000in}{-0.048611in}}%
\pgfusepath{stroke,fill}%
}%
\begin{pgfscope}%
\pgfsys@transformshift{3.633266in}{5.093981in}%
\pgfsys@useobject{currentmarker}{}%
\end{pgfscope}%
\end{pgfscope}%
\begin{pgfscope}%
\pgfsetbuttcap%
\pgfsetroundjoin%
\definecolor{currentfill}{rgb}{0.000000,0.000000,0.000000}%
\pgfsetfillcolor{currentfill}%
\pgfsetlinewidth{0.803000pt}%
\definecolor{currentstroke}{rgb}{0.000000,0.000000,0.000000}%
\pgfsetstrokecolor{currentstroke}%
\pgfsetdash{}{0pt}%
\pgfsys@defobject{currentmarker}{\pgfqpoint{0.000000in}{-0.048611in}}{\pgfqpoint{0.000000in}{0.000000in}}{%
\pgfpathmoveto{\pgfqpoint{0.000000in}{0.000000in}}%
\pgfpathlineto{\pgfqpoint{0.000000in}{-0.048611in}}%
\pgfusepath{stroke,fill}%
}%
\begin{pgfscope}%
\pgfsys@transformshift{4.220949in}{5.093981in}%
\pgfsys@useobject{currentmarker}{}%
\end{pgfscope}%
\end{pgfscope}%
\begin{pgfscope}%
\pgfsetbuttcap%
\pgfsetroundjoin%
\definecolor{currentfill}{rgb}{0.000000,0.000000,0.000000}%
\pgfsetfillcolor{currentfill}%
\pgfsetlinewidth{0.803000pt}%
\definecolor{currentstroke}{rgb}{0.000000,0.000000,0.000000}%
\pgfsetstrokecolor{currentstroke}%
\pgfsetdash{}{0pt}%
\pgfsys@defobject{currentmarker}{\pgfqpoint{0.000000in}{-0.048611in}}{\pgfqpoint{0.000000in}{0.000000in}}{%
\pgfpathmoveto{\pgfqpoint{0.000000in}{0.000000in}}%
\pgfpathlineto{\pgfqpoint{0.000000in}{-0.048611in}}%
\pgfusepath{stroke,fill}%
}%
\begin{pgfscope}%
\pgfsys@transformshift{4.808633in}{5.093981in}%
\pgfsys@useobject{currentmarker}{}%
\end{pgfscope}%
\end{pgfscope}%
\begin{pgfscope}%
\pgfsetbuttcap%
\pgfsetroundjoin%
\definecolor{currentfill}{rgb}{0.000000,0.000000,0.000000}%
\pgfsetfillcolor{currentfill}%
\pgfsetlinewidth{0.803000pt}%
\definecolor{currentstroke}{rgb}{0.000000,0.000000,0.000000}%
\pgfsetstrokecolor{currentstroke}%
\pgfsetdash{}{0pt}%
\pgfsys@defobject{currentmarker}{\pgfqpoint{0.000000in}{-0.048611in}}{\pgfqpoint{0.000000in}{0.000000in}}{%
\pgfpathmoveto{\pgfqpoint{0.000000in}{0.000000in}}%
\pgfpathlineto{\pgfqpoint{0.000000in}{-0.048611in}}%
\pgfusepath{stroke,fill}%
}%
\begin{pgfscope}%
\pgfsys@transformshift{5.396316in}{5.093981in}%
\pgfsys@useobject{currentmarker}{}%
\end{pgfscope}%
\end{pgfscope}%
\begin{pgfscope}%
\pgfsetbuttcap%
\pgfsetroundjoin%
\definecolor{currentfill}{rgb}{0.000000,0.000000,0.000000}%
\pgfsetfillcolor{currentfill}%
\pgfsetlinewidth{0.803000pt}%
\definecolor{currentstroke}{rgb}{0.000000,0.000000,0.000000}%
\pgfsetstrokecolor{currentstroke}%
\pgfsetdash{}{0pt}%
\pgfsys@defobject{currentmarker}{\pgfqpoint{-0.048611in}{0.000000in}}{\pgfqpoint{0.000000in}{0.000000in}}{%
\pgfpathmoveto{\pgfqpoint{0.000000in}{0.000000in}}%
\pgfpathlineto{\pgfqpoint{-0.048611in}{0.000000in}}%
\pgfusepath{stroke,fill}%
}%
\begin{pgfscope}%
\pgfsys@transformshift{0.459778in}{5.126603in}%
\pgfsys@useobject{currentmarker}{}%
\end{pgfscope}%
\end{pgfscope}%
\begin{pgfscope}%
\definecolor{textcolor}{rgb}{0.000000,0.000000,0.000000}%
\pgfsetstrokecolor{textcolor}%
\pgfsetfillcolor{textcolor}%
\pgftext[x=0.303556in,y=5.088047in,left,base]{\color{textcolor}\rmfamily\fontsize{8.000000}{9.600000}\selectfont 0}%
\end{pgfscope}%
\begin{pgfscope}%
\pgfsetbuttcap%
\pgfsetroundjoin%
\definecolor{currentfill}{rgb}{0.000000,0.000000,0.000000}%
\pgfsetfillcolor{currentfill}%
\pgfsetlinewidth{0.803000pt}%
\definecolor{currentstroke}{rgb}{0.000000,0.000000,0.000000}%
\pgfsetstrokecolor{currentstroke}%
\pgfsetdash{}{0pt}%
\pgfsys@defobject{currentmarker}{\pgfqpoint{-0.048611in}{0.000000in}}{\pgfqpoint{0.000000in}{0.000000in}}{%
\pgfpathmoveto{\pgfqpoint{0.000000in}{0.000000in}}%
\pgfpathlineto{\pgfqpoint{-0.048611in}{0.000000in}}%
\pgfusepath{stroke,fill}%
}%
\begin{pgfscope}%
\pgfsys@transformshift{0.459778in}{5.461296in}%
\pgfsys@useobject{currentmarker}{}%
\end{pgfscope}%
\end{pgfscope}%
\begin{pgfscope}%
\definecolor{textcolor}{rgb}{0.000000,0.000000,0.000000}%
\pgfsetstrokecolor{textcolor}%
\pgfsetfillcolor{textcolor}%
\pgftext[x=0.303556in,y=5.422740in,left,base]{\color{textcolor}\rmfamily\fontsize{8.000000}{9.600000}\selectfont 2}%
\end{pgfscope}%
\begin{pgfscope}%
\pgfsetbuttcap%
\pgfsetroundjoin%
\definecolor{currentfill}{rgb}{0.000000,0.000000,0.000000}%
\pgfsetfillcolor{currentfill}%
\pgfsetlinewidth{0.803000pt}%
\definecolor{currentstroke}{rgb}{0.000000,0.000000,0.000000}%
\pgfsetstrokecolor{currentstroke}%
\pgfsetdash{}{0pt}%
\pgfsys@defobject{currentmarker}{\pgfqpoint{-0.048611in}{0.000000in}}{\pgfqpoint{0.000000in}{0.000000in}}{%
\pgfpathmoveto{\pgfqpoint{0.000000in}{0.000000in}}%
\pgfpathlineto{\pgfqpoint{-0.048611in}{0.000000in}}%
\pgfusepath{stroke,fill}%
}%
\begin{pgfscope}%
\pgfsys@transformshift{0.459778in}{5.795989in}%
\pgfsys@useobject{currentmarker}{}%
\end{pgfscope}%
\end{pgfscope}%
\begin{pgfscope}%
\definecolor{textcolor}{rgb}{0.000000,0.000000,0.000000}%
\pgfsetstrokecolor{textcolor}%
\pgfsetfillcolor{textcolor}%
\pgftext[x=0.303556in,y=5.757433in,left,base]{\color{textcolor}\rmfamily\fontsize{8.000000}{9.600000}\selectfont 4}%
\end{pgfscope}%
\begin{pgfscope}%
\pgfpathrectangle{\pgfqpoint{0.459778in}{5.093981in}}{\pgfqpoint{5.171611in}{0.737408in}}%
\pgfusepath{clip}%
\pgfsetrectcap%
\pgfsetroundjoin%
\pgfsetlinewidth{1.505625pt}%
\definecolor{currentstroke}{rgb}{0.121569,0.466667,0.705882}%
\pgfsetstrokecolor{currentstroke}%
\pgfsetdash{}{0pt}%
\pgfpathmoveto{\pgfqpoint{0.694851in}{5.128545in}}%
\pgfpathlineto{\pgfqpoint{0.713288in}{5.132698in}}%
\pgfpathlineto{\pgfqpoint{0.768599in}{5.128634in}}%
\pgfpathlineto{\pgfqpoint{0.787037in}{5.132451in}}%
\pgfpathlineto{\pgfqpoint{0.805474in}{5.131221in}}%
\pgfpathlineto{\pgfqpoint{0.823911in}{5.127830in}}%
\pgfpathlineto{\pgfqpoint{0.842348in}{5.127500in}}%
\pgfpathlineto{\pgfqpoint{0.860785in}{5.132423in}}%
\pgfpathlineto{\pgfqpoint{0.879222in}{5.128807in}}%
\pgfpathlineto{\pgfqpoint{0.897659in}{5.127858in}}%
\pgfpathlineto{\pgfqpoint{0.952971in}{5.133625in}}%
\pgfpathlineto{\pgfqpoint{0.971408in}{5.128956in}}%
\pgfpathlineto{\pgfqpoint{0.989845in}{5.135915in}}%
\pgfpathlineto{\pgfqpoint{1.100468in}{5.133197in}}%
\pgfpathlineto{\pgfqpoint{1.118905in}{5.136854in}}%
\pgfpathlineto{\pgfqpoint{1.137342in}{5.136481in}}%
\pgfpathlineto{\pgfqpoint{1.155779in}{5.139927in}}%
\pgfpathlineto{\pgfqpoint{1.174216in}{5.139647in}}%
\pgfpathlineto{\pgfqpoint{1.192653in}{5.134119in}}%
\pgfpathlineto{\pgfqpoint{1.211090in}{5.135351in}}%
\pgfpathlineto{\pgfqpoint{1.229527in}{5.141849in}}%
\pgfpathlineto{\pgfqpoint{1.266402in}{5.143539in}}%
\pgfpathlineto{\pgfqpoint{1.284839in}{5.142386in}}%
\pgfpathlineto{\pgfqpoint{1.303276in}{5.145754in}}%
\pgfpathlineto{\pgfqpoint{1.321713in}{5.145142in}}%
\pgfpathlineto{\pgfqpoint{1.340150in}{5.142681in}}%
\pgfpathlineto{\pgfqpoint{1.358587in}{5.145781in}}%
\pgfpathlineto{\pgfqpoint{1.377024in}{5.141417in}}%
\pgfpathlineto{\pgfqpoint{1.413899in}{5.150057in}}%
\pgfpathlineto{\pgfqpoint{1.432336in}{5.146163in}}%
\pgfpathlineto{\pgfqpoint{1.450773in}{5.150069in}}%
\pgfpathlineto{\pgfqpoint{1.469210in}{5.149325in}}%
\pgfpathlineto{\pgfqpoint{1.487647in}{5.145869in}}%
\pgfpathlineto{\pgfqpoint{1.506084in}{5.147677in}}%
\pgfpathlineto{\pgfqpoint{1.524521in}{5.154141in}}%
\pgfpathlineto{\pgfqpoint{1.542958in}{5.155704in}}%
\pgfpathlineto{\pgfqpoint{1.561395in}{5.151541in}}%
\pgfpathlineto{\pgfqpoint{1.616707in}{5.156761in}}%
\pgfpathlineto{\pgfqpoint{1.635144in}{5.156061in}}%
\pgfpathlineto{\pgfqpoint{1.653581in}{5.161645in}}%
\pgfpathlineto{\pgfqpoint{1.672018in}{5.162341in}}%
\pgfpathlineto{\pgfqpoint{1.708892in}{5.158351in}}%
\pgfpathlineto{\pgfqpoint{1.745767in}{5.167929in}}%
\pgfpathlineto{\pgfqpoint{1.764204in}{5.161382in}}%
\pgfpathlineto{\pgfqpoint{1.782641in}{5.165994in}}%
\pgfpathlineto{\pgfqpoint{1.801078in}{5.166972in}}%
\pgfpathlineto{\pgfqpoint{1.819515in}{5.171008in}}%
\pgfpathlineto{\pgfqpoint{1.837952in}{5.170195in}}%
\pgfpathlineto{\pgfqpoint{1.856389in}{5.174773in}}%
\pgfpathlineto{\pgfqpoint{1.874826in}{5.176342in}}%
\pgfpathlineto{\pgfqpoint{1.893264in}{5.172002in}}%
\pgfpathlineto{\pgfqpoint{1.911701in}{5.174476in}}%
\pgfpathlineto{\pgfqpoint{1.948575in}{5.174354in}}%
\pgfpathlineto{\pgfqpoint{1.967012in}{5.183374in}}%
\pgfpathlineto{\pgfqpoint{1.985449in}{5.180593in}}%
\pgfpathlineto{\pgfqpoint{2.003886in}{5.186032in}}%
\pgfpathlineto{\pgfqpoint{2.022323in}{5.184009in}}%
\pgfpathlineto{\pgfqpoint{2.040760in}{5.185916in}}%
\pgfpathlineto{\pgfqpoint{2.059198in}{5.184339in}}%
\pgfpathlineto{\pgfqpoint{2.077635in}{5.190452in}}%
\pgfpathlineto{\pgfqpoint{2.132946in}{5.193100in}}%
\pgfpathlineto{\pgfqpoint{2.151383in}{5.198759in}}%
\pgfpathlineto{\pgfqpoint{2.188257in}{5.196107in}}%
\pgfpathlineto{\pgfqpoint{2.206695in}{5.203092in}}%
\pgfpathlineto{\pgfqpoint{2.225132in}{5.200804in}}%
\pgfpathlineto{\pgfqpoint{2.243569in}{5.201247in}}%
\pgfpathlineto{\pgfqpoint{2.280443in}{5.205623in}}%
\pgfpathlineto{\pgfqpoint{2.298880in}{5.209554in}}%
\pgfpathlineto{\pgfqpoint{2.317317in}{5.211286in}}%
\pgfpathlineto{\pgfqpoint{2.335754in}{5.209707in}}%
\pgfpathlineto{\pgfqpoint{2.354191in}{5.213623in}}%
\pgfpathlineto{\pgfqpoint{2.372629in}{5.213014in}}%
\pgfpathlineto{\pgfqpoint{2.391066in}{5.219926in}}%
\pgfpathlineto{\pgfqpoint{2.409503in}{5.218455in}}%
\pgfpathlineto{\pgfqpoint{2.446377in}{5.220379in}}%
\pgfpathlineto{\pgfqpoint{2.464814in}{5.226854in}}%
\pgfpathlineto{\pgfqpoint{2.483251in}{5.224133in}}%
\pgfpathlineto{\pgfqpoint{2.501688in}{5.232784in}}%
\pgfpathlineto{\pgfqpoint{2.520126in}{5.229949in}}%
\pgfpathlineto{\pgfqpoint{2.538563in}{5.231992in}}%
\pgfpathlineto{\pgfqpoint{2.557000in}{5.237664in}}%
\pgfpathlineto{\pgfqpoint{2.575437in}{5.234969in}}%
\pgfpathlineto{\pgfqpoint{2.593874in}{5.239746in}}%
\pgfpathlineto{\pgfqpoint{2.612311in}{5.240038in}}%
\pgfpathlineto{\pgfqpoint{2.630748in}{5.242174in}}%
\pgfpathlineto{\pgfqpoint{2.649185in}{5.242285in}}%
\pgfpathlineto{\pgfqpoint{2.667622in}{5.251311in}}%
\pgfpathlineto{\pgfqpoint{2.686060in}{5.247006in}}%
\pgfpathlineto{\pgfqpoint{2.722934in}{5.259344in}}%
\pgfpathlineto{\pgfqpoint{2.741371in}{5.257880in}}%
\pgfpathlineto{\pgfqpoint{2.759808in}{5.260747in}}%
\pgfpathlineto{\pgfqpoint{2.778245in}{5.261954in}}%
\pgfpathlineto{\pgfqpoint{2.796682in}{5.267211in}}%
\pgfpathlineto{\pgfqpoint{2.815119in}{5.266945in}}%
\pgfpathlineto{\pgfqpoint{2.851994in}{5.274513in}}%
\pgfpathlineto{\pgfqpoint{2.870431in}{5.273499in}}%
\pgfpathlineto{\pgfqpoint{2.888868in}{5.277256in}}%
\pgfpathlineto{\pgfqpoint{2.907305in}{5.275912in}}%
\pgfpathlineto{\pgfqpoint{2.925742in}{5.285072in}}%
\pgfpathlineto{\pgfqpoint{2.944179in}{5.283601in}}%
\pgfpathlineto{\pgfqpoint{2.962616in}{5.286248in}}%
\pgfpathlineto{\pgfqpoint{2.981053in}{5.293114in}}%
\pgfpathlineto{\pgfqpoint{3.017928in}{5.297791in}}%
\pgfpathlineto{\pgfqpoint{3.036365in}{5.296593in}}%
\pgfpathlineto{\pgfqpoint{3.054802in}{5.296948in}}%
\pgfpathlineto{\pgfqpoint{3.073239in}{5.302986in}}%
\pgfpathlineto{\pgfqpoint{3.110113in}{5.310057in}}%
\pgfpathlineto{\pgfqpoint{3.128550in}{5.309181in}}%
\pgfpathlineto{\pgfqpoint{3.146987in}{5.314532in}}%
\pgfpathlineto{\pgfqpoint{3.165425in}{5.311821in}}%
\pgfpathlineto{\pgfqpoint{3.183862in}{5.314296in}}%
\pgfpathlineto{\pgfqpoint{3.202299in}{5.325170in}}%
\pgfpathlineto{\pgfqpoint{3.220736in}{5.327168in}}%
\pgfpathlineto{\pgfqpoint{3.239173in}{5.323804in}}%
\pgfpathlineto{\pgfqpoint{3.257610in}{5.330690in}}%
\pgfpathlineto{\pgfqpoint{3.276047in}{5.328706in}}%
\pgfpathlineto{\pgfqpoint{3.294484in}{5.334497in}}%
\pgfpathlineto{\pgfqpoint{3.312922in}{5.336104in}}%
\pgfpathlineto{\pgfqpoint{3.349796in}{5.345858in}}%
\pgfpathlineto{\pgfqpoint{3.368233in}{5.350219in}}%
\pgfpathlineto{\pgfqpoint{3.405107in}{5.354017in}}%
\pgfpathlineto{\pgfqpoint{3.423544in}{5.359901in}}%
\pgfpathlineto{\pgfqpoint{3.441981in}{5.356841in}}%
\pgfpathlineto{\pgfqpoint{3.460418in}{5.359032in}}%
\pgfpathlineto{\pgfqpoint{3.478856in}{5.364931in}}%
\pgfpathlineto{\pgfqpoint{3.497293in}{5.372495in}}%
\pgfpathlineto{\pgfqpoint{3.515730in}{5.371485in}}%
\pgfpathlineto{\pgfqpoint{3.534167in}{5.376429in}}%
\pgfpathlineto{\pgfqpoint{3.552604in}{5.374035in}}%
\pgfpathlineto{\pgfqpoint{3.571041in}{5.385004in}}%
\pgfpathlineto{\pgfqpoint{3.644790in}{5.394709in}}%
\pgfpathlineto{\pgfqpoint{3.663227in}{5.400824in}}%
\pgfpathlineto{\pgfqpoint{3.681664in}{5.401708in}}%
\pgfpathlineto{\pgfqpoint{3.700101in}{5.407314in}}%
\pgfpathlineto{\pgfqpoint{3.718538in}{5.404671in}}%
\pgfpathlineto{\pgfqpoint{3.736975in}{5.412994in}}%
\pgfpathlineto{\pgfqpoint{3.755412in}{5.415773in}}%
\pgfpathlineto{\pgfqpoint{3.773849in}{5.416117in}}%
\pgfpathlineto{\pgfqpoint{3.792287in}{5.424201in}}%
\pgfpathlineto{\pgfqpoint{3.810724in}{5.427195in}}%
\pgfpathlineto{\pgfqpoint{3.829161in}{5.426484in}}%
\pgfpathlineto{\pgfqpoint{3.884472in}{5.442284in}}%
\pgfpathlineto{\pgfqpoint{3.902909in}{5.446593in}}%
\pgfpathlineto{\pgfqpoint{3.921346in}{5.443875in}}%
\pgfpathlineto{\pgfqpoint{3.939783in}{5.449414in}}%
\pgfpathlineto{\pgfqpoint{3.958221in}{5.450307in}}%
\pgfpathlineto{\pgfqpoint{3.976658in}{5.461116in}}%
\pgfpathlineto{\pgfqpoint{4.068843in}{5.477678in}}%
\pgfpathlineto{\pgfqpoint{4.087280in}{5.483403in}}%
\pgfpathlineto{\pgfqpoint{4.105718in}{5.479974in}}%
\pgfpathlineto{\pgfqpoint{4.142592in}{5.488435in}}%
\pgfpathlineto{\pgfqpoint{4.161029in}{5.498685in}}%
\pgfpathlineto{\pgfqpoint{4.179466in}{5.501094in}}%
\pgfpathlineto{\pgfqpoint{4.197903in}{5.499308in}}%
\pgfpathlineto{\pgfqpoint{4.234777in}{5.511187in}}%
\pgfpathlineto{\pgfqpoint{4.253214in}{5.510101in}}%
\pgfpathlineto{\pgfqpoint{4.271652in}{5.519752in}}%
\pgfpathlineto{\pgfqpoint{4.308526in}{5.522902in}}%
\pgfpathlineto{\pgfqpoint{4.345400in}{5.534734in}}%
\pgfpathlineto{\pgfqpoint{4.363837in}{5.539041in}}%
\pgfpathlineto{\pgfqpoint{4.382274in}{5.538390in}}%
\pgfpathlineto{\pgfqpoint{4.437586in}{5.552190in}}%
\pgfpathlineto{\pgfqpoint{4.474460in}{5.560513in}}%
\pgfpathlineto{\pgfqpoint{4.492897in}{5.569445in}}%
\pgfpathlineto{\pgfqpoint{4.511334in}{5.570108in}}%
\pgfpathlineto{\pgfqpoint{4.529771in}{5.580004in}}%
\pgfpathlineto{\pgfqpoint{4.548208in}{5.581232in}}%
\pgfpathlineto{\pgfqpoint{4.566645in}{5.587075in}}%
\pgfpathlineto{\pgfqpoint{4.585083in}{5.590381in}}%
\pgfpathlineto{\pgfqpoint{4.603520in}{5.596074in}}%
\pgfpathlineto{\pgfqpoint{4.621957in}{5.594206in}}%
\pgfpathlineto{\pgfqpoint{4.640394in}{5.598874in}}%
\pgfpathlineto{\pgfqpoint{4.658831in}{5.606560in}}%
\pgfpathlineto{\pgfqpoint{4.677268in}{5.608101in}}%
\pgfpathlineto{\pgfqpoint{4.695705in}{5.613041in}}%
\pgfpathlineto{\pgfqpoint{4.714142in}{5.616174in}}%
\pgfpathlineto{\pgfqpoint{4.751017in}{5.633028in}}%
\pgfpathlineto{\pgfqpoint{4.769454in}{5.630883in}}%
\pgfpathlineto{\pgfqpoint{4.787891in}{5.642012in}}%
\pgfpathlineto{\pgfqpoint{4.806328in}{5.643826in}}%
\pgfpathlineto{\pgfqpoint{4.824765in}{5.650631in}}%
\pgfpathlineto{\pgfqpoint{4.843202in}{5.647796in}}%
\pgfpathlineto{\pgfqpoint{4.861639in}{5.652777in}}%
\pgfpathlineto{\pgfqpoint{4.898514in}{5.669150in}}%
\pgfpathlineto{\pgfqpoint{4.916951in}{5.667934in}}%
\pgfpathlineto{\pgfqpoint{4.953825in}{5.678145in}}%
\pgfpathlineto{\pgfqpoint{4.972262in}{5.688960in}}%
\pgfpathlineto{\pgfqpoint{4.990699in}{5.690217in}}%
\pgfpathlineto{\pgfqpoint{5.009136in}{5.698533in}}%
\pgfpathlineto{\pgfqpoint{5.027573in}{5.703117in}}%
\pgfpathlineto{\pgfqpoint{5.046010in}{5.703875in}}%
\pgfpathlineto{\pgfqpoint{5.064448in}{5.710961in}}%
\pgfpathlineto{\pgfqpoint{5.082885in}{5.711510in}}%
\pgfpathlineto{\pgfqpoint{5.101322in}{5.718166in}}%
\pgfpathlineto{\pgfqpoint{5.119759in}{5.721677in}}%
\pgfpathlineto{\pgfqpoint{5.138196in}{5.731899in}}%
\pgfpathlineto{\pgfqpoint{5.156633in}{5.734690in}}%
\pgfpathlineto{\pgfqpoint{5.175070in}{5.742834in}}%
\pgfpathlineto{\pgfqpoint{5.211944in}{5.749612in}}%
\pgfpathlineto{\pgfqpoint{5.248819in}{5.760969in}}%
\pgfpathlineto{\pgfqpoint{5.267256in}{5.761675in}}%
\pgfpathlineto{\pgfqpoint{5.285693in}{5.773123in}}%
\pgfpathlineto{\pgfqpoint{5.322567in}{5.783133in}}%
\pgfpathlineto{\pgfqpoint{5.341004in}{5.781595in}}%
\pgfpathlineto{\pgfqpoint{5.359441in}{5.791653in}}%
\pgfpathlineto{\pgfqpoint{5.377879in}{5.791447in}}%
\pgfpathlineto{\pgfqpoint{5.396316in}{5.797870in}}%
\pgfpathlineto{\pgfqpoint{5.396316in}{5.797870in}}%
\pgfusepath{stroke}%
\end{pgfscope}%
\begin{pgfscope}%
\pgfpathrectangle{\pgfqpoint{0.459778in}{5.093981in}}{\pgfqpoint{5.171611in}{0.737408in}}%
\pgfusepath{clip}%
\pgfsetbuttcap%
\pgfsetroundjoin%
\definecolor{currentfill}{rgb}{0.121569,0.466667,0.705882}%
\pgfsetfillcolor{currentfill}%
\pgfsetlinewidth{1.003750pt}%
\definecolor{currentstroke}{rgb}{0.121569,0.466667,0.705882}%
\pgfsetstrokecolor{currentstroke}%
\pgfsetdash{}{0pt}%
\pgfsys@defobject{currentmarker}{\pgfqpoint{-0.041667in}{-0.041667in}}{\pgfqpoint{0.041667in}{0.041667in}}{%
\pgfpathmoveto{\pgfqpoint{0.000000in}{-0.041667in}}%
\pgfpathcurveto{\pgfqpoint{0.011050in}{-0.041667in}}{\pgfqpoint{0.021649in}{-0.037276in}}{\pgfqpoint{0.029463in}{-0.029463in}}%
\pgfpathcurveto{\pgfqpoint{0.037276in}{-0.021649in}}{\pgfqpoint{0.041667in}{-0.011050in}}{\pgfqpoint{0.041667in}{0.000000in}}%
\pgfpathcurveto{\pgfqpoint{0.041667in}{0.011050in}}{\pgfqpoint{0.037276in}{0.021649in}}{\pgfqpoint{0.029463in}{0.029463in}}%
\pgfpathcurveto{\pgfqpoint{0.021649in}{0.037276in}}{\pgfqpoint{0.011050in}{0.041667in}}{\pgfqpoint{0.000000in}{0.041667in}}%
\pgfpathcurveto{\pgfqpoint{-0.011050in}{0.041667in}}{\pgfqpoint{-0.021649in}{0.037276in}}{\pgfqpoint{-0.029463in}{0.029463in}}%
\pgfpathcurveto{\pgfqpoint{-0.037276in}{0.021649in}}{\pgfqpoint{-0.041667in}{0.011050in}}{\pgfqpoint{-0.041667in}{0.000000in}}%
\pgfpathcurveto{\pgfqpoint{-0.041667in}{-0.011050in}}{\pgfqpoint{-0.037276in}{-0.021649in}}{\pgfqpoint{-0.029463in}{-0.029463in}}%
\pgfpathcurveto{\pgfqpoint{-0.021649in}{-0.037276in}}{\pgfqpoint{-0.011050in}{-0.041667in}}{\pgfqpoint{0.000000in}{-0.041667in}}%
\pgfpathclose%
\pgfusepath{stroke,fill}%
}%
\begin{pgfscope}%
\pgfsys@transformshift{0.694851in}{5.128545in}%
\pgfsys@useobject{currentmarker}{}%
\end{pgfscope}%
\begin{pgfscope}%
\pgfsys@transformshift{0.713288in}{5.132698in}%
\pgfsys@useobject{currentmarker}{}%
\end{pgfscope}%
\begin{pgfscope}%
\pgfsys@transformshift{0.731725in}{5.131395in}%
\pgfsys@useobject{currentmarker}{}%
\end{pgfscope}%
\begin{pgfscope}%
\pgfsys@transformshift{0.750162in}{5.130133in}%
\pgfsys@useobject{currentmarker}{}%
\end{pgfscope}%
\begin{pgfscope}%
\pgfsys@transformshift{0.768599in}{5.128634in}%
\pgfsys@useobject{currentmarker}{}%
\end{pgfscope}%
\begin{pgfscope}%
\pgfsys@transformshift{0.787037in}{5.132451in}%
\pgfsys@useobject{currentmarker}{}%
\end{pgfscope}%
\begin{pgfscope}%
\pgfsys@transformshift{0.805474in}{5.131221in}%
\pgfsys@useobject{currentmarker}{}%
\end{pgfscope}%
\begin{pgfscope}%
\pgfsys@transformshift{0.823911in}{5.127830in}%
\pgfsys@useobject{currentmarker}{}%
\end{pgfscope}%
\begin{pgfscope}%
\pgfsys@transformshift{0.842348in}{5.127500in}%
\pgfsys@useobject{currentmarker}{}%
\end{pgfscope}%
\begin{pgfscope}%
\pgfsys@transformshift{0.860785in}{5.132423in}%
\pgfsys@useobject{currentmarker}{}%
\end{pgfscope}%
\begin{pgfscope}%
\pgfsys@transformshift{0.879222in}{5.128807in}%
\pgfsys@useobject{currentmarker}{}%
\end{pgfscope}%
\begin{pgfscope}%
\pgfsys@transformshift{0.897659in}{5.127858in}%
\pgfsys@useobject{currentmarker}{}%
\end{pgfscope}%
\begin{pgfscope}%
\pgfsys@transformshift{0.916096in}{5.129863in}%
\pgfsys@useobject{currentmarker}{}%
\end{pgfscope}%
\begin{pgfscope}%
\pgfsys@transformshift{0.934534in}{5.130499in}%
\pgfsys@useobject{currentmarker}{}%
\end{pgfscope}%
\begin{pgfscope}%
\pgfsys@transformshift{0.952971in}{5.133625in}%
\pgfsys@useobject{currentmarker}{}%
\end{pgfscope}%
\begin{pgfscope}%
\pgfsys@transformshift{0.971408in}{5.128956in}%
\pgfsys@useobject{currentmarker}{}%
\end{pgfscope}%
\begin{pgfscope}%
\pgfsys@transformshift{0.989845in}{5.135915in}%
\pgfsys@useobject{currentmarker}{}%
\end{pgfscope}%
\begin{pgfscope}%
\pgfsys@transformshift{1.008282in}{5.135525in}%
\pgfsys@useobject{currentmarker}{}%
\end{pgfscope}%
\begin{pgfscope}%
\pgfsys@transformshift{1.026719in}{5.136105in}%
\pgfsys@useobject{currentmarker}{}%
\end{pgfscope}%
\begin{pgfscope}%
\pgfsys@transformshift{1.045156in}{5.133987in}%
\pgfsys@useobject{currentmarker}{}%
\end{pgfscope}%
\begin{pgfscope}%
\pgfsys@transformshift{1.063593in}{5.134327in}%
\pgfsys@useobject{currentmarker}{}%
\end{pgfscope}%
\begin{pgfscope}%
\pgfsys@transformshift{1.082030in}{5.133881in}%
\pgfsys@useobject{currentmarker}{}%
\end{pgfscope}%
\begin{pgfscope}%
\pgfsys@transformshift{1.100468in}{5.133197in}%
\pgfsys@useobject{currentmarker}{}%
\end{pgfscope}%
\begin{pgfscope}%
\pgfsys@transformshift{1.118905in}{5.136854in}%
\pgfsys@useobject{currentmarker}{}%
\end{pgfscope}%
\begin{pgfscope}%
\pgfsys@transformshift{1.137342in}{5.136481in}%
\pgfsys@useobject{currentmarker}{}%
\end{pgfscope}%
\begin{pgfscope}%
\pgfsys@transformshift{1.155779in}{5.139927in}%
\pgfsys@useobject{currentmarker}{}%
\end{pgfscope}%
\begin{pgfscope}%
\pgfsys@transformshift{1.174216in}{5.139647in}%
\pgfsys@useobject{currentmarker}{}%
\end{pgfscope}%
\begin{pgfscope}%
\pgfsys@transformshift{1.192653in}{5.134119in}%
\pgfsys@useobject{currentmarker}{}%
\end{pgfscope}%
\begin{pgfscope}%
\pgfsys@transformshift{1.211090in}{5.135351in}%
\pgfsys@useobject{currentmarker}{}%
\end{pgfscope}%
\begin{pgfscope}%
\pgfsys@transformshift{1.229527in}{5.141849in}%
\pgfsys@useobject{currentmarker}{}%
\end{pgfscope}%
\begin{pgfscope}%
\pgfsys@transformshift{1.247964in}{5.143237in}%
\pgfsys@useobject{currentmarker}{}%
\end{pgfscope}%
\begin{pgfscope}%
\pgfsys@transformshift{1.266402in}{5.143539in}%
\pgfsys@useobject{currentmarker}{}%
\end{pgfscope}%
\begin{pgfscope}%
\pgfsys@transformshift{1.284839in}{5.142386in}%
\pgfsys@useobject{currentmarker}{}%
\end{pgfscope}%
\begin{pgfscope}%
\pgfsys@transformshift{1.303276in}{5.145754in}%
\pgfsys@useobject{currentmarker}{}%
\end{pgfscope}%
\begin{pgfscope}%
\pgfsys@transformshift{1.321713in}{5.145142in}%
\pgfsys@useobject{currentmarker}{}%
\end{pgfscope}%
\begin{pgfscope}%
\pgfsys@transformshift{1.340150in}{5.142681in}%
\pgfsys@useobject{currentmarker}{}%
\end{pgfscope}%
\begin{pgfscope}%
\pgfsys@transformshift{1.358587in}{5.145781in}%
\pgfsys@useobject{currentmarker}{}%
\end{pgfscope}%
\begin{pgfscope}%
\pgfsys@transformshift{1.377024in}{5.141417in}%
\pgfsys@useobject{currentmarker}{}%
\end{pgfscope}%
\begin{pgfscope}%
\pgfsys@transformshift{1.395461in}{5.146149in}%
\pgfsys@useobject{currentmarker}{}%
\end{pgfscope}%
\begin{pgfscope}%
\pgfsys@transformshift{1.413899in}{5.150057in}%
\pgfsys@useobject{currentmarker}{}%
\end{pgfscope}%
\begin{pgfscope}%
\pgfsys@transformshift{1.432336in}{5.146163in}%
\pgfsys@useobject{currentmarker}{}%
\end{pgfscope}%
\begin{pgfscope}%
\pgfsys@transformshift{1.450773in}{5.150069in}%
\pgfsys@useobject{currentmarker}{}%
\end{pgfscope}%
\begin{pgfscope}%
\pgfsys@transformshift{1.469210in}{5.149325in}%
\pgfsys@useobject{currentmarker}{}%
\end{pgfscope}%
\begin{pgfscope}%
\pgfsys@transformshift{1.487647in}{5.145869in}%
\pgfsys@useobject{currentmarker}{}%
\end{pgfscope}%
\begin{pgfscope}%
\pgfsys@transformshift{1.506084in}{5.147677in}%
\pgfsys@useobject{currentmarker}{}%
\end{pgfscope}%
\begin{pgfscope}%
\pgfsys@transformshift{1.524521in}{5.154141in}%
\pgfsys@useobject{currentmarker}{}%
\end{pgfscope}%
\begin{pgfscope}%
\pgfsys@transformshift{1.542958in}{5.155704in}%
\pgfsys@useobject{currentmarker}{}%
\end{pgfscope}%
\begin{pgfscope}%
\pgfsys@transformshift{1.561395in}{5.151541in}%
\pgfsys@useobject{currentmarker}{}%
\end{pgfscope}%
\begin{pgfscope}%
\pgfsys@transformshift{1.579833in}{5.153681in}%
\pgfsys@useobject{currentmarker}{}%
\end{pgfscope}%
\begin{pgfscope}%
\pgfsys@transformshift{1.598270in}{5.154314in}%
\pgfsys@useobject{currentmarker}{}%
\end{pgfscope}%
\begin{pgfscope}%
\pgfsys@transformshift{1.616707in}{5.156761in}%
\pgfsys@useobject{currentmarker}{}%
\end{pgfscope}%
\begin{pgfscope}%
\pgfsys@transformshift{1.635144in}{5.156061in}%
\pgfsys@useobject{currentmarker}{}%
\end{pgfscope}%
\begin{pgfscope}%
\pgfsys@transformshift{1.653581in}{5.161645in}%
\pgfsys@useobject{currentmarker}{}%
\end{pgfscope}%
\begin{pgfscope}%
\pgfsys@transformshift{1.672018in}{5.162341in}%
\pgfsys@useobject{currentmarker}{}%
\end{pgfscope}%
\begin{pgfscope}%
\pgfsys@transformshift{1.690455in}{5.159816in}%
\pgfsys@useobject{currentmarker}{}%
\end{pgfscope}%
\begin{pgfscope}%
\pgfsys@transformshift{1.708892in}{5.158351in}%
\pgfsys@useobject{currentmarker}{}%
\end{pgfscope}%
\begin{pgfscope}%
\pgfsys@transformshift{1.727330in}{5.163903in}%
\pgfsys@useobject{currentmarker}{}%
\end{pgfscope}%
\begin{pgfscope}%
\pgfsys@transformshift{1.745767in}{5.167929in}%
\pgfsys@useobject{currentmarker}{}%
\end{pgfscope}%
\begin{pgfscope}%
\pgfsys@transformshift{1.764204in}{5.161382in}%
\pgfsys@useobject{currentmarker}{}%
\end{pgfscope}%
\begin{pgfscope}%
\pgfsys@transformshift{1.782641in}{5.165994in}%
\pgfsys@useobject{currentmarker}{}%
\end{pgfscope}%
\begin{pgfscope}%
\pgfsys@transformshift{1.801078in}{5.166972in}%
\pgfsys@useobject{currentmarker}{}%
\end{pgfscope}%
\begin{pgfscope}%
\pgfsys@transformshift{1.819515in}{5.171008in}%
\pgfsys@useobject{currentmarker}{}%
\end{pgfscope}%
\begin{pgfscope}%
\pgfsys@transformshift{1.837952in}{5.170195in}%
\pgfsys@useobject{currentmarker}{}%
\end{pgfscope}%
\begin{pgfscope}%
\pgfsys@transformshift{1.856389in}{5.174773in}%
\pgfsys@useobject{currentmarker}{}%
\end{pgfscope}%
\begin{pgfscope}%
\pgfsys@transformshift{1.874826in}{5.176342in}%
\pgfsys@useobject{currentmarker}{}%
\end{pgfscope}%
\begin{pgfscope}%
\pgfsys@transformshift{1.893264in}{5.172002in}%
\pgfsys@useobject{currentmarker}{}%
\end{pgfscope}%
\begin{pgfscope}%
\pgfsys@transformshift{1.911701in}{5.174476in}%
\pgfsys@useobject{currentmarker}{}%
\end{pgfscope}%
\begin{pgfscope}%
\pgfsys@transformshift{1.930138in}{5.174486in}%
\pgfsys@useobject{currentmarker}{}%
\end{pgfscope}%
\begin{pgfscope}%
\pgfsys@transformshift{1.948575in}{5.174354in}%
\pgfsys@useobject{currentmarker}{}%
\end{pgfscope}%
\begin{pgfscope}%
\pgfsys@transformshift{1.967012in}{5.183374in}%
\pgfsys@useobject{currentmarker}{}%
\end{pgfscope}%
\begin{pgfscope}%
\pgfsys@transformshift{1.985449in}{5.180593in}%
\pgfsys@useobject{currentmarker}{}%
\end{pgfscope}%
\begin{pgfscope}%
\pgfsys@transformshift{2.003886in}{5.186032in}%
\pgfsys@useobject{currentmarker}{}%
\end{pgfscope}%
\begin{pgfscope}%
\pgfsys@transformshift{2.022323in}{5.184009in}%
\pgfsys@useobject{currentmarker}{}%
\end{pgfscope}%
\begin{pgfscope}%
\pgfsys@transformshift{2.040760in}{5.185916in}%
\pgfsys@useobject{currentmarker}{}%
\end{pgfscope}%
\begin{pgfscope}%
\pgfsys@transformshift{2.059198in}{5.184339in}%
\pgfsys@useobject{currentmarker}{}%
\end{pgfscope}%
\begin{pgfscope}%
\pgfsys@transformshift{2.077635in}{5.190452in}%
\pgfsys@useobject{currentmarker}{}%
\end{pgfscope}%
\begin{pgfscope}%
\pgfsys@transformshift{2.096072in}{5.191709in}%
\pgfsys@useobject{currentmarker}{}%
\end{pgfscope}%
\begin{pgfscope}%
\pgfsys@transformshift{2.114509in}{5.193571in}%
\pgfsys@useobject{currentmarker}{}%
\end{pgfscope}%
\begin{pgfscope}%
\pgfsys@transformshift{2.132946in}{5.193100in}%
\pgfsys@useobject{currentmarker}{}%
\end{pgfscope}%
\begin{pgfscope}%
\pgfsys@transformshift{2.151383in}{5.198759in}%
\pgfsys@useobject{currentmarker}{}%
\end{pgfscope}%
\begin{pgfscope}%
\pgfsys@transformshift{2.169820in}{5.198162in}%
\pgfsys@useobject{currentmarker}{}%
\end{pgfscope}%
\begin{pgfscope}%
\pgfsys@transformshift{2.188257in}{5.196107in}%
\pgfsys@useobject{currentmarker}{}%
\end{pgfscope}%
\begin{pgfscope}%
\pgfsys@transformshift{2.206695in}{5.203092in}%
\pgfsys@useobject{currentmarker}{}%
\end{pgfscope}%
\begin{pgfscope}%
\pgfsys@transformshift{2.225132in}{5.200804in}%
\pgfsys@useobject{currentmarker}{}%
\end{pgfscope}%
\begin{pgfscope}%
\pgfsys@transformshift{2.243569in}{5.201247in}%
\pgfsys@useobject{currentmarker}{}%
\end{pgfscope}%
\begin{pgfscope}%
\pgfsys@transformshift{2.262006in}{5.203432in}%
\pgfsys@useobject{currentmarker}{}%
\end{pgfscope}%
\begin{pgfscope}%
\pgfsys@transformshift{2.280443in}{5.205623in}%
\pgfsys@useobject{currentmarker}{}%
\end{pgfscope}%
\begin{pgfscope}%
\pgfsys@transformshift{2.298880in}{5.209554in}%
\pgfsys@useobject{currentmarker}{}%
\end{pgfscope}%
\begin{pgfscope}%
\pgfsys@transformshift{2.317317in}{5.211286in}%
\pgfsys@useobject{currentmarker}{}%
\end{pgfscope}%
\begin{pgfscope}%
\pgfsys@transformshift{2.335754in}{5.209707in}%
\pgfsys@useobject{currentmarker}{}%
\end{pgfscope}%
\begin{pgfscope}%
\pgfsys@transformshift{2.354191in}{5.213623in}%
\pgfsys@useobject{currentmarker}{}%
\end{pgfscope}%
\begin{pgfscope}%
\pgfsys@transformshift{2.372629in}{5.213014in}%
\pgfsys@useobject{currentmarker}{}%
\end{pgfscope}%
\begin{pgfscope}%
\pgfsys@transformshift{2.391066in}{5.219926in}%
\pgfsys@useobject{currentmarker}{}%
\end{pgfscope}%
\begin{pgfscope}%
\pgfsys@transformshift{2.409503in}{5.218455in}%
\pgfsys@useobject{currentmarker}{}%
\end{pgfscope}%
\begin{pgfscope}%
\pgfsys@transformshift{2.427940in}{5.220184in}%
\pgfsys@useobject{currentmarker}{}%
\end{pgfscope}%
\begin{pgfscope}%
\pgfsys@transformshift{2.446377in}{5.220379in}%
\pgfsys@useobject{currentmarker}{}%
\end{pgfscope}%
\begin{pgfscope}%
\pgfsys@transformshift{2.464814in}{5.226854in}%
\pgfsys@useobject{currentmarker}{}%
\end{pgfscope}%
\begin{pgfscope}%
\pgfsys@transformshift{2.483251in}{5.224133in}%
\pgfsys@useobject{currentmarker}{}%
\end{pgfscope}%
\begin{pgfscope}%
\pgfsys@transformshift{2.501688in}{5.232784in}%
\pgfsys@useobject{currentmarker}{}%
\end{pgfscope}%
\begin{pgfscope}%
\pgfsys@transformshift{2.520126in}{5.229949in}%
\pgfsys@useobject{currentmarker}{}%
\end{pgfscope}%
\begin{pgfscope}%
\pgfsys@transformshift{2.538563in}{5.231992in}%
\pgfsys@useobject{currentmarker}{}%
\end{pgfscope}%
\begin{pgfscope}%
\pgfsys@transformshift{2.557000in}{5.237664in}%
\pgfsys@useobject{currentmarker}{}%
\end{pgfscope}%
\begin{pgfscope}%
\pgfsys@transformshift{2.575437in}{5.234969in}%
\pgfsys@useobject{currentmarker}{}%
\end{pgfscope}%
\begin{pgfscope}%
\pgfsys@transformshift{2.593874in}{5.239746in}%
\pgfsys@useobject{currentmarker}{}%
\end{pgfscope}%
\begin{pgfscope}%
\pgfsys@transformshift{2.612311in}{5.240038in}%
\pgfsys@useobject{currentmarker}{}%
\end{pgfscope}%
\begin{pgfscope}%
\pgfsys@transformshift{2.630748in}{5.242174in}%
\pgfsys@useobject{currentmarker}{}%
\end{pgfscope}%
\begin{pgfscope}%
\pgfsys@transformshift{2.649185in}{5.242285in}%
\pgfsys@useobject{currentmarker}{}%
\end{pgfscope}%
\begin{pgfscope}%
\pgfsys@transformshift{2.667622in}{5.251311in}%
\pgfsys@useobject{currentmarker}{}%
\end{pgfscope}%
\begin{pgfscope}%
\pgfsys@transformshift{2.686060in}{5.247006in}%
\pgfsys@useobject{currentmarker}{}%
\end{pgfscope}%
\begin{pgfscope}%
\pgfsys@transformshift{2.704497in}{5.253077in}%
\pgfsys@useobject{currentmarker}{}%
\end{pgfscope}%
\begin{pgfscope}%
\pgfsys@transformshift{2.722934in}{5.259344in}%
\pgfsys@useobject{currentmarker}{}%
\end{pgfscope}%
\begin{pgfscope}%
\pgfsys@transformshift{2.741371in}{5.257880in}%
\pgfsys@useobject{currentmarker}{}%
\end{pgfscope}%
\begin{pgfscope}%
\pgfsys@transformshift{2.759808in}{5.260747in}%
\pgfsys@useobject{currentmarker}{}%
\end{pgfscope}%
\begin{pgfscope}%
\pgfsys@transformshift{2.778245in}{5.261954in}%
\pgfsys@useobject{currentmarker}{}%
\end{pgfscope}%
\begin{pgfscope}%
\pgfsys@transformshift{2.796682in}{5.267211in}%
\pgfsys@useobject{currentmarker}{}%
\end{pgfscope}%
\begin{pgfscope}%
\pgfsys@transformshift{2.815119in}{5.266945in}%
\pgfsys@useobject{currentmarker}{}%
\end{pgfscope}%
\begin{pgfscope}%
\pgfsys@transformshift{2.833556in}{5.270419in}%
\pgfsys@useobject{currentmarker}{}%
\end{pgfscope}%
\begin{pgfscope}%
\pgfsys@transformshift{2.851994in}{5.274513in}%
\pgfsys@useobject{currentmarker}{}%
\end{pgfscope}%
\begin{pgfscope}%
\pgfsys@transformshift{2.870431in}{5.273499in}%
\pgfsys@useobject{currentmarker}{}%
\end{pgfscope}%
\begin{pgfscope}%
\pgfsys@transformshift{2.888868in}{5.277256in}%
\pgfsys@useobject{currentmarker}{}%
\end{pgfscope}%
\begin{pgfscope}%
\pgfsys@transformshift{2.907305in}{5.275912in}%
\pgfsys@useobject{currentmarker}{}%
\end{pgfscope}%
\begin{pgfscope}%
\pgfsys@transformshift{2.925742in}{5.285072in}%
\pgfsys@useobject{currentmarker}{}%
\end{pgfscope}%
\begin{pgfscope}%
\pgfsys@transformshift{2.944179in}{5.283601in}%
\pgfsys@useobject{currentmarker}{}%
\end{pgfscope}%
\begin{pgfscope}%
\pgfsys@transformshift{2.962616in}{5.286248in}%
\pgfsys@useobject{currentmarker}{}%
\end{pgfscope}%
\begin{pgfscope}%
\pgfsys@transformshift{2.981053in}{5.293114in}%
\pgfsys@useobject{currentmarker}{}%
\end{pgfscope}%
\begin{pgfscope}%
\pgfsys@transformshift{2.999491in}{5.295644in}%
\pgfsys@useobject{currentmarker}{}%
\end{pgfscope}%
\begin{pgfscope}%
\pgfsys@transformshift{3.017928in}{5.297791in}%
\pgfsys@useobject{currentmarker}{}%
\end{pgfscope}%
\begin{pgfscope}%
\pgfsys@transformshift{3.036365in}{5.296593in}%
\pgfsys@useobject{currentmarker}{}%
\end{pgfscope}%
\begin{pgfscope}%
\pgfsys@transformshift{3.054802in}{5.296948in}%
\pgfsys@useobject{currentmarker}{}%
\end{pgfscope}%
\begin{pgfscope}%
\pgfsys@transformshift{3.073239in}{5.302986in}%
\pgfsys@useobject{currentmarker}{}%
\end{pgfscope}%
\begin{pgfscope}%
\pgfsys@transformshift{3.091676in}{5.307084in}%
\pgfsys@useobject{currentmarker}{}%
\end{pgfscope}%
\begin{pgfscope}%
\pgfsys@transformshift{3.110113in}{5.310057in}%
\pgfsys@useobject{currentmarker}{}%
\end{pgfscope}%
\begin{pgfscope}%
\pgfsys@transformshift{3.128550in}{5.309181in}%
\pgfsys@useobject{currentmarker}{}%
\end{pgfscope}%
\begin{pgfscope}%
\pgfsys@transformshift{3.146987in}{5.314532in}%
\pgfsys@useobject{currentmarker}{}%
\end{pgfscope}%
\begin{pgfscope}%
\pgfsys@transformshift{3.165425in}{5.311821in}%
\pgfsys@useobject{currentmarker}{}%
\end{pgfscope}%
\begin{pgfscope}%
\pgfsys@transformshift{3.183862in}{5.314296in}%
\pgfsys@useobject{currentmarker}{}%
\end{pgfscope}%
\begin{pgfscope}%
\pgfsys@transformshift{3.202299in}{5.325170in}%
\pgfsys@useobject{currentmarker}{}%
\end{pgfscope}%
\begin{pgfscope}%
\pgfsys@transformshift{3.220736in}{5.327168in}%
\pgfsys@useobject{currentmarker}{}%
\end{pgfscope}%
\begin{pgfscope}%
\pgfsys@transformshift{3.239173in}{5.323804in}%
\pgfsys@useobject{currentmarker}{}%
\end{pgfscope}%
\begin{pgfscope}%
\pgfsys@transformshift{3.257610in}{5.330690in}%
\pgfsys@useobject{currentmarker}{}%
\end{pgfscope}%
\begin{pgfscope}%
\pgfsys@transformshift{3.276047in}{5.328706in}%
\pgfsys@useobject{currentmarker}{}%
\end{pgfscope}%
\begin{pgfscope}%
\pgfsys@transformshift{3.294484in}{5.334497in}%
\pgfsys@useobject{currentmarker}{}%
\end{pgfscope}%
\begin{pgfscope}%
\pgfsys@transformshift{3.312922in}{5.336104in}%
\pgfsys@useobject{currentmarker}{}%
\end{pgfscope}%
\begin{pgfscope}%
\pgfsys@transformshift{3.331359in}{5.340262in}%
\pgfsys@useobject{currentmarker}{}%
\end{pgfscope}%
\begin{pgfscope}%
\pgfsys@transformshift{3.349796in}{5.345858in}%
\pgfsys@useobject{currentmarker}{}%
\end{pgfscope}%
\begin{pgfscope}%
\pgfsys@transformshift{3.368233in}{5.350219in}%
\pgfsys@useobject{currentmarker}{}%
\end{pgfscope}%
\begin{pgfscope}%
\pgfsys@transformshift{3.386670in}{5.352601in}%
\pgfsys@useobject{currentmarker}{}%
\end{pgfscope}%
\begin{pgfscope}%
\pgfsys@transformshift{3.405107in}{5.354017in}%
\pgfsys@useobject{currentmarker}{}%
\end{pgfscope}%
\begin{pgfscope}%
\pgfsys@transformshift{3.423544in}{5.359901in}%
\pgfsys@useobject{currentmarker}{}%
\end{pgfscope}%
\begin{pgfscope}%
\pgfsys@transformshift{3.441981in}{5.356841in}%
\pgfsys@useobject{currentmarker}{}%
\end{pgfscope}%
\begin{pgfscope}%
\pgfsys@transformshift{3.460418in}{5.359032in}%
\pgfsys@useobject{currentmarker}{}%
\end{pgfscope}%
\begin{pgfscope}%
\pgfsys@transformshift{3.478856in}{5.364931in}%
\pgfsys@useobject{currentmarker}{}%
\end{pgfscope}%
\begin{pgfscope}%
\pgfsys@transformshift{3.497293in}{5.372495in}%
\pgfsys@useobject{currentmarker}{}%
\end{pgfscope}%
\begin{pgfscope}%
\pgfsys@transformshift{3.515730in}{5.371485in}%
\pgfsys@useobject{currentmarker}{}%
\end{pgfscope}%
\begin{pgfscope}%
\pgfsys@transformshift{3.534167in}{5.376429in}%
\pgfsys@useobject{currentmarker}{}%
\end{pgfscope}%
\begin{pgfscope}%
\pgfsys@transformshift{3.552604in}{5.374035in}%
\pgfsys@useobject{currentmarker}{}%
\end{pgfscope}%
\begin{pgfscope}%
\pgfsys@transformshift{3.571041in}{5.385004in}%
\pgfsys@useobject{currentmarker}{}%
\end{pgfscope}%
\begin{pgfscope}%
\pgfsys@transformshift{3.589478in}{5.387747in}%
\pgfsys@useobject{currentmarker}{}%
\end{pgfscope}%
\begin{pgfscope}%
\pgfsys@transformshift{3.607915in}{5.391825in}%
\pgfsys@useobject{currentmarker}{}%
\end{pgfscope}%
\begin{pgfscope}%
\pgfsys@transformshift{3.626352in}{5.391742in}%
\pgfsys@useobject{currentmarker}{}%
\end{pgfscope}%
\begin{pgfscope}%
\pgfsys@transformshift{3.644790in}{5.394709in}%
\pgfsys@useobject{currentmarker}{}%
\end{pgfscope}%
\begin{pgfscope}%
\pgfsys@transformshift{3.663227in}{5.400824in}%
\pgfsys@useobject{currentmarker}{}%
\end{pgfscope}%
\begin{pgfscope}%
\pgfsys@transformshift{3.681664in}{5.401708in}%
\pgfsys@useobject{currentmarker}{}%
\end{pgfscope}%
\begin{pgfscope}%
\pgfsys@transformshift{3.700101in}{5.407314in}%
\pgfsys@useobject{currentmarker}{}%
\end{pgfscope}%
\begin{pgfscope}%
\pgfsys@transformshift{3.718538in}{5.404671in}%
\pgfsys@useobject{currentmarker}{}%
\end{pgfscope}%
\begin{pgfscope}%
\pgfsys@transformshift{3.736975in}{5.412994in}%
\pgfsys@useobject{currentmarker}{}%
\end{pgfscope}%
\begin{pgfscope}%
\pgfsys@transformshift{3.755412in}{5.415773in}%
\pgfsys@useobject{currentmarker}{}%
\end{pgfscope}%
\begin{pgfscope}%
\pgfsys@transformshift{3.773849in}{5.416117in}%
\pgfsys@useobject{currentmarker}{}%
\end{pgfscope}%
\begin{pgfscope}%
\pgfsys@transformshift{3.792287in}{5.424201in}%
\pgfsys@useobject{currentmarker}{}%
\end{pgfscope}%
\begin{pgfscope}%
\pgfsys@transformshift{3.810724in}{5.427195in}%
\pgfsys@useobject{currentmarker}{}%
\end{pgfscope}%
\begin{pgfscope}%
\pgfsys@transformshift{3.829161in}{5.426484in}%
\pgfsys@useobject{currentmarker}{}%
\end{pgfscope}%
\begin{pgfscope}%
\pgfsys@transformshift{3.847598in}{5.432176in}%
\pgfsys@useobject{currentmarker}{}%
\end{pgfscope}%
\begin{pgfscope}%
\pgfsys@transformshift{3.866035in}{5.438760in}%
\pgfsys@useobject{currentmarker}{}%
\end{pgfscope}%
\begin{pgfscope}%
\pgfsys@transformshift{3.884472in}{5.442284in}%
\pgfsys@useobject{currentmarker}{}%
\end{pgfscope}%
\begin{pgfscope}%
\pgfsys@transformshift{3.902909in}{5.446593in}%
\pgfsys@useobject{currentmarker}{}%
\end{pgfscope}%
\begin{pgfscope}%
\pgfsys@transformshift{3.921346in}{5.443875in}%
\pgfsys@useobject{currentmarker}{}%
\end{pgfscope}%
\begin{pgfscope}%
\pgfsys@transformshift{3.939783in}{5.449414in}%
\pgfsys@useobject{currentmarker}{}%
\end{pgfscope}%
\begin{pgfscope}%
\pgfsys@transformshift{3.958221in}{5.450307in}%
\pgfsys@useobject{currentmarker}{}%
\end{pgfscope}%
\begin{pgfscope}%
\pgfsys@transformshift{3.976658in}{5.461116in}%
\pgfsys@useobject{currentmarker}{}%
\end{pgfscope}%
\begin{pgfscope}%
\pgfsys@transformshift{3.995095in}{5.464364in}%
\pgfsys@useobject{currentmarker}{}%
\end{pgfscope}%
\begin{pgfscope}%
\pgfsys@transformshift{4.013532in}{5.467222in}%
\pgfsys@useobject{currentmarker}{}%
\end{pgfscope}%
\begin{pgfscope}%
\pgfsys@transformshift{4.031969in}{5.472085in}%
\pgfsys@useobject{currentmarker}{}%
\end{pgfscope}%
\begin{pgfscope}%
\pgfsys@transformshift{4.050406in}{5.473483in}%
\pgfsys@useobject{currentmarker}{}%
\end{pgfscope}%
\begin{pgfscope}%
\pgfsys@transformshift{4.068843in}{5.477678in}%
\pgfsys@useobject{currentmarker}{}%
\end{pgfscope}%
\begin{pgfscope}%
\pgfsys@transformshift{4.087280in}{5.483403in}%
\pgfsys@useobject{currentmarker}{}%
\end{pgfscope}%
\begin{pgfscope}%
\pgfsys@transformshift{4.105718in}{5.479974in}%
\pgfsys@useobject{currentmarker}{}%
\end{pgfscope}%
\begin{pgfscope}%
\pgfsys@transformshift{4.124155in}{5.484001in}%
\pgfsys@useobject{currentmarker}{}%
\end{pgfscope}%
\begin{pgfscope}%
\pgfsys@transformshift{4.142592in}{5.488435in}%
\pgfsys@useobject{currentmarker}{}%
\end{pgfscope}%
\begin{pgfscope}%
\pgfsys@transformshift{4.161029in}{5.498685in}%
\pgfsys@useobject{currentmarker}{}%
\end{pgfscope}%
\begin{pgfscope}%
\pgfsys@transformshift{4.179466in}{5.501094in}%
\pgfsys@useobject{currentmarker}{}%
\end{pgfscope}%
\begin{pgfscope}%
\pgfsys@transformshift{4.197903in}{5.499308in}%
\pgfsys@useobject{currentmarker}{}%
\end{pgfscope}%
\begin{pgfscope}%
\pgfsys@transformshift{4.216340in}{5.505231in}%
\pgfsys@useobject{currentmarker}{}%
\end{pgfscope}%
\begin{pgfscope}%
\pgfsys@transformshift{4.234777in}{5.511187in}%
\pgfsys@useobject{currentmarker}{}%
\end{pgfscope}%
\begin{pgfscope}%
\pgfsys@transformshift{4.253214in}{5.510101in}%
\pgfsys@useobject{currentmarker}{}%
\end{pgfscope}%
\begin{pgfscope}%
\pgfsys@transformshift{4.271652in}{5.519752in}%
\pgfsys@useobject{currentmarker}{}%
\end{pgfscope}%
\begin{pgfscope}%
\pgfsys@transformshift{4.290089in}{5.521282in}%
\pgfsys@useobject{currentmarker}{}%
\end{pgfscope}%
\begin{pgfscope}%
\pgfsys@transformshift{4.308526in}{5.522902in}%
\pgfsys@useobject{currentmarker}{}%
\end{pgfscope}%
\begin{pgfscope}%
\pgfsys@transformshift{4.326963in}{5.529162in}%
\pgfsys@useobject{currentmarker}{}%
\end{pgfscope}%
\begin{pgfscope}%
\pgfsys@transformshift{4.345400in}{5.534734in}%
\pgfsys@useobject{currentmarker}{}%
\end{pgfscope}%
\begin{pgfscope}%
\pgfsys@transformshift{4.363837in}{5.539041in}%
\pgfsys@useobject{currentmarker}{}%
\end{pgfscope}%
\begin{pgfscope}%
\pgfsys@transformshift{4.382274in}{5.538390in}%
\pgfsys@useobject{currentmarker}{}%
\end{pgfscope}%
\begin{pgfscope}%
\pgfsys@transformshift{4.400711in}{5.543495in}%
\pgfsys@useobject{currentmarker}{}%
\end{pgfscope}%
\begin{pgfscope}%
\pgfsys@transformshift{4.419148in}{5.548435in}%
\pgfsys@useobject{currentmarker}{}%
\end{pgfscope}%
\begin{pgfscope}%
\pgfsys@transformshift{4.437586in}{5.552190in}%
\pgfsys@useobject{currentmarker}{}%
\end{pgfscope}%
\begin{pgfscope}%
\pgfsys@transformshift{4.456023in}{5.555834in}%
\pgfsys@useobject{currentmarker}{}%
\end{pgfscope}%
\begin{pgfscope}%
\pgfsys@transformshift{4.474460in}{5.560513in}%
\pgfsys@useobject{currentmarker}{}%
\end{pgfscope}%
\begin{pgfscope}%
\pgfsys@transformshift{4.492897in}{5.569445in}%
\pgfsys@useobject{currentmarker}{}%
\end{pgfscope}%
\begin{pgfscope}%
\pgfsys@transformshift{4.511334in}{5.570108in}%
\pgfsys@useobject{currentmarker}{}%
\end{pgfscope}%
\begin{pgfscope}%
\pgfsys@transformshift{4.529771in}{5.580004in}%
\pgfsys@useobject{currentmarker}{}%
\end{pgfscope}%
\begin{pgfscope}%
\pgfsys@transformshift{4.548208in}{5.581232in}%
\pgfsys@useobject{currentmarker}{}%
\end{pgfscope}%
\begin{pgfscope}%
\pgfsys@transformshift{4.566645in}{5.587075in}%
\pgfsys@useobject{currentmarker}{}%
\end{pgfscope}%
\begin{pgfscope}%
\pgfsys@transformshift{4.585083in}{5.590381in}%
\pgfsys@useobject{currentmarker}{}%
\end{pgfscope}%
\begin{pgfscope}%
\pgfsys@transformshift{4.603520in}{5.596074in}%
\pgfsys@useobject{currentmarker}{}%
\end{pgfscope}%
\begin{pgfscope}%
\pgfsys@transformshift{4.621957in}{5.594206in}%
\pgfsys@useobject{currentmarker}{}%
\end{pgfscope}%
\begin{pgfscope}%
\pgfsys@transformshift{4.640394in}{5.598874in}%
\pgfsys@useobject{currentmarker}{}%
\end{pgfscope}%
\begin{pgfscope}%
\pgfsys@transformshift{4.658831in}{5.606560in}%
\pgfsys@useobject{currentmarker}{}%
\end{pgfscope}%
\begin{pgfscope}%
\pgfsys@transformshift{4.677268in}{5.608101in}%
\pgfsys@useobject{currentmarker}{}%
\end{pgfscope}%
\begin{pgfscope}%
\pgfsys@transformshift{4.695705in}{5.613041in}%
\pgfsys@useobject{currentmarker}{}%
\end{pgfscope}%
\begin{pgfscope}%
\pgfsys@transformshift{4.714142in}{5.616174in}%
\pgfsys@useobject{currentmarker}{}%
\end{pgfscope}%
\begin{pgfscope}%
\pgfsys@transformshift{4.732579in}{5.624314in}%
\pgfsys@useobject{currentmarker}{}%
\end{pgfscope}%
\begin{pgfscope}%
\pgfsys@transformshift{4.751017in}{5.633028in}%
\pgfsys@useobject{currentmarker}{}%
\end{pgfscope}%
\begin{pgfscope}%
\pgfsys@transformshift{4.769454in}{5.630883in}%
\pgfsys@useobject{currentmarker}{}%
\end{pgfscope}%
\begin{pgfscope}%
\pgfsys@transformshift{4.787891in}{5.642012in}%
\pgfsys@useobject{currentmarker}{}%
\end{pgfscope}%
\begin{pgfscope}%
\pgfsys@transformshift{4.806328in}{5.643826in}%
\pgfsys@useobject{currentmarker}{}%
\end{pgfscope}%
\begin{pgfscope}%
\pgfsys@transformshift{4.824765in}{5.650631in}%
\pgfsys@useobject{currentmarker}{}%
\end{pgfscope}%
\begin{pgfscope}%
\pgfsys@transformshift{4.843202in}{5.647796in}%
\pgfsys@useobject{currentmarker}{}%
\end{pgfscope}%
\begin{pgfscope}%
\pgfsys@transformshift{4.861639in}{5.652777in}%
\pgfsys@useobject{currentmarker}{}%
\end{pgfscope}%
\begin{pgfscope}%
\pgfsys@transformshift{4.880076in}{5.661371in}%
\pgfsys@useobject{currentmarker}{}%
\end{pgfscope}%
\begin{pgfscope}%
\pgfsys@transformshift{4.898514in}{5.669150in}%
\pgfsys@useobject{currentmarker}{}%
\end{pgfscope}%
\begin{pgfscope}%
\pgfsys@transformshift{4.916951in}{5.667934in}%
\pgfsys@useobject{currentmarker}{}%
\end{pgfscope}%
\begin{pgfscope}%
\pgfsys@transformshift{4.935388in}{5.673367in}%
\pgfsys@useobject{currentmarker}{}%
\end{pgfscope}%
\begin{pgfscope}%
\pgfsys@transformshift{4.953825in}{5.678145in}%
\pgfsys@useobject{currentmarker}{}%
\end{pgfscope}%
\begin{pgfscope}%
\pgfsys@transformshift{4.972262in}{5.688960in}%
\pgfsys@useobject{currentmarker}{}%
\end{pgfscope}%
\begin{pgfscope}%
\pgfsys@transformshift{4.990699in}{5.690217in}%
\pgfsys@useobject{currentmarker}{}%
\end{pgfscope}%
\begin{pgfscope}%
\pgfsys@transformshift{5.009136in}{5.698533in}%
\pgfsys@useobject{currentmarker}{}%
\end{pgfscope}%
\begin{pgfscope}%
\pgfsys@transformshift{5.027573in}{5.703117in}%
\pgfsys@useobject{currentmarker}{}%
\end{pgfscope}%
\begin{pgfscope}%
\pgfsys@transformshift{5.046010in}{5.703875in}%
\pgfsys@useobject{currentmarker}{}%
\end{pgfscope}%
\begin{pgfscope}%
\pgfsys@transformshift{5.064448in}{5.710961in}%
\pgfsys@useobject{currentmarker}{}%
\end{pgfscope}%
\begin{pgfscope}%
\pgfsys@transformshift{5.082885in}{5.711510in}%
\pgfsys@useobject{currentmarker}{}%
\end{pgfscope}%
\begin{pgfscope}%
\pgfsys@transformshift{5.101322in}{5.718166in}%
\pgfsys@useobject{currentmarker}{}%
\end{pgfscope}%
\begin{pgfscope}%
\pgfsys@transformshift{5.119759in}{5.721677in}%
\pgfsys@useobject{currentmarker}{}%
\end{pgfscope}%
\begin{pgfscope}%
\pgfsys@transformshift{5.138196in}{5.731899in}%
\pgfsys@useobject{currentmarker}{}%
\end{pgfscope}%
\begin{pgfscope}%
\pgfsys@transformshift{5.156633in}{5.734690in}%
\pgfsys@useobject{currentmarker}{}%
\end{pgfscope}%
\begin{pgfscope}%
\pgfsys@transformshift{5.175070in}{5.742834in}%
\pgfsys@useobject{currentmarker}{}%
\end{pgfscope}%
\begin{pgfscope}%
\pgfsys@transformshift{5.193507in}{5.745723in}%
\pgfsys@useobject{currentmarker}{}%
\end{pgfscope}%
\begin{pgfscope}%
\pgfsys@transformshift{5.211944in}{5.749612in}%
\pgfsys@useobject{currentmarker}{}%
\end{pgfscope}%
\begin{pgfscope}%
\pgfsys@transformshift{5.230382in}{5.755389in}%
\pgfsys@useobject{currentmarker}{}%
\end{pgfscope}%
\begin{pgfscope}%
\pgfsys@transformshift{5.248819in}{5.760969in}%
\pgfsys@useobject{currentmarker}{}%
\end{pgfscope}%
\begin{pgfscope}%
\pgfsys@transformshift{5.267256in}{5.761675in}%
\pgfsys@useobject{currentmarker}{}%
\end{pgfscope}%
\begin{pgfscope}%
\pgfsys@transformshift{5.285693in}{5.773123in}%
\pgfsys@useobject{currentmarker}{}%
\end{pgfscope}%
\begin{pgfscope}%
\pgfsys@transformshift{5.304130in}{5.777667in}%
\pgfsys@useobject{currentmarker}{}%
\end{pgfscope}%
\begin{pgfscope}%
\pgfsys@transformshift{5.322567in}{5.783133in}%
\pgfsys@useobject{currentmarker}{}%
\end{pgfscope}%
\begin{pgfscope}%
\pgfsys@transformshift{5.341004in}{5.781595in}%
\pgfsys@useobject{currentmarker}{}%
\end{pgfscope}%
\begin{pgfscope}%
\pgfsys@transformshift{5.359441in}{5.791653in}%
\pgfsys@useobject{currentmarker}{}%
\end{pgfscope}%
\begin{pgfscope}%
\pgfsys@transformshift{5.377879in}{5.791447in}%
\pgfsys@useobject{currentmarker}{}%
\end{pgfscope}%
\begin{pgfscope}%
\pgfsys@transformshift{5.396316in}{5.797870in}%
\pgfsys@useobject{currentmarker}{}%
\end{pgfscope}%
\end{pgfscope}%
\begin{pgfscope}%
\pgfsetrectcap%
\pgfsetmiterjoin%
\pgfsetlinewidth{0.803000pt}%
\definecolor{currentstroke}{rgb}{0.000000,0.000000,0.000000}%
\pgfsetstrokecolor{currentstroke}%
\pgfsetdash{}{0pt}%
\pgfpathmoveto{\pgfqpoint{0.459778in}{5.093981in}}%
\pgfpathlineto{\pgfqpoint{0.459778in}{5.831389in}}%
\pgfusepath{stroke}%
\end{pgfscope}%
\begin{pgfscope}%
\pgfsetrectcap%
\pgfsetmiterjoin%
\pgfsetlinewidth{0.803000pt}%
\definecolor{currentstroke}{rgb}{0.000000,0.000000,0.000000}%
\pgfsetstrokecolor{currentstroke}%
\pgfsetdash{}{0pt}%
\pgfpathmoveto{\pgfqpoint{5.631389in}{5.093981in}}%
\pgfpathlineto{\pgfqpoint{5.631389in}{5.831389in}}%
\pgfusepath{stroke}%
\end{pgfscope}%
\begin{pgfscope}%
\pgfsetrectcap%
\pgfsetmiterjoin%
\pgfsetlinewidth{0.803000pt}%
\definecolor{currentstroke}{rgb}{0.000000,0.000000,0.000000}%
\pgfsetstrokecolor{currentstroke}%
\pgfsetdash{}{0pt}%
\pgfpathmoveto{\pgfqpoint{0.459778in}{5.093981in}}%
\pgfpathlineto{\pgfqpoint{5.631389in}{5.093981in}}%
\pgfusepath{stroke}%
\end{pgfscope}%
\begin{pgfscope}%
\pgfsetrectcap%
\pgfsetmiterjoin%
\pgfsetlinewidth{0.803000pt}%
\definecolor{currentstroke}{rgb}{0.000000,0.000000,0.000000}%
\pgfsetstrokecolor{currentstroke}%
\pgfsetdash{}{0pt}%
\pgfpathmoveto{\pgfqpoint{0.459778in}{5.831389in}}%
\pgfpathlineto{\pgfqpoint{5.631389in}{5.831389in}}%
\pgfusepath{stroke}%
\end{pgfscope}%
\begin{pgfscope}%
\pgfsetbuttcap%
\pgfsetmiterjoin%
\definecolor{currentfill}{rgb}{1.000000,1.000000,1.000000}%
\pgfsetfillcolor{currentfill}%
\pgfsetlinewidth{0.000000pt}%
\definecolor{currentstroke}{rgb}{0.000000,0.000000,0.000000}%
\pgfsetstrokecolor{currentstroke}%
\pgfsetstrokeopacity{0.000000}%
\pgfsetdash{}{0pt}%
\pgfpathmoveto{\pgfqpoint{0.459778in}{4.138363in}}%
\pgfpathlineto{\pgfqpoint{5.631389in}{4.138363in}}%
\pgfpathlineto{\pgfqpoint{5.631389in}{4.875770in}}%
\pgfpathlineto{\pgfqpoint{0.459778in}{4.875770in}}%
\pgfpathclose%
\pgfusepath{fill}%
\end{pgfscope}%
\begin{pgfscope}%
\pgfsetbuttcap%
\pgfsetroundjoin%
\definecolor{currentfill}{rgb}{0.000000,0.000000,0.000000}%
\pgfsetfillcolor{currentfill}%
\pgfsetlinewidth{0.803000pt}%
\definecolor{currentstroke}{rgb}{0.000000,0.000000,0.000000}%
\pgfsetstrokecolor{currentstroke}%
\pgfsetdash{}{0pt}%
\pgfsys@defobject{currentmarker}{\pgfqpoint{0.000000in}{-0.048611in}}{\pgfqpoint{0.000000in}{0.000000in}}{%
\pgfpathmoveto{\pgfqpoint{0.000000in}{0.000000in}}%
\pgfpathlineto{\pgfqpoint{0.000000in}{-0.048611in}}%
\pgfusepath{stroke,fill}%
}%
\begin{pgfscope}%
\pgfsys@transformshift{0.694851in}{4.138363in}%
\pgfsys@useobject{currentmarker}{}%
\end{pgfscope}%
\end{pgfscope}%
\begin{pgfscope}%
\pgfsetbuttcap%
\pgfsetroundjoin%
\definecolor{currentfill}{rgb}{0.000000,0.000000,0.000000}%
\pgfsetfillcolor{currentfill}%
\pgfsetlinewidth{0.803000pt}%
\definecolor{currentstroke}{rgb}{0.000000,0.000000,0.000000}%
\pgfsetstrokecolor{currentstroke}%
\pgfsetdash{}{0pt}%
\pgfsys@defobject{currentmarker}{\pgfqpoint{0.000000in}{-0.048611in}}{\pgfqpoint{0.000000in}{0.000000in}}{%
\pgfpathmoveto{\pgfqpoint{0.000000in}{0.000000in}}%
\pgfpathlineto{\pgfqpoint{0.000000in}{-0.048611in}}%
\pgfusepath{stroke,fill}%
}%
\begin{pgfscope}%
\pgfsys@transformshift{1.282534in}{4.138363in}%
\pgfsys@useobject{currentmarker}{}%
\end{pgfscope}%
\end{pgfscope}%
\begin{pgfscope}%
\pgfsetbuttcap%
\pgfsetroundjoin%
\definecolor{currentfill}{rgb}{0.000000,0.000000,0.000000}%
\pgfsetfillcolor{currentfill}%
\pgfsetlinewidth{0.803000pt}%
\definecolor{currentstroke}{rgb}{0.000000,0.000000,0.000000}%
\pgfsetstrokecolor{currentstroke}%
\pgfsetdash{}{0pt}%
\pgfsys@defobject{currentmarker}{\pgfqpoint{0.000000in}{-0.048611in}}{\pgfqpoint{0.000000in}{0.000000in}}{%
\pgfpathmoveto{\pgfqpoint{0.000000in}{0.000000in}}%
\pgfpathlineto{\pgfqpoint{0.000000in}{-0.048611in}}%
\pgfusepath{stroke,fill}%
}%
\begin{pgfscope}%
\pgfsys@transformshift{1.870217in}{4.138363in}%
\pgfsys@useobject{currentmarker}{}%
\end{pgfscope}%
\end{pgfscope}%
\begin{pgfscope}%
\pgfsetbuttcap%
\pgfsetroundjoin%
\definecolor{currentfill}{rgb}{0.000000,0.000000,0.000000}%
\pgfsetfillcolor{currentfill}%
\pgfsetlinewidth{0.803000pt}%
\definecolor{currentstroke}{rgb}{0.000000,0.000000,0.000000}%
\pgfsetstrokecolor{currentstroke}%
\pgfsetdash{}{0pt}%
\pgfsys@defobject{currentmarker}{\pgfqpoint{0.000000in}{-0.048611in}}{\pgfqpoint{0.000000in}{0.000000in}}{%
\pgfpathmoveto{\pgfqpoint{0.000000in}{0.000000in}}%
\pgfpathlineto{\pgfqpoint{0.000000in}{-0.048611in}}%
\pgfusepath{stroke,fill}%
}%
\begin{pgfscope}%
\pgfsys@transformshift{2.457900in}{4.138363in}%
\pgfsys@useobject{currentmarker}{}%
\end{pgfscope}%
\end{pgfscope}%
\begin{pgfscope}%
\pgfsetbuttcap%
\pgfsetroundjoin%
\definecolor{currentfill}{rgb}{0.000000,0.000000,0.000000}%
\pgfsetfillcolor{currentfill}%
\pgfsetlinewidth{0.803000pt}%
\definecolor{currentstroke}{rgb}{0.000000,0.000000,0.000000}%
\pgfsetstrokecolor{currentstroke}%
\pgfsetdash{}{0pt}%
\pgfsys@defobject{currentmarker}{\pgfqpoint{0.000000in}{-0.048611in}}{\pgfqpoint{0.000000in}{0.000000in}}{%
\pgfpathmoveto{\pgfqpoint{0.000000in}{0.000000in}}%
\pgfpathlineto{\pgfqpoint{0.000000in}{-0.048611in}}%
\pgfusepath{stroke,fill}%
}%
\begin{pgfscope}%
\pgfsys@transformshift{3.045583in}{4.138363in}%
\pgfsys@useobject{currentmarker}{}%
\end{pgfscope}%
\end{pgfscope}%
\begin{pgfscope}%
\pgfsetbuttcap%
\pgfsetroundjoin%
\definecolor{currentfill}{rgb}{0.000000,0.000000,0.000000}%
\pgfsetfillcolor{currentfill}%
\pgfsetlinewidth{0.803000pt}%
\definecolor{currentstroke}{rgb}{0.000000,0.000000,0.000000}%
\pgfsetstrokecolor{currentstroke}%
\pgfsetdash{}{0pt}%
\pgfsys@defobject{currentmarker}{\pgfqpoint{0.000000in}{-0.048611in}}{\pgfqpoint{0.000000in}{0.000000in}}{%
\pgfpathmoveto{\pgfqpoint{0.000000in}{0.000000in}}%
\pgfpathlineto{\pgfqpoint{0.000000in}{-0.048611in}}%
\pgfusepath{stroke,fill}%
}%
\begin{pgfscope}%
\pgfsys@transformshift{3.633266in}{4.138363in}%
\pgfsys@useobject{currentmarker}{}%
\end{pgfscope}%
\end{pgfscope}%
\begin{pgfscope}%
\pgfsetbuttcap%
\pgfsetroundjoin%
\definecolor{currentfill}{rgb}{0.000000,0.000000,0.000000}%
\pgfsetfillcolor{currentfill}%
\pgfsetlinewidth{0.803000pt}%
\definecolor{currentstroke}{rgb}{0.000000,0.000000,0.000000}%
\pgfsetstrokecolor{currentstroke}%
\pgfsetdash{}{0pt}%
\pgfsys@defobject{currentmarker}{\pgfqpoint{0.000000in}{-0.048611in}}{\pgfqpoint{0.000000in}{0.000000in}}{%
\pgfpathmoveto{\pgfqpoint{0.000000in}{0.000000in}}%
\pgfpathlineto{\pgfqpoint{0.000000in}{-0.048611in}}%
\pgfusepath{stroke,fill}%
}%
\begin{pgfscope}%
\pgfsys@transformshift{4.220949in}{4.138363in}%
\pgfsys@useobject{currentmarker}{}%
\end{pgfscope}%
\end{pgfscope}%
\begin{pgfscope}%
\pgfsetbuttcap%
\pgfsetroundjoin%
\definecolor{currentfill}{rgb}{0.000000,0.000000,0.000000}%
\pgfsetfillcolor{currentfill}%
\pgfsetlinewidth{0.803000pt}%
\definecolor{currentstroke}{rgb}{0.000000,0.000000,0.000000}%
\pgfsetstrokecolor{currentstroke}%
\pgfsetdash{}{0pt}%
\pgfsys@defobject{currentmarker}{\pgfqpoint{0.000000in}{-0.048611in}}{\pgfqpoint{0.000000in}{0.000000in}}{%
\pgfpathmoveto{\pgfqpoint{0.000000in}{0.000000in}}%
\pgfpathlineto{\pgfqpoint{0.000000in}{-0.048611in}}%
\pgfusepath{stroke,fill}%
}%
\begin{pgfscope}%
\pgfsys@transformshift{4.808633in}{4.138363in}%
\pgfsys@useobject{currentmarker}{}%
\end{pgfscope}%
\end{pgfscope}%
\begin{pgfscope}%
\pgfsetbuttcap%
\pgfsetroundjoin%
\definecolor{currentfill}{rgb}{0.000000,0.000000,0.000000}%
\pgfsetfillcolor{currentfill}%
\pgfsetlinewidth{0.803000pt}%
\definecolor{currentstroke}{rgb}{0.000000,0.000000,0.000000}%
\pgfsetstrokecolor{currentstroke}%
\pgfsetdash{}{0pt}%
\pgfsys@defobject{currentmarker}{\pgfqpoint{0.000000in}{-0.048611in}}{\pgfqpoint{0.000000in}{0.000000in}}{%
\pgfpathmoveto{\pgfqpoint{0.000000in}{0.000000in}}%
\pgfpathlineto{\pgfqpoint{0.000000in}{-0.048611in}}%
\pgfusepath{stroke,fill}%
}%
\begin{pgfscope}%
\pgfsys@transformshift{5.396316in}{4.138363in}%
\pgfsys@useobject{currentmarker}{}%
\end{pgfscope}%
\end{pgfscope}%
\begin{pgfscope}%
\pgfsetbuttcap%
\pgfsetroundjoin%
\definecolor{currentfill}{rgb}{0.000000,0.000000,0.000000}%
\pgfsetfillcolor{currentfill}%
\pgfsetlinewidth{0.803000pt}%
\definecolor{currentstroke}{rgb}{0.000000,0.000000,0.000000}%
\pgfsetstrokecolor{currentstroke}%
\pgfsetdash{}{0pt}%
\pgfsys@defobject{currentmarker}{\pgfqpoint{-0.048611in}{0.000000in}}{\pgfqpoint{0.000000in}{0.000000in}}{%
\pgfpathmoveto{\pgfqpoint{0.000000in}{0.000000in}}%
\pgfpathlineto{\pgfqpoint{-0.048611in}{0.000000in}}%
\pgfusepath{stroke,fill}%
}%
\begin{pgfscope}%
\pgfsys@transformshift{0.459778in}{4.275740in}%
\pgfsys@useobject{currentmarker}{}%
\end{pgfscope}%
\end{pgfscope}%
\begin{pgfscope}%
\definecolor{textcolor}{rgb}{0.000000,0.000000,0.000000}%
\pgfsetstrokecolor{textcolor}%
\pgfsetfillcolor{textcolor}%
\pgftext[x=0.120000in,y=4.237185in,left,base]{\color{textcolor}\rmfamily\fontsize{8.000000}{9.600000}\selectfont −0.1}%
\end{pgfscope}%
\begin{pgfscope}%
\pgfsetbuttcap%
\pgfsetroundjoin%
\definecolor{currentfill}{rgb}{0.000000,0.000000,0.000000}%
\pgfsetfillcolor{currentfill}%
\pgfsetlinewidth{0.803000pt}%
\definecolor{currentstroke}{rgb}{0.000000,0.000000,0.000000}%
\pgfsetstrokecolor{currentstroke}%
\pgfsetdash{}{0pt}%
\pgfsys@defobject{currentmarker}{\pgfqpoint{-0.048611in}{0.000000in}}{\pgfqpoint{0.000000in}{0.000000in}}{%
\pgfpathmoveto{\pgfqpoint{0.000000in}{0.000000in}}%
\pgfpathlineto{\pgfqpoint{-0.048611in}{0.000000in}}%
\pgfusepath{stroke,fill}%
}%
\begin{pgfscope}%
\pgfsys@transformshift{0.459778in}{4.540135in}%
\pgfsys@useobject{currentmarker}{}%
\end{pgfscope}%
\end{pgfscope}%
\begin{pgfscope}%
\definecolor{textcolor}{rgb}{0.000000,0.000000,0.000000}%
\pgfsetstrokecolor{textcolor}%
\pgfsetfillcolor{textcolor}%
\pgftext[x=0.211778in,y=4.501579in,left,base]{\color{textcolor}\rmfamily\fontsize{8.000000}{9.600000}\selectfont 0.0}%
\end{pgfscope}%
\begin{pgfscope}%
\pgfsetbuttcap%
\pgfsetroundjoin%
\definecolor{currentfill}{rgb}{0.000000,0.000000,0.000000}%
\pgfsetfillcolor{currentfill}%
\pgfsetlinewidth{0.803000pt}%
\definecolor{currentstroke}{rgb}{0.000000,0.000000,0.000000}%
\pgfsetstrokecolor{currentstroke}%
\pgfsetdash{}{0pt}%
\pgfsys@defobject{currentmarker}{\pgfqpoint{-0.048611in}{0.000000in}}{\pgfqpoint{0.000000in}{0.000000in}}{%
\pgfpathmoveto{\pgfqpoint{0.000000in}{0.000000in}}%
\pgfpathlineto{\pgfqpoint{-0.048611in}{0.000000in}}%
\pgfusepath{stroke,fill}%
}%
\begin{pgfscope}%
\pgfsys@transformshift{0.459778in}{4.804529in}%
\pgfsys@useobject{currentmarker}{}%
\end{pgfscope}%
\end{pgfscope}%
\begin{pgfscope}%
\definecolor{textcolor}{rgb}{0.000000,0.000000,0.000000}%
\pgfsetstrokecolor{textcolor}%
\pgfsetfillcolor{textcolor}%
\pgftext[x=0.211778in,y=4.765974in,left,base]{\color{textcolor}\rmfamily\fontsize{8.000000}{9.600000}\selectfont 0.1}%
\end{pgfscope}%
\begin{pgfscope}%
\pgfpathrectangle{\pgfqpoint{0.459778in}{4.138363in}}{\pgfqpoint{5.171611in}{0.737408in}}%
\pgfusepath{clip}%
\pgfsetrectcap%
\pgfsetroundjoin%
\pgfsetlinewidth{1.505625pt}%
\definecolor{currentstroke}{rgb}{1.000000,0.498039,0.054902}%
\pgfsetstrokecolor{currentstroke}%
\pgfsetdash{}{0pt}%
\pgfpathmoveto{\pgfqpoint{0.704070in}{4.493740in}}%
\pgfpathlineto{\pgfqpoint{0.722507in}{4.493740in}}%
\pgfpathlineto{\pgfqpoint{0.722507in}{4.554231in}}%
\pgfpathlineto{\pgfqpoint{0.759381in}{4.554231in}}%
\pgfpathlineto{\pgfqpoint{0.759381in}{4.497503in}}%
\pgfpathlineto{\pgfqpoint{0.796255in}{4.497503in}}%
\pgfpathlineto{\pgfqpoint{0.796255in}{4.578014in}}%
\pgfpathlineto{\pgfqpoint{0.833129in}{4.578014in}}%
\pgfpathlineto{\pgfqpoint{0.833129in}{4.485134in}}%
\pgfpathlineto{\pgfqpoint{0.870004in}{4.485134in}}%
\pgfpathlineto{\pgfqpoint{0.870004in}{4.550741in}}%
\pgfpathlineto{\pgfqpoint{0.906878in}{4.550741in}}%
\pgfpathlineto{\pgfqpoint{0.906878in}{4.533035in}}%
\pgfpathlineto{\pgfqpoint{0.943752in}{4.533035in}}%
\pgfpathlineto{\pgfqpoint{0.943752in}{4.592296in}}%
\pgfpathlineto{\pgfqpoint{0.980626in}{4.592296in}}%
\pgfpathlineto{\pgfqpoint{0.980626in}{4.544493in}}%
\pgfpathlineto{\pgfqpoint{1.017501in}{4.544493in}}%
\pgfpathlineto{\pgfqpoint{1.017501in}{4.563799in}}%
\pgfpathlineto{\pgfqpoint{1.054375in}{4.563799in}}%
\pgfpathlineto{\pgfqpoint{1.054375in}{4.545125in}}%
\pgfpathlineto{\pgfqpoint{1.091249in}{4.545125in}}%
\pgfpathlineto{\pgfqpoint{1.091249in}{4.499276in}}%
\pgfpathlineto{\pgfqpoint{1.128123in}{4.499276in}}%
\pgfpathlineto{\pgfqpoint{1.128123in}{4.501642in}}%
\pgfpathlineto{\pgfqpoint{1.164997in}{4.501642in}}%
\pgfpathlineto{\pgfqpoint{1.164997in}{4.601893in}}%
\pgfpathlineto{\pgfqpoint{1.201872in}{4.601893in}}%
\pgfpathlineto{\pgfqpoint{1.201872in}{4.467549in}}%
\pgfpathlineto{\pgfqpoint{1.238746in}{4.467549in}}%
\pgfpathlineto{\pgfqpoint{1.238746in}{4.536756in}}%
\pgfpathlineto{\pgfqpoint{1.275620in}{4.536756in}}%
\pgfpathlineto{\pgfqpoint{1.275620in}{4.502513in}}%
\pgfpathlineto{\pgfqpoint{1.312494in}{4.502513in}}%
\pgfpathlineto{\pgfqpoint{1.312494in}{4.567620in}}%
\pgfpathlineto{\pgfqpoint{1.349369in}{4.567620in}}%
\pgfpathlineto{\pgfqpoint{1.349369in}{4.588884in}}%
\pgfpathlineto{\pgfqpoint{1.386243in}{4.588884in}}%
\pgfpathlineto{\pgfqpoint{1.386243in}{4.496476in}}%
\pgfpathlineto{\pgfqpoint{1.459991in}{4.496492in}}%
\pgfpathlineto{\pgfqpoint{1.459991in}{4.578745in}}%
\pgfpathlineto{\pgfqpoint{1.496866in}{4.578745in}}%
\pgfpathlineto{\pgfqpoint{1.496866in}{4.467925in}}%
\pgfpathlineto{\pgfqpoint{1.533740in}{4.467925in}}%
\pgfpathlineto{\pgfqpoint{1.533740in}{4.586637in}}%
\pgfpathlineto{\pgfqpoint{1.570614in}{4.586637in}}%
\pgfpathlineto{\pgfqpoint{1.570614in}{4.533060in}}%
\pgfpathlineto{\pgfqpoint{1.607488in}{4.533060in}}%
\pgfpathlineto{\pgfqpoint{1.607488in}{4.547959in}}%
\pgfpathlineto{\pgfqpoint{1.644362in}{4.547959in}}%
\pgfpathlineto{\pgfqpoint{1.644362in}{4.532363in}}%
\pgfpathlineto{\pgfqpoint{1.681237in}{4.532363in}}%
\pgfpathlineto{\pgfqpoint{1.681237in}{4.556502in}}%
\pgfpathlineto{\pgfqpoint{1.718111in}{4.556502in}}%
\pgfpathlineto{\pgfqpoint{1.718111in}{4.495154in}}%
\pgfpathlineto{\pgfqpoint{1.754985in}{4.495154in}}%
\pgfpathlineto{\pgfqpoint{1.754985in}{4.488610in}}%
\pgfpathlineto{\pgfqpoint{1.791859in}{4.488610in}}%
\pgfpathlineto{\pgfqpoint{1.791859in}{4.495047in}}%
\pgfpathlineto{\pgfqpoint{1.828734in}{4.495047in}}%
\pgfpathlineto{\pgfqpoint{1.828734in}{4.488993in}}%
\pgfpathlineto{\pgfqpoint{1.865608in}{4.488993in}}%
\pgfpathlineto{\pgfqpoint{1.865608in}{4.588622in}}%
\pgfpathlineto{\pgfqpoint{1.902482in}{4.588622in}}%
\pgfpathlineto{\pgfqpoint{1.902482in}{4.540026in}}%
\pgfpathlineto{\pgfqpoint{1.939356in}{4.540026in}}%
\pgfpathlineto{\pgfqpoint{1.939356in}{4.439366in}}%
\pgfpathlineto{\pgfqpoint{1.976231in}{4.439366in}}%
\pgfpathlineto{\pgfqpoint{1.976231in}{4.479372in}}%
\pgfpathlineto{\pgfqpoint{2.013105in}{4.479372in}}%
\pgfpathlineto{\pgfqpoint{2.013105in}{4.518823in}}%
\pgfpathlineto{\pgfqpoint{2.049979in}{4.518823in}}%
\pgfpathlineto{\pgfqpoint{2.049979in}{4.471840in}}%
\pgfpathlineto{\pgfqpoint{2.086853in}{4.471840in}}%
\pgfpathlineto{\pgfqpoint{2.086853in}{4.519341in}}%
\pgfpathlineto{\pgfqpoint{2.123728in}{4.519341in}}%
\pgfpathlineto{\pgfqpoint{2.123728in}{4.476920in}}%
\pgfpathlineto{\pgfqpoint{2.160602in}{4.476920in}}%
\pgfpathlineto{\pgfqpoint{2.160602in}{4.563100in}}%
\pgfpathlineto{\pgfqpoint{2.197476in}{4.563100in}}%
\pgfpathlineto{\pgfqpoint{2.197476in}{4.565694in}}%
\pgfpathlineto{\pgfqpoint{2.234350in}{4.565694in}}%
\pgfpathlineto{\pgfqpoint{2.234350in}{4.515725in}}%
\pgfpathlineto{\pgfqpoint{2.271224in}{4.515725in}}%
\pgfpathlineto{\pgfqpoint{2.271224in}{4.496220in}}%
\pgfpathlineto{\pgfqpoint{2.308099in}{4.496220in}}%
\pgfpathlineto{\pgfqpoint{2.308099in}{4.557776in}}%
\pgfpathlineto{\pgfqpoint{2.344973in}{4.557776in}}%
\pgfpathlineto{\pgfqpoint{2.344973in}{4.546939in}}%
\pgfpathlineto{\pgfqpoint{2.381847in}{4.546939in}}%
\pgfpathlineto{\pgfqpoint{2.381847in}{4.556569in}}%
\pgfpathlineto{\pgfqpoint{2.418721in}{4.556569in}}%
\pgfpathlineto{\pgfqpoint{2.418721in}{4.537958in}}%
\pgfpathlineto{\pgfqpoint{2.455596in}{4.537958in}}%
\pgfpathlineto{\pgfqpoint{2.455596in}{4.570535in}}%
\pgfpathlineto{\pgfqpoint{2.529344in}{4.571808in}}%
\pgfpathlineto{\pgfqpoint{2.529344in}{4.476770in}}%
\pgfpathlineto{\pgfqpoint{2.566218in}{4.476770in}}%
\pgfpathlineto{\pgfqpoint{2.566218in}{4.486770in}}%
\pgfpathlineto{\pgfqpoint{2.603093in}{4.486770in}}%
\pgfpathlineto{\pgfqpoint{2.603093in}{4.516272in}}%
\pgfpathlineto{\pgfqpoint{2.639967in}{4.516272in}}%
\pgfpathlineto{\pgfqpoint{2.639967in}{4.439304in}}%
\pgfpathlineto{\pgfqpoint{2.676841in}{4.439304in}}%
\pgfpathlineto{\pgfqpoint{2.676841in}{4.472317in}}%
\pgfpathlineto{\pgfqpoint{2.713715in}{4.472317in}}%
\pgfpathlineto{\pgfqpoint{2.713715in}{4.556497in}}%
\pgfpathlineto{\pgfqpoint{2.750589in}{4.556497in}}%
\pgfpathlineto{\pgfqpoint{2.750589in}{4.526655in}}%
\pgfpathlineto{\pgfqpoint{2.787464in}{4.526655in}}%
\pgfpathlineto{\pgfqpoint{2.787464in}{4.543102in}}%
\pgfpathlineto{\pgfqpoint{2.824338in}{4.543102in}}%
\pgfpathlineto{\pgfqpoint{2.824338in}{4.494399in}}%
\pgfpathlineto{\pgfqpoint{2.861212in}{4.494399in}}%
\pgfpathlineto{\pgfqpoint{2.861212in}{4.498161in}}%
\pgfpathlineto{\pgfqpoint{2.898086in}{4.498161in}}%
\pgfpathlineto{\pgfqpoint{2.898086in}{4.437810in}}%
\pgfpathlineto{\pgfqpoint{2.934961in}{4.437810in}}%
\pgfpathlineto{\pgfqpoint{2.934961in}{4.510559in}}%
\pgfpathlineto{\pgfqpoint{3.008709in}{4.511874in}}%
\pgfpathlineto{\pgfqpoint{3.008709in}{4.553520in}}%
\pgfpathlineto{\pgfqpoint{3.045583in}{4.553520in}}%
\pgfpathlineto{\pgfqpoint{3.045583in}{4.472677in}}%
\pgfpathlineto{\pgfqpoint{3.082458in}{4.472677in}}%
\pgfpathlineto{\pgfqpoint{3.082458in}{4.506919in}}%
\pgfpathlineto{\pgfqpoint{3.119332in}{4.506919in}}%
\pgfpathlineto{\pgfqpoint{3.119332in}{4.480355in}}%
\pgfpathlineto{\pgfqpoint{3.156206in}{4.480355in}}%
\pgfpathlineto{\pgfqpoint{3.156206in}{4.512485in}}%
\pgfpathlineto{\pgfqpoint{3.193080in}{4.512485in}}%
\pgfpathlineto{\pgfqpoint{3.193080in}{4.517817in}}%
\pgfpathlineto{\pgfqpoint{3.229954in}{4.517817in}}%
\pgfpathlineto{\pgfqpoint{3.229954in}{4.463198in}}%
\pgfpathlineto{\pgfqpoint{3.266829in}{4.463198in}}%
\pgfpathlineto{\pgfqpoint{3.266829in}{4.475442in}}%
\pgfpathlineto{\pgfqpoint{3.303703in}{4.475442in}}%
\pgfpathlineto{\pgfqpoint{3.303703in}{4.493681in}}%
\pgfpathlineto{\pgfqpoint{3.340577in}{4.493681in}}%
\pgfpathlineto{\pgfqpoint{3.340577in}{4.491418in}}%
\pgfpathlineto{\pgfqpoint{3.377451in}{4.491418in}}%
\pgfpathlineto{\pgfqpoint{3.377451in}{4.524318in}}%
\pgfpathlineto{\pgfqpoint{3.414326in}{4.524318in}}%
\pgfpathlineto{\pgfqpoint{3.414326in}{4.574329in}}%
\pgfpathlineto{\pgfqpoint{3.451200in}{4.574329in}}%
\pgfpathlineto{\pgfqpoint{3.451200in}{4.474223in}}%
\pgfpathlineto{\pgfqpoint{3.488074in}{4.474223in}}%
\pgfpathlineto{\pgfqpoint{3.488074in}{4.551424in}}%
\pgfpathlineto{\pgfqpoint{3.524948in}{4.551424in}}%
\pgfpathlineto{\pgfqpoint{3.524948in}{4.566879in}}%
\pgfpathlineto{\pgfqpoint{3.561823in}{4.566879in}}%
\pgfpathlineto{\pgfqpoint{3.561823in}{4.509485in}}%
\pgfpathlineto{\pgfqpoint{3.598697in}{4.509485in}}%
\pgfpathlineto{\pgfqpoint{3.598697in}{4.541057in}}%
\pgfpathlineto{\pgfqpoint{3.635571in}{4.541057in}}%
\pgfpathlineto{\pgfqpoint{3.635571in}{4.471820in}}%
\pgfpathlineto{\pgfqpoint{3.672445in}{4.471820in}}%
\pgfpathlineto{\pgfqpoint{3.672445in}{4.477506in}}%
\pgfpathlineto{\pgfqpoint{3.709320in}{4.477506in}}%
\pgfpathlineto{\pgfqpoint{3.709320in}{4.447147in}}%
\pgfpathlineto{\pgfqpoint{3.746194in}{4.447147in}}%
\pgfpathlineto{\pgfqpoint{3.746194in}{4.536290in}}%
\pgfpathlineto{\pgfqpoint{3.783068in}{4.536290in}}%
\pgfpathlineto{\pgfqpoint{3.783068in}{4.506690in}}%
\pgfpathlineto{\pgfqpoint{3.819942in}{4.506690in}}%
\pgfpathlineto{\pgfqpoint{3.819942in}{4.476542in}}%
\pgfpathlineto{\pgfqpoint{3.856816in}{4.476542in}}%
\pgfpathlineto{\pgfqpoint{3.856816in}{4.500769in}}%
\pgfpathlineto{\pgfqpoint{3.893691in}{4.500769in}}%
\pgfpathlineto{\pgfqpoint{3.893691in}{4.570500in}}%
\pgfpathlineto{\pgfqpoint{3.930565in}{4.570500in}}%
\pgfpathlineto{\pgfqpoint{3.930565in}{4.530163in}}%
\pgfpathlineto{\pgfqpoint{3.967439in}{4.530163in}}%
\pgfpathlineto{\pgfqpoint{3.967439in}{4.503846in}}%
\pgfpathlineto{\pgfqpoint{4.004313in}{4.503846in}}%
\pgfpathlineto{\pgfqpoint{4.004313in}{4.485804in}}%
\pgfpathlineto{\pgfqpoint{4.041188in}{4.485804in}}%
\pgfpathlineto{\pgfqpoint{4.041188in}{4.493266in}}%
\pgfpathlineto{\pgfqpoint{4.078062in}{4.493266in}}%
\pgfpathlineto{\pgfqpoint{4.078062in}{4.578435in}}%
\pgfpathlineto{\pgfqpoint{4.114936in}{4.578435in}}%
\pgfpathlineto{\pgfqpoint{4.114936in}{4.490599in}}%
\pgfpathlineto{\pgfqpoint{4.151810in}{4.490599in}}%
\pgfpathlineto{\pgfqpoint{4.151810in}{4.513224in}}%
\pgfpathlineto{\pgfqpoint{4.188685in}{4.513224in}}%
\pgfpathlineto{\pgfqpoint{4.188685in}{4.473963in}}%
\pgfpathlineto{\pgfqpoint{4.225559in}{4.473963in}}%
\pgfpathlineto{\pgfqpoint{4.225559in}{4.552264in}}%
\pgfpathlineto{\pgfqpoint{4.262433in}{4.552264in}}%
\pgfpathlineto{\pgfqpoint{4.262433in}{4.523042in}}%
\pgfpathlineto{\pgfqpoint{4.299307in}{4.523042in}}%
\pgfpathlineto{\pgfqpoint{4.299307in}{4.470196in}}%
\pgfpathlineto{\pgfqpoint{4.336181in}{4.470196in}}%
\pgfpathlineto{\pgfqpoint{4.336181in}{4.492019in}}%
\pgfpathlineto{\pgfqpoint{4.373056in}{4.492019in}}%
\pgfpathlineto{\pgfqpoint{4.373056in}{4.483102in}}%
\pgfpathlineto{\pgfqpoint{4.409930in}{4.483102in}}%
\pgfpathlineto{\pgfqpoint{4.409930in}{4.498193in}}%
\pgfpathlineto{\pgfqpoint{4.446804in}{4.498193in}}%
\pgfpathlineto{\pgfqpoint{4.446804in}{4.487860in}}%
\pgfpathlineto{\pgfqpoint{4.483678in}{4.487860in}}%
\pgfpathlineto{\pgfqpoint{4.483678in}{4.532723in}}%
\pgfpathlineto{\pgfqpoint{4.520553in}{4.532723in}}%
\pgfpathlineto{\pgfqpoint{4.520553in}{4.526418in}}%
\pgfpathlineto{\pgfqpoint{4.557427in}{4.526418in}}%
\pgfpathlineto{\pgfqpoint{4.557427in}{4.503205in}}%
\pgfpathlineto{\pgfqpoint{4.594301in}{4.503205in}}%
\pgfpathlineto{\pgfqpoint{4.594301in}{4.561001in}}%
\pgfpathlineto{\pgfqpoint{4.631175in}{4.561001in}}%
\pgfpathlineto{\pgfqpoint{4.631175in}{4.454272in}}%
\pgfpathlineto{\pgfqpoint{4.668050in}{4.454272in}}%
\pgfpathlineto{\pgfqpoint{4.668050in}{4.484957in}}%
\pgfpathlineto{\pgfqpoint{4.704924in}{4.484957in}}%
\pgfpathlineto{\pgfqpoint{4.704924in}{4.449191in}}%
\pgfpathlineto{\pgfqpoint{4.741798in}{4.449191in}}%
\pgfpathlineto{\pgfqpoint{4.741798in}{4.564099in}}%
\pgfpathlineto{\pgfqpoint{4.778672in}{4.564099in}}%
\pgfpathlineto{\pgfqpoint{4.778672in}{4.519861in}}%
\pgfpathlineto{\pgfqpoint{4.815546in}{4.519861in}}%
\pgfpathlineto{\pgfqpoint{4.815546in}{4.571802in}}%
\pgfpathlineto{\pgfqpoint{4.852421in}{4.571802in}}%
\pgfpathlineto{\pgfqpoint{4.852421in}{4.444127in}}%
\pgfpathlineto{\pgfqpoint{4.889295in}{4.444127in}}%
\pgfpathlineto{\pgfqpoint{4.889295in}{4.553714in}}%
\pgfpathlineto{\pgfqpoint{4.926169in}{4.553714in}}%
\pgfpathlineto{\pgfqpoint{4.926169in}{4.486754in}}%
\pgfpathlineto{\pgfqpoint{4.963043in}{4.486754in}}%
\pgfpathlineto{\pgfqpoint{4.963043in}{4.526096in}}%
\pgfpathlineto{\pgfqpoint{4.999918in}{4.526096in}}%
\pgfpathlineto{\pgfqpoint{4.999918in}{4.488921in}}%
\pgfpathlineto{\pgfqpoint{5.036792in}{4.488921in}}%
\pgfpathlineto{\pgfqpoint{5.036792in}{4.460972in}}%
\pgfpathlineto{\pgfqpoint{5.073666in}{4.460972in}}%
\pgfpathlineto{\pgfqpoint{5.073666in}{4.465779in}}%
\pgfpathlineto{\pgfqpoint{5.110540in}{4.465779in}}%
\pgfpathlineto{\pgfqpoint{5.110540in}{4.425943in}}%
\pgfpathlineto{\pgfqpoint{5.147415in}{4.425943in}}%
\pgfpathlineto{\pgfqpoint{5.147415in}{4.449160in}}%
\pgfpathlineto{\pgfqpoint{5.184289in}{4.449160in}}%
\pgfpathlineto{\pgfqpoint{5.184289in}{4.496683in}}%
\pgfpathlineto{\pgfqpoint{5.221163in}{4.496683in}}%
\pgfpathlineto{\pgfqpoint{5.221163in}{4.477798in}}%
\pgfpathlineto{\pgfqpoint{5.258037in}{4.477798in}}%
\pgfpathlineto{\pgfqpoint{5.258037in}{4.412240in}}%
\pgfpathlineto{\pgfqpoint{5.294912in}{4.412240in}}%
\pgfpathlineto{\pgfqpoint{5.294912in}{4.479067in}}%
\pgfpathlineto{\pgfqpoint{5.331786in}{4.479067in}}%
\pgfpathlineto{\pgfqpoint{5.331786in}{4.427770in}}%
\pgfpathlineto{\pgfqpoint{5.368660in}{4.427770in}}%
\pgfpathlineto{\pgfqpoint{5.368660in}{4.468374in}}%
\pgfpathlineto{\pgfqpoint{5.387097in}{4.468374in}}%
\pgfpathlineto{\pgfqpoint{5.387097in}{4.468374in}}%
\pgfusepath{stroke}%
\end{pgfscope}%
\begin{pgfscope}%
\pgfpathrectangle{\pgfqpoint{0.459778in}{4.138363in}}{\pgfqpoint{5.171611in}{0.737408in}}%
\pgfusepath{clip}%
\pgfsetbuttcap%
\pgfsetroundjoin%
\definecolor{currentfill}{rgb}{1.000000,0.498039,0.054902}%
\pgfsetfillcolor{currentfill}%
\pgfsetlinewidth{1.003750pt}%
\definecolor{currentstroke}{rgb}{1.000000,0.498039,0.054902}%
\pgfsetstrokecolor{currentstroke}%
\pgfsetdash{}{0pt}%
\pgfsys@defobject{currentmarker}{\pgfqpoint{-0.041667in}{-0.041667in}}{\pgfqpoint{0.041667in}{0.041667in}}{%
\pgfpathmoveto{\pgfqpoint{0.000000in}{-0.041667in}}%
\pgfpathcurveto{\pgfqpoint{0.011050in}{-0.041667in}}{\pgfqpoint{0.021649in}{-0.037276in}}{\pgfqpoint{0.029463in}{-0.029463in}}%
\pgfpathcurveto{\pgfqpoint{0.037276in}{-0.021649in}}{\pgfqpoint{0.041667in}{-0.011050in}}{\pgfqpoint{0.041667in}{0.000000in}}%
\pgfpathcurveto{\pgfqpoint{0.041667in}{0.011050in}}{\pgfqpoint{0.037276in}{0.021649in}}{\pgfqpoint{0.029463in}{0.029463in}}%
\pgfpathcurveto{\pgfqpoint{0.021649in}{0.037276in}}{\pgfqpoint{0.011050in}{0.041667in}}{\pgfqpoint{0.000000in}{0.041667in}}%
\pgfpathcurveto{\pgfqpoint{-0.011050in}{0.041667in}}{\pgfqpoint{-0.021649in}{0.037276in}}{\pgfqpoint{-0.029463in}{0.029463in}}%
\pgfpathcurveto{\pgfqpoint{-0.037276in}{0.021649in}}{\pgfqpoint{-0.041667in}{0.011050in}}{\pgfqpoint{-0.041667in}{0.000000in}}%
\pgfpathcurveto{\pgfqpoint{-0.041667in}{-0.011050in}}{\pgfqpoint{-0.037276in}{-0.021649in}}{\pgfqpoint{-0.029463in}{-0.029463in}}%
\pgfpathcurveto{\pgfqpoint{-0.021649in}{-0.037276in}}{\pgfqpoint{-0.011050in}{-0.041667in}}{\pgfqpoint{0.000000in}{-0.041667in}}%
\pgfpathclose%
\pgfusepath{stroke,fill}%
}%
\begin{pgfscope}%
\pgfsys@transformshift{0.704070in}{4.493740in}%
\pgfsys@useobject{currentmarker}{}%
\end{pgfscope}%
\begin{pgfscope}%
\pgfsys@transformshift{0.740944in}{4.554231in}%
\pgfsys@useobject{currentmarker}{}%
\end{pgfscope}%
\begin{pgfscope}%
\pgfsys@transformshift{0.777818in}{4.497503in}%
\pgfsys@useobject{currentmarker}{}%
\end{pgfscope}%
\begin{pgfscope}%
\pgfsys@transformshift{0.814692in}{4.578014in}%
\pgfsys@useobject{currentmarker}{}%
\end{pgfscope}%
\begin{pgfscope}%
\pgfsys@transformshift{0.851566in}{4.485134in}%
\pgfsys@useobject{currentmarker}{}%
\end{pgfscope}%
\begin{pgfscope}%
\pgfsys@transformshift{0.888441in}{4.550741in}%
\pgfsys@useobject{currentmarker}{}%
\end{pgfscope}%
\begin{pgfscope}%
\pgfsys@transformshift{0.925315in}{4.533035in}%
\pgfsys@useobject{currentmarker}{}%
\end{pgfscope}%
\begin{pgfscope}%
\pgfsys@transformshift{0.962189in}{4.592296in}%
\pgfsys@useobject{currentmarker}{}%
\end{pgfscope}%
\begin{pgfscope}%
\pgfsys@transformshift{0.999063in}{4.544493in}%
\pgfsys@useobject{currentmarker}{}%
\end{pgfscope}%
\begin{pgfscope}%
\pgfsys@transformshift{1.035938in}{4.563799in}%
\pgfsys@useobject{currentmarker}{}%
\end{pgfscope}%
\begin{pgfscope}%
\pgfsys@transformshift{1.072812in}{4.545125in}%
\pgfsys@useobject{currentmarker}{}%
\end{pgfscope}%
\begin{pgfscope}%
\pgfsys@transformshift{1.109686in}{4.499276in}%
\pgfsys@useobject{currentmarker}{}%
\end{pgfscope}%
\begin{pgfscope}%
\pgfsys@transformshift{1.146560in}{4.501642in}%
\pgfsys@useobject{currentmarker}{}%
\end{pgfscope}%
\begin{pgfscope}%
\pgfsys@transformshift{1.183435in}{4.601893in}%
\pgfsys@useobject{currentmarker}{}%
\end{pgfscope}%
\begin{pgfscope}%
\pgfsys@transformshift{1.220309in}{4.467549in}%
\pgfsys@useobject{currentmarker}{}%
\end{pgfscope}%
\begin{pgfscope}%
\pgfsys@transformshift{1.257183in}{4.536756in}%
\pgfsys@useobject{currentmarker}{}%
\end{pgfscope}%
\begin{pgfscope}%
\pgfsys@transformshift{1.294057in}{4.502513in}%
\pgfsys@useobject{currentmarker}{}%
\end{pgfscope}%
\begin{pgfscope}%
\pgfsys@transformshift{1.330932in}{4.567620in}%
\pgfsys@useobject{currentmarker}{}%
\end{pgfscope}%
\begin{pgfscope}%
\pgfsys@transformshift{1.367806in}{4.588884in}%
\pgfsys@useobject{currentmarker}{}%
\end{pgfscope}%
\begin{pgfscope}%
\pgfsys@transformshift{1.404680in}{4.496476in}%
\pgfsys@useobject{currentmarker}{}%
\end{pgfscope}%
\begin{pgfscope}%
\pgfsys@transformshift{1.441554in}{4.496492in}%
\pgfsys@useobject{currentmarker}{}%
\end{pgfscope}%
\begin{pgfscope}%
\pgfsys@transformshift{1.478428in}{4.578745in}%
\pgfsys@useobject{currentmarker}{}%
\end{pgfscope}%
\begin{pgfscope}%
\pgfsys@transformshift{1.515303in}{4.467925in}%
\pgfsys@useobject{currentmarker}{}%
\end{pgfscope}%
\begin{pgfscope}%
\pgfsys@transformshift{1.552177in}{4.586637in}%
\pgfsys@useobject{currentmarker}{}%
\end{pgfscope}%
\begin{pgfscope}%
\pgfsys@transformshift{1.589051in}{4.533060in}%
\pgfsys@useobject{currentmarker}{}%
\end{pgfscope}%
\begin{pgfscope}%
\pgfsys@transformshift{1.625925in}{4.547959in}%
\pgfsys@useobject{currentmarker}{}%
\end{pgfscope}%
\begin{pgfscope}%
\pgfsys@transformshift{1.662800in}{4.532363in}%
\pgfsys@useobject{currentmarker}{}%
\end{pgfscope}%
\begin{pgfscope}%
\pgfsys@transformshift{1.699674in}{4.556502in}%
\pgfsys@useobject{currentmarker}{}%
\end{pgfscope}%
\begin{pgfscope}%
\pgfsys@transformshift{1.736548in}{4.495154in}%
\pgfsys@useobject{currentmarker}{}%
\end{pgfscope}%
\begin{pgfscope}%
\pgfsys@transformshift{1.773422in}{4.488610in}%
\pgfsys@useobject{currentmarker}{}%
\end{pgfscope}%
\begin{pgfscope}%
\pgfsys@transformshift{1.810297in}{4.495047in}%
\pgfsys@useobject{currentmarker}{}%
\end{pgfscope}%
\begin{pgfscope}%
\pgfsys@transformshift{1.847171in}{4.488993in}%
\pgfsys@useobject{currentmarker}{}%
\end{pgfscope}%
\begin{pgfscope}%
\pgfsys@transformshift{1.884045in}{4.588622in}%
\pgfsys@useobject{currentmarker}{}%
\end{pgfscope}%
\begin{pgfscope}%
\pgfsys@transformshift{1.920919in}{4.540026in}%
\pgfsys@useobject{currentmarker}{}%
\end{pgfscope}%
\begin{pgfscope}%
\pgfsys@transformshift{1.957793in}{4.439366in}%
\pgfsys@useobject{currentmarker}{}%
\end{pgfscope}%
\begin{pgfscope}%
\pgfsys@transformshift{1.994668in}{4.479372in}%
\pgfsys@useobject{currentmarker}{}%
\end{pgfscope}%
\begin{pgfscope}%
\pgfsys@transformshift{2.031542in}{4.518823in}%
\pgfsys@useobject{currentmarker}{}%
\end{pgfscope}%
\begin{pgfscope}%
\pgfsys@transformshift{2.068416in}{4.471840in}%
\pgfsys@useobject{currentmarker}{}%
\end{pgfscope}%
\begin{pgfscope}%
\pgfsys@transformshift{2.105290in}{4.519341in}%
\pgfsys@useobject{currentmarker}{}%
\end{pgfscope}%
\begin{pgfscope}%
\pgfsys@transformshift{2.142165in}{4.476920in}%
\pgfsys@useobject{currentmarker}{}%
\end{pgfscope}%
\begin{pgfscope}%
\pgfsys@transformshift{2.179039in}{4.563100in}%
\pgfsys@useobject{currentmarker}{}%
\end{pgfscope}%
\begin{pgfscope}%
\pgfsys@transformshift{2.215913in}{4.565694in}%
\pgfsys@useobject{currentmarker}{}%
\end{pgfscope}%
\begin{pgfscope}%
\pgfsys@transformshift{2.252787in}{4.515725in}%
\pgfsys@useobject{currentmarker}{}%
\end{pgfscope}%
\begin{pgfscope}%
\pgfsys@transformshift{2.289662in}{4.496220in}%
\pgfsys@useobject{currentmarker}{}%
\end{pgfscope}%
\begin{pgfscope}%
\pgfsys@transformshift{2.326536in}{4.557776in}%
\pgfsys@useobject{currentmarker}{}%
\end{pgfscope}%
\begin{pgfscope}%
\pgfsys@transformshift{2.363410in}{4.546939in}%
\pgfsys@useobject{currentmarker}{}%
\end{pgfscope}%
\begin{pgfscope}%
\pgfsys@transformshift{2.400284in}{4.556569in}%
\pgfsys@useobject{currentmarker}{}%
\end{pgfscope}%
\begin{pgfscope}%
\pgfsys@transformshift{2.437158in}{4.537958in}%
\pgfsys@useobject{currentmarker}{}%
\end{pgfscope}%
\begin{pgfscope}%
\pgfsys@transformshift{2.474033in}{4.570535in}%
\pgfsys@useobject{currentmarker}{}%
\end{pgfscope}%
\begin{pgfscope}%
\pgfsys@transformshift{2.510907in}{4.571808in}%
\pgfsys@useobject{currentmarker}{}%
\end{pgfscope}%
\begin{pgfscope}%
\pgfsys@transformshift{2.547781in}{4.476770in}%
\pgfsys@useobject{currentmarker}{}%
\end{pgfscope}%
\begin{pgfscope}%
\pgfsys@transformshift{2.584655in}{4.486770in}%
\pgfsys@useobject{currentmarker}{}%
\end{pgfscope}%
\begin{pgfscope}%
\pgfsys@transformshift{2.621530in}{4.516272in}%
\pgfsys@useobject{currentmarker}{}%
\end{pgfscope}%
\begin{pgfscope}%
\pgfsys@transformshift{2.658404in}{4.439304in}%
\pgfsys@useobject{currentmarker}{}%
\end{pgfscope}%
\begin{pgfscope}%
\pgfsys@transformshift{2.695278in}{4.472317in}%
\pgfsys@useobject{currentmarker}{}%
\end{pgfscope}%
\begin{pgfscope}%
\pgfsys@transformshift{2.732152in}{4.556497in}%
\pgfsys@useobject{currentmarker}{}%
\end{pgfscope}%
\begin{pgfscope}%
\pgfsys@transformshift{2.769027in}{4.526655in}%
\pgfsys@useobject{currentmarker}{}%
\end{pgfscope}%
\begin{pgfscope}%
\pgfsys@transformshift{2.805901in}{4.543102in}%
\pgfsys@useobject{currentmarker}{}%
\end{pgfscope}%
\begin{pgfscope}%
\pgfsys@transformshift{2.842775in}{4.494399in}%
\pgfsys@useobject{currentmarker}{}%
\end{pgfscope}%
\begin{pgfscope}%
\pgfsys@transformshift{2.879649in}{4.498161in}%
\pgfsys@useobject{currentmarker}{}%
\end{pgfscope}%
\begin{pgfscope}%
\pgfsys@transformshift{2.916524in}{4.437810in}%
\pgfsys@useobject{currentmarker}{}%
\end{pgfscope}%
\begin{pgfscope}%
\pgfsys@transformshift{2.953398in}{4.510559in}%
\pgfsys@useobject{currentmarker}{}%
\end{pgfscope}%
\begin{pgfscope}%
\pgfsys@transformshift{2.990272in}{4.511874in}%
\pgfsys@useobject{currentmarker}{}%
\end{pgfscope}%
\begin{pgfscope}%
\pgfsys@transformshift{3.027146in}{4.553520in}%
\pgfsys@useobject{currentmarker}{}%
\end{pgfscope}%
\begin{pgfscope}%
\pgfsys@transformshift{3.064020in}{4.472677in}%
\pgfsys@useobject{currentmarker}{}%
\end{pgfscope}%
\begin{pgfscope}%
\pgfsys@transformshift{3.100895in}{4.506919in}%
\pgfsys@useobject{currentmarker}{}%
\end{pgfscope}%
\begin{pgfscope}%
\pgfsys@transformshift{3.137769in}{4.480355in}%
\pgfsys@useobject{currentmarker}{}%
\end{pgfscope}%
\begin{pgfscope}%
\pgfsys@transformshift{3.174643in}{4.512485in}%
\pgfsys@useobject{currentmarker}{}%
\end{pgfscope}%
\begin{pgfscope}%
\pgfsys@transformshift{3.211517in}{4.517817in}%
\pgfsys@useobject{currentmarker}{}%
\end{pgfscope}%
\begin{pgfscope}%
\pgfsys@transformshift{3.248392in}{4.463198in}%
\pgfsys@useobject{currentmarker}{}%
\end{pgfscope}%
\begin{pgfscope}%
\pgfsys@transformshift{3.285266in}{4.475442in}%
\pgfsys@useobject{currentmarker}{}%
\end{pgfscope}%
\begin{pgfscope}%
\pgfsys@transformshift{3.322140in}{4.493681in}%
\pgfsys@useobject{currentmarker}{}%
\end{pgfscope}%
\begin{pgfscope}%
\pgfsys@transformshift{3.359014in}{4.491418in}%
\pgfsys@useobject{currentmarker}{}%
\end{pgfscope}%
\begin{pgfscope}%
\pgfsys@transformshift{3.395889in}{4.524318in}%
\pgfsys@useobject{currentmarker}{}%
\end{pgfscope}%
\begin{pgfscope}%
\pgfsys@transformshift{3.432763in}{4.574329in}%
\pgfsys@useobject{currentmarker}{}%
\end{pgfscope}%
\begin{pgfscope}%
\pgfsys@transformshift{3.469637in}{4.474223in}%
\pgfsys@useobject{currentmarker}{}%
\end{pgfscope}%
\begin{pgfscope}%
\pgfsys@transformshift{3.506511in}{4.551424in}%
\pgfsys@useobject{currentmarker}{}%
\end{pgfscope}%
\begin{pgfscope}%
\pgfsys@transformshift{3.543385in}{4.566879in}%
\pgfsys@useobject{currentmarker}{}%
\end{pgfscope}%
\begin{pgfscope}%
\pgfsys@transformshift{3.580260in}{4.509485in}%
\pgfsys@useobject{currentmarker}{}%
\end{pgfscope}%
\begin{pgfscope}%
\pgfsys@transformshift{3.617134in}{4.541057in}%
\pgfsys@useobject{currentmarker}{}%
\end{pgfscope}%
\begin{pgfscope}%
\pgfsys@transformshift{3.654008in}{4.471820in}%
\pgfsys@useobject{currentmarker}{}%
\end{pgfscope}%
\begin{pgfscope}%
\pgfsys@transformshift{3.690882in}{4.477506in}%
\pgfsys@useobject{currentmarker}{}%
\end{pgfscope}%
\begin{pgfscope}%
\pgfsys@transformshift{3.727757in}{4.447147in}%
\pgfsys@useobject{currentmarker}{}%
\end{pgfscope}%
\begin{pgfscope}%
\pgfsys@transformshift{3.764631in}{4.536290in}%
\pgfsys@useobject{currentmarker}{}%
\end{pgfscope}%
\begin{pgfscope}%
\pgfsys@transformshift{3.801505in}{4.506690in}%
\pgfsys@useobject{currentmarker}{}%
\end{pgfscope}%
\begin{pgfscope}%
\pgfsys@transformshift{3.838379in}{4.476542in}%
\pgfsys@useobject{currentmarker}{}%
\end{pgfscope}%
\begin{pgfscope}%
\pgfsys@transformshift{3.875254in}{4.500769in}%
\pgfsys@useobject{currentmarker}{}%
\end{pgfscope}%
\begin{pgfscope}%
\pgfsys@transformshift{3.912128in}{4.570500in}%
\pgfsys@useobject{currentmarker}{}%
\end{pgfscope}%
\begin{pgfscope}%
\pgfsys@transformshift{3.949002in}{4.530163in}%
\pgfsys@useobject{currentmarker}{}%
\end{pgfscope}%
\begin{pgfscope}%
\pgfsys@transformshift{3.985876in}{4.503846in}%
\pgfsys@useobject{currentmarker}{}%
\end{pgfscope}%
\begin{pgfscope}%
\pgfsys@transformshift{4.022750in}{4.485804in}%
\pgfsys@useobject{currentmarker}{}%
\end{pgfscope}%
\begin{pgfscope}%
\pgfsys@transformshift{4.059625in}{4.493266in}%
\pgfsys@useobject{currentmarker}{}%
\end{pgfscope}%
\begin{pgfscope}%
\pgfsys@transformshift{4.096499in}{4.578435in}%
\pgfsys@useobject{currentmarker}{}%
\end{pgfscope}%
\begin{pgfscope}%
\pgfsys@transformshift{4.133373in}{4.490599in}%
\pgfsys@useobject{currentmarker}{}%
\end{pgfscope}%
\begin{pgfscope}%
\pgfsys@transformshift{4.170247in}{4.513224in}%
\pgfsys@useobject{currentmarker}{}%
\end{pgfscope}%
\begin{pgfscope}%
\pgfsys@transformshift{4.207122in}{4.473963in}%
\pgfsys@useobject{currentmarker}{}%
\end{pgfscope}%
\begin{pgfscope}%
\pgfsys@transformshift{4.243996in}{4.552264in}%
\pgfsys@useobject{currentmarker}{}%
\end{pgfscope}%
\begin{pgfscope}%
\pgfsys@transformshift{4.280870in}{4.523042in}%
\pgfsys@useobject{currentmarker}{}%
\end{pgfscope}%
\begin{pgfscope}%
\pgfsys@transformshift{4.317744in}{4.470196in}%
\pgfsys@useobject{currentmarker}{}%
\end{pgfscope}%
\begin{pgfscope}%
\pgfsys@transformshift{4.354619in}{4.492019in}%
\pgfsys@useobject{currentmarker}{}%
\end{pgfscope}%
\begin{pgfscope}%
\pgfsys@transformshift{4.391493in}{4.483102in}%
\pgfsys@useobject{currentmarker}{}%
\end{pgfscope}%
\begin{pgfscope}%
\pgfsys@transformshift{4.428367in}{4.498193in}%
\pgfsys@useobject{currentmarker}{}%
\end{pgfscope}%
\begin{pgfscope}%
\pgfsys@transformshift{4.465241in}{4.487860in}%
\pgfsys@useobject{currentmarker}{}%
\end{pgfscope}%
\begin{pgfscope}%
\pgfsys@transformshift{4.502116in}{4.532723in}%
\pgfsys@useobject{currentmarker}{}%
\end{pgfscope}%
\begin{pgfscope}%
\pgfsys@transformshift{4.538990in}{4.526418in}%
\pgfsys@useobject{currentmarker}{}%
\end{pgfscope}%
\begin{pgfscope}%
\pgfsys@transformshift{4.575864in}{4.503205in}%
\pgfsys@useobject{currentmarker}{}%
\end{pgfscope}%
\begin{pgfscope}%
\pgfsys@transformshift{4.612738in}{4.561001in}%
\pgfsys@useobject{currentmarker}{}%
\end{pgfscope}%
\begin{pgfscope}%
\pgfsys@transformshift{4.649612in}{4.454272in}%
\pgfsys@useobject{currentmarker}{}%
\end{pgfscope}%
\begin{pgfscope}%
\pgfsys@transformshift{4.686487in}{4.484957in}%
\pgfsys@useobject{currentmarker}{}%
\end{pgfscope}%
\begin{pgfscope}%
\pgfsys@transformshift{4.723361in}{4.449191in}%
\pgfsys@useobject{currentmarker}{}%
\end{pgfscope}%
\begin{pgfscope}%
\pgfsys@transformshift{4.760235in}{4.564099in}%
\pgfsys@useobject{currentmarker}{}%
\end{pgfscope}%
\begin{pgfscope}%
\pgfsys@transformshift{4.797109in}{4.519861in}%
\pgfsys@useobject{currentmarker}{}%
\end{pgfscope}%
\begin{pgfscope}%
\pgfsys@transformshift{4.833984in}{4.571802in}%
\pgfsys@useobject{currentmarker}{}%
\end{pgfscope}%
\begin{pgfscope}%
\pgfsys@transformshift{4.870858in}{4.444127in}%
\pgfsys@useobject{currentmarker}{}%
\end{pgfscope}%
\begin{pgfscope}%
\pgfsys@transformshift{4.907732in}{4.553714in}%
\pgfsys@useobject{currentmarker}{}%
\end{pgfscope}%
\begin{pgfscope}%
\pgfsys@transformshift{4.944606in}{4.486754in}%
\pgfsys@useobject{currentmarker}{}%
\end{pgfscope}%
\begin{pgfscope}%
\pgfsys@transformshift{4.981481in}{4.526096in}%
\pgfsys@useobject{currentmarker}{}%
\end{pgfscope}%
\begin{pgfscope}%
\pgfsys@transformshift{5.018355in}{4.488921in}%
\pgfsys@useobject{currentmarker}{}%
\end{pgfscope}%
\begin{pgfscope}%
\pgfsys@transformshift{5.055229in}{4.460972in}%
\pgfsys@useobject{currentmarker}{}%
\end{pgfscope}%
\begin{pgfscope}%
\pgfsys@transformshift{5.092103in}{4.465779in}%
\pgfsys@useobject{currentmarker}{}%
\end{pgfscope}%
\begin{pgfscope}%
\pgfsys@transformshift{5.128977in}{4.425943in}%
\pgfsys@useobject{currentmarker}{}%
\end{pgfscope}%
\begin{pgfscope}%
\pgfsys@transformshift{5.165852in}{4.449160in}%
\pgfsys@useobject{currentmarker}{}%
\end{pgfscope}%
\begin{pgfscope}%
\pgfsys@transformshift{5.202726in}{4.496683in}%
\pgfsys@useobject{currentmarker}{}%
\end{pgfscope}%
\begin{pgfscope}%
\pgfsys@transformshift{5.239600in}{4.477798in}%
\pgfsys@useobject{currentmarker}{}%
\end{pgfscope}%
\begin{pgfscope}%
\pgfsys@transformshift{5.276474in}{4.412240in}%
\pgfsys@useobject{currentmarker}{}%
\end{pgfscope}%
\begin{pgfscope}%
\pgfsys@transformshift{5.313349in}{4.479067in}%
\pgfsys@useobject{currentmarker}{}%
\end{pgfscope}%
\begin{pgfscope}%
\pgfsys@transformshift{5.350223in}{4.427770in}%
\pgfsys@useobject{currentmarker}{}%
\end{pgfscope}%
\begin{pgfscope}%
\pgfsys@transformshift{5.387097in}{4.468374in}%
\pgfsys@useobject{currentmarker}{}%
\end{pgfscope}%
\end{pgfscope}%
\begin{pgfscope}%
\pgfsetrectcap%
\pgfsetmiterjoin%
\pgfsetlinewidth{0.803000pt}%
\definecolor{currentstroke}{rgb}{0.000000,0.000000,0.000000}%
\pgfsetstrokecolor{currentstroke}%
\pgfsetdash{}{0pt}%
\pgfpathmoveto{\pgfqpoint{0.459778in}{4.138363in}}%
\pgfpathlineto{\pgfqpoint{0.459778in}{4.875770in}}%
\pgfusepath{stroke}%
\end{pgfscope}%
\begin{pgfscope}%
\pgfsetrectcap%
\pgfsetmiterjoin%
\pgfsetlinewidth{0.803000pt}%
\definecolor{currentstroke}{rgb}{0.000000,0.000000,0.000000}%
\pgfsetstrokecolor{currentstroke}%
\pgfsetdash{}{0pt}%
\pgfpathmoveto{\pgfqpoint{5.631389in}{4.138363in}}%
\pgfpathlineto{\pgfqpoint{5.631389in}{4.875770in}}%
\pgfusepath{stroke}%
\end{pgfscope}%
\begin{pgfscope}%
\pgfsetrectcap%
\pgfsetmiterjoin%
\pgfsetlinewidth{0.803000pt}%
\definecolor{currentstroke}{rgb}{0.000000,0.000000,0.000000}%
\pgfsetstrokecolor{currentstroke}%
\pgfsetdash{}{0pt}%
\pgfpathmoveto{\pgfqpoint{0.459778in}{4.138363in}}%
\pgfpathlineto{\pgfqpoint{5.631389in}{4.138363in}}%
\pgfusepath{stroke}%
\end{pgfscope}%
\begin{pgfscope}%
\pgfsetrectcap%
\pgfsetmiterjoin%
\pgfsetlinewidth{0.803000pt}%
\definecolor{currentstroke}{rgb}{0.000000,0.000000,0.000000}%
\pgfsetstrokecolor{currentstroke}%
\pgfsetdash{}{0pt}%
\pgfpathmoveto{\pgfqpoint{0.459778in}{4.875770in}}%
\pgfpathlineto{\pgfqpoint{5.631389in}{4.875770in}}%
\pgfusepath{stroke}%
\end{pgfscope}%
\begin{pgfscope}%
\pgfsetbuttcap%
\pgfsetmiterjoin%
\definecolor{currentfill}{rgb}{1.000000,1.000000,1.000000}%
\pgfsetfillcolor{currentfill}%
\pgfsetlinewidth{0.000000pt}%
\definecolor{currentstroke}{rgb}{0.000000,0.000000,0.000000}%
\pgfsetstrokecolor{currentstroke}%
\pgfsetstrokeopacity{0.000000}%
\pgfsetdash{}{0pt}%
\pgfpathmoveto{\pgfqpoint{0.459778in}{3.182744in}}%
\pgfpathlineto{\pgfqpoint{5.631389in}{3.182744in}}%
\pgfpathlineto{\pgfqpoint{5.631389in}{3.920152in}}%
\pgfpathlineto{\pgfqpoint{0.459778in}{3.920152in}}%
\pgfpathclose%
\pgfusepath{fill}%
\end{pgfscope}%
\begin{pgfscope}%
\pgfsetbuttcap%
\pgfsetroundjoin%
\definecolor{currentfill}{rgb}{0.000000,0.000000,0.000000}%
\pgfsetfillcolor{currentfill}%
\pgfsetlinewidth{0.803000pt}%
\definecolor{currentstroke}{rgb}{0.000000,0.000000,0.000000}%
\pgfsetstrokecolor{currentstroke}%
\pgfsetdash{}{0pt}%
\pgfsys@defobject{currentmarker}{\pgfqpoint{0.000000in}{-0.048611in}}{\pgfqpoint{0.000000in}{0.000000in}}{%
\pgfpathmoveto{\pgfqpoint{0.000000in}{0.000000in}}%
\pgfpathlineto{\pgfqpoint{0.000000in}{-0.048611in}}%
\pgfusepath{stroke,fill}%
}%
\begin{pgfscope}%
\pgfsys@transformshift{0.694851in}{3.182744in}%
\pgfsys@useobject{currentmarker}{}%
\end{pgfscope}%
\end{pgfscope}%
\begin{pgfscope}%
\pgfsetbuttcap%
\pgfsetroundjoin%
\definecolor{currentfill}{rgb}{0.000000,0.000000,0.000000}%
\pgfsetfillcolor{currentfill}%
\pgfsetlinewidth{0.803000pt}%
\definecolor{currentstroke}{rgb}{0.000000,0.000000,0.000000}%
\pgfsetstrokecolor{currentstroke}%
\pgfsetdash{}{0pt}%
\pgfsys@defobject{currentmarker}{\pgfqpoint{0.000000in}{-0.048611in}}{\pgfqpoint{0.000000in}{0.000000in}}{%
\pgfpathmoveto{\pgfqpoint{0.000000in}{0.000000in}}%
\pgfpathlineto{\pgfqpoint{0.000000in}{-0.048611in}}%
\pgfusepath{stroke,fill}%
}%
\begin{pgfscope}%
\pgfsys@transformshift{1.282534in}{3.182744in}%
\pgfsys@useobject{currentmarker}{}%
\end{pgfscope}%
\end{pgfscope}%
\begin{pgfscope}%
\pgfsetbuttcap%
\pgfsetroundjoin%
\definecolor{currentfill}{rgb}{0.000000,0.000000,0.000000}%
\pgfsetfillcolor{currentfill}%
\pgfsetlinewidth{0.803000pt}%
\definecolor{currentstroke}{rgb}{0.000000,0.000000,0.000000}%
\pgfsetstrokecolor{currentstroke}%
\pgfsetdash{}{0pt}%
\pgfsys@defobject{currentmarker}{\pgfqpoint{0.000000in}{-0.048611in}}{\pgfqpoint{0.000000in}{0.000000in}}{%
\pgfpathmoveto{\pgfqpoint{0.000000in}{0.000000in}}%
\pgfpathlineto{\pgfqpoint{0.000000in}{-0.048611in}}%
\pgfusepath{stroke,fill}%
}%
\begin{pgfscope}%
\pgfsys@transformshift{1.870217in}{3.182744in}%
\pgfsys@useobject{currentmarker}{}%
\end{pgfscope}%
\end{pgfscope}%
\begin{pgfscope}%
\pgfsetbuttcap%
\pgfsetroundjoin%
\definecolor{currentfill}{rgb}{0.000000,0.000000,0.000000}%
\pgfsetfillcolor{currentfill}%
\pgfsetlinewidth{0.803000pt}%
\definecolor{currentstroke}{rgb}{0.000000,0.000000,0.000000}%
\pgfsetstrokecolor{currentstroke}%
\pgfsetdash{}{0pt}%
\pgfsys@defobject{currentmarker}{\pgfqpoint{0.000000in}{-0.048611in}}{\pgfqpoint{0.000000in}{0.000000in}}{%
\pgfpathmoveto{\pgfqpoint{0.000000in}{0.000000in}}%
\pgfpathlineto{\pgfqpoint{0.000000in}{-0.048611in}}%
\pgfusepath{stroke,fill}%
}%
\begin{pgfscope}%
\pgfsys@transformshift{2.457900in}{3.182744in}%
\pgfsys@useobject{currentmarker}{}%
\end{pgfscope}%
\end{pgfscope}%
\begin{pgfscope}%
\pgfsetbuttcap%
\pgfsetroundjoin%
\definecolor{currentfill}{rgb}{0.000000,0.000000,0.000000}%
\pgfsetfillcolor{currentfill}%
\pgfsetlinewidth{0.803000pt}%
\definecolor{currentstroke}{rgb}{0.000000,0.000000,0.000000}%
\pgfsetstrokecolor{currentstroke}%
\pgfsetdash{}{0pt}%
\pgfsys@defobject{currentmarker}{\pgfqpoint{0.000000in}{-0.048611in}}{\pgfqpoint{0.000000in}{0.000000in}}{%
\pgfpathmoveto{\pgfqpoint{0.000000in}{0.000000in}}%
\pgfpathlineto{\pgfqpoint{0.000000in}{-0.048611in}}%
\pgfusepath{stroke,fill}%
}%
\begin{pgfscope}%
\pgfsys@transformshift{3.045583in}{3.182744in}%
\pgfsys@useobject{currentmarker}{}%
\end{pgfscope}%
\end{pgfscope}%
\begin{pgfscope}%
\pgfsetbuttcap%
\pgfsetroundjoin%
\definecolor{currentfill}{rgb}{0.000000,0.000000,0.000000}%
\pgfsetfillcolor{currentfill}%
\pgfsetlinewidth{0.803000pt}%
\definecolor{currentstroke}{rgb}{0.000000,0.000000,0.000000}%
\pgfsetstrokecolor{currentstroke}%
\pgfsetdash{}{0pt}%
\pgfsys@defobject{currentmarker}{\pgfqpoint{0.000000in}{-0.048611in}}{\pgfqpoint{0.000000in}{0.000000in}}{%
\pgfpathmoveto{\pgfqpoint{0.000000in}{0.000000in}}%
\pgfpathlineto{\pgfqpoint{0.000000in}{-0.048611in}}%
\pgfusepath{stroke,fill}%
}%
\begin{pgfscope}%
\pgfsys@transformshift{3.633266in}{3.182744in}%
\pgfsys@useobject{currentmarker}{}%
\end{pgfscope}%
\end{pgfscope}%
\begin{pgfscope}%
\pgfsetbuttcap%
\pgfsetroundjoin%
\definecolor{currentfill}{rgb}{0.000000,0.000000,0.000000}%
\pgfsetfillcolor{currentfill}%
\pgfsetlinewidth{0.803000pt}%
\definecolor{currentstroke}{rgb}{0.000000,0.000000,0.000000}%
\pgfsetstrokecolor{currentstroke}%
\pgfsetdash{}{0pt}%
\pgfsys@defobject{currentmarker}{\pgfqpoint{0.000000in}{-0.048611in}}{\pgfqpoint{0.000000in}{0.000000in}}{%
\pgfpathmoveto{\pgfqpoint{0.000000in}{0.000000in}}%
\pgfpathlineto{\pgfqpoint{0.000000in}{-0.048611in}}%
\pgfusepath{stroke,fill}%
}%
\begin{pgfscope}%
\pgfsys@transformshift{4.220949in}{3.182744in}%
\pgfsys@useobject{currentmarker}{}%
\end{pgfscope}%
\end{pgfscope}%
\begin{pgfscope}%
\pgfsetbuttcap%
\pgfsetroundjoin%
\definecolor{currentfill}{rgb}{0.000000,0.000000,0.000000}%
\pgfsetfillcolor{currentfill}%
\pgfsetlinewidth{0.803000pt}%
\definecolor{currentstroke}{rgb}{0.000000,0.000000,0.000000}%
\pgfsetstrokecolor{currentstroke}%
\pgfsetdash{}{0pt}%
\pgfsys@defobject{currentmarker}{\pgfqpoint{0.000000in}{-0.048611in}}{\pgfqpoint{0.000000in}{0.000000in}}{%
\pgfpathmoveto{\pgfqpoint{0.000000in}{0.000000in}}%
\pgfpathlineto{\pgfqpoint{0.000000in}{-0.048611in}}%
\pgfusepath{stroke,fill}%
}%
\begin{pgfscope}%
\pgfsys@transformshift{4.808633in}{3.182744in}%
\pgfsys@useobject{currentmarker}{}%
\end{pgfscope}%
\end{pgfscope}%
\begin{pgfscope}%
\pgfsetbuttcap%
\pgfsetroundjoin%
\definecolor{currentfill}{rgb}{0.000000,0.000000,0.000000}%
\pgfsetfillcolor{currentfill}%
\pgfsetlinewidth{0.803000pt}%
\definecolor{currentstroke}{rgb}{0.000000,0.000000,0.000000}%
\pgfsetstrokecolor{currentstroke}%
\pgfsetdash{}{0pt}%
\pgfsys@defobject{currentmarker}{\pgfqpoint{0.000000in}{-0.048611in}}{\pgfqpoint{0.000000in}{0.000000in}}{%
\pgfpathmoveto{\pgfqpoint{0.000000in}{0.000000in}}%
\pgfpathlineto{\pgfqpoint{0.000000in}{-0.048611in}}%
\pgfusepath{stroke,fill}%
}%
\begin{pgfscope}%
\pgfsys@transformshift{5.396316in}{3.182744in}%
\pgfsys@useobject{currentmarker}{}%
\end{pgfscope}%
\end{pgfscope}%
\begin{pgfscope}%
\pgfsetbuttcap%
\pgfsetroundjoin%
\definecolor{currentfill}{rgb}{0.000000,0.000000,0.000000}%
\pgfsetfillcolor{currentfill}%
\pgfsetlinewidth{0.803000pt}%
\definecolor{currentstroke}{rgb}{0.000000,0.000000,0.000000}%
\pgfsetstrokecolor{currentstroke}%
\pgfsetdash{}{0pt}%
\pgfsys@defobject{currentmarker}{\pgfqpoint{-0.048611in}{0.000000in}}{\pgfqpoint{0.000000in}{0.000000in}}{%
\pgfpathmoveto{\pgfqpoint{0.000000in}{0.000000in}}%
\pgfpathlineto{\pgfqpoint{-0.048611in}{0.000000in}}%
\pgfusepath{stroke,fill}%
}%
\begin{pgfscope}%
\pgfsys@transformshift{0.459778in}{3.382337in}%
\pgfsys@useobject{currentmarker}{}%
\end{pgfscope}%
\end{pgfscope}%
\begin{pgfscope}%
\definecolor{textcolor}{rgb}{0.000000,0.000000,0.000000}%
\pgfsetstrokecolor{textcolor}%
\pgfsetfillcolor{textcolor}%
\pgftext[x=0.120000in,y=3.343781in,left,base]{\color{textcolor}\rmfamily\fontsize{8.000000}{9.600000}\selectfont −0.1}%
\end{pgfscope}%
\begin{pgfscope}%
\pgfsetbuttcap%
\pgfsetroundjoin%
\definecolor{currentfill}{rgb}{0.000000,0.000000,0.000000}%
\pgfsetfillcolor{currentfill}%
\pgfsetlinewidth{0.803000pt}%
\definecolor{currentstroke}{rgb}{0.000000,0.000000,0.000000}%
\pgfsetstrokecolor{currentstroke}%
\pgfsetdash{}{0pt}%
\pgfsys@defobject{currentmarker}{\pgfqpoint{-0.048611in}{0.000000in}}{\pgfqpoint{0.000000in}{0.000000in}}{%
\pgfpathmoveto{\pgfqpoint{0.000000in}{0.000000in}}%
\pgfpathlineto{\pgfqpoint{-0.048611in}{0.000000in}}%
\pgfusepath{stroke,fill}%
}%
\begin{pgfscope}%
\pgfsys@transformshift{0.459778in}{3.624446in}%
\pgfsys@useobject{currentmarker}{}%
\end{pgfscope}%
\end{pgfscope}%
\begin{pgfscope}%
\definecolor{textcolor}{rgb}{0.000000,0.000000,0.000000}%
\pgfsetstrokecolor{textcolor}%
\pgfsetfillcolor{textcolor}%
\pgftext[x=0.211778in,y=3.585890in,left,base]{\color{textcolor}\rmfamily\fontsize{8.000000}{9.600000}\selectfont 0.0}%
\end{pgfscope}%
\begin{pgfscope}%
\pgfsetbuttcap%
\pgfsetroundjoin%
\definecolor{currentfill}{rgb}{0.000000,0.000000,0.000000}%
\pgfsetfillcolor{currentfill}%
\pgfsetlinewidth{0.803000pt}%
\definecolor{currentstroke}{rgb}{0.000000,0.000000,0.000000}%
\pgfsetstrokecolor{currentstroke}%
\pgfsetdash{}{0pt}%
\pgfsys@defobject{currentmarker}{\pgfqpoint{-0.048611in}{0.000000in}}{\pgfqpoint{0.000000in}{0.000000in}}{%
\pgfpathmoveto{\pgfqpoint{0.000000in}{0.000000in}}%
\pgfpathlineto{\pgfqpoint{-0.048611in}{0.000000in}}%
\pgfusepath{stroke,fill}%
}%
\begin{pgfscope}%
\pgfsys@transformshift{0.459778in}{3.866555in}%
\pgfsys@useobject{currentmarker}{}%
\end{pgfscope}%
\end{pgfscope}%
\begin{pgfscope}%
\definecolor{textcolor}{rgb}{0.000000,0.000000,0.000000}%
\pgfsetstrokecolor{textcolor}%
\pgfsetfillcolor{textcolor}%
\pgftext[x=0.211778in,y=3.828000in,left,base]{\color{textcolor}\rmfamily\fontsize{8.000000}{9.600000}\selectfont 0.1}%
\end{pgfscope}%
\begin{pgfscope}%
\pgfpathrectangle{\pgfqpoint{0.459778in}{3.182744in}}{\pgfqpoint{5.171611in}{0.737408in}}%
\pgfusepath{clip}%
\pgfsetrectcap%
\pgfsetroundjoin%
\pgfsetlinewidth{1.505625pt}%
\definecolor{currentstroke}{rgb}{1.000000,0.498039,0.054902}%
\pgfsetstrokecolor{currentstroke}%
\pgfsetdash{}{0pt}%
\pgfpathmoveto{\pgfqpoint{0.722507in}{3.622377in}}%
\pgfpathlineto{\pgfqpoint{0.759381in}{3.622377in}}%
\pgfpathlineto{\pgfqpoint{0.759381in}{3.639160in}}%
\pgfpathlineto{\pgfqpoint{0.833129in}{3.639160in}}%
\pgfpathlineto{\pgfqpoint{0.833129in}{3.648011in}}%
\pgfpathlineto{\pgfqpoint{0.906878in}{3.648011in}}%
\pgfpathlineto{\pgfqpoint{0.906878in}{3.608398in}}%
\pgfpathlineto{\pgfqpoint{0.980626in}{3.608398in}}%
\pgfpathlineto{\pgfqpoint{0.980626in}{3.634201in}}%
\pgfpathlineto{\pgfqpoint{1.054375in}{3.634201in}}%
\pgfpathlineto{\pgfqpoint{1.054375in}{3.611110in}}%
\pgfpathlineto{\pgfqpoint{1.128123in}{3.611110in}}%
\pgfpathlineto{\pgfqpoint{1.128123in}{3.643565in}}%
\pgfpathlineto{\pgfqpoint{1.201872in}{3.643565in}}%
\pgfpathlineto{\pgfqpoint{1.201872in}{3.555179in}}%
\pgfpathlineto{\pgfqpoint{1.275620in}{3.555179in}}%
\pgfpathlineto{\pgfqpoint{1.275620in}{3.626744in}}%
\pgfpathlineto{\pgfqpoint{1.349369in}{3.626744in}}%
\pgfpathlineto{\pgfqpoint{1.349369in}{3.559292in}}%
\pgfpathlineto{\pgfqpoint{1.423117in}{3.559292in}}%
\pgfpathlineto{\pgfqpoint{1.423117in}{3.631959in}}%
\pgfpathlineto{\pgfqpoint{1.496866in}{3.631959in}}%
\pgfpathlineto{\pgfqpoint{1.496866in}{3.585185in}}%
\pgfpathlineto{\pgfqpoint{1.570614in}{3.585185in}}%
\pgfpathlineto{\pgfqpoint{1.570614in}{3.589531in}}%
\pgfpathlineto{\pgfqpoint{1.644362in}{3.589531in}}%
\pgfpathlineto{\pgfqpoint{1.644362in}{3.666534in}}%
\pgfpathlineto{\pgfqpoint{1.718111in}{3.666534in}}%
\pgfpathlineto{\pgfqpoint{1.718111in}{3.656675in}}%
\pgfpathlineto{\pgfqpoint{1.791859in}{3.656675in}}%
\pgfpathlineto{\pgfqpoint{1.791859in}{3.573904in}}%
\pgfpathlineto{\pgfqpoint{1.865608in}{3.573904in}}%
\pgfpathlineto{\pgfqpoint{1.865608in}{3.619976in}}%
\pgfpathlineto{\pgfqpoint{1.939356in}{3.619976in}}%
\pgfpathlineto{\pgfqpoint{1.939356in}{3.560091in}}%
\pgfpathlineto{\pgfqpoint{2.013105in}{3.560091in}}%
\pgfpathlineto{\pgfqpoint{2.013105in}{3.589247in}}%
\pgfpathlineto{\pgfqpoint{2.086853in}{3.589247in}}%
\pgfpathlineto{\pgfqpoint{2.086853in}{3.576853in}}%
\pgfpathlineto{\pgfqpoint{2.160602in}{3.576853in}}%
\pgfpathlineto{\pgfqpoint{2.160602in}{3.554811in}}%
\pgfpathlineto{\pgfqpoint{2.234350in}{3.554811in}}%
\pgfpathlineto{\pgfqpoint{2.234350in}{3.548506in}}%
\pgfpathlineto{\pgfqpoint{2.308099in}{3.548506in}}%
\pgfpathlineto{\pgfqpoint{2.308099in}{3.583616in}}%
\pgfpathlineto{\pgfqpoint{2.381847in}{3.583616in}}%
\pgfpathlineto{\pgfqpoint{2.381847in}{3.608668in}}%
\pgfpathlineto{\pgfqpoint{2.455596in}{3.608668in}}%
\pgfpathlineto{\pgfqpoint{2.455596in}{3.539485in}}%
\pgfpathlineto{\pgfqpoint{2.529344in}{3.539485in}}%
\pgfpathlineto{\pgfqpoint{2.529344in}{3.587851in}}%
\pgfpathlineto{\pgfqpoint{2.603093in}{3.587851in}}%
\pgfpathlineto{\pgfqpoint{2.603093in}{3.542096in}}%
\pgfpathlineto{\pgfqpoint{2.676841in}{3.542096in}}%
\pgfpathlineto{\pgfqpoint{2.676841in}{3.500450in}}%
\pgfpathlineto{\pgfqpoint{2.750589in}{3.500450in}}%
\pgfpathlineto{\pgfqpoint{2.750589in}{3.541585in}}%
\pgfpathlineto{\pgfqpoint{2.824338in}{3.541585in}}%
\pgfpathlineto{\pgfqpoint{2.824338in}{3.582329in}}%
\pgfpathlineto{\pgfqpoint{2.898086in}{3.582329in}}%
\pgfpathlineto{\pgfqpoint{2.898086in}{3.560319in}}%
\pgfpathlineto{\pgfqpoint{2.971835in}{3.560319in}}%
\pgfpathlineto{\pgfqpoint{2.971835in}{3.583758in}}%
\pgfpathlineto{\pgfqpoint{3.045583in}{3.583758in}}%
\pgfpathlineto{\pgfqpoint{3.045583in}{3.499976in}}%
\pgfpathlineto{\pgfqpoint{3.119332in}{3.499976in}}%
\pgfpathlineto{\pgfqpoint{3.119332in}{3.607060in}}%
\pgfpathlineto{\pgfqpoint{3.193080in}{3.607060in}}%
\pgfpathlineto{\pgfqpoint{3.193080in}{3.608849in}}%
\pgfpathlineto{\pgfqpoint{3.266829in}{3.608849in}}%
\pgfpathlineto{\pgfqpoint{3.266829in}{3.529228in}}%
\pgfpathlineto{\pgfqpoint{3.340577in}{3.529228in}}%
\pgfpathlineto{\pgfqpoint{3.340577in}{3.548199in}}%
\pgfpathlineto{\pgfqpoint{3.414326in}{3.548199in}}%
\pgfpathlineto{\pgfqpoint{3.414326in}{3.572212in}}%
\pgfpathlineto{\pgfqpoint{3.488074in}{3.572212in}}%
\pgfpathlineto{\pgfqpoint{3.488074in}{3.577542in}}%
\pgfpathlineto{\pgfqpoint{3.561823in}{3.577542in}}%
\pgfpathlineto{\pgfqpoint{3.561823in}{3.546205in}}%
\pgfpathlineto{\pgfqpoint{3.635571in}{3.546205in}}%
\pgfpathlineto{\pgfqpoint{3.635571in}{3.526867in}}%
\pgfpathlineto{\pgfqpoint{3.709320in}{3.526867in}}%
\pgfpathlineto{\pgfqpoint{3.709320in}{3.521551in}}%
\pgfpathlineto{\pgfqpoint{3.783068in}{3.521551in}}%
\pgfpathlineto{\pgfqpoint{3.783068in}{3.571900in}}%
\pgfpathlineto{\pgfqpoint{3.856816in}{3.571900in}}%
\pgfpathlineto{\pgfqpoint{3.856816in}{3.556271in}}%
\pgfpathlineto{\pgfqpoint{3.930565in}{3.556271in}}%
\pgfpathlineto{\pgfqpoint{3.930565in}{3.438113in}}%
\pgfpathlineto{\pgfqpoint{4.004313in}{3.438113in}}%
\pgfpathlineto{\pgfqpoint{4.004313in}{3.538697in}}%
\pgfpathlineto{\pgfqpoint{4.078062in}{3.538697in}}%
\pgfpathlineto{\pgfqpoint{4.078062in}{3.558917in}}%
\pgfpathlineto{\pgfqpoint{4.151810in}{3.558917in}}%
\pgfpathlineto{\pgfqpoint{4.151810in}{3.590013in}}%
\pgfpathlineto{\pgfqpoint{4.225559in}{3.590013in}}%
\pgfpathlineto{\pgfqpoint{4.225559in}{3.481603in}}%
\pgfpathlineto{\pgfqpoint{4.299307in}{3.481603in}}%
\pgfpathlineto{\pgfqpoint{4.299307in}{3.467393in}}%
\pgfpathlineto{\pgfqpoint{4.373056in}{3.467393in}}%
\pgfpathlineto{\pgfqpoint{4.373056in}{3.488887in}}%
\pgfpathlineto{\pgfqpoint{4.446804in}{3.488887in}}%
\pgfpathlineto{\pgfqpoint{4.446804in}{3.456577in}}%
\pgfpathlineto{\pgfqpoint{4.520553in}{3.456577in}}%
\pgfpathlineto{\pgfqpoint{4.520553in}{3.507114in}}%
\pgfpathlineto{\pgfqpoint{4.594301in}{3.507114in}}%
\pgfpathlineto{\pgfqpoint{4.594301in}{3.514833in}}%
\pgfpathlineto{\pgfqpoint{4.668050in}{3.514833in}}%
\pgfpathlineto{\pgfqpoint{4.668050in}{3.484502in}}%
\pgfpathlineto{\pgfqpoint{4.741798in}{3.484502in}}%
\pgfpathlineto{\pgfqpoint{4.741798in}{3.465831in}}%
\pgfpathlineto{\pgfqpoint{4.815546in}{3.465831in}}%
\pgfpathlineto{\pgfqpoint{4.815546in}{3.510730in}}%
\pgfpathlineto{\pgfqpoint{4.889295in}{3.510730in}}%
\pgfpathlineto{\pgfqpoint{4.889295in}{3.520082in}}%
\pgfpathlineto{\pgfqpoint{4.963043in}{3.520082in}}%
\pgfpathlineto{\pgfqpoint{4.963043in}{3.461882in}}%
\pgfpathlineto{\pgfqpoint{5.036792in}{3.461882in}}%
\pgfpathlineto{\pgfqpoint{5.036792in}{3.517096in}}%
\pgfpathlineto{\pgfqpoint{5.110540in}{3.517096in}}%
\pgfpathlineto{\pgfqpoint{5.110540in}{3.451212in}}%
\pgfpathlineto{\pgfqpoint{5.184289in}{3.451212in}}%
\pgfpathlineto{\pgfqpoint{5.184289in}{3.472366in}}%
\pgfpathlineto{\pgfqpoint{5.258037in}{3.472366in}}%
\pgfpathlineto{\pgfqpoint{5.258037in}{3.436362in}}%
\pgfpathlineto{\pgfqpoint{5.331786in}{3.436362in}}%
\pgfpathlineto{\pgfqpoint{5.331786in}{3.508208in}}%
\pgfpathlineto{\pgfqpoint{5.368660in}{3.508208in}}%
\pgfpathlineto{\pgfqpoint{5.368660in}{3.508208in}}%
\pgfusepath{stroke}%
\end{pgfscope}%
\begin{pgfscope}%
\pgfpathrectangle{\pgfqpoint{0.459778in}{3.182744in}}{\pgfqpoint{5.171611in}{0.737408in}}%
\pgfusepath{clip}%
\pgfsetbuttcap%
\pgfsetroundjoin%
\definecolor{currentfill}{rgb}{1.000000,0.498039,0.054902}%
\pgfsetfillcolor{currentfill}%
\pgfsetlinewidth{1.003750pt}%
\definecolor{currentstroke}{rgb}{1.000000,0.498039,0.054902}%
\pgfsetstrokecolor{currentstroke}%
\pgfsetdash{}{0pt}%
\pgfsys@defobject{currentmarker}{\pgfqpoint{-0.041667in}{-0.041667in}}{\pgfqpoint{0.041667in}{0.041667in}}{%
\pgfpathmoveto{\pgfqpoint{0.000000in}{-0.041667in}}%
\pgfpathcurveto{\pgfqpoint{0.011050in}{-0.041667in}}{\pgfqpoint{0.021649in}{-0.037276in}}{\pgfqpoint{0.029463in}{-0.029463in}}%
\pgfpathcurveto{\pgfqpoint{0.037276in}{-0.021649in}}{\pgfqpoint{0.041667in}{-0.011050in}}{\pgfqpoint{0.041667in}{0.000000in}}%
\pgfpathcurveto{\pgfqpoint{0.041667in}{0.011050in}}{\pgfqpoint{0.037276in}{0.021649in}}{\pgfqpoint{0.029463in}{0.029463in}}%
\pgfpathcurveto{\pgfqpoint{0.021649in}{0.037276in}}{\pgfqpoint{0.011050in}{0.041667in}}{\pgfqpoint{0.000000in}{0.041667in}}%
\pgfpathcurveto{\pgfqpoint{-0.011050in}{0.041667in}}{\pgfqpoint{-0.021649in}{0.037276in}}{\pgfqpoint{-0.029463in}{0.029463in}}%
\pgfpathcurveto{\pgfqpoint{-0.037276in}{0.021649in}}{\pgfqpoint{-0.041667in}{0.011050in}}{\pgfqpoint{-0.041667in}{0.000000in}}%
\pgfpathcurveto{\pgfqpoint{-0.041667in}{-0.011050in}}{\pgfqpoint{-0.037276in}{-0.021649in}}{\pgfqpoint{-0.029463in}{-0.029463in}}%
\pgfpathcurveto{\pgfqpoint{-0.021649in}{-0.037276in}}{\pgfqpoint{-0.011050in}{-0.041667in}}{\pgfqpoint{0.000000in}{-0.041667in}}%
\pgfpathclose%
\pgfusepath{stroke,fill}%
}%
\begin{pgfscope}%
\pgfsys@transformshift{0.722507in}{3.622377in}%
\pgfsys@useobject{currentmarker}{}%
\end{pgfscope}%
\begin{pgfscope}%
\pgfsys@transformshift{0.796255in}{3.639160in}%
\pgfsys@useobject{currentmarker}{}%
\end{pgfscope}%
\begin{pgfscope}%
\pgfsys@transformshift{0.870004in}{3.648011in}%
\pgfsys@useobject{currentmarker}{}%
\end{pgfscope}%
\begin{pgfscope}%
\pgfsys@transformshift{0.943752in}{3.608398in}%
\pgfsys@useobject{currentmarker}{}%
\end{pgfscope}%
\begin{pgfscope}%
\pgfsys@transformshift{1.017501in}{3.634201in}%
\pgfsys@useobject{currentmarker}{}%
\end{pgfscope}%
\begin{pgfscope}%
\pgfsys@transformshift{1.091249in}{3.611110in}%
\pgfsys@useobject{currentmarker}{}%
\end{pgfscope}%
\begin{pgfscope}%
\pgfsys@transformshift{1.164997in}{3.643565in}%
\pgfsys@useobject{currentmarker}{}%
\end{pgfscope}%
\begin{pgfscope}%
\pgfsys@transformshift{1.238746in}{3.555179in}%
\pgfsys@useobject{currentmarker}{}%
\end{pgfscope}%
\begin{pgfscope}%
\pgfsys@transformshift{1.312494in}{3.626744in}%
\pgfsys@useobject{currentmarker}{}%
\end{pgfscope}%
\begin{pgfscope}%
\pgfsys@transformshift{1.386243in}{3.559292in}%
\pgfsys@useobject{currentmarker}{}%
\end{pgfscope}%
\begin{pgfscope}%
\pgfsys@transformshift{1.459991in}{3.631959in}%
\pgfsys@useobject{currentmarker}{}%
\end{pgfscope}%
\begin{pgfscope}%
\pgfsys@transformshift{1.533740in}{3.585185in}%
\pgfsys@useobject{currentmarker}{}%
\end{pgfscope}%
\begin{pgfscope}%
\pgfsys@transformshift{1.607488in}{3.589531in}%
\pgfsys@useobject{currentmarker}{}%
\end{pgfscope}%
\begin{pgfscope}%
\pgfsys@transformshift{1.681237in}{3.666534in}%
\pgfsys@useobject{currentmarker}{}%
\end{pgfscope}%
\begin{pgfscope}%
\pgfsys@transformshift{1.754985in}{3.656675in}%
\pgfsys@useobject{currentmarker}{}%
\end{pgfscope}%
\begin{pgfscope}%
\pgfsys@transformshift{1.828734in}{3.573904in}%
\pgfsys@useobject{currentmarker}{}%
\end{pgfscope}%
\begin{pgfscope}%
\pgfsys@transformshift{1.902482in}{3.619976in}%
\pgfsys@useobject{currentmarker}{}%
\end{pgfscope}%
\begin{pgfscope}%
\pgfsys@transformshift{1.976231in}{3.560091in}%
\pgfsys@useobject{currentmarker}{}%
\end{pgfscope}%
\begin{pgfscope}%
\pgfsys@transformshift{2.049979in}{3.589247in}%
\pgfsys@useobject{currentmarker}{}%
\end{pgfscope}%
\begin{pgfscope}%
\pgfsys@transformshift{2.123728in}{3.576853in}%
\pgfsys@useobject{currentmarker}{}%
\end{pgfscope}%
\begin{pgfscope}%
\pgfsys@transformshift{2.197476in}{3.554811in}%
\pgfsys@useobject{currentmarker}{}%
\end{pgfscope}%
\begin{pgfscope}%
\pgfsys@transformshift{2.271224in}{3.548506in}%
\pgfsys@useobject{currentmarker}{}%
\end{pgfscope}%
\begin{pgfscope}%
\pgfsys@transformshift{2.344973in}{3.583616in}%
\pgfsys@useobject{currentmarker}{}%
\end{pgfscope}%
\begin{pgfscope}%
\pgfsys@transformshift{2.418721in}{3.608668in}%
\pgfsys@useobject{currentmarker}{}%
\end{pgfscope}%
\begin{pgfscope}%
\pgfsys@transformshift{2.492470in}{3.539485in}%
\pgfsys@useobject{currentmarker}{}%
\end{pgfscope}%
\begin{pgfscope}%
\pgfsys@transformshift{2.566218in}{3.587851in}%
\pgfsys@useobject{currentmarker}{}%
\end{pgfscope}%
\begin{pgfscope}%
\pgfsys@transformshift{2.639967in}{3.542096in}%
\pgfsys@useobject{currentmarker}{}%
\end{pgfscope}%
\begin{pgfscope}%
\pgfsys@transformshift{2.713715in}{3.500450in}%
\pgfsys@useobject{currentmarker}{}%
\end{pgfscope}%
\begin{pgfscope}%
\pgfsys@transformshift{2.787464in}{3.541585in}%
\pgfsys@useobject{currentmarker}{}%
\end{pgfscope}%
\begin{pgfscope}%
\pgfsys@transformshift{2.861212in}{3.582329in}%
\pgfsys@useobject{currentmarker}{}%
\end{pgfscope}%
\begin{pgfscope}%
\pgfsys@transformshift{2.934961in}{3.560319in}%
\pgfsys@useobject{currentmarker}{}%
\end{pgfscope}%
\begin{pgfscope}%
\pgfsys@transformshift{3.008709in}{3.583758in}%
\pgfsys@useobject{currentmarker}{}%
\end{pgfscope}%
\begin{pgfscope}%
\pgfsys@transformshift{3.082458in}{3.499976in}%
\pgfsys@useobject{currentmarker}{}%
\end{pgfscope}%
\begin{pgfscope}%
\pgfsys@transformshift{3.156206in}{3.607060in}%
\pgfsys@useobject{currentmarker}{}%
\end{pgfscope}%
\begin{pgfscope}%
\pgfsys@transformshift{3.229954in}{3.608849in}%
\pgfsys@useobject{currentmarker}{}%
\end{pgfscope}%
\begin{pgfscope}%
\pgfsys@transformshift{3.303703in}{3.529228in}%
\pgfsys@useobject{currentmarker}{}%
\end{pgfscope}%
\begin{pgfscope}%
\pgfsys@transformshift{3.377451in}{3.548199in}%
\pgfsys@useobject{currentmarker}{}%
\end{pgfscope}%
\begin{pgfscope}%
\pgfsys@transformshift{3.451200in}{3.572212in}%
\pgfsys@useobject{currentmarker}{}%
\end{pgfscope}%
\begin{pgfscope}%
\pgfsys@transformshift{3.524948in}{3.577542in}%
\pgfsys@useobject{currentmarker}{}%
\end{pgfscope}%
\begin{pgfscope}%
\pgfsys@transformshift{3.598697in}{3.546205in}%
\pgfsys@useobject{currentmarker}{}%
\end{pgfscope}%
\begin{pgfscope}%
\pgfsys@transformshift{3.672445in}{3.526867in}%
\pgfsys@useobject{currentmarker}{}%
\end{pgfscope}%
\begin{pgfscope}%
\pgfsys@transformshift{3.746194in}{3.521551in}%
\pgfsys@useobject{currentmarker}{}%
\end{pgfscope}%
\begin{pgfscope}%
\pgfsys@transformshift{3.819942in}{3.571900in}%
\pgfsys@useobject{currentmarker}{}%
\end{pgfscope}%
\begin{pgfscope}%
\pgfsys@transformshift{3.893691in}{3.556271in}%
\pgfsys@useobject{currentmarker}{}%
\end{pgfscope}%
\begin{pgfscope}%
\pgfsys@transformshift{3.967439in}{3.438113in}%
\pgfsys@useobject{currentmarker}{}%
\end{pgfscope}%
\begin{pgfscope}%
\pgfsys@transformshift{4.041188in}{3.538697in}%
\pgfsys@useobject{currentmarker}{}%
\end{pgfscope}%
\begin{pgfscope}%
\pgfsys@transformshift{4.114936in}{3.558917in}%
\pgfsys@useobject{currentmarker}{}%
\end{pgfscope}%
\begin{pgfscope}%
\pgfsys@transformshift{4.188685in}{3.590013in}%
\pgfsys@useobject{currentmarker}{}%
\end{pgfscope}%
\begin{pgfscope}%
\pgfsys@transformshift{4.262433in}{3.481603in}%
\pgfsys@useobject{currentmarker}{}%
\end{pgfscope}%
\begin{pgfscope}%
\pgfsys@transformshift{4.336181in}{3.467393in}%
\pgfsys@useobject{currentmarker}{}%
\end{pgfscope}%
\begin{pgfscope}%
\pgfsys@transformshift{4.409930in}{3.488887in}%
\pgfsys@useobject{currentmarker}{}%
\end{pgfscope}%
\begin{pgfscope}%
\pgfsys@transformshift{4.483678in}{3.456577in}%
\pgfsys@useobject{currentmarker}{}%
\end{pgfscope}%
\begin{pgfscope}%
\pgfsys@transformshift{4.557427in}{3.507114in}%
\pgfsys@useobject{currentmarker}{}%
\end{pgfscope}%
\begin{pgfscope}%
\pgfsys@transformshift{4.631175in}{3.514833in}%
\pgfsys@useobject{currentmarker}{}%
\end{pgfscope}%
\begin{pgfscope}%
\pgfsys@transformshift{4.704924in}{3.484502in}%
\pgfsys@useobject{currentmarker}{}%
\end{pgfscope}%
\begin{pgfscope}%
\pgfsys@transformshift{4.778672in}{3.465831in}%
\pgfsys@useobject{currentmarker}{}%
\end{pgfscope}%
\begin{pgfscope}%
\pgfsys@transformshift{4.852421in}{3.510730in}%
\pgfsys@useobject{currentmarker}{}%
\end{pgfscope}%
\begin{pgfscope}%
\pgfsys@transformshift{4.926169in}{3.520082in}%
\pgfsys@useobject{currentmarker}{}%
\end{pgfscope}%
\begin{pgfscope}%
\pgfsys@transformshift{4.999918in}{3.461882in}%
\pgfsys@useobject{currentmarker}{}%
\end{pgfscope}%
\begin{pgfscope}%
\pgfsys@transformshift{5.073666in}{3.517096in}%
\pgfsys@useobject{currentmarker}{}%
\end{pgfscope}%
\begin{pgfscope}%
\pgfsys@transformshift{5.147415in}{3.451212in}%
\pgfsys@useobject{currentmarker}{}%
\end{pgfscope}%
\begin{pgfscope}%
\pgfsys@transformshift{5.221163in}{3.472366in}%
\pgfsys@useobject{currentmarker}{}%
\end{pgfscope}%
\begin{pgfscope}%
\pgfsys@transformshift{5.294912in}{3.436362in}%
\pgfsys@useobject{currentmarker}{}%
\end{pgfscope}%
\begin{pgfscope}%
\pgfsys@transformshift{5.368660in}{3.508208in}%
\pgfsys@useobject{currentmarker}{}%
\end{pgfscope}%
\end{pgfscope}%
\begin{pgfscope}%
\pgfsetrectcap%
\pgfsetmiterjoin%
\pgfsetlinewidth{0.803000pt}%
\definecolor{currentstroke}{rgb}{0.000000,0.000000,0.000000}%
\pgfsetstrokecolor{currentstroke}%
\pgfsetdash{}{0pt}%
\pgfpathmoveto{\pgfqpoint{0.459778in}{3.182744in}}%
\pgfpathlineto{\pgfqpoint{0.459778in}{3.920152in}}%
\pgfusepath{stroke}%
\end{pgfscope}%
\begin{pgfscope}%
\pgfsetrectcap%
\pgfsetmiterjoin%
\pgfsetlinewidth{0.803000pt}%
\definecolor{currentstroke}{rgb}{0.000000,0.000000,0.000000}%
\pgfsetstrokecolor{currentstroke}%
\pgfsetdash{}{0pt}%
\pgfpathmoveto{\pgfqpoint{5.631389in}{3.182744in}}%
\pgfpathlineto{\pgfqpoint{5.631389in}{3.920152in}}%
\pgfusepath{stroke}%
\end{pgfscope}%
\begin{pgfscope}%
\pgfsetrectcap%
\pgfsetmiterjoin%
\pgfsetlinewidth{0.803000pt}%
\definecolor{currentstroke}{rgb}{0.000000,0.000000,0.000000}%
\pgfsetstrokecolor{currentstroke}%
\pgfsetdash{}{0pt}%
\pgfpathmoveto{\pgfqpoint{0.459778in}{3.182744in}}%
\pgfpathlineto{\pgfqpoint{5.631389in}{3.182744in}}%
\pgfusepath{stroke}%
\end{pgfscope}%
\begin{pgfscope}%
\pgfsetrectcap%
\pgfsetmiterjoin%
\pgfsetlinewidth{0.803000pt}%
\definecolor{currentstroke}{rgb}{0.000000,0.000000,0.000000}%
\pgfsetstrokecolor{currentstroke}%
\pgfsetdash{}{0pt}%
\pgfpathmoveto{\pgfqpoint{0.459778in}{3.920152in}}%
\pgfpathlineto{\pgfqpoint{5.631389in}{3.920152in}}%
\pgfusepath{stroke}%
\end{pgfscope}%
\begin{pgfscope}%
\pgfsetbuttcap%
\pgfsetmiterjoin%
\definecolor{currentfill}{rgb}{1.000000,1.000000,1.000000}%
\pgfsetfillcolor{currentfill}%
\pgfsetlinewidth{0.000000pt}%
\definecolor{currentstroke}{rgb}{0.000000,0.000000,0.000000}%
\pgfsetstrokecolor{currentstroke}%
\pgfsetstrokeopacity{0.000000}%
\pgfsetdash{}{0pt}%
\pgfpathmoveto{\pgfqpoint{0.459778in}{2.227126in}}%
\pgfpathlineto{\pgfqpoint{5.631389in}{2.227126in}}%
\pgfpathlineto{\pgfqpoint{5.631389in}{2.964533in}}%
\pgfpathlineto{\pgfqpoint{0.459778in}{2.964533in}}%
\pgfpathclose%
\pgfusepath{fill}%
\end{pgfscope}%
\begin{pgfscope}%
\pgfsetbuttcap%
\pgfsetroundjoin%
\definecolor{currentfill}{rgb}{0.000000,0.000000,0.000000}%
\pgfsetfillcolor{currentfill}%
\pgfsetlinewidth{0.803000pt}%
\definecolor{currentstroke}{rgb}{0.000000,0.000000,0.000000}%
\pgfsetstrokecolor{currentstroke}%
\pgfsetdash{}{0pt}%
\pgfsys@defobject{currentmarker}{\pgfqpoint{0.000000in}{-0.048611in}}{\pgfqpoint{0.000000in}{0.000000in}}{%
\pgfpathmoveto{\pgfqpoint{0.000000in}{0.000000in}}%
\pgfpathlineto{\pgfqpoint{0.000000in}{-0.048611in}}%
\pgfusepath{stroke,fill}%
}%
\begin{pgfscope}%
\pgfsys@transformshift{0.694851in}{2.227126in}%
\pgfsys@useobject{currentmarker}{}%
\end{pgfscope}%
\end{pgfscope}%
\begin{pgfscope}%
\pgfsetbuttcap%
\pgfsetroundjoin%
\definecolor{currentfill}{rgb}{0.000000,0.000000,0.000000}%
\pgfsetfillcolor{currentfill}%
\pgfsetlinewidth{0.803000pt}%
\definecolor{currentstroke}{rgb}{0.000000,0.000000,0.000000}%
\pgfsetstrokecolor{currentstroke}%
\pgfsetdash{}{0pt}%
\pgfsys@defobject{currentmarker}{\pgfqpoint{0.000000in}{-0.048611in}}{\pgfqpoint{0.000000in}{0.000000in}}{%
\pgfpathmoveto{\pgfqpoint{0.000000in}{0.000000in}}%
\pgfpathlineto{\pgfqpoint{0.000000in}{-0.048611in}}%
\pgfusepath{stroke,fill}%
}%
\begin{pgfscope}%
\pgfsys@transformshift{1.282534in}{2.227126in}%
\pgfsys@useobject{currentmarker}{}%
\end{pgfscope}%
\end{pgfscope}%
\begin{pgfscope}%
\pgfsetbuttcap%
\pgfsetroundjoin%
\definecolor{currentfill}{rgb}{0.000000,0.000000,0.000000}%
\pgfsetfillcolor{currentfill}%
\pgfsetlinewidth{0.803000pt}%
\definecolor{currentstroke}{rgb}{0.000000,0.000000,0.000000}%
\pgfsetstrokecolor{currentstroke}%
\pgfsetdash{}{0pt}%
\pgfsys@defobject{currentmarker}{\pgfqpoint{0.000000in}{-0.048611in}}{\pgfqpoint{0.000000in}{0.000000in}}{%
\pgfpathmoveto{\pgfqpoint{0.000000in}{0.000000in}}%
\pgfpathlineto{\pgfqpoint{0.000000in}{-0.048611in}}%
\pgfusepath{stroke,fill}%
}%
\begin{pgfscope}%
\pgfsys@transformshift{1.870217in}{2.227126in}%
\pgfsys@useobject{currentmarker}{}%
\end{pgfscope}%
\end{pgfscope}%
\begin{pgfscope}%
\pgfsetbuttcap%
\pgfsetroundjoin%
\definecolor{currentfill}{rgb}{0.000000,0.000000,0.000000}%
\pgfsetfillcolor{currentfill}%
\pgfsetlinewidth{0.803000pt}%
\definecolor{currentstroke}{rgb}{0.000000,0.000000,0.000000}%
\pgfsetstrokecolor{currentstroke}%
\pgfsetdash{}{0pt}%
\pgfsys@defobject{currentmarker}{\pgfqpoint{0.000000in}{-0.048611in}}{\pgfqpoint{0.000000in}{0.000000in}}{%
\pgfpathmoveto{\pgfqpoint{0.000000in}{0.000000in}}%
\pgfpathlineto{\pgfqpoint{0.000000in}{-0.048611in}}%
\pgfusepath{stroke,fill}%
}%
\begin{pgfscope}%
\pgfsys@transformshift{2.457900in}{2.227126in}%
\pgfsys@useobject{currentmarker}{}%
\end{pgfscope}%
\end{pgfscope}%
\begin{pgfscope}%
\pgfsetbuttcap%
\pgfsetroundjoin%
\definecolor{currentfill}{rgb}{0.000000,0.000000,0.000000}%
\pgfsetfillcolor{currentfill}%
\pgfsetlinewidth{0.803000pt}%
\definecolor{currentstroke}{rgb}{0.000000,0.000000,0.000000}%
\pgfsetstrokecolor{currentstroke}%
\pgfsetdash{}{0pt}%
\pgfsys@defobject{currentmarker}{\pgfqpoint{0.000000in}{-0.048611in}}{\pgfqpoint{0.000000in}{0.000000in}}{%
\pgfpathmoveto{\pgfqpoint{0.000000in}{0.000000in}}%
\pgfpathlineto{\pgfqpoint{0.000000in}{-0.048611in}}%
\pgfusepath{stroke,fill}%
}%
\begin{pgfscope}%
\pgfsys@transformshift{3.045583in}{2.227126in}%
\pgfsys@useobject{currentmarker}{}%
\end{pgfscope}%
\end{pgfscope}%
\begin{pgfscope}%
\pgfsetbuttcap%
\pgfsetroundjoin%
\definecolor{currentfill}{rgb}{0.000000,0.000000,0.000000}%
\pgfsetfillcolor{currentfill}%
\pgfsetlinewidth{0.803000pt}%
\definecolor{currentstroke}{rgb}{0.000000,0.000000,0.000000}%
\pgfsetstrokecolor{currentstroke}%
\pgfsetdash{}{0pt}%
\pgfsys@defobject{currentmarker}{\pgfqpoint{0.000000in}{-0.048611in}}{\pgfqpoint{0.000000in}{0.000000in}}{%
\pgfpathmoveto{\pgfqpoint{0.000000in}{0.000000in}}%
\pgfpathlineto{\pgfqpoint{0.000000in}{-0.048611in}}%
\pgfusepath{stroke,fill}%
}%
\begin{pgfscope}%
\pgfsys@transformshift{3.633266in}{2.227126in}%
\pgfsys@useobject{currentmarker}{}%
\end{pgfscope}%
\end{pgfscope}%
\begin{pgfscope}%
\pgfsetbuttcap%
\pgfsetroundjoin%
\definecolor{currentfill}{rgb}{0.000000,0.000000,0.000000}%
\pgfsetfillcolor{currentfill}%
\pgfsetlinewidth{0.803000pt}%
\definecolor{currentstroke}{rgb}{0.000000,0.000000,0.000000}%
\pgfsetstrokecolor{currentstroke}%
\pgfsetdash{}{0pt}%
\pgfsys@defobject{currentmarker}{\pgfqpoint{0.000000in}{-0.048611in}}{\pgfqpoint{0.000000in}{0.000000in}}{%
\pgfpathmoveto{\pgfqpoint{0.000000in}{0.000000in}}%
\pgfpathlineto{\pgfqpoint{0.000000in}{-0.048611in}}%
\pgfusepath{stroke,fill}%
}%
\begin{pgfscope}%
\pgfsys@transformshift{4.220949in}{2.227126in}%
\pgfsys@useobject{currentmarker}{}%
\end{pgfscope}%
\end{pgfscope}%
\begin{pgfscope}%
\pgfsetbuttcap%
\pgfsetroundjoin%
\definecolor{currentfill}{rgb}{0.000000,0.000000,0.000000}%
\pgfsetfillcolor{currentfill}%
\pgfsetlinewidth{0.803000pt}%
\definecolor{currentstroke}{rgb}{0.000000,0.000000,0.000000}%
\pgfsetstrokecolor{currentstroke}%
\pgfsetdash{}{0pt}%
\pgfsys@defobject{currentmarker}{\pgfqpoint{0.000000in}{-0.048611in}}{\pgfqpoint{0.000000in}{0.000000in}}{%
\pgfpathmoveto{\pgfqpoint{0.000000in}{0.000000in}}%
\pgfpathlineto{\pgfqpoint{0.000000in}{-0.048611in}}%
\pgfusepath{stroke,fill}%
}%
\begin{pgfscope}%
\pgfsys@transformshift{4.808633in}{2.227126in}%
\pgfsys@useobject{currentmarker}{}%
\end{pgfscope}%
\end{pgfscope}%
\begin{pgfscope}%
\pgfsetbuttcap%
\pgfsetroundjoin%
\definecolor{currentfill}{rgb}{0.000000,0.000000,0.000000}%
\pgfsetfillcolor{currentfill}%
\pgfsetlinewidth{0.803000pt}%
\definecolor{currentstroke}{rgb}{0.000000,0.000000,0.000000}%
\pgfsetstrokecolor{currentstroke}%
\pgfsetdash{}{0pt}%
\pgfsys@defobject{currentmarker}{\pgfqpoint{0.000000in}{-0.048611in}}{\pgfqpoint{0.000000in}{0.000000in}}{%
\pgfpathmoveto{\pgfqpoint{0.000000in}{0.000000in}}%
\pgfpathlineto{\pgfqpoint{0.000000in}{-0.048611in}}%
\pgfusepath{stroke,fill}%
}%
\begin{pgfscope}%
\pgfsys@transformshift{5.396316in}{2.227126in}%
\pgfsys@useobject{currentmarker}{}%
\end{pgfscope}%
\end{pgfscope}%
\begin{pgfscope}%
\pgfsetbuttcap%
\pgfsetroundjoin%
\definecolor{currentfill}{rgb}{0.000000,0.000000,0.000000}%
\pgfsetfillcolor{currentfill}%
\pgfsetlinewidth{0.803000pt}%
\definecolor{currentstroke}{rgb}{0.000000,0.000000,0.000000}%
\pgfsetstrokecolor{currentstroke}%
\pgfsetdash{}{0pt}%
\pgfsys@defobject{currentmarker}{\pgfqpoint{-0.048611in}{0.000000in}}{\pgfqpoint{0.000000in}{0.000000in}}{%
\pgfpathmoveto{\pgfqpoint{0.000000in}{0.000000in}}%
\pgfpathlineto{\pgfqpoint{-0.048611in}{0.000000in}}%
\pgfusepath{stroke,fill}%
}%
\begin{pgfscope}%
\pgfsys@transformshift{0.459778in}{2.559816in}%
\pgfsys@useobject{currentmarker}{}%
\end{pgfscope}%
\end{pgfscope}%
\begin{pgfscope}%
\definecolor{textcolor}{rgb}{0.000000,0.000000,0.000000}%
\pgfsetstrokecolor{textcolor}%
\pgfsetfillcolor{textcolor}%
\pgftext[x=0.120000in,y=2.521260in,left,base]{\color{textcolor}\rmfamily\fontsize{8.000000}{9.600000}\selectfont −0.1}%
\end{pgfscope}%
\begin{pgfscope}%
\pgfsetbuttcap%
\pgfsetroundjoin%
\definecolor{currentfill}{rgb}{0.000000,0.000000,0.000000}%
\pgfsetfillcolor{currentfill}%
\pgfsetlinewidth{0.803000pt}%
\definecolor{currentstroke}{rgb}{0.000000,0.000000,0.000000}%
\pgfsetstrokecolor{currentstroke}%
\pgfsetdash{}{0pt}%
\pgfsys@defobject{currentmarker}{\pgfqpoint{-0.048611in}{0.000000in}}{\pgfqpoint{0.000000in}{0.000000in}}{%
\pgfpathmoveto{\pgfqpoint{0.000000in}{0.000000in}}%
\pgfpathlineto{\pgfqpoint{-0.048611in}{0.000000in}}%
\pgfusepath{stroke,fill}%
}%
\begin{pgfscope}%
\pgfsys@transformshift{0.459778in}{2.907004in}%
\pgfsys@useobject{currentmarker}{}%
\end{pgfscope}%
\end{pgfscope}%
\begin{pgfscope}%
\definecolor{textcolor}{rgb}{0.000000,0.000000,0.000000}%
\pgfsetstrokecolor{textcolor}%
\pgfsetfillcolor{textcolor}%
\pgftext[x=0.211778in,y=2.868448in,left,base]{\color{textcolor}\rmfamily\fontsize{8.000000}{9.600000}\selectfont 0.0}%
\end{pgfscope}%
\begin{pgfscope}%
\pgfpathrectangle{\pgfqpoint{0.459778in}{2.227126in}}{\pgfqpoint{5.171611in}{0.737408in}}%
\pgfusepath{clip}%
\pgfsetrectcap%
\pgfsetroundjoin%
\pgfsetlinewidth{1.505625pt}%
\definecolor{currentstroke}{rgb}{1.000000,0.498039,0.054902}%
\pgfsetstrokecolor{currentstroke}%
\pgfsetdash{}{0pt}%
\pgfpathmoveto{\pgfqpoint{0.759381in}{2.926330in}}%
\pgfpathlineto{\pgfqpoint{0.833129in}{2.926330in}}%
\pgfpathlineto{\pgfqpoint{0.833129in}{2.860397in}}%
\pgfpathlineto{\pgfqpoint{0.980626in}{2.860397in}}%
\pgfpathlineto{\pgfqpoint{0.980626in}{2.931015in}}%
\pgfpathlineto{\pgfqpoint{1.128123in}{2.931015in}}%
\pgfpathlineto{\pgfqpoint{1.128123in}{2.805764in}}%
\pgfpathlineto{\pgfqpoint{1.275620in}{2.805764in}}%
\pgfpathlineto{\pgfqpoint{1.275620in}{2.852430in}}%
\pgfpathlineto{\pgfqpoint{1.423117in}{2.852430in}}%
\pgfpathlineto{\pgfqpoint{1.423117in}{2.777634in}}%
\pgfpathlineto{\pgfqpoint{1.570614in}{2.777634in}}%
\pgfpathlineto{\pgfqpoint{1.570614in}{2.750488in}}%
\pgfpathlineto{\pgfqpoint{1.718111in}{2.750488in}}%
\pgfpathlineto{\pgfqpoint{1.718111in}{2.732877in}}%
\pgfpathlineto{\pgfqpoint{1.865608in}{2.732877in}}%
\pgfpathlineto{\pgfqpoint{1.865608in}{2.708621in}}%
\pgfpathlineto{\pgfqpoint{2.013105in}{2.708621in}}%
\pgfpathlineto{\pgfqpoint{2.013105in}{2.669173in}}%
\pgfpathlineto{\pgfqpoint{2.160602in}{2.669173in}}%
\pgfpathlineto{\pgfqpoint{2.160602in}{2.747880in}}%
\pgfpathlineto{\pgfqpoint{2.308099in}{2.747880in}}%
\pgfpathlineto{\pgfqpoint{2.308099in}{2.677327in}}%
\pgfpathlineto{\pgfqpoint{2.455596in}{2.677327in}}%
\pgfpathlineto{\pgfqpoint{2.455596in}{2.682187in}}%
\pgfpathlineto{\pgfqpoint{2.603093in}{2.682187in}}%
\pgfpathlineto{\pgfqpoint{2.603093in}{2.602606in}}%
\pgfpathlineto{\pgfqpoint{2.750589in}{2.602606in}}%
\pgfpathlineto{\pgfqpoint{2.750589in}{2.622184in}}%
\pgfpathlineto{\pgfqpoint{2.898086in}{2.622184in}}%
\pgfpathlineto{\pgfqpoint{2.898086in}{2.523317in}}%
\pgfpathlineto{\pgfqpoint{3.045583in}{2.523317in}}%
\pgfpathlineto{\pgfqpoint{3.045583in}{2.666747in}}%
\pgfpathlineto{\pgfqpoint{3.193080in}{2.666747in}}%
\pgfpathlineto{\pgfqpoint{3.193080in}{2.666871in}}%
\pgfpathlineto{\pgfqpoint{3.340577in}{2.666871in}}%
\pgfpathlineto{\pgfqpoint{3.340577in}{2.628193in}}%
\pgfpathlineto{\pgfqpoint{3.488074in}{2.628193in}}%
\pgfpathlineto{\pgfqpoint{3.488074in}{2.453158in}}%
\pgfpathlineto{\pgfqpoint{3.635571in}{2.453158in}}%
\pgfpathlineto{\pgfqpoint{3.635571in}{2.576925in}}%
\pgfpathlineto{\pgfqpoint{3.783068in}{2.576925in}}%
\pgfpathlineto{\pgfqpoint{3.783068in}{2.456207in}}%
\pgfpathlineto{\pgfqpoint{3.930565in}{2.456207in}}%
\pgfpathlineto{\pgfqpoint{3.930565in}{2.428274in}}%
\pgfpathlineto{\pgfqpoint{4.078062in}{2.428274in}}%
\pgfpathlineto{\pgfqpoint{4.078062in}{2.404503in}}%
\pgfpathlineto{\pgfqpoint{4.225559in}{2.404503in}}%
\pgfpathlineto{\pgfqpoint{4.225559in}{2.441093in}}%
\pgfpathlineto{\pgfqpoint{4.373056in}{2.441093in}}%
\pgfpathlineto{\pgfqpoint{4.373056in}{2.368687in}}%
\pgfpathlineto{\pgfqpoint{4.520553in}{2.368687in}}%
\pgfpathlineto{\pgfqpoint{4.520553in}{2.488748in}}%
\pgfpathlineto{\pgfqpoint{4.668050in}{2.488748in}}%
\pgfpathlineto{\pgfqpoint{4.668050in}{2.260644in}}%
\pgfpathlineto{\pgfqpoint{4.815546in}{2.260644in}}%
\pgfpathlineto{\pgfqpoint{4.815546in}{2.349388in}}%
\pgfpathlineto{\pgfqpoint{4.963043in}{2.349388in}}%
\pgfpathlineto{\pgfqpoint{4.963043in}{2.439864in}}%
\pgfpathlineto{\pgfqpoint{5.110540in}{2.439864in}}%
\pgfpathlineto{\pgfqpoint{5.110540in}{2.315840in}}%
\pgfpathlineto{\pgfqpoint{5.258037in}{2.315840in}}%
\pgfpathlineto{\pgfqpoint{5.258037in}{2.415792in}}%
\pgfpathlineto{\pgfqpoint{5.331786in}{2.415792in}}%
\pgfusepath{stroke}%
\end{pgfscope}%
\begin{pgfscope}%
\pgfpathrectangle{\pgfqpoint{0.459778in}{2.227126in}}{\pgfqpoint{5.171611in}{0.737408in}}%
\pgfusepath{clip}%
\pgfsetbuttcap%
\pgfsetroundjoin%
\definecolor{currentfill}{rgb}{1.000000,0.498039,0.054902}%
\pgfsetfillcolor{currentfill}%
\pgfsetlinewidth{1.003750pt}%
\definecolor{currentstroke}{rgb}{1.000000,0.498039,0.054902}%
\pgfsetstrokecolor{currentstroke}%
\pgfsetdash{}{0pt}%
\pgfsys@defobject{currentmarker}{\pgfqpoint{-0.041667in}{-0.041667in}}{\pgfqpoint{0.041667in}{0.041667in}}{%
\pgfpathmoveto{\pgfqpoint{0.000000in}{-0.041667in}}%
\pgfpathcurveto{\pgfqpoint{0.011050in}{-0.041667in}}{\pgfqpoint{0.021649in}{-0.037276in}}{\pgfqpoint{0.029463in}{-0.029463in}}%
\pgfpathcurveto{\pgfqpoint{0.037276in}{-0.021649in}}{\pgfqpoint{0.041667in}{-0.011050in}}{\pgfqpoint{0.041667in}{0.000000in}}%
\pgfpathcurveto{\pgfqpoint{0.041667in}{0.011050in}}{\pgfqpoint{0.037276in}{0.021649in}}{\pgfqpoint{0.029463in}{0.029463in}}%
\pgfpathcurveto{\pgfqpoint{0.021649in}{0.037276in}}{\pgfqpoint{0.011050in}{0.041667in}}{\pgfqpoint{0.000000in}{0.041667in}}%
\pgfpathcurveto{\pgfqpoint{-0.011050in}{0.041667in}}{\pgfqpoint{-0.021649in}{0.037276in}}{\pgfqpoint{-0.029463in}{0.029463in}}%
\pgfpathcurveto{\pgfqpoint{-0.037276in}{0.021649in}}{\pgfqpoint{-0.041667in}{0.011050in}}{\pgfqpoint{-0.041667in}{0.000000in}}%
\pgfpathcurveto{\pgfqpoint{-0.041667in}{-0.011050in}}{\pgfqpoint{-0.037276in}{-0.021649in}}{\pgfqpoint{-0.029463in}{-0.029463in}}%
\pgfpathcurveto{\pgfqpoint{-0.021649in}{-0.037276in}}{\pgfqpoint{-0.011050in}{-0.041667in}}{\pgfqpoint{0.000000in}{-0.041667in}}%
\pgfpathclose%
\pgfusepath{stroke,fill}%
}%
\begin{pgfscope}%
\pgfsys@transformshift{0.759381in}{2.926330in}%
\pgfsys@useobject{currentmarker}{}%
\end{pgfscope}%
\begin{pgfscope}%
\pgfsys@transformshift{0.906878in}{2.860397in}%
\pgfsys@useobject{currentmarker}{}%
\end{pgfscope}%
\begin{pgfscope}%
\pgfsys@transformshift{1.054375in}{2.931015in}%
\pgfsys@useobject{currentmarker}{}%
\end{pgfscope}%
\begin{pgfscope}%
\pgfsys@transformshift{1.201872in}{2.805764in}%
\pgfsys@useobject{currentmarker}{}%
\end{pgfscope}%
\begin{pgfscope}%
\pgfsys@transformshift{1.349369in}{2.852430in}%
\pgfsys@useobject{currentmarker}{}%
\end{pgfscope}%
\begin{pgfscope}%
\pgfsys@transformshift{1.496866in}{2.777634in}%
\pgfsys@useobject{currentmarker}{}%
\end{pgfscope}%
\begin{pgfscope}%
\pgfsys@transformshift{1.644362in}{2.750488in}%
\pgfsys@useobject{currentmarker}{}%
\end{pgfscope}%
\begin{pgfscope}%
\pgfsys@transformshift{1.791859in}{2.732877in}%
\pgfsys@useobject{currentmarker}{}%
\end{pgfscope}%
\begin{pgfscope}%
\pgfsys@transformshift{1.939356in}{2.708621in}%
\pgfsys@useobject{currentmarker}{}%
\end{pgfscope}%
\begin{pgfscope}%
\pgfsys@transformshift{2.086853in}{2.669173in}%
\pgfsys@useobject{currentmarker}{}%
\end{pgfscope}%
\begin{pgfscope}%
\pgfsys@transformshift{2.234350in}{2.747880in}%
\pgfsys@useobject{currentmarker}{}%
\end{pgfscope}%
\begin{pgfscope}%
\pgfsys@transformshift{2.381847in}{2.677327in}%
\pgfsys@useobject{currentmarker}{}%
\end{pgfscope}%
\begin{pgfscope}%
\pgfsys@transformshift{2.529344in}{2.682187in}%
\pgfsys@useobject{currentmarker}{}%
\end{pgfscope}%
\begin{pgfscope}%
\pgfsys@transformshift{2.676841in}{2.602606in}%
\pgfsys@useobject{currentmarker}{}%
\end{pgfscope}%
\begin{pgfscope}%
\pgfsys@transformshift{2.824338in}{2.622184in}%
\pgfsys@useobject{currentmarker}{}%
\end{pgfscope}%
\begin{pgfscope}%
\pgfsys@transformshift{2.971835in}{2.523317in}%
\pgfsys@useobject{currentmarker}{}%
\end{pgfscope}%
\begin{pgfscope}%
\pgfsys@transformshift{3.119332in}{2.666747in}%
\pgfsys@useobject{currentmarker}{}%
\end{pgfscope}%
\begin{pgfscope}%
\pgfsys@transformshift{3.266829in}{2.666871in}%
\pgfsys@useobject{currentmarker}{}%
\end{pgfscope}%
\begin{pgfscope}%
\pgfsys@transformshift{3.414326in}{2.628193in}%
\pgfsys@useobject{currentmarker}{}%
\end{pgfscope}%
\begin{pgfscope}%
\pgfsys@transformshift{3.561823in}{2.453158in}%
\pgfsys@useobject{currentmarker}{}%
\end{pgfscope}%
\begin{pgfscope}%
\pgfsys@transformshift{3.709320in}{2.576925in}%
\pgfsys@useobject{currentmarker}{}%
\end{pgfscope}%
\begin{pgfscope}%
\pgfsys@transformshift{3.856816in}{2.456207in}%
\pgfsys@useobject{currentmarker}{}%
\end{pgfscope}%
\begin{pgfscope}%
\pgfsys@transformshift{4.004313in}{2.428274in}%
\pgfsys@useobject{currentmarker}{}%
\end{pgfscope}%
\begin{pgfscope}%
\pgfsys@transformshift{4.151810in}{2.404503in}%
\pgfsys@useobject{currentmarker}{}%
\end{pgfscope}%
\begin{pgfscope}%
\pgfsys@transformshift{4.299307in}{2.441093in}%
\pgfsys@useobject{currentmarker}{}%
\end{pgfscope}%
\begin{pgfscope}%
\pgfsys@transformshift{4.446804in}{2.368687in}%
\pgfsys@useobject{currentmarker}{}%
\end{pgfscope}%
\begin{pgfscope}%
\pgfsys@transformshift{4.594301in}{2.488748in}%
\pgfsys@useobject{currentmarker}{}%
\end{pgfscope}%
\begin{pgfscope}%
\pgfsys@transformshift{4.741798in}{2.260644in}%
\pgfsys@useobject{currentmarker}{}%
\end{pgfscope}%
\begin{pgfscope}%
\pgfsys@transformshift{4.889295in}{2.349388in}%
\pgfsys@useobject{currentmarker}{}%
\end{pgfscope}%
\begin{pgfscope}%
\pgfsys@transformshift{5.036792in}{2.439864in}%
\pgfsys@useobject{currentmarker}{}%
\end{pgfscope}%
\begin{pgfscope}%
\pgfsys@transformshift{5.184289in}{2.315840in}%
\pgfsys@useobject{currentmarker}{}%
\end{pgfscope}%
\begin{pgfscope}%
\pgfsys@transformshift{5.331786in}{2.415792in}%
\pgfsys@useobject{currentmarker}{}%
\end{pgfscope}%
\end{pgfscope}%
\begin{pgfscope}%
\pgfsetrectcap%
\pgfsetmiterjoin%
\pgfsetlinewidth{0.803000pt}%
\definecolor{currentstroke}{rgb}{0.000000,0.000000,0.000000}%
\pgfsetstrokecolor{currentstroke}%
\pgfsetdash{}{0pt}%
\pgfpathmoveto{\pgfqpoint{0.459778in}{2.227126in}}%
\pgfpathlineto{\pgfqpoint{0.459778in}{2.964533in}}%
\pgfusepath{stroke}%
\end{pgfscope}%
\begin{pgfscope}%
\pgfsetrectcap%
\pgfsetmiterjoin%
\pgfsetlinewidth{0.803000pt}%
\definecolor{currentstroke}{rgb}{0.000000,0.000000,0.000000}%
\pgfsetstrokecolor{currentstroke}%
\pgfsetdash{}{0pt}%
\pgfpathmoveto{\pgfqpoint{5.631389in}{2.227126in}}%
\pgfpathlineto{\pgfqpoint{5.631389in}{2.964533in}}%
\pgfusepath{stroke}%
\end{pgfscope}%
\begin{pgfscope}%
\pgfsetrectcap%
\pgfsetmiterjoin%
\pgfsetlinewidth{0.803000pt}%
\definecolor{currentstroke}{rgb}{0.000000,0.000000,0.000000}%
\pgfsetstrokecolor{currentstroke}%
\pgfsetdash{}{0pt}%
\pgfpathmoveto{\pgfqpoint{0.459778in}{2.227126in}}%
\pgfpathlineto{\pgfqpoint{5.631389in}{2.227126in}}%
\pgfusepath{stroke}%
\end{pgfscope}%
\begin{pgfscope}%
\pgfsetrectcap%
\pgfsetmiterjoin%
\pgfsetlinewidth{0.803000pt}%
\definecolor{currentstroke}{rgb}{0.000000,0.000000,0.000000}%
\pgfsetstrokecolor{currentstroke}%
\pgfsetdash{}{0pt}%
\pgfpathmoveto{\pgfqpoint{0.459778in}{2.964533in}}%
\pgfpathlineto{\pgfqpoint{5.631389in}{2.964533in}}%
\pgfusepath{stroke}%
\end{pgfscope}%
\begin{pgfscope}%
\pgfsetbuttcap%
\pgfsetmiterjoin%
\definecolor{currentfill}{rgb}{1.000000,1.000000,1.000000}%
\pgfsetfillcolor{currentfill}%
\pgfsetlinewidth{0.000000pt}%
\definecolor{currentstroke}{rgb}{0.000000,0.000000,0.000000}%
\pgfsetstrokecolor{currentstroke}%
\pgfsetstrokeopacity{0.000000}%
\pgfsetdash{}{0pt}%
\pgfpathmoveto{\pgfqpoint{0.459778in}{1.271507in}}%
\pgfpathlineto{\pgfqpoint{5.631389in}{1.271507in}}%
\pgfpathlineto{\pgfqpoint{5.631389in}{2.008915in}}%
\pgfpathlineto{\pgfqpoint{0.459778in}{2.008915in}}%
\pgfpathclose%
\pgfusepath{fill}%
\end{pgfscope}%
\begin{pgfscope}%
\pgfsetbuttcap%
\pgfsetroundjoin%
\definecolor{currentfill}{rgb}{0.000000,0.000000,0.000000}%
\pgfsetfillcolor{currentfill}%
\pgfsetlinewidth{0.803000pt}%
\definecolor{currentstroke}{rgb}{0.000000,0.000000,0.000000}%
\pgfsetstrokecolor{currentstroke}%
\pgfsetdash{}{0pt}%
\pgfsys@defobject{currentmarker}{\pgfqpoint{0.000000in}{-0.048611in}}{\pgfqpoint{0.000000in}{0.000000in}}{%
\pgfpathmoveto{\pgfqpoint{0.000000in}{0.000000in}}%
\pgfpathlineto{\pgfqpoint{0.000000in}{-0.048611in}}%
\pgfusepath{stroke,fill}%
}%
\begin{pgfscope}%
\pgfsys@transformshift{0.694851in}{1.271507in}%
\pgfsys@useobject{currentmarker}{}%
\end{pgfscope}%
\end{pgfscope}%
\begin{pgfscope}%
\pgfsetbuttcap%
\pgfsetroundjoin%
\definecolor{currentfill}{rgb}{0.000000,0.000000,0.000000}%
\pgfsetfillcolor{currentfill}%
\pgfsetlinewidth{0.803000pt}%
\definecolor{currentstroke}{rgb}{0.000000,0.000000,0.000000}%
\pgfsetstrokecolor{currentstroke}%
\pgfsetdash{}{0pt}%
\pgfsys@defobject{currentmarker}{\pgfqpoint{0.000000in}{-0.048611in}}{\pgfqpoint{0.000000in}{0.000000in}}{%
\pgfpathmoveto{\pgfqpoint{0.000000in}{0.000000in}}%
\pgfpathlineto{\pgfqpoint{0.000000in}{-0.048611in}}%
\pgfusepath{stroke,fill}%
}%
\begin{pgfscope}%
\pgfsys@transformshift{1.282534in}{1.271507in}%
\pgfsys@useobject{currentmarker}{}%
\end{pgfscope}%
\end{pgfscope}%
\begin{pgfscope}%
\pgfsetbuttcap%
\pgfsetroundjoin%
\definecolor{currentfill}{rgb}{0.000000,0.000000,0.000000}%
\pgfsetfillcolor{currentfill}%
\pgfsetlinewidth{0.803000pt}%
\definecolor{currentstroke}{rgb}{0.000000,0.000000,0.000000}%
\pgfsetstrokecolor{currentstroke}%
\pgfsetdash{}{0pt}%
\pgfsys@defobject{currentmarker}{\pgfqpoint{0.000000in}{-0.048611in}}{\pgfqpoint{0.000000in}{0.000000in}}{%
\pgfpathmoveto{\pgfqpoint{0.000000in}{0.000000in}}%
\pgfpathlineto{\pgfqpoint{0.000000in}{-0.048611in}}%
\pgfusepath{stroke,fill}%
}%
\begin{pgfscope}%
\pgfsys@transformshift{1.870217in}{1.271507in}%
\pgfsys@useobject{currentmarker}{}%
\end{pgfscope}%
\end{pgfscope}%
\begin{pgfscope}%
\pgfsetbuttcap%
\pgfsetroundjoin%
\definecolor{currentfill}{rgb}{0.000000,0.000000,0.000000}%
\pgfsetfillcolor{currentfill}%
\pgfsetlinewidth{0.803000pt}%
\definecolor{currentstroke}{rgb}{0.000000,0.000000,0.000000}%
\pgfsetstrokecolor{currentstroke}%
\pgfsetdash{}{0pt}%
\pgfsys@defobject{currentmarker}{\pgfqpoint{0.000000in}{-0.048611in}}{\pgfqpoint{0.000000in}{0.000000in}}{%
\pgfpathmoveto{\pgfqpoint{0.000000in}{0.000000in}}%
\pgfpathlineto{\pgfqpoint{0.000000in}{-0.048611in}}%
\pgfusepath{stroke,fill}%
}%
\begin{pgfscope}%
\pgfsys@transformshift{2.457900in}{1.271507in}%
\pgfsys@useobject{currentmarker}{}%
\end{pgfscope}%
\end{pgfscope}%
\begin{pgfscope}%
\pgfsetbuttcap%
\pgfsetroundjoin%
\definecolor{currentfill}{rgb}{0.000000,0.000000,0.000000}%
\pgfsetfillcolor{currentfill}%
\pgfsetlinewidth{0.803000pt}%
\definecolor{currentstroke}{rgb}{0.000000,0.000000,0.000000}%
\pgfsetstrokecolor{currentstroke}%
\pgfsetdash{}{0pt}%
\pgfsys@defobject{currentmarker}{\pgfqpoint{0.000000in}{-0.048611in}}{\pgfqpoint{0.000000in}{0.000000in}}{%
\pgfpathmoveto{\pgfqpoint{0.000000in}{0.000000in}}%
\pgfpathlineto{\pgfqpoint{0.000000in}{-0.048611in}}%
\pgfusepath{stroke,fill}%
}%
\begin{pgfscope}%
\pgfsys@transformshift{3.045583in}{1.271507in}%
\pgfsys@useobject{currentmarker}{}%
\end{pgfscope}%
\end{pgfscope}%
\begin{pgfscope}%
\pgfsetbuttcap%
\pgfsetroundjoin%
\definecolor{currentfill}{rgb}{0.000000,0.000000,0.000000}%
\pgfsetfillcolor{currentfill}%
\pgfsetlinewidth{0.803000pt}%
\definecolor{currentstroke}{rgb}{0.000000,0.000000,0.000000}%
\pgfsetstrokecolor{currentstroke}%
\pgfsetdash{}{0pt}%
\pgfsys@defobject{currentmarker}{\pgfqpoint{0.000000in}{-0.048611in}}{\pgfqpoint{0.000000in}{0.000000in}}{%
\pgfpathmoveto{\pgfqpoint{0.000000in}{0.000000in}}%
\pgfpathlineto{\pgfqpoint{0.000000in}{-0.048611in}}%
\pgfusepath{stroke,fill}%
}%
\begin{pgfscope}%
\pgfsys@transformshift{3.633266in}{1.271507in}%
\pgfsys@useobject{currentmarker}{}%
\end{pgfscope}%
\end{pgfscope}%
\begin{pgfscope}%
\pgfsetbuttcap%
\pgfsetroundjoin%
\definecolor{currentfill}{rgb}{0.000000,0.000000,0.000000}%
\pgfsetfillcolor{currentfill}%
\pgfsetlinewidth{0.803000pt}%
\definecolor{currentstroke}{rgb}{0.000000,0.000000,0.000000}%
\pgfsetstrokecolor{currentstroke}%
\pgfsetdash{}{0pt}%
\pgfsys@defobject{currentmarker}{\pgfqpoint{0.000000in}{-0.048611in}}{\pgfqpoint{0.000000in}{0.000000in}}{%
\pgfpathmoveto{\pgfqpoint{0.000000in}{0.000000in}}%
\pgfpathlineto{\pgfqpoint{0.000000in}{-0.048611in}}%
\pgfusepath{stroke,fill}%
}%
\begin{pgfscope}%
\pgfsys@transformshift{4.220949in}{1.271507in}%
\pgfsys@useobject{currentmarker}{}%
\end{pgfscope}%
\end{pgfscope}%
\begin{pgfscope}%
\pgfsetbuttcap%
\pgfsetroundjoin%
\definecolor{currentfill}{rgb}{0.000000,0.000000,0.000000}%
\pgfsetfillcolor{currentfill}%
\pgfsetlinewidth{0.803000pt}%
\definecolor{currentstroke}{rgb}{0.000000,0.000000,0.000000}%
\pgfsetstrokecolor{currentstroke}%
\pgfsetdash{}{0pt}%
\pgfsys@defobject{currentmarker}{\pgfqpoint{0.000000in}{-0.048611in}}{\pgfqpoint{0.000000in}{0.000000in}}{%
\pgfpathmoveto{\pgfqpoint{0.000000in}{0.000000in}}%
\pgfpathlineto{\pgfqpoint{0.000000in}{-0.048611in}}%
\pgfusepath{stroke,fill}%
}%
\begin{pgfscope}%
\pgfsys@transformshift{4.808633in}{1.271507in}%
\pgfsys@useobject{currentmarker}{}%
\end{pgfscope}%
\end{pgfscope}%
\begin{pgfscope}%
\pgfsetbuttcap%
\pgfsetroundjoin%
\definecolor{currentfill}{rgb}{0.000000,0.000000,0.000000}%
\pgfsetfillcolor{currentfill}%
\pgfsetlinewidth{0.803000pt}%
\definecolor{currentstroke}{rgb}{0.000000,0.000000,0.000000}%
\pgfsetstrokecolor{currentstroke}%
\pgfsetdash{}{0pt}%
\pgfsys@defobject{currentmarker}{\pgfqpoint{0.000000in}{-0.048611in}}{\pgfqpoint{0.000000in}{0.000000in}}{%
\pgfpathmoveto{\pgfqpoint{0.000000in}{0.000000in}}%
\pgfpathlineto{\pgfqpoint{0.000000in}{-0.048611in}}%
\pgfusepath{stroke,fill}%
}%
\begin{pgfscope}%
\pgfsys@transformshift{5.396316in}{1.271507in}%
\pgfsys@useobject{currentmarker}{}%
\end{pgfscope}%
\end{pgfscope}%
\begin{pgfscope}%
\pgfsetbuttcap%
\pgfsetroundjoin%
\definecolor{currentfill}{rgb}{0.000000,0.000000,0.000000}%
\pgfsetfillcolor{currentfill}%
\pgfsetlinewidth{0.803000pt}%
\definecolor{currentstroke}{rgb}{0.000000,0.000000,0.000000}%
\pgfsetstrokecolor{currentstroke}%
\pgfsetdash{}{0pt}%
\pgfsys@defobject{currentmarker}{\pgfqpoint{-0.048611in}{0.000000in}}{\pgfqpoint{0.000000in}{0.000000in}}{%
\pgfpathmoveto{\pgfqpoint{0.000000in}{0.000000in}}%
\pgfpathlineto{\pgfqpoint{-0.048611in}{0.000000in}}%
\pgfusepath{stroke,fill}%
}%
\begin{pgfscope}%
\pgfsys@transformshift{0.459778in}{1.420485in}%
\pgfsys@useobject{currentmarker}{}%
\end{pgfscope}%
\end{pgfscope}%
\begin{pgfscope}%
\definecolor{textcolor}{rgb}{0.000000,0.000000,0.000000}%
\pgfsetstrokecolor{textcolor}%
\pgfsetfillcolor{textcolor}%
\pgftext[x=0.120000in,y=1.381929in,left,base]{\color{textcolor}\rmfamily\fontsize{8.000000}{9.600000}\selectfont −0.4}%
\end{pgfscope}%
\begin{pgfscope}%
\pgfsetbuttcap%
\pgfsetroundjoin%
\definecolor{currentfill}{rgb}{0.000000,0.000000,0.000000}%
\pgfsetfillcolor{currentfill}%
\pgfsetlinewidth{0.803000pt}%
\definecolor{currentstroke}{rgb}{0.000000,0.000000,0.000000}%
\pgfsetstrokecolor{currentstroke}%
\pgfsetdash{}{0pt}%
\pgfsys@defobject{currentmarker}{\pgfqpoint{-0.048611in}{0.000000in}}{\pgfqpoint{0.000000in}{0.000000in}}{%
\pgfpathmoveto{\pgfqpoint{0.000000in}{0.000000in}}%
\pgfpathlineto{\pgfqpoint{-0.048611in}{0.000000in}}%
\pgfusepath{stroke,fill}%
}%
\begin{pgfscope}%
\pgfsys@transformshift{0.459778in}{1.694485in}%
\pgfsys@useobject{currentmarker}{}%
\end{pgfscope}%
\end{pgfscope}%
\begin{pgfscope}%
\definecolor{textcolor}{rgb}{0.000000,0.000000,0.000000}%
\pgfsetstrokecolor{textcolor}%
\pgfsetfillcolor{textcolor}%
\pgftext[x=0.120000in,y=1.655930in,left,base]{\color{textcolor}\rmfamily\fontsize{8.000000}{9.600000}\selectfont −0.2}%
\end{pgfscope}%
\begin{pgfscope}%
\pgfsetbuttcap%
\pgfsetroundjoin%
\definecolor{currentfill}{rgb}{0.000000,0.000000,0.000000}%
\pgfsetfillcolor{currentfill}%
\pgfsetlinewidth{0.803000pt}%
\definecolor{currentstroke}{rgb}{0.000000,0.000000,0.000000}%
\pgfsetstrokecolor{currentstroke}%
\pgfsetdash{}{0pt}%
\pgfsys@defobject{currentmarker}{\pgfqpoint{-0.048611in}{0.000000in}}{\pgfqpoint{0.000000in}{0.000000in}}{%
\pgfpathmoveto{\pgfqpoint{0.000000in}{0.000000in}}%
\pgfpathlineto{\pgfqpoint{-0.048611in}{0.000000in}}%
\pgfusepath{stroke,fill}%
}%
\begin{pgfscope}%
\pgfsys@transformshift{0.459778in}{1.968486in}%
\pgfsys@useobject{currentmarker}{}%
\end{pgfscope}%
\end{pgfscope}%
\begin{pgfscope}%
\definecolor{textcolor}{rgb}{0.000000,0.000000,0.000000}%
\pgfsetstrokecolor{textcolor}%
\pgfsetfillcolor{textcolor}%
\pgftext[x=0.211778in,y=1.929930in,left,base]{\color{textcolor}\rmfamily\fontsize{8.000000}{9.600000}\selectfont 0.0}%
\end{pgfscope}%
\begin{pgfscope}%
\pgfpathrectangle{\pgfqpoint{0.459778in}{1.271507in}}{\pgfqpoint{5.171611in}{0.737408in}}%
\pgfusepath{clip}%
\pgfsetrectcap%
\pgfsetroundjoin%
\pgfsetlinewidth{1.505625pt}%
\definecolor{currentstroke}{rgb}{1.000000,0.498039,0.054902}%
\pgfsetstrokecolor{currentstroke}%
\pgfsetdash{}{0pt}%
\pgfpathmoveto{\pgfqpoint{0.833129in}{1.975396in}}%
\pgfpathlineto{\pgfqpoint{0.980626in}{1.975396in}}%
\pgfpathlineto{\pgfqpoint{0.980626in}{1.898169in}}%
\pgfpathlineto{\pgfqpoint{1.275620in}{1.898169in}}%
\pgfpathlineto{\pgfqpoint{1.275620in}{1.884325in}}%
\pgfpathlineto{\pgfqpoint{1.570614in}{1.884325in}}%
\pgfpathlineto{\pgfqpoint{1.570614in}{1.806417in}}%
\pgfpathlineto{\pgfqpoint{1.865608in}{1.806417in}}%
\pgfpathlineto{\pgfqpoint{1.865608in}{1.763418in}}%
\pgfpathlineto{\pgfqpoint{2.160602in}{1.763418in}}%
\pgfpathlineto{\pgfqpoint{2.160602in}{1.746315in}}%
\pgfpathlineto{\pgfqpoint{2.455596in}{1.746315in}}%
\pgfpathlineto{\pgfqpoint{2.455596in}{1.692138in}}%
\pgfpathlineto{\pgfqpoint{2.750589in}{1.692138in}}%
\pgfpathlineto{\pgfqpoint{2.750589in}{1.638097in}}%
\pgfpathlineto{\pgfqpoint{3.045583in}{1.638097in}}%
\pgfpathlineto{\pgfqpoint{3.045583in}{1.601124in}}%
\pgfpathlineto{\pgfqpoint{3.340577in}{1.601124in}}%
\pgfpathlineto{\pgfqpoint{3.340577in}{1.544088in}}%
\pgfpathlineto{\pgfqpoint{3.635571in}{1.544088in}}%
\pgfpathlineto{\pgfqpoint{3.635571in}{1.502953in}}%
\pgfpathlineto{\pgfqpoint{3.930565in}{1.502953in}}%
\pgfpathlineto{\pgfqpoint{3.930565in}{1.509086in}}%
\pgfpathlineto{\pgfqpoint{4.225559in}{1.509086in}}%
\pgfpathlineto{\pgfqpoint{4.225559in}{1.456316in}}%
\pgfpathlineto{\pgfqpoint{4.520553in}{1.456316in}}%
\pgfpathlineto{\pgfqpoint{4.520553in}{1.401619in}}%
\pgfpathlineto{\pgfqpoint{4.815546in}{1.401619in}}%
\pgfpathlineto{\pgfqpoint{4.815546in}{1.305026in}}%
\pgfpathlineto{\pgfqpoint{5.110540in}{1.305026in}}%
\pgfpathlineto{\pgfqpoint{5.110540in}{1.323032in}}%
\pgfpathlineto{\pgfqpoint{5.258037in}{1.323032in}}%
\pgfusepath{stroke}%
\end{pgfscope}%
\begin{pgfscope}%
\pgfpathrectangle{\pgfqpoint{0.459778in}{1.271507in}}{\pgfqpoint{5.171611in}{0.737408in}}%
\pgfusepath{clip}%
\pgfsetbuttcap%
\pgfsetroundjoin%
\definecolor{currentfill}{rgb}{1.000000,0.498039,0.054902}%
\pgfsetfillcolor{currentfill}%
\pgfsetlinewidth{1.003750pt}%
\definecolor{currentstroke}{rgb}{1.000000,0.498039,0.054902}%
\pgfsetstrokecolor{currentstroke}%
\pgfsetdash{}{0pt}%
\pgfsys@defobject{currentmarker}{\pgfqpoint{-0.041667in}{-0.041667in}}{\pgfqpoint{0.041667in}{0.041667in}}{%
\pgfpathmoveto{\pgfqpoint{0.000000in}{-0.041667in}}%
\pgfpathcurveto{\pgfqpoint{0.011050in}{-0.041667in}}{\pgfqpoint{0.021649in}{-0.037276in}}{\pgfqpoint{0.029463in}{-0.029463in}}%
\pgfpathcurveto{\pgfqpoint{0.037276in}{-0.021649in}}{\pgfqpoint{0.041667in}{-0.011050in}}{\pgfqpoint{0.041667in}{0.000000in}}%
\pgfpathcurveto{\pgfqpoint{0.041667in}{0.011050in}}{\pgfqpoint{0.037276in}{0.021649in}}{\pgfqpoint{0.029463in}{0.029463in}}%
\pgfpathcurveto{\pgfqpoint{0.021649in}{0.037276in}}{\pgfqpoint{0.011050in}{0.041667in}}{\pgfqpoint{0.000000in}{0.041667in}}%
\pgfpathcurveto{\pgfqpoint{-0.011050in}{0.041667in}}{\pgfqpoint{-0.021649in}{0.037276in}}{\pgfqpoint{-0.029463in}{0.029463in}}%
\pgfpathcurveto{\pgfqpoint{-0.037276in}{0.021649in}}{\pgfqpoint{-0.041667in}{0.011050in}}{\pgfqpoint{-0.041667in}{0.000000in}}%
\pgfpathcurveto{\pgfqpoint{-0.041667in}{-0.011050in}}{\pgfqpoint{-0.037276in}{-0.021649in}}{\pgfqpoint{-0.029463in}{-0.029463in}}%
\pgfpathcurveto{\pgfqpoint{-0.021649in}{-0.037276in}}{\pgfqpoint{-0.011050in}{-0.041667in}}{\pgfqpoint{0.000000in}{-0.041667in}}%
\pgfpathclose%
\pgfusepath{stroke,fill}%
}%
\begin{pgfscope}%
\pgfsys@transformshift{0.833129in}{1.975396in}%
\pgfsys@useobject{currentmarker}{}%
\end{pgfscope}%
\begin{pgfscope}%
\pgfsys@transformshift{1.128123in}{1.898169in}%
\pgfsys@useobject{currentmarker}{}%
\end{pgfscope}%
\begin{pgfscope}%
\pgfsys@transformshift{1.423117in}{1.884325in}%
\pgfsys@useobject{currentmarker}{}%
\end{pgfscope}%
\begin{pgfscope}%
\pgfsys@transformshift{1.718111in}{1.806417in}%
\pgfsys@useobject{currentmarker}{}%
\end{pgfscope}%
\begin{pgfscope}%
\pgfsys@transformshift{2.013105in}{1.763418in}%
\pgfsys@useobject{currentmarker}{}%
\end{pgfscope}%
\begin{pgfscope}%
\pgfsys@transformshift{2.308099in}{1.746315in}%
\pgfsys@useobject{currentmarker}{}%
\end{pgfscope}%
\begin{pgfscope}%
\pgfsys@transformshift{2.603093in}{1.692138in}%
\pgfsys@useobject{currentmarker}{}%
\end{pgfscope}%
\begin{pgfscope}%
\pgfsys@transformshift{2.898086in}{1.638097in}%
\pgfsys@useobject{currentmarker}{}%
\end{pgfscope}%
\begin{pgfscope}%
\pgfsys@transformshift{3.193080in}{1.601124in}%
\pgfsys@useobject{currentmarker}{}%
\end{pgfscope}%
\begin{pgfscope}%
\pgfsys@transformshift{3.488074in}{1.544088in}%
\pgfsys@useobject{currentmarker}{}%
\end{pgfscope}%
\begin{pgfscope}%
\pgfsys@transformshift{3.783068in}{1.502953in}%
\pgfsys@useobject{currentmarker}{}%
\end{pgfscope}%
\begin{pgfscope}%
\pgfsys@transformshift{4.078062in}{1.509086in}%
\pgfsys@useobject{currentmarker}{}%
\end{pgfscope}%
\begin{pgfscope}%
\pgfsys@transformshift{4.373056in}{1.456316in}%
\pgfsys@useobject{currentmarker}{}%
\end{pgfscope}%
\begin{pgfscope}%
\pgfsys@transformshift{4.668050in}{1.401619in}%
\pgfsys@useobject{currentmarker}{}%
\end{pgfscope}%
\begin{pgfscope}%
\pgfsys@transformshift{4.963043in}{1.305026in}%
\pgfsys@useobject{currentmarker}{}%
\end{pgfscope}%
\begin{pgfscope}%
\pgfsys@transformshift{5.258037in}{1.323032in}%
\pgfsys@useobject{currentmarker}{}%
\end{pgfscope}%
\end{pgfscope}%
\begin{pgfscope}%
\pgfsetrectcap%
\pgfsetmiterjoin%
\pgfsetlinewidth{0.803000pt}%
\definecolor{currentstroke}{rgb}{0.000000,0.000000,0.000000}%
\pgfsetstrokecolor{currentstroke}%
\pgfsetdash{}{0pt}%
\pgfpathmoveto{\pgfqpoint{0.459778in}{1.271507in}}%
\pgfpathlineto{\pgfqpoint{0.459778in}{2.008915in}}%
\pgfusepath{stroke}%
\end{pgfscope}%
\begin{pgfscope}%
\pgfsetrectcap%
\pgfsetmiterjoin%
\pgfsetlinewidth{0.803000pt}%
\definecolor{currentstroke}{rgb}{0.000000,0.000000,0.000000}%
\pgfsetstrokecolor{currentstroke}%
\pgfsetdash{}{0pt}%
\pgfpathmoveto{\pgfqpoint{5.631389in}{1.271507in}}%
\pgfpathlineto{\pgfqpoint{5.631389in}{2.008915in}}%
\pgfusepath{stroke}%
\end{pgfscope}%
\begin{pgfscope}%
\pgfsetrectcap%
\pgfsetmiterjoin%
\pgfsetlinewidth{0.803000pt}%
\definecolor{currentstroke}{rgb}{0.000000,0.000000,0.000000}%
\pgfsetstrokecolor{currentstroke}%
\pgfsetdash{}{0pt}%
\pgfpathmoveto{\pgfqpoint{0.459778in}{1.271507in}}%
\pgfpathlineto{\pgfqpoint{5.631389in}{1.271507in}}%
\pgfusepath{stroke}%
\end{pgfscope}%
\begin{pgfscope}%
\pgfsetrectcap%
\pgfsetmiterjoin%
\pgfsetlinewidth{0.803000pt}%
\definecolor{currentstroke}{rgb}{0.000000,0.000000,0.000000}%
\pgfsetstrokecolor{currentstroke}%
\pgfsetdash{}{0pt}%
\pgfpathmoveto{\pgfqpoint{0.459778in}{2.008915in}}%
\pgfpathlineto{\pgfqpoint{5.631389in}{2.008915in}}%
\pgfusepath{stroke}%
\end{pgfscope}%
\begin{pgfscope}%
\pgfsetbuttcap%
\pgfsetmiterjoin%
\definecolor{currentfill}{rgb}{1.000000,1.000000,1.000000}%
\pgfsetfillcolor{currentfill}%
\pgfsetlinewidth{0.000000pt}%
\definecolor{currentstroke}{rgb}{0.000000,0.000000,0.000000}%
\pgfsetstrokecolor{currentstroke}%
\pgfsetstrokeopacity{0.000000}%
\pgfsetdash{}{0pt}%
\pgfpathmoveto{\pgfqpoint{0.459778in}{0.315889in}}%
\pgfpathlineto{\pgfqpoint{5.631389in}{0.315889in}}%
\pgfpathlineto{\pgfqpoint{5.631389in}{1.053296in}}%
\pgfpathlineto{\pgfqpoint{0.459778in}{1.053296in}}%
\pgfpathclose%
\pgfusepath{fill}%
\end{pgfscope}%
\begin{pgfscope}%
\pgfsetbuttcap%
\pgfsetroundjoin%
\definecolor{currentfill}{rgb}{0.000000,0.000000,0.000000}%
\pgfsetfillcolor{currentfill}%
\pgfsetlinewidth{0.803000pt}%
\definecolor{currentstroke}{rgb}{0.000000,0.000000,0.000000}%
\pgfsetstrokecolor{currentstroke}%
\pgfsetdash{}{0pt}%
\pgfsys@defobject{currentmarker}{\pgfqpoint{0.000000in}{-0.048611in}}{\pgfqpoint{0.000000in}{0.000000in}}{%
\pgfpathmoveto{\pgfqpoint{0.000000in}{0.000000in}}%
\pgfpathlineto{\pgfqpoint{0.000000in}{-0.048611in}}%
\pgfusepath{stroke,fill}%
}%
\begin{pgfscope}%
\pgfsys@transformshift{0.694851in}{0.315889in}%
\pgfsys@useobject{currentmarker}{}%
\end{pgfscope}%
\end{pgfscope}%
\begin{pgfscope}%
\definecolor{textcolor}{rgb}{0.000000,0.000000,0.000000}%
\pgfsetstrokecolor{textcolor}%
\pgfsetfillcolor{textcolor}%
\pgftext[x=0.694851in,y=0.218667in,,top]{\color{textcolor}\rmfamily\fontsize{8.000000}{9.600000}\selectfont 0.0}%
\end{pgfscope}%
\begin{pgfscope}%
\pgfsetbuttcap%
\pgfsetroundjoin%
\definecolor{currentfill}{rgb}{0.000000,0.000000,0.000000}%
\pgfsetfillcolor{currentfill}%
\pgfsetlinewidth{0.803000pt}%
\definecolor{currentstroke}{rgb}{0.000000,0.000000,0.000000}%
\pgfsetstrokecolor{currentstroke}%
\pgfsetdash{}{0pt}%
\pgfsys@defobject{currentmarker}{\pgfqpoint{0.000000in}{-0.048611in}}{\pgfqpoint{0.000000in}{0.000000in}}{%
\pgfpathmoveto{\pgfqpoint{0.000000in}{0.000000in}}%
\pgfpathlineto{\pgfqpoint{0.000000in}{-0.048611in}}%
\pgfusepath{stroke,fill}%
}%
\begin{pgfscope}%
\pgfsys@transformshift{1.282534in}{0.315889in}%
\pgfsys@useobject{currentmarker}{}%
\end{pgfscope}%
\end{pgfscope}%
\begin{pgfscope}%
\definecolor{textcolor}{rgb}{0.000000,0.000000,0.000000}%
\pgfsetstrokecolor{textcolor}%
\pgfsetfillcolor{textcolor}%
\pgftext[x=1.282534in,y=0.218667in,,top]{\color{textcolor}\rmfamily\fontsize{8.000000}{9.600000}\selectfont 31.9}%
\end{pgfscope}%
\begin{pgfscope}%
\pgfsetbuttcap%
\pgfsetroundjoin%
\definecolor{currentfill}{rgb}{0.000000,0.000000,0.000000}%
\pgfsetfillcolor{currentfill}%
\pgfsetlinewidth{0.803000pt}%
\definecolor{currentstroke}{rgb}{0.000000,0.000000,0.000000}%
\pgfsetstrokecolor{currentstroke}%
\pgfsetdash{}{0pt}%
\pgfsys@defobject{currentmarker}{\pgfqpoint{0.000000in}{-0.048611in}}{\pgfqpoint{0.000000in}{0.000000in}}{%
\pgfpathmoveto{\pgfqpoint{0.000000in}{0.000000in}}%
\pgfpathlineto{\pgfqpoint{0.000000in}{-0.048611in}}%
\pgfusepath{stroke,fill}%
}%
\begin{pgfscope}%
\pgfsys@transformshift{1.870217in}{0.315889in}%
\pgfsys@useobject{currentmarker}{}%
\end{pgfscope}%
\end{pgfscope}%
\begin{pgfscope}%
\definecolor{textcolor}{rgb}{0.000000,0.000000,0.000000}%
\pgfsetstrokecolor{textcolor}%
\pgfsetfillcolor{textcolor}%
\pgftext[x=1.870217in,y=0.218667in,,top]{\color{textcolor}\rmfamily\fontsize{8.000000}{9.600000}\selectfont 63.8}%
\end{pgfscope}%
\begin{pgfscope}%
\pgfsetbuttcap%
\pgfsetroundjoin%
\definecolor{currentfill}{rgb}{0.000000,0.000000,0.000000}%
\pgfsetfillcolor{currentfill}%
\pgfsetlinewidth{0.803000pt}%
\definecolor{currentstroke}{rgb}{0.000000,0.000000,0.000000}%
\pgfsetstrokecolor{currentstroke}%
\pgfsetdash{}{0pt}%
\pgfsys@defobject{currentmarker}{\pgfqpoint{0.000000in}{-0.048611in}}{\pgfqpoint{0.000000in}{0.000000in}}{%
\pgfpathmoveto{\pgfqpoint{0.000000in}{0.000000in}}%
\pgfpathlineto{\pgfqpoint{0.000000in}{-0.048611in}}%
\pgfusepath{stroke,fill}%
}%
\begin{pgfscope}%
\pgfsys@transformshift{2.457900in}{0.315889in}%
\pgfsys@useobject{currentmarker}{}%
\end{pgfscope}%
\end{pgfscope}%
\begin{pgfscope}%
\definecolor{textcolor}{rgb}{0.000000,0.000000,0.000000}%
\pgfsetstrokecolor{textcolor}%
\pgfsetfillcolor{textcolor}%
\pgftext[x=2.457900in,y=0.218667in,,top]{\color{textcolor}\rmfamily\fontsize{8.000000}{9.600000}\selectfont 95.6}%
\end{pgfscope}%
\begin{pgfscope}%
\pgfsetbuttcap%
\pgfsetroundjoin%
\definecolor{currentfill}{rgb}{0.000000,0.000000,0.000000}%
\pgfsetfillcolor{currentfill}%
\pgfsetlinewidth{0.803000pt}%
\definecolor{currentstroke}{rgb}{0.000000,0.000000,0.000000}%
\pgfsetstrokecolor{currentstroke}%
\pgfsetdash{}{0pt}%
\pgfsys@defobject{currentmarker}{\pgfqpoint{0.000000in}{-0.048611in}}{\pgfqpoint{0.000000in}{0.000000in}}{%
\pgfpathmoveto{\pgfqpoint{0.000000in}{0.000000in}}%
\pgfpathlineto{\pgfqpoint{0.000000in}{-0.048611in}}%
\pgfusepath{stroke,fill}%
}%
\begin{pgfscope}%
\pgfsys@transformshift{3.045583in}{0.315889in}%
\pgfsys@useobject{currentmarker}{}%
\end{pgfscope}%
\end{pgfscope}%
\begin{pgfscope}%
\definecolor{textcolor}{rgb}{0.000000,0.000000,0.000000}%
\pgfsetstrokecolor{textcolor}%
\pgfsetfillcolor{textcolor}%
\pgftext[x=3.045583in,y=0.218667in,,top]{\color{textcolor}\rmfamily\fontsize{8.000000}{9.600000}\selectfont 127.5}%
\end{pgfscope}%
\begin{pgfscope}%
\pgfsetbuttcap%
\pgfsetroundjoin%
\definecolor{currentfill}{rgb}{0.000000,0.000000,0.000000}%
\pgfsetfillcolor{currentfill}%
\pgfsetlinewidth{0.803000pt}%
\definecolor{currentstroke}{rgb}{0.000000,0.000000,0.000000}%
\pgfsetstrokecolor{currentstroke}%
\pgfsetdash{}{0pt}%
\pgfsys@defobject{currentmarker}{\pgfqpoint{0.000000in}{-0.048611in}}{\pgfqpoint{0.000000in}{0.000000in}}{%
\pgfpathmoveto{\pgfqpoint{0.000000in}{0.000000in}}%
\pgfpathlineto{\pgfqpoint{0.000000in}{-0.048611in}}%
\pgfusepath{stroke,fill}%
}%
\begin{pgfscope}%
\pgfsys@transformshift{3.633266in}{0.315889in}%
\pgfsys@useobject{currentmarker}{}%
\end{pgfscope}%
\end{pgfscope}%
\begin{pgfscope}%
\definecolor{textcolor}{rgb}{0.000000,0.000000,0.000000}%
\pgfsetstrokecolor{textcolor}%
\pgfsetfillcolor{textcolor}%
\pgftext[x=3.633266in,y=0.218667in,,top]{\color{textcolor}\rmfamily\fontsize{8.000000}{9.600000}\selectfont 159.4}%
\end{pgfscope}%
\begin{pgfscope}%
\pgfsetbuttcap%
\pgfsetroundjoin%
\definecolor{currentfill}{rgb}{0.000000,0.000000,0.000000}%
\pgfsetfillcolor{currentfill}%
\pgfsetlinewidth{0.803000pt}%
\definecolor{currentstroke}{rgb}{0.000000,0.000000,0.000000}%
\pgfsetstrokecolor{currentstroke}%
\pgfsetdash{}{0pt}%
\pgfsys@defobject{currentmarker}{\pgfqpoint{0.000000in}{-0.048611in}}{\pgfqpoint{0.000000in}{0.000000in}}{%
\pgfpathmoveto{\pgfqpoint{0.000000in}{0.000000in}}%
\pgfpathlineto{\pgfqpoint{0.000000in}{-0.048611in}}%
\pgfusepath{stroke,fill}%
}%
\begin{pgfscope}%
\pgfsys@transformshift{4.220949in}{0.315889in}%
\pgfsys@useobject{currentmarker}{}%
\end{pgfscope}%
\end{pgfscope}%
\begin{pgfscope}%
\definecolor{textcolor}{rgb}{0.000000,0.000000,0.000000}%
\pgfsetstrokecolor{textcolor}%
\pgfsetfillcolor{textcolor}%
\pgftext[x=4.220949in,y=0.218667in,,top]{\color{textcolor}\rmfamily\fontsize{8.000000}{9.600000}\selectfont 191.2}%
\end{pgfscope}%
\begin{pgfscope}%
\pgfsetbuttcap%
\pgfsetroundjoin%
\definecolor{currentfill}{rgb}{0.000000,0.000000,0.000000}%
\pgfsetfillcolor{currentfill}%
\pgfsetlinewidth{0.803000pt}%
\definecolor{currentstroke}{rgb}{0.000000,0.000000,0.000000}%
\pgfsetstrokecolor{currentstroke}%
\pgfsetdash{}{0pt}%
\pgfsys@defobject{currentmarker}{\pgfqpoint{0.000000in}{-0.048611in}}{\pgfqpoint{0.000000in}{0.000000in}}{%
\pgfpathmoveto{\pgfqpoint{0.000000in}{0.000000in}}%
\pgfpathlineto{\pgfqpoint{0.000000in}{-0.048611in}}%
\pgfusepath{stroke,fill}%
}%
\begin{pgfscope}%
\pgfsys@transformshift{4.808633in}{0.315889in}%
\pgfsys@useobject{currentmarker}{}%
\end{pgfscope}%
\end{pgfscope}%
\begin{pgfscope}%
\definecolor{textcolor}{rgb}{0.000000,0.000000,0.000000}%
\pgfsetstrokecolor{textcolor}%
\pgfsetfillcolor{textcolor}%
\pgftext[x=4.808633in,y=0.218667in,,top]{\color{textcolor}\rmfamily\fontsize{8.000000}{9.600000}\selectfont 223.1}%
\end{pgfscope}%
\begin{pgfscope}%
\pgfsetbuttcap%
\pgfsetroundjoin%
\definecolor{currentfill}{rgb}{0.000000,0.000000,0.000000}%
\pgfsetfillcolor{currentfill}%
\pgfsetlinewidth{0.803000pt}%
\definecolor{currentstroke}{rgb}{0.000000,0.000000,0.000000}%
\pgfsetstrokecolor{currentstroke}%
\pgfsetdash{}{0pt}%
\pgfsys@defobject{currentmarker}{\pgfqpoint{0.000000in}{-0.048611in}}{\pgfqpoint{0.000000in}{0.000000in}}{%
\pgfpathmoveto{\pgfqpoint{0.000000in}{0.000000in}}%
\pgfpathlineto{\pgfqpoint{0.000000in}{-0.048611in}}%
\pgfusepath{stroke,fill}%
}%
\begin{pgfscope}%
\pgfsys@transformshift{5.396316in}{0.315889in}%
\pgfsys@useobject{currentmarker}{}%
\end{pgfscope}%
\end{pgfscope}%
\begin{pgfscope}%
\definecolor{textcolor}{rgb}{0.000000,0.000000,0.000000}%
\pgfsetstrokecolor{textcolor}%
\pgfsetfillcolor{textcolor}%
\pgftext[x=5.396316in,y=0.218667in,,top]{\color{textcolor}\rmfamily\fontsize{8.000000}{9.600000}\selectfont 255.0}%
\end{pgfscope}%
\begin{pgfscope}%
\pgfsetbuttcap%
\pgfsetroundjoin%
\definecolor{currentfill}{rgb}{0.000000,0.000000,0.000000}%
\pgfsetfillcolor{currentfill}%
\pgfsetlinewidth{0.803000pt}%
\definecolor{currentstroke}{rgb}{0.000000,0.000000,0.000000}%
\pgfsetstrokecolor{currentstroke}%
\pgfsetdash{}{0pt}%
\pgfsys@defobject{currentmarker}{\pgfqpoint{-0.048611in}{0.000000in}}{\pgfqpoint{0.000000in}{0.000000in}}{%
\pgfpathmoveto{\pgfqpoint{0.000000in}{0.000000in}}%
\pgfpathlineto{\pgfqpoint{-0.048611in}{0.000000in}}%
\pgfusepath{stroke,fill}%
}%
\begin{pgfscope}%
\pgfsys@transformshift{0.459778in}{0.345644in}%
\pgfsys@useobject{currentmarker}{}%
\end{pgfscope}%
\end{pgfscope}%
\begin{pgfscope}%
\definecolor{textcolor}{rgb}{0.000000,0.000000,0.000000}%
\pgfsetstrokecolor{textcolor}%
\pgfsetfillcolor{textcolor}%
\pgftext[x=0.303556in,y=0.307089in,left,base]{\color{textcolor}\rmfamily\fontsize{8.000000}{9.600000}\selectfont 0}%
\end{pgfscope}%
\begin{pgfscope}%
\pgfsetbuttcap%
\pgfsetroundjoin%
\definecolor{currentfill}{rgb}{0.000000,0.000000,0.000000}%
\pgfsetfillcolor{currentfill}%
\pgfsetlinewidth{0.803000pt}%
\definecolor{currentstroke}{rgb}{0.000000,0.000000,0.000000}%
\pgfsetstrokecolor{currentstroke}%
\pgfsetdash{}{0pt}%
\pgfsys@defobject{currentmarker}{\pgfqpoint{-0.048611in}{0.000000in}}{\pgfqpoint{0.000000in}{0.000000in}}{%
\pgfpathmoveto{\pgfqpoint{0.000000in}{0.000000in}}%
\pgfpathlineto{\pgfqpoint{-0.048611in}{0.000000in}}%
\pgfusepath{stroke,fill}%
}%
\begin{pgfscope}%
\pgfsys@transformshift{0.459778in}{0.789125in}%
\pgfsys@useobject{currentmarker}{}%
\end{pgfscope}%
\end{pgfscope}%
\begin{pgfscope}%
\definecolor{textcolor}{rgb}{0.000000,0.000000,0.000000}%
\pgfsetstrokecolor{textcolor}%
\pgfsetfillcolor{textcolor}%
\pgftext[x=0.244556in,y=0.750570in,left,base]{\color{textcolor}\rmfamily\fontsize{8.000000}{9.600000}\selectfont 10}%
\end{pgfscope}%
\begin{pgfscope}%
\pgfpathrectangle{\pgfqpoint{0.459778in}{0.315889in}}{\pgfqpoint{5.171611in}{0.737408in}}%
\pgfusepath{clip}%
\pgfsetrectcap%
\pgfsetroundjoin%
\pgfsetlinewidth{1.505625pt}%
\definecolor{currentstroke}{rgb}{0.172549,0.627451,0.172549}%
\pgfsetstrokecolor{currentstroke}%
\pgfsetdash{}{0pt}%
\pgfpathmoveto{\pgfqpoint{0.833129in}{0.349407in}}%
\pgfpathlineto{\pgfqpoint{0.980626in}{0.349407in}}%
\pgfpathlineto{\pgfqpoint{0.980626in}{0.356795in}}%
\pgfpathlineto{\pgfqpoint{1.275620in}{0.356795in}}%
\pgfpathlineto{\pgfqpoint{1.275620in}{0.367787in}}%
\pgfpathlineto{\pgfqpoint{1.570614in}{0.367787in}}%
\pgfpathlineto{\pgfqpoint{1.570614in}{0.384036in}}%
\pgfpathlineto{\pgfqpoint{1.865608in}{0.384036in}}%
\pgfpathlineto{\pgfqpoint{1.865608in}{0.406455in}}%
\pgfpathlineto{\pgfqpoint{2.160602in}{0.406455in}}%
\pgfpathlineto{\pgfqpoint{2.160602in}{0.433028in}}%
\pgfpathlineto{\pgfqpoint{2.455596in}{0.433028in}}%
\pgfpathlineto{\pgfqpoint{2.455596in}{0.466591in}}%
\pgfpathlineto{\pgfqpoint{2.750589in}{0.466591in}}%
\pgfpathlineto{\pgfqpoint{2.750589in}{0.507357in}}%
\pgfpathlineto{\pgfqpoint{3.045583in}{0.507357in}}%
\pgfpathlineto{\pgfqpoint{3.045583in}{0.550208in}}%
\pgfpathlineto{\pgfqpoint{3.340577in}{0.550208in}}%
\pgfpathlineto{\pgfqpoint{3.340577in}{0.601941in}}%
\pgfpathlineto{\pgfqpoint{3.635571in}{0.601941in}}%
\pgfpathlineto{\pgfqpoint{3.635571in}{0.657693in}}%
\pgfpathlineto{\pgfqpoint{3.930565in}{0.657693in}}%
\pgfpathlineto{\pgfqpoint{3.930565in}{0.718653in}}%
\pgfpathlineto{\pgfqpoint{4.225559in}{0.718653in}}%
\pgfpathlineto{\pgfqpoint{4.225559in}{0.782968in}}%
\pgfpathlineto{\pgfqpoint{4.520553in}{0.782968in}}%
\pgfpathlineto{\pgfqpoint{4.520553in}{0.857118in}}%
\pgfpathlineto{\pgfqpoint{4.815546in}{0.857118in}}%
\pgfpathlineto{\pgfqpoint{4.815546in}{0.935343in}}%
\pgfpathlineto{\pgfqpoint{5.110540in}{0.935343in}}%
\pgfpathlineto{\pgfqpoint{5.110540in}{1.019778in}}%
\pgfpathlineto{\pgfqpoint{5.258037in}{1.019778in}}%
\pgfusepath{stroke}%
\end{pgfscope}%
\begin{pgfscope}%
\pgfpathrectangle{\pgfqpoint{0.459778in}{0.315889in}}{\pgfqpoint{5.171611in}{0.737408in}}%
\pgfusepath{clip}%
\pgfsetbuttcap%
\pgfsetroundjoin%
\definecolor{currentfill}{rgb}{0.172549,0.627451,0.172549}%
\pgfsetfillcolor{currentfill}%
\pgfsetlinewidth{1.003750pt}%
\definecolor{currentstroke}{rgb}{0.172549,0.627451,0.172549}%
\pgfsetstrokecolor{currentstroke}%
\pgfsetdash{}{0pt}%
\pgfsys@defobject{currentmarker}{\pgfqpoint{-0.041667in}{-0.041667in}}{\pgfqpoint{0.041667in}{0.041667in}}{%
\pgfpathmoveto{\pgfqpoint{0.000000in}{-0.041667in}}%
\pgfpathcurveto{\pgfqpoint{0.011050in}{-0.041667in}}{\pgfqpoint{0.021649in}{-0.037276in}}{\pgfqpoint{0.029463in}{-0.029463in}}%
\pgfpathcurveto{\pgfqpoint{0.037276in}{-0.021649in}}{\pgfqpoint{0.041667in}{-0.011050in}}{\pgfqpoint{0.041667in}{0.000000in}}%
\pgfpathcurveto{\pgfqpoint{0.041667in}{0.011050in}}{\pgfqpoint{0.037276in}{0.021649in}}{\pgfqpoint{0.029463in}{0.029463in}}%
\pgfpathcurveto{\pgfqpoint{0.021649in}{0.037276in}}{\pgfqpoint{0.011050in}{0.041667in}}{\pgfqpoint{0.000000in}{0.041667in}}%
\pgfpathcurveto{\pgfqpoint{-0.011050in}{0.041667in}}{\pgfqpoint{-0.021649in}{0.037276in}}{\pgfqpoint{-0.029463in}{0.029463in}}%
\pgfpathcurveto{\pgfqpoint{-0.037276in}{0.021649in}}{\pgfqpoint{-0.041667in}{0.011050in}}{\pgfqpoint{-0.041667in}{0.000000in}}%
\pgfpathcurveto{\pgfqpoint{-0.041667in}{-0.011050in}}{\pgfqpoint{-0.037276in}{-0.021649in}}{\pgfqpoint{-0.029463in}{-0.029463in}}%
\pgfpathcurveto{\pgfqpoint{-0.021649in}{-0.037276in}}{\pgfqpoint{-0.011050in}{-0.041667in}}{\pgfqpoint{0.000000in}{-0.041667in}}%
\pgfpathclose%
\pgfusepath{stroke,fill}%
}%
\begin{pgfscope}%
\pgfsys@transformshift{0.833129in}{0.349407in}%
\pgfsys@useobject{currentmarker}{}%
\end{pgfscope}%
\begin{pgfscope}%
\pgfsys@transformshift{1.128123in}{0.356795in}%
\pgfsys@useobject{currentmarker}{}%
\end{pgfscope}%
\begin{pgfscope}%
\pgfsys@transformshift{1.423117in}{0.367787in}%
\pgfsys@useobject{currentmarker}{}%
\end{pgfscope}%
\begin{pgfscope}%
\pgfsys@transformshift{1.718111in}{0.384036in}%
\pgfsys@useobject{currentmarker}{}%
\end{pgfscope}%
\begin{pgfscope}%
\pgfsys@transformshift{2.013105in}{0.406455in}%
\pgfsys@useobject{currentmarker}{}%
\end{pgfscope}%
\begin{pgfscope}%
\pgfsys@transformshift{2.308099in}{0.433028in}%
\pgfsys@useobject{currentmarker}{}%
\end{pgfscope}%
\begin{pgfscope}%
\pgfsys@transformshift{2.603093in}{0.466591in}%
\pgfsys@useobject{currentmarker}{}%
\end{pgfscope}%
\begin{pgfscope}%
\pgfsys@transformshift{2.898086in}{0.507357in}%
\pgfsys@useobject{currentmarker}{}%
\end{pgfscope}%
\begin{pgfscope}%
\pgfsys@transformshift{3.193080in}{0.550208in}%
\pgfsys@useobject{currentmarker}{}%
\end{pgfscope}%
\begin{pgfscope}%
\pgfsys@transformshift{3.488074in}{0.601941in}%
\pgfsys@useobject{currentmarker}{}%
\end{pgfscope}%
\begin{pgfscope}%
\pgfsys@transformshift{3.783068in}{0.657693in}%
\pgfsys@useobject{currentmarker}{}%
\end{pgfscope}%
\begin{pgfscope}%
\pgfsys@transformshift{4.078062in}{0.718653in}%
\pgfsys@useobject{currentmarker}{}%
\end{pgfscope}%
\begin{pgfscope}%
\pgfsys@transformshift{4.373056in}{0.782968in}%
\pgfsys@useobject{currentmarker}{}%
\end{pgfscope}%
\begin{pgfscope}%
\pgfsys@transformshift{4.668050in}{0.857118in}%
\pgfsys@useobject{currentmarker}{}%
\end{pgfscope}%
\begin{pgfscope}%
\pgfsys@transformshift{4.963043in}{0.935343in}%
\pgfsys@useobject{currentmarker}{}%
\end{pgfscope}%
\begin{pgfscope}%
\pgfsys@transformshift{5.258037in}{1.019778in}%
\pgfsys@useobject{currentmarker}{}%
\end{pgfscope}%
\end{pgfscope}%
\begin{pgfscope}%
\pgfsetrectcap%
\pgfsetmiterjoin%
\pgfsetlinewidth{0.803000pt}%
\definecolor{currentstroke}{rgb}{0.000000,0.000000,0.000000}%
\pgfsetstrokecolor{currentstroke}%
\pgfsetdash{}{0pt}%
\pgfpathmoveto{\pgfqpoint{0.459778in}{0.315889in}}%
\pgfpathlineto{\pgfqpoint{0.459778in}{1.053296in}}%
\pgfusepath{stroke}%
\end{pgfscope}%
\begin{pgfscope}%
\pgfsetrectcap%
\pgfsetmiterjoin%
\pgfsetlinewidth{0.803000pt}%
\definecolor{currentstroke}{rgb}{0.000000,0.000000,0.000000}%
\pgfsetstrokecolor{currentstroke}%
\pgfsetdash{}{0pt}%
\pgfpathmoveto{\pgfqpoint{5.631389in}{0.315889in}}%
\pgfpathlineto{\pgfqpoint{5.631389in}{1.053296in}}%
\pgfusepath{stroke}%
\end{pgfscope}%
\begin{pgfscope}%
\pgfsetrectcap%
\pgfsetmiterjoin%
\pgfsetlinewidth{0.803000pt}%
\definecolor{currentstroke}{rgb}{0.000000,0.000000,0.000000}%
\pgfsetstrokecolor{currentstroke}%
\pgfsetdash{}{0pt}%
\pgfpathmoveto{\pgfqpoint{0.459778in}{0.315889in}}%
\pgfpathlineto{\pgfqpoint{5.631389in}{0.315889in}}%
\pgfusepath{stroke}%
\end{pgfscope}%
\begin{pgfscope}%
\pgfsetrectcap%
\pgfsetmiterjoin%
\pgfsetlinewidth{0.803000pt}%
\definecolor{currentstroke}{rgb}{0.000000,0.000000,0.000000}%
\pgfsetstrokecolor{currentstroke}%
\pgfsetdash{}{0pt}%
\pgfpathmoveto{\pgfqpoint{0.459778in}{1.053296in}}%
\pgfpathlineto{\pgfqpoint{5.631389in}{1.053296in}}%
\pgfusepath{stroke}%
\end{pgfscope}%
\end{pgfpicture}%
\makeatother%
\endgroup%

    \caption{Multiskalenanalyse von $x^2 + r$ mit Haar Wavelet\label{polynomials:noise:db1_multi}}
\end{figure}

Im Vergleich dazu sind in \autoref{polynomials:noise:average} die direkte
Ableitung und die Ableitungen nach der Mittelwertbildung über 2, 4 und 8 Werte
zu sehen.

\begin{figure}
    \centering
    %% Creator: Matplotlib, PGF backend
%%
%% To include the figure in your LaTeX document, write
%%   \input{<filename>.pgf}
%%
%% Make sure the required packages are loaded in your preamble
%%   \usepackage{pgf}
%%
%% Figures using additional raster images can only be included by \input if
%% they are in the same directory as the main LaTeX file. For loading figures
%% from other directories you can use the `import` package
%%   \usepackage{import}
%% and then include the figures with
%%   \import{<path to file>}{<filename>.pgf}
%%
%% Matplotlib used the following preamble
%%   \usepackage{fontspec}
%%
\begingroup%
\makeatletter%
\begin{pgfpicture}%
\pgfpathrectangle{\pgfpointorigin}{\pgfqpoint{5.800000in}{6.000000in}}%
\pgfusepath{use as bounding box, clip}%
\begin{pgfscope}%
\pgfsetbuttcap%
\pgfsetmiterjoin%
\definecolor{currentfill}{rgb}{1.000000,1.000000,1.000000}%
\pgfsetfillcolor{currentfill}%
\pgfsetlinewidth{0.000000pt}%
\definecolor{currentstroke}{rgb}{1.000000,1.000000,1.000000}%
\pgfsetstrokecolor{currentstroke}%
\pgfsetdash{}{0pt}%
\pgfpathmoveto{\pgfqpoint{0.000000in}{0.000000in}}%
\pgfpathlineto{\pgfqpoint{5.800000in}{0.000000in}}%
\pgfpathlineto{\pgfqpoint{5.800000in}{6.000000in}}%
\pgfpathlineto{\pgfqpoint{0.000000in}{6.000000in}}%
\pgfpathclose%
\pgfusepath{fill}%
\end{pgfscope}%
\begin{pgfscope}%
\pgfsetbuttcap%
\pgfsetmiterjoin%
\definecolor{currentfill}{rgb}{1.000000,1.000000,1.000000}%
\pgfsetfillcolor{currentfill}%
\pgfsetlinewidth{0.000000pt}%
\definecolor{currentstroke}{rgb}{0.000000,0.000000,0.000000}%
\pgfsetstrokecolor{currentstroke}%
\pgfsetstrokeopacity{0.000000}%
\pgfsetdash{}{0pt}%
\pgfpathmoveto{\pgfqpoint{0.725000in}{4.295217in}}%
\pgfpathlineto{\pgfqpoint{5.220000in}{4.295217in}}%
\pgfpathlineto{\pgfqpoint{5.220000in}{5.280000in}}%
\pgfpathlineto{\pgfqpoint{0.725000in}{5.280000in}}%
\pgfpathclose%
\pgfusepath{fill}%
\end{pgfscope}%
\begin{pgfscope}%
\pgfsetbuttcap%
\pgfsetroundjoin%
\definecolor{currentfill}{rgb}{0.000000,0.000000,0.000000}%
\pgfsetfillcolor{currentfill}%
\pgfsetlinewidth{0.803000pt}%
\definecolor{currentstroke}{rgb}{0.000000,0.000000,0.000000}%
\pgfsetstrokecolor{currentstroke}%
\pgfsetdash{}{0pt}%
\pgfsys@defobject{currentmarker}{\pgfqpoint{0.000000in}{-0.048611in}}{\pgfqpoint{0.000000in}{0.000000in}}{%
\pgfpathmoveto{\pgfqpoint{0.000000in}{0.000000in}}%
\pgfpathlineto{\pgfqpoint{0.000000in}{-0.048611in}}%
\pgfusepath{stroke,fill}%
}%
\begin{pgfscope}%
\pgfsys@transformshift{0.929318in}{4.295217in}%
\pgfsys@useobject{currentmarker}{}%
\end{pgfscope}%
\end{pgfscope}%
\begin{pgfscope}%
\pgfsetbuttcap%
\pgfsetroundjoin%
\definecolor{currentfill}{rgb}{0.000000,0.000000,0.000000}%
\pgfsetfillcolor{currentfill}%
\pgfsetlinewidth{0.803000pt}%
\definecolor{currentstroke}{rgb}{0.000000,0.000000,0.000000}%
\pgfsetstrokecolor{currentstroke}%
\pgfsetdash{}{0pt}%
\pgfsys@defobject{currentmarker}{\pgfqpoint{0.000000in}{-0.048611in}}{\pgfqpoint{0.000000in}{0.000000in}}{%
\pgfpathmoveto{\pgfqpoint{0.000000in}{0.000000in}}%
\pgfpathlineto{\pgfqpoint{0.000000in}{-0.048611in}}%
\pgfusepath{stroke,fill}%
}%
\begin{pgfscope}%
\pgfsys@transformshift{1.733720in}{4.295217in}%
\pgfsys@useobject{currentmarker}{}%
\end{pgfscope}%
\end{pgfscope}%
\begin{pgfscope}%
\pgfsetbuttcap%
\pgfsetroundjoin%
\definecolor{currentfill}{rgb}{0.000000,0.000000,0.000000}%
\pgfsetfillcolor{currentfill}%
\pgfsetlinewidth{0.803000pt}%
\definecolor{currentstroke}{rgb}{0.000000,0.000000,0.000000}%
\pgfsetstrokecolor{currentstroke}%
\pgfsetdash{}{0pt}%
\pgfsys@defobject{currentmarker}{\pgfqpoint{0.000000in}{-0.048611in}}{\pgfqpoint{0.000000in}{0.000000in}}{%
\pgfpathmoveto{\pgfqpoint{0.000000in}{0.000000in}}%
\pgfpathlineto{\pgfqpoint{0.000000in}{-0.048611in}}%
\pgfusepath{stroke,fill}%
}%
\begin{pgfscope}%
\pgfsys@transformshift{2.538123in}{4.295217in}%
\pgfsys@useobject{currentmarker}{}%
\end{pgfscope}%
\end{pgfscope}%
\begin{pgfscope}%
\pgfsetbuttcap%
\pgfsetroundjoin%
\definecolor{currentfill}{rgb}{0.000000,0.000000,0.000000}%
\pgfsetfillcolor{currentfill}%
\pgfsetlinewidth{0.803000pt}%
\definecolor{currentstroke}{rgb}{0.000000,0.000000,0.000000}%
\pgfsetstrokecolor{currentstroke}%
\pgfsetdash{}{0pt}%
\pgfsys@defobject{currentmarker}{\pgfqpoint{0.000000in}{-0.048611in}}{\pgfqpoint{0.000000in}{0.000000in}}{%
\pgfpathmoveto{\pgfqpoint{0.000000in}{0.000000in}}%
\pgfpathlineto{\pgfqpoint{0.000000in}{-0.048611in}}%
\pgfusepath{stroke,fill}%
}%
\begin{pgfscope}%
\pgfsys@transformshift{3.342525in}{4.295217in}%
\pgfsys@useobject{currentmarker}{}%
\end{pgfscope}%
\end{pgfscope}%
\begin{pgfscope}%
\pgfsetbuttcap%
\pgfsetroundjoin%
\definecolor{currentfill}{rgb}{0.000000,0.000000,0.000000}%
\pgfsetfillcolor{currentfill}%
\pgfsetlinewidth{0.803000pt}%
\definecolor{currentstroke}{rgb}{0.000000,0.000000,0.000000}%
\pgfsetstrokecolor{currentstroke}%
\pgfsetdash{}{0pt}%
\pgfsys@defobject{currentmarker}{\pgfqpoint{0.000000in}{-0.048611in}}{\pgfqpoint{0.000000in}{0.000000in}}{%
\pgfpathmoveto{\pgfqpoint{0.000000in}{0.000000in}}%
\pgfpathlineto{\pgfqpoint{0.000000in}{-0.048611in}}%
\pgfusepath{stroke,fill}%
}%
\begin{pgfscope}%
\pgfsys@transformshift{4.146927in}{4.295217in}%
\pgfsys@useobject{currentmarker}{}%
\end{pgfscope}%
\end{pgfscope}%
\begin{pgfscope}%
\pgfsetbuttcap%
\pgfsetroundjoin%
\definecolor{currentfill}{rgb}{0.000000,0.000000,0.000000}%
\pgfsetfillcolor{currentfill}%
\pgfsetlinewidth{0.803000pt}%
\definecolor{currentstroke}{rgb}{0.000000,0.000000,0.000000}%
\pgfsetstrokecolor{currentstroke}%
\pgfsetdash{}{0pt}%
\pgfsys@defobject{currentmarker}{\pgfqpoint{0.000000in}{-0.048611in}}{\pgfqpoint{0.000000in}{0.000000in}}{%
\pgfpathmoveto{\pgfqpoint{0.000000in}{0.000000in}}%
\pgfpathlineto{\pgfqpoint{0.000000in}{-0.048611in}}%
\pgfusepath{stroke,fill}%
}%
\begin{pgfscope}%
\pgfsys@transformshift{4.951330in}{4.295217in}%
\pgfsys@useobject{currentmarker}{}%
\end{pgfscope}%
\end{pgfscope}%
\begin{pgfscope}%
\pgfsetbuttcap%
\pgfsetroundjoin%
\definecolor{currentfill}{rgb}{0.000000,0.000000,0.000000}%
\pgfsetfillcolor{currentfill}%
\pgfsetlinewidth{0.803000pt}%
\definecolor{currentstroke}{rgb}{0.000000,0.000000,0.000000}%
\pgfsetstrokecolor{currentstroke}%
\pgfsetdash{}{0pt}%
\pgfsys@defobject{currentmarker}{\pgfqpoint{-0.048611in}{0.000000in}}{\pgfqpoint{0.000000in}{0.000000in}}{%
\pgfpathmoveto{\pgfqpoint{0.000000in}{0.000000in}}%
\pgfpathlineto{\pgfqpoint{-0.048611in}{0.000000in}}%
\pgfusepath{stroke,fill}%
}%
\begin{pgfscope}%
\pgfsys@transformshift{0.725000in}{4.635076in}%
\pgfsys@useobject{currentmarker}{}%
\end{pgfscope}%
\end{pgfscope}%
\begin{pgfscope}%
\definecolor{textcolor}{rgb}{0.000000,0.000000,0.000000}%
\pgfsetstrokecolor{textcolor}%
\pgfsetfillcolor{textcolor}%
\pgftext[x=0.418000in,y=4.596520in,left,base]{\color{textcolor}\rmfamily\fontsize{8.000000}{9.600000}\selectfont 0.00}%
\end{pgfscope}%
\begin{pgfscope}%
\pgfsetbuttcap%
\pgfsetroundjoin%
\definecolor{currentfill}{rgb}{0.000000,0.000000,0.000000}%
\pgfsetfillcolor{currentfill}%
\pgfsetlinewidth{0.803000pt}%
\definecolor{currentstroke}{rgb}{0.000000,0.000000,0.000000}%
\pgfsetstrokecolor{currentstroke}%
\pgfsetdash{}{0pt}%
\pgfsys@defobject{currentmarker}{\pgfqpoint{-0.048611in}{0.000000in}}{\pgfqpoint{0.000000in}{0.000000in}}{%
\pgfpathmoveto{\pgfqpoint{0.000000in}{0.000000in}}%
\pgfpathlineto{\pgfqpoint{-0.048611in}{0.000000in}}%
\pgfusepath{stroke,fill}%
}%
\begin{pgfscope}%
\pgfsys@transformshift{0.725000in}{5.058477in}%
\pgfsys@useobject{currentmarker}{}%
\end{pgfscope}%
\end{pgfscope}%
\begin{pgfscope}%
\definecolor{textcolor}{rgb}{0.000000,0.000000,0.000000}%
\pgfsetstrokecolor{textcolor}%
\pgfsetfillcolor{textcolor}%
\pgftext[x=0.418000in,y=5.019922in,left,base]{\color{textcolor}\rmfamily\fontsize{8.000000}{9.600000}\selectfont 0.05}%
\end{pgfscope}%
\begin{pgfscope}%
\pgfpathrectangle{\pgfqpoint{0.725000in}{4.295217in}}{\pgfqpoint{4.495000in}{0.984783in}}%
\pgfusepath{clip}%
\pgfsetrectcap%
\pgfsetroundjoin%
\pgfsetlinewidth{1.505625pt}%
\definecolor{currentstroke}{rgb}{1.000000,0.498039,0.054902}%
\pgfsetstrokecolor{currentstroke}%
\pgfsetdash{}{0pt}%
\pgfpathmoveto{\pgfqpoint{0.929318in}{4.561811in}}%
\pgfpathlineto{\pgfqpoint{0.937362in}{4.561811in}}%
\pgfpathlineto{\pgfqpoint{0.937362in}{4.587443in}}%
\pgfpathlineto{\pgfqpoint{0.953450in}{4.587443in}}%
\pgfpathlineto{\pgfqpoint{0.953450in}{4.455279in}}%
\pgfpathlineto{\pgfqpoint{0.969538in}{4.455279in}}%
\pgfpathlineto{\pgfqpoint{0.969538in}{4.626684in}}%
\pgfpathlineto{\pgfqpoint{0.985626in}{4.626684in}}%
\pgfpathlineto{\pgfqpoint{0.985626in}{4.618420in}}%
\pgfpathlineto{\pgfqpoint{1.001714in}{4.618420in}}%
\pgfpathlineto{\pgfqpoint{1.001714in}{4.961197in}}%
\pgfpathlineto{\pgfqpoint{1.017802in}{4.961197in}}%
\pgfpathlineto{\pgfqpoint{1.017802in}{4.375663in}}%
\pgfpathlineto{\pgfqpoint{1.033890in}{4.375663in}}%
\pgfpathlineto{\pgfqpoint{1.033890in}{4.710218in}}%
\pgfpathlineto{\pgfqpoint{1.049979in}{4.710218in}}%
\pgfpathlineto{\pgfqpoint{1.049979in}{4.661233in}}%
\pgfpathlineto{\pgfqpoint{1.066067in}{4.661233in}}%
\pgfpathlineto{\pgfqpoint{1.066067in}{4.765309in}}%
\pgfpathlineto{\pgfqpoint{1.082155in}{4.765309in}}%
\pgfpathlineto{\pgfqpoint{1.082155in}{4.673986in}}%
\pgfpathlineto{\pgfqpoint{1.098243in}{4.673986in}}%
\pgfpathlineto{\pgfqpoint{1.098243in}{4.348490in}}%
\pgfpathlineto{\pgfqpoint{1.114331in}{4.348490in}}%
\pgfpathlineto{\pgfqpoint{1.114331in}{4.769967in}}%
\pgfpathlineto{\pgfqpoint{1.130419in}{4.769967in}}%
\pgfpathlineto{\pgfqpoint{1.130419in}{4.825254in}}%
\pgfpathlineto{\pgfqpoint{1.146507in}{4.825254in}}%
\pgfpathlineto{\pgfqpoint{1.146507in}{4.639673in}}%
\pgfpathlineto{\pgfqpoint{1.162595in}{4.639673in}}%
\pgfpathlineto{\pgfqpoint{1.162595in}{4.348229in}}%
\pgfpathlineto{\pgfqpoint{1.178683in}{4.348229in}}%
\pgfpathlineto{\pgfqpoint{1.178683in}{4.675116in}}%
\pgfpathlineto{\pgfqpoint{1.194771in}{4.675116in}}%
\pgfpathlineto{\pgfqpoint{1.194771in}{4.715893in}}%
\pgfpathlineto{\pgfqpoint{1.210859in}{4.715893in}}%
\pgfpathlineto{\pgfqpoint{1.210859in}{4.557458in}}%
\pgfpathlineto{\pgfqpoint{1.226947in}{4.557458in}}%
\pgfpathlineto{\pgfqpoint{1.226947in}{4.803287in}}%
\pgfpathlineto{\pgfqpoint{1.243035in}{4.803287in}}%
\pgfpathlineto{\pgfqpoint{1.243035in}{4.522195in}}%
\pgfpathlineto{\pgfqpoint{1.259123in}{4.522195in}}%
\pgfpathlineto{\pgfqpoint{1.259123in}{4.852493in}}%
\pgfpathlineto{\pgfqpoint{1.275211in}{4.852493in}}%
\pgfpathlineto{\pgfqpoint{1.275211in}{4.522325in}}%
\pgfpathlineto{\pgfqpoint{1.291299in}{4.522325in}}%
\pgfpathlineto{\pgfqpoint{1.291299in}{4.807696in}}%
\pgfpathlineto{\pgfqpoint{1.307387in}{4.807696in}}%
\pgfpathlineto{\pgfqpoint{1.307387in}{4.457830in}}%
\pgfpathlineto{\pgfqpoint{1.323475in}{4.457830in}}%
\pgfpathlineto{\pgfqpoint{1.323475in}{5.025072in}}%
\pgfpathlineto{\pgfqpoint{1.339563in}{5.025072in}}%
\pgfpathlineto{\pgfqpoint{1.339563in}{4.644341in}}%
\pgfpathlineto{\pgfqpoint{1.355651in}{4.644341in}}%
\pgfpathlineto{\pgfqpoint{1.355651in}{4.434765in}}%
\pgfpathlineto{\pgfqpoint{1.371739in}{4.434765in}}%
\pgfpathlineto{\pgfqpoint{1.371739in}{4.601419in}}%
\pgfpathlineto{\pgfqpoint{1.387827in}{4.601419in}}%
\pgfpathlineto{\pgfqpoint{1.387827in}{4.669290in}}%
\pgfpathlineto{\pgfqpoint{1.403916in}{4.669290in}}%
\pgfpathlineto{\pgfqpoint{1.403916in}{4.765502in}}%
\pgfpathlineto{\pgfqpoint{1.420004in}{4.765502in}}%
\pgfpathlineto{\pgfqpoint{1.420004in}{4.480156in}}%
\pgfpathlineto{\pgfqpoint{1.436092in}{4.480156in}}%
\pgfpathlineto{\pgfqpoint{1.436092in}{4.922854in}}%
\pgfpathlineto{\pgfqpoint{1.452180in}{4.922854in}}%
\pgfpathlineto{\pgfqpoint{1.452180in}{4.572001in}}%
\pgfpathlineto{\pgfqpoint{1.468268in}{4.572001in}}%
\pgfpathlineto{\pgfqpoint{1.468268in}{4.657937in}}%
\pgfpathlineto{\pgfqpoint{1.484356in}{4.657937in}}%
\pgfpathlineto{\pgfqpoint{1.484356in}{4.854952in}}%
\pgfpathlineto{\pgfqpoint{1.500444in}{4.854952in}}%
\pgfpathlineto{\pgfqpoint{1.500444in}{4.630622in}}%
\pgfpathlineto{\pgfqpoint{1.516532in}{4.630622in}}%
\pgfpathlineto{\pgfqpoint{1.516532in}{4.778004in}}%
\pgfpathlineto{\pgfqpoint{1.532620in}{4.778004in}}%
\pgfpathlineto{\pgfqpoint{1.532620in}{4.420130in}}%
\pgfpathlineto{\pgfqpoint{1.548708in}{4.420130in}}%
\pgfpathlineto{\pgfqpoint{1.548708in}{4.907238in}}%
\pgfpathlineto{\pgfqpoint{1.564796in}{4.907238in}}%
\pgfpathlineto{\pgfqpoint{1.564796in}{4.623254in}}%
\pgfpathlineto{\pgfqpoint{1.580884in}{4.623254in}}%
\pgfpathlineto{\pgfqpoint{1.580884in}{4.503640in}}%
\pgfpathlineto{\pgfqpoint{1.596972in}{4.503640in}}%
\pgfpathlineto{\pgfqpoint{1.596972in}{4.791233in}}%
\pgfpathlineto{\pgfqpoint{1.613060in}{4.791233in}}%
\pgfpathlineto{\pgfqpoint{1.613060in}{4.539362in}}%
\pgfpathlineto{\pgfqpoint{1.629148in}{4.539362in}}%
\pgfpathlineto{\pgfqpoint{1.629148in}{4.929478in}}%
\pgfpathlineto{\pgfqpoint{1.645236in}{4.929478in}}%
\pgfpathlineto{\pgfqpoint{1.645236in}{4.359858in}}%
\pgfpathlineto{\pgfqpoint{1.661324in}{4.359858in}}%
\pgfpathlineto{\pgfqpoint{1.661324in}{4.725863in}}%
\pgfpathlineto{\pgfqpoint{1.677412in}{4.725863in}}%
\pgfpathlineto{\pgfqpoint{1.677412in}{4.672114in}}%
\pgfpathlineto{\pgfqpoint{1.693500in}{4.672114in}}%
\pgfpathlineto{\pgfqpoint{1.693500in}{4.778749in}}%
\pgfpathlineto{\pgfqpoint{1.709588in}{4.778749in}}%
\pgfpathlineto{\pgfqpoint{1.709588in}{4.831537in}}%
\pgfpathlineto{\pgfqpoint{1.725676in}{4.831537in}}%
\pgfpathlineto{\pgfqpoint{1.725676in}{4.595248in}}%
\pgfpathlineto{\pgfqpoint{1.741764in}{4.595248in}}%
\pgfpathlineto{\pgfqpoint{1.741764in}{4.672119in}}%
\pgfpathlineto{\pgfqpoint{1.757853in}{4.672119in}}%
\pgfpathlineto{\pgfqpoint{1.757853in}{4.483519in}}%
\pgfpathlineto{\pgfqpoint{1.773941in}{4.483519in}}%
\pgfpathlineto{\pgfqpoint{1.773941in}{4.870698in}}%
\pgfpathlineto{\pgfqpoint{1.790029in}{4.870698in}}%
\pgfpathlineto{\pgfqpoint{1.790029in}{4.538648in}}%
\pgfpathlineto{\pgfqpoint{1.806117in}{4.538648in}}%
\pgfpathlineto{\pgfqpoint{1.806117in}{4.754163in}}%
\pgfpathlineto{\pgfqpoint{1.822205in}{4.754163in}}%
\pgfpathlineto{\pgfqpoint{1.822205in}{5.000541in}}%
\pgfpathlineto{\pgfqpoint{1.838293in}{5.000541in}}%
\pgfpathlineto{\pgfqpoint{1.838293in}{4.391403in}}%
\pgfpathlineto{\pgfqpoint{1.854381in}{4.391403in}}%
\pgfpathlineto{\pgfqpoint{1.854381in}{4.968244in}}%
\pgfpathlineto{\pgfqpoint{1.870469in}{4.968244in}}%
\pgfpathlineto{\pgfqpoint{1.870469in}{4.673589in}}%
\pgfpathlineto{\pgfqpoint{1.886557in}{4.673589in}}%
\pgfpathlineto{\pgfqpoint{1.886557in}{4.425542in}}%
\pgfpathlineto{\pgfqpoint{1.902645in}{4.425542in}}%
\pgfpathlineto{\pgfqpoint{1.902645in}{4.888224in}}%
\pgfpathlineto{\pgfqpoint{1.918733in}{4.888224in}}%
\pgfpathlineto{\pgfqpoint{1.918733in}{4.492701in}}%
\pgfpathlineto{\pgfqpoint{1.934821in}{4.492701in}}%
\pgfpathlineto{\pgfqpoint{1.934821in}{4.960449in}}%
\pgfpathlineto{\pgfqpoint{1.950909in}{4.960449in}}%
\pgfpathlineto{\pgfqpoint{1.950909in}{4.663986in}}%
\pgfpathlineto{\pgfqpoint{1.966997in}{4.663986in}}%
\pgfpathlineto{\pgfqpoint{1.966997in}{4.777188in}}%
\pgfpathlineto{\pgfqpoint{1.983085in}{4.777188in}}%
\pgfpathlineto{\pgfqpoint{1.983085in}{4.563553in}}%
\pgfpathlineto{\pgfqpoint{1.999173in}{4.563553in}}%
\pgfpathlineto{\pgfqpoint{1.999173in}{4.889781in}}%
\pgfpathlineto{\pgfqpoint{2.015261in}{4.889781in}}%
\pgfpathlineto{\pgfqpoint{2.015261in}{4.339980in}}%
\pgfpathlineto{\pgfqpoint{2.031349in}{4.339980in}}%
\pgfpathlineto{\pgfqpoint{2.031349in}{4.810597in}}%
\pgfpathlineto{\pgfqpoint{2.047437in}{4.810597in}}%
\pgfpathlineto{\pgfqpoint{2.047437in}{4.887269in}}%
\pgfpathlineto{\pgfqpoint{2.063525in}{4.887269in}}%
\pgfpathlineto{\pgfqpoint{2.063525in}{4.583839in}}%
\pgfpathlineto{\pgfqpoint{2.079613in}{4.583839in}}%
\pgfpathlineto{\pgfqpoint{2.079613in}{4.606807in}}%
\pgfpathlineto{\pgfqpoint{2.095702in}{4.606807in}}%
\pgfpathlineto{\pgfqpoint{2.095702in}{4.683815in}}%
\pgfpathlineto{\pgfqpoint{2.111790in}{4.683815in}}%
\pgfpathlineto{\pgfqpoint{2.111790in}{5.013361in}}%
\pgfpathlineto{\pgfqpoint{2.127878in}{5.013361in}}%
\pgfpathlineto{\pgfqpoint{2.127878in}{4.568000in}}%
\pgfpathlineto{\pgfqpoint{2.143966in}{4.568000in}}%
\pgfpathlineto{\pgfqpoint{2.143966in}{4.888006in}}%
\pgfpathlineto{\pgfqpoint{2.160054in}{4.888006in}}%
\pgfpathlineto{\pgfqpoint{2.160054in}{4.652732in}}%
\pgfpathlineto{\pgfqpoint{2.176142in}{4.652732in}}%
\pgfpathlineto{\pgfqpoint{2.176142in}{4.579861in}}%
\pgfpathlineto{\pgfqpoint{2.192230in}{4.579861in}}%
\pgfpathlineto{\pgfqpoint{2.192230in}{4.702567in}}%
\pgfpathlineto{\pgfqpoint{2.208318in}{4.702567in}}%
\pgfpathlineto{\pgfqpoint{2.208318in}{4.779119in}}%
\pgfpathlineto{\pgfqpoint{2.224406in}{4.779119in}}%
\pgfpathlineto{\pgfqpoint{2.224406in}{4.878091in}}%
\pgfpathlineto{\pgfqpoint{2.240494in}{4.878091in}}%
\pgfpathlineto{\pgfqpoint{2.240494in}{4.656761in}}%
\pgfpathlineto{\pgfqpoint{2.256582in}{4.656761in}}%
\pgfpathlineto{\pgfqpoint{2.256582in}{4.412084in}}%
\pgfpathlineto{\pgfqpoint{2.272670in}{4.412084in}}%
\pgfpathlineto{\pgfqpoint{2.272670in}{5.014835in}}%
\pgfpathlineto{\pgfqpoint{2.288758in}{5.014835in}}%
\pgfpathlineto{\pgfqpoint{2.288758in}{4.819564in}}%
\pgfpathlineto{\pgfqpoint{2.304846in}{4.819564in}}%
\pgfpathlineto{\pgfqpoint{2.304846in}{4.636680in}}%
\pgfpathlineto{\pgfqpoint{2.320934in}{4.636680in}}%
\pgfpathlineto{\pgfqpoint{2.320934in}{4.500371in}}%
\pgfpathlineto{\pgfqpoint{2.337022in}{4.500371in}}%
\pgfpathlineto{\pgfqpoint{2.337022in}{4.857054in}}%
\pgfpathlineto{\pgfqpoint{2.353110in}{4.857054in}}%
\pgfpathlineto{\pgfqpoint{2.353110in}{4.583006in}}%
\pgfpathlineto{\pgfqpoint{2.369198in}{4.583006in}}%
\pgfpathlineto{\pgfqpoint{2.369198in}{4.821949in}}%
\pgfpathlineto{\pgfqpoint{2.385286in}{4.821949in}}%
\pgfpathlineto{\pgfqpoint{2.385286in}{4.975180in}}%
\pgfpathlineto{\pgfqpoint{2.401374in}{4.975180in}}%
\pgfpathlineto{\pgfqpoint{2.401374in}{4.725596in}}%
\pgfpathlineto{\pgfqpoint{2.417462in}{4.725596in}}%
\pgfpathlineto{\pgfqpoint{2.417462in}{4.461310in}}%
\pgfpathlineto{\pgfqpoint{2.433550in}{4.461310in}}%
\pgfpathlineto{\pgfqpoint{2.433550in}{4.757574in}}%
\pgfpathlineto{\pgfqpoint{2.449639in}{4.757574in}}%
\pgfpathlineto{\pgfqpoint{2.449639in}{4.844900in}}%
\pgfpathlineto{\pgfqpoint{2.465727in}{4.844900in}}%
\pgfpathlineto{\pgfqpoint{2.465727in}{4.889299in}}%
\pgfpathlineto{\pgfqpoint{2.481815in}{4.889299in}}%
\pgfpathlineto{\pgfqpoint{2.481815in}{4.391413in}}%
\pgfpathlineto{\pgfqpoint{2.497903in}{4.391413in}}%
\pgfpathlineto{\pgfqpoint{2.497903in}{4.783894in}}%
\pgfpathlineto{\pgfqpoint{2.513991in}{4.783894in}}%
\pgfpathlineto{\pgfqpoint{2.513991in}{4.999823in}}%
\pgfpathlineto{\pgfqpoint{2.530079in}{4.999823in}}%
\pgfpathlineto{\pgfqpoint{2.530079in}{4.578181in}}%
\pgfpathlineto{\pgfqpoint{2.546167in}{4.578181in}}%
\pgfpathlineto{\pgfqpoint{2.546167in}{4.913174in}}%
\pgfpathlineto{\pgfqpoint{2.562255in}{4.913174in}}%
\pgfpathlineto{\pgfqpoint{2.562255in}{4.439789in}}%
\pgfpathlineto{\pgfqpoint{2.578343in}{4.439789in}}%
\pgfpathlineto{\pgfqpoint{2.578343in}{4.889888in}}%
\pgfpathlineto{\pgfqpoint{2.594431in}{4.889888in}}%
\pgfpathlineto{\pgfqpoint{2.594431in}{4.656556in}}%
\pgfpathlineto{\pgfqpoint{2.610519in}{4.656556in}}%
\pgfpathlineto{\pgfqpoint{2.610519in}{4.740869in}}%
\pgfpathlineto{\pgfqpoint{2.626607in}{4.740869in}}%
\pgfpathlineto{\pgfqpoint{2.626607in}{4.801213in}}%
\pgfpathlineto{\pgfqpoint{2.642695in}{4.801213in}}%
\pgfpathlineto{\pgfqpoint{2.642695in}{4.741228in}}%
\pgfpathlineto{\pgfqpoint{2.658783in}{4.741228in}}%
\pgfpathlineto{\pgfqpoint{2.658783in}{4.716049in}}%
\pgfpathlineto{\pgfqpoint{2.674871in}{4.716049in}}%
\pgfpathlineto{\pgfqpoint{2.674871in}{4.985254in}}%
\pgfpathlineto{\pgfqpoint{2.690959in}{4.985254in}}%
\pgfpathlineto{\pgfqpoint{2.690959in}{4.672618in}}%
\pgfpathlineto{\pgfqpoint{2.707047in}{4.672618in}}%
\pgfpathlineto{\pgfqpoint{2.707047in}{4.749461in}}%
\pgfpathlineto{\pgfqpoint{2.723135in}{4.749461in}}%
\pgfpathlineto{\pgfqpoint{2.723135in}{4.566986in}}%
\pgfpathlineto{\pgfqpoint{2.739223in}{4.566986in}}%
\pgfpathlineto{\pgfqpoint{2.739223in}{4.875175in}}%
\pgfpathlineto{\pgfqpoint{2.755311in}{4.875175in}}%
\pgfpathlineto{\pgfqpoint{2.755311in}{4.839667in}}%
\pgfpathlineto{\pgfqpoint{2.771399in}{4.839667in}}%
\pgfpathlineto{\pgfqpoint{2.771399in}{4.762354in}}%
\pgfpathlineto{\pgfqpoint{2.787487in}{4.762354in}}%
\pgfpathlineto{\pgfqpoint{2.787487in}{4.565468in}}%
\pgfpathlineto{\pgfqpoint{2.803576in}{4.565468in}}%
\pgfpathlineto{\pgfqpoint{2.803576in}{4.938003in}}%
\pgfpathlineto{\pgfqpoint{2.819664in}{4.938003in}}%
\pgfpathlineto{\pgfqpoint{2.819664in}{4.743721in}}%
\pgfpathlineto{\pgfqpoint{2.835752in}{4.743721in}}%
\pgfpathlineto{\pgfqpoint{2.835752in}{4.507908in}}%
\pgfpathlineto{\pgfqpoint{2.851840in}{4.507908in}}%
\pgfpathlineto{\pgfqpoint{2.851840in}{5.067343in}}%
\pgfpathlineto{\pgfqpoint{2.867928in}{5.067343in}}%
\pgfpathlineto{\pgfqpoint{2.867928in}{4.581870in}}%
\pgfpathlineto{\pgfqpoint{2.884016in}{4.581870in}}%
\pgfpathlineto{\pgfqpoint{2.884016in}{4.781789in}}%
\pgfpathlineto{\pgfqpoint{2.900104in}{4.781789in}}%
\pgfpathlineto{\pgfqpoint{2.900104in}{4.747633in}}%
\pgfpathlineto{\pgfqpoint{2.916192in}{4.747633in}}%
\pgfpathlineto{\pgfqpoint{2.916192in}{4.815369in}}%
\pgfpathlineto{\pgfqpoint{2.932280in}{4.815369in}}%
\pgfpathlineto{\pgfqpoint{2.932280in}{4.788401in}}%
\pgfpathlineto{\pgfqpoint{2.948368in}{4.788401in}}%
\pgfpathlineto{\pgfqpoint{2.948368in}{4.714792in}}%
\pgfpathlineto{\pgfqpoint{2.964456in}{4.714792in}}%
\pgfpathlineto{\pgfqpoint{2.964456in}{4.923977in}}%
\pgfpathlineto{\pgfqpoint{2.980544in}{4.923977in}}%
\pgfpathlineto{\pgfqpoint{2.980544in}{4.667734in}}%
\pgfpathlineto{\pgfqpoint{2.996632in}{4.667734in}}%
\pgfpathlineto{\pgfqpoint{2.996632in}{4.674172in}}%
\pgfpathlineto{\pgfqpoint{3.012720in}{4.674172in}}%
\pgfpathlineto{\pgfqpoint{3.012720in}{4.777496in}}%
\pgfpathlineto{\pgfqpoint{3.028808in}{4.777496in}}%
\pgfpathlineto{\pgfqpoint{3.028808in}{4.938957in}}%
\pgfpathlineto{\pgfqpoint{3.044896in}{4.938957in}}%
\pgfpathlineto{\pgfqpoint{3.044896in}{4.577231in}}%
\pgfpathlineto{\pgfqpoint{3.060984in}{4.577231in}}%
\pgfpathlineto{\pgfqpoint{3.060984in}{4.716264in}}%
\pgfpathlineto{\pgfqpoint{3.077072in}{4.716264in}}%
\pgfpathlineto{\pgfqpoint{3.077072in}{4.710813in}}%
\pgfpathlineto{\pgfqpoint{3.093160in}{4.710813in}}%
\pgfpathlineto{\pgfqpoint{3.093160in}{5.122010in}}%
\pgfpathlineto{\pgfqpoint{3.109248in}{5.122010in}}%
\pgfpathlineto{\pgfqpoint{3.109248in}{4.795287in}}%
\pgfpathlineto{\pgfqpoint{3.125336in}{4.795287in}}%
\pgfpathlineto{\pgfqpoint{3.125336in}{4.440271in}}%
\pgfpathlineto{\pgfqpoint{3.141424in}{4.440271in}}%
\pgfpathlineto{\pgfqpoint{3.141424in}{4.870125in}}%
\pgfpathlineto{\pgfqpoint{3.157513in}{4.870125in}}%
\pgfpathlineto{\pgfqpoint{3.157513in}{4.713045in}}%
\pgfpathlineto{\pgfqpoint{3.173601in}{4.713045in}}%
\pgfpathlineto{\pgfqpoint{3.173601in}{4.756270in}}%
\pgfpathlineto{\pgfqpoint{3.189689in}{4.756270in}}%
\pgfpathlineto{\pgfqpoint{3.189689in}{5.114426in}}%
\pgfpathlineto{\pgfqpoint{3.205777in}{5.114426in}}%
\pgfpathlineto{\pgfqpoint{3.205777in}{4.426612in}}%
\pgfpathlineto{\pgfqpoint{3.221865in}{4.426612in}}%
\pgfpathlineto{\pgfqpoint{3.221865in}{5.175634in}}%
\pgfpathlineto{\pgfqpoint{3.237953in}{5.175634in}}%
\pgfpathlineto{\pgfqpoint{3.237953in}{4.411317in}}%
\pgfpathlineto{\pgfqpoint{3.254041in}{4.411317in}}%
\pgfpathlineto{\pgfqpoint{3.254041in}{5.111418in}}%
\pgfpathlineto{\pgfqpoint{3.270129in}{5.111418in}}%
\pgfpathlineto{\pgfqpoint{3.270129in}{4.661967in}}%
\pgfpathlineto{\pgfqpoint{3.286217in}{4.661967in}}%
\pgfpathlineto{\pgfqpoint{3.286217in}{4.704661in}}%
\pgfpathlineto{\pgfqpoint{3.302305in}{4.704661in}}%
\pgfpathlineto{\pgfqpoint{3.302305in}{4.778957in}}%
\pgfpathlineto{\pgfqpoint{3.318393in}{4.778957in}}%
\pgfpathlineto{\pgfqpoint{3.318393in}{4.966548in}}%
\pgfpathlineto{\pgfqpoint{3.334481in}{4.966548in}}%
\pgfpathlineto{\pgfqpoint{3.334481in}{4.604391in}}%
\pgfpathlineto{\pgfqpoint{3.350569in}{4.604391in}}%
\pgfpathlineto{\pgfqpoint{3.350569in}{4.938030in}}%
\pgfpathlineto{\pgfqpoint{3.366657in}{4.938030in}}%
\pgfpathlineto{\pgfqpoint{3.366657in}{4.734680in}}%
\pgfpathlineto{\pgfqpoint{3.382745in}{4.734680in}}%
\pgfpathlineto{\pgfqpoint{3.382745in}{4.781293in}}%
\pgfpathlineto{\pgfqpoint{3.398833in}{4.781293in}}%
\pgfpathlineto{\pgfqpoint{3.398833in}{4.972255in}}%
\pgfpathlineto{\pgfqpoint{3.414921in}{4.972255in}}%
\pgfpathlineto{\pgfqpoint{3.414921in}{4.650363in}}%
\pgfpathlineto{\pgfqpoint{3.431009in}{4.650363in}}%
\pgfpathlineto{\pgfqpoint{3.431009in}{4.913377in}}%
\pgfpathlineto{\pgfqpoint{3.447097in}{4.913377in}}%
\pgfpathlineto{\pgfqpoint{3.447097in}{4.756277in}}%
\pgfpathlineto{\pgfqpoint{3.463185in}{4.756277in}}%
\pgfpathlineto{\pgfqpoint{3.463185in}{4.509320in}}%
\pgfpathlineto{\pgfqpoint{3.479273in}{4.509320in}}%
\pgfpathlineto{\pgfqpoint{3.479273in}{5.112256in}}%
\pgfpathlineto{\pgfqpoint{3.495361in}{5.112256in}}%
\pgfpathlineto{\pgfqpoint{3.495361in}{4.566775in}}%
\pgfpathlineto{\pgfqpoint{3.511450in}{4.566775in}}%
\pgfpathlineto{\pgfqpoint{3.511450in}{5.093901in}}%
\pgfpathlineto{\pgfqpoint{3.527538in}{5.093901in}}%
\pgfpathlineto{\pgfqpoint{3.527538in}{4.440150in}}%
\pgfpathlineto{\pgfqpoint{3.543626in}{4.440150in}}%
\pgfpathlineto{\pgfqpoint{3.543626in}{4.928262in}}%
\pgfpathlineto{\pgfqpoint{3.559714in}{4.928262in}}%
\pgfpathlineto{\pgfqpoint{3.559714in}{4.707659in}}%
\pgfpathlineto{\pgfqpoint{3.575802in}{4.707659in}}%
\pgfpathlineto{\pgfqpoint{3.575802in}{4.966635in}}%
\pgfpathlineto{\pgfqpoint{3.591890in}{4.966635in}}%
\pgfpathlineto{\pgfqpoint{3.591890in}{4.954372in}}%
\pgfpathlineto{\pgfqpoint{3.607978in}{4.954372in}}%
\pgfpathlineto{\pgfqpoint{3.607978in}{4.588634in}}%
\pgfpathlineto{\pgfqpoint{3.624066in}{4.588634in}}%
\pgfpathlineto{\pgfqpoint{3.624066in}{4.748275in}}%
\pgfpathlineto{\pgfqpoint{3.640154in}{4.748275in}}%
\pgfpathlineto{\pgfqpoint{3.640154in}{4.887593in}}%
\pgfpathlineto{\pgfqpoint{3.656242in}{4.887593in}}%
\pgfpathlineto{\pgfqpoint{3.656242in}{5.017468in}}%
\pgfpathlineto{\pgfqpoint{3.672330in}{5.017468in}}%
\pgfpathlineto{\pgfqpoint{3.672330in}{4.528588in}}%
\pgfpathlineto{\pgfqpoint{3.688418in}{4.528588in}}%
\pgfpathlineto{\pgfqpoint{3.688418in}{4.851616in}}%
\pgfpathlineto{\pgfqpoint{3.704506in}{4.851616in}}%
\pgfpathlineto{\pgfqpoint{3.704506in}{4.748718in}}%
\pgfpathlineto{\pgfqpoint{3.720594in}{4.748718in}}%
\pgfpathlineto{\pgfqpoint{3.720594in}{5.086520in}}%
\pgfpathlineto{\pgfqpoint{3.736682in}{5.086520in}}%
\pgfpathlineto{\pgfqpoint{3.736682in}{4.686458in}}%
\pgfpathlineto{\pgfqpoint{3.752770in}{4.686458in}}%
\pgfpathlineto{\pgfqpoint{3.752770in}{4.783224in}}%
\pgfpathlineto{\pgfqpoint{3.768858in}{4.783224in}}%
\pgfpathlineto{\pgfqpoint{3.768858in}{4.992656in}}%
\pgfpathlineto{\pgfqpoint{3.784946in}{4.992656in}}%
\pgfpathlineto{\pgfqpoint{3.784946in}{4.659026in}}%
\pgfpathlineto{\pgfqpoint{3.801034in}{4.659026in}}%
\pgfpathlineto{\pgfqpoint{3.801034in}{4.987968in}}%
\pgfpathlineto{\pgfqpoint{3.817122in}{4.987968in}}%
\pgfpathlineto{\pgfqpoint{3.817122in}{4.860141in}}%
\pgfpathlineto{\pgfqpoint{3.833210in}{4.860141in}}%
\pgfpathlineto{\pgfqpoint{3.833210in}{4.571168in}}%
\pgfpathlineto{\pgfqpoint{3.849298in}{4.571168in}}%
\pgfpathlineto{\pgfqpoint{3.849298in}{5.145365in}}%
\pgfpathlineto{\pgfqpoint{3.865387in}{5.145365in}}%
\pgfpathlineto{\pgfqpoint{3.865387in}{4.482676in}}%
\pgfpathlineto{\pgfqpoint{3.881475in}{4.482676in}}%
\pgfpathlineto{\pgfqpoint{3.881475in}{4.881691in}}%
\pgfpathlineto{\pgfqpoint{3.897563in}{4.881691in}}%
\pgfpathlineto{\pgfqpoint{3.897563in}{4.781316in}}%
\pgfpathlineto{\pgfqpoint{3.913651in}{4.781316in}}%
\pgfpathlineto{\pgfqpoint{3.913651in}{4.899537in}}%
\pgfpathlineto{\pgfqpoint{3.929739in}{4.899537in}}%
\pgfpathlineto{\pgfqpoint{3.929739in}{4.887387in}}%
\pgfpathlineto{\pgfqpoint{3.945827in}{4.887387in}}%
\pgfpathlineto{\pgfqpoint{3.945827in}{4.849290in}}%
\pgfpathlineto{\pgfqpoint{3.961915in}{4.849290in}}%
\pgfpathlineto{\pgfqpoint{3.961915in}{4.972828in}}%
\pgfpathlineto{\pgfqpoint{3.978003in}{4.972828in}}%
\pgfpathlineto{\pgfqpoint{3.978003in}{4.848718in}}%
\pgfpathlineto{\pgfqpoint{3.994091in}{4.848718in}}%
\pgfpathlineto{\pgfqpoint{3.994091in}{4.551183in}}%
\pgfpathlineto{\pgfqpoint{4.010179in}{4.551183in}}%
\pgfpathlineto{\pgfqpoint{4.010179in}{5.006168in}}%
\pgfpathlineto{\pgfqpoint{4.026267in}{5.006168in}}%
\pgfpathlineto{\pgfqpoint{4.026267in}{4.620141in}}%
\pgfpathlineto{\pgfqpoint{4.042355in}{4.620141in}}%
\pgfpathlineto{\pgfqpoint{4.042355in}{4.984830in}}%
\pgfpathlineto{\pgfqpoint{4.058443in}{4.984830in}}%
\pgfpathlineto{\pgfqpoint{4.058443in}{4.780899in}}%
\pgfpathlineto{\pgfqpoint{4.074531in}{4.780899in}}%
\pgfpathlineto{\pgfqpoint{4.074531in}{4.837323in}}%
\pgfpathlineto{\pgfqpoint{4.090619in}{4.837323in}}%
\pgfpathlineto{\pgfqpoint{4.090619in}{4.911155in}}%
\pgfpathlineto{\pgfqpoint{4.106707in}{4.911155in}}%
\pgfpathlineto{\pgfqpoint{4.106707in}{4.946560in}}%
\pgfpathlineto{\pgfqpoint{4.122795in}{4.946560in}}%
\pgfpathlineto{\pgfqpoint{4.122795in}{4.563200in}}%
\pgfpathlineto{\pgfqpoint{4.138883in}{4.563200in}}%
\pgfpathlineto{\pgfqpoint{4.138883in}{5.186912in}}%
\pgfpathlineto{\pgfqpoint{4.154971in}{5.186912in}}%
\pgfpathlineto{\pgfqpoint{4.154971in}{4.783880in}}%
\pgfpathlineto{\pgfqpoint{4.171059in}{4.783880in}}%
\pgfpathlineto{\pgfqpoint{4.171059in}{4.609653in}}%
\pgfpathlineto{\pgfqpoint{4.187147in}{4.609653in}}%
\pgfpathlineto{\pgfqpoint{4.187147in}{5.169998in}}%
\pgfpathlineto{\pgfqpoint{4.203236in}{5.169998in}}%
\pgfpathlineto{\pgfqpoint{4.203236in}{4.664859in}}%
\pgfpathlineto{\pgfqpoint{4.219324in}{4.664859in}}%
\pgfpathlineto{\pgfqpoint{4.219324in}{5.058116in}}%
\pgfpathlineto{\pgfqpoint{4.235412in}{5.058116in}}%
\pgfpathlineto{\pgfqpoint{4.235412in}{4.510411in}}%
\pgfpathlineto{\pgfqpoint{4.251500in}{4.510411in}}%
\pgfpathlineto{\pgfqpoint{4.251500in}{4.994214in}}%
\pgfpathlineto{\pgfqpoint{4.267588in}{4.994214in}}%
\pgfpathlineto{\pgfqpoint{4.267588in}{4.748705in}}%
\pgfpathlineto{\pgfqpoint{4.283676in}{4.748705in}}%
\pgfpathlineto{\pgfqpoint{4.283676in}{4.859233in}}%
\pgfpathlineto{\pgfqpoint{4.299764in}{4.859233in}}%
\pgfpathlineto{\pgfqpoint{4.299764in}{4.975849in}}%
\pgfpathlineto{\pgfqpoint{4.315852in}{4.975849in}}%
\pgfpathlineto{\pgfqpoint{4.315852in}{4.995420in}}%
\pgfpathlineto{\pgfqpoint{4.331940in}{4.995420in}}%
\pgfpathlineto{\pgfqpoint{4.331940in}{4.807155in}}%
\pgfpathlineto{\pgfqpoint{4.348028in}{4.807155in}}%
\pgfpathlineto{\pgfqpoint{4.348028in}{4.823072in}}%
\pgfpathlineto{\pgfqpoint{4.364116in}{4.823072in}}%
\pgfpathlineto{\pgfqpoint{4.364116in}{4.788783in}}%
\pgfpathlineto{\pgfqpoint{4.380204in}{4.788783in}}%
\pgfpathlineto{\pgfqpoint{4.380204in}{5.013335in}}%
\pgfpathlineto{\pgfqpoint{4.396292in}{5.013335in}}%
\pgfpathlineto{\pgfqpoint{4.396292in}{4.678593in}}%
\pgfpathlineto{\pgfqpoint{4.412380in}{4.678593in}}%
\pgfpathlineto{\pgfqpoint{4.412380in}{5.023186in}}%
\pgfpathlineto{\pgfqpoint{4.428468in}{5.023186in}}%
\pgfpathlineto{\pgfqpoint{4.428468in}{4.845840in}}%
\pgfpathlineto{\pgfqpoint{4.444556in}{4.845840in}}%
\pgfpathlineto{\pgfqpoint{4.444556in}{4.601184in}}%
\pgfpathlineto{\pgfqpoint{4.460644in}{4.601184in}}%
\pgfpathlineto{\pgfqpoint{4.460644in}{5.100796in}}%
\pgfpathlineto{\pgfqpoint{4.476732in}{5.100796in}}%
\pgfpathlineto{\pgfqpoint{4.476732in}{4.544454in}}%
\pgfpathlineto{\pgfqpoint{4.492820in}{4.544454in}}%
\pgfpathlineto{\pgfqpoint{4.492820in}{4.998413in}}%
\pgfpathlineto{\pgfqpoint{4.508908in}{4.998413in}}%
\pgfpathlineto{\pgfqpoint{4.508908in}{4.828435in}}%
\pgfpathlineto{\pgfqpoint{4.524996in}{4.828435in}}%
\pgfpathlineto{\pgfqpoint{4.524996in}{4.986371in}}%
\pgfpathlineto{\pgfqpoint{4.541084in}{4.986371in}}%
\pgfpathlineto{\pgfqpoint{4.541084in}{4.967384in}}%
\pgfpathlineto{\pgfqpoint{4.557173in}{4.967384in}}%
\pgfpathlineto{\pgfqpoint{4.557173in}{4.600419in}}%
\pgfpathlineto{\pgfqpoint{4.573261in}{4.600419in}}%
\pgfpathlineto{\pgfqpoint{4.573261in}{5.048752in}}%
\pgfpathlineto{\pgfqpoint{4.589349in}{5.048752in}}%
\pgfpathlineto{\pgfqpoint{4.589349in}{5.052437in}}%
\pgfpathlineto{\pgfqpoint{4.605437in}{5.052437in}}%
\pgfpathlineto{\pgfqpoint{4.605437in}{4.573657in}}%
\pgfpathlineto{\pgfqpoint{4.621525in}{4.573657in}}%
\pgfpathlineto{\pgfqpoint{4.621525in}{5.065108in}}%
\pgfpathlineto{\pgfqpoint{4.637613in}{5.065108in}}%
\pgfpathlineto{\pgfqpoint{4.637613in}{4.840580in}}%
\pgfpathlineto{\pgfqpoint{4.653701in}{4.840580in}}%
\pgfpathlineto{\pgfqpoint{4.653701in}{4.630533in}}%
\pgfpathlineto{\pgfqpoint{4.669789in}{4.630533in}}%
\pgfpathlineto{\pgfqpoint{4.669789in}{5.075545in}}%
\pgfpathlineto{\pgfqpoint{4.685877in}{5.075545in}}%
\pgfpathlineto{\pgfqpoint{4.685877in}{4.978011in}}%
\pgfpathlineto{\pgfqpoint{4.701965in}{4.978011in}}%
\pgfpathlineto{\pgfqpoint{4.701965in}{4.598949in}}%
\pgfpathlineto{\pgfqpoint{4.718053in}{4.598949in}}%
\pgfpathlineto{\pgfqpoint{4.718053in}{5.235237in}}%
\pgfpathlineto{\pgfqpoint{4.734141in}{5.235237in}}%
\pgfpathlineto{\pgfqpoint{4.734141in}{4.730320in}}%
\pgfpathlineto{\pgfqpoint{4.750229in}{4.730320in}}%
\pgfpathlineto{\pgfqpoint{4.750229in}{4.676328in}}%
\pgfpathlineto{\pgfqpoint{4.766317in}{4.676328in}}%
\pgfpathlineto{\pgfqpoint{4.766317in}{4.966626in}}%
\pgfpathlineto{\pgfqpoint{4.782405in}{4.966626in}}%
\pgfpathlineto{\pgfqpoint{4.782405in}{5.194404in}}%
\pgfpathlineto{\pgfqpoint{4.798493in}{5.194404in}}%
\pgfpathlineto{\pgfqpoint{4.798493in}{4.729442in}}%
\pgfpathlineto{\pgfqpoint{4.814581in}{4.729442in}}%
\pgfpathlineto{\pgfqpoint{4.814581in}{4.686666in}}%
\pgfpathlineto{\pgfqpoint{4.830669in}{4.686666in}}%
\pgfpathlineto{\pgfqpoint{4.830669in}{4.965001in}}%
\pgfpathlineto{\pgfqpoint{4.846757in}{4.965001in}}%
\pgfpathlineto{\pgfqpoint{4.846757in}{4.961422in}}%
\pgfpathlineto{\pgfqpoint{4.862845in}{4.961422in}}%
\pgfpathlineto{\pgfqpoint{4.862845in}{4.706343in}}%
\pgfpathlineto{\pgfqpoint{4.878933in}{4.706343in}}%
\pgfpathlineto{\pgfqpoint{4.878933in}{5.057292in}}%
\pgfpathlineto{\pgfqpoint{4.895021in}{5.057292in}}%
\pgfpathlineto{\pgfqpoint{4.895021in}{4.925885in}}%
\pgfpathlineto{\pgfqpoint{4.911110in}{4.925885in}}%
\pgfpathlineto{\pgfqpoint{4.911110in}{4.789904in}}%
\pgfpathlineto{\pgfqpoint{4.927198in}{4.789904in}}%
\pgfpathlineto{\pgfqpoint{4.927198in}{5.041066in}}%
\pgfpathlineto{\pgfqpoint{4.943286in}{5.041066in}}%
\pgfpathlineto{\pgfqpoint{4.943286in}{4.965744in}}%
\pgfpathlineto{\pgfqpoint{4.959374in}{4.965744in}}%
\pgfpathlineto{\pgfqpoint{4.959374in}{4.678023in}}%
\pgfpathlineto{\pgfqpoint{4.975462in}{4.678023in}}%
\pgfpathlineto{\pgfqpoint{4.975462in}{5.061003in}}%
\pgfpathlineto{\pgfqpoint{4.991550in}{5.061003in}}%
\pgfpathlineto{\pgfqpoint{4.991550in}{4.907273in}}%
\pgfpathlineto{\pgfqpoint{5.007638in}{4.907273in}}%
\pgfpathlineto{\pgfqpoint{5.007638in}{5.031015in}}%
\pgfpathlineto{\pgfqpoint{5.015682in}{5.031015in}}%
\pgfpathlineto{\pgfqpoint{5.015682in}{5.031015in}}%
\pgfusepath{stroke}%
\end{pgfscope}%
\begin{pgfscope}%
\pgfpathrectangle{\pgfqpoint{0.725000in}{4.295217in}}{\pgfqpoint{4.495000in}{0.984783in}}%
\pgfusepath{clip}%
\pgfsetbuttcap%
\pgfsetroundjoin%
\definecolor{currentfill}{rgb}{1.000000,0.498039,0.054902}%
\pgfsetfillcolor{currentfill}%
\pgfsetlinewidth{1.003750pt}%
\definecolor{currentstroke}{rgb}{1.000000,0.498039,0.054902}%
\pgfsetstrokecolor{currentstroke}%
\pgfsetdash{}{0pt}%
\pgfsys@defobject{currentmarker}{\pgfqpoint{-0.041667in}{-0.041667in}}{\pgfqpoint{0.041667in}{0.041667in}}{%
\pgfpathmoveto{\pgfqpoint{0.000000in}{-0.041667in}}%
\pgfpathcurveto{\pgfqpoint{0.011050in}{-0.041667in}}{\pgfqpoint{0.021649in}{-0.037276in}}{\pgfqpoint{0.029463in}{-0.029463in}}%
\pgfpathcurveto{\pgfqpoint{0.037276in}{-0.021649in}}{\pgfqpoint{0.041667in}{-0.011050in}}{\pgfqpoint{0.041667in}{0.000000in}}%
\pgfpathcurveto{\pgfqpoint{0.041667in}{0.011050in}}{\pgfqpoint{0.037276in}{0.021649in}}{\pgfqpoint{0.029463in}{0.029463in}}%
\pgfpathcurveto{\pgfqpoint{0.021649in}{0.037276in}}{\pgfqpoint{0.011050in}{0.041667in}}{\pgfqpoint{0.000000in}{0.041667in}}%
\pgfpathcurveto{\pgfqpoint{-0.011050in}{0.041667in}}{\pgfqpoint{-0.021649in}{0.037276in}}{\pgfqpoint{-0.029463in}{0.029463in}}%
\pgfpathcurveto{\pgfqpoint{-0.037276in}{0.021649in}}{\pgfqpoint{-0.041667in}{0.011050in}}{\pgfqpoint{-0.041667in}{0.000000in}}%
\pgfpathcurveto{\pgfqpoint{-0.041667in}{-0.011050in}}{\pgfqpoint{-0.037276in}{-0.021649in}}{\pgfqpoint{-0.029463in}{-0.029463in}}%
\pgfpathcurveto{\pgfqpoint{-0.021649in}{-0.037276in}}{\pgfqpoint{-0.011050in}{-0.041667in}}{\pgfqpoint{0.000000in}{-0.041667in}}%
\pgfpathclose%
\pgfusepath{stroke,fill}%
}%
\begin{pgfscope}%
\pgfsys@transformshift{0.929318in}{4.561811in}%
\pgfsys@useobject{currentmarker}{}%
\end{pgfscope}%
\begin{pgfscope}%
\pgfsys@transformshift{0.945406in}{4.587443in}%
\pgfsys@useobject{currentmarker}{}%
\end{pgfscope}%
\begin{pgfscope}%
\pgfsys@transformshift{0.961494in}{4.455279in}%
\pgfsys@useobject{currentmarker}{}%
\end{pgfscope}%
\begin{pgfscope}%
\pgfsys@transformshift{0.977582in}{4.626684in}%
\pgfsys@useobject{currentmarker}{}%
\end{pgfscope}%
\begin{pgfscope}%
\pgfsys@transformshift{0.993670in}{4.618420in}%
\pgfsys@useobject{currentmarker}{}%
\end{pgfscope}%
\begin{pgfscope}%
\pgfsys@transformshift{1.009758in}{4.961197in}%
\pgfsys@useobject{currentmarker}{}%
\end{pgfscope}%
\begin{pgfscope}%
\pgfsys@transformshift{1.025846in}{4.375663in}%
\pgfsys@useobject{currentmarker}{}%
\end{pgfscope}%
\begin{pgfscope}%
\pgfsys@transformshift{1.041935in}{4.710218in}%
\pgfsys@useobject{currentmarker}{}%
\end{pgfscope}%
\begin{pgfscope}%
\pgfsys@transformshift{1.058023in}{4.661233in}%
\pgfsys@useobject{currentmarker}{}%
\end{pgfscope}%
\begin{pgfscope}%
\pgfsys@transformshift{1.074111in}{4.765309in}%
\pgfsys@useobject{currentmarker}{}%
\end{pgfscope}%
\begin{pgfscope}%
\pgfsys@transformshift{1.090199in}{4.673986in}%
\pgfsys@useobject{currentmarker}{}%
\end{pgfscope}%
\begin{pgfscope}%
\pgfsys@transformshift{1.106287in}{4.348490in}%
\pgfsys@useobject{currentmarker}{}%
\end{pgfscope}%
\begin{pgfscope}%
\pgfsys@transformshift{1.122375in}{4.769967in}%
\pgfsys@useobject{currentmarker}{}%
\end{pgfscope}%
\begin{pgfscope}%
\pgfsys@transformshift{1.138463in}{4.825254in}%
\pgfsys@useobject{currentmarker}{}%
\end{pgfscope}%
\begin{pgfscope}%
\pgfsys@transformshift{1.154551in}{4.639673in}%
\pgfsys@useobject{currentmarker}{}%
\end{pgfscope}%
\begin{pgfscope}%
\pgfsys@transformshift{1.170639in}{4.348229in}%
\pgfsys@useobject{currentmarker}{}%
\end{pgfscope}%
\begin{pgfscope}%
\pgfsys@transformshift{1.186727in}{4.675116in}%
\pgfsys@useobject{currentmarker}{}%
\end{pgfscope}%
\begin{pgfscope}%
\pgfsys@transformshift{1.202815in}{4.715893in}%
\pgfsys@useobject{currentmarker}{}%
\end{pgfscope}%
\begin{pgfscope}%
\pgfsys@transformshift{1.218903in}{4.557458in}%
\pgfsys@useobject{currentmarker}{}%
\end{pgfscope}%
\begin{pgfscope}%
\pgfsys@transformshift{1.234991in}{4.803287in}%
\pgfsys@useobject{currentmarker}{}%
\end{pgfscope}%
\begin{pgfscope}%
\pgfsys@transformshift{1.251079in}{4.522195in}%
\pgfsys@useobject{currentmarker}{}%
\end{pgfscope}%
\begin{pgfscope}%
\pgfsys@transformshift{1.267167in}{4.852493in}%
\pgfsys@useobject{currentmarker}{}%
\end{pgfscope}%
\begin{pgfscope}%
\pgfsys@transformshift{1.283255in}{4.522325in}%
\pgfsys@useobject{currentmarker}{}%
\end{pgfscope}%
\begin{pgfscope}%
\pgfsys@transformshift{1.299343in}{4.807696in}%
\pgfsys@useobject{currentmarker}{}%
\end{pgfscope}%
\begin{pgfscope}%
\pgfsys@transformshift{1.315431in}{4.457830in}%
\pgfsys@useobject{currentmarker}{}%
\end{pgfscope}%
\begin{pgfscope}%
\pgfsys@transformshift{1.331519in}{5.025072in}%
\pgfsys@useobject{currentmarker}{}%
\end{pgfscope}%
\begin{pgfscope}%
\pgfsys@transformshift{1.347607in}{4.644341in}%
\pgfsys@useobject{currentmarker}{}%
\end{pgfscope}%
\begin{pgfscope}%
\pgfsys@transformshift{1.363695in}{4.434765in}%
\pgfsys@useobject{currentmarker}{}%
\end{pgfscope}%
\begin{pgfscope}%
\pgfsys@transformshift{1.379783in}{4.601419in}%
\pgfsys@useobject{currentmarker}{}%
\end{pgfscope}%
\begin{pgfscope}%
\pgfsys@transformshift{1.395872in}{4.669290in}%
\pgfsys@useobject{currentmarker}{}%
\end{pgfscope}%
\begin{pgfscope}%
\pgfsys@transformshift{1.411960in}{4.765502in}%
\pgfsys@useobject{currentmarker}{}%
\end{pgfscope}%
\begin{pgfscope}%
\pgfsys@transformshift{1.428048in}{4.480156in}%
\pgfsys@useobject{currentmarker}{}%
\end{pgfscope}%
\begin{pgfscope}%
\pgfsys@transformshift{1.444136in}{4.922854in}%
\pgfsys@useobject{currentmarker}{}%
\end{pgfscope}%
\begin{pgfscope}%
\pgfsys@transformshift{1.460224in}{4.572001in}%
\pgfsys@useobject{currentmarker}{}%
\end{pgfscope}%
\begin{pgfscope}%
\pgfsys@transformshift{1.476312in}{4.657937in}%
\pgfsys@useobject{currentmarker}{}%
\end{pgfscope}%
\begin{pgfscope}%
\pgfsys@transformshift{1.492400in}{4.854952in}%
\pgfsys@useobject{currentmarker}{}%
\end{pgfscope}%
\begin{pgfscope}%
\pgfsys@transformshift{1.508488in}{4.630622in}%
\pgfsys@useobject{currentmarker}{}%
\end{pgfscope}%
\begin{pgfscope}%
\pgfsys@transformshift{1.524576in}{4.778004in}%
\pgfsys@useobject{currentmarker}{}%
\end{pgfscope}%
\begin{pgfscope}%
\pgfsys@transformshift{1.540664in}{4.420130in}%
\pgfsys@useobject{currentmarker}{}%
\end{pgfscope}%
\begin{pgfscope}%
\pgfsys@transformshift{1.556752in}{4.907238in}%
\pgfsys@useobject{currentmarker}{}%
\end{pgfscope}%
\begin{pgfscope}%
\pgfsys@transformshift{1.572840in}{4.623254in}%
\pgfsys@useobject{currentmarker}{}%
\end{pgfscope}%
\begin{pgfscope}%
\pgfsys@transformshift{1.588928in}{4.503640in}%
\pgfsys@useobject{currentmarker}{}%
\end{pgfscope}%
\begin{pgfscope}%
\pgfsys@transformshift{1.605016in}{4.791233in}%
\pgfsys@useobject{currentmarker}{}%
\end{pgfscope}%
\begin{pgfscope}%
\pgfsys@transformshift{1.621104in}{4.539362in}%
\pgfsys@useobject{currentmarker}{}%
\end{pgfscope}%
\begin{pgfscope}%
\pgfsys@transformshift{1.637192in}{4.929478in}%
\pgfsys@useobject{currentmarker}{}%
\end{pgfscope}%
\begin{pgfscope}%
\pgfsys@transformshift{1.653280in}{4.359858in}%
\pgfsys@useobject{currentmarker}{}%
\end{pgfscope}%
\begin{pgfscope}%
\pgfsys@transformshift{1.669368in}{4.725863in}%
\pgfsys@useobject{currentmarker}{}%
\end{pgfscope}%
\begin{pgfscope}%
\pgfsys@transformshift{1.685456in}{4.672114in}%
\pgfsys@useobject{currentmarker}{}%
\end{pgfscope}%
\begin{pgfscope}%
\pgfsys@transformshift{1.701544in}{4.778749in}%
\pgfsys@useobject{currentmarker}{}%
\end{pgfscope}%
\begin{pgfscope}%
\pgfsys@transformshift{1.717632in}{4.831537in}%
\pgfsys@useobject{currentmarker}{}%
\end{pgfscope}%
\begin{pgfscope}%
\pgfsys@transformshift{1.733720in}{4.595248in}%
\pgfsys@useobject{currentmarker}{}%
\end{pgfscope}%
\begin{pgfscope}%
\pgfsys@transformshift{1.749809in}{4.672119in}%
\pgfsys@useobject{currentmarker}{}%
\end{pgfscope}%
\begin{pgfscope}%
\pgfsys@transformshift{1.765897in}{4.483519in}%
\pgfsys@useobject{currentmarker}{}%
\end{pgfscope}%
\begin{pgfscope}%
\pgfsys@transformshift{1.781985in}{4.870698in}%
\pgfsys@useobject{currentmarker}{}%
\end{pgfscope}%
\begin{pgfscope}%
\pgfsys@transformshift{1.798073in}{4.538648in}%
\pgfsys@useobject{currentmarker}{}%
\end{pgfscope}%
\begin{pgfscope}%
\pgfsys@transformshift{1.814161in}{4.754163in}%
\pgfsys@useobject{currentmarker}{}%
\end{pgfscope}%
\begin{pgfscope}%
\pgfsys@transformshift{1.830249in}{5.000541in}%
\pgfsys@useobject{currentmarker}{}%
\end{pgfscope}%
\begin{pgfscope}%
\pgfsys@transformshift{1.846337in}{4.391403in}%
\pgfsys@useobject{currentmarker}{}%
\end{pgfscope}%
\begin{pgfscope}%
\pgfsys@transformshift{1.862425in}{4.968244in}%
\pgfsys@useobject{currentmarker}{}%
\end{pgfscope}%
\begin{pgfscope}%
\pgfsys@transformshift{1.878513in}{4.673589in}%
\pgfsys@useobject{currentmarker}{}%
\end{pgfscope}%
\begin{pgfscope}%
\pgfsys@transformshift{1.894601in}{4.425542in}%
\pgfsys@useobject{currentmarker}{}%
\end{pgfscope}%
\begin{pgfscope}%
\pgfsys@transformshift{1.910689in}{4.888224in}%
\pgfsys@useobject{currentmarker}{}%
\end{pgfscope}%
\begin{pgfscope}%
\pgfsys@transformshift{1.926777in}{4.492701in}%
\pgfsys@useobject{currentmarker}{}%
\end{pgfscope}%
\begin{pgfscope}%
\pgfsys@transformshift{1.942865in}{4.960449in}%
\pgfsys@useobject{currentmarker}{}%
\end{pgfscope}%
\begin{pgfscope}%
\pgfsys@transformshift{1.958953in}{4.663986in}%
\pgfsys@useobject{currentmarker}{}%
\end{pgfscope}%
\begin{pgfscope}%
\pgfsys@transformshift{1.975041in}{4.777188in}%
\pgfsys@useobject{currentmarker}{}%
\end{pgfscope}%
\begin{pgfscope}%
\pgfsys@transformshift{1.991129in}{4.563553in}%
\pgfsys@useobject{currentmarker}{}%
\end{pgfscope}%
\begin{pgfscope}%
\pgfsys@transformshift{2.007217in}{4.889781in}%
\pgfsys@useobject{currentmarker}{}%
\end{pgfscope}%
\begin{pgfscope}%
\pgfsys@transformshift{2.023305in}{4.339980in}%
\pgfsys@useobject{currentmarker}{}%
\end{pgfscope}%
\begin{pgfscope}%
\pgfsys@transformshift{2.039393in}{4.810597in}%
\pgfsys@useobject{currentmarker}{}%
\end{pgfscope}%
\begin{pgfscope}%
\pgfsys@transformshift{2.055481in}{4.887269in}%
\pgfsys@useobject{currentmarker}{}%
\end{pgfscope}%
\begin{pgfscope}%
\pgfsys@transformshift{2.071569in}{4.583839in}%
\pgfsys@useobject{currentmarker}{}%
\end{pgfscope}%
\begin{pgfscope}%
\pgfsys@transformshift{2.087657in}{4.606807in}%
\pgfsys@useobject{currentmarker}{}%
\end{pgfscope}%
\begin{pgfscope}%
\pgfsys@transformshift{2.103746in}{4.683815in}%
\pgfsys@useobject{currentmarker}{}%
\end{pgfscope}%
\begin{pgfscope}%
\pgfsys@transformshift{2.119834in}{5.013361in}%
\pgfsys@useobject{currentmarker}{}%
\end{pgfscope}%
\begin{pgfscope}%
\pgfsys@transformshift{2.135922in}{4.568000in}%
\pgfsys@useobject{currentmarker}{}%
\end{pgfscope}%
\begin{pgfscope}%
\pgfsys@transformshift{2.152010in}{4.888006in}%
\pgfsys@useobject{currentmarker}{}%
\end{pgfscope}%
\begin{pgfscope}%
\pgfsys@transformshift{2.168098in}{4.652732in}%
\pgfsys@useobject{currentmarker}{}%
\end{pgfscope}%
\begin{pgfscope}%
\pgfsys@transformshift{2.184186in}{4.579861in}%
\pgfsys@useobject{currentmarker}{}%
\end{pgfscope}%
\begin{pgfscope}%
\pgfsys@transformshift{2.200274in}{4.702567in}%
\pgfsys@useobject{currentmarker}{}%
\end{pgfscope}%
\begin{pgfscope}%
\pgfsys@transformshift{2.216362in}{4.779119in}%
\pgfsys@useobject{currentmarker}{}%
\end{pgfscope}%
\begin{pgfscope}%
\pgfsys@transformshift{2.232450in}{4.878091in}%
\pgfsys@useobject{currentmarker}{}%
\end{pgfscope}%
\begin{pgfscope}%
\pgfsys@transformshift{2.248538in}{4.656761in}%
\pgfsys@useobject{currentmarker}{}%
\end{pgfscope}%
\begin{pgfscope}%
\pgfsys@transformshift{2.264626in}{4.412084in}%
\pgfsys@useobject{currentmarker}{}%
\end{pgfscope}%
\begin{pgfscope}%
\pgfsys@transformshift{2.280714in}{5.014835in}%
\pgfsys@useobject{currentmarker}{}%
\end{pgfscope}%
\begin{pgfscope}%
\pgfsys@transformshift{2.296802in}{4.819564in}%
\pgfsys@useobject{currentmarker}{}%
\end{pgfscope}%
\begin{pgfscope}%
\pgfsys@transformshift{2.312890in}{4.636680in}%
\pgfsys@useobject{currentmarker}{}%
\end{pgfscope}%
\begin{pgfscope}%
\pgfsys@transformshift{2.328978in}{4.500371in}%
\pgfsys@useobject{currentmarker}{}%
\end{pgfscope}%
\begin{pgfscope}%
\pgfsys@transformshift{2.345066in}{4.857054in}%
\pgfsys@useobject{currentmarker}{}%
\end{pgfscope}%
\begin{pgfscope}%
\pgfsys@transformshift{2.361154in}{4.583006in}%
\pgfsys@useobject{currentmarker}{}%
\end{pgfscope}%
\begin{pgfscope}%
\pgfsys@transformshift{2.377242in}{4.821949in}%
\pgfsys@useobject{currentmarker}{}%
\end{pgfscope}%
\begin{pgfscope}%
\pgfsys@transformshift{2.393330in}{4.975180in}%
\pgfsys@useobject{currentmarker}{}%
\end{pgfscope}%
\begin{pgfscope}%
\pgfsys@transformshift{2.409418in}{4.725596in}%
\pgfsys@useobject{currentmarker}{}%
\end{pgfscope}%
\begin{pgfscope}%
\pgfsys@transformshift{2.425506in}{4.461310in}%
\pgfsys@useobject{currentmarker}{}%
\end{pgfscope}%
\begin{pgfscope}%
\pgfsys@transformshift{2.441594in}{4.757574in}%
\pgfsys@useobject{currentmarker}{}%
\end{pgfscope}%
\begin{pgfscope}%
\pgfsys@transformshift{2.457683in}{4.844900in}%
\pgfsys@useobject{currentmarker}{}%
\end{pgfscope}%
\begin{pgfscope}%
\pgfsys@transformshift{2.473771in}{4.889299in}%
\pgfsys@useobject{currentmarker}{}%
\end{pgfscope}%
\begin{pgfscope}%
\pgfsys@transformshift{2.489859in}{4.391413in}%
\pgfsys@useobject{currentmarker}{}%
\end{pgfscope}%
\begin{pgfscope}%
\pgfsys@transformshift{2.505947in}{4.783894in}%
\pgfsys@useobject{currentmarker}{}%
\end{pgfscope}%
\begin{pgfscope}%
\pgfsys@transformshift{2.522035in}{4.999823in}%
\pgfsys@useobject{currentmarker}{}%
\end{pgfscope}%
\begin{pgfscope}%
\pgfsys@transformshift{2.538123in}{4.578181in}%
\pgfsys@useobject{currentmarker}{}%
\end{pgfscope}%
\begin{pgfscope}%
\pgfsys@transformshift{2.554211in}{4.913174in}%
\pgfsys@useobject{currentmarker}{}%
\end{pgfscope}%
\begin{pgfscope}%
\pgfsys@transformshift{2.570299in}{4.439789in}%
\pgfsys@useobject{currentmarker}{}%
\end{pgfscope}%
\begin{pgfscope}%
\pgfsys@transformshift{2.586387in}{4.889888in}%
\pgfsys@useobject{currentmarker}{}%
\end{pgfscope}%
\begin{pgfscope}%
\pgfsys@transformshift{2.602475in}{4.656556in}%
\pgfsys@useobject{currentmarker}{}%
\end{pgfscope}%
\begin{pgfscope}%
\pgfsys@transformshift{2.618563in}{4.740869in}%
\pgfsys@useobject{currentmarker}{}%
\end{pgfscope}%
\begin{pgfscope}%
\pgfsys@transformshift{2.634651in}{4.801213in}%
\pgfsys@useobject{currentmarker}{}%
\end{pgfscope}%
\begin{pgfscope}%
\pgfsys@transformshift{2.650739in}{4.741228in}%
\pgfsys@useobject{currentmarker}{}%
\end{pgfscope}%
\begin{pgfscope}%
\pgfsys@transformshift{2.666827in}{4.716049in}%
\pgfsys@useobject{currentmarker}{}%
\end{pgfscope}%
\begin{pgfscope}%
\pgfsys@transformshift{2.682915in}{4.985254in}%
\pgfsys@useobject{currentmarker}{}%
\end{pgfscope}%
\begin{pgfscope}%
\pgfsys@transformshift{2.699003in}{4.672618in}%
\pgfsys@useobject{currentmarker}{}%
\end{pgfscope}%
\begin{pgfscope}%
\pgfsys@transformshift{2.715091in}{4.749461in}%
\pgfsys@useobject{currentmarker}{}%
\end{pgfscope}%
\begin{pgfscope}%
\pgfsys@transformshift{2.731179in}{4.566986in}%
\pgfsys@useobject{currentmarker}{}%
\end{pgfscope}%
\begin{pgfscope}%
\pgfsys@transformshift{2.747267in}{4.875175in}%
\pgfsys@useobject{currentmarker}{}%
\end{pgfscope}%
\begin{pgfscope}%
\pgfsys@transformshift{2.763355in}{4.839667in}%
\pgfsys@useobject{currentmarker}{}%
\end{pgfscope}%
\begin{pgfscope}%
\pgfsys@transformshift{2.779443in}{4.762354in}%
\pgfsys@useobject{currentmarker}{}%
\end{pgfscope}%
\begin{pgfscope}%
\pgfsys@transformshift{2.795531in}{4.565468in}%
\pgfsys@useobject{currentmarker}{}%
\end{pgfscope}%
\begin{pgfscope}%
\pgfsys@transformshift{2.811620in}{4.938003in}%
\pgfsys@useobject{currentmarker}{}%
\end{pgfscope}%
\begin{pgfscope}%
\pgfsys@transformshift{2.827708in}{4.743721in}%
\pgfsys@useobject{currentmarker}{}%
\end{pgfscope}%
\begin{pgfscope}%
\pgfsys@transformshift{2.843796in}{4.507908in}%
\pgfsys@useobject{currentmarker}{}%
\end{pgfscope}%
\begin{pgfscope}%
\pgfsys@transformshift{2.859884in}{5.067343in}%
\pgfsys@useobject{currentmarker}{}%
\end{pgfscope}%
\begin{pgfscope}%
\pgfsys@transformshift{2.875972in}{4.581870in}%
\pgfsys@useobject{currentmarker}{}%
\end{pgfscope}%
\begin{pgfscope}%
\pgfsys@transformshift{2.892060in}{4.781789in}%
\pgfsys@useobject{currentmarker}{}%
\end{pgfscope}%
\begin{pgfscope}%
\pgfsys@transformshift{2.908148in}{4.747633in}%
\pgfsys@useobject{currentmarker}{}%
\end{pgfscope}%
\begin{pgfscope}%
\pgfsys@transformshift{2.924236in}{4.815369in}%
\pgfsys@useobject{currentmarker}{}%
\end{pgfscope}%
\begin{pgfscope}%
\pgfsys@transformshift{2.940324in}{4.788401in}%
\pgfsys@useobject{currentmarker}{}%
\end{pgfscope}%
\begin{pgfscope}%
\pgfsys@transformshift{2.956412in}{4.714792in}%
\pgfsys@useobject{currentmarker}{}%
\end{pgfscope}%
\begin{pgfscope}%
\pgfsys@transformshift{2.972500in}{4.923977in}%
\pgfsys@useobject{currentmarker}{}%
\end{pgfscope}%
\begin{pgfscope}%
\pgfsys@transformshift{2.988588in}{4.667734in}%
\pgfsys@useobject{currentmarker}{}%
\end{pgfscope}%
\begin{pgfscope}%
\pgfsys@transformshift{3.004676in}{4.674172in}%
\pgfsys@useobject{currentmarker}{}%
\end{pgfscope}%
\begin{pgfscope}%
\pgfsys@transformshift{3.020764in}{4.777496in}%
\pgfsys@useobject{currentmarker}{}%
\end{pgfscope}%
\begin{pgfscope}%
\pgfsys@transformshift{3.036852in}{4.938957in}%
\pgfsys@useobject{currentmarker}{}%
\end{pgfscope}%
\begin{pgfscope}%
\pgfsys@transformshift{3.052940in}{4.577231in}%
\pgfsys@useobject{currentmarker}{}%
\end{pgfscope}%
\begin{pgfscope}%
\pgfsys@transformshift{3.069028in}{4.716264in}%
\pgfsys@useobject{currentmarker}{}%
\end{pgfscope}%
\begin{pgfscope}%
\pgfsys@transformshift{3.085116in}{4.710813in}%
\pgfsys@useobject{currentmarker}{}%
\end{pgfscope}%
\begin{pgfscope}%
\pgfsys@transformshift{3.101204in}{5.122010in}%
\pgfsys@useobject{currentmarker}{}%
\end{pgfscope}%
\begin{pgfscope}%
\pgfsys@transformshift{3.117292in}{4.795287in}%
\pgfsys@useobject{currentmarker}{}%
\end{pgfscope}%
\begin{pgfscope}%
\pgfsys@transformshift{3.133380in}{4.440271in}%
\pgfsys@useobject{currentmarker}{}%
\end{pgfscope}%
\begin{pgfscope}%
\pgfsys@transformshift{3.149469in}{4.870125in}%
\pgfsys@useobject{currentmarker}{}%
\end{pgfscope}%
\begin{pgfscope}%
\pgfsys@transformshift{3.165557in}{4.713045in}%
\pgfsys@useobject{currentmarker}{}%
\end{pgfscope}%
\begin{pgfscope}%
\pgfsys@transformshift{3.181645in}{4.756270in}%
\pgfsys@useobject{currentmarker}{}%
\end{pgfscope}%
\begin{pgfscope}%
\pgfsys@transformshift{3.197733in}{5.114426in}%
\pgfsys@useobject{currentmarker}{}%
\end{pgfscope}%
\begin{pgfscope}%
\pgfsys@transformshift{3.213821in}{4.426612in}%
\pgfsys@useobject{currentmarker}{}%
\end{pgfscope}%
\begin{pgfscope}%
\pgfsys@transformshift{3.229909in}{5.175634in}%
\pgfsys@useobject{currentmarker}{}%
\end{pgfscope}%
\begin{pgfscope}%
\pgfsys@transformshift{3.245997in}{4.411317in}%
\pgfsys@useobject{currentmarker}{}%
\end{pgfscope}%
\begin{pgfscope}%
\pgfsys@transformshift{3.262085in}{5.111418in}%
\pgfsys@useobject{currentmarker}{}%
\end{pgfscope}%
\begin{pgfscope}%
\pgfsys@transformshift{3.278173in}{4.661967in}%
\pgfsys@useobject{currentmarker}{}%
\end{pgfscope}%
\begin{pgfscope}%
\pgfsys@transformshift{3.294261in}{4.704661in}%
\pgfsys@useobject{currentmarker}{}%
\end{pgfscope}%
\begin{pgfscope}%
\pgfsys@transformshift{3.310349in}{4.778957in}%
\pgfsys@useobject{currentmarker}{}%
\end{pgfscope}%
\begin{pgfscope}%
\pgfsys@transformshift{3.326437in}{4.966548in}%
\pgfsys@useobject{currentmarker}{}%
\end{pgfscope}%
\begin{pgfscope}%
\pgfsys@transformshift{3.342525in}{4.604391in}%
\pgfsys@useobject{currentmarker}{}%
\end{pgfscope}%
\begin{pgfscope}%
\pgfsys@transformshift{3.358613in}{4.938030in}%
\pgfsys@useobject{currentmarker}{}%
\end{pgfscope}%
\begin{pgfscope}%
\pgfsys@transformshift{3.374701in}{4.734680in}%
\pgfsys@useobject{currentmarker}{}%
\end{pgfscope}%
\begin{pgfscope}%
\pgfsys@transformshift{3.390789in}{4.781293in}%
\pgfsys@useobject{currentmarker}{}%
\end{pgfscope}%
\begin{pgfscope}%
\pgfsys@transformshift{3.406877in}{4.972255in}%
\pgfsys@useobject{currentmarker}{}%
\end{pgfscope}%
\begin{pgfscope}%
\pgfsys@transformshift{3.422965in}{4.650363in}%
\pgfsys@useobject{currentmarker}{}%
\end{pgfscope}%
\begin{pgfscope}%
\pgfsys@transformshift{3.439053in}{4.913377in}%
\pgfsys@useobject{currentmarker}{}%
\end{pgfscope}%
\begin{pgfscope}%
\pgfsys@transformshift{3.455141in}{4.756277in}%
\pgfsys@useobject{currentmarker}{}%
\end{pgfscope}%
\begin{pgfscope}%
\pgfsys@transformshift{3.471229in}{4.509320in}%
\pgfsys@useobject{currentmarker}{}%
\end{pgfscope}%
\begin{pgfscope}%
\pgfsys@transformshift{3.487317in}{5.112256in}%
\pgfsys@useobject{currentmarker}{}%
\end{pgfscope}%
\begin{pgfscope}%
\pgfsys@transformshift{3.503406in}{4.566775in}%
\pgfsys@useobject{currentmarker}{}%
\end{pgfscope}%
\begin{pgfscope}%
\pgfsys@transformshift{3.519494in}{5.093901in}%
\pgfsys@useobject{currentmarker}{}%
\end{pgfscope}%
\begin{pgfscope}%
\pgfsys@transformshift{3.535582in}{4.440150in}%
\pgfsys@useobject{currentmarker}{}%
\end{pgfscope}%
\begin{pgfscope}%
\pgfsys@transformshift{3.551670in}{4.928262in}%
\pgfsys@useobject{currentmarker}{}%
\end{pgfscope}%
\begin{pgfscope}%
\pgfsys@transformshift{3.567758in}{4.707659in}%
\pgfsys@useobject{currentmarker}{}%
\end{pgfscope}%
\begin{pgfscope}%
\pgfsys@transformshift{3.583846in}{4.966635in}%
\pgfsys@useobject{currentmarker}{}%
\end{pgfscope}%
\begin{pgfscope}%
\pgfsys@transformshift{3.599934in}{4.954372in}%
\pgfsys@useobject{currentmarker}{}%
\end{pgfscope}%
\begin{pgfscope}%
\pgfsys@transformshift{3.616022in}{4.588634in}%
\pgfsys@useobject{currentmarker}{}%
\end{pgfscope}%
\begin{pgfscope}%
\pgfsys@transformshift{3.632110in}{4.748275in}%
\pgfsys@useobject{currentmarker}{}%
\end{pgfscope}%
\begin{pgfscope}%
\pgfsys@transformshift{3.648198in}{4.887593in}%
\pgfsys@useobject{currentmarker}{}%
\end{pgfscope}%
\begin{pgfscope}%
\pgfsys@transformshift{3.664286in}{5.017468in}%
\pgfsys@useobject{currentmarker}{}%
\end{pgfscope}%
\begin{pgfscope}%
\pgfsys@transformshift{3.680374in}{4.528588in}%
\pgfsys@useobject{currentmarker}{}%
\end{pgfscope}%
\begin{pgfscope}%
\pgfsys@transformshift{3.696462in}{4.851616in}%
\pgfsys@useobject{currentmarker}{}%
\end{pgfscope}%
\begin{pgfscope}%
\pgfsys@transformshift{3.712550in}{4.748718in}%
\pgfsys@useobject{currentmarker}{}%
\end{pgfscope}%
\begin{pgfscope}%
\pgfsys@transformshift{3.728638in}{5.086520in}%
\pgfsys@useobject{currentmarker}{}%
\end{pgfscope}%
\begin{pgfscope}%
\pgfsys@transformshift{3.744726in}{4.686458in}%
\pgfsys@useobject{currentmarker}{}%
\end{pgfscope}%
\begin{pgfscope}%
\pgfsys@transformshift{3.760814in}{4.783224in}%
\pgfsys@useobject{currentmarker}{}%
\end{pgfscope}%
\begin{pgfscope}%
\pgfsys@transformshift{3.776902in}{4.992656in}%
\pgfsys@useobject{currentmarker}{}%
\end{pgfscope}%
\begin{pgfscope}%
\pgfsys@transformshift{3.792990in}{4.659026in}%
\pgfsys@useobject{currentmarker}{}%
\end{pgfscope}%
\begin{pgfscope}%
\pgfsys@transformshift{3.809078in}{4.987968in}%
\pgfsys@useobject{currentmarker}{}%
\end{pgfscope}%
\begin{pgfscope}%
\pgfsys@transformshift{3.825166in}{4.860141in}%
\pgfsys@useobject{currentmarker}{}%
\end{pgfscope}%
\begin{pgfscope}%
\pgfsys@transformshift{3.841254in}{4.571168in}%
\pgfsys@useobject{currentmarker}{}%
\end{pgfscope}%
\begin{pgfscope}%
\pgfsys@transformshift{3.857343in}{5.145365in}%
\pgfsys@useobject{currentmarker}{}%
\end{pgfscope}%
\begin{pgfscope}%
\pgfsys@transformshift{3.873431in}{4.482676in}%
\pgfsys@useobject{currentmarker}{}%
\end{pgfscope}%
\begin{pgfscope}%
\pgfsys@transformshift{3.889519in}{4.881691in}%
\pgfsys@useobject{currentmarker}{}%
\end{pgfscope}%
\begin{pgfscope}%
\pgfsys@transformshift{3.905607in}{4.781316in}%
\pgfsys@useobject{currentmarker}{}%
\end{pgfscope}%
\begin{pgfscope}%
\pgfsys@transformshift{3.921695in}{4.899537in}%
\pgfsys@useobject{currentmarker}{}%
\end{pgfscope}%
\begin{pgfscope}%
\pgfsys@transformshift{3.937783in}{4.887387in}%
\pgfsys@useobject{currentmarker}{}%
\end{pgfscope}%
\begin{pgfscope}%
\pgfsys@transformshift{3.953871in}{4.849290in}%
\pgfsys@useobject{currentmarker}{}%
\end{pgfscope}%
\begin{pgfscope}%
\pgfsys@transformshift{3.969959in}{4.972828in}%
\pgfsys@useobject{currentmarker}{}%
\end{pgfscope}%
\begin{pgfscope}%
\pgfsys@transformshift{3.986047in}{4.848718in}%
\pgfsys@useobject{currentmarker}{}%
\end{pgfscope}%
\begin{pgfscope}%
\pgfsys@transformshift{4.002135in}{4.551183in}%
\pgfsys@useobject{currentmarker}{}%
\end{pgfscope}%
\begin{pgfscope}%
\pgfsys@transformshift{4.018223in}{5.006168in}%
\pgfsys@useobject{currentmarker}{}%
\end{pgfscope}%
\begin{pgfscope}%
\pgfsys@transformshift{4.034311in}{4.620141in}%
\pgfsys@useobject{currentmarker}{}%
\end{pgfscope}%
\begin{pgfscope}%
\pgfsys@transformshift{4.050399in}{4.984830in}%
\pgfsys@useobject{currentmarker}{}%
\end{pgfscope}%
\begin{pgfscope}%
\pgfsys@transformshift{4.066487in}{4.780899in}%
\pgfsys@useobject{currentmarker}{}%
\end{pgfscope}%
\begin{pgfscope}%
\pgfsys@transformshift{4.082575in}{4.837323in}%
\pgfsys@useobject{currentmarker}{}%
\end{pgfscope}%
\begin{pgfscope}%
\pgfsys@transformshift{4.098663in}{4.911155in}%
\pgfsys@useobject{currentmarker}{}%
\end{pgfscope}%
\begin{pgfscope}%
\pgfsys@transformshift{4.114751in}{4.946560in}%
\pgfsys@useobject{currentmarker}{}%
\end{pgfscope}%
\begin{pgfscope}%
\pgfsys@transformshift{4.130839in}{4.563200in}%
\pgfsys@useobject{currentmarker}{}%
\end{pgfscope}%
\begin{pgfscope}%
\pgfsys@transformshift{4.146927in}{5.186912in}%
\pgfsys@useobject{currentmarker}{}%
\end{pgfscope}%
\begin{pgfscope}%
\pgfsys@transformshift{4.163015in}{4.783880in}%
\pgfsys@useobject{currentmarker}{}%
\end{pgfscope}%
\begin{pgfscope}%
\pgfsys@transformshift{4.179103in}{4.609653in}%
\pgfsys@useobject{currentmarker}{}%
\end{pgfscope}%
\begin{pgfscope}%
\pgfsys@transformshift{4.195191in}{5.169998in}%
\pgfsys@useobject{currentmarker}{}%
\end{pgfscope}%
\begin{pgfscope}%
\pgfsys@transformshift{4.211280in}{4.664859in}%
\pgfsys@useobject{currentmarker}{}%
\end{pgfscope}%
\begin{pgfscope}%
\pgfsys@transformshift{4.227368in}{5.058116in}%
\pgfsys@useobject{currentmarker}{}%
\end{pgfscope}%
\begin{pgfscope}%
\pgfsys@transformshift{4.243456in}{4.510411in}%
\pgfsys@useobject{currentmarker}{}%
\end{pgfscope}%
\begin{pgfscope}%
\pgfsys@transformshift{4.259544in}{4.994214in}%
\pgfsys@useobject{currentmarker}{}%
\end{pgfscope}%
\begin{pgfscope}%
\pgfsys@transformshift{4.275632in}{4.748705in}%
\pgfsys@useobject{currentmarker}{}%
\end{pgfscope}%
\begin{pgfscope}%
\pgfsys@transformshift{4.291720in}{4.859233in}%
\pgfsys@useobject{currentmarker}{}%
\end{pgfscope}%
\begin{pgfscope}%
\pgfsys@transformshift{4.307808in}{4.975849in}%
\pgfsys@useobject{currentmarker}{}%
\end{pgfscope}%
\begin{pgfscope}%
\pgfsys@transformshift{4.323896in}{4.995420in}%
\pgfsys@useobject{currentmarker}{}%
\end{pgfscope}%
\begin{pgfscope}%
\pgfsys@transformshift{4.339984in}{4.807155in}%
\pgfsys@useobject{currentmarker}{}%
\end{pgfscope}%
\begin{pgfscope}%
\pgfsys@transformshift{4.356072in}{4.823072in}%
\pgfsys@useobject{currentmarker}{}%
\end{pgfscope}%
\begin{pgfscope}%
\pgfsys@transformshift{4.372160in}{4.788783in}%
\pgfsys@useobject{currentmarker}{}%
\end{pgfscope}%
\begin{pgfscope}%
\pgfsys@transformshift{4.388248in}{5.013335in}%
\pgfsys@useobject{currentmarker}{}%
\end{pgfscope}%
\begin{pgfscope}%
\pgfsys@transformshift{4.404336in}{4.678593in}%
\pgfsys@useobject{currentmarker}{}%
\end{pgfscope}%
\begin{pgfscope}%
\pgfsys@transformshift{4.420424in}{5.023186in}%
\pgfsys@useobject{currentmarker}{}%
\end{pgfscope}%
\begin{pgfscope}%
\pgfsys@transformshift{4.436512in}{4.845840in}%
\pgfsys@useobject{currentmarker}{}%
\end{pgfscope}%
\begin{pgfscope}%
\pgfsys@transformshift{4.452600in}{4.601184in}%
\pgfsys@useobject{currentmarker}{}%
\end{pgfscope}%
\begin{pgfscope}%
\pgfsys@transformshift{4.468688in}{5.100796in}%
\pgfsys@useobject{currentmarker}{}%
\end{pgfscope}%
\begin{pgfscope}%
\pgfsys@transformshift{4.484776in}{4.544454in}%
\pgfsys@useobject{currentmarker}{}%
\end{pgfscope}%
\begin{pgfscope}%
\pgfsys@transformshift{4.500864in}{4.998413in}%
\pgfsys@useobject{currentmarker}{}%
\end{pgfscope}%
\begin{pgfscope}%
\pgfsys@transformshift{4.516952in}{4.828435in}%
\pgfsys@useobject{currentmarker}{}%
\end{pgfscope}%
\begin{pgfscope}%
\pgfsys@transformshift{4.533040in}{4.986371in}%
\pgfsys@useobject{currentmarker}{}%
\end{pgfscope}%
\begin{pgfscope}%
\pgfsys@transformshift{4.549128in}{4.967384in}%
\pgfsys@useobject{currentmarker}{}%
\end{pgfscope}%
\begin{pgfscope}%
\pgfsys@transformshift{4.565217in}{4.600419in}%
\pgfsys@useobject{currentmarker}{}%
\end{pgfscope}%
\begin{pgfscope}%
\pgfsys@transformshift{4.581305in}{5.048752in}%
\pgfsys@useobject{currentmarker}{}%
\end{pgfscope}%
\begin{pgfscope}%
\pgfsys@transformshift{4.597393in}{5.052437in}%
\pgfsys@useobject{currentmarker}{}%
\end{pgfscope}%
\begin{pgfscope}%
\pgfsys@transformshift{4.613481in}{4.573657in}%
\pgfsys@useobject{currentmarker}{}%
\end{pgfscope}%
\begin{pgfscope}%
\pgfsys@transformshift{4.629569in}{5.065108in}%
\pgfsys@useobject{currentmarker}{}%
\end{pgfscope}%
\begin{pgfscope}%
\pgfsys@transformshift{4.645657in}{4.840580in}%
\pgfsys@useobject{currentmarker}{}%
\end{pgfscope}%
\begin{pgfscope}%
\pgfsys@transformshift{4.661745in}{4.630533in}%
\pgfsys@useobject{currentmarker}{}%
\end{pgfscope}%
\begin{pgfscope}%
\pgfsys@transformshift{4.677833in}{5.075545in}%
\pgfsys@useobject{currentmarker}{}%
\end{pgfscope}%
\begin{pgfscope}%
\pgfsys@transformshift{4.693921in}{4.978011in}%
\pgfsys@useobject{currentmarker}{}%
\end{pgfscope}%
\begin{pgfscope}%
\pgfsys@transformshift{4.710009in}{4.598949in}%
\pgfsys@useobject{currentmarker}{}%
\end{pgfscope}%
\begin{pgfscope}%
\pgfsys@transformshift{4.726097in}{5.235237in}%
\pgfsys@useobject{currentmarker}{}%
\end{pgfscope}%
\begin{pgfscope}%
\pgfsys@transformshift{4.742185in}{4.730320in}%
\pgfsys@useobject{currentmarker}{}%
\end{pgfscope}%
\begin{pgfscope}%
\pgfsys@transformshift{4.758273in}{4.676328in}%
\pgfsys@useobject{currentmarker}{}%
\end{pgfscope}%
\begin{pgfscope}%
\pgfsys@transformshift{4.774361in}{4.966626in}%
\pgfsys@useobject{currentmarker}{}%
\end{pgfscope}%
\begin{pgfscope}%
\pgfsys@transformshift{4.790449in}{5.194404in}%
\pgfsys@useobject{currentmarker}{}%
\end{pgfscope}%
\begin{pgfscope}%
\pgfsys@transformshift{4.806537in}{4.729442in}%
\pgfsys@useobject{currentmarker}{}%
\end{pgfscope}%
\begin{pgfscope}%
\pgfsys@transformshift{4.822625in}{4.686666in}%
\pgfsys@useobject{currentmarker}{}%
\end{pgfscope}%
\begin{pgfscope}%
\pgfsys@transformshift{4.838713in}{4.965001in}%
\pgfsys@useobject{currentmarker}{}%
\end{pgfscope}%
\begin{pgfscope}%
\pgfsys@transformshift{4.854801in}{4.961422in}%
\pgfsys@useobject{currentmarker}{}%
\end{pgfscope}%
\begin{pgfscope}%
\pgfsys@transformshift{4.870889in}{4.706343in}%
\pgfsys@useobject{currentmarker}{}%
\end{pgfscope}%
\begin{pgfscope}%
\pgfsys@transformshift{4.886977in}{5.057292in}%
\pgfsys@useobject{currentmarker}{}%
\end{pgfscope}%
\begin{pgfscope}%
\pgfsys@transformshift{4.903065in}{4.925885in}%
\pgfsys@useobject{currentmarker}{}%
\end{pgfscope}%
\begin{pgfscope}%
\pgfsys@transformshift{4.919154in}{4.789904in}%
\pgfsys@useobject{currentmarker}{}%
\end{pgfscope}%
\begin{pgfscope}%
\pgfsys@transformshift{4.935242in}{5.041066in}%
\pgfsys@useobject{currentmarker}{}%
\end{pgfscope}%
\begin{pgfscope}%
\pgfsys@transformshift{4.951330in}{4.965744in}%
\pgfsys@useobject{currentmarker}{}%
\end{pgfscope}%
\begin{pgfscope}%
\pgfsys@transformshift{4.967418in}{4.678023in}%
\pgfsys@useobject{currentmarker}{}%
\end{pgfscope}%
\begin{pgfscope}%
\pgfsys@transformshift{4.983506in}{5.061003in}%
\pgfsys@useobject{currentmarker}{}%
\end{pgfscope}%
\begin{pgfscope}%
\pgfsys@transformshift{4.999594in}{4.907273in}%
\pgfsys@useobject{currentmarker}{}%
\end{pgfscope}%
\begin{pgfscope}%
\pgfsys@transformshift{5.015682in}{5.031015in}%
\pgfsys@useobject{currentmarker}{}%
\end{pgfscope}%
\end{pgfscope}%
\begin{pgfscope}%
\pgfsetrectcap%
\pgfsetmiterjoin%
\pgfsetlinewidth{0.803000pt}%
\definecolor{currentstroke}{rgb}{0.000000,0.000000,0.000000}%
\pgfsetstrokecolor{currentstroke}%
\pgfsetdash{}{0pt}%
\pgfpathmoveto{\pgfqpoint{0.725000in}{4.295217in}}%
\pgfpathlineto{\pgfqpoint{0.725000in}{5.280000in}}%
\pgfusepath{stroke}%
\end{pgfscope}%
\begin{pgfscope}%
\pgfsetrectcap%
\pgfsetmiterjoin%
\pgfsetlinewidth{0.803000pt}%
\definecolor{currentstroke}{rgb}{0.000000,0.000000,0.000000}%
\pgfsetstrokecolor{currentstroke}%
\pgfsetdash{}{0pt}%
\pgfpathmoveto{\pgfqpoint{5.220000in}{4.295217in}}%
\pgfpathlineto{\pgfqpoint{5.220000in}{5.280000in}}%
\pgfusepath{stroke}%
\end{pgfscope}%
\begin{pgfscope}%
\pgfsetrectcap%
\pgfsetmiterjoin%
\pgfsetlinewidth{0.803000pt}%
\definecolor{currentstroke}{rgb}{0.000000,0.000000,0.000000}%
\pgfsetstrokecolor{currentstroke}%
\pgfsetdash{}{0pt}%
\pgfpathmoveto{\pgfqpoint{0.725000in}{4.295217in}}%
\pgfpathlineto{\pgfqpoint{5.220000in}{4.295217in}}%
\pgfusepath{stroke}%
\end{pgfscope}%
\begin{pgfscope}%
\pgfsetrectcap%
\pgfsetmiterjoin%
\pgfsetlinewidth{0.803000pt}%
\definecolor{currentstroke}{rgb}{0.000000,0.000000,0.000000}%
\pgfsetstrokecolor{currentstroke}%
\pgfsetdash{}{0pt}%
\pgfpathmoveto{\pgfqpoint{0.725000in}{5.280000in}}%
\pgfpathlineto{\pgfqpoint{5.220000in}{5.280000in}}%
\pgfusepath{stroke}%
\end{pgfscope}%
\begin{pgfscope}%
\pgfsetbuttcap%
\pgfsetmiterjoin%
\definecolor{currentfill}{rgb}{1.000000,1.000000,1.000000}%
\pgfsetfillcolor{currentfill}%
\pgfsetlinewidth{0.000000pt}%
\definecolor{currentstroke}{rgb}{0.000000,0.000000,0.000000}%
\pgfsetstrokecolor{currentstroke}%
\pgfsetstrokeopacity{0.000000}%
\pgfsetdash{}{0pt}%
\pgfpathmoveto{\pgfqpoint{0.725000in}{3.113478in}}%
\pgfpathlineto{\pgfqpoint{5.220000in}{3.113478in}}%
\pgfpathlineto{\pgfqpoint{5.220000in}{4.098261in}}%
\pgfpathlineto{\pgfqpoint{0.725000in}{4.098261in}}%
\pgfpathclose%
\pgfusepath{fill}%
\end{pgfscope}%
\begin{pgfscope}%
\pgfsetbuttcap%
\pgfsetroundjoin%
\definecolor{currentfill}{rgb}{0.000000,0.000000,0.000000}%
\pgfsetfillcolor{currentfill}%
\pgfsetlinewidth{0.803000pt}%
\definecolor{currentstroke}{rgb}{0.000000,0.000000,0.000000}%
\pgfsetstrokecolor{currentstroke}%
\pgfsetdash{}{0pt}%
\pgfsys@defobject{currentmarker}{\pgfqpoint{0.000000in}{-0.048611in}}{\pgfqpoint{0.000000in}{0.000000in}}{%
\pgfpathmoveto{\pgfqpoint{0.000000in}{0.000000in}}%
\pgfpathlineto{\pgfqpoint{0.000000in}{-0.048611in}}%
\pgfusepath{stroke,fill}%
}%
\begin{pgfscope}%
\pgfsys@transformshift{0.929318in}{3.113478in}%
\pgfsys@useobject{currentmarker}{}%
\end{pgfscope}%
\end{pgfscope}%
\begin{pgfscope}%
\pgfsetbuttcap%
\pgfsetroundjoin%
\definecolor{currentfill}{rgb}{0.000000,0.000000,0.000000}%
\pgfsetfillcolor{currentfill}%
\pgfsetlinewidth{0.803000pt}%
\definecolor{currentstroke}{rgb}{0.000000,0.000000,0.000000}%
\pgfsetstrokecolor{currentstroke}%
\pgfsetdash{}{0pt}%
\pgfsys@defobject{currentmarker}{\pgfqpoint{0.000000in}{-0.048611in}}{\pgfqpoint{0.000000in}{0.000000in}}{%
\pgfpathmoveto{\pgfqpoint{0.000000in}{0.000000in}}%
\pgfpathlineto{\pgfqpoint{0.000000in}{-0.048611in}}%
\pgfusepath{stroke,fill}%
}%
\begin{pgfscope}%
\pgfsys@transformshift{1.733720in}{3.113478in}%
\pgfsys@useobject{currentmarker}{}%
\end{pgfscope}%
\end{pgfscope}%
\begin{pgfscope}%
\pgfsetbuttcap%
\pgfsetroundjoin%
\definecolor{currentfill}{rgb}{0.000000,0.000000,0.000000}%
\pgfsetfillcolor{currentfill}%
\pgfsetlinewidth{0.803000pt}%
\definecolor{currentstroke}{rgb}{0.000000,0.000000,0.000000}%
\pgfsetstrokecolor{currentstroke}%
\pgfsetdash{}{0pt}%
\pgfsys@defobject{currentmarker}{\pgfqpoint{0.000000in}{-0.048611in}}{\pgfqpoint{0.000000in}{0.000000in}}{%
\pgfpathmoveto{\pgfqpoint{0.000000in}{0.000000in}}%
\pgfpathlineto{\pgfqpoint{0.000000in}{-0.048611in}}%
\pgfusepath{stroke,fill}%
}%
\begin{pgfscope}%
\pgfsys@transformshift{2.538123in}{3.113478in}%
\pgfsys@useobject{currentmarker}{}%
\end{pgfscope}%
\end{pgfscope}%
\begin{pgfscope}%
\pgfsetbuttcap%
\pgfsetroundjoin%
\definecolor{currentfill}{rgb}{0.000000,0.000000,0.000000}%
\pgfsetfillcolor{currentfill}%
\pgfsetlinewidth{0.803000pt}%
\definecolor{currentstroke}{rgb}{0.000000,0.000000,0.000000}%
\pgfsetstrokecolor{currentstroke}%
\pgfsetdash{}{0pt}%
\pgfsys@defobject{currentmarker}{\pgfqpoint{0.000000in}{-0.048611in}}{\pgfqpoint{0.000000in}{0.000000in}}{%
\pgfpathmoveto{\pgfqpoint{0.000000in}{0.000000in}}%
\pgfpathlineto{\pgfqpoint{0.000000in}{-0.048611in}}%
\pgfusepath{stroke,fill}%
}%
\begin{pgfscope}%
\pgfsys@transformshift{3.342525in}{3.113478in}%
\pgfsys@useobject{currentmarker}{}%
\end{pgfscope}%
\end{pgfscope}%
\begin{pgfscope}%
\pgfsetbuttcap%
\pgfsetroundjoin%
\definecolor{currentfill}{rgb}{0.000000,0.000000,0.000000}%
\pgfsetfillcolor{currentfill}%
\pgfsetlinewidth{0.803000pt}%
\definecolor{currentstroke}{rgb}{0.000000,0.000000,0.000000}%
\pgfsetstrokecolor{currentstroke}%
\pgfsetdash{}{0pt}%
\pgfsys@defobject{currentmarker}{\pgfqpoint{0.000000in}{-0.048611in}}{\pgfqpoint{0.000000in}{0.000000in}}{%
\pgfpathmoveto{\pgfqpoint{0.000000in}{0.000000in}}%
\pgfpathlineto{\pgfqpoint{0.000000in}{-0.048611in}}%
\pgfusepath{stroke,fill}%
}%
\begin{pgfscope}%
\pgfsys@transformshift{4.146927in}{3.113478in}%
\pgfsys@useobject{currentmarker}{}%
\end{pgfscope}%
\end{pgfscope}%
\begin{pgfscope}%
\pgfsetbuttcap%
\pgfsetroundjoin%
\definecolor{currentfill}{rgb}{0.000000,0.000000,0.000000}%
\pgfsetfillcolor{currentfill}%
\pgfsetlinewidth{0.803000pt}%
\definecolor{currentstroke}{rgb}{0.000000,0.000000,0.000000}%
\pgfsetstrokecolor{currentstroke}%
\pgfsetdash{}{0pt}%
\pgfsys@defobject{currentmarker}{\pgfqpoint{0.000000in}{-0.048611in}}{\pgfqpoint{0.000000in}{0.000000in}}{%
\pgfpathmoveto{\pgfqpoint{0.000000in}{0.000000in}}%
\pgfpathlineto{\pgfqpoint{0.000000in}{-0.048611in}}%
\pgfusepath{stroke,fill}%
}%
\begin{pgfscope}%
\pgfsys@transformshift{4.951330in}{3.113478in}%
\pgfsys@useobject{currentmarker}{}%
\end{pgfscope}%
\end{pgfscope}%
\begin{pgfscope}%
\pgfsetbuttcap%
\pgfsetroundjoin%
\definecolor{currentfill}{rgb}{0.000000,0.000000,0.000000}%
\pgfsetfillcolor{currentfill}%
\pgfsetlinewidth{0.803000pt}%
\definecolor{currentstroke}{rgb}{0.000000,0.000000,0.000000}%
\pgfsetstrokecolor{currentstroke}%
\pgfsetdash{}{0pt}%
\pgfsys@defobject{currentmarker}{\pgfqpoint{-0.048611in}{0.000000in}}{\pgfqpoint{0.000000in}{0.000000in}}{%
\pgfpathmoveto{\pgfqpoint{0.000000in}{0.000000in}}%
\pgfpathlineto{\pgfqpoint{-0.048611in}{0.000000in}}%
\pgfusepath{stroke,fill}%
}%
\begin{pgfscope}%
\pgfsys@transformshift{0.725000in}{3.408141in}%
\pgfsys@useobject{currentmarker}{}%
\end{pgfscope}%
\end{pgfscope}%
\begin{pgfscope}%
\definecolor{textcolor}{rgb}{0.000000,0.000000,0.000000}%
\pgfsetstrokecolor{textcolor}%
\pgfsetfillcolor{textcolor}%
\pgftext[x=0.418000in,y=3.369586in,left,base]{\color{textcolor}\rmfamily\fontsize{8.000000}{9.600000}\selectfont 0.00}%
\end{pgfscope}%
\begin{pgfscope}%
\pgfsetbuttcap%
\pgfsetroundjoin%
\definecolor{currentfill}{rgb}{0.000000,0.000000,0.000000}%
\pgfsetfillcolor{currentfill}%
\pgfsetlinewidth{0.803000pt}%
\definecolor{currentstroke}{rgb}{0.000000,0.000000,0.000000}%
\pgfsetstrokecolor{currentstroke}%
\pgfsetdash{}{0pt}%
\pgfsys@defobject{currentmarker}{\pgfqpoint{-0.048611in}{0.000000in}}{\pgfqpoint{0.000000in}{0.000000in}}{%
\pgfpathmoveto{\pgfqpoint{0.000000in}{0.000000in}}%
\pgfpathlineto{\pgfqpoint{-0.048611in}{0.000000in}}%
\pgfusepath{stroke,fill}%
}%
\begin{pgfscope}%
\pgfsys@transformshift{0.725000in}{3.728133in}%
\pgfsys@useobject{currentmarker}{}%
\end{pgfscope}%
\end{pgfscope}%
\begin{pgfscope}%
\definecolor{textcolor}{rgb}{0.000000,0.000000,0.000000}%
\pgfsetstrokecolor{textcolor}%
\pgfsetfillcolor{textcolor}%
\pgftext[x=0.418000in,y=3.689577in,left,base]{\color{textcolor}\rmfamily\fontsize{8.000000}{9.600000}\selectfont 0.02}%
\end{pgfscope}%
\begin{pgfscope}%
\pgfsetbuttcap%
\pgfsetroundjoin%
\definecolor{currentfill}{rgb}{0.000000,0.000000,0.000000}%
\pgfsetfillcolor{currentfill}%
\pgfsetlinewidth{0.803000pt}%
\definecolor{currentstroke}{rgb}{0.000000,0.000000,0.000000}%
\pgfsetstrokecolor{currentstroke}%
\pgfsetdash{}{0pt}%
\pgfsys@defobject{currentmarker}{\pgfqpoint{-0.048611in}{0.000000in}}{\pgfqpoint{0.000000in}{0.000000in}}{%
\pgfpathmoveto{\pgfqpoint{0.000000in}{0.000000in}}%
\pgfpathlineto{\pgfqpoint{-0.048611in}{0.000000in}}%
\pgfusepath{stroke,fill}%
}%
\begin{pgfscope}%
\pgfsys@transformshift{0.725000in}{4.048124in}%
\pgfsys@useobject{currentmarker}{}%
\end{pgfscope}%
\end{pgfscope}%
\begin{pgfscope}%
\definecolor{textcolor}{rgb}{0.000000,0.000000,0.000000}%
\pgfsetstrokecolor{textcolor}%
\pgfsetfillcolor{textcolor}%
\pgftext[x=0.418000in,y=4.009569in,left,base]{\color{textcolor}\rmfamily\fontsize{8.000000}{9.600000}\selectfont 0.04}%
\end{pgfscope}%
\begin{pgfscope}%
\pgfpathrectangle{\pgfqpoint{0.725000in}{3.113478in}}{\pgfqpoint{4.495000in}{0.984783in}}%
\pgfusepath{clip}%
\pgfsetrectcap%
\pgfsetroundjoin%
\pgfsetlinewidth{1.505625pt}%
\definecolor{currentstroke}{rgb}{1.000000,0.498039,0.054902}%
\pgfsetstrokecolor{currentstroke}%
\pgfsetdash{}{0pt}%
\pgfpathmoveto{\pgfqpoint{0.937362in}{3.243609in}}%
\pgfpathlineto{\pgfqpoint{0.953450in}{3.243609in}}%
\pgfpathlineto{\pgfqpoint{0.953450in}{3.307419in}}%
\pgfpathlineto{\pgfqpoint{0.985626in}{3.307419in}}%
\pgfpathlineto{\pgfqpoint{0.985626in}{3.585829in}}%
\pgfpathlineto{\pgfqpoint{1.017802in}{3.585829in}}%
\pgfpathlineto{\pgfqpoint{1.017802in}{3.368950in}}%
\pgfpathlineto{\pgfqpoint{1.049979in}{3.368950in}}%
\pgfpathlineto{\pgfqpoint{1.049979in}{3.561909in}}%
\pgfpathlineto{\pgfqpoint{1.082155in}{3.561909in}}%
\pgfpathlineto{\pgfqpoint{1.082155in}{3.219498in}}%
\pgfpathlineto{\pgfqpoint{1.114331in}{3.219498in}}%
\pgfpathlineto{\pgfqpoint{1.114331in}{3.653691in}}%
\pgfpathlineto{\pgfqpoint{1.146507in}{3.653691in}}%
\pgfpathlineto{\pgfqpoint{1.146507in}{3.158241in}}%
\pgfpathlineto{\pgfqpoint{1.178683in}{3.158241in}}%
\pgfpathlineto{\pgfqpoint{1.178683in}{3.466740in}}%
\pgfpathlineto{\pgfqpoint{1.210859in}{3.466740in}}%
\pgfpathlineto{\pgfqpoint{1.210859in}{3.477069in}}%
\pgfpathlineto{\pgfqpoint{1.243035in}{3.477069in}}%
\pgfpathlineto{\pgfqpoint{1.243035in}{3.506960in}}%
\pgfpathlineto{\pgfqpoint{1.275211in}{3.506960in}}%
\pgfpathlineto{\pgfqpoint{1.275211in}{3.434237in}}%
\pgfpathlineto{\pgfqpoint{1.307387in}{3.434237in}}%
\pgfpathlineto{\pgfqpoint{1.307387in}{3.697227in}}%
\pgfpathlineto{\pgfqpoint{1.339563in}{3.697227in}}%
\pgfpathlineto{\pgfqpoint{1.339563in}{3.207386in}}%
\pgfpathlineto{\pgfqpoint{1.371739in}{3.207386in}}%
\pgfpathlineto{\pgfqpoint{1.371739in}{3.486173in}}%
\pgfpathlineto{\pgfqpoint{1.403916in}{3.486173in}}%
\pgfpathlineto{\pgfqpoint{1.403916in}{3.459328in}}%
\pgfpathlineto{\pgfqpoint{1.436092in}{3.459328in}}%
\pgfpathlineto{\pgfqpoint{1.436092in}{3.495285in}}%
\pgfpathlineto{\pgfqpoint{1.468268in}{3.495285in}}%
\pgfpathlineto{\pgfqpoint{1.468268in}{3.624555in}}%
\pgfpathlineto{\pgfqpoint{1.500444in}{3.624555in}}%
\pgfpathlineto{\pgfqpoint{1.500444in}{3.439532in}}%
\pgfpathlineto{\pgfqpoint{1.532620in}{3.439532in}}%
\pgfpathlineto{\pgfqpoint{1.532620in}{3.558140in}}%
\pgfpathlineto{\pgfqpoint{1.564796in}{3.558140in}}%
\pgfpathlineto{\pgfqpoint{1.564796in}{3.352151in}}%
\pgfpathlineto{\pgfqpoint{1.596972in}{3.352151in}}%
\pgfpathlineto{\pgfqpoint{1.596972in}{3.530543in}}%
\pgfpathlineto{\pgfqpoint{1.629148in}{3.530543in}}%
\pgfpathlineto{\pgfqpoint{1.629148in}{3.330087in}}%
\pgfpathlineto{\pgfqpoint{1.661324in}{3.330087in}}%
\pgfpathlineto{\pgfqpoint{1.661324in}{3.553880in}}%
\pgfpathlineto{\pgfqpoint{1.693500in}{3.553880in}}%
\pgfpathlineto{\pgfqpoint{1.693500in}{3.642791in}}%
\pgfpathlineto{\pgfqpoint{1.725676in}{3.642791in}}%
\pgfpathlineto{\pgfqpoint{1.725676in}{3.352736in}}%
\pgfpathlineto{\pgfqpoint{1.757853in}{3.352736in}}%
\pgfpathlineto{\pgfqpoint{1.757853in}{3.513599in}}%
\pgfpathlineto{\pgfqpoint{1.790029in}{3.513599in}}%
\pgfpathlineto{\pgfqpoint{1.790029in}{3.647724in}}%
\pgfpathlineto{\pgfqpoint{1.822205in}{3.647724in}}%
\pgfpathlineto{\pgfqpoint{1.822205in}{3.507944in}}%
\pgfpathlineto{\pgfqpoint{1.854381in}{3.507944in}}%
\pgfpathlineto{\pgfqpoint{1.854381in}{3.502924in}}%
\pgfpathlineto{\pgfqpoint{1.886557in}{3.502924in}}%
\pgfpathlineto{\pgfqpoint{1.886557in}{3.481067in}}%
\pgfpathlineto{\pgfqpoint{1.918733in}{3.481067in}}%
\pgfpathlineto{\pgfqpoint{1.918733in}{3.661928in}}%
\pgfpathlineto{\pgfqpoint{1.950909in}{3.661928in}}%
\pgfpathlineto{\pgfqpoint{1.950909in}{3.522268in}}%
\pgfpathlineto{\pgfqpoint{1.983085in}{3.522268in}}%
\pgfpathlineto{\pgfqpoint{1.983085in}{3.475589in}}%
\pgfpathlineto{\pgfqpoint{2.015261in}{3.475589in}}%
\pgfpathlineto{\pgfqpoint{2.015261in}{3.553692in}}%
\pgfpathlineto{\pgfqpoint{2.047437in}{3.553692in}}%
\pgfpathlineto{\pgfqpoint{2.047437in}{3.465510in}}%
\pgfpathlineto{\pgfqpoint{2.079613in}{3.465510in}}%
\pgfpathlineto{\pgfqpoint{2.079613in}{3.619516in}}%
\pgfpathlineto{\pgfqpoint{2.111790in}{3.619516in}}%
\pgfpathlineto{\pgfqpoint{2.111790in}{3.642930in}}%
\pgfpathlineto{\pgfqpoint{2.143966in}{3.642930in}}%
\pgfpathlineto{\pgfqpoint{2.143966in}{3.518213in}}%
\pgfpathlineto{\pgfqpoint{2.176142in}{3.518213in}}%
\pgfpathlineto{\pgfqpoint{2.176142in}{3.513859in}}%
\pgfpathlineto{\pgfqpoint{2.208318in}{3.513859in}}%
\pgfpathlineto{\pgfqpoint{2.208318in}{3.716002in}}%
\pgfpathlineto{\pgfqpoint{2.240494in}{3.716002in}}%
\pgfpathlineto{\pgfqpoint{2.240494in}{3.387104in}}%
\pgfpathlineto{\pgfqpoint{2.272670in}{3.387104in}}%
\pgfpathlineto{\pgfqpoint{2.272670in}{3.762567in}}%
\pgfpathlineto{\pgfqpoint{2.304846in}{3.762567in}}%
\pgfpathlineto{\pgfqpoint{2.304846in}{3.386496in}}%
\pgfpathlineto{\pgfqpoint{2.337022in}{3.386496in}}%
\pgfpathlineto{\pgfqpoint{2.337022in}{3.552073in}}%
\pgfpathlineto{\pgfqpoint{2.369198in}{3.552073in}}%
\pgfpathlineto{\pgfqpoint{2.369198in}{3.860467in}}%
\pgfpathlineto{\pgfqpoint{2.401374in}{3.860467in}}%
\pgfpathlineto{\pgfqpoint{2.401374in}{3.344604in}}%
\pgfpathlineto{\pgfqpoint{2.433550in}{3.344604in}}%
\pgfpathlineto{\pgfqpoint{2.433550in}{3.784309in}}%
\pgfpathlineto{\pgfqpoint{2.465727in}{3.784309in}}%
\pgfpathlineto{\pgfqpoint{2.465727in}{3.368330in}}%
\pgfpathlineto{\pgfqpoint{2.497903in}{3.368330in}}%
\pgfpathlineto{\pgfqpoint{2.497903in}{3.796140in}}%
\pgfpathlineto{\pgfqpoint{2.530079in}{3.796140in}}%
\pgfpathlineto{\pgfqpoint{2.530079in}{3.551744in}}%
\pgfpathlineto{\pgfqpoint{2.562255in}{3.551744in}}%
\pgfpathlineto{\pgfqpoint{2.562255in}{3.566766in}}%
\pgfpathlineto{\pgfqpoint{2.594431in}{3.566766in}}%
\pgfpathlineto{\pgfqpoint{2.594431in}{3.596707in}}%
\pgfpathlineto{\pgfqpoint{2.626607in}{3.596707in}}%
\pgfpathlineto{\pgfqpoint{2.626607in}{3.625147in}}%
\pgfpathlineto{\pgfqpoint{2.658783in}{3.625147in}}%
\pgfpathlineto{\pgfqpoint{2.658783in}{3.794937in}}%
\pgfpathlineto{\pgfqpoint{2.690959in}{3.794937in}}%
\pgfpathlineto{\pgfqpoint{2.690959in}{3.501773in}}%
\pgfpathlineto{\pgfqpoint{2.723135in}{3.501773in}}%
\pgfpathlineto{\pgfqpoint{2.723135in}{3.699441in}}%
\pgfpathlineto{\pgfqpoint{2.755311in}{3.699441in}}%
\pgfpathlineto{\pgfqpoint{2.755311in}{3.592142in}}%
\pgfpathlineto{\pgfqpoint{2.787487in}{3.592142in}}%
\pgfpathlineto{\pgfqpoint{2.787487in}{3.712758in}}%
\pgfpathlineto{\pgfqpoint{2.819664in}{3.712758in}}%
\pgfpathlineto{\pgfqpoint{2.819664in}{3.543507in}}%
\pgfpathlineto{\pgfqpoint{2.851840in}{3.543507in}}%
\pgfpathlineto{\pgfqpoint{2.851840in}{3.631361in}}%
\pgfpathlineto{\pgfqpoint{2.884016in}{3.631361in}}%
\pgfpathlineto{\pgfqpoint{2.884016in}{3.668938in}}%
\pgfpathlineto{\pgfqpoint{2.916192in}{3.668938in}}%
\pgfpathlineto{\pgfqpoint{2.916192in}{3.675804in}}%
\pgfpathlineto{\pgfqpoint{2.948368in}{3.675804in}}%
\pgfpathlineto{\pgfqpoint{2.948368in}{3.734148in}}%
\pgfpathlineto{\pgfqpoint{2.980544in}{3.734148in}}%
\pgfpathlineto{\pgfqpoint{2.980544in}{3.527775in}}%
\pgfpathlineto{\pgfqpoint{3.012720in}{3.527775in}}%
\pgfpathlineto{\pgfqpoint{3.012720in}{3.735169in}}%
\pgfpathlineto{\pgfqpoint{3.044896in}{3.735169in}}%
\pgfpathlineto{\pgfqpoint{3.044896in}{3.493292in}}%
\pgfpathlineto{\pgfqpoint{3.077072in}{3.493292in}}%
\pgfpathlineto{\pgfqpoint{3.077072in}{3.979601in}}%
\pgfpathlineto{\pgfqpoint{3.109248in}{3.979601in}}%
\pgfpathlineto{\pgfqpoint{3.109248in}{3.410811in}}%
\pgfpathlineto{\pgfqpoint{3.141424in}{3.410811in}}%
\pgfpathlineto{\pgfqpoint{3.141424in}{3.650072in}}%
\pgfpathlineto{\pgfqpoint{3.173601in}{3.650072in}}%
\pgfpathlineto{\pgfqpoint{3.173601in}{3.819764in}}%
\pgfpathlineto{\pgfqpoint{3.205777in}{3.819764in}}%
\pgfpathlineto{\pgfqpoint{3.205777in}{3.714648in}}%
\pgfpathlineto{\pgfqpoint{3.237953in}{3.714648in}}%
\pgfpathlineto{\pgfqpoint{3.237953in}{3.765153in}}%
\pgfpathlineto{\pgfqpoint{3.270129in}{3.765153in}}%
\pgfpathlineto{\pgfqpoint{3.270129in}{3.554544in}}%
\pgfpathlineto{\pgfqpoint{3.302305in}{3.554544in}}%
\pgfpathlineto{\pgfqpoint{3.302305in}{3.774754in}}%
\pgfpathlineto{\pgfqpoint{3.334481in}{3.774754in}}%
\pgfpathlineto{\pgfqpoint{3.334481in}{3.726898in}}%
\pgfpathlineto{\pgfqpoint{3.366657in}{3.726898in}}%
\pgfpathlineto{\pgfqpoint{3.366657in}{3.752590in}}%
\pgfpathlineto{\pgfqpoint{3.398833in}{3.752590in}}%
\pgfpathlineto{\pgfqpoint{3.398833in}{3.713307in}}%
\pgfpathlineto{\pgfqpoint{3.431009in}{3.713307in}}%
\pgfpathlineto{\pgfqpoint{3.431009in}{3.594696in}}%
\pgfpathlineto{\pgfqpoint{3.463185in}{3.594696in}}%
\pgfpathlineto{\pgfqpoint{3.463185in}{3.767273in}}%
\pgfpathlineto{\pgfqpoint{3.495361in}{3.767273in}}%
\pgfpathlineto{\pgfqpoint{3.495361in}{3.717260in}}%
\pgfpathlineto{\pgfqpoint{3.527538in}{3.717260in}}%
\pgfpathlineto{\pgfqpoint{3.527538in}{3.627327in}}%
\pgfpathlineto{\pgfqpoint{3.559714in}{3.627327in}}%
\pgfpathlineto{\pgfqpoint{3.559714in}{3.906472in}}%
\pgfpathlineto{\pgfqpoint{3.591890in}{3.906472in}}%
\pgfpathlineto{\pgfqpoint{3.591890in}{3.568559in}}%
\pgfpathlineto{\pgfqpoint{3.624066in}{3.568559in}}%
\pgfpathlineto{\pgfqpoint{3.624066in}{3.880790in}}%
\pgfpathlineto{\pgfqpoint{3.656242in}{3.880790in}}%
\pgfpathlineto{\pgfqpoint{3.656242in}{3.590450in}}%
\pgfpathlineto{\pgfqpoint{3.688418in}{3.590450in}}%
\pgfpathlineto{\pgfqpoint{3.688418in}{3.831023in}}%
\pgfpathlineto{\pgfqpoint{3.720594in}{3.831023in}}%
\pgfpathlineto{\pgfqpoint{3.720594in}{3.739902in}}%
\pgfpathlineto{\pgfqpoint{3.752770in}{3.739902in}}%
\pgfpathlineto{\pgfqpoint{3.752770in}{3.827240in}}%
\pgfpathlineto{\pgfqpoint{3.784946in}{3.827240in}}%
\pgfpathlineto{\pgfqpoint{3.784946in}{3.859144in}}%
\pgfpathlineto{\pgfqpoint{3.817122in}{3.859144in}}%
\pgfpathlineto{\pgfqpoint{3.817122in}{3.695114in}}%
\pgfpathlineto{\pgfqpoint{3.849298in}{3.695114in}}%
\pgfpathlineto{\pgfqpoint{3.849298in}{3.621694in}}%
\pgfpathlineto{\pgfqpoint{3.881475in}{3.621694in}}%
\pgfpathlineto{\pgfqpoint{3.881475in}{3.787704in}}%
\pgfpathlineto{\pgfqpoint{3.913651in}{3.787704in}}%
\pgfpathlineto{\pgfqpoint{3.913651in}{3.872605in}}%
\pgfpathlineto{\pgfqpoint{3.945827in}{3.872605in}}%
\pgfpathlineto{\pgfqpoint{3.945827in}{3.929317in}}%
\pgfpathlineto{\pgfqpoint{3.978003in}{3.929317in}}%
\pgfpathlineto{\pgfqpoint{3.978003in}{3.605088in}}%
\pgfpathlineto{\pgfqpoint{4.010179in}{3.605088in}}%
\pgfpathlineto{\pgfqpoint{4.010179in}{3.734526in}}%
\pgfpathlineto{\pgfqpoint{4.042355in}{3.734526in}}%
\pgfpathlineto{\pgfqpoint{4.042355in}{3.806641in}}%
\pgfpathlineto{\pgfqpoint{4.074531in}{3.806641in}}%
\pgfpathlineto{\pgfqpoint{4.074531in}{3.911617in}}%
\pgfpathlineto{\pgfqpoint{4.106707in}{3.911617in}}%
\pgfpathlineto{\pgfqpoint{4.106707in}{3.748032in}}%
\pgfpathlineto{\pgfqpoint{4.138883in}{3.748032in}}%
\pgfpathlineto{\pgfqpoint{4.138883in}{3.797370in}}%
\pgfpathlineto{\pgfqpoint{4.171059in}{3.797370in}}%
\pgfpathlineto{\pgfqpoint{4.171059in}{3.915544in}}%
\pgfpathlineto{\pgfqpoint{4.203236in}{3.915544in}}%
\pgfpathlineto{\pgfqpoint{4.203236in}{3.762972in}}%
\pgfpathlineto{\pgfqpoint{4.235412in}{3.762972in}}%
\pgfpathlineto{\pgfqpoint{4.235412in}{3.742208in}}%
\pgfpathlineto{\pgfqpoint{4.267588in}{3.742208in}}%
\pgfpathlineto{\pgfqpoint{4.267588in}{3.834542in}}%
\pgfpathlineto{\pgfqpoint{4.299764in}{3.834542in}}%
\pgfpathlineto{\pgfqpoint{4.299764in}{3.990807in}}%
\pgfpathlineto{\pgfqpoint{4.331940in}{3.990807in}}%
\pgfpathlineto{\pgfqpoint{4.331940in}{3.739628in}}%
\pgfpathlineto{\pgfqpoint{4.364116in}{3.739628in}}%
\pgfpathlineto{\pgfqpoint{4.364116in}{3.858644in}}%
\pgfpathlineto{\pgfqpoint{4.396292in}{3.858644in}}%
\pgfpathlineto{\pgfqpoint{4.396292in}{3.894902in}}%
\pgfpathlineto{\pgfqpoint{4.428468in}{3.894902in}}%
\pgfpathlineto{\pgfqpoint{4.428468in}{3.695663in}}%
\pgfpathlineto{\pgfqpoint{4.460644in}{3.695663in}}%
\pgfpathlineto{\pgfqpoint{4.460644in}{3.714138in}}%
\pgfpathlineto{\pgfqpoint{4.492820in}{3.714138in}}%
\pgfpathlineto{\pgfqpoint{4.492820in}{3.928367in}}%
\pgfpathlineto{\pgfqpoint{4.524996in}{3.928367in}}%
\pgfpathlineto{\pgfqpoint{4.524996in}{3.871640in}}%
\pgfpathlineto{\pgfqpoint{4.557173in}{3.871640in}}%
\pgfpathlineto{\pgfqpoint{4.557173in}{3.979715in}}%
\pgfpathlineto{\pgfqpoint{4.589349in}{3.979715in}}%
\pgfpathlineto{\pgfqpoint{4.589349in}{3.750388in}}%
\pgfpathlineto{\pgfqpoint{4.621525in}{3.750388in}}%
\pgfpathlineto{\pgfqpoint{4.621525in}{3.803264in}}%
\pgfpathlineto{\pgfqpoint{4.653701in}{3.803264in}}%
\pgfpathlineto{\pgfqpoint{4.653701in}{3.984095in}}%
\pgfpathlineto{\pgfqpoint{4.685877in}{3.984095in}}%
\pgfpathlineto{\pgfqpoint{4.685877in}{3.819486in}}%
\pgfpathlineto{\pgfqpoint{4.718053in}{3.819486in}}%
\pgfpathlineto{\pgfqpoint{4.718053in}{3.801093in}}%
\pgfpathlineto{\pgfqpoint{4.750229in}{3.801093in}}%
\pgfpathlineto{\pgfqpoint{4.750229in}{4.005044in}}%
\pgfpathlineto{\pgfqpoint{4.782405in}{4.005044in}}%
\pgfpathlineto{\pgfqpoint{4.782405in}{3.785858in}}%
\pgfpathlineto{\pgfqpoint{4.814581in}{3.785858in}}%
\pgfpathlineto{\pgfqpoint{4.814581in}{3.898343in}}%
\pgfpathlineto{\pgfqpoint{4.846757in}{3.898343in}}%
\pgfpathlineto{\pgfqpoint{4.846757in}{3.829054in}}%
\pgfpathlineto{\pgfqpoint{4.878933in}{3.829054in}}%
\pgfpathlineto{\pgfqpoint{4.878933in}{3.955438in}}%
\pgfpathlineto{\pgfqpoint{4.911110in}{3.955438in}}%
\pgfpathlineto{\pgfqpoint{4.911110in}{4.021008in}}%
\pgfpathlineto{\pgfqpoint{4.943286in}{4.021008in}}%
\pgfpathlineto{\pgfqpoint{4.943286in}{3.806093in}}%
\pgfpathlineto{\pgfqpoint{4.975462in}{3.806093in}}%
\pgfpathlineto{\pgfqpoint{4.975462in}{4.053498in}}%
\pgfpathlineto{\pgfqpoint{4.991550in}{4.053498in}}%
\pgfpathlineto{\pgfqpoint{4.991550in}{4.053498in}}%
\pgfusepath{stroke}%
\end{pgfscope}%
\begin{pgfscope}%
\pgfpathrectangle{\pgfqpoint{0.725000in}{3.113478in}}{\pgfqpoint{4.495000in}{0.984783in}}%
\pgfusepath{clip}%
\pgfsetbuttcap%
\pgfsetroundjoin%
\definecolor{currentfill}{rgb}{1.000000,0.498039,0.054902}%
\pgfsetfillcolor{currentfill}%
\pgfsetlinewidth{1.003750pt}%
\definecolor{currentstroke}{rgb}{1.000000,0.498039,0.054902}%
\pgfsetstrokecolor{currentstroke}%
\pgfsetdash{}{0pt}%
\pgfsys@defobject{currentmarker}{\pgfqpoint{-0.041667in}{-0.041667in}}{\pgfqpoint{0.041667in}{0.041667in}}{%
\pgfpathmoveto{\pgfqpoint{0.000000in}{-0.041667in}}%
\pgfpathcurveto{\pgfqpoint{0.011050in}{-0.041667in}}{\pgfqpoint{0.021649in}{-0.037276in}}{\pgfqpoint{0.029463in}{-0.029463in}}%
\pgfpathcurveto{\pgfqpoint{0.037276in}{-0.021649in}}{\pgfqpoint{0.041667in}{-0.011050in}}{\pgfqpoint{0.041667in}{0.000000in}}%
\pgfpathcurveto{\pgfqpoint{0.041667in}{0.011050in}}{\pgfqpoint{0.037276in}{0.021649in}}{\pgfqpoint{0.029463in}{0.029463in}}%
\pgfpathcurveto{\pgfqpoint{0.021649in}{0.037276in}}{\pgfqpoint{0.011050in}{0.041667in}}{\pgfqpoint{0.000000in}{0.041667in}}%
\pgfpathcurveto{\pgfqpoint{-0.011050in}{0.041667in}}{\pgfqpoint{-0.021649in}{0.037276in}}{\pgfqpoint{-0.029463in}{0.029463in}}%
\pgfpathcurveto{\pgfqpoint{-0.037276in}{0.021649in}}{\pgfqpoint{-0.041667in}{0.011050in}}{\pgfqpoint{-0.041667in}{0.000000in}}%
\pgfpathcurveto{\pgfqpoint{-0.041667in}{-0.011050in}}{\pgfqpoint{-0.037276in}{-0.021649in}}{\pgfqpoint{-0.029463in}{-0.029463in}}%
\pgfpathcurveto{\pgfqpoint{-0.021649in}{-0.037276in}}{\pgfqpoint{-0.011050in}{-0.041667in}}{\pgfqpoint{0.000000in}{-0.041667in}}%
\pgfpathclose%
\pgfusepath{stroke,fill}%
}%
\begin{pgfscope}%
\pgfsys@transformshift{0.937362in}{3.243609in}%
\pgfsys@useobject{currentmarker}{}%
\end{pgfscope}%
\begin{pgfscope}%
\pgfsys@transformshift{0.969538in}{3.307419in}%
\pgfsys@useobject{currentmarker}{}%
\end{pgfscope}%
\begin{pgfscope}%
\pgfsys@transformshift{1.001714in}{3.585829in}%
\pgfsys@useobject{currentmarker}{}%
\end{pgfscope}%
\begin{pgfscope}%
\pgfsys@transformshift{1.033890in}{3.368950in}%
\pgfsys@useobject{currentmarker}{}%
\end{pgfscope}%
\begin{pgfscope}%
\pgfsys@transformshift{1.066067in}{3.561909in}%
\pgfsys@useobject{currentmarker}{}%
\end{pgfscope}%
\begin{pgfscope}%
\pgfsys@transformshift{1.098243in}{3.219498in}%
\pgfsys@useobject{currentmarker}{}%
\end{pgfscope}%
\begin{pgfscope}%
\pgfsys@transformshift{1.130419in}{3.653691in}%
\pgfsys@useobject{currentmarker}{}%
\end{pgfscope}%
\begin{pgfscope}%
\pgfsys@transformshift{1.162595in}{3.158241in}%
\pgfsys@useobject{currentmarker}{}%
\end{pgfscope}%
\begin{pgfscope}%
\pgfsys@transformshift{1.194771in}{3.466740in}%
\pgfsys@useobject{currentmarker}{}%
\end{pgfscope}%
\begin{pgfscope}%
\pgfsys@transformshift{1.226947in}{3.477069in}%
\pgfsys@useobject{currentmarker}{}%
\end{pgfscope}%
\begin{pgfscope}%
\pgfsys@transformshift{1.259123in}{3.506960in}%
\pgfsys@useobject{currentmarker}{}%
\end{pgfscope}%
\begin{pgfscope}%
\pgfsys@transformshift{1.291299in}{3.434237in}%
\pgfsys@useobject{currentmarker}{}%
\end{pgfscope}%
\begin{pgfscope}%
\pgfsys@transformshift{1.323475in}{3.697227in}%
\pgfsys@useobject{currentmarker}{}%
\end{pgfscope}%
\begin{pgfscope}%
\pgfsys@transformshift{1.355651in}{3.207386in}%
\pgfsys@useobject{currentmarker}{}%
\end{pgfscope}%
\begin{pgfscope}%
\pgfsys@transformshift{1.387827in}{3.486173in}%
\pgfsys@useobject{currentmarker}{}%
\end{pgfscope}%
\begin{pgfscope}%
\pgfsys@transformshift{1.420004in}{3.459328in}%
\pgfsys@useobject{currentmarker}{}%
\end{pgfscope}%
\begin{pgfscope}%
\pgfsys@transformshift{1.452180in}{3.495285in}%
\pgfsys@useobject{currentmarker}{}%
\end{pgfscope}%
\begin{pgfscope}%
\pgfsys@transformshift{1.484356in}{3.624555in}%
\pgfsys@useobject{currentmarker}{}%
\end{pgfscope}%
\begin{pgfscope}%
\pgfsys@transformshift{1.516532in}{3.439532in}%
\pgfsys@useobject{currentmarker}{}%
\end{pgfscope}%
\begin{pgfscope}%
\pgfsys@transformshift{1.548708in}{3.558140in}%
\pgfsys@useobject{currentmarker}{}%
\end{pgfscope}%
\begin{pgfscope}%
\pgfsys@transformshift{1.580884in}{3.352151in}%
\pgfsys@useobject{currentmarker}{}%
\end{pgfscope}%
\begin{pgfscope}%
\pgfsys@transformshift{1.613060in}{3.530543in}%
\pgfsys@useobject{currentmarker}{}%
\end{pgfscope}%
\begin{pgfscope}%
\pgfsys@transformshift{1.645236in}{3.330087in}%
\pgfsys@useobject{currentmarker}{}%
\end{pgfscope}%
\begin{pgfscope}%
\pgfsys@transformshift{1.677412in}{3.553880in}%
\pgfsys@useobject{currentmarker}{}%
\end{pgfscope}%
\begin{pgfscope}%
\pgfsys@transformshift{1.709588in}{3.642791in}%
\pgfsys@useobject{currentmarker}{}%
\end{pgfscope}%
\begin{pgfscope}%
\pgfsys@transformshift{1.741764in}{3.352736in}%
\pgfsys@useobject{currentmarker}{}%
\end{pgfscope}%
\begin{pgfscope}%
\pgfsys@transformshift{1.773941in}{3.513599in}%
\pgfsys@useobject{currentmarker}{}%
\end{pgfscope}%
\begin{pgfscope}%
\pgfsys@transformshift{1.806117in}{3.647724in}%
\pgfsys@useobject{currentmarker}{}%
\end{pgfscope}%
\begin{pgfscope}%
\pgfsys@transformshift{1.838293in}{3.507944in}%
\pgfsys@useobject{currentmarker}{}%
\end{pgfscope}%
\begin{pgfscope}%
\pgfsys@transformshift{1.870469in}{3.502924in}%
\pgfsys@useobject{currentmarker}{}%
\end{pgfscope}%
\begin{pgfscope}%
\pgfsys@transformshift{1.902645in}{3.481067in}%
\pgfsys@useobject{currentmarker}{}%
\end{pgfscope}%
\begin{pgfscope}%
\pgfsys@transformshift{1.934821in}{3.661928in}%
\pgfsys@useobject{currentmarker}{}%
\end{pgfscope}%
\begin{pgfscope}%
\pgfsys@transformshift{1.966997in}{3.522268in}%
\pgfsys@useobject{currentmarker}{}%
\end{pgfscope}%
\begin{pgfscope}%
\pgfsys@transformshift{1.999173in}{3.475589in}%
\pgfsys@useobject{currentmarker}{}%
\end{pgfscope}%
\begin{pgfscope}%
\pgfsys@transformshift{2.031349in}{3.553692in}%
\pgfsys@useobject{currentmarker}{}%
\end{pgfscope}%
\begin{pgfscope}%
\pgfsys@transformshift{2.063525in}{3.465510in}%
\pgfsys@useobject{currentmarker}{}%
\end{pgfscope}%
\begin{pgfscope}%
\pgfsys@transformshift{2.095702in}{3.619516in}%
\pgfsys@useobject{currentmarker}{}%
\end{pgfscope}%
\begin{pgfscope}%
\pgfsys@transformshift{2.127878in}{3.642930in}%
\pgfsys@useobject{currentmarker}{}%
\end{pgfscope}%
\begin{pgfscope}%
\pgfsys@transformshift{2.160054in}{3.518213in}%
\pgfsys@useobject{currentmarker}{}%
\end{pgfscope}%
\begin{pgfscope}%
\pgfsys@transformshift{2.192230in}{3.513859in}%
\pgfsys@useobject{currentmarker}{}%
\end{pgfscope}%
\begin{pgfscope}%
\pgfsys@transformshift{2.224406in}{3.716002in}%
\pgfsys@useobject{currentmarker}{}%
\end{pgfscope}%
\begin{pgfscope}%
\pgfsys@transformshift{2.256582in}{3.387104in}%
\pgfsys@useobject{currentmarker}{}%
\end{pgfscope}%
\begin{pgfscope}%
\pgfsys@transformshift{2.288758in}{3.762567in}%
\pgfsys@useobject{currentmarker}{}%
\end{pgfscope}%
\begin{pgfscope}%
\pgfsys@transformshift{2.320934in}{3.386496in}%
\pgfsys@useobject{currentmarker}{}%
\end{pgfscope}%
\begin{pgfscope}%
\pgfsys@transformshift{2.353110in}{3.552073in}%
\pgfsys@useobject{currentmarker}{}%
\end{pgfscope}%
\begin{pgfscope}%
\pgfsys@transformshift{2.385286in}{3.860467in}%
\pgfsys@useobject{currentmarker}{}%
\end{pgfscope}%
\begin{pgfscope}%
\pgfsys@transformshift{2.417462in}{3.344604in}%
\pgfsys@useobject{currentmarker}{}%
\end{pgfscope}%
\begin{pgfscope}%
\pgfsys@transformshift{2.449639in}{3.784309in}%
\pgfsys@useobject{currentmarker}{}%
\end{pgfscope}%
\begin{pgfscope}%
\pgfsys@transformshift{2.481815in}{3.368330in}%
\pgfsys@useobject{currentmarker}{}%
\end{pgfscope}%
\begin{pgfscope}%
\pgfsys@transformshift{2.513991in}{3.796140in}%
\pgfsys@useobject{currentmarker}{}%
\end{pgfscope}%
\begin{pgfscope}%
\pgfsys@transformshift{2.546167in}{3.551744in}%
\pgfsys@useobject{currentmarker}{}%
\end{pgfscope}%
\begin{pgfscope}%
\pgfsys@transformshift{2.578343in}{3.566766in}%
\pgfsys@useobject{currentmarker}{}%
\end{pgfscope}%
\begin{pgfscope}%
\pgfsys@transformshift{2.610519in}{3.596707in}%
\pgfsys@useobject{currentmarker}{}%
\end{pgfscope}%
\begin{pgfscope}%
\pgfsys@transformshift{2.642695in}{3.625147in}%
\pgfsys@useobject{currentmarker}{}%
\end{pgfscope}%
\begin{pgfscope}%
\pgfsys@transformshift{2.674871in}{3.794937in}%
\pgfsys@useobject{currentmarker}{}%
\end{pgfscope}%
\begin{pgfscope}%
\pgfsys@transformshift{2.707047in}{3.501773in}%
\pgfsys@useobject{currentmarker}{}%
\end{pgfscope}%
\begin{pgfscope}%
\pgfsys@transformshift{2.739223in}{3.699441in}%
\pgfsys@useobject{currentmarker}{}%
\end{pgfscope}%
\begin{pgfscope}%
\pgfsys@transformshift{2.771399in}{3.592142in}%
\pgfsys@useobject{currentmarker}{}%
\end{pgfscope}%
\begin{pgfscope}%
\pgfsys@transformshift{2.803576in}{3.712758in}%
\pgfsys@useobject{currentmarker}{}%
\end{pgfscope}%
\begin{pgfscope}%
\pgfsys@transformshift{2.835752in}{3.543507in}%
\pgfsys@useobject{currentmarker}{}%
\end{pgfscope}%
\begin{pgfscope}%
\pgfsys@transformshift{2.867928in}{3.631361in}%
\pgfsys@useobject{currentmarker}{}%
\end{pgfscope}%
\begin{pgfscope}%
\pgfsys@transformshift{2.900104in}{3.668938in}%
\pgfsys@useobject{currentmarker}{}%
\end{pgfscope}%
\begin{pgfscope}%
\pgfsys@transformshift{2.932280in}{3.675804in}%
\pgfsys@useobject{currentmarker}{}%
\end{pgfscope}%
\begin{pgfscope}%
\pgfsys@transformshift{2.964456in}{3.734148in}%
\pgfsys@useobject{currentmarker}{}%
\end{pgfscope}%
\begin{pgfscope}%
\pgfsys@transformshift{2.996632in}{3.527775in}%
\pgfsys@useobject{currentmarker}{}%
\end{pgfscope}%
\begin{pgfscope}%
\pgfsys@transformshift{3.028808in}{3.735169in}%
\pgfsys@useobject{currentmarker}{}%
\end{pgfscope}%
\begin{pgfscope}%
\pgfsys@transformshift{3.060984in}{3.493292in}%
\pgfsys@useobject{currentmarker}{}%
\end{pgfscope}%
\begin{pgfscope}%
\pgfsys@transformshift{3.093160in}{3.979601in}%
\pgfsys@useobject{currentmarker}{}%
\end{pgfscope}%
\begin{pgfscope}%
\pgfsys@transformshift{3.125336in}{3.410811in}%
\pgfsys@useobject{currentmarker}{}%
\end{pgfscope}%
\begin{pgfscope}%
\pgfsys@transformshift{3.157513in}{3.650072in}%
\pgfsys@useobject{currentmarker}{}%
\end{pgfscope}%
\begin{pgfscope}%
\pgfsys@transformshift{3.189689in}{3.819764in}%
\pgfsys@useobject{currentmarker}{}%
\end{pgfscope}%
\begin{pgfscope}%
\pgfsys@transformshift{3.221865in}{3.714648in}%
\pgfsys@useobject{currentmarker}{}%
\end{pgfscope}%
\begin{pgfscope}%
\pgfsys@transformshift{3.254041in}{3.765153in}%
\pgfsys@useobject{currentmarker}{}%
\end{pgfscope}%
\begin{pgfscope}%
\pgfsys@transformshift{3.286217in}{3.554544in}%
\pgfsys@useobject{currentmarker}{}%
\end{pgfscope}%
\begin{pgfscope}%
\pgfsys@transformshift{3.318393in}{3.774754in}%
\pgfsys@useobject{currentmarker}{}%
\end{pgfscope}%
\begin{pgfscope}%
\pgfsys@transformshift{3.350569in}{3.726898in}%
\pgfsys@useobject{currentmarker}{}%
\end{pgfscope}%
\begin{pgfscope}%
\pgfsys@transformshift{3.382745in}{3.752590in}%
\pgfsys@useobject{currentmarker}{}%
\end{pgfscope}%
\begin{pgfscope}%
\pgfsys@transformshift{3.414921in}{3.713307in}%
\pgfsys@useobject{currentmarker}{}%
\end{pgfscope}%
\begin{pgfscope}%
\pgfsys@transformshift{3.447097in}{3.594696in}%
\pgfsys@useobject{currentmarker}{}%
\end{pgfscope}%
\begin{pgfscope}%
\pgfsys@transformshift{3.479273in}{3.767273in}%
\pgfsys@useobject{currentmarker}{}%
\end{pgfscope}%
\begin{pgfscope}%
\pgfsys@transformshift{3.511450in}{3.717260in}%
\pgfsys@useobject{currentmarker}{}%
\end{pgfscope}%
\begin{pgfscope}%
\pgfsys@transformshift{3.543626in}{3.627327in}%
\pgfsys@useobject{currentmarker}{}%
\end{pgfscope}%
\begin{pgfscope}%
\pgfsys@transformshift{3.575802in}{3.906472in}%
\pgfsys@useobject{currentmarker}{}%
\end{pgfscope}%
\begin{pgfscope}%
\pgfsys@transformshift{3.607978in}{3.568559in}%
\pgfsys@useobject{currentmarker}{}%
\end{pgfscope}%
\begin{pgfscope}%
\pgfsys@transformshift{3.640154in}{3.880790in}%
\pgfsys@useobject{currentmarker}{}%
\end{pgfscope}%
\begin{pgfscope}%
\pgfsys@transformshift{3.672330in}{3.590450in}%
\pgfsys@useobject{currentmarker}{}%
\end{pgfscope}%
\begin{pgfscope}%
\pgfsys@transformshift{3.704506in}{3.831023in}%
\pgfsys@useobject{currentmarker}{}%
\end{pgfscope}%
\begin{pgfscope}%
\pgfsys@transformshift{3.736682in}{3.739902in}%
\pgfsys@useobject{currentmarker}{}%
\end{pgfscope}%
\begin{pgfscope}%
\pgfsys@transformshift{3.768858in}{3.827240in}%
\pgfsys@useobject{currentmarker}{}%
\end{pgfscope}%
\begin{pgfscope}%
\pgfsys@transformshift{3.801034in}{3.859144in}%
\pgfsys@useobject{currentmarker}{}%
\end{pgfscope}%
\begin{pgfscope}%
\pgfsys@transformshift{3.833210in}{3.695114in}%
\pgfsys@useobject{currentmarker}{}%
\end{pgfscope}%
\begin{pgfscope}%
\pgfsys@transformshift{3.865387in}{3.621694in}%
\pgfsys@useobject{currentmarker}{}%
\end{pgfscope}%
\begin{pgfscope}%
\pgfsys@transformshift{3.897563in}{3.787704in}%
\pgfsys@useobject{currentmarker}{}%
\end{pgfscope}%
\begin{pgfscope}%
\pgfsys@transformshift{3.929739in}{3.872605in}%
\pgfsys@useobject{currentmarker}{}%
\end{pgfscope}%
\begin{pgfscope}%
\pgfsys@transformshift{3.961915in}{3.929317in}%
\pgfsys@useobject{currentmarker}{}%
\end{pgfscope}%
\begin{pgfscope}%
\pgfsys@transformshift{3.994091in}{3.605088in}%
\pgfsys@useobject{currentmarker}{}%
\end{pgfscope}%
\begin{pgfscope}%
\pgfsys@transformshift{4.026267in}{3.734526in}%
\pgfsys@useobject{currentmarker}{}%
\end{pgfscope}%
\begin{pgfscope}%
\pgfsys@transformshift{4.058443in}{3.806641in}%
\pgfsys@useobject{currentmarker}{}%
\end{pgfscope}%
\begin{pgfscope}%
\pgfsys@transformshift{4.090619in}{3.911617in}%
\pgfsys@useobject{currentmarker}{}%
\end{pgfscope}%
\begin{pgfscope}%
\pgfsys@transformshift{4.122795in}{3.748032in}%
\pgfsys@useobject{currentmarker}{}%
\end{pgfscope}%
\begin{pgfscope}%
\pgfsys@transformshift{4.154971in}{3.797370in}%
\pgfsys@useobject{currentmarker}{}%
\end{pgfscope}%
\begin{pgfscope}%
\pgfsys@transformshift{4.187147in}{3.915544in}%
\pgfsys@useobject{currentmarker}{}%
\end{pgfscope}%
\begin{pgfscope}%
\pgfsys@transformshift{4.219324in}{3.762972in}%
\pgfsys@useobject{currentmarker}{}%
\end{pgfscope}%
\begin{pgfscope}%
\pgfsys@transformshift{4.251500in}{3.742208in}%
\pgfsys@useobject{currentmarker}{}%
\end{pgfscope}%
\begin{pgfscope}%
\pgfsys@transformshift{4.283676in}{3.834542in}%
\pgfsys@useobject{currentmarker}{}%
\end{pgfscope}%
\begin{pgfscope}%
\pgfsys@transformshift{4.315852in}{3.990807in}%
\pgfsys@useobject{currentmarker}{}%
\end{pgfscope}%
\begin{pgfscope}%
\pgfsys@transformshift{4.348028in}{3.739628in}%
\pgfsys@useobject{currentmarker}{}%
\end{pgfscope}%
\begin{pgfscope}%
\pgfsys@transformshift{4.380204in}{3.858644in}%
\pgfsys@useobject{currentmarker}{}%
\end{pgfscope}%
\begin{pgfscope}%
\pgfsys@transformshift{4.412380in}{3.894902in}%
\pgfsys@useobject{currentmarker}{}%
\end{pgfscope}%
\begin{pgfscope}%
\pgfsys@transformshift{4.444556in}{3.695663in}%
\pgfsys@useobject{currentmarker}{}%
\end{pgfscope}%
\begin{pgfscope}%
\pgfsys@transformshift{4.476732in}{3.714138in}%
\pgfsys@useobject{currentmarker}{}%
\end{pgfscope}%
\begin{pgfscope}%
\pgfsys@transformshift{4.508908in}{3.928367in}%
\pgfsys@useobject{currentmarker}{}%
\end{pgfscope}%
\begin{pgfscope}%
\pgfsys@transformshift{4.541084in}{3.871640in}%
\pgfsys@useobject{currentmarker}{}%
\end{pgfscope}%
\begin{pgfscope}%
\pgfsys@transformshift{4.573261in}{3.979715in}%
\pgfsys@useobject{currentmarker}{}%
\end{pgfscope}%
\begin{pgfscope}%
\pgfsys@transformshift{4.605437in}{3.750388in}%
\pgfsys@useobject{currentmarker}{}%
\end{pgfscope}%
\begin{pgfscope}%
\pgfsys@transformshift{4.637613in}{3.803264in}%
\pgfsys@useobject{currentmarker}{}%
\end{pgfscope}%
\begin{pgfscope}%
\pgfsys@transformshift{4.669789in}{3.984095in}%
\pgfsys@useobject{currentmarker}{}%
\end{pgfscope}%
\begin{pgfscope}%
\pgfsys@transformshift{4.701965in}{3.819486in}%
\pgfsys@useobject{currentmarker}{}%
\end{pgfscope}%
\begin{pgfscope}%
\pgfsys@transformshift{4.734141in}{3.801093in}%
\pgfsys@useobject{currentmarker}{}%
\end{pgfscope}%
\begin{pgfscope}%
\pgfsys@transformshift{4.766317in}{4.005044in}%
\pgfsys@useobject{currentmarker}{}%
\end{pgfscope}%
\begin{pgfscope}%
\pgfsys@transformshift{4.798493in}{3.785858in}%
\pgfsys@useobject{currentmarker}{}%
\end{pgfscope}%
\begin{pgfscope}%
\pgfsys@transformshift{4.830669in}{3.898343in}%
\pgfsys@useobject{currentmarker}{}%
\end{pgfscope}%
\begin{pgfscope}%
\pgfsys@transformshift{4.862845in}{3.829054in}%
\pgfsys@useobject{currentmarker}{}%
\end{pgfscope}%
\begin{pgfscope}%
\pgfsys@transformshift{4.895021in}{3.955438in}%
\pgfsys@useobject{currentmarker}{}%
\end{pgfscope}%
\begin{pgfscope}%
\pgfsys@transformshift{4.927198in}{4.021008in}%
\pgfsys@useobject{currentmarker}{}%
\end{pgfscope}%
\begin{pgfscope}%
\pgfsys@transformshift{4.959374in}{3.806093in}%
\pgfsys@useobject{currentmarker}{}%
\end{pgfscope}%
\begin{pgfscope}%
\pgfsys@transformshift{4.991550in}{4.053498in}%
\pgfsys@useobject{currentmarker}{}%
\end{pgfscope}%
\end{pgfscope}%
\begin{pgfscope}%
\pgfsetrectcap%
\pgfsetmiterjoin%
\pgfsetlinewidth{0.803000pt}%
\definecolor{currentstroke}{rgb}{0.000000,0.000000,0.000000}%
\pgfsetstrokecolor{currentstroke}%
\pgfsetdash{}{0pt}%
\pgfpathmoveto{\pgfqpoint{0.725000in}{3.113478in}}%
\pgfpathlineto{\pgfqpoint{0.725000in}{4.098261in}}%
\pgfusepath{stroke}%
\end{pgfscope}%
\begin{pgfscope}%
\pgfsetrectcap%
\pgfsetmiterjoin%
\pgfsetlinewidth{0.803000pt}%
\definecolor{currentstroke}{rgb}{0.000000,0.000000,0.000000}%
\pgfsetstrokecolor{currentstroke}%
\pgfsetdash{}{0pt}%
\pgfpathmoveto{\pgfqpoint{5.220000in}{3.113478in}}%
\pgfpathlineto{\pgfqpoint{5.220000in}{4.098261in}}%
\pgfusepath{stroke}%
\end{pgfscope}%
\begin{pgfscope}%
\pgfsetrectcap%
\pgfsetmiterjoin%
\pgfsetlinewidth{0.803000pt}%
\definecolor{currentstroke}{rgb}{0.000000,0.000000,0.000000}%
\pgfsetstrokecolor{currentstroke}%
\pgfsetdash{}{0pt}%
\pgfpathmoveto{\pgfqpoint{0.725000in}{3.113478in}}%
\pgfpathlineto{\pgfqpoint{5.220000in}{3.113478in}}%
\pgfusepath{stroke}%
\end{pgfscope}%
\begin{pgfscope}%
\pgfsetrectcap%
\pgfsetmiterjoin%
\pgfsetlinewidth{0.803000pt}%
\definecolor{currentstroke}{rgb}{0.000000,0.000000,0.000000}%
\pgfsetstrokecolor{currentstroke}%
\pgfsetdash{}{0pt}%
\pgfpathmoveto{\pgfqpoint{0.725000in}{4.098261in}}%
\pgfpathlineto{\pgfqpoint{5.220000in}{4.098261in}}%
\pgfusepath{stroke}%
\end{pgfscope}%
\begin{pgfscope}%
\pgfsetbuttcap%
\pgfsetmiterjoin%
\definecolor{currentfill}{rgb}{1.000000,1.000000,1.000000}%
\pgfsetfillcolor{currentfill}%
\pgfsetlinewidth{0.000000pt}%
\definecolor{currentstroke}{rgb}{0.000000,0.000000,0.000000}%
\pgfsetstrokecolor{currentstroke}%
\pgfsetstrokeopacity{0.000000}%
\pgfsetdash{}{0pt}%
\pgfpathmoveto{\pgfqpoint{0.725000in}{1.931739in}}%
\pgfpathlineto{\pgfqpoint{5.220000in}{1.931739in}}%
\pgfpathlineto{\pgfqpoint{5.220000in}{2.916522in}}%
\pgfpathlineto{\pgfqpoint{0.725000in}{2.916522in}}%
\pgfpathclose%
\pgfusepath{fill}%
\end{pgfscope}%
\begin{pgfscope}%
\pgfsetbuttcap%
\pgfsetroundjoin%
\definecolor{currentfill}{rgb}{0.000000,0.000000,0.000000}%
\pgfsetfillcolor{currentfill}%
\pgfsetlinewidth{0.803000pt}%
\definecolor{currentstroke}{rgb}{0.000000,0.000000,0.000000}%
\pgfsetstrokecolor{currentstroke}%
\pgfsetdash{}{0pt}%
\pgfsys@defobject{currentmarker}{\pgfqpoint{0.000000in}{-0.048611in}}{\pgfqpoint{0.000000in}{0.000000in}}{%
\pgfpathmoveto{\pgfqpoint{0.000000in}{0.000000in}}%
\pgfpathlineto{\pgfqpoint{0.000000in}{-0.048611in}}%
\pgfusepath{stroke,fill}%
}%
\begin{pgfscope}%
\pgfsys@transformshift{0.929318in}{1.931739in}%
\pgfsys@useobject{currentmarker}{}%
\end{pgfscope}%
\end{pgfscope}%
\begin{pgfscope}%
\pgfsetbuttcap%
\pgfsetroundjoin%
\definecolor{currentfill}{rgb}{0.000000,0.000000,0.000000}%
\pgfsetfillcolor{currentfill}%
\pgfsetlinewidth{0.803000pt}%
\definecolor{currentstroke}{rgb}{0.000000,0.000000,0.000000}%
\pgfsetstrokecolor{currentstroke}%
\pgfsetdash{}{0pt}%
\pgfsys@defobject{currentmarker}{\pgfqpoint{0.000000in}{-0.048611in}}{\pgfqpoint{0.000000in}{0.000000in}}{%
\pgfpathmoveto{\pgfqpoint{0.000000in}{0.000000in}}%
\pgfpathlineto{\pgfqpoint{0.000000in}{-0.048611in}}%
\pgfusepath{stroke,fill}%
}%
\begin{pgfscope}%
\pgfsys@transformshift{1.733720in}{1.931739in}%
\pgfsys@useobject{currentmarker}{}%
\end{pgfscope}%
\end{pgfscope}%
\begin{pgfscope}%
\pgfsetbuttcap%
\pgfsetroundjoin%
\definecolor{currentfill}{rgb}{0.000000,0.000000,0.000000}%
\pgfsetfillcolor{currentfill}%
\pgfsetlinewidth{0.803000pt}%
\definecolor{currentstroke}{rgb}{0.000000,0.000000,0.000000}%
\pgfsetstrokecolor{currentstroke}%
\pgfsetdash{}{0pt}%
\pgfsys@defobject{currentmarker}{\pgfqpoint{0.000000in}{-0.048611in}}{\pgfqpoint{0.000000in}{0.000000in}}{%
\pgfpathmoveto{\pgfqpoint{0.000000in}{0.000000in}}%
\pgfpathlineto{\pgfqpoint{0.000000in}{-0.048611in}}%
\pgfusepath{stroke,fill}%
}%
\begin{pgfscope}%
\pgfsys@transformshift{2.538123in}{1.931739in}%
\pgfsys@useobject{currentmarker}{}%
\end{pgfscope}%
\end{pgfscope}%
\begin{pgfscope}%
\pgfsetbuttcap%
\pgfsetroundjoin%
\definecolor{currentfill}{rgb}{0.000000,0.000000,0.000000}%
\pgfsetfillcolor{currentfill}%
\pgfsetlinewidth{0.803000pt}%
\definecolor{currentstroke}{rgb}{0.000000,0.000000,0.000000}%
\pgfsetstrokecolor{currentstroke}%
\pgfsetdash{}{0pt}%
\pgfsys@defobject{currentmarker}{\pgfqpoint{0.000000in}{-0.048611in}}{\pgfqpoint{0.000000in}{0.000000in}}{%
\pgfpathmoveto{\pgfqpoint{0.000000in}{0.000000in}}%
\pgfpathlineto{\pgfqpoint{0.000000in}{-0.048611in}}%
\pgfusepath{stroke,fill}%
}%
\begin{pgfscope}%
\pgfsys@transformshift{3.342525in}{1.931739in}%
\pgfsys@useobject{currentmarker}{}%
\end{pgfscope}%
\end{pgfscope}%
\begin{pgfscope}%
\pgfsetbuttcap%
\pgfsetroundjoin%
\definecolor{currentfill}{rgb}{0.000000,0.000000,0.000000}%
\pgfsetfillcolor{currentfill}%
\pgfsetlinewidth{0.803000pt}%
\definecolor{currentstroke}{rgb}{0.000000,0.000000,0.000000}%
\pgfsetstrokecolor{currentstroke}%
\pgfsetdash{}{0pt}%
\pgfsys@defobject{currentmarker}{\pgfqpoint{0.000000in}{-0.048611in}}{\pgfqpoint{0.000000in}{0.000000in}}{%
\pgfpathmoveto{\pgfqpoint{0.000000in}{0.000000in}}%
\pgfpathlineto{\pgfqpoint{0.000000in}{-0.048611in}}%
\pgfusepath{stroke,fill}%
}%
\begin{pgfscope}%
\pgfsys@transformshift{4.146927in}{1.931739in}%
\pgfsys@useobject{currentmarker}{}%
\end{pgfscope}%
\end{pgfscope}%
\begin{pgfscope}%
\pgfsetbuttcap%
\pgfsetroundjoin%
\definecolor{currentfill}{rgb}{0.000000,0.000000,0.000000}%
\pgfsetfillcolor{currentfill}%
\pgfsetlinewidth{0.803000pt}%
\definecolor{currentstroke}{rgb}{0.000000,0.000000,0.000000}%
\pgfsetstrokecolor{currentstroke}%
\pgfsetdash{}{0pt}%
\pgfsys@defobject{currentmarker}{\pgfqpoint{0.000000in}{-0.048611in}}{\pgfqpoint{0.000000in}{0.000000in}}{%
\pgfpathmoveto{\pgfqpoint{0.000000in}{0.000000in}}%
\pgfpathlineto{\pgfqpoint{0.000000in}{-0.048611in}}%
\pgfusepath{stroke,fill}%
}%
\begin{pgfscope}%
\pgfsys@transformshift{4.951330in}{1.931739in}%
\pgfsys@useobject{currentmarker}{}%
\end{pgfscope}%
\end{pgfscope}%
\begin{pgfscope}%
\pgfsetbuttcap%
\pgfsetroundjoin%
\definecolor{currentfill}{rgb}{0.000000,0.000000,0.000000}%
\pgfsetfillcolor{currentfill}%
\pgfsetlinewidth{0.803000pt}%
\definecolor{currentstroke}{rgb}{0.000000,0.000000,0.000000}%
\pgfsetstrokecolor{currentstroke}%
\pgfsetdash{}{0pt}%
\pgfsys@defobject{currentmarker}{\pgfqpoint{-0.048611in}{0.000000in}}{\pgfqpoint{0.000000in}{0.000000in}}{%
\pgfpathmoveto{\pgfqpoint{0.000000in}{0.000000in}}%
\pgfpathlineto{\pgfqpoint{-0.048611in}{0.000000in}}%
\pgfusepath{stroke,fill}%
}%
\begin{pgfscope}%
\pgfsys@transformshift{0.725000in}{2.051871in}%
\pgfsys@useobject{currentmarker}{}%
\end{pgfscope}%
\end{pgfscope}%
\begin{pgfscope}%
\definecolor{textcolor}{rgb}{0.000000,0.000000,0.000000}%
\pgfsetstrokecolor{textcolor}%
\pgfsetfillcolor{textcolor}%
\pgftext[x=0.418000in,y=2.013315in,left,base]{\color{textcolor}\rmfamily\fontsize{8.000000}{9.600000}\selectfont 0.00}%
\end{pgfscope}%
\begin{pgfscope}%
\pgfsetbuttcap%
\pgfsetroundjoin%
\definecolor{currentfill}{rgb}{0.000000,0.000000,0.000000}%
\pgfsetfillcolor{currentfill}%
\pgfsetlinewidth{0.803000pt}%
\definecolor{currentstroke}{rgb}{0.000000,0.000000,0.000000}%
\pgfsetstrokecolor{currentstroke}%
\pgfsetdash{}{0pt}%
\pgfsys@defobject{currentmarker}{\pgfqpoint{-0.048611in}{0.000000in}}{\pgfqpoint{0.000000in}{0.000000in}}{%
\pgfpathmoveto{\pgfqpoint{0.000000in}{0.000000in}}%
\pgfpathlineto{\pgfqpoint{-0.048611in}{0.000000in}}%
\pgfusepath{stroke,fill}%
}%
\begin{pgfscope}%
\pgfsys@transformshift{0.725000in}{2.298404in}%
\pgfsys@useobject{currentmarker}{}%
\end{pgfscope}%
\end{pgfscope}%
\begin{pgfscope}%
\definecolor{textcolor}{rgb}{0.000000,0.000000,0.000000}%
\pgfsetstrokecolor{textcolor}%
\pgfsetfillcolor{textcolor}%
\pgftext[x=0.418000in,y=2.259848in,left,base]{\color{textcolor}\rmfamily\fontsize{8.000000}{9.600000}\selectfont 0.01}%
\end{pgfscope}%
\begin{pgfscope}%
\pgfsetbuttcap%
\pgfsetroundjoin%
\definecolor{currentfill}{rgb}{0.000000,0.000000,0.000000}%
\pgfsetfillcolor{currentfill}%
\pgfsetlinewidth{0.803000pt}%
\definecolor{currentstroke}{rgb}{0.000000,0.000000,0.000000}%
\pgfsetstrokecolor{currentstroke}%
\pgfsetdash{}{0pt}%
\pgfsys@defobject{currentmarker}{\pgfqpoint{-0.048611in}{0.000000in}}{\pgfqpoint{0.000000in}{0.000000in}}{%
\pgfpathmoveto{\pgfqpoint{0.000000in}{0.000000in}}%
\pgfpathlineto{\pgfqpoint{-0.048611in}{0.000000in}}%
\pgfusepath{stroke,fill}%
}%
\begin{pgfscope}%
\pgfsys@transformshift{0.725000in}{2.544937in}%
\pgfsys@useobject{currentmarker}{}%
\end{pgfscope}%
\end{pgfscope}%
\begin{pgfscope}%
\definecolor{textcolor}{rgb}{0.000000,0.000000,0.000000}%
\pgfsetstrokecolor{textcolor}%
\pgfsetfillcolor{textcolor}%
\pgftext[x=0.418000in,y=2.506381in,left,base]{\color{textcolor}\rmfamily\fontsize{8.000000}{9.600000}\selectfont 0.02}%
\end{pgfscope}%
\begin{pgfscope}%
\pgfsetbuttcap%
\pgfsetroundjoin%
\definecolor{currentfill}{rgb}{0.000000,0.000000,0.000000}%
\pgfsetfillcolor{currentfill}%
\pgfsetlinewidth{0.803000pt}%
\definecolor{currentstroke}{rgb}{0.000000,0.000000,0.000000}%
\pgfsetstrokecolor{currentstroke}%
\pgfsetdash{}{0pt}%
\pgfsys@defobject{currentmarker}{\pgfqpoint{-0.048611in}{0.000000in}}{\pgfqpoint{0.000000in}{0.000000in}}{%
\pgfpathmoveto{\pgfqpoint{0.000000in}{0.000000in}}%
\pgfpathlineto{\pgfqpoint{-0.048611in}{0.000000in}}%
\pgfusepath{stroke,fill}%
}%
\begin{pgfscope}%
\pgfsys@transformshift{0.725000in}{2.791470in}%
\pgfsys@useobject{currentmarker}{}%
\end{pgfscope}%
\end{pgfscope}%
\begin{pgfscope}%
\definecolor{textcolor}{rgb}{0.000000,0.000000,0.000000}%
\pgfsetstrokecolor{textcolor}%
\pgfsetfillcolor{textcolor}%
\pgftext[x=0.418000in,y=2.752915in,left,base]{\color{textcolor}\rmfamily\fontsize{8.000000}{9.600000}\selectfont 0.03}%
\end{pgfscope}%
\begin{pgfscope}%
\pgfpathrectangle{\pgfqpoint{0.725000in}{1.931739in}}{\pgfqpoint{4.495000in}{0.984783in}}%
\pgfusepath{clip}%
\pgfsetrectcap%
\pgfsetroundjoin%
\pgfsetlinewidth{1.505625pt}%
\definecolor{currentstroke}{rgb}{1.000000,0.498039,0.054902}%
\pgfsetstrokecolor{currentstroke}%
\pgfsetdash{}{0pt}%
\pgfpathmoveto{\pgfqpoint{0.953450in}{1.979338in}}%
\pgfpathlineto{\pgfqpoint{0.985626in}{1.979338in}}%
\pgfpathlineto{\pgfqpoint{0.985626in}{2.149358in}}%
\pgfpathlineto{\pgfqpoint{1.049979in}{2.149358in}}%
\pgfpathlineto{\pgfqpoint{1.049979in}{2.060357in}}%
\pgfpathlineto{\pgfqpoint{1.114331in}{2.060357in}}%
\pgfpathlineto{\pgfqpoint{1.114331in}{1.976502in}}%
\pgfpathlineto{\pgfqpoint{1.178683in}{1.976502in}}%
\pgfpathlineto{\pgfqpoint{1.178683in}{2.165615in}}%
\pgfpathlineto{\pgfqpoint{1.243035in}{2.165615in}}%
\pgfpathlineto{\pgfqpoint{1.243035in}{2.221403in}}%
\pgfpathlineto{\pgfqpoint{1.307387in}{2.221403in}}%
\pgfpathlineto{\pgfqpoint{1.307387in}{2.038621in}}%
\pgfpathlineto{\pgfqpoint{1.371739in}{2.038621in}}%
\pgfpathlineto{\pgfqpoint{1.371739in}{2.154935in}}%
\pgfpathlineto{\pgfqpoint{1.436092in}{2.154935in}}%
\pgfpathlineto{\pgfqpoint{1.436092in}{2.264265in}}%
\pgfpathlineto{\pgfqpoint{1.500444in}{2.264265in}}%
\pgfpathlineto{\pgfqpoint{1.500444in}{2.157959in}}%
\pgfpathlineto{\pgfqpoint{1.564796in}{2.157959in}}%
\pgfpathlineto{\pgfqpoint{1.564796in}{2.094536in}}%
\pgfpathlineto{\pgfqpoint{1.629148in}{2.094536in}}%
\pgfpathlineto{\pgfqpoint{1.629148in}{2.224476in}}%
\pgfpathlineto{\pgfqpoint{1.693500in}{2.224476in}}%
\pgfpathlineto{\pgfqpoint{1.693500in}{2.140199in}}%
\pgfpathlineto{\pgfqpoint{1.757853in}{2.140199in}}%
\pgfpathlineto{\pgfqpoint{1.757853in}{2.315523in}}%
\pgfpathlineto{\pgfqpoint{1.822205in}{2.315523in}}%
\pgfpathlineto{\pgfqpoint{1.822205in}{2.191433in}}%
\pgfpathlineto{\pgfqpoint{1.886557in}{2.191433in}}%
\pgfpathlineto{\pgfqpoint{1.886557in}{2.319453in}}%
\pgfpathlineto{\pgfqpoint{1.950909in}{2.319453in}}%
\pgfpathlineto{\pgfqpoint{1.950909in}{2.203867in}}%
\pgfpathlineto{\pgfqpoint{2.015261in}{2.203867in}}%
\pgfpathlineto{\pgfqpoint{2.015261in}{2.233564in}}%
\pgfpathlineto{\pgfqpoint{2.079613in}{2.233564in}}%
\pgfpathlineto{\pgfqpoint{2.079613in}{2.356588in}}%
\pgfpathlineto{\pgfqpoint{2.143966in}{2.356588in}}%
\pgfpathlineto{\pgfqpoint{2.143966in}{2.294314in}}%
\pgfpathlineto{\pgfqpoint{2.208318in}{2.294314in}}%
\pgfpathlineto{\pgfqpoint{2.208318in}{2.290787in}}%
\pgfpathlineto{\pgfqpoint{2.272670in}{2.290787in}}%
\pgfpathlineto{\pgfqpoint{2.272670in}{2.227170in}}%
\pgfpathlineto{\pgfqpoint{2.337022in}{2.227170in}}%
\pgfpathlineto{\pgfqpoint{2.337022in}{2.431328in}}%
\pgfpathlineto{\pgfqpoint{2.401374in}{2.431328in}}%
\pgfpathlineto{\pgfqpoint{2.401374in}{2.301872in}}%
\pgfpathlineto{\pgfqpoint{2.465727in}{2.301872in}}%
\pgfpathlineto{\pgfqpoint{2.465727in}{2.390781in}}%
\pgfpathlineto{\pgfqpoint{2.530079in}{2.390781in}}%
\pgfpathlineto{\pgfqpoint{2.530079in}{2.302038in}}%
\pgfpathlineto{\pgfqpoint{2.594431in}{2.302038in}}%
\pgfpathlineto{\pgfqpoint{2.594431in}{2.440700in}}%
\pgfpathlineto{\pgfqpoint{2.658783in}{2.440700in}}%
\pgfpathlineto{\pgfqpoint{2.658783in}{2.385222in}}%
\pgfpathlineto{\pgfqpoint{2.723135in}{2.385222in}}%
\pgfpathlineto{\pgfqpoint{2.723135in}{2.423189in}}%
\pgfpathlineto{\pgfqpoint{2.787487in}{2.423189in}}%
\pgfpathlineto{\pgfqpoint{2.787487in}{2.359494in}}%
\pgfpathlineto{\pgfqpoint{2.851840in}{2.359494in}}%
\pgfpathlineto{\pgfqpoint{2.851840in}{2.441895in}}%
\pgfpathlineto{\pgfqpoint{2.916192in}{2.441895in}}%
\pgfpathlineto{\pgfqpoint{2.916192in}{2.452232in}}%
\pgfpathlineto{\pgfqpoint{2.980544in}{2.452232in}}%
\pgfpathlineto{\pgfqpoint{2.980544in}{2.382711in}}%
\pgfpathlineto{\pgfqpoint{3.044896in}{2.382711in}}%
\pgfpathlineto{\pgfqpoint{3.044896in}{2.525974in}}%
\pgfpathlineto{\pgfqpoint{3.109248in}{2.525974in}}%
\pgfpathlineto{\pgfqpoint{3.109248in}{2.397856in}}%
\pgfpathlineto{\pgfqpoint{3.173601in}{2.397856in}}%
\pgfpathlineto{\pgfqpoint{3.173601in}{2.584106in}}%
\pgfpathlineto{\pgfqpoint{3.237953in}{2.584106in}}%
\pgfpathlineto{\pgfqpoint{3.237953in}{2.443418in}}%
\pgfpathlineto{\pgfqpoint{3.302305in}{2.443418in}}%
\pgfpathlineto{\pgfqpoint{3.302305in}{2.571366in}}%
\pgfpathlineto{\pgfqpoint{3.366657in}{2.571366in}}%
\pgfpathlineto{\pgfqpoint{3.366657in}{2.491533in}}%
\pgfpathlineto{\pgfqpoint{3.431009in}{2.491533in}}%
\pgfpathlineto{\pgfqpoint{3.431009in}{2.519502in}}%
\pgfpathlineto{\pgfqpoint{3.495361in}{2.519502in}}%
\pgfpathlineto{\pgfqpoint{3.495361in}{2.531784in}}%
\pgfpathlineto{\pgfqpoint{3.559714in}{2.531784in}}%
\pgfpathlineto{\pgfqpoint{3.559714in}{2.549501in}}%
\pgfpathlineto{\pgfqpoint{3.624066in}{2.549501in}}%
\pgfpathlineto{\pgfqpoint{3.624066in}{2.537302in}}%
\pgfpathlineto{\pgfqpoint{3.688418in}{2.537302in}}%
\pgfpathlineto{\pgfqpoint{3.688418in}{2.631818in}}%
\pgfpathlineto{\pgfqpoint{3.752770in}{2.631818in}}%
\pgfpathlineto{\pgfqpoint{3.752770in}{2.671331in}}%
\pgfpathlineto{\pgfqpoint{3.817122in}{2.671331in}}%
\pgfpathlineto{\pgfqpoint{3.817122in}{2.473161in}}%
\pgfpathlineto{\pgfqpoint{3.881475in}{2.473161in}}%
\pgfpathlineto{\pgfqpoint{3.881475in}{2.756691in}}%
\pgfpathlineto{\pgfqpoint{3.945827in}{2.756691in}}%
\pgfpathlineto{\pgfqpoint{3.945827in}{2.530101in}}%
\pgfpathlineto{\pgfqpoint{4.010179in}{2.530101in}}%
\pgfpathlineto{\pgfqpoint{4.010179in}{2.678567in}}%
\pgfpathlineto{\pgfqpoint{4.074531in}{2.678567in}}%
\pgfpathlineto{\pgfqpoint{4.074531in}{2.657621in}}%
\pgfpathlineto{\pgfqpoint{4.138883in}{2.657621in}}%
\pgfpathlineto{\pgfqpoint{4.138883in}{2.729418in}}%
\pgfpathlineto{\pgfqpoint{4.203236in}{2.729418in}}%
\pgfpathlineto{\pgfqpoint{4.203236in}{2.610193in}}%
\pgfpathlineto{\pgfqpoint{4.267588in}{2.610193in}}%
\pgfpathlineto{\pgfqpoint{4.267588in}{2.792730in}}%
\pgfpathlineto{\pgfqpoint{4.331940in}{2.792730in}}%
\pgfpathlineto{\pgfqpoint{4.331940in}{2.714159in}}%
\pgfpathlineto{\pgfqpoint{4.396292in}{2.714159in}}%
\pgfpathlineto{\pgfqpoint{4.396292in}{2.578773in}}%
\pgfpathlineto{\pgfqpoint{4.460644in}{2.578773in}}%
\pgfpathlineto{\pgfqpoint{4.460644in}{2.749095in}}%
\pgfpathlineto{\pgfqpoint{4.524996in}{2.749095in}}%
\pgfpathlineto{\pgfqpoint{4.524996in}{2.802619in}}%
\pgfpathlineto{\pgfqpoint{4.589349in}{2.802619in}}%
\pgfpathlineto{\pgfqpoint{4.589349in}{2.709995in}}%
\pgfpathlineto{\pgfqpoint{4.653701in}{2.709995in}}%
\pgfpathlineto{\pgfqpoint{4.653701in}{2.742026in}}%
\pgfpathlineto{\pgfqpoint{4.718053in}{2.742026in}}%
\pgfpathlineto{\pgfqpoint{4.718053in}{2.808622in}}%
\pgfpathlineto{\pgfqpoint{4.782405in}{2.808622in}}%
\pgfpathlineto{\pgfqpoint{4.782405in}{2.737187in}}%
\pgfpathlineto{\pgfqpoint{4.846757in}{2.737187in}}%
\pgfpathlineto{\pgfqpoint{4.846757in}{2.871759in}}%
\pgfpathlineto{\pgfqpoint{4.911110in}{2.871759in}}%
\pgfpathlineto{\pgfqpoint{4.911110in}{2.843158in}}%
\pgfpathlineto{\pgfqpoint{4.943286in}{2.843158in}}%
\pgfusepath{stroke}%
\end{pgfscope}%
\begin{pgfscope}%
\pgfpathrectangle{\pgfqpoint{0.725000in}{1.931739in}}{\pgfqpoint{4.495000in}{0.984783in}}%
\pgfusepath{clip}%
\pgfsetbuttcap%
\pgfsetroundjoin%
\definecolor{currentfill}{rgb}{1.000000,0.498039,0.054902}%
\pgfsetfillcolor{currentfill}%
\pgfsetlinewidth{1.003750pt}%
\definecolor{currentstroke}{rgb}{1.000000,0.498039,0.054902}%
\pgfsetstrokecolor{currentstroke}%
\pgfsetdash{}{0pt}%
\pgfsys@defobject{currentmarker}{\pgfqpoint{-0.041667in}{-0.041667in}}{\pgfqpoint{0.041667in}{0.041667in}}{%
\pgfpathmoveto{\pgfqpoint{0.000000in}{-0.041667in}}%
\pgfpathcurveto{\pgfqpoint{0.011050in}{-0.041667in}}{\pgfqpoint{0.021649in}{-0.037276in}}{\pgfqpoint{0.029463in}{-0.029463in}}%
\pgfpathcurveto{\pgfqpoint{0.037276in}{-0.021649in}}{\pgfqpoint{0.041667in}{-0.011050in}}{\pgfqpoint{0.041667in}{0.000000in}}%
\pgfpathcurveto{\pgfqpoint{0.041667in}{0.011050in}}{\pgfqpoint{0.037276in}{0.021649in}}{\pgfqpoint{0.029463in}{0.029463in}}%
\pgfpathcurveto{\pgfqpoint{0.021649in}{0.037276in}}{\pgfqpoint{0.011050in}{0.041667in}}{\pgfqpoint{0.000000in}{0.041667in}}%
\pgfpathcurveto{\pgfqpoint{-0.011050in}{0.041667in}}{\pgfqpoint{-0.021649in}{0.037276in}}{\pgfqpoint{-0.029463in}{0.029463in}}%
\pgfpathcurveto{\pgfqpoint{-0.037276in}{0.021649in}}{\pgfqpoint{-0.041667in}{0.011050in}}{\pgfqpoint{-0.041667in}{0.000000in}}%
\pgfpathcurveto{\pgfqpoint{-0.041667in}{-0.011050in}}{\pgfqpoint{-0.037276in}{-0.021649in}}{\pgfqpoint{-0.029463in}{-0.029463in}}%
\pgfpathcurveto{\pgfqpoint{-0.021649in}{-0.037276in}}{\pgfqpoint{-0.011050in}{-0.041667in}}{\pgfqpoint{0.000000in}{-0.041667in}}%
\pgfpathclose%
\pgfusepath{stroke,fill}%
}%
\begin{pgfscope}%
\pgfsys@transformshift{0.953450in}{1.979338in}%
\pgfsys@useobject{currentmarker}{}%
\end{pgfscope}%
\begin{pgfscope}%
\pgfsys@transformshift{1.017802in}{2.149358in}%
\pgfsys@useobject{currentmarker}{}%
\end{pgfscope}%
\begin{pgfscope}%
\pgfsys@transformshift{1.082155in}{2.060357in}%
\pgfsys@useobject{currentmarker}{}%
\end{pgfscope}%
\begin{pgfscope}%
\pgfsys@transformshift{1.146507in}{1.976502in}%
\pgfsys@useobject{currentmarker}{}%
\end{pgfscope}%
\begin{pgfscope}%
\pgfsys@transformshift{1.210859in}{2.165615in}%
\pgfsys@useobject{currentmarker}{}%
\end{pgfscope}%
\begin{pgfscope}%
\pgfsys@transformshift{1.275211in}{2.221403in}%
\pgfsys@useobject{currentmarker}{}%
\end{pgfscope}%
\begin{pgfscope}%
\pgfsys@transformshift{1.339563in}{2.038621in}%
\pgfsys@useobject{currentmarker}{}%
\end{pgfscope}%
\begin{pgfscope}%
\pgfsys@transformshift{1.403916in}{2.154935in}%
\pgfsys@useobject{currentmarker}{}%
\end{pgfscope}%
\begin{pgfscope}%
\pgfsys@transformshift{1.468268in}{2.264265in}%
\pgfsys@useobject{currentmarker}{}%
\end{pgfscope}%
\begin{pgfscope}%
\pgfsys@transformshift{1.532620in}{2.157959in}%
\pgfsys@useobject{currentmarker}{}%
\end{pgfscope}%
\begin{pgfscope}%
\pgfsys@transformshift{1.596972in}{2.094536in}%
\pgfsys@useobject{currentmarker}{}%
\end{pgfscope}%
\begin{pgfscope}%
\pgfsys@transformshift{1.661324in}{2.224476in}%
\pgfsys@useobject{currentmarker}{}%
\end{pgfscope}%
\begin{pgfscope}%
\pgfsys@transformshift{1.725676in}{2.140199in}%
\pgfsys@useobject{currentmarker}{}%
\end{pgfscope}%
\begin{pgfscope}%
\pgfsys@transformshift{1.790029in}{2.315523in}%
\pgfsys@useobject{currentmarker}{}%
\end{pgfscope}%
\begin{pgfscope}%
\pgfsys@transformshift{1.854381in}{2.191433in}%
\pgfsys@useobject{currentmarker}{}%
\end{pgfscope}%
\begin{pgfscope}%
\pgfsys@transformshift{1.918733in}{2.319453in}%
\pgfsys@useobject{currentmarker}{}%
\end{pgfscope}%
\begin{pgfscope}%
\pgfsys@transformshift{1.983085in}{2.203867in}%
\pgfsys@useobject{currentmarker}{}%
\end{pgfscope}%
\begin{pgfscope}%
\pgfsys@transformshift{2.047437in}{2.233564in}%
\pgfsys@useobject{currentmarker}{}%
\end{pgfscope}%
\begin{pgfscope}%
\pgfsys@transformshift{2.111790in}{2.356588in}%
\pgfsys@useobject{currentmarker}{}%
\end{pgfscope}%
\begin{pgfscope}%
\pgfsys@transformshift{2.176142in}{2.294314in}%
\pgfsys@useobject{currentmarker}{}%
\end{pgfscope}%
\begin{pgfscope}%
\pgfsys@transformshift{2.240494in}{2.290787in}%
\pgfsys@useobject{currentmarker}{}%
\end{pgfscope}%
\begin{pgfscope}%
\pgfsys@transformshift{2.304846in}{2.227170in}%
\pgfsys@useobject{currentmarker}{}%
\end{pgfscope}%
\begin{pgfscope}%
\pgfsys@transformshift{2.369198in}{2.431328in}%
\pgfsys@useobject{currentmarker}{}%
\end{pgfscope}%
\begin{pgfscope}%
\pgfsys@transformshift{2.433550in}{2.301872in}%
\pgfsys@useobject{currentmarker}{}%
\end{pgfscope}%
\begin{pgfscope}%
\pgfsys@transformshift{2.497903in}{2.390781in}%
\pgfsys@useobject{currentmarker}{}%
\end{pgfscope}%
\begin{pgfscope}%
\pgfsys@transformshift{2.562255in}{2.302038in}%
\pgfsys@useobject{currentmarker}{}%
\end{pgfscope}%
\begin{pgfscope}%
\pgfsys@transformshift{2.626607in}{2.440700in}%
\pgfsys@useobject{currentmarker}{}%
\end{pgfscope}%
\begin{pgfscope}%
\pgfsys@transformshift{2.690959in}{2.385222in}%
\pgfsys@useobject{currentmarker}{}%
\end{pgfscope}%
\begin{pgfscope}%
\pgfsys@transformshift{2.755311in}{2.423189in}%
\pgfsys@useobject{currentmarker}{}%
\end{pgfscope}%
\begin{pgfscope}%
\pgfsys@transformshift{2.819664in}{2.359494in}%
\pgfsys@useobject{currentmarker}{}%
\end{pgfscope}%
\begin{pgfscope}%
\pgfsys@transformshift{2.884016in}{2.441895in}%
\pgfsys@useobject{currentmarker}{}%
\end{pgfscope}%
\begin{pgfscope}%
\pgfsys@transformshift{2.948368in}{2.452232in}%
\pgfsys@useobject{currentmarker}{}%
\end{pgfscope}%
\begin{pgfscope}%
\pgfsys@transformshift{3.012720in}{2.382711in}%
\pgfsys@useobject{currentmarker}{}%
\end{pgfscope}%
\begin{pgfscope}%
\pgfsys@transformshift{3.077072in}{2.525974in}%
\pgfsys@useobject{currentmarker}{}%
\end{pgfscope}%
\begin{pgfscope}%
\pgfsys@transformshift{3.141424in}{2.397856in}%
\pgfsys@useobject{currentmarker}{}%
\end{pgfscope}%
\begin{pgfscope}%
\pgfsys@transformshift{3.205777in}{2.584106in}%
\pgfsys@useobject{currentmarker}{}%
\end{pgfscope}%
\begin{pgfscope}%
\pgfsys@transformshift{3.270129in}{2.443418in}%
\pgfsys@useobject{currentmarker}{}%
\end{pgfscope}%
\begin{pgfscope}%
\pgfsys@transformshift{3.334481in}{2.571366in}%
\pgfsys@useobject{currentmarker}{}%
\end{pgfscope}%
\begin{pgfscope}%
\pgfsys@transformshift{3.398833in}{2.491533in}%
\pgfsys@useobject{currentmarker}{}%
\end{pgfscope}%
\begin{pgfscope}%
\pgfsys@transformshift{3.463185in}{2.519502in}%
\pgfsys@useobject{currentmarker}{}%
\end{pgfscope}%
\begin{pgfscope}%
\pgfsys@transformshift{3.527538in}{2.531784in}%
\pgfsys@useobject{currentmarker}{}%
\end{pgfscope}%
\begin{pgfscope}%
\pgfsys@transformshift{3.591890in}{2.549501in}%
\pgfsys@useobject{currentmarker}{}%
\end{pgfscope}%
\begin{pgfscope}%
\pgfsys@transformshift{3.656242in}{2.537302in}%
\pgfsys@useobject{currentmarker}{}%
\end{pgfscope}%
\begin{pgfscope}%
\pgfsys@transformshift{3.720594in}{2.631818in}%
\pgfsys@useobject{currentmarker}{}%
\end{pgfscope}%
\begin{pgfscope}%
\pgfsys@transformshift{3.784946in}{2.671331in}%
\pgfsys@useobject{currentmarker}{}%
\end{pgfscope}%
\begin{pgfscope}%
\pgfsys@transformshift{3.849298in}{2.473161in}%
\pgfsys@useobject{currentmarker}{}%
\end{pgfscope}%
\begin{pgfscope}%
\pgfsys@transformshift{3.913651in}{2.756691in}%
\pgfsys@useobject{currentmarker}{}%
\end{pgfscope}%
\begin{pgfscope}%
\pgfsys@transformshift{3.978003in}{2.530101in}%
\pgfsys@useobject{currentmarker}{}%
\end{pgfscope}%
\begin{pgfscope}%
\pgfsys@transformshift{4.042355in}{2.678567in}%
\pgfsys@useobject{currentmarker}{}%
\end{pgfscope}%
\begin{pgfscope}%
\pgfsys@transformshift{4.106707in}{2.657621in}%
\pgfsys@useobject{currentmarker}{}%
\end{pgfscope}%
\begin{pgfscope}%
\pgfsys@transformshift{4.171059in}{2.729418in}%
\pgfsys@useobject{currentmarker}{}%
\end{pgfscope}%
\begin{pgfscope}%
\pgfsys@transformshift{4.235412in}{2.610193in}%
\pgfsys@useobject{currentmarker}{}%
\end{pgfscope}%
\begin{pgfscope}%
\pgfsys@transformshift{4.299764in}{2.792730in}%
\pgfsys@useobject{currentmarker}{}%
\end{pgfscope}%
\begin{pgfscope}%
\pgfsys@transformshift{4.364116in}{2.714159in}%
\pgfsys@useobject{currentmarker}{}%
\end{pgfscope}%
\begin{pgfscope}%
\pgfsys@transformshift{4.428468in}{2.578773in}%
\pgfsys@useobject{currentmarker}{}%
\end{pgfscope}%
\begin{pgfscope}%
\pgfsys@transformshift{4.492820in}{2.749095in}%
\pgfsys@useobject{currentmarker}{}%
\end{pgfscope}%
\begin{pgfscope}%
\pgfsys@transformshift{4.557173in}{2.802619in}%
\pgfsys@useobject{currentmarker}{}%
\end{pgfscope}%
\begin{pgfscope}%
\pgfsys@transformshift{4.621525in}{2.709995in}%
\pgfsys@useobject{currentmarker}{}%
\end{pgfscope}%
\begin{pgfscope}%
\pgfsys@transformshift{4.685877in}{2.742026in}%
\pgfsys@useobject{currentmarker}{}%
\end{pgfscope}%
\begin{pgfscope}%
\pgfsys@transformshift{4.750229in}{2.808622in}%
\pgfsys@useobject{currentmarker}{}%
\end{pgfscope}%
\begin{pgfscope}%
\pgfsys@transformshift{4.814581in}{2.737187in}%
\pgfsys@useobject{currentmarker}{}%
\end{pgfscope}%
\begin{pgfscope}%
\pgfsys@transformshift{4.878933in}{2.871759in}%
\pgfsys@useobject{currentmarker}{}%
\end{pgfscope}%
\begin{pgfscope}%
\pgfsys@transformshift{4.943286in}{2.843158in}%
\pgfsys@useobject{currentmarker}{}%
\end{pgfscope}%
\end{pgfscope}%
\begin{pgfscope}%
\pgfsetrectcap%
\pgfsetmiterjoin%
\pgfsetlinewidth{0.803000pt}%
\definecolor{currentstroke}{rgb}{0.000000,0.000000,0.000000}%
\pgfsetstrokecolor{currentstroke}%
\pgfsetdash{}{0pt}%
\pgfpathmoveto{\pgfqpoint{0.725000in}{1.931739in}}%
\pgfpathlineto{\pgfqpoint{0.725000in}{2.916522in}}%
\pgfusepath{stroke}%
\end{pgfscope}%
\begin{pgfscope}%
\pgfsetrectcap%
\pgfsetmiterjoin%
\pgfsetlinewidth{0.803000pt}%
\definecolor{currentstroke}{rgb}{0.000000,0.000000,0.000000}%
\pgfsetstrokecolor{currentstroke}%
\pgfsetdash{}{0pt}%
\pgfpathmoveto{\pgfqpoint{5.220000in}{1.931739in}}%
\pgfpathlineto{\pgfqpoint{5.220000in}{2.916522in}}%
\pgfusepath{stroke}%
\end{pgfscope}%
\begin{pgfscope}%
\pgfsetrectcap%
\pgfsetmiterjoin%
\pgfsetlinewidth{0.803000pt}%
\definecolor{currentstroke}{rgb}{0.000000,0.000000,0.000000}%
\pgfsetstrokecolor{currentstroke}%
\pgfsetdash{}{0pt}%
\pgfpathmoveto{\pgfqpoint{0.725000in}{1.931739in}}%
\pgfpathlineto{\pgfqpoint{5.220000in}{1.931739in}}%
\pgfusepath{stroke}%
\end{pgfscope}%
\begin{pgfscope}%
\pgfsetrectcap%
\pgfsetmiterjoin%
\pgfsetlinewidth{0.803000pt}%
\definecolor{currentstroke}{rgb}{0.000000,0.000000,0.000000}%
\pgfsetstrokecolor{currentstroke}%
\pgfsetdash{}{0pt}%
\pgfpathmoveto{\pgfqpoint{0.725000in}{2.916522in}}%
\pgfpathlineto{\pgfqpoint{5.220000in}{2.916522in}}%
\pgfusepath{stroke}%
\end{pgfscope}%
\begin{pgfscope}%
\pgfsetbuttcap%
\pgfsetmiterjoin%
\definecolor{currentfill}{rgb}{1.000000,1.000000,1.000000}%
\pgfsetfillcolor{currentfill}%
\pgfsetlinewidth{0.000000pt}%
\definecolor{currentstroke}{rgb}{0.000000,0.000000,0.000000}%
\pgfsetstrokecolor{currentstroke}%
\pgfsetstrokeopacity{0.000000}%
\pgfsetdash{}{0pt}%
\pgfpathmoveto{\pgfqpoint{0.725000in}{0.750000in}}%
\pgfpathlineto{\pgfqpoint{5.220000in}{0.750000in}}%
\pgfpathlineto{\pgfqpoint{5.220000in}{1.734783in}}%
\pgfpathlineto{\pgfqpoint{0.725000in}{1.734783in}}%
\pgfpathclose%
\pgfusepath{fill}%
\end{pgfscope}%
\begin{pgfscope}%
\pgfsetbuttcap%
\pgfsetroundjoin%
\definecolor{currentfill}{rgb}{0.000000,0.000000,0.000000}%
\pgfsetfillcolor{currentfill}%
\pgfsetlinewidth{0.803000pt}%
\definecolor{currentstroke}{rgb}{0.000000,0.000000,0.000000}%
\pgfsetstrokecolor{currentstroke}%
\pgfsetdash{}{0pt}%
\pgfsys@defobject{currentmarker}{\pgfqpoint{0.000000in}{-0.048611in}}{\pgfqpoint{0.000000in}{0.000000in}}{%
\pgfpathmoveto{\pgfqpoint{0.000000in}{0.000000in}}%
\pgfpathlineto{\pgfqpoint{0.000000in}{-0.048611in}}%
\pgfusepath{stroke,fill}%
}%
\begin{pgfscope}%
\pgfsys@transformshift{0.929318in}{0.750000in}%
\pgfsys@useobject{currentmarker}{}%
\end{pgfscope}%
\end{pgfscope}%
\begin{pgfscope}%
\definecolor{textcolor}{rgb}{0.000000,0.000000,0.000000}%
\pgfsetstrokecolor{textcolor}%
\pgfsetfillcolor{textcolor}%
\pgftext[x=0.929318in,y=0.652778in,,top]{\color{textcolor}\rmfamily\fontsize{8.000000}{9.600000}\selectfont 0}%
\end{pgfscope}%
\begin{pgfscope}%
\pgfsetbuttcap%
\pgfsetroundjoin%
\definecolor{currentfill}{rgb}{0.000000,0.000000,0.000000}%
\pgfsetfillcolor{currentfill}%
\pgfsetlinewidth{0.803000pt}%
\definecolor{currentstroke}{rgb}{0.000000,0.000000,0.000000}%
\pgfsetstrokecolor{currentstroke}%
\pgfsetdash{}{0pt}%
\pgfsys@defobject{currentmarker}{\pgfqpoint{0.000000in}{-0.048611in}}{\pgfqpoint{0.000000in}{0.000000in}}{%
\pgfpathmoveto{\pgfqpoint{0.000000in}{0.000000in}}%
\pgfpathlineto{\pgfqpoint{0.000000in}{-0.048611in}}%
\pgfusepath{stroke,fill}%
}%
\begin{pgfscope}%
\pgfsys@transformshift{1.733720in}{0.750000in}%
\pgfsys@useobject{currentmarker}{}%
\end{pgfscope}%
\end{pgfscope}%
\begin{pgfscope}%
\definecolor{textcolor}{rgb}{0.000000,0.000000,0.000000}%
\pgfsetstrokecolor{textcolor}%
\pgfsetfillcolor{textcolor}%
\pgftext[x=1.733720in,y=0.652778in,,top]{\color{textcolor}\rmfamily\fontsize{8.000000}{9.600000}\selectfont 50}%
\end{pgfscope}%
\begin{pgfscope}%
\pgfsetbuttcap%
\pgfsetroundjoin%
\definecolor{currentfill}{rgb}{0.000000,0.000000,0.000000}%
\pgfsetfillcolor{currentfill}%
\pgfsetlinewidth{0.803000pt}%
\definecolor{currentstroke}{rgb}{0.000000,0.000000,0.000000}%
\pgfsetstrokecolor{currentstroke}%
\pgfsetdash{}{0pt}%
\pgfsys@defobject{currentmarker}{\pgfqpoint{0.000000in}{-0.048611in}}{\pgfqpoint{0.000000in}{0.000000in}}{%
\pgfpathmoveto{\pgfqpoint{0.000000in}{0.000000in}}%
\pgfpathlineto{\pgfqpoint{0.000000in}{-0.048611in}}%
\pgfusepath{stroke,fill}%
}%
\begin{pgfscope}%
\pgfsys@transformshift{2.538123in}{0.750000in}%
\pgfsys@useobject{currentmarker}{}%
\end{pgfscope}%
\end{pgfscope}%
\begin{pgfscope}%
\definecolor{textcolor}{rgb}{0.000000,0.000000,0.000000}%
\pgfsetstrokecolor{textcolor}%
\pgfsetfillcolor{textcolor}%
\pgftext[x=2.538123in,y=0.652778in,,top]{\color{textcolor}\rmfamily\fontsize{8.000000}{9.600000}\selectfont 100}%
\end{pgfscope}%
\begin{pgfscope}%
\pgfsetbuttcap%
\pgfsetroundjoin%
\definecolor{currentfill}{rgb}{0.000000,0.000000,0.000000}%
\pgfsetfillcolor{currentfill}%
\pgfsetlinewidth{0.803000pt}%
\definecolor{currentstroke}{rgb}{0.000000,0.000000,0.000000}%
\pgfsetstrokecolor{currentstroke}%
\pgfsetdash{}{0pt}%
\pgfsys@defobject{currentmarker}{\pgfqpoint{0.000000in}{-0.048611in}}{\pgfqpoint{0.000000in}{0.000000in}}{%
\pgfpathmoveto{\pgfqpoint{0.000000in}{0.000000in}}%
\pgfpathlineto{\pgfqpoint{0.000000in}{-0.048611in}}%
\pgfusepath{stroke,fill}%
}%
\begin{pgfscope}%
\pgfsys@transformshift{3.342525in}{0.750000in}%
\pgfsys@useobject{currentmarker}{}%
\end{pgfscope}%
\end{pgfscope}%
\begin{pgfscope}%
\definecolor{textcolor}{rgb}{0.000000,0.000000,0.000000}%
\pgfsetstrokecolor{textcolor}%
\pgfsetfillcolor{textcolor}%
\pgftext[x=3.342525in,y=0.652778in,,top]{\color{textcolor}\rmfamily\fontsize{8.000000}{9.600000}\selectfont 150}%
\end{pgfscope}%
\begin{pgfscope}%
\pgfsetbuttcap%
\pgfsetroundjoin%
\definecolor{currentfill}{rgb}{0.000000,0.000000,0.000000}%
\pgfsetfillcolor{currentfill}%
\pgfsetlinewidth{0.803000pt}%
\definecolor{currentstroke}{rgb}{0.000000,0.000000,0.000000}%
\pgfsetstrokecolor{currentstroke}%
\pgfsetdash{}{0pt}%
\pgfsys@defobject{currentmarker}{\pgfqpoint{0.000000in}{-0.048611in}}{\pgfqpoint{0.000000in}{0.000000in}}{%
\pgfpathmoveto{\pgfqpoint{0.000000in}{0.000000in}}%
\pgfpathlineto{\pgfqpoint{0.000000in}{-0.048611in}}%
\pgfusepath{stroke,fill}%
}%
\begin{pgfscope}%
\pgfsys@transformshift{4.146927in}{0.750000in}%
\pgfsys@useobject{currentmarker}{}%
\end{pgfscope}%
\end{pgfscope}%
\begin{pgfscope}%
\definecolor{textcolor}{rgb}{0.000000,0.000000,0.000000}%
\pgfsetstrokecolor{textcolor}%
\pgfsetfillcolor{textcolor}%
\pgftext[x=4.146927in,y=0.652778in,,top]{\color{textcolor}\rmfamily\fontsize{8.000000}{9.600000}\selectfont 200}%
\end{pgfscope}%
\begin{pgfscope}%
\pgfsetbuttcap%
\pgfsetroundjoin%
\definecolor{currentfill}{rgb}{0.000000,0.000000,0.000000}%
\pgfsetfillcolor{currentfill}%
\pgfsetlinewidth{0.803000pt}%
\definecolor{currentstroke}{rgb}{0.000000,0.000000,0.000000}%
\pgfsetstrokecolor{currentstroke}%
\pgfsetdash{}{0pt}%
\pgfsys@defobject{currentmarker}{\pgfqpoint{0.000000in}{-0.048611in}}{\pgfqpoint{0.000000in}{0.000000in}}{%
\pgfpathmoveto{\pgfqpoint{0.000000in}{0.000000in}}%
\pgfpathlineto{\pgfqpoint{0.000000in}{-0.048611in}}%
\pgfusepath{stroke,fill}%
}%
\begin{pgfscope}%
\pgfsys@transformshift{4.951330in}{0.750000in}%
\pgfsys@useobject{currentmarker}{}%
\end{pgfscope}%
\end{pgfscope}%
\begin{pgfscope}%
\definecolor{textcolor}{rgb}{0.000000,0.000000,0.000000}%
\pgfsetstrokecolor{textcolor}%
\pgfsetfillcolor{textcolor}%
\pgftext[x=4.951330in,y=0.652778in,,top]{\color{textcolor}\rmfamily\fontsize{8.000000}{9.600000}\selectfont 250}%
\end{pgfscope}%
\begin{pgfscope}%
\pgfsetbuttcap%
\pgfsetroundjoin%
\definecolor{currentfill}{rgb}{0.000000,0.000000,0.000000}%
\pgfsetfillcolor{currentfill}%
\pgfsetlinewidth{0.803000pt}%
\definecolor{currentstroke}{rgb}{0.000000,0.000000,0.000000}%
\pgfsetstrokecolor{currentstroke}%
\pgfsetdash{}{0pt}%
\pgfsys@defobject{currentmarker}{\pgfqpoint{-0.048611in}{0.000000in}}{\pgfqpoint{0.000000in}{0.000000in}}{%
\pgfpathmoveto{\pgfqpoint{0.000000in}{0.000000in}}%
\pgfpathlineto{\pgfqpoint{-0.048611in}{0.000000in}}%
\pgfusepath{stroke,fill}%
}%
\begin{pgfscope}%
\pgfsys@transformshift{0.725000in}{0.802878in}%
\pgfsys@useobject{currentmarker}{}%
\end{pgfscope}%
\end{pgfscope}%
\begin{pgfscope}%
\definecolor{textcolor}{rgb}{0.000000,0.000000,0.000000}%
\pgfsetstrokecolor{textcolor}%
\pgfsetfillcolor{textcolor}%
\pgftext[x=0.418000in,y=0.764322in,left,base]{\color{textcolor}\rmfamily\fontsize{8.000000}{9.600000}\selectfont 0.00}%
\end{pgfscope}%
\begin{pgfscope}%
\pgfsetbuttcap%
\pgfsetroundjoin%
\definecolor{currentfill}{rgb}{0.000000,0.000000,0.000000}%
\pgfsetfillcolor{currentfill}%
\pgfsetlinewidth{0.803000pt}%
\definecolor{currentstroke}{rgb}{0.000000,0.000000,0.000000}%
\pgfsetstrokecolor{currentstroke}%
\pgfsetdash{}{0pt}%
\pgfsys@defobject{currentmarker}{\pgfqpoint{-0.048611in}{0.000000in}}{\pgfqpoint{0.000000in}{0.000000in}}{%
\pgfpathmoveto{\pgfqpoint{0.000000in}{0.000000in}}%
\pgfpathlineto{\pgfqpoint{-0.048611in}{0.000000in}}%
\pgfusepath{stroke,fill}%
}%
\begin{pgfscope}%
\pgfsys@transformshift{0.725000in}{1.083601in}%
\pgfsys@useobject{currentmarker}{}%
\end{pgfscope}%
\end{pgfscope}%
\begin{pgfscope}%
\definecolor{textcolor}{rgb}{0.000000,0.000000,0.000000}%
\pgfsetstrokecolor{textcolor}%
\pgfsetfillcolor{textcolor}%
\pgftext[x=0.418000in,y=1.045045in,left,base]{\color{textcolor}\rmfamily\fontsize{8.000000}{9.600000}\selectfont 0.01}%
\end{pgfscope}%
\begin{pgfscope}%
\pgfsetbuttcap%
\pgfsetroundjoin%
\definecolor{currentfill}{rgb}{0.000000,0.000000,0.000000}%
\pgfsetfillcolor{currentfill}%
\pgfsetlinewidth{0.803000pt}%
\definecolor{currentstroke}{rgb}{0.000000,0.000000,0.000000}%
\pgfsetstrokecolor{currentstroke}%
\pgfsetdash{}{0pt}%
\pgfsys@defobject{currentmarker}{\pgfqpoint{-0.048611in}{0.000000in}}{\pgfqpoint{0.000000in}{0.000000in}}{%
\pgfpathmoveto{\pgfqpoint{0.000000in}{0.000000in}}%
\pgfpathlineto{\pgfqpoint{-0.048611in}{0.000000in}}%
\pgfusepath{stroke,fill}%
}%
\begin{pgfscope}%
\pgfsys@transformshift{0.725000in}{1.364324in}%
\pgfsys@useobject{currentmarker}{}%
\end{pgfscope}%
\end{pgfscope}%
\begin{pgfscope}%
\definecolor{textcolor}{rgb}{0.000000,0.000000,0.000000}%
\pgfsetstrokecolor{textcolor}%
\pgfsetfillcolor{textcolor}%
\pgftext[x=0.418000in,y=1.325768in,left,base]{\color{textcolor}\rmfamily\fontsize{8.000000}{9.600000}\selectfont 0.02}%
\end{pgfscope}%
\begin{pgfscope}%
\pgfsetbuttcap%
\pgfsetroundjoin%
\definecolor{currentfill}{rgb}{0.000000,0.000000,0.000000}%
\pgfsetfillcolor{currentfill}%
\pgfsetlinewidth{0.803000pt}%
\definecolor{currentstroke}{rgb}{0.000000,0.000000,0.000000}%
\pgfsetstrokecolor{currentstroke}%
\pgfsetdash{}{0pt}%
\pgfsys@defobject{currentmarker}{\pgfqpoint{-0.048611in}{0.000000in}}{\pgfqpoint{0.000000in}{0.000000in}}{%
\pgfpathmoveto{\pgfqpoint{0.000000in}{0.000000in}}%
\pgfpathlineto{\pgfqpoint{-0.048611in}{0.000000in}}%
\pgfusepath{stroke,fill}%
}%
\begin{pgfscope}%
\pgfsys@transformshift{0.725000in}{1.645047in}%
\pgfsys@useobject{currentmarker}{}%
\end{pgfscope}%
\end{pgfscope}%
\begin{pgfscope}%
\definecolor{textcolor}{rgb}{0.000000,0.000000,0.000000}%
\pgfsetstrokecolor{textcolor}%
\pgfsetfillcolor{textcolor}%
\pgftext[x=0.418000in,y=1.606492in,left,base]{\color{textcolor}\rmfamily\fontsize{8.000000}{9.600000}\selectfont 0.03}%
\end{pgfscope}%
\begin{pgfscope}%
\pgfpathrectangle{\pgfqpoint{0.725000in}{0.750000in}}{\pgfqpoint{4.495000in}{0.984783in}}%
\pgfusepath{clip}%
\pgfsetrectcap%
\pgfsetroundjoin%
\pgfsetlinewidth{1.505625pt}%
\definecolor{currentstroke}{rgb}{1.000000,0.498039,0.054902}%
\pgfsetstrokecolor{currentstroke}%
\pgfsetdash{}{0pt}%
\pgfpathmoveto{\pgfqpoint{0.985626in}{0.840149in}}%
\pgfpathlineto{\pgfqpoint{1.049979in}{0.840149in}}%
\pgfpathlineto{\pgfqpoint{1.049979in}{0.794763in}}%
\pgfpathlineto{\pgfqpoint{1.178683in}{0.794763in}}%
\pgfpathlineto{\pgfqpoint{1.178683in}{0.928008in}}%
\pgfpathlineto{\pgfqpoint{1.307387in}{0.928008in}}%
\pgfpathlineto{\pgfqpoint{1.307387in}{0.918247in}}%
\pgfpathlineto{\pgfqpoint{1.436092in}{0.918247in}}%
\pgfpathlineto{\pgfqpoint{1.436092in}{0.935886in}}%
\pgfpathlineto{\pgfqpoint{1.564796in}{0.935886in}}%
\pgfpathlineto{\pgfqpoint{1.564796in}{0.938439in}}%
\pgfpathlineto{\pgfqpoint{1.693500in}{0.938439in}}%
\pgfpathlineto{\pgfqpoint{1.693500in}{1.017860in}}%
\pgfpathlineto{\pgfqpoint{1.822205in}{1.017860in}}%
\pgfpathlineto{\pgfqpoint{1.822205in}{1.038222in}}%
\pgfpathlineto{\pgfqpoint{1.950909in}{1.038222in}}%
\pgfpathlineto{\pgfqpoint{1.950909in}{1.036336in}}%
\pgfpathlineto{\pgfqpoint{2.079613in}{1.036336in}}%
\pgfpathlineto{\pgfqpoint{2.079613in}{1.095668in}}%
\pgfpathlineto{\pgfqpoint{2.208318in}{1.095668in}}%
\pgfpathlineto{\pgfqpoint{2.208318in}{1.078716in}}%
\pgfpathlineto{\pgfqpoint{2.337022in}{1.078716in}}%
\pgfpathlineto{\pgfqpoint{2.337022in}{1.149713in}}%
\pgfpathlineto{\pgfqpoint{2.465727in}{1.149713in}}%
\pgfpathlineto{\pgfqpoint{2.465727in}{1.152475in}}%
\pgfpathlineto{\pgfqpoint{2.594431in}{1.152475in}}%
\pgfpathlineto{\pgfqpoint{2.594431in}{1.209060in}}%
\pgfpathlineto{\pgfqpoint{2.723135in}{1.209060in}}%
\pgfpathlineto{\pgfqpoint{2.723135in}{1.194752in}}%
\pgfpathlineto{\pgfqpoint{2.851840in}{1.194752in}}%
\pgfpathlineto{\pgfqpoint{2.851840in}{1.236029in}}%
\pgfpathlineto{\pgfqpoint{2.980544in}{1.236029in}}%
\pgfpathlineto{\pgfqpoint{2.980544in}{1.265477in}}%
\pgfpathlineto{\pgfqpoint{3.109248in}{1.265477in}}%
\pgfpathlineto{\pgfqpoint{3.109248in}{1.315856in}}%
\pgfpathlineto{\pgfqpoint{3.237953in}{1.315856in}}%
\pgfpathlineto{\pgfqpoint{3.237953in}{1.335269in}}%
\pgfpathlineto{\pgfqpoint{3.366657in}{1.335269in}}%
\pgfpathlineto{\pgfqpoint{3.366657in}{1.330896in}}%
\pgfpathlineto{\pgfqpoint{3.495361in}{1.330896in}}%
\pgfpathlineto{\pgfqpoint{3.495361in}{1.361005in}}%
\pgfpathlineto{\pgfqpoint{3.624066in}{1.361005in}}%
\pgfpathlineto{\pgfqpoint{3.624066in}{1.447596in}}%
\pgfpathlineto{\pgfqpoint{3.752770in}{1.447596in}}%
\pgfpathlineto{\pgfqpoint{3.752770in}{1.419720in}}%
\pgfpathlineto{\pgfqpoint{3.881475in}{1.419720in}}%
\pgfpathlineto{\pgfqpoint{3.881475in}{1.454198in}}%
\pgfpathlineto{\pgfqpoint{4.010179in}{1.454198in}}%
\pgfpathlineto{\pgfqpoint{4.010179in}{1.519036in}}%
\pgfpathlineto{\pgfqpoint{4.138883in}{1.519036in}}%
\pgfpathlineto{\pgfqpoint{4.138883in}{1.524532in}}%
\pgfpathlineto{\pgfqpoint{4.267588in}{1.524532in}}%
\pgfpathlineto{\pgfqpoint{4.267588in}{1.540840in}}%
\pgfpathlineto{\pgfqpoint{4.396292in}{1.540840in}}%
\pgfpathlineto{\pgfqpoint{4.396292in}{1.563547in}}%
\pgfpathlineto{\pgfqpoint{4.524996in}{1.563547in}}%
\pgfpathlineto{\pgfqpoint{4.524996in}{1.587758in}}%
\pgfpathlineto{\pgfqpoint{4.653701in}{1.587758in}}%
\pgfpathlineto{\pgfqpoint{4.653701in}{1.625284in}}%
\pgfpathlineto{\pgfqpoint{4.782405in}{1.625284in}}%
\pgfpathlineto{\pgfqpoint{4.782405in}{1.690020in}}%
\pgfpathlineto{\pgfqpoint{4.846757in}{1.690020in}}%
\pgfusepath{stroke}%
\end{pgfscope}%
\begin{pgfscope}%
\pgfpathrectangle{\pgfqpoint{0.725000in}{0.750000in}}{\pgfqpoint{4.495000in}{0.984783in}}%
\pgfusepath{clip}%
\pgfsetbuttcap%
\pgfsetroundjoin%
\definecolor{currentfill}{rgb}{1.000000,0.498039,0.054902}%
\pgfsetfillcolor{currentfill}%
\pgfsetlinewidth{1.003750pt}%
\definecolor{currentstroke}{rgb}{1.000000,0.498039,0.054902}%
\pgfsetstrokecolor{currentstroke}%
\pgfsetdash{}{0pt}%
\pgfsys@defobject{currentmarker}{\pgfqpoint{-0.041667in}{-0.041667in}}{\pgfqpoint{0.041667in}{0.041667in}}{%
\pgfpathmoveto{\pgfqpoint{0.000000in}{-0.041667in}}%
\pgfpathcurveto{\pgfqpoint{0.011050in}{-0.041667in}}{\pgfqpoint{0.021649in}{-0.037276in}}{\pgfqpoint{0.029463in}{-0.029463in}}%
\pgfpathcurveto{\pgfqpoint{0.037276in}{-0.021649in}}{\pgfqpoint{0.041667in}{-0.011050in}}{\pgfqpoint{0.041667in}{0.000000in}}%
\pgfpathcurveto{\pgfqpoint{0.041667in}{0.011050in}}{\pgfqpoint{0.037276in}{0.021649in}}{\pgfqpoint{0.029463in}{0.029463in}}%
\pgfpathcurveto{\pgfqpoint{0.021649in}{0.037276in}}{\pgfqpoint{0.011050in}{0.041667in}}{\pgfqpoint{0.000000in}{0.041667in}}%
\pgfpathcurveto{\pgfqpoint{-0.011050in}{0.041667in}}{\pgfqpoint{-0.021649in}{0.037276in}}{\pgfqpoint{-0.029463in}{0.029463in}}%
\pgfpathcurveto{\pgfqpoint{-0.037276in}{0.021649in}}{\pgfqpoint{-0.041667in}{0.011050in}}{\pgfqpoint{-0.041667in}{0.000000in}}%
\pgfpathcurveto{\pgfqpoint{-0.041667in}{-0.011050in}}{\pgfqpoint{-0.037276in}{-0.021649in}}{\pgfqpoint{-0.029463in}{-0.029463in}}%
\pgfpathcurveto{\pgfqpoint{-0.021649in}{-0.037276in}}{\pgfqpoint{-0.011050in}{-0.041667in}}{\pgfqpoint{0.000000in}{-0.041667in}}%
\pgfpathclose%
\pgfusepath{stroke,fill}%
}%
\begin{pgfscope}%
\pgfsys@transformshift{0.985626in}{0.840149in}%
\pgfsys@useobject{currentmarker}{}%
\end{pgfscope}%
\begin{pgfscope}%
\pgfsys@transformshift{1.114331in}{0.794763in}%
\pgfsys@useobject{currentmarker}{}%
\end{pgfscope}%
\begin{pgfscope}%
\pgfsys@transformshift{1.243035in}{0.928008in}%
\pgfsys@useobject{currentmarker}{}%
\end{pgfscope}%
\begin{pgfscope}%
\pgfsys@transformshift{1.371739in}{0.918247in}%
\pgfsys@useobject{currentmarker}{}%
\end{pgfscope}%
\begin{pgfscope}%
\pgfsys@transformshift{1.500444in}{0.935886in}%
\pgfsys@useobject{currentmarker}{}%
\end{pgfscope}%
\begin{pgfscope}%
\pgfsys@transformshift{1.629148in}{0.938439in}%
\pgfsys@useobject{currentmarker}{}%
\end{pgfscope}%
\begin{pgfscope}%
\pgfsys@transformshift{1.757853in}{1.017860in}%
\pgfsys@useobject{currentmarker}{}%
\end{pgfscope}%
\begin{pgfscope}%
\pgfsys@transformshift{1.886557in}{1.038222in}%
\pgfsys@useobject{currentmarker}{}%
\end{pgfscope}%
\begin{pgfscope}%
\pgfsys@transformshift{2.015261in}{1.036336in}%
\pgfsys@useobject{currentmarker}{}%
\end{pgfscope}%
\begin{pgfscope}%
\pgfsys@transformshift{2.143966in}{1.095668in}%
\pgfsys@useobject{currentmarker}{}%
\end{pgfscope}%
\begin{pgfscope}%
\pgfsys@transformshift{2.272670in}{1.078716in}%
\pgfsys@useobject{currentmarker}{}%
\end{pgfscope}%
\begin{pgfscope}%
\pgfsys@transformshift{2.401374in}{1.149713in}%
\pgfsys@useobject{currentmarker}{}%
\end{pgfscope}%
\begin{pgfscope}%
\pgfsys@transformshift{2.530079in}{1.152475in}%
\pgfsys@useobject{currentmarker}{}%
\end{pgfscope}%
\begin{pgfscope}%
\pgfsys@transformshift{2.658783in}{1.209060in}%
\pgfsys@useobject{currentmarker}{}%
\end{pgfscope}%
\begin{pgfscope}%
\pgfsys@transformshift{2.787487in}{1.194752in}%
\pgfsys@useobject{currentmarker}{}%
\end{pgfscope}%
\begin{pgfscope}%
\pgfsys@transformshift{2.916192in}{1.236029in}%
\pgfsys@useobject{currentmarker}{}%
\end{pgfscope}%
\begin{pgfscope}%
\pgfsys@transformshift{3.044896in}{1.265477in}%
\pgfsys@useobject{currentmarker}{}%
\end{pgfscope}%
\begin{pgfscope}%
\pgfsys@transformshift{3.173601in}{1.315856in}%
\pgfsys@useobject{currentmarker}{}%
\end{pgfscope}%
\begin{pgfscope}%
\pgfsys@transformshift{3.302305in}{1.335269in}%
\pgfsys@useobject{currentmarker}{}%
\end{pgfscope}%
\begin{pgfscope}%
\pgfsys@transformshift{3.431009in}{1.330896in}%
\pgfsys@useobject{currentmarker}{}%
\end{pgfscope}%
\begin{pgfscope}%
\pgfsys@transformshift{3.559714in}{1.361005in}%
\pgfsys@useobject{currentmarker}{}%
\end{pgfscope}%
\begin{pgfscope}%
\pgfsys@transformshift{3.688418in}{1.447596in}%
\pgfsys@useobject{currentmarker}{}%
\end{pgfscope}%
\begin{pgfscope}%
\pgfsys@transformshift{3.817122in}{1.419720in}%
\pgfsys@useobject{currentmarker}{}%
\end{pgfscope}%
\begin{pgfscope}%
\pgfsys@transformshift{3.945827in}{1.454198in}%
\pgfsys@useobject{currentmarker}{}%
\end{pgfscope}%
\begin{pgfscope}%
\pgfsys@transformshift{4.074531in}{1.519036in}%
\pgfsys@useobject{currentmarker}{}%
\end{pgfscope}%
\begin{pgfscope}%
\pgfsys@transformshift{4.203236in}{1.524532in}%
\pgfsys@useobject{currentmarker}{}%
\end{pgfscope}%
\begin{pgfscope}%
\pgfsys@transformshift{4.331940in}{1.540840in}%
\pgfsys@useobject{currentmarker}{}%
\end{pgfscope}%
\begin{pgfscope}%
\pgfsys@transformshift{4.460644in}{1.563547in}%
\pgfsys@useobject{currentmarker}{}%
\end{pgfscope}%
\begin{pgfscope}%
\pgfsys@transformshift{4.589349in}{1.587758in}%
\pgfsys@useobject{currentmarker}{}%
\end{pgfscope}%
\begin{pgfscope}%
\pgfsys@transformshift{4.718053in}{1.625284in}%
\pgfsys@useobject{currentmarker}{}%
\end{pgfscope}%
\begin{pgfscope}%
\pgfsys@transformshift{4.846757in}{1.690020in}%
\pgfsys@useobject{currentmarker}{}%
\end{pgfscope}%
\end{pgfscope}%
\begin{pgfscope}%
\pgfsetrectcap%
\pgfsetmiterjoin%
\pgfsetlinewidth{0.803000pt}%
\definecolor{currentstroke}{rgb}{0.000000,0.000000,0.000000}%
\pgfsetstrokecolor{currentstroke}%
\pgfsetdash{}{0pt}%
\pgfpathmoveto{\pgfqpoint{0.725000in}{0.750000in}}%
\pgfpathlineto{\pgfqpoint{0.725000in}{1.734783in}}%
\pgfusepath{stroke}%
\end{pgfscope}%
\begin{pgfscope}%
\pgfsetrectcap%
\pgfsetmiterjoin%
\pgfsetlinewidth{0.803000pt}%
\definecolor{currentstroke}{rgb}{0.000000,0.000000,0.000000}%
\pgfsetstrokecolor{currentstroke}%
\pgfsetdash{}{0pt}%
\pgfpathmoveto{\pgfqpoint{5.220000in}{0.750000in}}%
\pgfpathlineto{\pgfqpoint{5.220000in}{1.734783in}}%
\pgfusepath{stroke}%
\end{pgfscope}%
\begin{pgfscope}%
\pgfsetrectcap%
\pgfsetmiterjoin%
\pgfsetlinewidth{0.803000pt}%
\definecolor{currentstroke}{rgb}{0.000000,0.000000,0.000000}%
\pgfsetstrokecolor{currentstroke}%
\pgfsetdash{}{0pt}%
\pgfpathmoveto{\pgfqpoint{0.725000in}{0.750000in}}%
\pgfpathlineto{\pgfqpoint{5.220000in}{0.750000in}}%
\pgfusepath{stroke}%
\end{pgfscope}%
\begin{pgfscope}%
\pgfsetrectcap%
\pgfsetmiterjoin%
\pgfsetlinewidth{0.803000pt}%
\definecolor{currentstroke}{rgb}{0.000000,0.000000,0.000000}%
\pgfsetstrokecolor{currentstroke}%
\pgfsetdash{}{0pt}%
\pgfpathmoveto{\pgfqpoint{0.725000in}{1.734783in}}%
\pgfpathlineto{\pgfqpoint{5.220000in}{1.734783in}}%
\pgfusepath{stroke}%
\end{pgfscope}%
\end{pgfpicture}%
\makeatother%
\endgroup%

    \caption{Ableitung nach Mittelwertbildung\label{polynomials:noise:average}}
\end{figure}

Nun haben wir aber mit dem Verfahren mit den Wavelets den Vorteil, dass wir bei
entsprechender Wahl des Wavelets direkt die $A$te Ableitung berechnen können.

In \autoref{polynomials:noise:db2_multi} ist das Signal $x^3 + r$, die
Detailkoeffizienten der verschiedenen Stufen der Multiskalenanalyse mit dem db2
Wavelet und die Approximationskoeffizienten der letzten Stufe abgebildet.

\begin{figure}
    \centering
    %% Creator: Matplotlib, PGF backend
%%
%% To include the figure in your LaTeX document, write
%%   \input{<filename>.pgf}
%%
%% Make sure the required packages are loaded in your preamble
%%   \usepackage{pgf}
%%
%% Figures using additional raster images can only be included by \input if
%% they are in the same directory as the main LaTeX file. For loading figures
%% from other directories you can use the `import` package
%%   \usepackage{import}
%% and then include the figures with
%%   \import{<path to file>}{<filename>.pgf}
%%
%% Matplotlib used the following preamble
%%   \usepackage{fontspec}
%%
\begingroup%
\makeatletter%
\begin{pgfpicture}%
\pgfpathrectangle{\pgfpointorigin}{\pgfqpoint{5.800000in}{6.000000in}}%
\pgfusepath{use as bounding box, clip}%
\begin{pgfscope}%
\pgfsetbuttcap%
\pgfsetmiterjoin%
\definecolor{currentfill}{rgb}{1.000000,1.000000,1.000000}%
\pgfsetfillcolor{currentfill}%
\pgfsetlinewidth{0.000000pt}%
\definecolor{currentstroke}{rgb}{1.000000,1.000000,1.000000}%
\pgfsetstrokecolor{currentstroke}%
\pgfsetdash{}{0pt}%
\pgfpathmoveto{\pgfqpoint{0.000000in}{0.000000in}}%
\pgfpathlineto{\pgfqpoint{5.800000in}{0.000000in}}%
\pgfpathlineto{\pgfqpoint{5.800000in}{6.000000in}}%
\pgfpathlineto{\pgfqpoint{0.000000in}{6.000000in}}%
\pgfpathclose%
\pgfusepath{fill}%
\end{pgfscope}%
\begin{pgfscope}%
\pgfsetbuttcap%
\pgfsetmiterjoin%
\definecolor{currentfill}{rgb}{1.000000,1.000000,1.000000}%
\pgfsetfillcolor{currentfill}%
\pgfsetlinewidth{0.000000pt}%
\definecolor{currentstroke}{rgb}{0.000000,0.000000,0.000000}%
\pgfsetstrokecolor{currentstroke}%
\pgfsetstrokeopacity{0.000000}%
\pgfsetdash{}{0pt}%
\pgfpathmoveto{\pgfqpoint{0.759375in}{5.109574in}}%
\pgfpathlineto{\pgfqpoint{5.601389in}{5.109574in}}%
\pgfpathlineto{\pgfqpoint{5.601389in}{5.801389in}}%
\pgfpathlineto{\pgfqpoint{0.759375in}{5.801389in}}%
\pgfpathclose%
\pgfusepath{fill}%
\end{pgfscope}%
\begin{pgfscope}%
\pgfsetbuttcap%
\pgfsetroundjoin%
\definecolor{currentfill}{rgb}{0.000000,0.000000,0.000000}%
\pgfsetfillcolor{currentfill}%
\pgfsetlinewidth{0.803000pt}%
\definecolor{currentstroke}{rgb}{0.000000,0.000000,0.000000}%
\pgfsetstrokecolor{currentstroke}%
\pgfsetdash{}{0pt}%
\pgfsys@defobject{currentmarker}{\pgfqpoint{0.000000in}{-0.048611in}}{\pgfqpoint{0.000000in}{0.000000in}}{%
\pgfpathmoveto{\pgfqpoint{0.000000in}{0.000000in}}%
\pgfpathlineto{\pgfqpoint{0.000000in}{-0.048611in}}%
\pgfusepath{stroke,fill}%
}%
\begin{pgfscope}%
\pgfsys@transformshift{0.979467in}{5.109574in}%
\pgfsys@useobject{currentmarker}{}%
\end{pgfscope}%
\end{pgfscope}%
\begin{pgfscope}%
\pgfsetbuttcap%
\pgfsetroundjoin%
\definecolor{currentfill}{rgb}{0.000000,0.000000,0.000000}%
\pgfsetfillcolor{currentfill}%
\pgfsetlinewidth{0.803000pt}%
\definecolor{currentstroke}{rgb}{0.000000,0.000000,0.000000}%
\pgfsetstrokecolor{currentstroke}%
\pgfsetdash{}{0pt}%
\pgfsys@defobject{currentmarker}{\pgfqpoint{0.000000in}{-0.048611in}}{\pgfqpoint{0.000000in}{0.000000in}}{%
\pgfpathmoveto{\pgfqpoint{0.000000in}{0.000000in}}%
\pgfpathlineto{\pgfqpoint{0.000000in}{-0.048611in}}%
\pgfusepath{stroke,fill}%
}%
\begin{pgfscope}%
\pgfsys@transformshift{1.529695in}{5.109574in}%
\pgfsys@useobject{currentmarker}{}%
\end{pgfscope}%
\end{pgfscope}%
\begin{pgfscope}%
\pgfsetbuttcap%
\pgfsetroundjoin%
\definecolor{currentfill}{rgb}{0.000000,0.000000,0.000000}%
\pgfsetfillcolor{currentfill}%
\pgfsetlinewidth{0.803000pt}%
\definecolor{currentstroke}{rgb}{0.000000,0.000000,0.000000}%
\pgfsetstrokecolor{currentstroke}%
\pgfsetdash{}{0pt}%
\pgfsys@defobject{currentmarker}{\pgfqpoint{0.000000in}{-0.048611in}}{\pgfqpoint{0.000000in}{0.000000in}}{%
\pgfpathmoveto{\pgfqpoint{0.000000in}{0.000000in}}%
\pgfpathlineto{\pgfqpoint{0.000000in}{-0.048611in}}%
\pgfusepath{stroke,fill}%
}%
\begin{pgfscope}%
\pgfsys@transformshift{2.079924in}{5.109574in}%
\pgfsys@useobject{currentmarker}{}%
\end{pgfscope}%
\end{pgfscope}%
\begin{pgfscope}%
\pgfsetbuttcap%
\pgfsetroundjoin%
\definecolor{currentfill}{rgb}{0.000000,0.000000,0.000000}%
\pgfsetfillcolor{currentfill}%
\pgfsetlinewidth{0.803000pt}%
\definecolor{currentstroke}{rgb}{0.000000,0.000000,0.000000}%
\pgfsetstrokecolor{currentstroke}%
\pgfsetdash{}{0pt}%
\pgfsys@defobject{currentmarker}{\pgfqpoint{0.000000in}{-0.048611in}}{\pgfqpoint{0.000000in}{0.000000in}}{%
\pgfpathmoveto{\pgfqpoint{0.000000in}{0.000000in}}%
\pgfpathlineto{\pgfqpoint{0.000000in}{-0.048611in}}%
\pgfusepath{stroke,fill}%
}%
\begin{pgfscope}%
\pgfsys@transformshift{2.630153in}{5.109574in}%
\pgfsys@useobject{currentmarker}{}%
\end{pgfscope}%
\end{pgfscope}%
\begin{pgfscope}%
\pgfsetbuttcap%
\pgfsetroundjoin%
\definecolor{currentfill}{rgb}{0.000000,0.000000,0.000000}%
\pgfsetfillcolor{currentfill}%
\pgfsetlinewidth{0.803000pt}%
\definecolor{currentstroke}{rgb}{0.000000,0.000000,0.000000}%
\pgfsetstrokecolor{currentstroke}%
\pgfsetdash{}{0pt}%
\pgfsys@defobject{currentmarker}{\pgfqpoint{0.000000in}{-0.048611in}}{\pgfqpoint{0.000000in}{0.000000in}}{%
\pgfpathmoveto{\pgfqpoint{0.000000in}{0.000000in}}%
\pgfpathlineto{\pgfqpoint{0.000000in}{-0.048611in}}%
\pgfusepath{stroke,fill}%
}%
\begin{pgfscope}%
\pgfsys@transformshift{3.180382in}{5.109574in}%
\pgfsys@useobject{currentmarker}{}%
\end{pgfscope}%
\end{pgfscope}%
\begin{pgfscope}%
\pgfsetbuttcap%
\pgfsetroundjoin%
\definecolor{currentfill}{rgb}{0.000000,0.000000,0.000000}%
\pgfsetfillcolor{currentfill}%
\pgfsetlinewidth{0.803000pt}%
\definecolor{currentstroke}{rgb}{0.000000,0.000000,0.000000}%
\pgfsetstrokecolor{currentstroke}%
\pgfsetdash{}{0pt}%
\pgfsys@defobject{currentmarker}{\pgfqpoint{0.000000in}{-0.048611in}}{\pgfqpoint{0.000000in}{0.000000in}}{%
\pgfpathmoveto{\pgfqpoint{0.000000in}{0.000000in}}%
\pgfpathlineto{\pgfqpoint{0.000000in}{-0.048611in}}%
\pgfusepath{stroke,fill}%
}%
\begin{pgfscope}%
\pgfsys@transformshift{3.730611in}{5.109574in}%
\pgfsys@useobject{currentmarker}{}%
\end{pgfscope}%
\end{pgfscope}%
\begin{pgfscope}%
\pgfsetbuttcap%
\pgfsetroundjoin%
\definecolor{currentfill}{rgb}{0.000000,0.000000,0.000000}%
\pgfsetfillcolor{currentfill}%
\pgfsetlinewidth{0.803000pt}%
\definecolor{currentstroke}{rgb}{0.000000,0.000000,0.000000}%
\pgfsetstrokecolor{currentstroke}%
\pgfsetdash{}{0pt}%
\pgfsys@defobject{currentmarker}{\pgfqpoint{0.000000in}{-0.048611in}}{\pgfqpoint{0.000000in}{0.000000in}}{%
\pgfpathmoveto{\pgfqpoint{0.000000in}{0.000000in}}%
\pgfpathlineto{\pgfqpoint{0.000000in}{-0.048611in}}%
\pgfusepath{stroke,fill}%
}%
\begin{pgfscope}%
\pgfsys@transformshift{4.280840in}{5.109574in}%
\pgfsys@useobject{currentmarker}{}%
\end{pgfscope}%
\end{pgfscope}%
\begin{pgfscope}%
\pgfsetbuttcap%
\pgfsetroundjoin%
\definecolor{currentfill}{rgb}{0.000000,0.000000,0.000000}%
\pgfsetfillcolor{currentfill}%
\pgfsetlinewidth{0.803000pt}%
\definecolor{currentstroke}{rgb}{0.000000,0.000000,0.000000}%
\pgfsetstrokecolor{currentstroke}%
\pgfsetdash{}{0pt}%
\pgfsys@defobject{currentmarker}{\pgfqpoint{0.000000in}{-0.048611in}}{\pgfqpoint{0.000000in}{0.000000in}}{%
\pgfpathmoveto{\pgfqpoint{0.000000in}{0.000000in}}%
\pgfpathlineto{\pgfqpoint{0.000000in}{-0.048611in}}%
\pgfusepath{stroke,fill}%
}%
\begin{pgfscope}%
\pgfsys@transformshift{4.831068in}{5.109574in}%
\pgfsys@useobject{currentmarker}{}%
\end{pgfscope}%
\end{pgfscope}%
\begin{pgfscope}%
\pgfsetbuttcap%
\pgfsetroundjoin%
\definecolor{currentfill}{rgb}{0.000000,0.000000,0.000000}%
\pgfsetfillcolor{currentfill}%
\pgfsetlinewidth{0.803000pt}%
\definecolor{currentstroke}{rgb}{0.000000,0.000000,0.000000}%
\pgfsetstrokecolor{currentstroke}%
\pgfsetdash{}{0pt}%
\pgfsys@defobject{currentmarker}{\pgfqpoint{0.000000in}{-0.048611in}}{\pgfqpoint{0.000000in}{0.000000in}}{%
\pgfpathmoveto{\pgfqpoint{0.000000in}{0.000000in}}%
\pgfpathlineto{\pgfqpoint{0.000000in}{-0.048611in}}%
\pgfusepath{stroke,fill}%
}%
\begin{pgfscope}%
\pgfsys@transformshift{5.381297in}{5.109574in}%
\pgfsys@useobject{currentmarker}{}%
\end{pgfscope}%
\end{pgfscope}%
\begin{pgfscope}%
\pgfsetbuttcap%
\pgfsetroundjoin%
\definecolor{currentfill}{rgb}{0.000000,0.000000,0.000000}%
\pgfsetfillcolor{currentfill}%
\pgfsetlinewidth{0.803000pt}%
\definecolor{currentstroke}{rgb}{0.000000,0.000000,0.000000}%
\pgfsetstrokecolor{currentstroke}%
\pgfsetdash{}{0pt}%
\pgfsys@defobject{currentmarker}{\pgfqpoint{-0.048611in}{0.000000in}}{\pgfqpoint{0.000000in}{0.000000in}}{%
\pgfpathmoveto{\pgfqpoint{0.000000in}{0.000000in}}%
\pgfpathlineto{\pgfqpoint{-0.048611in}{0.000000in}}%
\pgfusepath{stroke,fill}%
}%
\begin{pgfscope}%
\pgfsys@transformshift{0.759375in}{5.140647in}%
\pgfsys@useobject{currentmarker}{}%
\end{pgfscope}%
\end{pgfscope}%
\begin{pgfscope}%
\definecolor{textcolor}{rgb}{0.000000,0.000000,0.000000}%
\pgfsetstrokecolor{textcolor}%
\pgfsetfillcolor{textcolor}%
\pgftext[x=0.592708in,y=5.092453in,left,base]{\color{textcolor}\rmfamily\fontsize{10.000000}{12.000000}\selectfont 0}%
\end{pgfscope}%
\begin{pgfscope}%
\pgfsetbuttcap%
\pgfsetroundjoin%
\definecolor{currentfill}{rgb}{0.000000,0.000000,0.000000}%
\pgfsetfillcolor{currentfill}%
\pgfsetlinewidth{0.803000pt}%
\definecolor{currentstroke}{rgb}{0.000000,0.000000,0.000000}%
\pgfsetstrokecolor{currentstroke}%
\pgfsetdash{}{0pt}%
\pgfsys@defobject{currentmarker}{\pgfqpoint{-0.048611in}{0.000000in}}{\pgfqpoint{0.000000in}{0.000000in}}{%
\pgfpathmoveto{\pgfqpoint{0.000000in}{0.000000in}}%
\pgfpathlineto{\pgfqpoint{-0.048611in}{0.000000in}}%
\pgfusepath{stroke,fill}%
}%
\begin{pgfscope}%
\pgfsys@transformshift{0.759375in}{5.532290in}%
\pgfsys@useobject{currentmarker}{}%
\end{pgfscope}%
\end{pgfscope}%
\begin{pgfscope}%
\definecolor{textcolor}{rgb}{0.000000,0.000000,0.000000}%
\pgfsetstrokecolor{textcolor}%
\pgfsetfillcolor{textcolor}%
\pgftext[x=0.592708in,y=5.484096in,left,base]{\color{textcolor}\rmfamily\fontsize{10.000000}{12.000000}\selectfont 5}%
\end{pgfscope}%
\begin{pgfscope}%
\pgfpathrectangle{\pgfqpoint{0.759375in}{5.109574in}}{\pgfqpoint{4.842014in}{0.691815in}}%
\pgfusepath{clip}%
\pgfsetrectcap%
\pgfsetroundjoin%
\pgfsetlinewidth{1.505625pt}%
\definecolor{currentstroke}{rgb}{0.121569,0.466667,0.705882}%
\pgfsetstrokecolor{currentstroke}%
\pgfsetdash{}{0pt}%
\pgfpathmoveto{\pgfqpoint{0.979467in}{5.142422in}}%
\pgfpathlineto{\pgfqpoint{0.996729in}{5.144214in}}%
\pgfpathlineto{\pgfqpoint{1.048515in}{5.144298in}}%
\pgfpathlineto{\pgfqpoint{1.083039in}{5.141221in}}%
\pgfpathlineto{\pgfqpoint{1.117563in}{5.143454in}}%
\pgfpathlineto{\pgfqpoint{1.152087in}{5.143996in}}%
\pgfpathlineto{\pgfqpoint{1.169349in}{5.141249in}}%
\pgfpathlineto{\pgfqpoint{1.203874in}{5.144238in}}%
\pgfpathlineto{\pgfqpoint{1.238398in}{5.141020in}}%
\pgfpathlineto{\pgfqpoint{1.255660in}{5.143561in}}%
\pgfpathlineto{\pgfqpoint{1.272922in}{5.143989in}}%
\pgfpathlineto{\pgfqpoint{1.290184in}{5.142495in}}%
\pgfpathlineto{\pgfqpoint{1.307446in}{5.143920in}}%
\pgfpathlineto{\pgfqpoint{1.324708in}{5.141772in}}%
\pgfpathlineto{\pgfqpoint{1.376494in}{5.143789in}}%
\pgfpathlineto{\pgfqpoint{1.393756in}{5.141424in}}%
\pgfpathlineto{\pgfqpoint{1.411019in}{5.143990in}}%
\pgfpathlineto{\pgfqpoint{1.428281in}{5.144307in}}%
\pgfpathlineto{\pgfqpoint{1.445543in}{5.142597in}}%
\pgfpathlineto{\pgfqpoint{1.462805in}{5.145206in}}%
\pgfpathlineto{\pgfqpoint{1.480067in}{5.143730in}}%
\pgfpathlineto{\pgfqpoint{1.514591in}{5.145267in}}%
\pgfpathlineto{\pgfqpoint{1.531853in}{5.143180in}}%
\pgfpathlineto{\pgfqpoint{1.549115in}{5.145289in}}%
\pgfpathlineto{\pgfqpoint{1.566377in}{5.142379in}}%
\pgfpathlineto{\pgfqpoint{1.583639in}{5.145744in}}%
\pgfpathlineto{\pgfqpoint{1.600901in}{5.144047in}}%
\pgfpathlineto{\pgfqpoint{1.652688in}{5.146304in}}%
\pgfpathlineto{\pgfqpoint{1.669950in}{5.143287in}}%
\pgfpathlineto{\pgfqpoint{1.721736in}{5.144653in}}%
\pgfpathlineto{\pgfqpoint{1.738998in}{5.147200in}}%
\pgfpathlineto{\pgfqpoint{1.756260in}{5.144716in}}%
\pgfpathlineto{\pgfqpoint{1.808046in}{5.147282in}}%
\pgfpathlineto{\pgfqpoint{1.825309in}{5.145165in}}%
\pgfpathlineto{\pgfqpoint{1.842571in}{5.148780in}}%
\pgfpathlineto{\pgfqpoint{1.859833in}{5.148550in}}%
\pgfpathlineto{\pgfqpoint{1.877095in}{5.146200in}}%
\pgfpathlineto{\pgfqpoint{1.894357in}{5.149275in}}%
\pgfpathlineto{\pgfqpoint{1.911619in}{5.147620in}}%
\pgfpathlineto{\pgfqpoint{1.946143in}{5.147431in}}%
\pgfpathlineto{\pgfqpoint{1.963405in}{5.150569in}}%
\pgfpathlineto{\pgfqpoint{1.980667in}{5.151412in}}%
\pgfpathlineto{\pgfqpoint{1.997929in}{5.148537in}}%
\pgfpathlineto{\pgfqpoint{2.015191in}{5.152635in}}%
\pgfpathlineto{\pgfqpoint{2.032454in}{5.149864in}}%
\pgfpathlineto{\pgfqpoint{2.066978in}{5.150155in}}%
\pgfpathlineto{\pgfqpoint{2.084240in}{5.154173in}}%
\pgfpathlineto{\pgfqpoint{2.101502in}{5.151158in}}%
\pgfpathlineto{\pgfqpoint{2.153288in}{5.155797in}}%
\pgfpathlineto{\pgfqpoint{2.187812in}{5.155053in}}%
\pgfpathlineto{\pgfqpoint{2.205074in}{5.157322in}}%
\pgfpathlineto{\pgfqpoint{2.222336in}{5.154955in}}%
\pgfpathlineto{\pgfqpoint{2.239598in}{5.155695in}}%
\pgfpathlineto{\pgfqpoint{2.256861in}{5.159326in}}%
\pgfpathlineto{\pgfqpoint{2.291385in}{5.157922in}}%
\pgfpathlineto{\pgfqpoint{2.325909in}{5.158684in}}%
\pgfpathlineto{\pgfqpoint{2.377695in}{5.162466in}}%
\pgfpathlineto{\pgfqpoint{2.394957in}{5.164915in}}%
\pgfpathlineto{\pgfqpoint{2.412219in}{5.165490in}}%
\pgfpathlineto{\pgfqpoint{2.429481in}{5.163595in}}%
\pgfpathlineto{\pgfqpoint{2.446743in}{5.166483in}}%
\pgfpathlineto{\pgfqpoint{2.464006in}{5.165517in}}%
\pgfpathlineto{\pgfqpoint{2.515792in}{5.168109in}}%
\pgfpathlineto{\pgfqpoint{2.533054in}{5.171343in}}%
\pgfpathlineto{\pgfqpoint{2.567578in}{5.170322in}}%
\pgfpathlineto{\pgfqpoint{2.584840in}{5.173737in}}%
\pgfpathlineto{\pgfqpoint{2.602102in}{5.172999in}}%
\pgfpathlineto{\pgfqpoint{2.619364in}{5.174062in}}%
\pgfpathlineto{\pgfqpoint{2.636626in}{5.177512in}}%
\pgfpathlineto{\pgfqpoint{2.653888in}{5.175196in}}%
\pgfpathlineto{\pgfqpoint{2.688413in}{5.180176in}}%
\pgfpathlineto{\pgfqpoint{2.705675in}{5.179084in}}%
\pgfpathlineto{\pgfqpoint{2.722937in}{5.181955in}}%
\pgfpathlineto{\pgfqpoint{2.757461in}{5.183513in}}%
\pgfpathlineto{\pgfqpoint{2.826509in}{5.190784in}}%
\pgfpathlineto{\pgfqpoint{2.843771in}{5.189891in}}%
\pgfpathlineto{\pgfqpoint{2.861033in}{5.192286in}}%
\pgfpathlineto{\pgfqpoint{2.878296in}{5.193012in}}%
\pgfpathlineto{\pgfqpoint{2.912820in}{5.197638in}}%
\pgfpathlineto{\pgfqpoint{2.964606in}{5.198246in}}%
\pgfpathlineto{\pgfqpoint{2.981868in}{5.202095in}}%
\pgfpathlineto{\pgfqpoint{3.016392in}{5.205331in}}%
\pgfpathlineto{\pgfqpoint{3.050916in}{5.209606in}}%
\pgfpathlineto{\pgfqpoint{3.085440in}{5.211078in}}%
\pgfpathlineto{\pgfqpoint{3.102703in}{5.212501in}}%
\pgfpathlineto{\pgfqpoint{3.119965in}{5.215742in}}%
\pgfpathlineto{\pgfqpoint{3.137227in}{5.216482in}}%
\pgfpathlineto{\pgfqpoint{3.154489in}{5.220070in}}%
\pgfpathlineto{\pgfqpoint{3.171751in}{5.218900in}}%
\pgfpathlineto{\pgfqpoint{3.206275in}{5.222752in}}%
\pgfpathlineto{\pgfqpoint{3.223537in}{5.226971in}}%
\pgfpathlineto{\pgfqpoint{3.240799in}{5.229487in}}%
\pgfpathlineto{\pgfqpoint{3.275323in}{5.230586in}}%
\pgfpathlineto{\pgfqpoint{3.292585in}{5.235242in}}%
\pgfpathlineto{\pgfqpoint{3.327110in}{5.236301in}}%
\pgfpathlineto{\pgfqpoint{3.344372in}{5.240786in}}%
\pgfpathlineto{\pgfqpoint{3.378896in}{5.245871in}}%
\pgfpathlineto{\pgfqpoint{3.396158in}{5.244798in}}%
\pgfpathlineto{\pgfqpoint{3.413420in}{5.250360in}}%
\pgfpathlineto{\pgfqpoint{3.499730in}{5.258557in}}%
\pgfpathlineto{\pgfqpoint{3.534255in}{5.266406in}}%
\pgfpathlineto{\pgfqpoint{3.551517in}{5.266870in}}%
\pgfpathlineto{\pgfqpoint{3.568779in}{5.271540in}}%
\pgfpathlineto{\pgfqpoint{3.586041in}{5.270980in}}%
\pgfpathlineto{\pgfqpoint{3.603303in}{5.273718in}}%
\pgfpathlineto{\pgfqpoint{3.637827in}{5.282221in}}%
\pgfpathlineto{\pgfqpoint{3.655089in}{5.281895in}}%
\pgfpathlineto{\pgfqpoint{3.689613in}{5.290696in}}%
\pgfpathlineto{\pgfqpoint{3.724138in}{5.292829in}}%
\pgfpathlineto{\pgfqpoint{3.758662in}{5.300862in}}%
\pgfpathlineto{\pgfqpoint{3.775924in}{5.301318in}}%
\pgfpathlineto{\pgfqpoint{3.810448in}{5.310831in}}%
\pgfpathlineto{\pgfqpoint{3.827710in}{5.311237in}}%
\pgfpathlineto{\pgfqpoint{3.844972in}{5.317238in}}%
\pgfpathlineto{\pgfqpoint{3.879496in}{5.322884in}}%
\pgfpathlineto{\pgfqpoint{3.896758in}{5.323373in}}%
\pgfpathlineto{\pgfqpoint{3.914020in}{5.329980in}}%
\pgfpathlineto{\pgfqpoint{3.931282in}{5.329823in}}%
\pgfpathlineto{\pgfqpoint{4.000331in}{5.344243in}}%
\pgfpathlineto{\pgfqpoint{4.017593in}{5.346861in}}%
\pgfpathlineto{\pgfqpoint{4.034855in}{5.353639in}}%
\pgfpathlineto{\pgfqpoint{4.052117in}{5.357567in}}%
\pgfpathlineto{\pgfqpoint{4.069379in}{5.359245in}}%
\pgfpathlineto{\pgfqpoint{4.086641in}{5.364589in}}%
\pgfpathlineto{\pgfqpoint{4.103903in}{5.366701in}}%
\pgfpathlineto{\pgfqpoint{4.121165in}{5.371724in}}%
\pgfpathlineto{\pgfqpoint{4.138427in}{5.372278in}}%
\pgfpathlineto{\pgfqpoint{4.155690in}{5.376209in}}%
\pgfpathlineto{\pgfqpoint{4.172952in}{5.382532in}}%
\pgfpathlineto{\pgfqpoint{4.190214in}{5.386993in}}%
\pgfpathlineto{\pgfqpoint{4.224738in}{5.392169in}}%
\pgfpathlineto{\pgfqpoint{4.242000in}{5.397651in}}%
\pgfpathlineto{\pgfqpoint{4.276524in}{5.404108in}}%
\pgfpathlineto{\pgfqpoint{4.293786in}{5.411188in}}%
\pgfpathlineto{\pgfqpoint{4.311048in}{5.416165in}}%
\pgfpathlineto{\pgfqpoint{4.328310in}{5.416727in}}%
\pgfpathlineto{\pgfqpoint{4.345572in}{5.420937in}}%
\pgfpathlineto{\pgfqpoint{4.362835in}{5.429039in}}%
\pgfpathlineto{\pgfqpoint{4.380097in}{5.429932in}}%
\pgfpathlineto{\pgfqpoint{4.431883in}{5.445826in}}%
\pgfpathlineto{\pgfqpoint{4.449145in}{5.448866in}}%
\pgfpathlineto{\pgfqpoint{4.656290in}{5.508408in}}%
\pgfpathlineto{\pgfqpoint{4.673552in}{5.514912in}}%
\pgfpathlineto{\pgfqpoint{4.690814in}{5.516901in}}%
\pgfpathlineto{\pgfqpoint{4.725338in}{5.528558in}}%
\pgfpathlineto{\pgfqpoint{4.777124in}{5.545043in}}%
\pgfpathlineto{\pgfqpoint{4.811649in}{5.556616in}}%
\pgfpathlineto{\pgfqpoint{4.828911in}{5.561593in}}%
\pgfpathlineto{\pgfqpoint{4.863435in}{5.574951in}}%
\pgfpathlineto{\pgfqpoint{4.880697in}{5.578370in}}%
\pgfpathlineto{\pgfqpoint{4.897959in}{5.584807in}}%
\pgfpathlineto{\pgfqpoint{4.932483in}{5.594937in}}%
\pgfpathlineto{\pgfqpoint{4.949745in}{5.603877in}}%
\pgfpathlineto{\pgfqpoint{4.984269in}{5.616268in}}%
\pgfpathlineto{\pgfqpoint{5.001532in}{5.619993in}}%
\pgfpathlineto{\pgfqpoint{5.036056in}{5.633685in}}%
\pgfpathlineto{\pgfqpoint{5.070580in}{5.645269in}}%
\pgfpathlineto{\pgfqpoint{5.087842in}{5.653616in}}%
\pgfpathlineto{\pgfqpoint{5.105104in}{5.657161in}}%
\pgfpathlineto{\pgfqpoint{5.122366in}{5.666256in}}%
\pgfpathlineto{\pgfqpoint{5.174152in}{5.685056in}}%
\pgfpathlineto{\pgfqpoint{5.191414in}{5.692145in}}%
\pgfpathlineto{\pgfqpoint{5.208677in}{5.697511in}}%
\pgfpathlineto{\pgfqpoint{5.243201in}{5.713061in}}%
\pgfpathlineto{\pgfqpoint{5.260463in}{5.717734in}}%
\pgfpathlineto{\pgfqpoint{5.277725in}{5.727620in}}%
\pgfpathlineto{\pgfqpoint{5.312249in}{5.741670in}}%
\pgfpathlineto{\pgfqpoint{5.329511in}{5.746418in}}%
\pgfpathlineto{\pgfqpoint{5.346773in}{5.755724in}}%
\pgfpathlineto{\pgfqpoint{5.381297in}{5.769943in}}%
\pgfpathlineto{\pgfqpoint{5.381297in}{5.769943in}}%
\pgfusepath{stroke}%
\end{pgfscope}%
\begin{pgfscope}%
\pgfpathrectangle{\pgfqpoint{0.759375in}{5.109574in}}{\pgfqpoint{4.842014in}{0.691815in}}%
\pgfusepath{clip}%
\pgfsetbuttcap%
\pgfsetroundjoin%
\definecolor{currentfill}{rgb}{0.121569,0.466667,0.705882}%
\pgfsetfillcolor{currentfill}%
\pgfsetlinewidth{1.003750pt}%
\definecolor{currentstroke}{rgb}{0.121569,0.466667,0.705882}%
\pgfsetstrokecolor{currentstroke}%
\pgfsetdash{}{0pt}%
\pgfsys@defobject{currentmarker}{\pgfqpoint{-0.041667in}{-0.041667in}}{\pgfqpoint{0.041667in}{0.041667in}}{%
\pgfpathmoveto{\pgfqpoint{0.000000in}{-0.041667in}}%
\pgfpathcurveto{\pgfqpoint{0.011050in}{-0.041667in}}{\pgfqpoint{0.021649in}{-0.037276in}}{\pgfqpoint{0.029463in}{-0.029463in}}%
\pgfpathcurveto{\pgfqpoint{0.037276in}{-0.021649in}}{\pgfqpoint{0.041667in}{-0.011050in}}{\pgfqpoint{0.041667in}{0.000000in}}%
\pgfpathcurveto{\pgfqpoint{0.041667in}{0.011050in}}{\pgfqpoint{0.037276in}{0.021649in}}{\pgfqpoint{0.029463in}{0.029463in}}%
\pgfpathcurveto{\pgfqpoint{0.021649in}{0.037276in}}{\pgfqpoint{0.011050in}{0.041667in}}{\pgfqpoint{0.000000in}{0.041667in}}%
\pgfpathcurveto{\pgfqpoint{-0.011050in}{0.041667in}}{\pgfqpoint{-0.021649in}{0.037276in}}{\pgfqpoint{-0.029463in}{0.029463in}}%
\pgfpathcurveto{\pgfqpoint{-0.037276in}{0.021649in}}{\pgfqpoint{-0.041667in}{0.011050in}}{\pgfqpoint{-0.041667in}{0.000000in}}%
\pgfpathcurveto{\pgfqpoint{-0.041667in}{-0.011050in}}{\pgfqpoint{-0.037276in}{-0.021649in}}{\pgfqpoint{-0.029463in}{-0.029463in}}%
\pgfpathcurveto{\pgfqpoint{-0.021649in}{-0.037276in}}{\pgfqpoint{-0.011050in}{-0.041667in}}{\pgfqpoint{0.000000in}{-0.041667in}}%
\pgfpathclose%
\pgfusepath{stroke,fill}%
}%
\begin{pgfscope}%
\pgfsys@transformshift{0.979467in}{5.142422in}%
\pgfsys@useobject{currentmarker}{}%
\end{pgfscope}%
\begin{pgfscope}%
\pgfsys@transformshift{0.996729in}{5.144214in}%
\pgfsys@useobject{currentmarker}{}%
\end{pgfscope}%
\begin{pgfscope}%
\pgfsys@transformshift{1.013991in}{5.144367in}%
\pgfsys@useobject{currentmarker}{}%
\end{pgfscope}%
\begin{pgfscope}%
\pgfsys@transformshift{1.031253in}{5.144278in}%
\pgfsys@useobject{currentmarker}{}%
\end{pgfscope}%
\begin{pgfscope}%
\pgfsys@transformshift{1.048515in}{5.144298in}%
\pgfsys@useobject{currentmarker}{}%
\end{pgfscope}%
\begin{pgfscope}%
\pgfsys@transformshift{1.065777in}{5.142088in}%
\pgfsys@useobject{currentmarker}{}%
\end{pgfscope}%
\begin{pgfscope}%
\pgfsys@transformshift{1.083039in}{5.141221in}%
\pgfsys@useobject{currentmarker}{}%
\end{pgfscope}%
\begin{pgfscope}%
\pgfsys@transformshift{1.100301in}{5.142849in}%
\pgfsys@useobject{currentmarker}{}%
\end{pgfscope}%
\begin{pgfscope}%
\pgfsys@transformshift{1.117563in}{5.143454in}%
\pgfsys@useobject{currentmarker}{}%
\end{pgfscope}%
\begin{pgfscope}%
\pgfsys@transformshift{1.134825in}{5.143771in}%
\pgfsys@useobject{currentmarker}{}%
\end{pgfscope}%
\begin{pgfscope}%
\pgfsys@transformshift{1.152087in}{5.143996in}%
\pgfsys@useobject{currentmarker}{}%
\end{pgfscope}%
\begin{pgfscope}%
\pgfsys@transformshift{1.169349in}{5.141249in}%
\pgfsys@useobject{currentmarker}{}%
\end{pgfscope}%
\begin{pgfscope}%
\pgfsys@transformshift{1.186612in}{5.142577in}%
\pgfsys@useobject{currentmarker}{}%
\end{pgfscope}%
\begin{pgfscope}%
\pgfsys@transformshift{1.203874in}{5.144238in}%
\pgfsys@useobject{currentmarker}{}%
\end{pgfscope}%
\begin{pgfscope}%
\pgfsys@transformshift{1.221136in}{5.143380in}%
\pgfsys@useobject{currentmarker}{}%
\end{pgfscope}%
\begin{pgfscope}%
\pgfsys@transformshift{1.238398in}{5.141020in}%
\pgfsys@useobject{currentmarker}{}%
\end{pgfscope}%
\begin{pgfscope}%
\pgfsys@transformshift{1.255660in}{5.143561in}%
\pgfsys@useobject{currentmarker}{}%
\end{pgfscope}%
\begin{pgfscope}%
\pgfsys@transformshift{1.272922in}{5.143989in}%
\pgfsys@useobject{currentmarker}{}%
\end{pgfscope}%
\begin{pgfscope}%
\pgfsys@transformshift{1.290184in}{5.142495in}%
\pgfsys@useobject{currentmarker}{}%
\end{pgfscope}%
\begin{pgfscope}%
\pgfsys@transformshift{1.307446in}{5.143920in}%
\pgfsys@useobject{currentmarker}{}%
\end{pgfscope}%
\begin{pgfscope}%
\pgfsys@transformshift{1.324708in}{5.141772in}%
\pgfsys@useobject{currentmarker}{}%
\end{pgfscope}%
\begin{pgfscope}%
\pgfsys@transformshift{1.341970in}{5.142172in}%
\pgfsys@useobject{currentmarker}{}%
\end{pgfscope}%
\begin{pgfscope}%
\pgfsys@transformshift{1.359232in}{5.141094in}%
\pgfsys@useobject{currentmarker}{}%
\end{pgfscope}%
\begin{pgfscope}%
\pgfsys@transformshift{1.376494in}{5.143789in}%
\pgfsys@useobject{currentmarker}{}%
\end{pgfscope}%
\begin{pgfscope}%
\pgfsys@transformshift{1.393756in}{5.141424in}%
\pgfsys@useobject{currentmarker}{}%
\end{pgfscope}%
\begin{pgfscope}%
\pgfsys@transformshift{1.411019in}{5.143990in}%
\pgfsys@useobject{currentmarker}{}%
\end{pgfscope}%
\begin{pgfscope}%
\pgfsys@transformshift{1.428281in}{5.144307in}%
\pgfsys@useobject{currentmarker}{}%
\end{pgfscope}%
\begin{pgfscope}%
\pgfsys@transformshift{1.445543in}{5.142597in}%
\pgfsys@useobject{currentmarker}{}%
\end{pgfscope}%
\begin{pgfscope}%
\pgfsys@transformshift{1.462805in}{5.145206in}%
\pgfsys@useobject{currentmarker}{}%
\end{pgfscope}%
\begin{pgfscope}%
\pgfsys@transformshift{1.480067in}{5.143730in}%
\pgfsys@useobject{currentmarker}{}%
\end{pgfscope}%
\begin{pgfscope}%
\pgfsys@transformshift{1.497329in}{5.144354in}%
\pgfsys@useobject{currentmarker}{}%
\end{pgfscope}%
\begin{pgfscope}%
\pgfsys@transformshift{1.514591in}{5.145267in}%
\pgfsys@useobject{currentmarker}{}%
\end{pgfscope}%
\begin{pgfscope}%
\pgfsys@transformshift{1.531853in}{5.143180in}%
\pgfsys@useobject{currentmarker}{}%
\end{pgfscope}%
\begin{pgfscope}%
\pgfsys@transformshift{1.549115in}{5.145289in}%
\pgfsys@useobject{currentmarker}{}%
\end{pgfscope}%
\begin{pgfscope}%
\pgfsys@transformshift{1.566377in}{5.142379in}%
\pgfsys@useobject{currentmarker}{}%
\end{pgfscope}%
\begin{pgfscope}%
\pgfsys@transformshift{1.583639in}{5.145744in}%
\pgfsys@useobject{currentmarker}{}%
\end{pgfscope}%
\begin{pgfscope}%
\pgfsys@transformshift{1.600901in}{5.144047in}%
\pgfsys@useobject{currentmarker}{}%
\end{pgfscope}%
\begin{pgfscope}%
\pgfsys@transformshift{1.618164in}{5.145011in}%
\pgfsys@useobject{currentmarker}{}%
\end{pgfscope}%
\begin{pgfscope}%
\pgfsys@transformshift{1.635426in}{5.144996in}%
\pgfsys@useobject{currentmarker}{}%
\end{pgfscope}%
\begin{pgfscope}%
\pgfsys@transformshift{1.652688in}{5.146304in}%
\pgfsys@useobject{currentmarker}{}%
\end{pgfscope}%
\begin{pgfscope}%
\pgfsys@transformshift{1.669950in}{5.143287in}%
\pgfsys@useobject{currentmarker}{}%
\end{pgfscope}%
\begin{pgfscope}%
\pgfsys@transformshift{1.687212in}{5.143847in}%
\pgfsys@useobject{currentmarker}{}%
\end{pgfscope}%
\begin{pgfscope}%
\pgfsys@transformshift{1.704474in}{5.143699in}%
\pgfsys@useobject{currentmarker}{}%
\end{pgfscope}%
\begin{pgfscope}%
\pgfsys@transformshift{1.721736in}{5.144653in}%
\pgfsys@useobject{currentmarker}{}%
\end{pgfscope}%
\begin{pgfscope}%
\pgfsys@transformshift{1.738998in}{5.147200in}%
\pgfsys@useobject{currentmarker}{}%
\end{pgfscope}%
\begin{pgfscope}%
\pgfsys@transformshift{1.756260in}{5.144716in}%
\pgfsys@useobject{currentmarker}{}%
\end{pgfscope}%
\begin{pgfscope}%
\pgfsys@transformshift{1.773522in}{5.145842in}%
\pgfsys@useobject{currentmarker}{}%
\end{pgfscope}%
\begin{pgfscope}%
\pgfsys@transformshift{1.790784in}{5.146620in}%
\pgfsys@useobject{currentmarker}{}%
\end{pgfscope}%
\begin{pgfscope}%
\pgfsys@transformshift{1.808046in}{5.147282in}%
\pgfsys@useobject{currentmarker}{}%
\end{pgfscope}%
\begin{pgfscope}%
\pgfsys@transformshift{1.825309in}{5.145165in}%
\pgfsys@useobject{currentmarker}{}%
\end{pgfscope}%
\begin{pgfscope}%
\pgfsys@transformshift{1.842571in}{5.148780in}%
\pgfsys@useobject{currentmarker}{}%
\end{pgfscope}%
\begin{pgfscope}%
\pgfsys@transformshift{1.859833in}{5.148550in}%
\pgfsys@useobject{currentmarker}{}%
\end{pgfscope}%
\begin{pgfscope}%
\pgfsys@transformshift{1.877095in}{5.146200in}%
\pgfsys@useobject{currentmarker}{}%
\end{pgfscope}%
\begin{pgfscope}%
\pgfsys@transformshift{1.894357in}{5.149275in}%
\pgfsys@useobject{currentmarker}{}%
\end{pgfscope}%
\begin{pgfscope}%
\pgfsys@transformshift{1.911619in}{5.147620in}%
\pgfsys@useobject{currentmarker}{}%
\end{pgfscope}%
\begin{pgfscope}%
\pgfsys@transformshift{1.928881in}{5.147563in}%
\pgfsys@useobject{currentmarker}{}%
\end{pgfscope}%
\begin{pgfscope}%
\pgfsys@transformshift{1.946143in}{5.147431in}%
\pgfsys@useobject{currentmarker}{}%
\end{pgfscope}%
\begin{pgfscope}%
\pgfsys@transformshift{1.963405in}{5.150569in}%
\pgfsys@useobject{currentmarker}{}%
\end{pgfscope}%
\begin{pgfscope}%
\pgfsys@transformshift{1.980667in}{5.151412in}%
\pgfsys@useobject{currentmarker}{}%
\end{pgfscope}%
\begin{pgfscope}%
\pgfsys@transformshift{1.997929in}{5.148537in}%
\pgfsys@useobject{currentmarker}{}%
\end{pgfscope}%
\begin{pgfscope}%
\pgfsys@transformshift{2.015191in}{5.152635in}%
\pgfsys@useobject{currentmarker}{}%
\end{pgfscope}%
\begin{pgfscope}%
\pgfsys@transformshift{2.032454in}{5.149864in}%
\pgfsys@useobject{currentmarker}{}%
\end{pgfscope}%
\begin{pgfscope}%
\pgfsys@transformshift{2.049716in}{5.150236in}%
\pgfsys@useobject{currentmarker}{}%
\end{pgfscope}%
\begin{pgfscope}%
\pgfsys@transformshift{2.066978in}{5.150155in}%
\pgfsys@useobject{currentmarker}{}%
\end{pgfscope}%
\begin{pgfscope}%
\pgfsys@transformshift{2.084240in}{5.154173in}%
\pgfsys@useobject{currentmarker}{}%
\end{pgfscope}%
\begin{pgfscope}%
\pgfsys@transformshift{2.101502in}{5.151158in}%
\pgfsys@useobject{currentmarker}{}%
\end{pgfscope}%
\begin{pgfscope}%
\pgfsys@transformshift{2.118764in}{5.152625in}%
\pgfsys@useobject{currentmarker}{}%
\end{pgfscope}%
\begin{pgfscope}%
\pgfsys@transformshift{2.136026in}{5.152701in}%
\pgfsys@useobject{currentmarker}{}%
\end{pgfscope}%
\begin{pgfscope}%
\pgfsys@transformshift{2.153288in}{5.155797in}%
\pgfsys@useobject{currentmarker}{}%
\end{pgfscope}%
\begin{pgfscope}%
\pgfsys@transformshift{2.170550in}{5.154692in}%
\pgfsys@useobject{currentmarker}{}%
\end{pgfscope}%
\begin{pgfscope}%
\pgfsys@transformshift{2.187812in}{5.155053in}%
\pgfsys@useobject{currentmarker}{}%
\end{pgfscope}%
\begin{pgfscope}%
\pgfsys@transformshift{2.205074in}{5.157322in}%
\pgfsys@useobject{currentmarker}{}%
\end{pgfscope}%
\begin{pgfscope}%
\pgfsys@transformshift{2.222336in}{5.154955in}%
\pgfsys@useobject{currentmarker}{}%
\end{pgfscope}%
\begin{pgfscope}%
\pgfsys@transformshift{2.239598in}{5.155695in}%
\pgfsys@useobject{currentmarker}{}%
\end{pgfscope}%
\begin{pgfscope}%
\pgfsys@transformshift{2.256861in}{5.159326in}%
\pgfsys@useobject{currentmarker}{}%
\end{pgfscope}%
\begin{pgfscope}%
\pgfsys@transformshift{2.274123in}{5.158530in}%
\pgfsys@useobject{currentmarker}{}%
\end{pgfscope}%
\begin{pgfscope}%
\pgfsys@transformshift{2.291385in}{5.157922in}%
\pgfsys@useobject{currentmarker}{}%
\end{pgfscope}%
\begin{pgfscope}%
\pgfsys@transformshift{2.308647in}{5.159034in}%
\pgfsys@useobject{currentmarker}{}%
\end{pgfscope}%
\begin{pgfscope}%
\pgfsys@transformshift{2.325909in}{5.158684in}%
\pgfsys@useobject{currentmarker}{}%
\end{pgfscope}%
\begin{pgfscope}%
\pgfsys@transformshift{2.343171in}{5.159692in}%
\pgfsys@useobject{currentmarker}{}%
\end{pgfscope}%
\begin{pgfscope}%
\pgfsys@transformshift{2.360433in}{5.161357in}%
\pgfsys@useobject{currentmarker}{}%
\end{pgfscope}%
\begin{pgfscope}%
\pgfsys@transformshift{2.377695in}{5.162466in}%
\pgfsys@useobject{currentmarker}{}%
\end{pgfscope}%
\begin{pgfscope}%
\pgfsys@transformshift{2.394957in}{5.164915in}%
\pgfsys@useobject{currentmarker}{}%
\end{pgfscope}%
\begin{pgfscope}%
\pgfsys@transformshift{2.412219in}{5.165490in}%
\pgfsys@useobject{currentmarker}{}%
\end{pgfscope}%
\begin{pgfscope}%
\pgfsys@transformshift{2.429481in}{5.163595in}%
\pgfsys@useobject{currentmarker}{}%
\end{pgfscope}%
\begin{pgfscope}%
\pgfsys@transformshift{2.446743in}{5.166483in}%
\pgfsys@useobject{currentmarker}{}%
\end{pgfscope}%
\begin{pgfscope}%
\pgfsys@transformshift{2.464006in}{5.165517in}%
\pgfsys@useobject{currentmarker}{}%
\end{pgfscope}%
\begin{pgfscope}%
\pgfsys@transformshift{2.481268in}{5.166438in}%
\pgfsys@useobject{currentmarker}{}%
\end{pgfscope}%
\begin{pgfscope}%
\pgfsys@transformshift{2.498530in}{5.166962in}%
\pgfsys@useobject{currentmarker}{}%
\end{pgfscope}%
\begin{pgfscope}%
\pgfsys@transformshift{2.515792in}{5.168109in}%
\pgfsys@useobject{currentmarker}{}%
\end{pgfscope}%
\begin{pgfscope}%
\pgfsys@transformshift{2.533054in}{5.171343in}%
\pgfsys@useobject{currentmarker}{}%
\end{pgfscope}%
\begin{pgfscope}%
\pgfsys@transformshift{2.550316in}{5.170427in}%
\pgfsys@useobject{currentmarker}{}%
\end{pgfscope}%
\begin{pgfscope}%
\pgfsys@transformshift{2.567578in}{5.170322in}%
\pgfsys@useobject{currentmarker}{}%
\end{pgfscope}%
\begin{pgfscope}%
\pgfsys@transformshift{2.584840in}{5.173737in}%
\pgfsys@useobject{currentmarker}{}%
\end{pgfscope}%
\begin{pgfscope}%
\pgfsys@transformshift{2.602102in}{5.172999in}%
\pgfsys@useobject{currentmarker}{}%
\end{pgfscope}%
\begin{pgfscope}%
\pgfsys@transformshift{2.619364in}{5.174062in}%
\pgfsys@useobject{currentmarker}{}%
\end{pgfscope}%
\begin{pgfscope}%
\pgfsys@transformshift{2.636626in}{5.177512in}%
\pgfsys@useobject{currentmarker}{}%
\end{pgfscope}%
\begin{pgfscope}%
\pgfsys@transformshift{2.653888in}{5.175196in}%
\pgfsys@useobject{currentmarker}{}%
\end{pgfscope}%
\begin{pgfscope}%
\pgfsys@transformshift{2.671151in}{5.177186in}%
\pgfsys@useobject{currentmarker}{}%
\end{pgfscope}%
\begin{pgfscope}%
\pgfsys@transformshift{2.688413in}{5.180176in}%
\pgfsys@useobject{currentmarker}{}%
\end{pgfscope}%
\begin{pgfscope}%
\pgfsys@transformshift{2.705675in}{5.179084in}%
\pgfsys@useobject{currentmarker}{}%
\end{pgfscope}%
\begin{pgfscope}%
\pgfsys@transformshift{2.722937in}{5.181955in}%
\pgfsys@useobject{currentmarker}{}%
\end{pgfscope}%
\begin{pgfscope}%
\pgfsys@transformshift{2.740199in}{5.182397in}%
\pgfsys@useobject{currentmarker}{}%
\end{pgfscope}%
\begin{pgfscope}%
\pgfsys@transformshift{2.757461in}{5.183513in}%
\pgfsys@useobject{currentmarker}{}%
\end{pgfscope}%
\begin{pgfscope}%
\pgfsys@transformshift{2.774723in}{5.185492in}%
\pgfsys@useobject{currentmarker}{}%
\end{pgfscope}%
\begin{pgfscope}%
\pgfsys@transformshift{2.791985in}{5.187923in}%
\pgfsys@useobject{currentmarker}{}%
\end{pgfscope}%
\begin{pgfscope}%
\pgfsys@transformshift{2.809247in}{5.188566in}%
\pgfsys@useobject{currentmarker}{}%
\end{pgfscope}%
\begin{pgfscope}%
\pgfsys@transformshift{2.826509in}{5.190784in}%
\pgfsys@useobject{currentmarker}{}%
\end{pgfscope}%
\begin{pgfscope}%
\pgfsys@transformshift{2.843771in}{5.189891in}%
\pgfsys@useobject{currentmarker}{}%
\end{pgfscope}%
\begin{pgfscope}%
\pgfsys@transformshift{2.861033in}{5.192286in}%
\pgfsys@useobject{currentmarker}{}%
\end{pgfscope}%
\begin{pgfscope}%
\pgfsys@transformshift{2.878296in}{5.193012in}%
\pgfsys@useobject{currentmarker}{}%
\end{pgfscope}%
\begin{pgfscope}%
\pgfsys@transformshift{2.895558in}{5.195772in}%
\pgfsys@useobject{currentmarker}{}%
\end{pgfscope}%
\begin{pgfscope}%
\pgfsys@transformshift{2.912820in}{5.197638in}%
\pgfsys@useobject{currentmarker}{}%
\end{pgfscope}%
\begin{pgfscope}%
\pgfsys@transformshift{2.930082in}{5.197684in}%
\pgfsys@useobject{currentmarker}{}%
\end{pgfscope}%
\begin{pgfscope}%
\pgfsys@transformshift{2.947344in}{5.198612in}%
\pgfsys@useobject{currentmarker}{}%
\end{pgfscope}%
\begin{pgfscope}%
\pgfsys@transformshift{2.964606in}{5.198246in}%
\pgfsys@useobject{currentmarker}{}%
\end{pgfscope}%
\begin{pgfscope}%
\pgfsys@transformshift{2.981868in}{5.202095in}%
\pgfsys@useobject{currentmarker}{}%
\end{pgfscope}%
\begin{pgfscope}%
\pgfsys@transformshift{2.999130in}{5.203188in}%
\pgfsys@useobject{currentmarker}{}%
\end{pgfscope}%
\begin{pgfscope}%
\pgfsys@transformshift{3.016392in}{5.205331in}%
\pgfsys@useobject{currentmarker}{}%
\end{pgfscope}%
\begin{pgfscope}%
\pgfsys@transformshift{3.033654in}{5.208235in}%
\pgfsys@useobject{currentmarker}{}%
\end{pgfscope}%
\begin{pgfscope}%
\pgfsys@transformshift{3.050916in}{5.209606in}%
\pgfsys@useobject{currentmarker}{}%
\end{pgfscope}%
\begin{pgfscope}%
\pgfsys@transformshift{3.068178in}{5.209628in}%
\pgfsys@useobject{currentmarker}{}%
\end{pgfscope}%
\begin{pgfscope}%
\pgfsys@transformshift{3.085440in}{5.211078in}%
\pgfsys@useobject{currentmarker}{}%
\end{pgfscope}%
\begin{pgfscope}%
\pgfsys@transformshift{3.102703in}{5.212501in}%
\pgfsys@useobject{currentmarker}{}%
\end{pgfscope}%
\begin{pgfscope}%
\pgfsys@transformshift{3.119965in}{5.215742in}%
\pgfsys@useobject{currentmarker}{}%
\end{pgfscope}%
\begin{pgfscope}%
\pgfsys@transformshift{3.137227in}{5.216482in}%
\pgfsys@useobject{currentmarker}{}%
\end{pgfscope}%
\begin{pgfscope}%
\pgfsys@transformshift{3.154489in}{5.220070in}%
\pgfsys@useobject{currentmarker}{}%
\end{pgfscope}%
\begin{pgfscope}%
\pgfsys@transformshift{3.171751in}{5.218900in}%
\pgfsys@useobject{currentmarker}{}%
\end{pgfscope}%
\begin{pgfscope}%
\pgfsys@transformshift{3.189013in}{5.220646in}%
\pgfsys@useobject{currentmarker}{}%
\end{pgfscope}%
\begin{pgfscope}%
\pgfsys@transformshift{3.206275in}{5.222752in}%
\pgfsys@useobject{currentmarker}{}%
\end{pgfscope}%
\begin{pgfscope}%
\pgfsys@transformshift{3.223537in}{5.226971in}%
\pgfsys@useobject{currentmarker}{}%
\end{pgfscope}%
\begin{pgfscope}%
\pgfsys@transformshift{3.240799in}{5.229487in}%
\pgfsys@useobject{currentmarker}{}%
\end{pgfscope}%
\begin{pgfscope}%
\pgfsys@transformshift{3.258061in}{5.229664in}%
\pgfsys@useobject{currentmarker}{}%
\end{pgfscope}%
\begin{pgfscope}%
\pgfsys@transformshift{3.275323in}{5.230586in}%
\pgfsys@useobject{currentmarker}{}%
\end{pgfscope}%
\begin{pgfscope}%
\pgfsys@transformshift{3.292585in}{5.235242in}%
\pgfsys@useobject{currentmarker}{}%
\end{pgfscope}%
\begin{pgfscope}%
\pgfsys@transformshift{3.309848in}{5.235008in}%
\pgfsys@useobject{currentmarker}{}%
\end{pgfscope}%
\begin{pgfscope}%
\pgfsys@transformshift{3.327110in}{5.236301in}%
\pgfsys@useobject{currentmarker}{}%
\end{pgfscope}%
\begin{pgfscope}%
\pgfsys@transformshift{3.344372in}{5.240786in}%
\pgfsys@useobject{currentmarker}{}%
\end{pgfscope}%
\begin{pgfscope}%
\pgfsys@transformshift{3.361634in}{5.242990in}%
\pgfsys@useobject{currentmarker}{}%
\end{pgfscope}%
\begin{pgfscope}%
\pgfsys@transformshift{3.378896in}{5.245871in}%
\pgfsys@useobject{currentmarker}{}%
\end{pgfscope}%
\begin{pgfscope}%
\pgfsys@transformshift{3.396158in}{5.244798in}%
\pgfsys@useobject{currentmarker}{}%
\end{pgfscope}%
\begin{pgfscope}%
\pgfsys@transformshift{3.413420in}{5.250360in}%
\pgfsys@useobject{currentmarker}{}%
\end{pgfscope}%
\begin{pgfscope}%
\pgfsys@transformshift{3.430682in}{5.252123in}%
\pgfsys@useobject{currentmarker}{}%
\end{pgfscope}%
\begin{pgfscope}%
\pgfsys@transformshift{3.447944in}{5.254107in}%
\pgfsys@useobject{currentmarker}{}%
\end{pgfscope}%
\begin{pgfscope}%
\pgfsys@transformshift{3.465206in}{5.255683in}%
\pgfsys@useobject{currentmarker}{}%
\end{pgfscope}%
\begin{pgfscope}%
\pgfsys@transformshift{3.482468in}{5.257144in}%
\pgfsys@useobject{currentmarker}{}%
\end{pgfscope}%
\begin{pgfscope}%
\pgfsys@transformshift{3.499730in}{5.258557in}%
\pgfsys@useobject{currentmarker}{}%
\end{pgfscope}%
\begin{pgfscope}%
\pgfsys@transformshift{3.516993in}{5.263145in}%
\pgfsys@useobject{currentmarker}{}%
\end{pgfscope}%
\begin{pgfscope}%
\pgfsys@transformshift{3.534255in}{5.266406in}%
\pgfsys@useobject{currentmarker}{}%
\end{pgfscope}%
\begin{pgfscope}%
\pgfsys@transformshift{3.551517in}{5.266870in}%
\pgfsys@useobject{currentmarker}{}%
\end{pgfscope}%
\begin{pgfscope}%
\pgfsys@transformshift{3.568779in}{5.271540in}%
\pgfsys@useobject{currentmarker}{}%
\end{pgfscope}%
\begin{pgfscope}%
\pgfsys@transformshift{3.586041in}{5.270980in}%
\pgfsys@useobject{currentmarker}{}%
\end{pgfscope}%
\begin{pgfscope}%
\pgfsys@transformshift{3.603303in}{5.273718in}%
\pgfsys@useobject{currentmarker}{}%
\end{pgfscope}%
\begin{pgfscope}%
\pgfsys@transformshift{3.620565in}{5.278111in}%
\pgfsys@useobject{currentmarker}{}%
\end{pgfscope}%
\begin{pgfscope}%
\pgfsys@transformshift{3.637827in}{5.282221in}%
\pgfsys@useobject{currentmarker}{}%
\end{pgfscope}%
\begin{pgfscope}%
\pgfsys@transformshift{3.655089in}{5.281895in}%
\pgfsys@useobject{currentmarker}{}%
\end{pgfscope}%
\begin{pgfscope}%
\pgfsys@transformshift{3.672351in}{5.286865in}%
\pgfsys@useobject{currentmarker}{}%
\end{pgfscope}%
\begin{pgfscope}%
\pgfsys@transformshift{3.689613in}{5.290696in}%
\pgfsys@useobject{currentmarker}{}%
\end{pgfscope}%
\begin{pgfscope}%
\pgfsys@transformshift{3.706875in}{5.292279in}%
\pgfsys@useobject{currentmarker}{}%
\end{pgfscope}%
\begin{pgfscope}%
\pgfsys@transformshift{3.724138in}{5.292829in}%
\pgfsys@useobject{currentmarker}{}%
\end{pgfscope}%
\begin{pgfscope}%
\pgfsys@transformshift{3.741400in}{5.297388in}%
\pgfsys@useobject{currentmarker}{}%
\end{pgfscope}%
\begin{pgfscope}%
\pgfsys@transformshift{3.758662in}{5.300862in}%
\pgfsys@useobject{currentmarker}{}%
\end{pgfscope}%
\begin{pgfscope}%
\pgfsys@transformshift{3.775924in}{5.301318in}%
\pgfsys@useobject{currentmarker}{}%
\end{pgfscope}%
\begin{pgfscope}%
\pgfsys@transformshift{3.793186in}{5.305517in}%
\pgfsys@useobject{currentmarker}{}%
\end{pgfscope}%
\begin{pgfscope}%
\pgfsys@transformshift{3.810448in}{5.310831in}%
\pgfsys@useobject{currentmarker}{}%
\end{pgfscope}%
\begin{pgfscope}%
\pgfsys@transformshift{3.827710in}{5.311237in}%
\pgfsys@useobject{currentmarker}{}%
\end{pgfscope}%
\begin{pgfscope}%
\pgfsys@transformshift{3.844972in}{5.317238in}%
\pgfsys@useobject{currentmarker}{}%
\end{pgfscope}%
\begin{pgfscope}%
\pgfsys@transformshift{3.862234in}{5.320027in}%
\pgfsys@useobject{currentmarker}{}%
\end{pgfscope}%
\begin{pgfscope}%
\pgfsys@transformshift{3.879496in}{5.322884in}%
\pgfsys@useobject{currentmarker}{}%
\end{pgfscope}%
\begin{pgfscope}%
\pgfsys@transformshift{3.896758in}{5.323373in}%
\pgfsys@useobject{currentmarker}{}%
\end{pgfscope}%
\begin{pgfscope}%
\pgfsys@transformshift{3.914020in}{5.329980in}%
\pgfsys@useobject{currentmarker}{}%
\end{pgfscope}%
\begin{pgfscope}%
\pgfsys@transformshift{3.931282in}{5.329823in}%
\pgfsys@useobject{currentmarker}{}%
\end{pgfscope}%
\begin{pgfscope}%
\pgfsys@transformshift{3.948545in}{5.333706in}%
\pgfsys@useobject{currentmarker}{}%
\end{pgfscope}%
\begin{pgfscope}%
\pgfsys@transformshift{3.965807in}{5.338863in}%
\pgfsys@useobject{currentmarker}{}%
\end{pgfscope}%
\begin{pgfscope}%
\pgfsys@transformshift{3.983069in}{5.340081in}%
\pgfsys@useobject{currentmarker}{}%
\end{pgfscope}%
\begin{pgfscope}%
\pgfsys@transformshift{4.000331in}{5.344243in}%
\pgfsys@useobject{currentmarker}{}%
\end{pgfscope}%
\begin{pgfscope}%
\pgfsys@transformshift{4.017593in}{5.346861in}%
\pgfsys@useobject{currentmarker}{}%
\end{pgfscope}%
\begin{pgfscope}%
\pgfsys@transformshift{4.034855in}{5.353639in}%
\pgfsys@useobject{currentmarker}{}%
\end{pgfscope}%
\begin{pgfscope}%
\pgfsys@transformshift{4.052117in}{5.357567in}%
\pgfsys@useobject{currentmarker}{}%
\end{pgfscope}%
\begin{pgfscope}%
\pgfsys@transformshift{4.069379in}{5.359245in}%
\pgfsys@useobject{currentmarker}{}%
\end{pgfscope}%
\begin{pgfscope}%
\pgfsys@transformshift{4.086641in}{5.364589in}%
\pgfsys@useobject{currentmarker}{}%
\end{pgfscope}%
\begin{pgfscope}%
\pgfsys@transformshift{4.103903in}{5.366701in}%
\pgfsys@useobject{currentmarker}{}%
\end{pgfscope}%
\begin{pgfscope}%
\pgfsys@transformshift{4.121165in}{5.371724in}%
\pgfsys@useobject{currentmarker}{}%
\end{pgfscope}%
\begin{pgfscope}%
\pgfsys@transformshift{4.138427in}{5.372278in}%
\pgfsys@useobject{currentmarker}{}%
\end{pgfscope}%
\begin{pgfscope}%
\pgfsys@transformshift{4.155690in}{5.376209in}%
\pgfsys@useobject{currentmarker}{}%
\end{pgfscope}%
\begin{pgfscope}%
\pgfsys@transformshift{4.172952in}{5.382532in}%
\pgfsys@useobject{currentmarker}{}%
\end{pgfscope}%
\begin{pgfscope}%
\pgfsys@transformshift{4.190214in}{5.386993in}%
\pgfsys@useobject{currentmarker}{}%
\end{pgfscope}%
\begin{pgfscope}%
\pgfsys@transformshift{4.207476in}{5.389458in}%
\pgfsys@useobject{currentmarker}{}%
\end{pgfscope}%
\begin{pgfscope}%
\pgfsys@transformshift{4.224738in}{5.392169in}%
\pgfsys@useobject{currentmarker}{}%
\end{pgfscope}%
\begin{pgfscope}%
\pgfsys@transformshift{4.242000in}{5.397651in}%
\pgfsys@useobject{currentmarker}{}%
\end{pgfscope}%
\begin{pgfscope}%
\pgfsys@transformshift{4.259262in}{5.400320in}%
\pgfsys@useobject{currentmarker}{}%
\end{pgfscope}%
\begin{pgfscope}%
\pgfsys@transformshift{4.276524in}{5.404108in}%
\pgfsys@useobject{currentmarker}{}%
\end{pgfscope}%
\begin{pgfscope}%
\pgfsys@transformshift{4.293786in}{5.411188in}%
\pgfsys@useobject{currentmarker}{}%
\end{pgfscope}%
\begin{pgfscope}%
\pgfsys@transformshift{4.311048in}{5.416165in}%
\pgfsys@useobject{currentmarker}{}%
\end{pgfscope}%
\begin{pgfscope}%
\pgfsys@transformshift{4.328310in}{5.416727in}%
\pgfsys@useobject{currentmarker}{}%
\end{pgfscope}%
\begin{pgfscope}%
\pgfsys@transformshift{4.345572in}{5.420937in}%
\pgfsys@useobject{currentmarker}{}%
\end{pgfscope}%
\begin{pgfscope}%
\pgfsys@transformshift{4.362835in}{5.429039in}%
\pgfsys@useobject{currentmarker}{}%
\end{pgfscope}%
\begin{pgfscope}%
\pgfsys@transformshift{4.380097in}{5.429932in}%
\pgfsys@useobject{currentmarker}{}%
\end{pgfscope}%
\begin{pgfscope}%
\pgfsys@transformshift{4.397359in}{5.435144in}%
\pgfsys@useobject{currentmarker}{}%
\end{pgfscope}%
\begin{pgfscope}%
\pgfsys@transformshift{4.414621in}{5.441052in}%
\pgfsys@useobject{currentmarker}{}%
\end{pgfscope}%
\begin{pgfscope}%
\pgfsys@transformshift{4.431883in}{5.445826in}%
\pgfsys@useobject{currentmarker}{}%
\end{pgfscope}%
\begin{pgfscope}%
\pgfsys@transformshift{4.449145in}{5.448866in}%
\pgfsys@useobject{currentmarker}{}%
\end{pgfscope}%
\begin{pgfscope}%
\pgfsys@transformshift{4.466407in}{5.453697in}%
\pgfsys@useobject{currentmarker}{}%
\end{pgfscope}%
\begin{pgfscope}%
\pgfsys@transformshift{4.483669in}{5.457880in}%
\pgfsys@useobject{currentmarker}{}%
\end{pgfscope}%
\begin{pgfscope}%
\pgfsys@transformshift{4.500931in}{5.462669in}%
\pgfsys@useobject{currentmarker}{}%
\end{pgfscope}%
\begin{pgfscope}%
\pgfsys@transformshift{4.518193in}{5.467025in}%
\pgfsys@useobject{currentmarker}{}%
\end{pgfscope}%
\begin{pgfscope}%
\pgfsys@transformshift{4.535455in}{5.473465in}%
\pgfsys@useobject{currentmarker}{}%
\end{pgfscope}%
\begin{pgfscope}%
\pgfsys@transformshift{4.552717in}{5.478114in}%
\pgfsys@useobject{currentmarker}{}%
\end{pgfscope}%
\begin{pgfscope}%
\pgfsys@transformshift{4.569980in}{5.482715in}%
\pgfsys@useobject{currentmarker}{}%
\end{pgfscope}%
\begin{pgfscope}%
\pgfsys@transformshift{4.587242in}{5.487734in}%
\pgfsys@useobject{currentmarker}{}%
\end{pgfscope}%
\begin{pgfscope}%
\pgfsys@transformshift{4.604504in}{5.492698in}%
\pgfsys@useobject{currentmarker}{}%
\end{pgfscope}%
\begin{pgfscope}%
\pgfsys@transformshift{4.621766in}{5.498304in}%
\pgfsys@useobject{currentmarker}{}%
\end{pgfscope}%
\begin{pgfscope}%
\pgfsys@transformshift{4.639028in}{5.501196in}%
\pgfsys@useobject{currentmarker}{}%
\end{pgfscope}%
\begin{pgfscope}%
\pgfsys@transformshift{4.656290in}{5.508408in}%
\pgfsys@useobject{currentmarker}{}%
\end{pgfscope}%
\begin{pgfscope}%
\pgfsys@transformshift{4.673552in}{5.514912in}%
\pgfsys@useobject{currentmarker}{}%
\end{pgfscope}%
\begin{pgfscope}%
\pgfsys@transformshift{4.690814in}{5.516901in}%
\pgfsys@useobject{currentmarker}{}%
\end{pgfscope}%
\begin{pgfscope}%
\pgfsys@transformshift{4.708076in}{5.523216in}%
\pgfsys@useobject{currentmarker}{}%
\end{pgfscope}%
\begin{pgfscope}%
\pgfsys@transformshift{4.725338in}{5.528558in}%
\pgfsys@useobject{currentmarker}{}%
\end{pgfscope}%
\begin{pgfscope}%
\pgfsys@transformshift{4.742600in}{5.533820in}%
\pgfsys@useobject{currentmarker}{}%
\end{pgfscope}%
\begin{pgfscope}%
\pgfsys@transformshift{4.759862in}{5.537872in}%
\pgfsys@useobject{currentmarker}{}%
\end{pgfscope}%
\begin{pgfscope}%
\pgfsys@transformshift{4.777124in}{5.545043in}%
\pgfsys@useobject{currentmarker}{}%
\end{pgfscope}%
\begin{pgfscope}%
\pgfsys@transformshift{4.794387in}{5.551327in}%
\pgfsys@useobject{currentmarker}{}%
\end{pgfscope}%
\begin{pgfscope}%
\pgfsys@transformshift{4.811649in}{5.556616in}%
\pgfsys@useobject{currentmarker}{}%
\end{pgfscope}%
\begin{pgfscope}%
\pgfsys@transformshift{4.828911in}{5.561593in}%
\pgfsys@useobject{currentmarker}{}%
\end{pgfscope}%
\begin{pgfscope}%
\pgfsys@transformshift{4.846173in}{5.568382in}%
\pgfsys@useobject{currentmarker}{}%
\end{pgfscope}%
\begin{pgfscope}%
\pgfsys@transformshift{4.863435in}{5.574951in}%
\pgfsys@useobject{currentmarker}{}%
\end{pgfscope}%
\begin{pgfscope}%
\pgfsys@transformshift{4.880697in}{5.578370in}%
\pgfsys@useobject{currentmarker}{}%
\end{pgfscope}%
\begin{pgfscope}%
\pgfsys@transformshift{4.897959in}{5.584807in}%
\pgfsys@useobject{currentmarker}{}%
\end{pgfscope}%
\begin{pgfscope}%
\pgfsys@transformshift{4.915221in}{5.589212in}%
\pgfsys@useobject{currentmarker}{}%
\end{pgfscope}%
\begin{pgfscope}%
\pgfsys@transformshift{4.932483in}{5.594937in}%
\pgfsys@useobject{currentmarker}{}%
\end{pgfscope}%
\begin{pgfscope}%
\pgfsys@transformshift{4.949745in}{5.603877in}%
\pgfsys@useobject{currentmarker}{}%
\end{pgfscope}%
\begin{pgfscope}%
\pgfsys@transformshift{4.967007in}{5.610084in}%
\pgfsys@useobject{currentmarker}{}%
\end{pgfscope}%
\begin{pgfscope}%
\pgfsys@transformshift{4.984269in}{5.616268in}%
\pgfsys@useobject{currentmarker}{}%
\end{pgfscope}%
\begin{pgfscope}%
\pgfsys@transformshift{5.001532in}{5.619993in}%
\pgfsys@useobject{currentmarker}{}%
\end{pgfscope}%
\begin{pgfscope}%
\pgfsys@transformshift{5.018794in}{5.627019in}%
\pgfsys@useobject{currentmarker}{}%
\end{pgfscope}%
\begin{pgfscope}%
\pgfsys@transformshift{5.036056in}{5.633685in}%
\pgfsys@useobject{currentmarker}{}%
\end{pgfscope}%
\begin{pgfscope}%
\pgfsys@transformshift{5.053318in}{5.639114in}%
\pgfsys@useobject{currentmarker}{}%
\end{pgfscope}%
\begin{pgfscope}%
\pgfsys@transformshift{5.070580in}{5.645269in}%
\pgfsys@useobject{currentmarker}{}%
\end{pgfscope}%
\begin{pgfscope}%
\pgfsys@transformshift{5.087842in}{5.653616in}%
\pgfsys@useobject{currentmarker}{}%
\end{pgfscope}%
\begin{pgfscope}%
\pgfsys@transformshift{5.105104in}{5.657161in}%
\pgfsys@useobject{currentmarker}{}%
\end{pgfscope}%
\begin{pgfscope}%
\pgfsys@transformshift{5.122366in}{5.666256in}%
\pgfsys@useobject{currentmarker}{}%
\end{pgfscope}%
\begin{pgfscope}%
\pgfsys@transformshift{5.139628in}{5.672168in}%
\pgfsys@useobject{currentmarker}{}%
\end{pgfscope}%
\begin{pgfscope}%
\pgfsys@transformshift{5.156890in}{5.676509in}%
\pgfsys@useobject{currentmarker}{}%
\end{pgfscope}%
\begin{pgfscope}%
\pgfsys@transformshift{5.174152in}{5.685056in}%
\pgfsys@useobject{currentmarker}{}%
\end{pgfscope}%
\begin{pgfscope}%
\pgfsys@transformshift{5.191414in}{5.692145in}%
\pgfsys@useobject{currentmarker}{}%
\end{pgfscope}%
\begin{pgfscope}%
\pgfsys@transformshift{5.208677in}{5.697511in}%
\pgfsys@useobject{currentmarker}{}%
\end{pgfscope}%
\begin{pgfscope}%
\pgfsys@transformshift{5.225939in}{5.705845in}%
\pgfsys@useobject{currentmarker}{}%
\end{pgfscope}%
\begin{pgfscope}%
\pgfsys@transformshift{5.243201in}{5.713061in}%
\pgfsys@useobject{currentmarker}{}%
\end{pgfscope}%
\begin{pgfscope}%
\pgfsys@transformshift{5.260463in}{5.717734in}%
\pgfsys@useobject{currentmarker}{}%
\end{pgfscope}%
\begin{pgfscope}%
\pgfsys@transformshift{5.277725in}{5.727620in}%
\pgfsys@useobject{currentmarker}{}%
\end{pgfscope}%
\begin{pgfscope}%
\pgfsys@transformshift{5.294987in}{5.734461in}%
\pgfsys@useobject{currentmarker}{}%
\end{pgfscope}%
\begin{pgfscope}%
\pgfsys@transformshift{5.312249in}{5.741670in}%
\pgfsys@useobject{currentmarker}{}%
\end{pgfscope}%
\begin{pgfscope}%
\pgfsys@transformshift{5.329511in}{5.746418in}%
\pgfsys@useobject{currentmarker}{}%
\end{pgfscope}%
\begin{pgfscope}%
\pgfsys@transformshift{5.346773in}{5.755724in}%
\pgfsys@useobject{currentmarker}{}%
\end{pgfscope}%
\begin{pgfscope}%
\pgfsys@transformshift{5.364035in}{5.763113in}%
\pgfsys@useobject{currentmarker}{}%
\end{pgfscope}%
\begin{pgfscope}%
\pgfsys@transformshift{5.381297in}{5.769943in}%
\pgfsys@useobject{currentmarker}{}%
\end{pgfscope}%
\end{pgfscope}%
\begin{pgfscope}%
\pgfsetrectcap%
\pgfsetmiterjoin%
\pgfsetlinewidth{0.803000pt}%
\definecolor{currentstroke}{rgb}{0.000000,0.000000,0.000000}%
\pgfsetstrokecolor{currentstroke}%
\pgfsetdash{}{0pt}%
\pgfpathmoveto{\pgfqpoint{0.759375in}{5.109574in}}%
\pgfpathlineto{\pgfqpoint{0.759375in}{5.801389in}}%
\pgfusepath{stroke}%
\end{pgfscope}%
\begin{pgfscope}%
\pgfsetrectcap%
\pgfsetmiterjoin%
\pgfsetlinewidth{0.803000pt}%
\definecolor{currentstroke}{rgb}{0.000000,0.000000,0.000000}%
\pgfsetstrokecolor{currentstroke}%
\pgfsetdash{}{0pt}%
\pgfpathmoveto{\pgfqpoint{5.601389in}{5.109574in}}%
\pgfpathlineto{\pgfqpoint{5.601389in}{5.801389in}}%
\pgfusepath{stroke}%
\end{pgfscope}%
\begin{pgfscope}%
\pgfsetrectcap%
\pgfsetmiterjoin%
\pgfsetlinewidth{0.803000pt}%
\definecolor{currentstroke}{rgb}{0.000000,0.000000,0.000000}%
\pgfsetstrokecolor{currentstroke}%
\pgfsetdash{}{0pt}%
\pgfpathmoveto{\pgfqpoint{0.759375in}{5.109574in}}%
\pgfpathlineto{\pgfqpoint{5.601389in}{5.109574in}}%
\pgfusepath{stroke}%
\end{pgfscope}%
\begin{pgfscope}%
\pgfsetrectcap%
\pgfsetmiterjoin%
\pgfsetlinewidth{0.803000pt}%
\definecolor{currentstroke}{rgb}{0.000000,0.000000,0.000000}%
\pgfsetstrokecolor{currentstroke}%
\pgfsetdash{}{0pt}%
\pgfpathmoveto{\pgfqpoint{0.759375in}{5.801389in}}%
\pgfpathlineto{\pgfqpoint{5.601389in}{5.801389in}}%
\pgfusepath{stroke}%
\end{pgfscope}%
\begin{pgfscope}%
\pgfsetbuttcap%
\pgfsetmiterjoin%
\definecolor{currentfill}{rgb}{1.000000,1.000000,1.000000}%
\pgfsetfillcolor{currentfill}%
\pgfsetlinewidth{0.000000pt}%
\definecolor{currentstroke}{rgb}{0.000000,0.000000,0.000000}%
\pgfsetstrokecolor{currentstroke}%
\pgfsetstrokeopacity{0.000000}%
\pgfsetdash{}{0pt}%
\pgfpathmoveto{\pgfqpoint{0.759375in}{4.164881in}}%
\pgfpathlineto{\pgfqpoint{5.601389in}{4.164881in}}%
\pgfpathlineto{\pgfqpoint{5.601389in}{4.856696in}}%
\pgfpathlineto{\pgfqpoint{0.759375in}{4.856696in}}%
\pgfpathclose%
\pgfusepath{fill}%
\end{pgfscope}%
\begin{pgfscope}%
\pgfsetbuttcap%
\pgfsetroundjoin%
\definecolor{currentfill}{rgb}{0.000000,0.000000,0.000000}%
\pgfsetfillcolor{currentfill}%
\pgfsetlinewidth{0.803000pt}%
\definecolor{currentstroke}{rgb}{0.000000,0.000000,0.000000}%
\pgfsetstrokecolor{currentstroke}%
\pgfsetdash{}{0pt}%
\pgfsys@defobject{currentmarker}{\pgfqpoint{0.000000in}{-0.048611in}}{\pgfqpoint{0.000000in}{0.000000in}}{%
\pgfpathmoveto{\pgfqpoint{0.000000in}{0.000000in}}%
\pgfpathlineto{\pgfqpoint{0.000000in}{-0.048611in}}%
\pgfusepath{stroke,fill}%
}%
\begin{pgfscope}%
\pgfsys@transformshift{0.979467in}{4.164881in}%
\pgfsys@useobject{currentmarker}{}%
\end{pgfscope}%
\end{pgfscope}%
\begin{pgfscope}%
\pgfsetbuttcap%
\pgfsetroundjoin%
\definecolor{currentfill}{rgb}{0.000000,0.000000,0.000000}%
\pgfsetfillcolor{currentfill}%
\pgfsetlinewidth{0.803000pt}%
\definecolor{currentstroke}{rgb}{0.000000,0.000000,0.000000}%
\pgfsetstrokecolor{currentstroke}%
\pgfsetdash{}{0pt}%
\pgfsys@defobject{currentmarker}{\pgfqpoint{0.000000in}{-0.048611in}}{\pgfqpoint{0.000000in}{0.000000in}}{%
\pgfpathmoveto{\pgfqpoint{0.000000in}{0.000000in}}%
\pgfpathlineto{\pgfqpoint{0.000000in}{-0.048611in}}%
\pgfusepath{stroke,fill}%
}%
\begin{pgfscope}%
\pgfsys@transformshift{1.529695in}{4.164881in}%
\pgfsys@useobject{currentmarker}{}%
\end{pgfscope}%
\end{pgfscope}%
\begin{pgfscope}%
\pgfsetbuttcap%
\pgfsetroundjoin%
\definecolor{currentfill}{rgb}{0.000000,0.000000,0.000000}%
\pgfsetfillcolor{currentfill}%
\pgfsetlinewidth{0.803000pt}%
\definecolor{currentstroke}{rgb}{0.000000,0.000000,0.000000}%
\pgfsetstrokecolor{currentstroke}%
\pgfsetdash{}{0pt}%
\pgfsys@defobject{currentmarker}{\pgfqpoint{0.000000in}{-0.048611in}}{\pgfqpoint{0.000000in}{0.000000in}}{%
\pgfpathmoveto{\pgfqpoint{0.000000in}{0.000000in}}%
\pgfpathlineto{\pgfqpoint{0.000000in}{-0.048611in}}%
\pgfusepath{stroke,fill}%
}%
\begin{pgfscope}%
\pgfsys@transformshift{2.079924in}{4.164881in}%
\pgfsys@useobject{currentmarker}{}%
\end{pgfscope}%
\end{pgfscope}%
\begin{pgfscope}%
\pgfsetbuttcap%
\pgfsetroundjoin%
\definecolor{currentfill}{rgb}{0.000000,0.000000,0.000000}%
\pgfsetfillcolor{currentfill}%
\pgfsetlinewidth{0.803000pt}%
\definecolor{currentstroke}{rgb}{0.000000,0.000000,0.000000}%
\pgfsetstrokecolor{currentstroke}%
\pgfsetdash{}{0pt}%
\pgfsys@defobject{currentmarker}{\pgfqpoint{0.000000in}{-0.048611in}}{\pgfqpoint{0.000000in}{0.000000in}}{%
\pgfpathmoveto{\pgfqpoint{0.000000in}{0.000000in}}%
\pgfpathlineto{\pgfqpoint{0.000000in}{-0.048611in}}%
\pgfusepath{stroke,fill}%
}%
\begin{pgfscope}%
\pgfsys@transformshift{2.630153in}{4.164881in}%
\pgfsys@useobject{currentmarker}{}%
\end{pgfscope}%
\end{pgfscope}%
\begin{pgfscope}%
\pgfsetbuttcap%
\pgfsetroundjoin%
\definecolor{currentfill}{rgb}{0.000000,0.000000,0.000000}%
\pgfsetfillcolor{currentfill}%
\pgfsetlinewidth{0.803000pt}%
\definecolor{currentstroke}{rgb}{0.000000,0.000000,0.000000}%
\pgfsetstrokecolor{currentstroke}%
\pgfsetdash{}{0pt}%
\pgfsys@defobject{currentmarker}{\pgfqpoint{0.000000in}{-0.048611in}}{\pgfqpoint{0.000000in}{0.000000in}}{%
\pgfpathmoveto{\pgfqpoint{0.000000in}{0.000000in}}%
\pgfpathlineto{\pgfqpoint{0.000000in}{-0.048611in}}%
\pgfusepath{stroke,fill}%
}%
\begin{pgfscope}%
\pgfsys@transformshift{3.180382in}{4.164881in}%
\pgfsys@useobject{currentmarker}{}%
\end{pgfscope}%
\end{pgfscope}%
\begin{pgfscope}%
\pgfsetbuttcap%
\pgfsetroundjoin%
\definecolor{currentfill}{rgb}{0.000000,0.000000,0.000000}%
\pgfsetfillcolor{currentfill}%
\pgfsetlinewidth{0.803000pt}%
\definecolor{currentstroke}{rgb}{0.000000,0.000000,0.000000}%
\pgfsetstrokecolor{currentstroke}%
\pgfsetdash{}{0pt}%
\pgfsys@defobject{currentmarker}{\pgfqpoint{0.000000in}{-0.048611in}}{\pgfqpoint{0.000000in}{0.000000in}}{%
\pgfpathmoveto{\pgfqpoint{0.000000in}{0.000000in}}%
\pgfpathlineto{\pgfqpoint{0.000000in}{-0.048611in}}%
\pgfusepath{stroke,fill}%
}%
\begin{pgfscope}%
\pgfsys@transformshift{3.730611in}{4.164881in}%
\pgfsys@useobject{currentmarker}{}%
\end{pgfscope}%
\end{pgfscope}%
\begin{pgfscope}%
\pgfsetbuttcap%
\pgfsetroundjoin%
\definecolor{currentfill}{rgb}{0.000000,0.000000,0.000000}%
\pgfsetfillcolor{currentfill}%
\pgfsetlinewidth{0.803000pt}%
\definecolor{currentstroke}{rgb}{0.000000,0.000000,0.000000}%
\pgfsetstrokecolor{currentstroke}%
\pgfsetdash{}{0pt}%
\pgfsys@defobject{currentmarker}{\pgfqpoint{0.000000in}{-0.048611in}}{\pgfqpoint{0.000000in}{0.000000in}}{%
\pgfpathmoveto{\pgfqpoint{0.000000in}{0.000000in}}%
\pgfpathlineto{\pgfqpoint{0.000000in}{-0.048611in}}%
\pgfusepath{stroke,fill}%
}%
\begin{pgfscope}%
\pgfsys@transformshift{4.280840in}{4.164881in}%
\pgfsys@useobject{currentmarker}{}%
\end{pgfscope}%
\end{pgfscope}%
\begin{pgfscope}%
\pgfsetbuttcap%
\pgfsetroundjoin%
\definecolor{currentfill}{rgb}{0.000000,0.000000,0.000000}%
\pgfsetfillcolor{currentfill}%
\pgfsetlinewidth{0.803000pt}%
\definecolor{currentstroke}{rgb}{0.000000,0.000000,0.000000}%
\pgfsetstrokecolor{currentstroke}%
\pgfsetdash{}{0pt}%
\pgfsys@defobject{currentmarker}{\pgfqpoint{0.000000in}{-0.048611in}}{\pgfqpoint{0.000000in}{0.000000in}}{%
\pgfpathmoveto{\pgfqpoint{0.000000in}{0.000000in}}%
\pgfpathlineto{\pgfqpoint{0.000000in}{-0.048611in}}%
\pgfusepath{stroke,fill}%
}%
\begin{pgfscope}%
\pgfsys@transformshift{4.831068in}{4.164881in}%
\pgfsys@useobject{currentmarker}{}%
\end{pgfscope}%
\end{pgfscope}%
\begin{pgfscope}%
\pgfsetbuttcap%
\pgfsetroundjoin%
\definecolor{currentfill}{rgb}{0.000000,0.000000,0.000000}%
\pgfsetfillcolor{currentfill}%
\pgfsetlinewidth{0.803000pt}%
\definecolor{currentstroke}{rgb}{0.000000,0.000000,0.000000}%
\pgfsetstrokecolor{currentstroke}%
\pgfsetdash{}{0pt}%
\pgfsys@defobject{currentmarker}{\pgfqpoint{0.000000in}{-0.048611in}}{\pgfqpoint{0.000000in}{0.000000in}}{%
\pgfpathmoveto{\pgfqpoint{0.000000in}{0.000000in}}%
\pgfpathlineto{\pgfqpoint{0.000000in}{-0.048611in}}%
\pgfusepath{stroke,fill}%
}%
\begin{pgfscope}%
\pgfsys@transformshift{5.381297in}{4.164881in}%
\pgfsys@useobject{currentmarker}{}%
\end{pgfscope}%
\end{pgfscope}%
\begin{pgfscope}%
\pgfsetbuttcap%
\pgfsetroundjoin%
\definecolor{currentfill}{rgb}{0.000000,0.000000,0.000000}%
\pgfsetfillcolor{currentfill}%
\pgfsetlinewidth{0.803000pt}%
\definecolor{currentstroke}{rgb}{0.000000,0.000000,0.000000}%
\pgfsetstrokecolor{currentstroke}%
\pgfsetdash{}{0pt}%
\pgfsys@defobject{currentmarker}{\pgfqpoint{-0.048611in}{0.000000in}}{\pgfqpoint{0.000000in}{0.000000in}}{%
\pgfpathmoveto{\pgfqpoint{0.000000in}{0.000000in}}%
\pgfpathlineto{\pgfqpoint{-0.048611in}{0.000000in}}%
\pgfusepath{stroke,fill}%
}%
\begin{pgfscope}%
\pgfsys@transformshift{0.759375in}{4.277770in}%
\pgfsys@useobject{currentmarker}{}%
\end{pgfscope}%
\end{pgfscope}%
\begin{pgfscope}%
\definecolor{textcolor}{rgb}{0.000000,0.000000,0.000000}%
\pgfsetstrokecolor{textcolor}%
\pgfsetfillcolor{textcolor}%
\pgftext[x=0.237708in,y=4.229576in,left,base]{\color{textcolor}\rmfamily\fontsize{10.000000}{12.000000}\selectfont −0.025}%
\end{pgfscope}%
\begin{pgfscope}%
\pgfsetbuttcap%
\pgfsetroundjoin%
\definecolor{currentfill}{rgb}{0.000000,0.000000,0.000000}%
\pgfsetfillcolor{currentfill}%
\pgfsetlinewidth{0.803000pt}%
\definecolor{currentstroke}{rgb}{0.000000,0.000000,0.000000}%
\pgfsetstrokecolor{currentstroke}%
\pgfsetdash{}{0pt}%
\pgfsys@defobject{currentmarker}{\pgfqpoint{-0.048611in}{0.000000in}}{\pgfqpoint{0.000000in}{0.000000in}}{%
\pgfpathmoveto{\pgfqpoint{0.000000in}{0.000000in}}%
\pgfpathlineto{\pgfqpoint{-0.048611in}{0.000000in}}%
\pgfusepath{stroke,fill}%
}%
\begin{pgfscope}%
\pgfsys@transformshift{0.759375in}{4.494945in}%
\pgfsys@useobject{currentmarker}{}%
\end{pgfscope}%
\end{pgfscope}%
\begin{pgfscope}%
\definecolor{textcolor}{rgb}{0.000000,0.000000,0.000000}%
\pgfsetstrokecolor{textcolor}%
\pgfsetfillcolor{textcolor}%
\pgftext[x=0.345764in,y=4.446751in,left,base]{\color{textcolor}\rmfamily\fontsize{10.000000}{12.000000}\selectfont 0.000}%
\end{pgfscope}%
\begin{pgfscope}%
\pgfsetbuttcap%
\pgfsetroundjoin%
\definecolor{currentfill}{rgb}{0.000000,0.000000,0.000000}%
\pgfsetfillcolor{currentfill}%
\pgfsetlinewidth{0.803000pt}%
\definecolor{currentstroke}{rgb}{0.000000,0.000000,0.000000}%
\pgfsetstrokecolor{currentstroke}%
\pgfsetdash{}{0pt}%
\pgfsys@defobject{currentmarker}{\pgfqpoint{-0.048611in}{0.000000in}}{\pgfqpoint{0.000000in}{0.000000in}}{%
\pgfpathmoveto{\pgfqpoint{0.000000in}{0.000000in}}%
\pgfpathlineto{\pgfqpoint{-0.048611in}{0.000000in}}%
\pgfusepath{stroke,fill}%
}%
\begin{pgfscope}%
\pgfsys@transformshift{0.759375in}{4.712121in}%
\pgfsys@useobject{currentmarker}{}%
\end{pgfscope}%
\end{pgfscope}%
\begin{pgfscope}%
\definecolor{textcolor}{rgb}{0.000000,0.000000,0.000000}%
\pgfsetstrokecolor{textcolor}%
\pgfsetfillcolor{textcolor}%
\pgftext[x=0.345764in,y=4.663926in,left,base]{\color{textcolor}\rmfamily\fontsize{10.000000}{12.000000}\selectfont 0.025}%
\end{pgfscope}%
\begin{pgfscope}%
\pgfpathrectangle{\pgfqpoint{0.759375in}{4.164881in}}{\pgfqpoint{4.842014in}{0.691815in}}%
\pgfusepath{clip}%
\pgfsetrectcap%
\pgfsetroundjoin%
\pgfsetlinewidth{1.505625pt}%
\definecolor{currentstroke}{rgb}{1.000000,0.498039,0.054902}%
\pgfsetstrokecolor{currentstroke}%
\pgfsetdash{}{0pt}%
\pgfpathmoveto{\pgfqpoint{0.988098in}{4.612823in}}%
\pgfpathlineto{\pgfqpoint{1.005777in}{4.612823in}}%
\pgfpathlineto{\pgfqpoint{1.005777in}{4.342015in}}%
\pgfpathlineto{\pgfqpoint{1.041137in}{4.342015in}}%
\pgfpathlineto{\pgfqpoint{1.041137in}{4.525058in}}%
\pgfpathlineto{\pgfqpoint{1.076496in}{4.525058in}}%
\pgfpathlineto{\pgfqpoint{1.076496in}{4.655434in}}%
\pgfpathlineto{\pgfqpoint{1.111856in}{4.655434in}}%
\pgfpathlineto{\pgfqpoint{1.111856in}{4.418634in}}%
\pgfpathlineto{\pgfqpoint{1.147215in}{4.418634in}}%
\pgfpathlineto{\pgfqpoint{1.147215in}{4.611580in}}%
\pgfpathlineto{\pgfqpoint{1.182574in}{4.611580in}}%
\pgfpathlineto{\pgfqpoint{1.182574in}{4.537789in}}%
\pgfpathlineto{\pgfqpoint{1.217934in}{4.537789in}}%
\pgfpathlineto{\pgfqpoint{1.217934in}{4.366177in}}%
\pgfpathlineto{\pgfqpoint{1.253293in}{4.366177in}}%
\pgfpathlineto{\pgfqpoint{1.253293in}{4.409802in}}%
\pgfpathlineto{\pgfqpoint{1.288653in}{4.409802in}}%
\pgfpathlineto{\pgfqpoint{1.288653in}{4.314122in}}%
\pgfpathlineto{\pgfqpoint{1.324012in}{4.314122in}}%
\pgfpathlineto{\pgfqpoint{1.324012in}{4.303470in}}%
\pgfpathlineto{\pgfqpoint{1.359372in}{4.303470in}}%
\pgfpathlineto{\pgfqpoint{1.359372in}{4.635835in}}%
\pgfpathlineto{\pgfqpoint{1.394731in}{4.635835in}}%
\pgfpathlineto{\pgfqpoint{1.394731in}{4.651717in}}%
\pgfpathlineto{\pgfqpoint{1.430090in}{4.651717in}}%
\pgfpathlineto{\pgfqpoint{1.430090in}{4.449343in}}%
\pgfpathlineto{\pgfqpoint{1.465450in}{4.449343in}}%
\pgfpathlineto{\pgfqpoint{1.465450in}{4.313245in}}%
\pgfpathlineto{\pgfqpoint{1.500809in}{4.313245in}}%
\pgfpathlineto{\pgfqpoint{1.500809in}{4.230843in}}%
\pgfpathlineto{\pgfqpoint{1.536169in}{4.230843in}}%
\pgfpathlineto{\pgfqpoint{1.536169in}{4.425061in}}%
\pgfpathlineto{\pgfqpoint{1.571528in}{4.425061in}}%
\pgfpathlineto{\pgfqpoint{1.571528in}{4.438196in}}%
\pgfpathlineto{\pgfqpoint{1.606888in}{4.438196in}}%
\pgfpathlineto{\pgfqpoint{1.606888in}{4.365454in}}%
\pgfpathlineto{\pgfqpoint{1.642247in}{4.365454in}}%
\pgfpathlineto{\pgfqpoint{1.642247in}{4.446010in}}%
\pgfpathlineto{\pgfqpoint{1.677606in}{4.446010in}}%
\pgfpathlineto{\pgfqpoint{1.677606in}{4.741503in}}%
\pgfpathlineto{\pgfqpoint{1.712966in}{4.741503in}}%
\pgfpathlineto{\pgfqpoint{1.712966in}{4.461749in}}%
\pgfpathlineto{\pgfqpoint{1.748325in}{4.461749in}}%
\pgfpathlineto{\pgfqpoint{1.748325in}{4.645485in}}%
\pgfpathlineto{\pgfqpoint{1.783685in}{4.645485in}}%
\pgfpathlineto{\pgfqpoint{1.783685in}{4.618607in}}%
\pgfpathlineto{\pgfqpoint{1.819044in}{4.618607in}}%
\pgfpathlineto{\pgfqpoint{1.819044in}{4.234765in}}%
\pgfpathlineto{\pgfqpoint{1.854403in}{4.234765in}}%
\pgfpathlineto{\pgfqpoint{1.854403in}{4.477300in}}%
\pgfpathlineto{\pgfqpoint{1.889763in}{4.477300in}}%
\pgfpathlineto{\pgfqpoint{1.889763in}{4.320838in}}%
\pgfpathlineto{\pgfqpoint{1.925122in}{4.320838in}}%
\pgfpathlineto{\pgfqpoint{1.925122in}{4.726977in}}%
\pgfpathlineto{\pgfqpoint{1.960482in}{4.726977in}}%
\pgfpathlineto{\pgfqpoint{1.960482in}{4.762795in}}%
\pgfpathlineto{\pgfqpoint{1.995841in}{4.762795in}}%
\pgfpathlineto{\pgfqpoint{1.995841in}{4.474019in}}%
\pgfpathlineto{\pgfqpoint{2.031201in}{4.474019in}}%
\pgfpathlineto{\pgfqpoint{2.031201in}{4.812848in}}%
\pgfpathlineto{\pgfqpoint{2.066560in}{4.812848in}}%
\pgfpathlineto{\pgfqpoint{2.066560in}{4.505128in}}%
\pgfpathlineto{\pgfqpoint{2.101919in}{4.505128in}}%
\pgfpathlineto{\pgfqpoint{2.101919in}{4.676645in}}%
\pgfpathlineto{\pgfqpoint{2.137279in}{4.676645in}}%
\pgfpathlineto{\pgfqpoint{2.137279in}{4.371727in}}%
\pgfpathlineto{\pgfqpoint{2.172638in}{4.371727in}}%
\pgfpathlineto{\pgfqpoint{2.172638in}{4.395113in}}%
\pgfpathlineto{\pgfqpoint{2.207998in}{4.395113in}}%
\pgfpathlineto{\pgfqpoint{2.207998in}{4.690557in}}%
\pgfpathlineto{\pgfqpoint{2.243357in}{4.690557in}}%
\pgfpathlineto{\pgfqpoint{2.243357in}{4.400103in}}%
\pgfpathlineto{\pgfqpoint{2.278717in}{4.400103in}}%
\pgfpathlineto{\pgfqpoint{2.278717in}{4.443152in}}%
\pgfpathlineto{\pgfqpoint{2.314076in}{4.443152in}}%
\pgfpathlineto{\pgfqpoint{2.314076in}{4.515329in}}%
\pgfpathlineto{\pgfqpoint{2.349435in}{4.515329in}}%
\pgfpathlineto{\pgfqpoint{2.349435in}{4.576077in}}%
\pgfpathlineto{\pgfqpoint{2.384795in}{4.576077in}}%
\pgfpathlineto{\pgfqpoint{2.384795in}{4.274239in}}%
\pgfpathlineto{\pgfqpoint{2.420154in}{4.274239in}}%
\pgfpathlineto{\pgfqpoint{2.420154in}{4.449230in}}%
\pgfpathlineto{\pgfqpoint{2.455514in}{4.449230in}}%
\pgfpathlineto{\pgfqpoint{2.455514in}{4.467345in}}%
\pgfpathlineto{\pgfqpoint{2.490873in}{4.467345in}}%
\pgfpathlineto{\pgfqpoint{2.490873in}{4.687298in}}%
\pgfpathlineto{\pgfqpoint{2.526233in}{4.687298in}}%
\pgfpathlineto{\pgfqpoint{2.526233in}{4.294721in}}%
\pgfpathlineto{\pgfqpoint{2.561592in}{4.294721in}}%
\pgfpathlineto{\pgfqpoint{2.561592in}{4.458073in}}%
\pgfpathlineto{\pgfqpoint{2.596951in}{4.458073in}}%
\pgfpathlineto{\pgfqpoint{2.596951in}{4.769474in}}%
\pgfpathlineto{\pgfqpoint{2.632311in}{4.769474in}}%
\pgfpathlineto{\pgfqpoint{2.632311in}{4.379586in}}%
\pgfpathlineto{\pgfqpoint{2.667670in}{4.379586in}}%
\pgfpathlineto{\pgfqpoint{2.667670in}{4.341308in}}%
\pgfpathlineto{\pgfqpoint{2.703030in}{4.341308in}}%
\pgfpathlineto{\pgfqpoint{2.703030in}{4.493692in}}%
\pgfpathlineto{\pgfqpoint{2.738389in}{4.493692in}}%
\pgfpathlineto{\pgfqpoint{2.738389in}{4.458276in}}%
\pgfpathlineto{\pgfqpoint{2.773749in}{4.458276in}}%
\pgfpathlineto{\pgfqpoint{2.773749in}{4.436234in}}%
\pgfpathlineto{\pgfqpoint{2.809108in}{4.436234in}}%
\pgfpathlineto{\pgfqpoint{2.809108in}{4.363483in}}%
\pgfpathlineto{\pgfqpoint{2.844467in}{4.363483in}}%
\pgfpathlineto{\pgfqpoint{2.844467in}{4.410004in}}%
\pgfpathlineto{\pgfqpoint{2.879827in}{4.410004in}}%
\pgfpathlineto{\pgfqpoint{2.879827in}{4.605228in}}%
\pgfpathlineto{\pgfqpoint{2.915186in}{4.605228in}}%
\pgfpathlineto{\pgfqpoint{2.915186in}{4.551649in}}%
\pgfpathlineto{\pgfqpoint{2.950546in}{4.551649in}}%
\pgfpathlineto{\pgfqpoint{2.950546in}{4.582049in}}%
\pgfpathlineto{\pgfqpoint{2.985905in}{4.582049in}}%
\pgfpathlineto{\pgfqpoint{2.985905in}{4.439149in}}%
\pgfpathlineto{\pgfqpoint{3.021265in}{4.439149in}}%
\pgfpathlineto{\pgfqpoint{3.021265in}{4.589145in}}%
\pgfpathlineto{\pgfqpoint{3.056624in}{4.589145in}}%
\pgfpathlineto{\pgfqpoint{3.056624in}{4.475914in}}%
\pgfpathlineto{\pgfqpoint{3.091983in}{4.475914in}}%
\pgfpathlineto{\pgfqpoint{3.091983in}{4.602851in}}%
\pgfpathlineto{\pgfqpoint{3.127343in}{4.602851in}}%
\pgfpathlineto{\pgfqpoint{3.127343in}{4.708926in}}%
\pgfpathlineto{\pgfqpoint{3.162702in}{4.708926in}}%
\pgfpathlineto{\pgfqpoint{3.162702in}{4.433795in}}%
\pgfpathlineto{\pgfqpoint{3.198062in}{4.433795in}}%
\pgfpathlineto{\pgfqpoint{3.198062in}{4.555799in}}%
\pgfpathlineto{\pgfqpoint{3.233421in}{4.555799in}}%
\pgfpathlineto{\pgfqpoint{3.233421in}{4.488624in}}%
\pgfpathlineto{\pgfqpoint{3.268781in}{4.488624in}}%
\pgfpathlineto{\pgfqpoint{3.268781in}{4.703248in}}%
\pgfpathlineto{\pgfqpoint{3.304140in}{4.703248in}}%
\pgfpathlineto{\pgfqpoint{3.304140in}{4.302053in}}%
\pgfpathlineto{\pgfqpoint{3.339499in}{4.302053in}}%
\pgfpathlineto{\pgfqpoint{3.339499in}{4.491431in}}%
\pgfpathlineto{\pgfqpoint{3.374859in}{4.491431in}}%
\pgfpathlineto{\pgfqpoint{3.374859in}{4.196327in}}%
\pgfpathlineto{\pgfqpoint{3.410218in}{4.196327in}}%
\pgfpathlineto{\pgfqpoint{3.410218in}{4.537594in}}%
\pgfpathlineto{\pgfqpoint{3.445578in}{4.537594in}}%
\pgfpathlineto{\pgfqpoint{3.445578in}{4.506972in}}%
\pgfpathlineto{\pgfqpoint{3.480937in}{4.506972in}}%
\pgfpathlineto{\pgfqpoint{3.480937in}{4.325512in}}%
\pgfpathlineto{\pgfqpoint{3.516296in}{4.325512in}}%
\pgfpathlineto{\pgfqpoint{3.516296in}{4.663800in}}%
\pgfpathlineto{\pgfqpoint{3.551656in}{4.663800in}}%
\pgfpathlineto{\pgfqpoint{3.551656in}{4.714699in}}%
\pgfpathlineto{\pgfqpoint{3.587015in}{4.714699in}}%
\pgfpathlineto{\pgfqpoint{3.587015in}{4.358987in}}%
\pgfpathlineto{\pgfqpoint{3.622375in}{4.358987in}}%
\pgfpathlineto{\pgfqpoint{3.622375in}{4.736670in}}%
\pgfpathlineto{\pgfqpoint{3.657734in}{4.736670in}}%
\pgfpathlineto{\pgfqpoint{3.657734in}{4.479895in}}%
\pgfpathlineto{\pgfqpoint{3.693094in}{4.479895in}}%
\pgfpathlineto{\pgfqpoint{3.693094in}{4.582590in}}%
\pgfpathlineto{\pgfqpoint{3.728453in}{4.582590in}}%
\pgfpathlineto{\pgfqpoint{3.728453in}{4.495574in}}%
\pgfpathlineto{\pgfqpoint{3.763812in}{4.495574in}}%
\pgfpathlineto{\pgfqpoint{3.763812in}{4.337749in}}%
\pgfpathlineto{\pgfqpoint{3.799172in}{4.337749in}}%
\pgfpathlineto{\pgfqpoint{3.799172in}{4.741839in}}%
\pgfpathlineto{\pgfqpoint{3.834531in}{4.741839in}}%
\pgfpathlineto{\pgfqpoint{3.834531in}{4.586630in}}%
\pgfpathlineto{\pgfqpoint{3.869891in}{4.586630in}}%
\pgfpathlineto{\pgfqpoint{3.869891in}{4.620765in}}%
\pgfpathlineto{\pgfqpoint{3.905250in}{4.620765in}}%
\pgfpathlineto{\pgfqpoint{3.905250in}{4.769401in}}%
\pgfpathlineto{\pgfqpoint{3.940610in}{4.769401in}}%
\pgfpathlineto{\pgfqpoint{3.940610in}{4.368755in}}%
\pgfpathlineto{\pgfqpoint{3.975969in}{4.368755in}}%
\pgfpathlineto{\pgfqpoint{3.975969in}{4.393781in}}%
\pgfpathlineto{\pgfqpoint{4.011328in}{4.393781in}}%
\pgfpathlineto{\pgfqpoint{4.011328in}{4.294250in}}%
\pgfpathlineto{\pgfqpoint{4.046688in}{4.294250in}}%
\pgfpathlineto{\pgfqpoint{4.046688in}{4.656395in}}%
\pgfpathlineto{\pgfqpoint{4.082047in}{4.656395in}}%
\pgfpathlineto{\pgfqpoint{4.082047in}{4.615443in}}%
\pgfpathlineto{\pgfqpoint{4.117407in}{4.615443in}}%
\pgfpathlineto{\pgfqpoint{4.117407in}{4.692520in}}%
\pgfpathlineto{\pgfqpoint{4.152766in}{4.692520in}}%
\pgfpathlineto{\pgfqpoint{4.152766in}{4.318325in}}%
\pgfpathlineto{\pgfqpoint{4.188126in}{4.318325in}}%
\pgfpathlineto{\pgfqpoint{4.188126in}{4.628600in}}%
\pgfpathlineto{\pgfqpoint{4.223485in}{4.628600in}}%
\pgfpathlineto{\pgfqpoint{4.223485in}{4.342954in}}%
\pgfpathlineto{\pgfqpoint{4.258844in}{4.342954in}}%
\pgfpathlineto{\pgfqpoint{4.258844in}{4.475326in}}%
\pgfpathlineto{\pgfqpoint{4.294204in}{4.475326in}}%
\pgfpathlineto{\pgfqpoint{4.294204in}{4.560362in}}%
\pgfpathlineto{\pgfqpoint{4.329563in}{4.560362in}}%
\pgfpathlineto{\pgfqpoint{4.329563in}{4.362908in}}%
\pgfpathlineto{\pgfqpoint{4.364923in}{4.362908in}}%
\pgfpathlineto{\pgfqpoint{4.364923in}{4.825250in}}%
\pgfpathlineto{\pgfqpoint{4.400282in}{4.825250in}}%
\pgfpathlineto{\pgfqpoint{4.400282in}{4.395663in}}%
\pgfpathlineto{\pgfqpoint{4.435642in}{4.395663in}}%
\pgfpathlineto{\pgfqpoint{4.435642in}{4.604156in}}%
\pgfpathlineto{\pgfqpoint{4.471001in}{4.604156in}}%
\pgfpathlineto{\pgfqpoint{4.471001in}{4.503919in}}%
\pgfpathlineto{\pgfqpoint{4.506360in}{4.503919in}}%
\pgfpathlineto{\pgfqpoint{4.506360in}{4.509435in}}%
\pgfpathlineto{\pgfqpoint{4.541720in}{4.509435in}}%
\pgfpathlineto{\pgfqpoint{4.541720in}{4.560932in}}%
\pgfpathlineto{\pgfqpoint{4.577079in}{4.560932in}}%
\pgfpathlineto{\pgfqpoint{4.577079in}{4.473255in}}%
\pgfpathlineto{\pgfqpoint{4.612439in}{4.473255in}}%
\pgfpathlineto{\pgfqpoint{4.612439in}{4.461286in}}%
\pgfpathlineto{\pgfqpoint{4.647798in}{4.461286in}}%
\pgfpathlineto{\pgfqpoint{4.647798in}{4.302494in}}%
\pgfpathlineto{\pgfqpoint{4.683158in}{4.302494in}}%
\pgfpathlineto{\pgfqpoint{4.683158in}{4.746930in}}%
\pgfpathlineto{\pgfqpoint{4.718517in}{4.746930in}}%
\pgfpathlineto{\pgfqpoint{4.718517in}{4.485012in}}%
\pgfpathlineto{\pgfqpoint{4.753876in}{4.485012in}}%
\pgfpathlineto{\pgfqpoint{4.753876in}{4.560873in}}%
\pgfpathlineto{\pgfqpoint{4.789236in}{4.560873in}}%
\pgfpathlineto{\pgfqpoint{4.789236in}{4.497625in}}%
\pgfpathlineto{\pgfqpoint{4.824595in}{4.497625in}}%
\pgfpathlineto{\pgfqpoint{4.824595in}{4.525927in}}%
\pgfpathlineto{\pgfqpoint{4.859955in}{4.525927in}}%
\pgfpathlineto{\pgfqpoint{4.859955in}{4.480726in}}%
\pgfpathlineto{\pgfqpoint{4.895314in}{4.480726in}}%
\pgfpathlineto{\pgfqpoint{4.895314in}{4.378537in}}%
\pgfpathlineto{\pgfqpoint{4.930673in}{4.378537in}}%
\pgfpathlineto{\pgfqpoint{4.930673in}{4.453412in}}%
\pgfpathlineto{\pgfqpoint{4.966033in}{4.453412in}}%
\pgfpathlineto{\pgfqpoint{4.966033in}{4.595158in}}%
\pgfpathlineto{\pgfqpoint{5.001392in}{4.595158in}}%
\pgfpathlineto{\pgfqpoint{5.001392in}{4.626938in}}%
\pgfpathlineto{\pgfqpoint{5.036752in}{4.626938in}}%
\pgfpathlineto{\pgfqpoint{5.036752in}{4.466863in}}%
\pgfpathlineto{\pgfqpoint{5.072111in}{4.466863in}}%
\pgfpathlineto{\pgfqpoint{5.072111in}{4.473746in}}%
\pgfpathlineto{\pgfqpoint{5.107471in}{4.473746in}}%
\pgfpathlineto{\pgfqpoint{5.107471in}{4.720729in}}%
\pgfpathlineto{\pgfqpoint{5.142830in}{4.720729in}}%
\pgfpathlineto{\pgfqpoint{5.142830in}{4.585743in}}%
\pgfpathlineto{\pgfqpoint{5.178189in}{4.585743in}}%
\pgfpathlineto{\pgfqpoint{5.178189in}{4.292220in}}%
\pgfpathlineto{\pgfqpoint{5.213549in}{4.292220in}}%
\pgfpathlineto{\pgfqpoint{5.213549in}{4.608144in}}%
\pgfpathlineto{\pgfqpoint{5.248908in}{4.608144in}}%
\pgfpathlineto{\pgfqpoint{5.248908in}{4.512262in}}%
\pgfpathlineto{\pgfqpoint{5.284268in}{4.512262in}}%
\pgfpathlineto{\pgfqpoint{5.284268in}{4.252261in}}%
\pgfpathlineto{\pgfqpoint{5.319627in}{4.252261in}}%
\pgfpathlineto{\pgfqpoint{5.319627in}{4.518959in}}%
\pgfpathlineto{\pgfqpoint{5.354987in}{4.518959in}}%
\pgfpathlineto{\pgfqpoint{5.354987in}{4.286159in}}%
\pgfpathlineto{\pgfqpoint{5.372666in}{4.286159in}}%
\pgfpathlineto{\pgfqpoint{5.372666in}{4.286159in}}%
\pgfusepath{stroke}%
\end{pgfscope}%
\begin{pgfscope}%
\pgfpathrectangle{\pgfqpoint{0.759375in}{4.164881in}}{\pgfqpoint{4.842014in}{0.691815in}}%
\pgfusepath{clip}%
\pgfsetbuttcap%
\pgfsetroundjoin%
\definecolor{currentfill}{rgb}{1.000000,0.498039,0.054902}%
\pgfsetfillcolor{currentfill}%
\pgfsetlinewidth{1.003750pt}%
\definecolor{currentstroke}{rgb}{1.000000,0.498039,0.054902}%
\pgfsetstrokecolor{currentstroke}%
\pgfsetdash{}{0pt}%
\pgfsys@defobject{currentmarker}{\pgfqpoint{-0.041667in}{-0.041667in}}{\pgfqpoint{0.041667in}{0.041667in}}{%
\pgfpathmoveto{\pgfqpoint{0.000000in}{-0.041667in}}%
\pgfpathcurveto{\pgfqpoint{0.011050in}{-0.041667in}}{\pgfqpoint{0.021649in}{-0.037276in}}{\pgfqpoint{0.029463in}{-0.029463in}}%
\pgfpathcurveto{\pgfqpoint{0.037276in}{-0.021649in}}{\pgfqpoint{0.041667in}{-0.011050in}}{\pgfqpoint{0.041667in}{0.000000in}}%
\pgfpathcurveto{\pgfqpoint{0.041667in}{0.011050in}}{\pgfqpoint{0.037276in}{0.021649in}}{\pgfqpoint{0.029463in}{0.029463in}}%
\pgfpathcurveto{\pgfqpoint{0.021649in}{0.037276in}}{\pgfqpoint{0.011050in}{0.041667in}}{\pgfqpoint{0.000000in}{0.041667in}}%
\pgfpathcurveto{\pgfqpoint{-0.011050in}{0.041667in}}{\pgfqpoint{-0.021649in}{0.037276in}}{\pgfqpoint{-0.029463in}{0.029463in}}%
\pgfpathcurveto{\pgfqpoint{-0.037276in}{0.021649in}}{\pgfqpoint{-0.041667in}{0.011050in}}{\pgfqpoint{-0.041667in}{0.000000in}}%
\pgfpathcurveto{\pgfqpoint{-0.041667in}{-0.011050in}}{\pgfqpoint{-0.037276in}{-0.021649in}}{\pgfqpoint{-0.029463in}{-0.029463in}}%
\pgfpathcurveto{\pgfqpoint{-0.021649in}{-0.037276in}}{\pgfqpoint{-0.011050in}{-0.041667in}}{\pgfqpoint{0.000000in}{-0.041667in}}%
\pgfpathclose%
\pgfusepath{stroke,fill}%
}%
\begin{pgfscope}%
\pgfsys@transformshift{0.988098in}{4.612823in}%
\pgfsys@useobject{currentmarker}{}%
\end{pgfscope}%
\begin{pgfscope}%
\pgfsys@transformshift{1.023457in}{4.342015in}%
\pgfsys@useobject{currentmarker}{}%
\end{pgfscope}%
\begin{pgfscope}%
\pgfsys@transformshift{1.058816in}{4.525058in}%
\pgfsys@useobject{currentmarker}{}%
\end{pgfscope}%
\begin{pgfscope}%
\pgfsys@transformshift{1.094176in}{4.655434in}%
\pgfsys@useobject{currentmarker}{}%
\end{pgfscope}%
\begin{pgfscope}%
\pgfsys@transformshift{1.129535in}{4.418634in}%
\pgfsys@useobject{currentmarker}{}%
\end{pgfscope}%
\begin{pgfscope}%
\pgfsys@transformshift{1.164895in}{4.611580in}%
\pgfsys@useobject{currentmarker}{}%
\end{pgfscope}%
\begin{pgfscope}%
\pgfsys@transformshift{1.200254in}{4.537789in}%
\pgfsys@useobject{currentmarker}{}%
\end{pgfscope}%
\begin{pgfscope}%
\pgfsys@transformshift{1.235614in}{4.366177in}%
\pgfsys@useobject{currentmarker}{}%
\end{pgfscope}%
\begin{pgfscope}%
\pgfsys@transformshift{1.270973in}{4.409802in}%
\pgfsys@useobject{currentmarker}{}%
\end{pgfscope}%
\begin{pgfscope}%
\pgfsys@transformshift{1.306332in}{4.314122in}%
\pgfsys@useobject{currentmarker}{}%
\end{pgfscope}%
\begin{pgfscope}%
\pgfsys@transformshift{1.341692in}{4.303470in}%
\pgfsys@useobject{currentmarker}{}%
\end{pgfscope}%
\begin{pgfscope}%
\pgfsys@transformshift{1.377051in}{4.635835in}%
\pgfsys@useobject{currentmarker}{}%
\end{pgfscope}%
\begin{pgfscope}%
\pgfsys@transformshift{1.412411in}{4.651717in}%
\pgfsys@useobject{currentmarker}{}%
\end{pgfscope}%
\begin{pgfscope}%
\pgfsys@transformshift{1.447770in}{4.449343in}%
\pgfsys@useobject{currentmarker}{}%
\end{pgfscope}%
\begin{pgfscope}%
\pgfsys@transformshift{1.483130in}{4.313245in}%
\pgfsys@useobject{currentmarker}{}%
\end{pgfscope}%
\begin{pgfscope}%
\pgfsys@transformshift{1.518489in}{4.230843in}%
\pgfsys@useobject{currentmarker}{}%
\end{pgfscope}%
\begin{pgfscope}%
\pgfsys@transformshift{1.553848in}{4.425061in}%
\pgfsys@useobject{currentmarker}{}%
\end{pgfscope}%
\begin{pgfscope}%
\pgfsys@transformshift{1.589208in}{4.438196in}%
\pgfsys@useobject{currentmarker}{}%
\end{pgfscope}%
\begin{pgfscope}%
\pgfsys@transformshift{1.624567in}{4.365454in}%
\pgfsys@useobject{currentmarker}{}%
\end{pgfscope}%
\begin{pgfscope}%
\pgfsys@transformshift{1.659927in}{4.446010in}%
\pgfsys@useobject{currentmarker}{}%
\end{pgfscope}%
\begin{pgfscope}%
\pgfsys@transformshift{1.695286in}{4.741503in}%
\pgfsys@useobject{currentmarker}{}%
\end{pgfscope}%
\begin{pgfscope}%
\pgfsys@transformshift{1.730646in}{4.461749in}%
\pgfsys@useobject{currentmarker}{}%
\end{pgfscope}%
\begin{pgfscope}%
\pgfsys@transformshift{1.766005in}{4.645485in}%
\pgfsys@useobject{currentmarker}{}%
\end{pgfscope}%
\begin{pgfscope}%
\pgfsys@transformshift{1.801364in}{4.618607in}%
\pgfsys@useobject{currentmarker}{}%
\end{pgfscope}%
\begin{pgfscope}%
\pgfsys@transformshift{1.836724in}{4.234765in}%
\pgfsys@useobject{currentmarker}{}%
\end{pgfscope}%
\begin{pgfscope}%
\pgfsys@transformshift{1.872083in}{4.477300in}%
\pgfsys@useobject{currentmarker}{}%
\end{pgfscope}%
\begin{pgfscope}%
\pgfsys@transformshift{1.907443in}{4.320838in}%
\pgfsys@useobject{currentmarker}{}%
\end{pgfscope}%
\begin{pgfscope}%
\pgfsys@transformshift{1.942802in}{4.726977in}%
\pgfsys@useobject{currentmarker}{}%
\end{pgfscope}%
\begin{pgfscope}%
\pgfsys@transformshift{1.978161in}{4.762795in}%
\pgfsys@useobject{currentmarker}{}%
\end{pgfscope}%
\begin{pgfscope}%
\pgfsys@transformshift{2.013521in}{4.474019in}%
\pgfsys@useobject{currentmarker}{}%
\end{pgfscope}%
\begin{pgfscope}%
\pgfsys@transformshift{2.048880in}{4.812848in}%
\pgfsys@useobject{currentmarker}{}%
\end{pgfscope}%
\begin{pgfscope}%
\pgfsys@transformshift{2.084240in}{4.505128in}%
\pgfsys@useobject{currentmarker}{}%
\end{pgfscope}%
\begin{pgfscope}%
\pgfsys@transformshift{2.119599in}{4.676645in}%
\pgfsys@useobject{currentmarker}{}%
\end{pgfscope}%
\begin{pgfscope}%
\pgfsys@transformshift{2.154959in}{4.371727in}%
\pgfsys@useobject{currentmarker}{}%
\end{pgfscope}%
\begin{pgfscope}%
\pgfsys@transformshift{2.190318in}{4.395113in}%
\pgfsys@useobject{currentmarker}{}%
\end{pgfscope}%
\begin{pgfscope}%
\pgfsys@transformshift{2.225677in}{4.690557in}%
\pgfsys@useobject{currentmarker}{}%
\end{pgfscope}%
\begin{pgfscope}%
\pgfsys@transformshift{2.261037in}{4.400103in}%
\pgfsys@useobject{currentmarker}{}%
\end{pgfscope}%
\begin{pgfscope}%
\pgfsys@transformshift{2.296396in}{4.443152in}%
\pgfsys@useobject{currentmarker}{}%
\end{pgfscope}%
\begin{pgfscope}%
\pgfsys@transformshift{2.331756in}{4.515329in}%
\pgfsys@useobject{currentmarker}{}%
\end{pgfscope}%
\begin{pgfscope}%
\pgfsys@transformshift{2.367115in}{4.576077in}%
\pgfsys@useobject{currentmarker}{}%
\end{pgfscope}%
\begin{pgfscope}%
\pgfsys@transformshift{2.402475in}{4.274239in}%
\pgfsys@useobject{currentmarker}{}%
\end{pgfscope}%
\begin{pgfscope}%
\pgfsys@transformshift{2.437834in}{4.449230in}%
\pgfsys@useobject{currentmarker}{}%
\end{pgfscope}%
\begin{pgfscope}%
\pgfsys@transformshift{2.473193in}{4.467345in}%
\pgfsys@useobject{currentmarker}{}%
\end{pgfscope}%
\begin{pgfscope}%
\pgfsys@transformshift{2.508553in}{4.687298in}%
\pgfsys@useobject{currentmarker}{}%
\end{pgfscope}%
\begin{pgfscope}%
\pgfsys@transformshift{2.543912in}{4.294721in}%
\pgfsys@useobject{currentmarker}{}%
\end{pgfscope}%
\begin{pgfscope}%
\pgfsys@transformshift{2.579272in}{4.458073in}%
\pgfsys@useobject{currentmarker}{}%
\end{pgfscope}%
\begin{pgfscope}%
\pgfsys@transformshift{2.614631in}{4.769474in}%
\pgfsys@useobject{currentmarker}{}%
\end{pgfscope}%
\begin{pgfscope}%
\pgfsys@transformshift{2.649991in}{4.379586in}%
\pgfsys@useobject{currentmarker}{}%
\end{pgfscope}%
\begin{pgfscope}%
\pgfsys@transformshift{2.685350in}{4.341308in}%
\pgfsys@useobject{currentmarker}{}%
\end{pgfscope}%
\begin{pgfscope}%
\pgfsys@transformshift{2.720709in}{4.493692in}%
\pgfsys@useobject{currentmarker}{}%
\end{pgfscope}%
\begin{pgfscope}%
\pgfsys@transformshift{2.756069in}{4.458276in}%
\pgfsys@useobject{currentmarker}{}%
\end{pgfscope}%
\begin{pgfscope}%
\pgfsys@transformshift{2.791428in}{4.436234in}%
\pgfsys@useobject{currentmarker}{}%
\end{pgfscope}%
\begin{pgfscope}%
\pgfsys@transformshift{2.826788in}{4.363483in}%
\pgfsys@useobject{currentmarker}{}%
\end{pgfscope}%
\begin{pgfscope}%
\pgfsys@transformshift{2.862147in}{4.410004in}%
\pgfsys@useobject{currentmarker}{}%
\end{pgfscope}%
\begin{pgfscope}%
\pgfsys@transformshift{2.897507in}{4.605228in}%
\pgfsys@useobject{currentmarker}{}%
\end{pgfscope}%
\begin{pgfscope}%
\pgfsys@transformshift{2.932866in}{4.551649in}%
\pgfsys@useobject{currentmarker}{}%
\end{pgfscope}%
\begin{pgfscope}%
\pgfsys@transformshift{2.968225in}{4.582049in}%
\pgfsys@useobject{currentmarker}{}%
\end{pgfscope}%
\begin{pgfscope}%
\pgfsys@transformshift{3.003585in}{4.439149in}%
\pgfsys@useobject{currentmarker}{}%
\end{pgfscope}%
\begin{pgfscope}%
\pgfsys@transformshift{3.038944in}{4.589145in}%
\pgfsys@useobject{currentmarker}{}%
\end{pgfscope}%
\begin{pgfscope}%
\pgfsys@transformshift{3.074304in}{4.475914in}%
\pgfsys@useobject{currentmarker}{}%
\end{pgfscope}%
\begin{pgfscope}%
\pgfsys@transformshift{3.109663in}{4.602851in}%
\pgfsys@useobject{currentmarker}{}%
\end{pgfscope}%
\begin{pgfscope}%
\pgfsys@transformshift{3.145023in}{4.708926in}%
\pgfsys@useobject{currentmarker}{}%
\end{pgfscope}%
\begin{pgfscope}%
\pgfsys@transformshift{3.180382in}{4.433795in}%
\pgfsys@useobject{currentmarker}{}%
\end{pgfscope}%
\begin{pgfscope}%
\pgfsys@transformshift{3.215741in}{4.555799in}%
\pgfsys@useobject{currentmarker}{}%
\end{pgfscope}%
\begin{pgfscope}%
\pgfsys@transformshift{3.251101in}{4.488624in}%
\pgfsys@useobject{currentmarker}{}%
\end{pgfscope}%
\begin{pgfscope}%
\pgfsys@transformshift{3.286460in}{4.703248in}%
\pgfsys@useobject{currentmarker}{}%
\end{pgfscope}%
\begin{pgfscope}%
\pgfsys@transformshift{3.321820in}{4.302053in}%
\pgfsys@useobject{currentmarker}{}%
\end{pgfscope}%
\begin{pgfscope}%
\pgfsys@transformshift{3.357179in}{4.491431in}%
\pgfsys@useobject{currentmarker}{}%
\end{pgfscope}%
\begin{pgfscope}%
\pgfsys@transformshift{3.392538in}{4.196327in}%
\pgfsys@useobject{currentmarker}{}%
\end{pgfscope}%
\begin{pgfscope}%
\pgfsys@transformshift{3.427898in}{4.537594in}%
\pgfsys@useobject{currentmarker}{}%
\end{pgfscope}%
\begin{pgfscope}%
\pgfsys@transformshift{3.463257in}{4.506972in}%
\pgfsys@useobject{currentmarker}{}%
\end{pgfscope}%
\begin{pgfscope}%
\pgfsys@transformshift{3.498617in}{4.325512in}%
\pgfsys@useobject{currentmarker}{}%
\end{pgfscope}%
\begin{pgfscope}%
\pgfsys@transformshift{3.533976in}{4.663800in}%
\pgfsys@useobject{currentmarker}{}%
\end{pgfscope}%
\begin{pgfscope}%
\pgfsys@transformshift{3.569336in}{4.714699in}%
\pgfsys@useobject{currentmarker}{}%
\end{pgfscope}%
\begin{pgfscope}%
\pgfsys@transformshift{3.604695in}{4.358987in}%
\pgfsys@useobject{currentmarker}{}%
\end{pgfscope}%
\begin{pgfscope}%
\pgfsys@transformshift{3.640054in}{4.736670in}%
\pgfsys@useobject{currentmarker}{}%
\end{pgfscope}%
\begin{pgfscope}%
\pgfsys@transformshift{3.675414in}{4.479895in}%
\pgfsys@useobject{currentmarker}{}%
\end{pgfscope}%
\begin{pgfscope}%
\pgfsys@transformshift{3.710773in}{4.582590in}%
\pgfsys@useobject{currentmarker}{}%
\end{pgfscope}%
\begin{pgfscope}%
\pgfsys@transformshift{3.746133in}{4.495574in}%
\pgfsys@useobject{currentmarker}{}%
\end{pgfscope}%
\begin{pgfscope}%
\pgfsys@transformshift{3.781492in}{4.337749in}%
\pgfsys@useobject{currentmarker}{}%
\end{pgfscope}%
\begin{pgfscope}%
\pgfsys@transformshift{3.816852in}{4.741839in}%
\pgfsys@useobject{currentmarker}{}%
\end{pgfscope}%
\begin{pgfscope}%
\pgfsys@transformshift{3.852211in}{4.586630in}%
\pgfsys@useobject{currentmarker}{}%
\end{pgfscope}%
\begin{pgfscope}%
\pgfsys@transformshift{3.887570in}{4.620765in}%
\pgfsys@useobject{currentmarker}{}%
\end{pgfscope}%
\begin{pgfscope}%
\pgfsys@transformshift{3.922930in}{4.769401in}%
\pgfsys@useobject{currentmarker}{}%
\end{pgfscope}%
\begin{pgfscope}%
\pgfsys@transformshift{3.958289in}{4.368755in}%
\pgfsys@useobject{currentmarker}{}%
\end{pgfscope}%
\begin{pgfscope}%
\pgfsys@transformshift{3.993649in}{4.393781in}%
\pgfsys@useobject{currentmarker}{}%
\end{pgfscope}%
\begin{pgfscope}%
\pgfsys@transformshift{4.029008in}{4.294250in}%
\pgfsys@useobject{currentmarker}{}%
\end{pgfscope}%
\begin{pgfscope}%
\pgfsys@transformshift{4.064368in}{4.656395in}%
\pgfsys@useobject{currentmarker}{}%
\end{pgfscope}%
\begin{pgfscope}%
\pgfsys@transformshift{4.099727in}{4.615443in}%
\pgfsys@useobject{currentmarker}{}%
\end{pgfscope}%
\begin{pgfscope}%
\pgfsys@transformshift{4.135086in}{4.692520in}%
\pgfsys@useobject{currentmarker}{}%
\end{pgfscope}%
\begin{pgfscope}%
\pgfsys@transformshift{4.170446in}{4.318325in}%
\pgfsys@useobject{currentmarker}{}%
\end{pgfscope}%
\begin{pgfscope}%
\pgfsys@transformshift{4.205805in}{4.628600in}%
\pgfsys@useobject{currentmarker}{}%
\end{pgfscope}%
\begin{pgfscope}%
\pgfsys@transformshift{4.241165in}{4.342954in}%
\pgfsys@useobject{currentmarker}{}%
\end{pgfscope}%
\begin{pgfscope}%
\pgfsys@transformshift{4.276524in}{4.475326in}%
\pgfsys@useobject{currentmarker}{}%
\end{pgfscope}%
\begin{pgfscope}%
\pgfsys@transformshift{4.311884in}{4.560362in}%
\pgfsys@useobject{currentmarker}{}%
\end{pgfscope}%
\begin{pgfscope}%
\pgfsys@transformshift{4.347243in}{4.362908in}%
\pgfsys@useobject{currentmarker}{}%
\end{pgfscope}%
\begin{pgfscope}%
\pgfsys@transformshift{4.382602in}{4.825250in}%
\pgfsys@useobject{currentmarker}{}%
\end{pgfscope}%
\begin{pgfscope}%
\pgfsys@transformshift{4.417962in}{4.395663in}%
\pgfsys@useobject{currentmarker}{}%
\end{pgfscope}%
\begin{pgfscope}%
\pgfsys@transformshift{4.453321in}{4.604156in}%
\pgfsys@useobject{currentmarker}{}%
\end{pgfscope}%
\begin{pgfscope}%
\pgfsys@transformshift{4.488681in}{4.503919in}%
\pgfsys@useobject{currentmarker}{}%
\end{pgfscope}%
\begin{pgfscope}%
\pgfsys@transformshift{4.524040in}{4.509435in}%
\pgfsys@useobject{currentmarker}{}%
\end{pgfscope}%
\begin{pgfscope}%
\pgfsys@transformshift{4.559400in}{4.560932in}%
\pgfsys@useobject{currentmarker}{}%
\end{pgfscope}%
\begin{pgfscope}%
\pgfsys@transformshift{4.594759in}{4.473255in}%
\pgfsys@useobject{currentmarker}{}%
\end{pgfscope}%
\begin{pgfscope}%
\pgfsys@transformshift{4.630118in}{4.461286in}%
\pgfsys@useobject{currentmarker}{}%
\end{pgfscope}%
\begin{pgfscope}%
\pgfsys@transformshift{4.665478in}{4.302494in}%
\pgfsys@useobject{currentmarker}{}%
\end{pgfscope}%
\begin{pgfscope}%
\pgfsys@transformshift{4.700837in}{4.746930in}%
\pgfsys@useobject{currentmarker}{}%
\end{pgfscope}%
\begin{pgfscope}%
\pgfsys@transformshift{4.736197in}{4.485012in}%
\pgfsys@useobject{currentmarker}{}%
\end{pgfscope}%
\begin{pgfscope}%
\pgfsys@transformshift{4.771556in}{4.560873in}%
\pgfsys@useobject{currentmarker}{}%
\end{pgfscope}%
\begin{pgfscope}%
\pgfsys@transformshift{4.806916in}{4.497625in}%
\pgfsys@useobject{currentmarker}{}%
\end{pgfscope}%
\begin{pgfscope}%
\pgfsys@transformshift{4.842275in}{4.525927in}%
\pgfsys@useobject{currentmarker}{}%
\end{pgfscope}%
\begin{pgfscope}%
\pgfsys@transformshift{4.877634in}{4.480726in}%
\pgfsys@useobject{currentmarker}{}%
\end{pgfscope}%
\begin{pgfscope}%
\pgfsys@transformshift{4.912994in}{4.378537in}%
\pgfsys@useobject{currentmarker}{}%
\end{pgfscope}%
\begin{pgfscope}%
\pgfsys@transformshift{4.948353in}{4.453412in}%
\pgfsys@useobject{currentmarker}{}%
\end{pgfscope}%
\begin{pgfscope}%
\pgfsys@transformshift{4.983713in}{4.595158in}%
\pgfsys@useobject{currentmarker}{}%
\end{pgfscope}%
\begin{pgfscope}%
\pgfsys@transformshift{5.019072in}{4.626938in}%
\pgfsys@useobject{currentmarker}{}%
\end{pgfscope}%
\begin{pgfscope}%
\pgfsys@transformshift{5.054431in}{4.466863in}%
\pgfsys@useobject{currentmarker}{}%
\end{pgfscope}%
\begin{pgfscope}%
\pgfsys@transformshift{5.089791in}{4.473746in}%
\pgfsys@useobject{currentmarker}{}%
\end{pgfscope}%
\begin{pgfscope}%
\pgfsys@transformshift{5.125150in}{4.720729in}%
\pgfsys@useobject{currentmarker}{}%
\end{pgfscope}%
\begin{pgfscope}%
\pgfsys@transformshift{5.160510in}{4.585743in}%
\pgfsys@useobject{currentmarker}{}%
\end{pgfscope}%
\begin{pgfscope}%
\pgfsys@transformshift{5.195869in}{4.292220in}%
\pgfsys@useobject{currentmarker}{}%
\end{pgfscope}%
\begin{pgfscope}%
\pgfsys@transformshift{5.231229in}{4.608144in}%
\pgfsys@useobject{currentmarker}{}%
\end{pgfscope}%
\begin{pgfscope}%
\pgfsys@transformshift{5.266588in}{4.512262in}%
\pgfsys@useobject{currentmarker}{}%
\end{pgfscope}%
\begin{pgfscope}%
\pgfsys@transformshift{5.301947in}{4.252261in}%
\pgfsys@useobject{currentmarker}{}%
\end{pgfscope}%
\begin{pgfscope}%
\pgfsys@transformshift{5.337307in}{4.518959in}%
\pgfsys@useobject{currentmarker}{}%
\end{pgfscope}%
\begin{pgfscope}%
\pgfsys@transformshift{5.372666in}{4.286159in}%
\pgfsys@useobject{currentmarker}{}%
\end{pgfscope}%
\end{pgfscope}%
\begin{pgfscope}%
\pgfsetrectcap%
\pgfsetmiterjoin%
\pgfsetlinewidth{0.803000pt}%
\definecolor{currentstroke}{rgb}{0.000000,0.000000,0.000000}%
\pgfsetstrokecolor{currentstroke}%
\pgfsetdash{}{0pt}%
\pgfpathmoveto{\pgfqpoint{0.759375in}{4.164881in}}%
\pgfpathlineto{\pgfqpoint{0.759375in}{4.856696in}}%
\pgfusepath{stroke}%
\end{pgfscope}%
\begin{pgfscope}%
\pgfsetrectcap%
\pgfsetmiterjoin%
\pgfsetlinewidth{0.803000pt}%
\definecolor{currentstroke}{rgb}{0.000000,0.000000,0.000000}%
\pgfsetstrokecolor{currentstroke}%
\pgfsetdash{}{0pt}%
\pgfpathmoveto{\pgfqpoint{5.601389in}{4.164881in}}%
\pgfpathlineto{\pgfqpoint{5.601389in}{4.856696in}}%
\pgfusepath{stroke}%
\end{pgfscope}%
\begin{pgfscope}%
\pgfsetrectcap%
\pgfsetmiterjoin%
\pgfsetlinewidth{0.803000pt}%
\definecolor{currentstroke}{rgb}{0.000000,0.000000,0.000000}%
\pgfsetstrokecolor{currentstroke}%
\pgfsetdash{}{0pt}%
\pgfpathmoveto{\pgfqpoint{0.759375in}{4.164881in}}%
\pgfpathlineto{\pgfqpoint{5.601389in}{4.164881in}}%
\pgfusepath{stroke}%
\end{pgfscope}%
\begin{pgfscope}%
\pgfsetrectcap%
\pgfsetmiterjoin%
\pgfsetlinewidth{0.803000pt}%
\definecolor{currentstroke}{rgb}{0.000000,0.000000,0.000000}%
\pgfsetstrokecolor{currentstroke}%
\pgfsetdash{}{0pt}%
\pgfpathmoveto{\pgfqpoint{0.759375in}{4.856696in}}%
\pgfpathlineto{\pgfqpoint{5.601389in}{4.856696in}}%
\pgfusepath{stroke}%
\end{pgfscope}%
\begin{pgfscope}%
\pgfsetbuttcap%
\pgfsetmiterjoin%
\definecolor{currentfill}{rgb}{1.000000,1.000000,1.000000}%
\pgfsetfillcolor{currentfill}%
\pgfsetlinewidth{0.000000pt}%
\definecolor{currentstroke}{rgb}{0.000000,0.000000,0.000000}%
\pgfsetstrokecolor{currentstroke}%
\pgfsetstrokeopacity{0.000000}%
\pgfsetdash{}{0pt}%
\pgfpathmoveto{\pgfqpoint{0.759375in}{3.220189in}}%
\pgfpathlineto{\pgfqpoint{5.601389in}{3.220189in}}%
\pgfpathlineto{\pgfqpoint{5.601389in}{3.912004in}}%
\pgfpathlineto{\pgfqpoint{0.759375in}{3.912004in}}%
\pgfpathclose%
\pgfusepath{fill}%
\end{pgfscope}%
\begin{pgfscope}%
\pgfsetbuttcap%
\pgfsetroundjoin%
\definecolor{currentfill}{rgb}{0.000000,0.000000,0.000000}%
\pgfsetfillcolor{currentfill}%
\pgfsetlinewidth{0.803000pt}%
\definecolor{currentstroke}{rgb}{0.000000,0.000000,0.000000}%
\pgfsetstrokecolor{currentstroke}%
\pgfsetdash{}{0pt}%
\pgfsys@defobject{currentmarker}{\pgfqpoint{0.000000in}{-0.048611in}}{\pgfqpoint{0.000000in}{0.000000in}}{%
\pgfpathmoveto{\pgfqpoint{0.000000in}{0.000000in}}%
\pgfpathlineto{\pgfqpoint{0.000000in}{-0.048611in}}%
\pgfusepath{stroke,fill}%
}%
\begin{pgfscope}%
\pgfsys@transformshift{0.979467in}{3.220189in}%
\pgfsys@useobject{currentmarker}{}%
\end{pgfscope}%
\end{pgfscope}%
\begin{pgfscope}%
\pgfsetbuttcap%
\pgfsetroundjoin%
\definecolor{currentfill}{rgb}{0.000000,0.000000,0.000000}%
\pgfsetfillcolor{currentfill}%
\pgfsetlinewidth{0.803000pt}%
\definecolor{currentstroke}{rgb}{0.000000,0.000000,0.000000}%
\pgfsetstrokecolor{currentstroke}%
\pgfsetdash{}{0pt}%
\pgfsys@defobject{currentmarker}{\pgfqpoint{0.000000in}{-0.048611in}}{\pgfqpoint{0.000000in}{0.000000in}}{%
\pgfpathmoveto{\pgfqpoint{0.000000in}{0.000000in}}%
\pgfpathlineto{\pgfqpoint{0.000000in}{-0.048611in}}%
\pgfusepath{stroke,fill}%
}%
\begin{pgfscope}%
\pgfsys@transformshift{1.529695in}{3.220189in}%
\pgfsys@useobject{currentmarker}{}%
\end{pgfscope}%
\end{pgfscope}%
\begin{pgfscope}%
\pgfsetbuttcap%
\pgfsetroundjoin%
\definecolor{currentfill}{rgb}{0.000000,0.000000,0.000000}%
\pgfsetfillcolor{currentfill}%
\pgfsetlinewidth{0.803000pt}%
\definecolor{currentstroke}{rgb}{0.000000,0.000000,0.000000}%
\pgfsetstrokecolor{currentstroke}%
\pgfsetdash{}{0pt}%
\pgfsys@defobject{currentmarker}{\pgfqpoint{0.000000in}{-0.048611in}}{\pgfqpoint{0.000000in}{0.000000in}}{%
\pgfpathmoveto{\pgfqpoint{0.000000in}{0.000000in}}%
\pgfpathlineto{\pgfqpoint{0.000000in}{-0.048611in}}%
\pgfusepath{stroke,fill}%
}%
\begin{pgfscope}%
\pgfsys@transformshift{2.079924in}{3.220189in}%
\pgfsys@useobject{currentmarker}{}%
\end{pgfscope}%
\end{pgfscope}%
\begin{pgfscope}%
\pgfsetbuttcap%
\pgfsetroundjoin%
\definecolor{currentfill}{rgb}{0.000000,0.000000,0.000000}%
\pgfsetfillcolor{currentfill}%
\pgfsetlinewidth{0.803000pt}%
\definecolor{currentstroke}{rgb}{0.000000,0.000000,0.000000}%
\pgfsetstrokecolor{currentstroke}%
\pgfsetdash{}{0pt}%
\pgfsys@defobject{currentmarker}{\pgfqpoint{0.000000in}{-0.048611in}}{\pgfqpoint{0.000000in}{0.000000in}}{%
\pgfpathmoveto{\pgfqpoint{0.000000in}{0.000000in}}%
\pgfpathlineto{\pgfqpoint{0.000000in}{-0.048611in}}%
\pgfusepath{stroke,fill}%
}%
\begin{pgfscope}%
\pgfsys@transformshift{2.630153in}{3.220189in}%
\pgfsys@useobject{currentmarker}{}%
\end{pgfscope}%
\end{pgfscope}%
\begin{pgfscope}%
\pgfsetbuttcap%
\pgfsetroundjoin%
\definecolor{currentfill}{rgb}{0.000000,0.000000,0.000000}%
\pgfsetfillcolor{currentfill}%
\pgfsetlinewidth{0.803000pt}%
\definecolor{currentstroke}{rgb}{0.000000,0.000000,0.000000}%
\pgfsetstrokecolor{currentstroke}%
\pgfsetdash{}{0pt}%
\pgfsys@defobject{currentmarker}{\pgfqpoint{0.000000in}{-0.048611in}}{\pgfqpoint{0.000000in}{0.000000in}}{%
\pgfpathmoveto{\pgfqpoint{0.000000in}{0.000000in}}%
\pgfpathlineto{\pgfqpoint{0.000000in}{-0.048611in}}%
\pgfusepath{stroke,fill}%
}%
\begin{pgfscope}%
\pgfsys@transformshift{3.180382in}{3.220189in}%
\pgfsys@useobject{currentmarker}{}%
\end{pgfscope}%
\end{pgfscope}%
\begin{pgfscope}%
\pgfsetbuttcap%
\pgfsetroundjoin%
\definecolor{currentfill}{rgb}{0.000000,0.000000,0.000000}%
\pgfsetfillcolor{currentfill}%
\pgfsetlinewidth{0.803000pt}%
\definecolor{currentstroke}{rgb}{0.000000,0.000000,0.000000}%
\pgfsetstrokecolor{currentstroke}%
\pgfsetdash{}{0pt}%
\pgfsys@defobject{currentmarker}{\pgfqpoint{0.000000in}{-0.048611in}}{\pgfqpoint{0.000000in}{0.000000in}}{%
\pgfpathmoveto{\pgfqpoint{0.000000in}{0.000000in}}%
\pgfpathlineto{\pgfqpoint{0.000000in}{-0.048611in}}%
\pgfusepath{stroke,fill}%
}%
\begin{pgfscope}%
\pgfsys@transformshift{3.730611in}{3.220189in}%
\pgfsys@useobject{currentmarker}{}%
\end{pgfscope}%
\end{pgfscope}%
\begin{pgfscope}%
\pgfsetbuttcap%
\pgfsetroundjoin%
\definecolor{currentfill}{rgb}{0.000000,0.000000,0.000000}%
\pgfsetfillcolor{currentfill}%
\pgfsetlinewidth{0.803000pt}%
\definecolor{currentstroke}{rgb}{0.000000,0.000000,0.000000}%
\pgfsetstrokecolor{currentstroke}%
\pgfsetdash{}{0pt}%
\pgfsys@defobject{currentmarker}{\pgfqpoint{0.000000in}{-0.048611in}}{\pgfqpoint{0.000000in}{0.000000in}}{%
\pgfpathmoveto{\pgfqpoint{0.000000in}{0.000000in}}%
\pgfpathlineto{\pgfqpoint{0.000000in}{-0.048611in}}%
\pgfusepath{stroke,fill}%
}%
\begin{pgfscope}%
\pgfsys@transformshift{4.280840in}{3.220189in}%
\pgfsys@useobject{currentmarker}{}%
\end{pgfscope}%
\end{pgfscope}%
\begin{pgfscope}%
\pgfsetbuttcap%
\pgfsetroundjoin%
\definecolor{currentfill}{rgb}{0.000000,0.000000,0.000000}%
\pgfsetfillcolor{currentfill}%
\pgfsetlinewidth{0.803000pt}%
\definecolor{currentstroke}{rgb}{0.000000,0.000000,0.000000}%
\pgfsetstrokecolor{currentstroke}%
\pgfsetdash{}{0pt}%
\pgfsys@defobject{currentmarker}{\pgfqpoint{0.000000in}{-0.048611in}}{\pgfqpoint{0.000000in}{0.000000in}}{%
\pgfpathmoveto{\pgfqpoint{0.000000in}{0.000000in}}%
\pgfpathlineto{\pgfqpoint{0.000000in}{-0.048611in}}%
\pgfusepath{stroke,fill}%
}%
\begin{pgfscope}%
\pgfsys@transformshift{4.831068in}{3.220189in}%
\pgfsys@useobject{currentmarker}{}%
\end{pgfscope}%
\end{pgfscope}%
\begin{pgfscope}%
\pgfsetbuttcap%
\pgfsetroundjoin%
\definecolor{currentfill}{rgb}{0.000000,0.000000,0.000000}%
\pgfsetfillcolor{currentfill}%
\pgfsetlinewidth{0.803000pt}%
\definecolor{currentstroke}{rgb}{0.000000,0.000000,0.000000}%
\pgfsetstrokecolor{currentstroke}%
\pgfsetdash{}{0pt}%
\pgfsys@defobject{currentmarker}{\pgfqpoint{0.000000in}{-0.048611in}}{\pgfqpoint{0.000000in}{0.000000in}}{%
\pgfpathmoveto{\pgfqpoint{0.000000in}{0.000000in}}%
\pgfpathlineto{\pgfqpoint{0.000000in}{-0.048611in}}%
\pgfusepath{stroke,fill}%
}%
\begin{pgfscope}%
\pgfsys@transformshift{5.381297in}{3.220189in}%
\pgfsys@useobject{currentmarker}{}%
\end{pgfscope}%
\end{pgfscope}%
\begin{pgfscope}%
\pgfsetbuttcap%
\pgfsetroundjoin%
\definecolor{currentfill}{rgb}{0.000000,0.000000,0.000000}%
\pgfsetfillcolor{currentfill}%
\pgfsetlinewidth{0.803000pt}%
\definecolor{currentstroke}{rgb}{0.000000,0.000000,0.000000}%
\pgfsetstrokecolor{currentstroke}%
\pgfsetdash{}{0pt}%
\pgfsys@defobject{currentmarker}{\pgfqpoint{-0.048611in}{0.000000in}}{\pgfqpoint{0.000000in}{0.000000in}}{%
\pgfpathmoveto{\pgfqpoint{0.000000in}{0.000000in}}%
\pgfpathlineto{\pgfqpoint{-0.048611in}{0.000000in}}%
\pgfusepath{stroke,fill}%
}%
\begin{pgfscope}%
\pgfsys@transformshift{0.759375in}{3.349888in}%
\pgfsys@useobject{currentmarker}{}%
\end{pgfscope}%
\end{pgfscope}%
\begin{pgfscope}%
\definecolor{textcolor}{rgb}{0.000000,0.000000,0.000000}%
\pgfsetstrokecolor{textcolor}%
\pgfsetfillcolor{textcolor}%
\pgftext[x=0.237708in,y=3.301693in,left,base]{\color{textcolor}\rmfamily\fontsize{10.000000}{12.000000}\selectfont −0.025}%
\end{pgfscope}%
\begin{pgfscope}%
\pgfsetbuttcap%
\pgfsetroundjoin%
\definecolor{currentfill}{rgb}{0.000000,0.000000,0.000000}%
\pgfsetfillcolor{currentfill}%
\pgfsetlinewidth{0.803000pt}%
\definecolor{currentstroke}{rgb}{0.000000,0.000000,0.000000}%
\pgfsetstrokecolor{currentstroke}%
\pgfsetdash{}{0pt}%
\pgfsys@defobject{currentmarker}{\pgfqpoint{-0.048611in}{0.000000in}}{\pgfqpoint{0.000000in}{0.000000in}}{%
\pgfpathmoveto{\pgfqpoint{0.000000in}{0.000000in}}%
\pgfpathlineto{\pgfqpoint{-0.048611in}{0.000000in}}%
\pgfusepath{stroke,fill}%
}%
\begin{pgfscope}%
\pgfsys@transformshift{0.759375in}{3.590604in}%
\pgfsys@useobject{currentmarker}{}%
\end{pgfscope}%
\end{pgfscope}%
\begin{pgfscope}%
\definecolor{textcolor}{rgb}{0.000000,0.000000,0.000000}%
\pgfsetstrokecolor{textcolor}%
\pgfsetfillcolor{textcolor}%
\pgftext[x=0.345764in,y=3.542410in,left,base]{\color{textcolor}\rmfamily\fontsize{10.000000}{12.000000}\selectfont 0.000}%
\end{pgfscope}%
\begin{pgfscope}%
\pgfsetbuttcap%
\pgfsetroundjoin%
\definecolor{currentfill}{rgb}{0.000000,0.000000,0.000000}%
\pgfsetfillcolor{currentfill}%
\pgfsetlinewidth{0.803000pt}%
\definecolor{currentstroke}{rgb}{0.000000,0.000000,0.000000}%
\pgfsetstrokecolor{currentstroke}%
\pgfsetdash{}{0pt}%
\pgfsys@defobject{currentmarker}{\pgfqpoint{-0.048611in}{0.000000in}}{\pgfqpoint{0.000000in}{0.000000in}}{%
\pgfpathmoveto{\pgfqpoint{0.000000in}{0.000000in}}%
\pgfpathlineto{\pgfqpoint{-0.048611in}{0.000000in}}%
\pgfusepath{stroke,fill}%
}%
\begin{pgfscope}%
\pgfsys@transformshift{0.759375in}{3.831321in}%
\pgfsys@useobject{currentmarker}{}%
\end{pgfscope}%
\end{pgfscope}%
\begin{pgfscope}%
\definecolor{textcolor}{rgb}{0.000000,0.000000,0.000000}%
\pgfsetstrokecolor{textcolor}%
\pgfsetfillcolor{textcolor}%
\pgftext[x=0.345764in,y=3.783126in,left,base]{\color{textcolor}\rmfamily\fontsize{10.000000}{12.000000}\selectfont 0.025}%
\end{pgfscope}%
\begin{pgfscope}%
\pgfpathrectangle{\pgfqpoint{0.759375in}{3.220189in}}{\pgfqpoint{4.842014in}{0.691815in}}%
\pgfusepath{clip}%
\pgfsetrectcap%
\pgfsetroundjoin%
\pgfsetlinewidth{1.505625pt}%
\definecolor{currentstroke}{rgb}{1.000000,0.498039,0.054902}%
\pgfsetstrokecolor{currentstroke}%
\pgfsetdash{}{0pt}%
\pgfpathmoveto{\pgfqpoint{1.005777in}{3.391662in}}%
\pgfpathlineto{\pgfqpoint{1.041427in}{3.391662in}}%
\pgfpathlineto{\pgfqpoint{1.041427in}{3.373013in}}%
\pgfpathlineto{\pgfqpoint{1.112725in}{3.373013in}}%
\pgfpathlineto{\pgfqpoint{1.112725in}{3.374902in}}%
\pgfpathlineto{\pgfqpoint{1.184024in}{3.374902in}}%
\pgfpathlineto{\pgfqpoint{1.184024in}{3.734717in}}%
\pgfpathlineto{\pgfqpoint{1.255322in}{3.734717in}}%
\pgfpathlineto{\pgfqpoint{1.255322in}{3.534121in}}%
\pgfpathlineto{\pgfqpoint{1.326621in}{3.534121in}}%
\pgfpathlineto{\pgfqpoint{1.326621in}{3.562295in}}%
\pgfpathlineto{\pgfqpoint{1.397919in}{3.562295in}}%
\pgfpathlineto{\pgfqpoint{1.397919in}{3.679523in}}%
\pgfpathlineto{\pgfqpoint{1.469218in}{3.679523in}}%
\pgfpathlineto{\pgfqpoint{1.469218in}{3.584103in}}%
\pgfpathlineto{\pgfqpoint{1.540516in}{3.584103in}}%
\pgfpathlineto{\pgfqpoint{1.540516in}{3.830209in}}%
\pgfpathlineto{\pgfqpoint{1.611815in}{3.830209in}}%
\pgfpathlineto{\pgfqpoint{1.611815in}{3.545463in}}%
\pgfpathlineto{\pgfqpoint{1.683113in}{3.545463in}}%
\pgfpathlineto{\pgfqpoint{1.683113in}{3.709537in}}%
\pgfpathlineto{\pgfqpoint{1.754412in}{3.709537in}}%
\pgfpathlineto{\pgfqpoint{1.754412in}{3.711821in}}%
\pgfpathlineto{\pgfqpoint{1.825710in}{3.711821in}}%
\pgfpathlineto{\pgfqpoint{1.825710in}{3.325357in}}%
\pgfpathlineto{\pgfqpoint{1.897009in}{3.325357in}}%
\pgfpathlineto{\pgfqpoint{1.897009in}{3.594810in}}%
\pgfpathlineto{\pgfqpoint{1.968307in}{3.594810in}}%
\pgfpathlineto{\pgfqpoint{1.968307in}{3.471188in}}%
\pgfpathlineto{\pgfqpoint{2.039606in}{3.471188in}}%
\pgfpathlineto{\pgfqpoint{2.039606in}{3.509119in}}%
\pgfpathlineto{\pgfqpoint{2.110904in}{3.509119in}}%
\pgfpathlineto{\pgfqpoint{2.110904in}{3.768842in}}%
\pgfpathlineto{\pgfqpoint{2.182203in}{3.768842in}}%
\pgfpathlineto{\pgfqpoint{2.182203in}{3.766371in}}%
\pgfpathlineto{\pgfqpoint{2.253501in}{3.766371in}}%
\pgfpathlineto{\pgfqpoint{2.253501in}{3.399801in}}%
\pgfpathlineto{\pgfqpoint{2.324800in}{3.399801in}}%
\pgfpathlineto{\pgfqpoint{2.324800in}{3.783343in}}%
\pgfpathlineto{\pgfqpoint{2.396098in}{3.783343in}}%
\pgfpathlineto{\pgfqpoint{2.396098in}{3.479092in}}%
\pgfpathlineto{\pgfqpoint{2.467397in}{3.479092in}}%
\pgfpathlineto{\pgfqpoint{2.467397in}{3.614306in}}%
\pgfpathlineto{\pgfqpoint{2.538695in}{3.614306in}}%
\pgfpathlineto{\pgfqpoint{2.538695in}{3.598177in}}%
\pgfpathlineto{\pgfqpoint{2.609994in}{3.598177in}}%
\pgfpathlineto{\pgfqpoint{2.609994in}{3.636778in}}%
\pgfpathlineto{\pgfqpoint{2.681292in}{3.636778in}}%
\pgfpathlineto{\pgfqpoint{2.681292in}{3.442338in}}%
\pgfpathlineto{\pgfqpoint{2.752591in}{3.442338in}}%
\pgfpathlineto{\pgfqpoint{2.752591in}{3.723791in}}%
\pgfpathlineto{\pgfqpoint{2.823889in}{3.723791in}}%
\pgfpathlineto{\pgfqpoint{2.823889in}{3.614965in}}%
\pgfpathlineto{\pgfqpoint{2.895188in}{3.614965in}}%
\pgfpathlineto{\pgfqpoint{2.895188in}{3.351282in}}%
\pgfpathlineto{\pgfqpoint{2.966486in}{3.351282in}}%
\pgfpathlineto{\pgfqpoint{2.966486in}{3.780196in}}%
\pgfpathlineto{\pgfqpoint{3.037785in}{3.780196in}}%
\pgfpathlineto{\pgfqpoint{3.037785in}{3.429245in}}%
\pgfpathlineto{\pgfqpoint{3.109083in}{3.429245in}}%
\pgfpathlineto{\pgfqpoint{3.109083in}{3.612354in}}%
\pgfpathlineto{\pgfqpoint{3.180382in}{3.612354in}}%
\pgfpathlineto{\pgfqpoint{3.180382in}{3.880558in}}%
\pgfpathlineto{\pgfqpoint{3.251680in}{3.880558in}}%
\pgfpathlineto{\pgfqpoint{3.251680in}{3.525053in}}%
\pgfpathlineto{\pgfqpoint{3.322979in}{3.525053in}}%
\pgfpathlineto{\pgfqpoint{3.322979in}{3.663245in}}%
\pgfpathlineto{\pgfqpoint{3.394277in}{3.663245in}}%
\pgfpathlineto{\pgfqpoint{3.394277in}{3.745132in}}%
\pgfpathlineto{\pgfqpoint{3.465576in}{3.745132in}}%
\pgfpathlineto{\pgfqpoint{3.465576in}{3.543473in}}%
\pgfpathlineto{\pgfqpoint{3.536875in}{3.543473in}}%
\pgfpathlineto{\pgfqpoint{3.536875in}{3.447882in}}%
\pgfpathlineto{\pgfqpoint{3.608173in}{3.447882in}}%
\pgfpathlineto{\pgfqpoint{3.608173in}{3.410656in}}%
\pgfpathlineto{\pgfqpoint{3.679472in}{3.410656in}}%
\pgfpathlineto{\pgfqpoint{3.679472in}{3.392146in}}%
\pgfpathlineto{\pgfqpoint{3.750770in}{3.392146in}}%
\pgfpathlineto{\pgfqpoint{3.750770in}{3.478844in}}%
\pgfpathlineto{\pgfqpoint{3.822069in}{3.478844in}}%
\pgfpathlineto{\pgfqpoint{3.822069in}{3.861889in}}%
\pgfpathlineto{\pgfqpoint{3.893367in}{3.861889in}}%
\pgfpathlineto{\pgfqpoint{3.893367in}{3.460176in}}%
\pgfpathlineto{\pgfqpoint{3.964666in}{3.460176in}}%
\pgfpathlineto{\pgfqpoint{3.964666in}{3.333327in}}%
\pgfpathlineto{\pgfqpoint{4.035964in}{3.333327in}}%
\pgfpathlineto{\pgfqpoint{4.035964in}{3.625697in}}%
\pgfpathlineto{\pgfqpoint{4.107263in}{3.625697in}}%
\pgfpathlineto{\pgfqpoint{4.107263in}{3.368084in}}%
\pgfpathlineto{\pgfqpoint{4.178561in}{3.368084in}}%
\pgfpathlineto{\pgfqpoint{4.178561in}{3.654081in}}%
\pgfpathlineto{\pgfqpoint{4.249860in}{3.654081in}}%
\pgfpathlineto{\pgfqpoint{4.249860in}{3.251635in}}%
\pgfpathlineto{\pgfqpoint{4.321158in}{3.251635in}}%
\pgfpathlineto{\pgfqpoint{4.321158in}{3.407957in}}%
\pgfpathlineto{\pgfqpoint{4.392457in}{3.407957in}}%
\pgfpathlineto{\pgfqpoint{4.392457in}{3.666425in}}%
\pgfpathlineto{\pgfqpoint{4.463755in}{3.666425in}}%
\pgfpathlineto{\pgfqpoint{4.463755in}{3.545169in}}%
\pgfpathlineto{\pgfqpoint{4.535054in}{3.545169in}}%
\pgfpathlineto{\pgfqpoint{4.535054in}{3.667987in}}%
\pgfpathlineto{\pgfqpoint{4.606352in}{3.667987in}}%
\pgfpathlineto{\pgfqpoint{4.606352in}{3.560569in}}%
\pgfpathlineto{\pgfqpoint{4.677651in}{3.560569in}}%
\pgfpathlineto{\pgfqpoint{4.677651in}{3.547180in}}%
\pgfpathlineto{\pgfqpoint{4.748949in}{3.547180in}}%
\pgfpathlineto{\pgfqpoint{4.748949in}{3.357873in}}%
\pgfpathlineto{\pgfqpoint{4.820248in}{3.357873in}}%
\pgfpathlineto{\pgfqpoint{4.820248in}{3.473717in}}%
\pgfpathlineto{\pgfqpoint{4.891546in}{3.473717in}}%
\pgfpathlineto{\pgfqpoint{4.891546in}{3.597955in}}%
\pgfpathlineto{\pgfqpoint{4.962845in}{3.597955in}}%
\pgfpathlineto{\pgfqpoint{4.962845in}{3.868054in}}%
\pgfpathlineto{\pgfqpoint{5.034143in}{3.868054in}}%
\pgfpathlineto{\pgfqpoint{5.034143in}{3.580220in}}%
\pgfpathlineto{\pgfqpoint{5.105442in}{3.580220in}}%
\pgfpathlineto{\pgfqpoint{5.105442in}{3.495275in}}%
\pgfpathlineto{\pgfqpoint{5.176740in}{3.495275in}}%
\pgfpathlineto{\pgfqpoint{5.176740in}{3.533702in}}%
\pgfpathlineto{\pgfqpoint{5.248039in}{3.533702in}}%
\pgfpathlineto{\pgfqpoint{5.248039in}{3.601650in}}%
\pgfpathlineto{\pgfqpoint{5.319337in}{3.601650in}}%
\pgfpathlineto{\pgfqpoint{5.319337in}{3.651986in}}%
\pgfpathlineto{\pgfqpoint{5.354987in}{3.651986in}}%
\pgfusepath{stroke}%
\end{pgfscope}%
\begin{pgfscope}%
\pgfpathrectangle{\pgfqpoint{0.759375in}{3.220189in}}{\pgfqpoint{4.842014in}{0.691815in}}%
\pgfusepath{clip}%
\pgfsetbuttcap%
\pgfsetroundjoin%
\definecolor{currentfill}{rgb}{1.000000,0.498039,0.054902}%
\pgfsetfillcolor{currentfill}%
\pgfsetlinewidth{1.003750pt}%
\definecolor{currentstroke}{rgb}{1.000000,0.498039,0.054902}%
\pgfsetstrokecolor{currentstroke}%
\pgfsetdash{}{0pt}%
\pgfsys@defobject{currentmarker}{\pgfqpoint{-0.041667in}{-0.041667in}}{\pgfqpoint{0.041667in}{0.041667in}}{%
\pgfpathmoveto{\pgfqpoint{0.000000in}{-0.041667in}}%
\pgfpathcurveto{\pgfqpoint{0.011050in}{-0.041667in}}{\pgfqpoint{0.021649in}{-0.037276in}}{\pgfqpoint{0.029463in}{-0.029463in}}%
\pgfpathcurveto{\pgfqpoint{0.037276in}{-0.021649in}}{\pgfqpoint{0.041667in}{-0.011050in}}{\pgfqpoint{0.041667in}{0.000000in}}%
\pgfpathcurveto{\pgfqpoint{0.041667in}{0.011050in}}{\pgfqpoint{0.037276in}{0.021649in}}{\pgfqpoint{0.029463in}{0.029463in}}%
\pgfpathcurveto{\pgfqpoint{0.021649in}{0.037276in}}{\pgfqpoint{0.011050in}{0.041667in}}{\pgfqpoint{0.000000in}{0.041667in}}%
\pgfpathcurveto{\pgfqpoint{-0.011050in}{0.041667in}}{\pgfqpoint{-0.021649in}{0.037276in}}{\pgfqpoint{-0.029463in}{0.029463in}}%
\pgfpathcurveto{\pgfqpoint{-0.037276in}{0.021649in}}{\pgfqpoint{-0.041667in}{0.011050in}}{\pgfqpoint{-0.041667in}{0.000000in}}%
\pgfpathcurveto{\pgfqpoint{-0.041667in}{-0.011050in}}{\pgfqpoint{-0.037276in}{-0.021649in}}{\pgfqpoint{-0.029463in}{-0.029463in}}%
\pgfpathcurveto{\pgfqpoint{-0.021649in}{-0.037276in}}{\pgfqpoint{-0.011050in}{-0.041667in}}{\pgfqpoint{0.000000in}{-0.041667in}}%
\pgfpathclose%
\pgfusepath{stroke,fill}%
}%
\begin{pgfscope}%
\pgfsys@transformshift{1.005777in}{3.391662in}%
\pgfsys@useobject{currentmarker}{}%
\end{pgfscope}%
\begin{pgfscope}%
\pgfsys@transformshift{1.077076in}{3.373013in}%
\pgfsys@useobject{currentmarker}{}%
\end{pgfscope}%
\begin{pgfscope}%
\pgfsys@transformshift{1.148374in}{3.374902in}%
\pgfsys@useobject{currentmarker}{}%
\end{pgfscope}%
\begin{pgfscope}%
\pgfsys@transformshift{1.219673in}{3.734717in}%
\pgfsys@useobject{currentmarker}{}%
\end{pgfscope}%
\begin{pgfscope}%
\pgfsys@transformshift{1.290971in}{3.534121in}%
\pgfsys@useobject{currentmarker}{}%
\end{pgfscope}%
\begin{pgfscope}%
\pgfsys@transformshift{1.362270in}{3.562295in}%
\pgfsys@useobject{currentmarker}{}%
\end{pgfscope}%
\begin{pgfscope}%
\pgfsys@transformshift{1.433568in}{3.679523in}%
\pgfsys@useobject{currentmarker}{}%
\end{pgfscope}%
\begin{pgfscope}%
\pgfsys@transformshift{1.504867in}{3.584103in}%
\pgfsys@useobject{currentmarker}{}%
\end{pgfscope}%
\begin{pgfscope}%
\pgfsys@transformshift{1.576165in}{3.830209in}%
\pgfsys@useobject{currentmarker}{}%
\end{pgfscope}%
\begin{pgfscope}%
\pgfsys@transformshift{1.647464in}{3.545463in}%
\pgfsys@useobject{currentmarker}{}%
\end{pgfscope}%
\begin{pgfscope}%
\pgfsys@transformshift{1.718762in}{3.709537in}%
\pgfsys@useobject{currentmarker}{}%
\end{pgfscope}%
\begin{pgfscope}%
\pgfsys@transformshift{1.790061in}{3.711821in}%
\pgfsys@useobject{currentmarker}{}%
\end{pgfscope}%
\begin{pgfscope}%
\pgfsys@transformshift{1.861359in}{3.325357in}%
\pgfsys@useobject{currentmarker}{}%
\end{pgfscope}%
\begin{pgfscope}%
\pgfsys@transformshift{1.932658in}{3.594810in}%
\pgfsys@useobject{currentmarker}{}%
\end{pgfscope}%
\begin{pgfscope}%
\pgfsys@transformshift{2.003956in}{3.471188in}%
\pgfsys@useobject{currentmarker}{}%
\end{pgfscope}%
\begin{pgfscope}%
\pgfsys@transformshift{2.075255in}{3.509119in}%
\pgfsys@useobject{currentmarker}{}%
\end{pgfscope}%
\begin{pgfscope}%
\pgfsys@transformshift{2.146554in}{3.768842in}%
\pgfsys@useobject{currentmarker}{}%
\end{pgfscope}%
\begin{pgfscope}%
\pgfsys@transformshift{2.217852in}{3.766371in}%
\pgfsys@useobject{currentmarker}{}%
\end{pgfscope}%
\begin{pgfscope}%
\pgfsys@transformshift{2.289151in}{3.399801in}%
\pgfsys@useobject{currentmarker}{}%
\end{pgfscope}%
\begin{pgfscope}%
\pgfsys@transformshift{2.360449in}{3.783343in}%
\pgfsys@useobject{currentmarker}{}%
\end{pgfscope}%
\begin{pgfscope}%
\pgfsys@transformshift{2.431748in}{3.479092in}%
\pgfsys@useobject{currentmarker}{}%
\end{pgfscope}%
\begin{pgfscope}%
\pgfsys@transformshift{2.503046in}{3.614306in}%
\pgfsys@useobject{currentmarker}{}%
\end{pgfscope}%
\begin{pgfscope}%
\pgfsys@transformshift{2.574345in}{3.598177in}%
\pgfsys@useobject{currentmarker}{}%
\end{pgfscope}%
\begin{pgfscope}%
\pgfsys@transformshift{2.645643in}{3.636778in}%
\pgfsys@useobject{currentmarker}{}%
\end{pgfscope}%
\begin{pgfscope}%
\pgfsys@transformshift{2.716942in}{3.442338in}%
\pgfsys@useobject{currentmarker}{}%
\end{pgfscope}%
\begin{pgfscope}%
\pgfsys@transformshift{2.788240in}{3.723791in}%
\pgfsys@useobject{currentmarker}{}%
\end{pgfscope}%
\begin{pgfscope}%
\pgfsys@transformshift{2.859539in}{3.614965in}%
\pgfsys@useobject{currentmarker}{}%
\end{pgfscope}%
\begin{pgfscope}%
\pgfsys@transformshift{2.930837in}{3.351282in}%
\pgfsys@useobject{currentmarker}{}%
\end{pgfscope}%
\begin{pgfscope}%
\pgfsys@transformshift{3.002136in}{3.780196in}%
\pgfsys@useobject{currentmarker}{}%
\end{pgfscope}%
\begin{pgfscope}%
\pgfsys@transformshift{3.073434in}{3.429245in}%
\pgfsys@useobject{currentmarker}{}%
\end{pgfscope}%
\begin{pgfscope}%
\pgfsys@transformshift{3.144733in}{3.612354in}%
\pgfsys@useobject{currentmarker}{}%
\end{pgfscope}%
\begin{pgfscope}%
\pgfsys@transformshift{3.216031in}{3.880558in}%
\pgfsys@useobject{currentmarker}{}%
\end{pgfscope}%
\begin{pgfscope}%
\pgfsys@transformshift{3.287330in}{3.525053in}%
\pgfsys@useobject{currentmarker}{}%
\end{pgfscope}%
\begin{pgfscope}%
\pgfsys@transformshift{3.358628in}{3.663245in}%
\pgfsys@useobject{currentmarker}{}%
\end{pgfscope}%
\begin{pgfscope}%
\pgfsys@transformshift{3.429927in}{3.745132in}%
\pgfsys@useobject{currentmarker}{}%
\end{pgfscope}%
\begin{pgfscope}%
\pgfsys@transformshift{3.501225in}{3.543473in}%
\pgfsys@useobject{currentmarker}{}%
\end{pgfscope}%
\begin{pgfscope}%
\pgfsys@transformshift{3.572524in}{3.447882in}%
\pgfsys@useobject{currentmarker}{}%
\end{pgfscope}%
\begin{pgfscope}%
\pgfsys@transformshift{3.643822in}{3.410656in}%
\pgfsys@useobject{currentmarker}{}%
\end{pgfscope}%
\begin{pgfscope}%
\pgfsys@transformshift{3.715121in}{3.392146in}%
\pgfsys@useobject{currentmarker}{}%
\end{pgfscope}%
\begin{pgfscope}%
\pgfsys@transformshift{3.786419in}{3.478844in}%
\pgfsys@useobject{currentmarker}{}%
\end{pgfscope}%
\begin{pgfscope}%
\pgfsys@transformshift{3.857718in}{3.861889in}%
\pgfsys@useobject{currentmarker}{}%
\end{pgfscope}%
\begin{pgfscope}%
\pgfsys@transformshift{3.929016in}{3.460176in}%
\pgfsys@useobject{currentmarker}{}%
\end{pgfscope}%
\begin{pgfscope}%
\pgfsys@transformshift{4.000315in}{3.333327in}%
\pgfsys@useobject{currentmarker}{}%
\end{pgfscope}%
\begin{pgfscope}%
\pgfsys@transformshift{4.071613in}{3.625697in}%
\pgfsys@useobject{currentmarker}{}%
\end{pgfscope}%
\begin{pgfscope}%
\pgfsys@transformshift{4.142912in}{3.368084in}%
\pgfsys@useobject{currentmarker}{}%
\end{pgfscope}%
\begin{pgfscope}%
\pgfsys@transformshift{4.214210in}{3.654081in}%
\pgfsys@useobject{currentmarker}{}%
\end{pgfscope}%
\begin{pgfscope}%
\pgfsys@transformshift{4.285509in}{3.251635in}%
\pgfsys@useobject{currentmarker}{}%
\end{pgfscope}%
\begin{pgfscope}%
\pgfsys@transformshift{4.356807in}{3.407957in}%
\pgfsys@useobject{currentmarker}{}%
\end{pgfscope}%
\begin{pgfscope}%
\pgfsys@transformshift{4.428106in}{3.666425in}%
\pgfsys@useobject{currentmarker}{}%
\end{pgfscope}%
\begin{pgfscope}%
\pgfsys@transformshift{4.499404in}{3.545169in}%
\pgfsys@useobject{currentmarker}{}%
\end{pgfscope}%
\begin{pgfscope}%
\pgfsys@transformshift{4.570703in}{3.667987in}%
\pgfsys@useobject{currentmarker}{}%
\end{pgfscope}%
\begin{pgfscope}%
\pgfsys@transformshift{4.642001in}{3.560569in}%
\pgfsys@useobject{currentmarker}{}%
\end{pgfscope}%
\begin{pgfscope}%
\pgfsys@transformshift{4.713300in}{3.547180in}%
\pgfsys@useobject{currentmarker}{}%
\end{pgfscope}%
\begin{pgfscope}%
\pgfsys@transformshift{4.784598in}{3.357873in}%
\pgfsys@useobject{currentmarker}{}%
\end{pgfscope}%
\begin{pgfscope}%
\pgfsys@transformshift{4.855897in}{3.473717in}%
\pgfsys@useobject{currentmarker}{}%
\end{pgfscope}%
\begin{pgfscope}%
\pgfsys@transformshift{4.927196in}{3.597955in}%
\pgfsys@useobject{currentmarker}{}%
\end{pgfscope}%
\begin{pgfscope}%
\pgfsys@transformshift{4.998494in}{3.868054in}%
\pgfsys@useobject{currentmarker}{}%
\end{pgfscope}%
\begin{pgfscope}%
\pgfsys@transformshift{5.069793in}{3.580220in}%
\pgfsys@useobject{currentmarker}{}%
\end{pgfscope}%
\begin{pgfscope}%
\pgfsys@transformshift{5.141091in}{3.495275in}%
\pgfsys@useobject{currentmarker}{}%
\end{pgfscope}%
\begin{pgfscope}%
\pgfsys@transformshift{5.212390in}{3.533702in}%
\pgfsys@useobject{currentmarker}{}%
\end{pgfscope}%
\begin{pgfscope}%
\pgfsys@transformshift{5.283688in}{3.601650in}%
\pgfsys@useobject{currentmarker}{}%
\end{pgfscope}%
\begin{pgfscope}%
\pgfsys@transformshift{5.354987in}{3.651986in}%
\pgfsys@useobject{currentmarker}{}%
\end{pgfscope}%
\end{pgfscope}%
\begin{pgfscope}%
\pgfsetrectcap%
\pgfsetmiterjoin%
\pgfsetlinewidth{0.803000pt}%
\definecolor{currentstroke}{rgb}{0.000000,0.000000,0.000000}%
\pgfsetstrokecolor{currentstroke}%
\pgfsetdash{}{0pt}%
\pgfpathmoveto{\pgfqpoint{0.759375in}{3.220189in}}%
\pgfpathlineto{\pgfqpoint{0.759375in}{3.912004in}}%
\pgfusepath{stroke}%
\end{pgfscope}%
\begin{pgfscope}%
\pgfsetrectcap%
\pgfsetmiterjoin%
\pgfsetlinewidth{0.803000pt}%
\definecolor{currentstroke}{rgb}{0.000000,0.000000,0.000000}%
\pgfsetstrokecolor{currentstroke}%
\pgfsetdash{}{0pt}%
\pgfpathmoveto{\pgfqpoint{5.601389in}{3.220189in}}%
\pgfpathlineto{\pgfqpoint{5.601389in}{3.912004in}}%
\pgfusepath{stroke}%
\end{pgfscope}%
\begin{pgfscope}%
\pgfsetrectcap%
\pgfsetmiterjoin%
\pgfsetlinewidth{0.803000pt}%
\definecolor{currentstroke}{rgb}{0.000000,0.000000,0.000000}%
\pgfsetstrokecolor{currentstroke}%
\pgfsetdash{}{0pt}%
\pgfpathmoveto{\pgfqpoint{0.759375in}{3.220189in}}%
\pgfpathlineto{\pgfqpoint{5.601389in}{3.220189in}}%
\pgfusepath{stroke}%
\end{pgfscope}%
\begin{pgfscope}%
\pgfsetrectcap%
\pgfsetmiterjoin%
\pgfsetlinewidth{0.803000pt}%
\definecolor{currentstroke}{rgb}{0.000000,0.000000,0.000000}%
\pgfsetstrokecolor{currentstroke}%
\pgfsetdash{}{0pt}%
\pgfpathmoveto{\pgfqpoint{0.759375in}{3.912004in}}%
\pgfpathlineto{\pgfqpoint{5.601389in}{3.912004in}}%
\pgfusepath{stroke}%
\end{pgfscope}%
\begin{pgfscope}%
\pgfsetbuttcap%
\pgfsetmiterjoin%
\definecolor{currentfill}{rgb}{1.000000,1.000000,1.000000}%
\pgfsetfillcolor{currentfill}%
\pgfsetlinewidth{0.000000pt}%
\definecolor{currentstroke}{rgb}{0.000000,0.000000,0.000000}%
\pgfsetstrokecolor{currentstroke}%
\pgfsetstrokeopacity{0.000000}%
\pgfsetdash{}{0pt}%
\pgfpathmoveto{\pgfqpoint{0.759375in}{2.275496in}}%
\pgfpathlineto{\pgfqpoint{5.601389in}{2.275496in}}%
\pgfpathlineto{\pgfqpoint{5.601389in}{2.967311in}}%
\pgfpathlineto{\pgfqpoint{0.759375in}{2.967311in}}%
\pgfpathclose%
\pgfusepath{fill}%
\end{pgfscope}%
\begin{pgfscope}%
\pgfsetbuttcap%
\pgfsetroundjoin%
\definecolor{currentfill}{rgb}{0.000000,0.000000,0.000000}%
\pgfsetfillcolor{currentfill}%
\pgfsetlinewidth{0.803000pt}%
\definecolor{currentstroke}{rgb}{0.000000,0.000000,0.000000}%
\pgfsetstrokecolor{currentstroke}%
\pgfsetdash{}{0pt}%
\pgfsys@defobject{currentmarker}{\pgfqpoint{0.000000in}{-0.048611in}}{\pgfqpoint{0.000000in}{0.000000in}}{%
\pgfpathmoveto{\pgfqpoint{0.000000in}{0.000000in}}%
\pgfpathlineto{\pgfqpoint{0.000000in}{-0.048611in}}%
\pgfusepath{stroke,fill}%
}%
\begin{pgfscope}%
\pgfsys@transformshift{0.979467in}{2.275496in}%
\pgfsys@useobject{currentmarker}{}%
\end{pgfscope}%
\end{pgfscope}%
\begin{pgfscope}%
\pgfsetbuttcap%
\pgfsetroundjoin%
\definecolor{currentfill}{rgb}{0.000000,0.000000,0.000000}%
\pgfsetfillcolor{currentfill}%
\pgfsetlinewidth{0.803000pt}%
\definecolor{currentstroke}{rgb}{0.000000,0.000000,0.000000}%
\pgfsetstrokecolor{currentstroke}%
\pgfsetdash{}{0pt}%
\pgfsys@defobject{currentmarker}{\pgfqpoint{0.000000in}{-0.048611in}}{\pgfqpoint{0.000000in}{0.000000in}}{%
\pgfpathmoveto{\pgfqpoint{0.000000in}{0.000000in}}%
\pgfpathlineto{\pgfqpoint{0.000000in}{-0.048611in}}%
\pgfusepath{stroke,fill}%
}%
\begin{pgfscope}%
\pgfsys@transformshift{1.529695in}{2.275496in}%
\pgfsys@useobject{currentmarker}{}%
\end{pgfscope}%
\end{pgfscope}%
\begin{pgfscope}%
\pgfsetbuttcap%
\pgfsetroundjoin%
\definecolor{currentfill}{rgb}{0.000000,0.000000,0.000000}%
\pgfsetfillcolor{currentfill}%
\pgfsetlinewidth{0.803000pt}%
\definecolor{currentstroke}{rgb}{0.000000,0.000000,0.000000}%
\pgfsetstrokecolor{currentstroke}%
\pgfsetdash{}{0pt}%
\pgfsys@defobject{currentmarker}{\pgfqpoint{0.000000in}{-0.048611in}}{\pgfqpoint{0.000000in}{0.000000in}}{%
\pgfpathmoveto{\pgfqpoint{0.000000in}{0.000000in}}%
\pgfpathlineto{\pgfqpoint{0.000000in}{-0.048611in}}%
\pgfusepath{stroke,fill}%
}%
\begin{pgfscope}%
\pgfsys@transformshift{2.079924in}{2.275496in}%
\pgfsys@useobject{currentmarker}{}%
\end{pgfscope}%
\end{pgfscope}%
\begin{pgfscope}%
\pgfsetbuttcap%
\pgfsetroundjoin%
\definecolor{currentfill}{rgb}{0.000000,0.000000,0.000000}%
\pgfsetfillcolor{currentfill}%
\pgfsetlinewidth{0.803000pt}%
\definecolor{currentstroke}{rgb}{0.000000,0.000000,0.000000}%
\pgfsetstrokecolor{currentstroke}%
\pgfsetdash{}{0pt}%
\pgfsys@defobject{currentmarker}{\pgfqpoint{0.000000in}{-0.048611in}}{\pgfqpoint{0.000000in}{0.000000in}}{%
\pgfpathmoveto{\pgfqpoint{0.000000in}{0.000000in}}%
\pgfpathlineto{\pgfqpoint{0.000000in}{-0.048611in}}%
\pgfusepath{stroke,fill}%
}%
\begin{pgfscope}%
\pgfsys@transformshift{2.630153in}{2.275496in}%
\pgfsys@useobject{currentmarker}{}%
\end{pgfscope}%
\end{pgfscope}%
\begin{pgfscope}%
\pgfsetbuttcap%
\pgfsetroundjoin%
\definecolor{currentfill}{rgb}{0.000000,0.000000,0.000000}%
\pgfsetfillcolor{currentfill}%
\pgfsetlinewidth{0.803000pt}%
\definecolor{currentstroke}{rgb}{0.000000,0.000000,0.000000}%
\pgfsetstrokecolor{currentstroke}%
\pgfsetdash{}{0pt}%
\pgfsys@defobject{currentmarker}{\pgfqpoint{0.000000in}{-0.048611in}}{\pgfqpoint{0.000000in}{0.000000in}}{%
\pgfpathmoveto{\pgfqpoint{0.000000in}{0.000000in}}%
\pgfpathlineto{\pgfqpoint{0.000000in}{-0.048611in}}%
\pgfusepath{stroke,fill}%
}%
\begin{pgfscope}%
\pgfsys@transformshift{3.180382in}{2.275496in}%
\pgfsys@useobject{currentmarker}{}%
\end{pgfscope}%
\end{pgfscope}%
\begin{pgfscope}%
\pgfsetbuttcap%
\pgfsetroundjoin%
\definecolor{currentfill}{rgb}{0.000000,0.000000,0.000000}%
\pgfsetfillcolor{currentfill}%
\pgfsetlinewidth{0.803000pt}%
\definecolor{currentstroke}{rgb}{0.000000,0.000000,0.000000}%
\pgfsetstrokecolor{currentstroke}%
\pgfsetdash{}{0pt}%
\pgfsys@defobject{currentmarker}{\pgfqpoint{0.000000in}{-0.048611in}}{\pgfqpoint{0.000000in}{0.000000in}}{%
\pgfpathmoveto{\pgfqpoint{0.000000in}{0.000000in}}%
\pgfpathlineto{\pgfqpoint{0.000000in}{-0.048611in}}%
\pgfusepath{stroke,fill}%
}%
\begin{pgfscope}%
\pgfsys@transformshift{3.730611in}{2.275496in}%
\pgfsys@useobject{currentmarker}{}%
\end{pgfscope}%
\end{pgfscope}%
\begin{pgfscope}%
\pgfsetbuttcap%
\pgfsetroundjoin%
\definecolor{currentfill}{rgb}{0.000000,0.000000,0.000000}%
\pgfsetfillcolor{currentfill}%
\pgfsetlinewidth{0.803000pt}%
\definecolor{currentstroke}{rgb}{0.000000,0.000000,0.000000}%
\pgfsetstrokecolor{currentstroke}%
\pgfsetdash{}{0pt}%
\pgfsys@defobject{currentmarker}{\pgfqpoint{0.000000in}{-0.048611in}}{\pgfqpoint{0.000000in}{0.000000in}}{%
\pgfpathmoveto{\pgfqpoint{0.000000in}{0.000000in}}%
\pgfpathlineto{\pgfqpoint{0.000000in}{-0.048611in}}%
\pgfusepath{stroke,fill}%
}%
\begin{pgfscope}%
\pgfsys@transformshift{4.280840in}{2.275496in}%
\pgfsys@useobject{currentmarker}{}%
\end{pgfscope}%
\end{pgfscope}%
\begin{pgfscope}%
\pgfsetbuttcap%
\pgfsetroundjoin%
\definecolor{currentfill}{rgb}{0.000000,0.000000,0.000000}%
\pgfsetfillcolor{currentfill}%
\pgfsetlinewidth{0.803000pt}%
\definecolor{currentstroke}{rgb}{0.000000,0.000000,0.000000}%
\pgfsetstrokecolor{currentstroke}%
\pgfsetdash{}{0pt}%
\pgfsys@defobject{currentmarker}{\pgfqpoint{0.000000in}{-0.048611in}}{\pgfqpoint{0.000000in}{0.000000in}}{%
\pgfpathmoveto{\pgfqpoint{0.000000in}{0.000000in}}%
\pgfpathlineto{\pgfqpoint{0.000000in}{-0.048611in}}%
\pgfusepath{stroke,fill}%
}%
\begin{pgfscope}%
\pgfsys@transformshift{4.831068in}{2.275496in}%
\pgfsys@useobject{currentmarker}{}%
\end{pgfscope}%
\end{pgfscope}%
\begin{pgfscope}%
\pgfsetbuttcap%
\pgfsetroundjoin%
\definecolor{currentfill}{rgb}{0.000000,0.000000,0.000000}%
\pgfsetfillcolor{currentfill}%
\pgfsetlinewidth{0.803000pt}%
\definecolor{currentstroke}{rgb}{0.000000,0.000000,0.000000}%
\pgfsetstrokecolor{currentstroke}%
\pgfsetdash{}{0pt}%
\pgfsys@defobject{currentmarker}{\pgfqpoint{0.000000in}{-0.048611in}}{\pgfqpoint{0.000000in}{0.000000in}}{%
\pgfpathmoveto{\pgfqpoint{0.000000in}{0.000000in}}%
\pgfpathlineto{\pgfqpoint{0.000000in}{-0.048611in}}%
\pgfusepath{stroke,fill}%
}%
\begin{pgfscope}%
\pgfsys@transformshift{5.381297in}{2.275496in}%
\pgfsys@useobject{currentmarker}{}%
\end{pgfscope}%
\end{pgfscope}%
\begin{pgfscope}%
\pgfsetbuttcap%
\pgfsetroundjoin%
\definecolor{currentfill}{rgb}{0.000000,0.000000,0.000000}%
\pgfsetfillcolor{currentfill}%
\pgfsetlinewidth{0.803000pt}%
\definecolor{currentstroke}{rgb}{0.000000,0.000000,0.000000}%
\pgfsetstrokecolor{currentstroke}%
\pgfsetdash{}{0pt}%
\pgfsys@defobject{currentmarker}{\pgfqpoint{-0.048611in}{0.000000in}}{\pgfqpoint{0.000000in}{0.000000in}}{%
\pgfpathmoveto{\pgfqpoint{0.000000in}{0.000000in}}%
\pgfpathlineto{\pgfqpoint{-0.048611in}{0.000000in}}%
\pgfusepath{stroke,fill}%
}%
\begin{pgfscope}%
\pgfsys@transformshift{0.759375in}{2.466246in}%
\pgfsys@useobject{currentmarker}{}%
\end{pgfscope}%
\end{pgfscope}%
\begin{pgfscope}%
\definecolor{textcolor}{rgb}{0.000000,0.000000,0.000000}%
\pgfsetstrokecolor{textcolor}%
\pgfsetfillcolor{textcolor}%
\pgftext[x=0.237708in,y=2.418051in,left,base]{\color{textcolor}\rmfamily\fontsize{10.000000}{12.000000}\selectfont −0.025}%
\end{pgfscope}%
\begin{pgfscope}%
\pgfsetbuttcap%
\pgfsetroundjoin%
\definecolor{currentfill}{rgb}{0.000000,0.000000,0.000000}%
\pgfsetfillcolor{currentfill}%
\pgfsetlinewidth{0.803000pt}%
\definecolor{currentstroke}{rgb}{0.000000,0.000000,0.000000}%
\pgfsetstrokecolor{currentstroke}%
\pgfsetdash{}{0pt}%
\pgfsys@defobject{currentmarker}{\pgfqpoint{-0.048611in}{0.000000in}}{\pgfqpoint{0.000000in}{0.000000in}}{%
\pgfpathmoveto{\pgfqpoint{0.000000in}{0.000000in}}%
\pgfpathlineto{\pgfqpoint{-0.048611in}{0.000000in}}%
\pgfusepath{stroke,fill}%
}%
\begin{pgfscope}%
\pgfsys@transformshift{0.759375in}{2.757133in}%
\pgfsys@useobject{currentmarker}{}%
\end{pgfscope}%
\end{pgfscope}%
\begin{pgfscope}%
\definecolor{textcolor}{rgb}{0.000000,0.000000,0.000000}%
\pgfsetstrokecolor{textcolor}%
\pgfsetfillcolor{textcolor}%
\pgftext[x=0.345764in,y=2.708938in,left,base]{\color{textcolor}\rmfamily\fontsize{10.000000}{12.000000}\selectfont 0.000}%
\end{pgfscope}%
\begin{pgfscope}%
\pgfpathrectangle{\pgfqpoint{0.759375in}{2.275496in}}{\pgfqpoint{4.842014in}{0.691815in}}%
\pgfusepath{clip}%
\pgfsetrectcap%
\pgfsetroundjoin%
\pgfsetlinewidth{1.505625pt}%
\definecolor{currentstroke}{rgb}{1.000000,0.498039,0.054902}%
\pgfsetstrokecolor{currentstroke}%
\pgfsetdash{}{0pt}%
\pgfpathmoveto{\pgfqpoint{1.041427in}{2.715804in}}%
\pgfpathlineto{\pgfqpoint{1.115184in}{2.715804in}}%
\pgfpathlineto{\pgfqpoint{1.115184in}{2.558268in}}%
\pgfpathlineto{\pgfqpoint{1.262698in}{2.558268in}}%
\pgfpathlineto{\pgfqpoint{1.262698in}{2.865168in}}%
\pgfpathlineto{\pgfqpoint{1.410212in}{2.865168in}}%
\pgfpathlineto{\pgfqpoint{1.410212in}{2.890418in}}%
\pgfpathlineto{\pgfqpoint{1.557726in}{2.890418in}}%
\pgfpathlineto{\pgfqpoint{1.557726in}{2.705214in}}%
\pgfpathlineto{\pgfqpoint{1.705240in}{2.705214in}}%
\pgfpathlineto{\pgfqpoint{1.705240in}{2.796741in}}%
\pgfpathlineto{\pgfqpoint{1.852754in}{2.796741in}}%
\pgfpathlineto{\pgfqpoint{1.852754in}{2.785040in}}%
\pgfpathlineto{\pgfqpoint{2.000269in}{2.785040in}}%
\pgfpathlineto{\pgfqpoint{2.000269in}{2.839130in}}%
\pgfpathlineto{\pgfqpoint{2.147783in}{2.839130in}}%
\pgfpathlineto{\pgfqpoint{2.147783in}{2.628644in}}%
\pgfpathlineto{\pgfqpoint{2.295297in}{2.628644in}}%
\pgfpathlineto{\pgfqpoint{2.295297in}{2.935865in}}%
\pgfpathlineto{\pgfqpoint{2.442811in}{2.935865in}}%
\pgfpathlineto{\pgfqpoint{2.442811in}{2.790666in}}%
\pgfpathlineto{\pgfqpoint{2.590325in}{2.790666in}}%
\pgfpathlineto{\pgfqpoint{2.590325in}{2.574308in}}%
\pgfpathlineto{\pgfqpoint{2.737839in}{2.574308in}}%
\pgfpathlineto{\pgfqpoint{2.737839in}{2.685957in}}%
\pgfpathlineto{\pgfqpoint{2.885354in}{2.685957in}}%
\pgfpathlineto{\pgfqpoint{2.885354in}{2.457886in}}%
\pgfpathlineto{\pgfqpoint{3.032868in}{2.457886in}}%
\pgfpathlineto{\pgfqpoint{3.032868in}{2.792706in}}%
\pgfpathlineto{\pgfqpoint{3.180382in}{2.792706in}}%
\pgfpathlineto{\pgfqpoint{3.180382in}{2.763382in}}%
\pgfpathlineto{\pgfqpoint{3.327896in}{2.763382in}}%
\pgfpathlineto{\pgfqpoint{3.327896in}{2.884272in}}%
\pgfpathlineto{\pgfqpoint{3.475410in}{2.884272in}}%
\pgfpathlineto{\pgfqpoint{3.475410in}{2.723829in}}%
\pgfpathlineto{\pgfqpoint{3.622924in}{2.723829in}}%
\pgfpathlineto{\pgfqpoint{3.622924in}{2.910310in}}%
\pgfpathlineto{\pgfqpoint{3.770439in}{2.910310in}}%
\pgfpathlineto{\pgfqpoint{3.770439in}{2.596628in}}%
\pgfpathlineto{\pgfqpoint{3.917953in}{2.596628in}}%
\pgfpathlineto{\pgfqpoint{3.917953in}{2.469297in}}%
\pgfpathlineto{\pgfqpoint{4.065467in}{2.469297in}}%
\pgfpathlineto{\pgfqpoint{4.065467in}{2.881845in}}%
\pgfpathlineto{\pgfqpoint{4.212981in}{2.881845in}}%
\pgfpathlineto{\pgfqpoint{4.212981in}{2.424489in}}%
\pgfpathlineto{\pgfqpoint{4.360495in}{2.424489in}}%
\pgfpathlineto{\pgfqpoint{4.360495in}{2.474825in}}%
\pgfpathlineto{\pgfqpoint{4.508009in}{2.474825in}}%
\pgfpathlineto{\pgfqpoint{4.508009in}{2.450773in}}%
\pgfpathlineto{\pgfqpoint{4.655524in}{2.450773in}}%
\pgfpathlineto{\pgfqpoint{4.655524in}{2.735241in}}%
\pgfpathlineto{\pgfqpoint{4.803038in}{2.735241in}}%
\pgfpathlineto{\pgfqpoint{4.803038in}{2.452171in}}%
\pgfpathlineto{\pgfqpoint{4.950552in}{2.452171in}}%
\pgfpathlineto{\pgfqpoint{4.950552in}{2.306942in}}%
\pgfpathlineto{\pgfqpoint{5.098066in}{2.306942in}}%
\pgfpathlineto{\pgfqpoint{5.098066in}{2.565952in}}%
\pgfpathlineto{\pgfqpoint{5.245580in}{2.565952in}}%
\pgfpathlineto{\pgfqpoint{5.245580in}{2.423319in}}%
\pgfpathlineto{\pgfqpoint{5.319337in}{2.423319in}}%
\pgfusepath{stroke}%
\end{pgfscope}%
\begin{pgfscope}%
\pgfpathrectangle{\pgfqpoint{0.759375in}{2.275496in}}{\pgfqpoint{4.842014in}{0.691815in}}%
\pgfusepath{clip}%
\pgfsetbuttcap%
\pgfsetroundjoin%
\definecolor{currentfill}{rgb}{1.000000,0.498039,0.054902}%
\pgfsetfillcolor{currentfill}%
\pgfsetlinewidth{1.003750pt}%
\definecolor{currentstroke}{rgb}{1.000000,0.498039,0.054902}%
\pgfsetstrokecolor{currentstroke}%
\pgfsetdash{}{0pt}%
\pgfsys@defobject{currentmarker}{\pgfqpoint{-0.041667in}{-0.041667in}}{\pgfqpoint{0.041667in}{0.041667in}}{%
\pgfpathmoveto{\pgfqpoint{0.000000in}{-0.041667in}}%
\pgfpathcurveto{\pgfqpoint{0.011050in}{-0.041667in}}{\pgfqpoint{0.021649in}{-0.037276in}}{\pgfqpoint{0.029463in}{-0.029463in}}%
\pgfpathcurveto{\pgfqpoint{0.037276in}{-0.021649in}}{\pgfqpoint{0.041667in}{-0.011050in}}{\pgfqpoint{0.041667in}{0.000000in}}%
\pgfpathcurveto{\pgfqpoint{0.041667in}{0.011050in}}{\pgfqpoint{0.037276in}{0.021649in}}{\pgfqpoint{0.029463in}{0.029463in}}%
\pgfpathcurveto{\pgfqpoint{0.021649in}{0.037276in}}{\pgfqpoint{0.011050in}{0.041667in}}{\pgfqpoint{0.000000in}{0.041667in}}%
\pgfpathcurveto{\pgfqpoint{-0.011050in}{0.041667in}}{\pgfqpoint{-0.021649in}{0.037276in}}{\pgfqpoint{-0.029463in}{0.029463in}}%
\pgfpathcurveto{\pgfqpoint{-0.037276in}{0.021649in}}{\pgfqpoint{-0.041667in}{0.011050in}}{\pgfqpoint{-0.041667in}{0.000000in}}%
\pgfpathcurveto{\pgfqpoint{-0.041667in}{-0.011050in}}{\pgfqpoint{-0.037276in}{-0.021649in}}{\pgfqpoint{-0.029463in}{-0.029463in}}%
\pgfpathcurveto{\pgfqpoint{-0.021649in}{-0.037276in}}{\pgfqpoint{-0.011050in}{-0.041667in}}{\pgfqpoint{0.000000in}{-0.041667in}}%
\pgfpathclose%
\pgfusepath{stroke,fill}%
}%
\begin{pgfscope}%
\pgfsys@transformshift{1.041427in}{2.715804in}%
\pgfsys@useobject{currentmarker}{}%
\end{pgfscope}%
\begin{pgfscope}%
\pgfsys@transformshift{1.188941in}{2.558268in}%
\pgfsys@useobject{currentmarker}{}%
\end{pgfscope}%
\begin{pgfscope}%
\pgfsys@transformshift{1.336455in}{2.865168in}%
\pgfsys@useobject{currentmarker}{}%
\end{pgfscope}%
\begin{pgfscope}%
\pgfsys@transformshift{1.483969in}{2.890418in}%
\pgfsys@useobject{currentmarker}{}%
\end{pgfscope}%
\begin{pgfscope}%
\pgfsys@transformshift{1.631483in}{2.705214in}%
\pgfsys@useobject{currentmarker}{}%
\end{pgfscope}%
\begin{pgfscope}%
\pgfsys@transformshift{1.778997in}{2.796741in}%
\pgfsys@useobject{currentmarker}{}%
\end{pgfscope}%
\begin{pgfscope}%
\pgfsys@transformshift{1.926512in}{2.785040in}%
\pgfsys@useobject{currentmarker}{}%
\end{pgfscope}%
\begin{pgfscope}%
\pgfsys@transformshift{2.074026in}{2.839130in}%
\pgfsys@useobject{currentmarker}{}%
\end{pgfscope}%
\begin{pgfscope}%
\pgfsys@transformshift{2.221540in}{2.628644in}%
\pgfsys@useobject{currentmarker}{}%
\end{pgfscope}%
\begin{pgfscope}%
\pgfsys@transformshift{2.369054in}{2.935865in}%
\pgfsys@useobject{currentmarker}{}%
\end{pgfscope}%
\begin{pgfscope}%
\pgfsys@transformshift{2.516568in}{2.790666in}%
\pgfsys@useobject{currentmarker}{}%
\end{pgfscope}%
\begin{pgfscope}%
\pgfsys@transformshift{2.664082in}{2.574308in}%
\pgfsys@useobject{currentmarker}{}%
\end{pgfscope}%
\begin{pgfscope}%
\pgfsys@transformshift{2.811597in}{2.685957in}%
\pgfsys@useobject{currentmarker}{}%
\end{pgfscope}%
\begin{pgfscope}%
\pgfsys@transformshift{2.959111in}{2.457886in}%
\pgfsys@useobject{currentmarker}{}%
\end{pgfscope}%
\begin{pgfscope}%
\pgfsys@transformshift{3.106625in}{2.792706in}%
\pgfsys@useobject{currentmarker}{}%
\end{pgfscope}%
\begin{pgfscope}%
\pgfsys@transformshift{3.254139in}{2.763382in}%
\pgfsys@useobject{currentmarker}{}%
\end{pgfscope}%
\begin{pgfscope}%
\pgfsys@transformshift{3.401653in}{2.884272in}%
\pgfsys@useobject{currentmarker}{}%
\end{pgfscope}%
\begin{pgfscope}%
\pgfsys@transformshift{3.549167in}{2.723829in}%
\pgfsys@useobject{currentmarker}{}%
\end{pgfscope}%
\begin{pgfscope}%
\pgfsys@transformshift{3.696682in}{2.910310in}%
\pgfsys@useobject{currentmarker}{}%
\end{pgfscope}%
\begin{pgfscope}%
\pgfsys@transformshift{3.844196in}{2.596628in}%
\pgfsys@useobject{currentmarker}{}%
\end{pgfscope}%
\begin{pgfscope}%
\pgfsys@transformshift{3.991710in}{2.469297in}%
\pgfsys@useobject{currentmarker}{}%
\end{pgfscope}%
\begin{pgfscope}%
\pgfsys@transformshift{4.139224in}{2.881845in}%
\pgfsys@useobject{currentmarker}{}%
\end{pgfscope}%
\begin{pgfscope}%
\pgfsys@transformshift{4.286738in}{2.424489in}%
\pgfsys@useobject{currentmarker}{}%
\end{pgfscope}%
\begin{pgfscope}%
\pgfsys@transformshift{4.434252in}{2.474825in}%
\pgfsys@useobject{currentmarker}{}%
\end{pgfscope}%
\begin{pgfscope}%
\pgfsys@transformshift{4.581767in}{2.450773in}%
\pgfsys@useobject{currentmarker}{}%
\end{pgfscope}%
\begin{pgfscope}%
\pgfsys@transformshift{4.729281in}{2.735241in}%
\pgfsys@useobject{currentmarker}{}%
\end{pgfscope}%
\begin{pgfscope}%
\pgfsys@transformshift{4.876795in}{2.452171in}%
\pgfsys@useobject{currentmarker}{}%
\end{pgfscope}%
\begin{pgfscope}%
\pgfsys@transformshift{5.024309in}{2.306942in}%
\pgfsys@useobject{currentmarker}{}%
\end{pgfscope}%
\begin{pgfscope}%
\pgfsys@transformshift{5.171823in}{2.565952in}%
\pgfsys@useobject{currentmarker}{}%
\end{pgfscope}%
\begin{pgfscope}%
\pgfsys@transformshift{5.319337in}{2.423319in}%
\pgfsys@useobject{currentmarker}{}%
\end{pgfscope}%
\end{pgfscope}%
\begin{pgfscope}%
\pgfsetrectcap%
\pgfsetmiterjoin%
\pgfsetlinewidth{0.803000pt}%
\definecolor{currentstroke}{rgb}{0.000000,0.000000,0.000000}%
\pgfsetstrokecolor{currentstroke}%
\pgfsetdash{}{0pt}%
\pgfpathmoveto{\pgfqpoint{0.759375in}{2.275496in}}%
\pgfpathlineto{\pgfqpoint{0.759375in}{2.967311in}}%
\pgfusepath{stroke}%
\end{pgfscope}%
\begin{pgfscope}%
\pgfsetrectcap%
\pgfsetmiterjoin%
\pgfsetlinewidth{0.803000pt}%
\definecolor{currentstroke}{rgb}{0.000000,0.000000,0.000000}%
\pgfsetstrokecolor{currentstroke}%
\pgfsetdash{}{0pt}%
\pgfpathmoveto{\pgfqpoint{5.601389in}{2.275496in}}%
\pgfpathlineto{\pgfqpoint{5.601389in}{2.967311in}}%
\pgfusepath{stroke}%
\end{pgfscope}%
\begin{pgfscope}%
\pgfsetrectcap%
\pgfsetmiterjoin%
\pgfsetlinewidth{0.803000pt}%
\definecolor{currentstroke}{rgb}{0.000000,0.000000,0.000000}%
\pgfsetstrokecolor{currentstroke}%
\pgfsetdash{}{0pt}%
\pgfpathmoveto{\pgfqpoint{0.759375in}{2.275496in}}%
\pgfpathlineto{\pgfqpoint{5.601389in}{2.275496in}}%
\pgfusepath{stroke}%
\end{pgfscope}%
\begin{pgfscope}%
\pgfsetrectcap%
\pgfsetmiterjoin%
\pgfsetlinewidth{0.803000pt}%
\definecolor{currentstroke}{rgb}{0.000000,0.000000,0.000000}%
\pgfsetstrokecolor{currentstroke}%
\pgfsetdash{}{0pt}%
\pgfpathmoveto{\pgfqpoint{0.759375in}{2.967311in}}%
\pgfpathlineto{\pgfqpoint{5.601389in}{2.967311in}}%
\pgfusepath{stroke}%
\end{pgfscope}%
\begin{pgfscope}%
\pgfsetbuttcap%
\pgfsetmiterjoin%
\definecolor{currentfill}{rgb}{1.000000,1.000000,1.000000}%
\pgfsetfillcolor{currentfill}%
\pgfsetlinewidth{0.000000pt}%
\definecolor{currentstroke}{rgb}{0.000000,0.000000,0.000000}%
\pgfsetstrokecolor{currentstroke}%
\pgfsetstrokeopacity{0.000000}%
\pgfsetdash{}{0pt}%
\pgfpathmoveto{\pgfqpoint{0.759375in}{1.330804in}}%
\pgfpathlineto{\pgfqpoint{5.601389in}{1.330804in}}%
\pgfpathlineto{\pgfqpoint{5.601389in}{2.022619in}}%
\pgfpathlineto{\pgfqpoint{0.759375in}{2.022619in}}%
\pgfpathclose%
\pgfusepath{fill}%
\end{pgfscope}%
\begin{pgfscope}%
\pgfsetbuttcap%
\pgfsetroundjoin%
\definecolor{currentfill}{rgb}{0.000000,0.000000,0.000000}%
\pgfsetfillcolor{currentfill}%
\pgfsetlinewidth{0.803000pt}%
\definecolor{currentstroke}{rgb}{0.000000,0.000000,0.000000}%
\pgfsetstrokecolor{currentstroke}%
\pgfsetdash{}{0pt}%
\pgfsys@defobject{currentmarker}{\pgfqpoint{0.000000in}{-0.048611in}}{\pgfqpoint{0.000000in}{0.000000in}}{%
\pgfpathmoveto{\pgfqpoint{0.000000in}{0.000000in}}%
\pgfpathlineto{\pgfqpoint{0.000000in}{-0.048611in}}%
\pgfusepath{stroke,fill}%
}%
\begin{pgfscope}%
\pgfsys@transformshift{0.979467in}{1.330804in}%
\pgfsys@useobject{currentmarker}{}%
\end{pgfscope}%
\end{pgfscope}%
\begin{pgfscope}%
\pgfsetbuttcap%
\pgfsetroundjoin%
\definecolor{currentfill}{rgb}{0.000000,0.000000,0.000000}%
\pgfsetfillcolor{currentfill}%
\pgfsetlinewidth{0.803000pt}%
\definecolor{currentstroke}{rgb}{0.000000,0.000000,0.000000}%
\pgfsetstrokecolor{currentstroke}%
\pgfsetdash{}{0pt}%
\pgfsys@defobject{currentmarker}{\pgfqpoint{0.000000in}{-0.048611in}}{\pgfqpoint{0.000000in}{0.000000in}}{%
\pgfpathmoveto{\pgfqpoint{0.000000in}{0.000000in}}%
\pgfpathlineto{\pgfqpoint{0.000000in}{-0.048611in}}%
\pgfusepath{stroke,fill}%
}%
\begin{pgfscope}%
\pgfsys@transformshift{1.529695in}{1.330804in}%
\pgfsys@useobject{currentmarker}{}%
\end{pgfscope}%
\end{pgfscope}%
\begin{pgfscope}%
\pgfsetbuttcap%
\pgfsetroundjoin%
\definecolor{currentfill}{rgb}{0.000000,0.000000,0.000000}%
\pgfsetfillcolor{currentfill}%
\pgfsetlinewidth{0.803000pt}%
\definecolor{currentstroke}{rgb}{0.000000,0.000000,0.000000}%
\pgfsetstrokecolor{currentstroke}%
\pgfsetdash{}{0pt}%
\pgfsys@defobject{currentmarker}{\pgfqpoint{0.000000in}{-0.048611in}}{\pgfqpoint{0.000000in}{0.000000in}}{%
\pgfpathmoveto{\pgfqpoint{0.000000in}{0.000000in}}%
\pgfpathlineto{\pgfqpoint{0.000000in}{-0.048611in}}%
\pgfusepath{stroke,fill}%
}%
\begin{pgfscope}%
\pgfsys@transformshift{2.079924in}{1.330804in}%
\pgfsys@useobject{currentmarker}{}%
\end{pgfscope}%
\end{pgfscope}%
\begin{pgfscope}%
\pgfsetbuttcap%
\pgfsetroundjoin%
\definecolor{currentfill}{rgb}{0.000000,0.000000,0.000000}%
\pgfsetfillcolor{currentfill}%
\pgfsetlinewidth{0.803000pt}%
\definecolor{currentstroke}{rgb}{0.000000,0.000000,0.000000}%
\pgfsetstrokecolor{currentstroke}%
\pgfsetdash{}{0pt}%
\pgfsys@defobject{currentmarker}{\pgfqpoint{0.000000in}{-0.048611in}}{\pgfqpoint{0.000000in}{0.000000in}}{%
\pgfpathmoveto{\pgfqpoint{0.000000in}{0.000000in}}%
\pgfpathlineto{\pgfqpoint{0.000000in}{-0.048611in}}%
\pgfusepath{stroke,fill}%
}%
\begin{pgfscope}%
\pgfsys@transformshift{2.630153in}{1.330804in}%
\pgfsys@useobject{currentmarker}{}%
\end{pgfscope}%
\end{pgfscope}%
\begin{pgfscope}%
\pgfsetbuttcap%
\pgfsetroundjoin%
\definecolor{currentfill}{rgb}{0.000000,0.000000,0.000000}%
\pgfsetfillcolor{currentfill}%
\pgfsetlinewidth{0.803000pt}%
\definecolor{currentstroke}{rgb}{0.000000,0.000000,0.000000}%
\pgfsetstrokecolor{currentstroke}%
\pgfsetdash{}{0pt}%
\pgfsys@defobject{currentmarker}{\pgfqpoint{0.000000in}{-0.048611in}}{\pgfqpoint{0.000000in}{0.000000in}}{%
\pgfpathmoveto{\pgfqpoint{0.000000in}{0.000000in}}%
\pgfpathlineto{\pgfqpoint{0.000000in}{-0.048611in}}%
\pgfusepath{stroke,fill}%
}%
\begin{pgfscope}%
\pgfsys@transformshift{3.180382in}{1.330804in}%
\pgfsys@useobject{currentmarker}{}%
\end{pgfscope}%
\end{pgfscope}%
\begin{pgfscope}%
\pgfsetbuttcap%
\pgfsetroundjoin%
\definecolor{currentfill}{rgb}{0.000000,0.000000,0.000000}%
\pgfsetfillcolor{currentfill}%
\pgfsetlinewidth{0.803000pt}%
\definecolor{currentstroke}{rgb}{0.000000,0.000000,0.000000}%
\pgfsetstrokecolor{currentstroke}%
\pgfsetdash{}{0pt}%
\pgfsys@defobject{currentmarker}{\pgfqpoint{0.000000in}{-0.048611in}}{\pgfqpoint{0.000000in}{0.000000in}}{%
\pgfpathmoveto{\pgfqpoint{0.000000in}{0.000000in}}%
\pgfpathlineto{\pgfqpoint{0.000000in}{-0.048611in}}%
\pgfusepath{stroke,fill}%
}%
\begin{pgfscope}%
\pgfsys@transformshift{3.730611in}{1.330804in}%
\pgfsys@useobject{currentmarker}{}%
\end{pgfscope}%
\end{pgfscope}%
\begin{pgfscope}%
\pgfsetbuttcap%
\pgfsetroundjoin%
\definecolor{currentfill}{rgb}{0.000000,0.000000,0.000000}%
\pgfsetfillcolor{currentfill}%
\pgfsetlinewidth{0.803000pt}%
\definecolor{currentstroke}{rgb}{0.000000,0.000000,0.000000}%
\pgfsetstrokecolor{currentstroke}%
\pgfsetdash{}{0pt}%
\pgfsys@defobject{currentmarker}{\pgfqpoint{0.000000in}{-0.048611in}}{\pgfqpoint{0.000000in}{0.000000in}}{%
\pgfpathmoveto{\pgfqpoint{0.000000in}{0.000000in}}%
\pgfpathlineto{\pgfqpoint{0.000000in}{-0.048611in}}%
\pgfusepath{stroke,fill}%
}%
\begin{pgfscope}%
\pgfsys@transformshift{4.280840in}{1.330804in}%
\pgfsys@useobject{currentmarker}{}%
\end{pgfscope}%
\end{pgfscope}%
\begin{pgfscope}%
\pgfsetbuttcap%
\pgfsetroundjoin%
\definecolor{currentfill}{rgb}{0.000000,0.000000,0.000000}%
\pgfsetfillcolor{currentfill}%
\pgfsetlinewidth{0.803000pt}%
\definecolor{currentstroke}{rgb}{0.000000,0.000000,0.000000}%
\pgfsetstrokecolor{currentstroke}%
\pgfsetdash{}{0pt}%
\pgfsys@defobject{currentmarker}{\pgfqpoint{0.000000in}{-0.048611in}}{\pgfqpoint{0.000000in}{0.000000in}}{%
\pgfpathmoveto{\pgfqpoint{0.000000in}{0.000000in}}%
\pgfpathlineto{\pgfqpoint{0.000000in}{-0.048611in}}%
\pgfusepath{stroke,fill}%
}%
\begin{pgfscope}%
\pgfsys@transformshift{4.831068in}{1.330804in}%
\pgfsys@useobject{currentmarker}{}%
\end{pgfscope}%
\end{pgfscope}%
\begin{pgfscope}%
\pgfsetbuttcap%
\pgfsetroundjoin%
\definecolor{currentfill}{rgb}{0.000000,0.000000,0.000000}%
\pgfsetfillcolor{currentfill}%
\pgfsetlinewidth{0.803000pt}%
\definecolor{currentstroke}{rgb}{0.000000,0.000000,0.000000}%
\pgfsetstrokecolor{currentstroke}%
\pgfsetdash{}{0pt}%
\pgfsys@defobject{currentmarker}{\pgfqpoint{0.000000in}{-0.048611in}}{\pgfqpoint{0.000000in}{0.000000in}}{%
\pgfpathmoveto{\pgfqpoint{0.000000in}{0.000000in}}%
\pgfpathlineto{\pgfqpoint{0.000000in}{-0.048611in}}%
\pgfusepath{stroke,fill}%
}%
\begin{pgfscope}%
\pgfsys@transformshift{5.381297in}{1.330804in}%
\pgfsys@useobject{currentmarker}{}%
\end{pgfscope}%
\end{pgfscope}%
\begin{pgfscope}%
\pgfsetbuttcap%
\pgfsetroundjoin%
\definecolor{currentfill}{rgb}{0.000000,0.000000,0.000000}%
\pgfsetfillcolor{currentfill}%
\pgfsetlinewidth{0.803000pt}%
\definecolor{currentstroke}{rgb}{0.000000,0.000000,0.000000}%
\pgfsetstrokecolor{currentstroke}%
\pgfsetdash{}{0pt}%
\pgfsys@defobject{currentmarker}{\pgfqpoint{-0.048611in}{0.000000in}}{\pgfqpoint{0.000000in}{0.000000in}}{%
\pgfpathmoveto{\pgfqpoint{0.000000in}{0.000000in}}%
\pgfpathlineto{\pgfqpoint{-0.048611in}{0.000000in}}%
\pgfusepath{stroke,fill}%
}%
\begin{pgfscope}%
\pgfsys@transformshift{0.759375in}{1.400637in}%
\pgfsys@useobject{currentmarker}{}%
\end{pgfscope}%
\end{pgfscope}%
\begin{pgfscope}%
\definecolor{textcolor}{rgb}{0.000000,0.000000,0.000000}%
\pgfsetstrokecolor{textcolor}%
\pgfsetfillcolor{textcolor}%
\pgftext[x=0.237708in,y=1.352443in,left,base]{\color{textcolor}\rmfamily\fontsize{10.000000}{12.000000}\selectfont −0.075}%
\end{pgfscope}%
\begin{pgfscope}%
\pgfsetbuttcap%
\pgfsetroundjoin%
\definecolor{currentfill}{rgb}{0.000000,0.000000,0.000000}%
\pgfsetfillcolor{currentfill}%
\pgfsetlinewidth{0.803000pt}%
\definecolor{currentstroke}{rgb}{0.000000,0.000000,0.000000}%
\pgfsetstrokecolor{currentstroke}%
\pgfsetdash{}{0pt}%
\pgfsys@defobject{currentmarker}{\pgfqpoint{-0.048611in}{0.000000in}}{\pgfqpoint{0.000000in}{0.000000in}}{%
\pgfpathmoveto{\pgfqpoint{0.000000in}{0.000000in}}%
\pgfpathlineto{\pgfqpoint{-0.048611in}{0.000000in}}%
\pgfusepath{stroke,fill}%
}%
\begin{pgfscope}%
\pgfsys@transformshift{0.759375in}{1.621123in}%
\pgfsys@useobject{currentmarker}{}%
\end{pgfscope}%
\end{pgfscope}%
\begin{pgfscope}%
\definecolor{textcolor}{rgb}{0.000000,0.000000,0.000000}%
\pgfsetstrokecolor{textcolor}%
\pgfsetfillcolor{textcolor}%
\pgftext[x=0.237708in,y=1.572928in,left,base]{\color{textcolor}\rmfamily\fontsize{10.000000}{12.000000}\selectfont −0.050}%
\end{pgfscope}%
\begin{pgfscope}%
\pgfsetbuttcap%
\pgfsetroundjoin%
\definecolor{currentfill}{rgb}{0.000000,0.000000,0.000000}%
\pgfsetfillcolor{currentfill}%
\pgfsetlinewidth{0.803000pt}%
\definecolor{currentstroke}{rgb}{0.000000,0.000000,0.000000}%
\pgfsetstrokecolor{currentstroke}%
\pgfsetdash{}{0pt}%
\pgfsys@defobject{currentmarker}{\pgfqpoint{-0.048611in}{0.000000in}}{\pgfqpoint{0.000000in}{0.000000in}}{%
\pgfpathmoveto{\pgfqpoint{0.000000in}{0.000000in}}%
\pgfpathlineto{\pgfqpoint{-0.048611in}{0.000000in}}%
\pgfusepath{stroke,fill}%
}%
\begin{pgfscope}%
\pgfsys@transformshift{0.759375in}{1.841608in}%
\pgfsys@useobject{currentmarker}{}%
\end{pgfscope}%
\end{pgfscope}%
\begin{pgfscope}%
\definecolor{textcolor}{rgb}{0.000000,0.000000,0.000000}%
\pgfsetstrokecolor{textcolor}%
\pgfsetfillcolor{textcolor}%
\pgftext[x=0.237708in,y=1.793414in,left,base]{\color{textcolor}\rmfamily\fontsize{10.000000}{12.000000}\selectfont −0.025}%
\end{pgfscope}%
\begin{pgfscope}%
\pgfpathrectangle{\pgfqpoint{0.759375in}{1.330804in}}{\pgfqpoint{4.842014in}{0.691815in}}%
\pgfusepath{clip}%
\pgfsetrectcap%
\pgfsetroundjoin%
\pgfsetlinewidth{1.505625pt}%
\definecolor{currentstroke}{rgb}{1.000000,0.498039,0.054902}%
\pgfsetstrokecolor{currentstroke}%
\pgfsetdash{}{0pt}%
\pgfpathmoveto{\pgfqpoint{1.115184in}{1.802791in}}%
\pgfpathlineto{\pgfqpoint{1.274045in}{1.802791in}}%
\pgfpathlineto{\pgfqpoint{1.274045in}{1.882913in}}%
\pgfpathlineto{\pgfqpoint{1.591768in}{1.882913in}}%
\pgfpathlineto{\pgfqpoint{1.591768in}{1.991173in}}%
\pgfpathlineto{\pgfqpoint{1.909491in}{1.991173in}}%
\pgfpathlineto{\pgfqpoint{1.909491in}{1.882165in}}%
\pgfpathlineto{\pgfqpoint{2.227213in}{1.882165in}}%
\pgfpathlineto{\pgfqpoint{2.227213in}{1.829533in}}%
\pgfpathlineto{\pgfqpoint{2.544936in}{1.829533in}}%
\pgfpathlineto{\pgfqpoint{2.544936in}{1.816133in}}%
\pgfpathlineto{\pgfqpoint{2.862659in}{1.816133in}}%
\pgfpathlineto{\pgfqpoint{2.862659in}{1.819946in}}%
\pgfpathlineto{\pgfqpoint{3.180382in}{1.819946in}}%
\pgfpathlineto{\pgfqpoint{3.180382in}{1.707597in}}%
\pgfpathlineto{\pgfqpoint{3.498105in}{1.707597in}}%
\pgfpathlineto{\pgfqpoint{3.498105in}{1.562482in}}%
\pgfpathlineto{\pgfqpoint{3.815828in}{1.562482in}}%
\pgfpathlineto{\pgfqpoint{3.815828in}{1.671920in}}%
\pgfpathlineto{\pgfqpoint{4.133550in}{1.671920in}}%
\pgfpathlineto{\pgfqpoint{4.133550in}{1.579903in}}%
\pgfpathlineto{\pgfqpoint{4.451273in}{1.579903in}}%
\pgfpathlineto{\pgfqpoint{4.451273in}{1.457879in}}%
\pgfpathlineto{\pgfqpoint{4.768996in}{1.457879in}}%
\pgfpathlineto{\pgfqpoint{4.768996in}{1.362250in}}%
\pgfpathlineto{\pgfqpoint{5.086719in}{1.362250in}}%
\pgfpathlineto{\pgfqpoint{5.086719in}{1.467935in}}%
\pgfpathlineto{\pgfqpoint{5.245580in}{1.467935in}}%
\pgfusepath{stroke}%
\end{pgfscope}%
\begin{pgfscope}%
\pgfpathrectangle{\pgfqpoint{0.759375in}{1.330804in}}{\pgfqpoint{4.842014in}{0.691815in}}%
\pgfusepath{clip}%
\pgfsetbuttcap%
\pgfsetroundjoin%
\definecolor{currentfill}{rgb}{1.000000,0.498039,0.054902}%
\pgfsetfillcolor{currentfill}%
\pgfsetlinewidth{1.003750pt}%
\definecolor{currentstroke}{rgb}{1.000000,0.498039,0.054902}%
\pgfsetstrokecolor{currentstroke}%
\pgfsetdash{}{0pt}%
\pgfsys@defobject{currentmarker}{\pgfqpoint{-0.041667in}{-0.041667in}}{\pgfqpoint{0.041667in}{0.041667in}}{%
\pgfpathmoveto{\pgfqpoint{0.000000in}{-0.041667in}}%
\pgfpathcurveto{\pgfqpoint{0.011050in}{-0.041667in}}{\pgfqpoint{0.021649in}{-0.037276in}}{\pgfqpoint{0.029463in}{-0.029463in}}%
\pgfpathcurveto{\pgfqpoint{0.037276in}{-0.021649in}}{\pgfqpoint{0.041667in}{-0.011050in}}{\pgfqpoint{0.041667in}{0.000000in}}%
\pgfpathcurveto{\pgfqpoint{0.041667in}{0.011050in}}{\pgfqpoint{0.037276in}{0.021649in}}{\pgfqpoint{0.029463in}{0.029463in}}%
\pgfpathcurveto{\pgfqpoint{0.021649in}{0.037276in}}{\pgfqpoint{0.011050in}{0.041667in}}{\pgfqpoint{0.000000in}{0.041667in}}%
\pgfpathcurveto{\pgfqpoint{-0.011050in}{0.041667in}}{\pgfqpoint{-0.021649in}{0.037276in}}{\pgfqpoint{-0.029463in}{0.029463in}}%
\pgfpathcurveto{\pgfqpoint{-0.037276in}{0.021649in}}{\pgfqpoint{-0.041667in}{0.011050in}}{\pgfqpoint{-0.041667in}{0.000000in}}%
\pgfpathcurveto{\pgfqpoint{-0.041667in}{-0.011050in}}{\pgfqpoint{-0.037276in}{-0.021649in}}{\pgfqpoint{-0.029463in}{-0.029463in}}%
\pgfpathcurveto{\pgfqpoint{-0.021649in}{-0.037276in}}{\pgfqpoint{-0.011050in}{-0.041667in}}{\pgfqpoint{0.000000in}{-0.041667in}}%
\pgfpathclose%
\pgfusepath{stroke,fill}%
}%
\begin{pgfscope}%
\pgfsys@transformshift{1.115184in}{1.802791in}%
\pgfsys@useobject{currentmarker}{}%
\end{pgfscope}%
\begin{pgfscope}%
\pgfsys@transformshift{1.432906in}{1.882913in}%
\pgfsys@useobject{currentmarker}{}%
\end{pgfscope}%
\begin{pgfscope}%
\pgfsys@transformshift{1.750629in}{1.991173in}%
\pgfsys@useobject{currentmarker}{}%
\end{pgfscope}%
\begin{pgfscope}%
\pgfsys@transformshift{2.068352in}{1.882165in}%
\pgfsys@useobject{currentmarker}{}%
\end{pgfscope}%
\begin{pgfscope}%
\pgfsys@transformshift{2.386075in}{1.829533in}%
\pgfsys@useobject{currentmarker}{}%
\end{pgfscope}%
\begin{pgfscope}%
\pgfsys@transformshift{2.703798in}{1.816133in}%
\pgfsys@useobject{currentmarker}{}%
\end{pgfscope}%
\begin{pgfscope}%
\pgfsys@transformshift{3.021521in}{1.819946in}%
\pgfsys@useobject{currentmarker}{}%
\end{pgfscope}%
\begin{pgfscope}%
\pgfsys@transformshift{3.339243in}{1.707597in}%
\pgfsys@useobject{currentmarker}{}%
\end{pgfscope}%
\begin{pgfscope}%
\pgfsys@transformshift{3.656966in}{1.562482in}%
\pgfsys@useobject{currentmarker}{}%
\end{pgfscope}%
\begin{pgfscope}%
\pgfsys@transformshift{3.974689in}{1.671920in}%
\pgfsys@useobject{currentmarker}{}%
\end{pgfscope}%
\begin{pgfscope}%
\pgfsys@transformshift{4.292412in}{1.579903in}%
\pgfsys@useobject{currentmarker}{}%
\end{pgfscope}%
\begin{pgfscope}%
\pgfsys@transformshift{4.610135in}{1.457879in}%
\pgfsys@useobject{currentmarker}{}%
\end{pgfscope}%
\begin{pgfscope}%
\pgfsys@transformshift{4.927857in}{1.362250in}%
\pgfsys@useobject{currentmarker}{}%
\end{pgfscope}%
\begin{pgfscope}%
\pgfsys@transformshift{5.245580in}{1.467935in}%
\pgfsys@useobject{currentmarker}{}%
\end{pgfscope}%
\end{pgfscope}%
\begin{pgfscope}%
\pgfsetrectcap%
\pgfsetmiterjoin%
\pgfsetlinewidth{0.803000pt}%
\definecolor{currentstroke}{rgb}{0.000000,0.000000,0.000000}%
\pgfsetstrokecolor{currentstroke}%
\pgfsetdash{}{0pt}%
\pgfpathmoveto{\pgfqpoint{0.759375in}{1.330804in}}%
\pgfpathlineto{\pgfqpoint{0.759375in}{2.022619in}}%
\pgfusepath{stroke}%
\end{pgfscope}%
\begin{pgfscope}%
\pgfsetrectcap%
\pgfsetmiterjoin%
\pgfsetlinewidth{0.803000pt}%
\definecolor{currentstroke}{rgb}{0.000000,0.000000,0.000000}%
\pgfsetstrokecolor{currentstroke}%
\pgfsetdash{}{0pt}%
\pgfpathmoveto{\pgfqpoint{5.601389in}{1.330804in}}%
\pgfpathlineto{\pgfqpoint{5.601389in}{2.022619in}}%
\pgfusepath{stroke}%
\end{pgfscope}%
\begin{pgfscope}%
\pgfsetrectcap%
\pgfsetmiterjoin%
\pgfsetlinewidth{0.803000pt}%
\definecolor{currentstroke}{rgb}{0.000000,0.000000,0.000000}%
\pgfsetstrokecolor{currentstroke}%
\pgfsetdash{}{0pt}%
\pgfpathmoveto{\pgfqpoint{0.759375in}{1.330804in}}%
\pgfpathlineto{\pgfqpoint{5.601389in}{1.330804in}}%
\pgfusepath{stroke}%
\end{pgfscope}%
\begin{pgfscope}%
\pgfsetrectcap%
\pgfsetmiterjoin%
\pgfsetlinewidth{0.803000pt}%
\definecolor{currentstroke}{rgb}{0.000000,0.000000,0.000000}%
\pgfsetstrokecolor{currentstroke}%
\pgfsetdash{}{0pt}%
\pgfpathmoveto{\pgfqpoint{0.759375in}{2.022619in}}%
\pgfpathlineto{\pgfqpoint{5.601389in}{2.022619in}}%
\pgfusepath{stroke}%
\end{pgfscope}%
\begin{pgfscope}%
\pgfsetbuttcap%
\pgfsetmiterjoin%
\definecolor{currentfill}{rgb}{1.000000,1.000000,1.000000}%
\pgfsetfillcolor{currentfill}%
\pgfsetlinewidth{0.000000pt}%
\definecolor{currentstroke}{rgb}{0.000000,0.000000,0.000000}%
\pgfsetstrokecolor{currentstroke}%
\pgfsetstrokeopacity{0.000000}%
\pgfsetdash{}{0pt}%
\pgfpathmoveto{\pgfqpoint{0.759375in}{0.386111in}}%
\pgfpathlineto{\pgfqpoint{5.601389in}{0.386111in}}%
\pgfpathlineto{\pgfqpoint{5.601389in}{1.077926in}}%
\pgfpathlineto{\pgfqpoint{0.759375in}{1.077926in}}%
\pgfpathclose%
\pgfusepath{fill}%
\end{pgfscope}%
\begin{pgfscope}%
\pgfsetbuttcap%
\pgfsetroundjoin%
\definecolor{currentfill}{rgb}{0.000000,0.000000,0.000000}%
\pgfsetfillcolor{currentfill}%
\pgfsetlinewidth{0.803000pt}%
\definecolor{currentstroke}{rgb}{0.000000,0.000000,0.000000}%
\pgfsetstrokecolor{currentstroke}%
\pgfsetdash{}{0pt}%
\pgfsys@defobject{currentmarker}{\pgfqpoint{0.000000in}{-0.048611in}}{\pgfqpoint{0.000000in}{0.000000in}}{%
\pgfpathmoveto{\pgfqpoint{0.000000in}{0.000000in}}%
\pgfpathlineto{\pgfqpoint{0.000000in}{-0.048611in}}%
\pgfusepath{stroke,fill}%
}%
\begin{pgfscope}%
\pgfsys@transformshift{0.979467in}{0.386111in}%
\pgfsys@useobject{currentmarker}{}%
\end{pgfscope}%
\end{pgfscope}%
\begin{pgfscope}%
\definecolor{textcolor}{rgb}{0.000000,0.000000,0.000000}%
\pgfsetstrokecolor{textcolor}%
\pgfsetfillcolor{textcolor}%
\pgftext[x=0.979467in,y=0.288889in,,top]{\color{textcolor}\rmfamily\fontsize{10.000000}{12.000000}\selectfont 0.0}%
\end{pgfscope}%
\begin{pgfscope}%
\pgfsetbuttcap%
\pgfsetroundjoin%
\definecolor{currentfill}{rgb}{0.000000,0.000000,0.000000}%
\pgfsetfillcolor{currentfill}%
\pgfsetlinewidth{0.803000pt}%
\definecolor{currentstroke}{rgb}{0.000000,0.000000,0.000000}%
\pgfsetstrokecolor{currentstroke}%
\pgfsetdash{}{0pt}%
\pgfsys@defobject{currentmarker}{\pgfqpoint{0.000000in}{-0.048611in}}{\pgfqpoint{0.000000in}{0.000000in}}{%
\pgfpathmoveto{\pgfqpoint{0.000000in}{0.000000in}}%
\pgfpathlineto{\pgfqpoint{0.000000in}{-0.048611in}}%
\pgfusepath{stroke,fill}%
}%
\begin{pgfscope}%
\pgfsys@transformshift{1.529695in}{0.386111in}%
\pgfsys@useobject{currentmarker}{}%
\end{pgfscope}%
\end{pgfscope}%
\begin{pgfscope}%
\definecolor{textcolor}{rgb}{0.000000,0.000000,0.000000}%
\pgfsetstrokecolor{textcolor}%
\pgfsetfillcolor{textcolor}%
\pgftext[x=1.529695in,y=0.288889in,,top]{\color{textcolor}\rmfamily\fontsize{10.000000}{12.000000}\selectfont 31.9}%
\end{pgfscope}%
\begin{pgfscope}%
\pgfsetbuttcap%
\pgfsetroundjoin%
\definecolor{currentfill}{rgb}{0.000000,0.000000,0.000000}%
\pgfsetfillcolor{currentfill}%
\pgfsetlinewidth{0.803000pt}%
\definecolor{currentstroke}{rgb}{0.000000,0.000000,0.000000}%
\pgfsetstrokecolor{currentstroke}%
\pgfsetdash{}{0pt}%
\pgfsys@defobject{currentmarker}{\pgfqpoint{0.000000in}{-0.048611in}}{\pgfqpoint{0.000000in}{0.000000in}}{%
\pgfpathmoveto{\pgfqpoint{0.000000in}{0.000000in}}%
\pgfpathlineto{\pgfqpoint{0.000000in}{-0.048611in}}%
\pgfusepath{stroke,fill}%
}%
\begin{pgfscope}%
\pgfsys@transformshift{2.079924in}{0.386111in}%
\pgfsys@useobject{currentmarker}{}%
\end{pgfscope}%
\end{pgfscope}%
\begin{pgfscope}%
\definecolor{textcolor}{rgb}{0.000000,0.000000,0.000000}%
\pgfsetstrokecolor{textcolor}%
\pgfsetfillcolor{textcolor}%
\pgftext[x=2.079924in,y=0.288889in,,top]{\color{textcolor}\rmfamily\fontsize{10.000000}{12.000000}\selectfont 63.8}%
\end{pgfscope}%
\begin{pgfscope}%
\pgfsetbuttcap%
\pgfsetroundjoin%
\definecolor{currentfill}{rgb}{0.000000,0.000000,0.000000}%
\pgfsetfillcolor{currentfill}%
\pgfsetlinewidth{0.803000pt}%
\definecolor{currentstroke}{rgb}{0.000000,0.000000,0.000000}%
\pgfsetstrokecolor{currentstroke}%
\pgfsetdash{}{0pt}%
\pgfsys@defobject{currentmarker}{\pgfqpoint{0.000000in}{-0.048611in}}{\pgfqpoint{0.000000in}{0.000000in}}{%
\pgfpathmoveto{\pgfqpoint{0.000000in}{0.000000in}}%
\pgfpathlineto{\pgfqpoint{0.000000in}{-0.048611in}}%
\pgfusepath{stroke,fill}%
}%
\begin{pgfscope}%
\pgfsys@transformshift{2.630153in}{0.386111in}%
\pgfsys@useobject{currentmarker}{}%
\end{pgfscope}%
\end{pgfscope}%
\begin{pgfscope}%
\definecolor{textcolor}{rgb}{0.000000,0.000000,0.000000}%
\pgfsetstrokecolor{textcolor}%
\pgfsetfillcolor{textcolor}%
\pgftext[x=2.630153in,y=0.288889in,,top]{\color{textcolor}\rmfamily\fontsize{10.000000}{12.000000}\selectfont 95.6}%
\end{pgfscope}%
\begin{pgfscope}%
\pgfsetbuttcap%
\pgfsetroundjoin%
\definecolor{currentfill}{rgb}{0.000000,0.000000,0.000000}%
\pgfsetfillcolor{currentfill}%
\pgfsetlinewidth{0.803000pt}%
\definecolor{currentstroke}{rgb}{0.000000,0.000000,0.000000}%
\pgfsetstrokecolor{currentstroke}%
\pgfsetdash{}{0pt}%
\pgfsys@defobject{currentmarker}{\pgfqpoint{0.000000in}{-0.048611in}}{\pgfqpoint{0.000000in}{0.000000in}}{%
\pgfpathmoveto{\pgfqpoint{0.000000in}{0.000000in}}%
\pgfpathlineto{\pgfqpoint{0.000000in}{-0.048611in}}%
\pgfusepath{stroke,fill}%
}%
\begin{pgfscope}%
\pgfsys@transformshift{3.180382in}{0.386111in}%
\pgfsys@useobject{currentmarker}{}%
\end{pgfscope}%
\end{pgfscope}%
\begin{pgfscope}%
\definecolor{textcolor}{rgb}{0.000000,0.000000,0.000000}%
\pgfsetstrokecolor{textcolor}%
\pgfsetfillcolor{textcolor}%
\pgftext[x=3.180382in,y=0.288889in,,top]{\color{textcolor}\rmfamily\fontsize{10.000000}{12.000000}\selectfont 127.5}%
\end{pgfscope}%
\begin{pgfscope}%
\pgfsetbuttcap%
\pgfsetroundjoin%
\definecolor{currentfill}{rgb}{0.000000,0.000000,0.000000}%
\pgfsetfillcolor{currentfill}%
\pgfsetlinewidth{0.803000pt}%
\definecolor{currentstroke}{rgb}{0.000000,0.000000,0.000000}%
\pgfsetstrokecolor{currentstroke}%
\pgfsetdash{}{0pt}%
\pgfsys@defobject{currentmarker}{\pgfqpoint{0.000000in}{-0.048611in}}{\pgfqpoint{0.000000in}{0.000000in}}{%
\pgfpathmoveto{\pgfqpoint{0.000000in}{0.000000in}}%
\pgfpathlineto{\pgfqpoint{0.000000in}{-0.048611in}}%
\pgfusepath{stroke,fill}%
}%
\begin{pgfscope}%
\pgfsys@transformshift{3.730611in}{0.386111in}%
\pgfsys@useobject{currentmarker}{}%
\end{pgfscope}%
\end{pgfscope}%
\begin{pgfscope}%
\definecolor{textcolor}{rgb}{0.000000,0.000000,0.000000}%
\pgfsetstrokecolor{textcolor}%
\pgfsetfillcolor{textcolor}%
\pgftext[x=3.730611in,y=0.288889in,,top]{\color{textcolor}\rmfamily\fontsize{10.000000}{12.000000}\selectfont 159.4}%
\end{pgfscope}%
\begin{pgfscope}%
\pgfsetbuttcap%
\pgfsetroundjoin%
\definecolor{currentfill}{rgb}{0.000000,0.000000,0.000000}%
\pgfsetfillcolor{currentfill}%
\pgfsetlinewidth{0.803000pt}%
\definecolor{currentstroke}{rgb}{0.000000,0.000000,0.000000}%
\pgfsetstrokecolor{currentstroke}%
\pgfsetdash{}{0pt}%
\pgfsys@defobject{currentmarker}{\pgfqpoint{0.000000in}{-0.048611in}}{\pgfqpoint{0.000000in}{0.000000in}}{%
\pgfpathmoveto{\pgfqpoint{0.000000in}{0.000000in}}%
\pgfpathlineto{\pgfqpoint{0.000000in}{-0.048611in}}%
\pgfusepath{stroke,fill}%
}%
\begin{pgfscope}%
\pgfsys@transformshift{4.280840in}{0.386111in}%
\pgfsys@useobject{currentmarker}{}%
\end{pgfscope}%
\end{pgfscope}%
\begin{pgfscope}%
\definecolor{textcolor}{rgb}{0.000000,0.000000,0.000000}%
\pgfsetstrokecolor{textcolor}%
\pgfsetfillcolor{textcolor}%
\pgftext[x=4.280840in,y=0.288889in,,top]{\color{textcolor}\rmfamily\fontsize{10.000000}{12.000000}\selectfont 191.2}%
\end{pgfscope}%
\begin{pgfscope}%
\pgfsetbuttcap%
\pgfsetroundjoin%
\definecolor{currentfill}{rgb}{0.000000,0.000000,0.000000}%
\pgfsetfillcolor{currentfill}%
\pgfsetlinewidth{0.803000pt}%
\definecolor{currentstroke}{rgb}{0.000000,0.000000,0.000000}%
\pgfsetstrokecolor{currentstroke}%
\pgfsetdash{}{0pt}%
\pgfsys@defobject{currentmarker}{\pgfqpoint{0.000000in}{-0.048611in}}{\pgfqpoint{0.000000in}{0.000000in}}{%
\pgfpathmoveto{\pgfqpoint{0.000000in}{0.000000in}}%
\pgfpathlineto{\pgfqpoint{0.000000in}{-0.048611in}}%
\pgfusepath{stroke,fill}%
}%
\begin{pgfscope}%
\pgfsys@transformshift{4.831068in}{0.386111in}%
\pgfsys@useobject{currentmarker}{}%
\end{pgfscope}%
\end{pgfscope}%
\begin{pgfscope}%
\definecolor{textcolor}{rgb}{0.000000,0.000000,0.000000}%
\pgfsetstrokecolor{textcolor}%
\pgfsetfillcolor{textcolor}%
\pgftext[x=4.831068in,y=0.288889in,,top]{\color{textcolor}\rmfamily\fontsize{10.000000}{12.000000}\selectfont 223.1}%
\end{pgfscope}%
\begin{pgfscope}%
\pgfsetbuttcap%
\pgfsetroundjoin%
\definecolor{currentfill}{rgb}{0.000000,0.000000,0.000000}%
\pgfsetfillcolor{currentfill}%
\pgfsetlinewidth{0.803000pt}%
\definecolor{currentstroke}{rgb}{0.000000,0.000000,0.000000}%
\pgfsetstrokecolor{currentstroke}%
\pgfsetdash{}{0pt}%
\pgfsys@defobject{currentmarker}{\pgfqpoint{0.000000in}{-0.048611in}}{\pgfqpoint{0.000000in}{0.000000in}}{%
\pgfpathmoveto{\pgfqpoint{0.000000in}{0.000000in}}%
\pgfpathlineto{\pgfqpoint{0.000000in}{-0.048611in}}%
\pgfusepath{stroke,fill}%
}%
\begin{pgfscope}%
\pgfsys@transformshift{5.381297in}{0.386111in}%
\pgfsys@useobject{currentmarker}{}%
\end{pgfscope}%
\end{pgfscope}%
\begin{pgfscope}%
\definecolor{textcolor}{rgb}{0.000000,0.000000,0.000000}%
\pgfsetstrokecolor{textcolor}%
\pgfsetfillcolor{textcolor}%
\pgftext[x=5.381297in,y=0.288889in,,top]{\color{textcolor}\rmfamily\fontsize{10.000000}{12.000000}\selectfont 255.0}%
\end{pgfscope}%
\begin{pgfscope}%
\pgfsetbuttcap%
\pgfsetroundjoin%
\definecolor{currentfill}{rgb}{0.000000,0.000000,0.000000}%
\pgfsetfillcolor{currentfill}%
\pgfsetlinewidth{0.803000pt}%
\definecolor{currentstroke}{rgb}{0.000000,0.000000,0.000000}%
\pgfsetstrokecolor{currentstroke}%
\pgfsetdash{}{0pt}%
\pgfsys@defobject{currentmarker}{\pgfqpoint{-0.048611in}{0.000000in}}{\pgfqpoint{0.000000in}{0.000000in}}{%
\pgfpathmoveto{\pgfqpoint{0.000000in}{0.000000in}}%
\pgfpathlineto{\pgfqpoint{-0.048611in}{0.000000in}}%
\pgfusepath{stroke,fill}%
}%
\begin{pgfscope}%
\pgfsys@transformshift{0.759375in}{0.414089in}%
\pgfsys@useobject{currentmarker}{}%
\end{pgfscope}%
\end{pgfscope}%
\begin{pgfscope}%
\definecolor{textcolor}{rgb}{0.000000,0.000000,0.000000}%
\pgfsetstrokecolor{textcolor}%
\pgfsetfillcolor{textcolor}%
\pgftext[x=0.592708in,y=0.365894in,left,base]{\color{textcolor}\rmfamily\fontsize{10.000000}{12.000000}\selectfont 0}%
\end{pgfscope}%
\begin{pgfscope}%
\pgfsetbuttcap%
\pgfsetroundjoin%
\definecolor{currentfill}{rgb}{0.000000,0.000000,0.000000}%
\pgfsetfillcolor{currentfill}%
\pgfsetlinewidth{0.803000pt}%
\definecolor{currentstroke}{rgb}{0.000000,0.000000,0.000000}%
\pgfsetstrokecolor{currentstroke}%
\pgfsetdash{}{0pt}%
\pgfsys@defobject{currentmarker}{\pgfqpoint{-0.048611in}{0.000000in}}{\pgfqpoint{0.000000in}{0.000000in}}{%
\pgfpathmoveto{\pgfqpoint{0.000000in}{0.000000in}}%
\pgfpathlineto{\pgfqpoint{-0.048611in}{0.000000in}}%
\pgfusepath{stroke,fill}%
}%
\begin{pgfscope}%
\pgfsys@transformshift{0.759375in}{1.030464in}%
\pgfsys@useobject{currentmarker}{}%
\end{pgfscope}%
\end{pgfscope}%
\begin{pgfscope}%
\definecolor{textcolor}{rgb}{0.000000,0.000000,0.000000}%
\pgfsetstrokecolor{textcolor}%
\pgfsetfillcolor{textcolor}%
\pgftext[x=0.523264in,y=0.982270in,left,base]{\color{textcolor}\rmfamily\fontsize{10.000000}{12.000000}\selectfont 20}%
\end{pgfscope}%
\begin{pgfscope}%
\pgfpathrectangle{\pgfqpoint{0.759375in}{0.386111in}}{\pgfqpoint{4.842014in}{0.691815in}}%
\pgfusepath{clip}%
\pgfsetrectcap%
\pgfsetroundjoin%
\pgfsetlinewidth{1.505625pt}%
\definecolor{currentstroke}{rgb}{0.172549,0.627451,0.172549}%
\pgfsetstrokecolor{currentstroke}%
\pgfsetdash{}{0pt}%
\pgfpathmoveto{\pgfqpoint{1.115184in}{0.417557in}}%
\pgfpathlineto{\pgfqpoint{1.274045in}{0.417557in}}%
\pgfpathlineto{\pgfqpoint{1.274045in}{0.418745in}}%
\pgfpathlineto{\pgfqpoint{1.591768in}{0.418745in}}%
\pgfpathlineto{\pgfqpoint{1.591768in}{0.421614in}}%
\pgfpathlineto{\pgfqpoint{1.909491in}{0.421614in}}%
\pgfpathlineto{\pgfqpoint{1.909491in}{0.429027in}}%
\pgfpathlineto{\pgfqpoint{2.227213in}{0.429027in}}%
\pgfpathlineto{\pgfqpoint{2.227213in}{0.442079in}}%
\pgfpathlineto{\pgfqpoint{2.544936in}{0.442079in}}%
\pgfpathlineto{\pgfqpoint{2.544936in}{0.461888in}}%
\pgfpathlineto{\pgfqpoint{2.862659in}{0.461888in}}%
\pgfpathlineto{\pgfqpoint{2.862659in}{0.491981in}}%
\pgfpathlineto{\pgfqpoint{3.180382in}{0.491981in}}%
\pgfpathlineto{\pgfqpoint{3.180382in}{0.529590in}}%
\pgfpathlineto{\pgfqpoint{3.498105in}{0.529590in}}%
\pgfpathlineto{\pgfqpoint{3.498105in}{0.579181in}}%
\pgfpathlineto{\pgfqpoint{3.815828in}{0.579181in}}%
\pgfpathlineto{\pgfqpoint{3.815828in}{0.640799in}}%
\pgfpathlineto{\pgfqpoint{4.133550in}{0.640799in}}%
\pgfpathlineto{\pgfqpoint{4.133550in}{0.717151in}}%
\pgfpathlineto{\pgfqpoint{4.451273in}{0.717151in}}%
\pgfpathlineto{\pgfqpoint{4.451273in}{0.808859in}}%
\pgfpathlineto{\pgfqpoint{4.768996in}{0.808859in}}%
\pgfpathlineto{\pgfqpoint{4.768996in}{0.918068in}}%
\pgfpathlineto{\pgfqpoint{5.086719in}{0.918068in}}%
\pgfpathlineto{\pgfqpoint{5.086719in}{1.046480in}}%
\pgfpathlineto{\pgfqpoint{5.245580in}{1.046480in}}%
\pgfusepath{stroke}%
\end{pgfscope}%
\begin{pgfscope}%
\pgfpathrectangle{\pgfqpoint{0.759375in}{0.386111in}}{\pgfqpoint{4.842014in}{0.691815in}}%
\pgfusepath{clip}%
\pgfsetbuttcap%
\pgfsetroundjoin%
\definecolor{currentfill}{rgb}{0.172549,0.627451,0.172549}%
\pgfsetfillcolor{currentfill}%
\pgfsetlinewidth{1.003750pt}%
\definecolor{currentstroke}{rgb}{0.172549,0.627451,0.172549}%
\pgfsetstrokecolor{currentstroke}%
\pgfsetdash{}{0pt}%
\pgfsys@defobject{currentmarker}{\pgfqpoint{-0.041667in}{-0.041667in}}{\pgfqpoint{0.041667in}{0.041667in}}{%
\pgfpathmoveto{\pgfqpoint{0.000000in}{-0.041667in}}%
\pgfpathcurveto{\pgfqpoint{0.011050in}{-0.041667in}}{\pgfqpoint{0.021649in}{-0.037276in}}{\pgfqpoint{0.029463in}{-0.029463in}}%
\pgfpathcurveto{\pgfqpoint{0.037276in}{-0.021649in}}{\pgfqpoint{0.041667in}{-0.011050in}}{\pgfqpoint{0.041667in}{0.000000in}}%
\pgfpathcurveto{\pgfqpoint{0.041667in}{0.011050in}}{\pgfqpoint{0.037276in}{0.021649in}}{\pgfqpoint{0.029463in}{0.029463in}}%
\pgfpathcurveto{\pgfqpoint{0.021649in}{0.037276in}}{\pgfqpoint{0.011050in}{0.041667in}}{\pgfqpoint{0.000000in}{0.041667in}}%
\pgfpathcurveto{\pgfqpoint{-0.011050in}{0.041667in}}{\pgfqpoint{-0.021649in}{0.037276in}}{\pgfqpoint{-0.029463in}{0.029463in}}%
\pgfpathcurveto{\pgfqpoint{-0.037276in}{0.021649in}}{\pgfqpoint{-0.041667in}{0.011050in}}{\pgfqpoint{-0.041667in}{0.000000in}}%
\pgfpathcurveto{\pgfqpoint{-0.041667in}{-0.011050in}}{\pgfqpoint{-0.037276in}{-0.021649in}}{\pgfqpoint{-0.029463in}{-0.029463in}}%
\pgfpathcurveto{\pgfqpoint{-0.021649in}{-0.037276in}}{\pgfqpoint{-0.011050in}{-0.041667in}}{\pgfqpoint{0.000000in}{-0.041667in}}%
\pgfpathclose%
\pgfusepath{stroke,fill}%
}%
\begin{pgfscope}%
\pgfsys@transformshift{1.115184in}{0.417557in}%
\pgfsys@useobject{currentmarker}{}%
\end{pgfscope}%
\begin{pgfscope}%
\pgfsys@transformshift{1.432906in}{0.418745in}%
\pgfsys@useobject{currentmarker}{}%
\end{pgfscope}%
\begin{pgfscope}%
\pgfsys@transformshift{1.750629in}{0.421614in}%
\pgfsys@useobject{currentmarker}{}%
\end{pgfscope}%
\begin{pgfscope}%
\pgfsys@transformshift{2.068352in}{0.429027in}%
\pgfsys@useobject{currentmarker}{}%
\end{pgfscope}%
\begin{pgfscope}%
\pgfsys@transformshift{2.386075in}{0.442079in}%
\pgfsys@useobject{currentmarker}{}%
\end{pgfscope}%
\begin{pgfscope}%
\pgfsys@transformshift{2.703798in}{0.461888in}%
\pgfsys@useobject{currentmarker}{}%
\end{pgfscope}%
\begin{pgfscope}%
\pgfsys@transformshift{3.021521in}{0.491981in}%
\pgfsys@useobject{currentmarker}{}%
\end{pgfscope}%
\begin{pgfscope}%
\pgfsys@transformshift{3.339243in}{0.529590in}%
\pgfsys@useobject{currentmarker}{}%
\end{pgfscope}%
\begin{pgfscope}%
\pgfsys@transformshift{3.656966in}{0.579181in}%
\pgfsys@useobject{currentmarker}{}%
\end{pgfscope}%
\begin{pgfscope}%
\pgfsys@transformshift{3.974689in}{0.640799in}%
\pgfsys@useobject{currentmarker}{}%
\end{pgfscope}%
\begin{pgfscope}%
\pgfsys@transformshift{4.292412in}{0.717151in}%
\pgfsys@useobject{currentmarker}{}%
\end{pgfscope}%
\begin{pgfscope}%
\pgfsys@transformshift{4.610135in}{0.808859in}%
\pgfsys@useobject{currentmarker}{}%
\end{pgfscope}%
\begin{pgfscope}%
\pgfsys@transformshift{4.927857in}{0.918068in}%
\pgfsys@useobject{currentmarker}{}%
\end{pgfscope}%
\begin{pgfscope}%
\pgfsys@transformshift{5.245580in}{1.046480in}%
\pgfsys@useobject{currentmarker}{}%
\end{pgfscope}%
\end{pgfscope}%
\begin{pgfscope}%
\pgfsetrectcap%
\pgfsetmiterjoin%
\pgfsetlinewidth{0.803000pt}%
\definecolor{currentstroke}{rgb}{0.000000,0.000000,0.000000}%
\pgfsetstrokecolor{currentstroke}%
\pgfsetdash{}{0pt}%
\pgfpathmoveto{\pgfqpoint{0.759375in}{0.386111in}}%
\pgfpathlineto{\pgfqpoint{0.759375in}{1.077926in}}%
\pgfusepath{stroke}%
\end{pgfscope}%
\begin{pgfscope}%
\pgfsetrectcap%
\pgfsetmiterjoin%
\pgfsetlinewidth{0.803000pt}%
\definecolor{currentstroke}{rgb}{0.000000,0.000000,0.000000}%
\pgfsetstrokecolor{currentstroke}%
\pgfsetdash{}{0pt}%
\pgfpathmoveto{\pgfqpoint{5.601389in}{0.386111in}}%
\pgfpathlineto{\pgfqpoint{5.601389in}{1.077926in}}%
\pgfusepath{stroke}%
\end{pgfscope}%
\begin{pgfscope}%
\pgfsetrectcap%
\pgfsetmiterjoin%
\pgfsetlinewidth{0.803000pt}%
\definecolor{currentstroke}{rgb}{0.000000,0.000000,0.000000}%
\pgfsetstrokecolor{currentstroke}%
\pgfsetdash{}{0pt}%
\pgfpathmoveto{\pgfqpoint{0.759375in}{0.386111in}}%
\pgfpathlineto{\pgfqpoint{5.601389in}{0.386111in}}%
\pgfusepath{stroke}%
\end{pgfscope}%
\begin{pgfscope}%
\pgfsetrectcap%
\pgfsetmiterjoin%
\pgfsetlinewidth{0.803000pt}%
\definecolor{currentstroke}{rgb}{0.000000,0.000000,0.000000}%
\pgfsetstrokecolor{currentstroke}%
\pgfsetdash{}{0pt}%
\pgfpathmoveto{\pgfqpoint{0.759375in}{1.077926in}}%
\pgfpathlineto{\pgfqpoint{5.601389in}{1.077926in}}%
\pgfusepath{stroke}%
\end{pgfscope}%
\end{pgfpicture}%
\makeatother%
\endgroup%

    \caption{Multiskalenanalyse von $x^3 + r$ mit db2 Wavelet\label{polynomials:noise:db2_multi}}
\end{figure}

Im Vergleich dazu sind in \autoref{polynomials:noise:average} die direkt
berechnete zweite Ableitung und die zweite Ableitungen nach der
Mittelwertbildung über 2, 4 und 8 Werte zu sehen.

\begin{figure}
    \centering
    %% Creator: Matplotlib, PGF backend
%%
%% To include the figure in your LaTeX document, write
%%   \input{<filename>.pgf}
%%
%% Make sure the required packages are loaded in your preamble
%%   \usepackage{pgf}
%%
%% Figures using additional raster images can only be included by \input if
%% they are in the same directory as the main LaTeX file. For loading figures
%% from other directories you can use the `import` package
%%   \usepackage{import}
%% and then include the figures with
%%   \import{<path to file>}{<filename>.pgf}
%%
%% Matplotlib used the following preamble
%%   \usepackage{fontspec}
%%
\begingroup%
\makeatletter%
\begin{pgfpicture}%
\pgfpathrectangle{\pgfpointorigin}{\pgfqpoint{5.800000in}{6.000000in}}%
\pgfusepath{use as bounding box, clip}%
\begin{pgfscope}%
\pgfsetbuttcap%
\pgfsetmiterjoin%
\definecolor{currentfill}{rgb}{1.000000,1.000000,1.000000}%
\pgfsetfillcolor{currentfill}%
\pgfsetlinewidth{0.000000pt}%
\definecolor{currentstroke}{rgb}{1.000000,1.000000,1.000000}%
\pgfsetstrokecolor{currentstroke}%
\pgfsetdash{}{0pt}%
\pgfpathmoveto{\pgfqpoint{0.000000in}{0.000000in}}%
\pgfpathlineto{\pgfqpoint{5.800000in}{0.000000in}}%
\pgfpathlineto{\pgfqpoint{5.800000in}{6.000000in}}%
\pgfpathlineto{\pgfqpoint{0.000000in}{6.000000in}}%
\pgfpathclose%
\pgfusepath{fill}%
\end{pgfscope}%
\begin{pgfscope}%
\pgfsetbuttcap%
\pgfsetmiterjoin%
\definecolor{currentfill}{rgb}{1.000000,1.000000,1.000000}%
\pgfsetfillcolor{currentfill}%
\pgfsetlinewidth{0.000000pt}%
\definecolor{currentstroke}{rgb}{0.000000,0.000000,0.000000}%
\pgfsetstrokecolor{currentstroke}%
\pgfsetstrokeopacity{0.000000}%
\pgfsetdash{}{0pt}%
\pgfpathmoveto{\pgfqpoint{0.725000in}{4.295217in}}%
\pgfpathlineto{\pgfqpoint{5.220000in}{4.295217in}}%
\pgfpathlineto{\pgfqpoint{5.220000in}{5.280000in}}%
\pgfpathlineto{\pgfqpoint{0.725000in}{5.280000in}}%
\pgfpathclose%
\pgfusepath{fill}%
\end{pgfscope}%
\begin{pgfscope}%
\pgfsetbuttcap%
\pgfsetroundjoin%
\definecolor{currentfill}{rgb}{0.000000,0.000000,0.000000}%
\pgfsetfillcolor{currentfill}%
\pgfsetlinewidth{0.803000pt}%
\definecolor{currentstroke}{rgb}{0.000000,0.000000,0.000000}%
\pgfsetstrokecolor{currentstroke}%
\pgfsetdash{}{0pt}%
\pgfsys@defobject{currentmarker}{\pgfqpoint{0.000000in}{-0.048611in}}{\pgfqpoint{0.000000in}{0.000000in}}{%
\pgfpathmoveto{\pgfqpoint{0.000000in}{0.000000in}}%
\pgfpathlineto{\pgfqpoint{0.000000in}{-0.048611in}}%
\pgfusepath{stroke,fill}%
}%
\begin{pgfscope}%
\pgfsys@transformshift{0.913167in}{4.295217in}%
\pgfsys@useobject{currentmarker}{}%
\end{pgfscope}%
\end{pgfscope}%
\begin{pgfscope}%
\pgfsetbuttcap%
\pgfsetroundjoin%
\definecolor{currentfill}{rgb}{0.000000,0.000000,0.000000}%
\pgfsetfillcolor{currentfill}%
\pgfsetlinewidth{0.803000pt}%
\definecolor{currentstroke}{rgb}{0.000000,0.000000,0.000000}%
\pgfsetstrokecolor{currentstroke}%
\pgfsetdash{}{0pt}%
\pgfsys@defobject{currentmarker}{\pgfqpoint{0.000000in}{-0.048611in}}{\pgfqpoint{0.000000in}{0.000000in}}{%
\pgfpathmoveto{\pgfqpoint{0.000000in}{0.000000in}}%
\pgfpathlineto{\pgfqpoint{0.000000in}{-0.048611in}}%
\pgfusepath{stroke,fill}%
}%
\begin{pgfscope}%
\pgfsys@transformshift{1.720748in}{4.295217in}%
\pgfsys@useobject{currentmarker}{}%
\end{pgfscope}%
\end{pgfscope}%
\begin{pgfscope}%
\pgfsetbuttcap%
\pgfsetroundjoin%
\definecolor{currentfill}{rgb}{0.000000,0.000000,0.000000}%
\pgfsetfillcolor{currentfill}%
\pgfsetlinewidth{0.803000pt}%
\definecolor{currentstroke}{rgb}{0.000000,0.000000,0.000000}%
\pgfsetstrokecolor{currentstroke}%
\pgfsetdash{}{0pt}%
\pgfsys@defobject{currentmarker}{\pgfqpoint{0.000000in}{-0.048611in}}{\pgfqpoint{0.000000in}{0.000000in}}{%
\pgfpathmoveto{\pgfqpoint{0.000000in}{0.000000in}}%
\pgfpathlineto{\pgfqpoint{0.000000in}{-0.048611in}}%
\pgfusepath{stroke,fill}%
}%
\begin{pgfscope}%
\pgfsys@transformshift{2.528330in}{4.295217in}%
\pgfsys@useobject{currentmarker}{}%
\end{pgfscope}%
\end{pgfscope}%
\begin{pgfscope}%
\pgfsetbuttcap%
\pgfsetroundjoin%
\definecolor{currentfill}{rgb}{0.000000,0.000000,0.000000}%
\pgfsetfillcolor{currentfill}%
\pgfsetlinewidth{0.803000pt}%
\definecolor{currentstroke}{rgb}{0.000000,0.000000,0.000000}%
\pgfsetstrokecolor{currentstroke}%
\pgfsetdash{}{0pt}%
\pgfsys@defobject{currentmarker}{\pgfqpoint{0.000000in}{-0.048611in}}{\pgfqpoint{0.000000in}{0.000000in}}{%
\pgfpathmoveto{\pgfqpoint{0.000000in}{0.000000in}}%
\pgfpathlineto{\pgfqpoint{0.000000in}{-0.048611in}}%
\pgfusepath{stroke,fill}%
}%
\begin{pgfscope}%
\pgfsys@transformshift{3.335912in}{4.295217in}%
\pgfsys@useobject{currentmarker}{}%
\end{pgfscope}%
\end{pgfscope}%
\begin{pgfscope}%
\pgfsetbuttcap%
\pgfsetroundjoin%
\definecolor{currentfill}{rgb}{0.000000,0.000000,0.000000}%
\pgfsetfillcolor{currentfill}%
\pgfsetlinewidth{0.803000pt}%
\definecolor{currentstroke}{rgb}{0.000000,0.000000,0.000000}%
\pgfsetstrokecolor{currentstroke}%
\pgfsetdash{}{0pt}%
\pgfsys@defobject{currentmarker}{\pgfqpoint{0.000000in}{-0.048611in}}{\pgfqpoint{0.000000in}{0.000000in}}{%
\pgfpathmoveto{\pgfqpoint{0.000000in}{0.000000in}}%
\pgfpathlineto{\pgfqpoint{0.000000in}{-0.048611in}}%
\pgfusepath{stroke,fill}%
}%
\begin{pgfscope}%
\pgfsys@transformshift{4.143494in}{4.295217in}%
\pgfsys@useobject{currentmarker}{}%
\end{pgfscope}%
\end{pgfscope}%
\begin{pgfscope}%
\pgfsetbuttcap%
\pgfsetroundjoin%
\definecolor{currentfill}{rgb}{0.000000,0.000000,0.000000}%
\pgfsetfillcolor{currentfill}%
\pgfsetlinewidth{0.803000pt}%
\definecolor{currentstroke}{rgb}{0.000000,0.000000,0.000000}%
\pgfsetstrokecolor{currentstroke}%
\pgfsetdash{}{0pt}%
\pgfsys@defobject{currentmarker}{\pgfqpoint{0.000000in}{-0.048611in}}{\pgfqpoint{0.000000in}{0.000000in}}{%
\pgfpathmoveto{\pgfqpoint{0.000000in}{0.000000in}}%
\pgfpathlineto{\pgfqpoint{0.000000in}{-0.048611in}}%
\pgfusepath{stroke,fill}%
}%
\begin{pgfscope}%
\pgfsys@transformshift{4.951075in}{4.295217in}%
\pgfsys@useobject{currentmarker}{}%
\end{pgfscope}%
\end{pgfscope}%
\begin{pgfscope}%
\pgfsetbuttcap%
\pgfsetroundjoin%
\definecolor{currentfill}{rgb}{0.000000,0.000000,0.000000}%
\pgfsetfillcolor{currentfill}%
\pgfsetlinewidth{0.803000pt}%
\definecolor{currentstroke}{rgb}{0.000000,0.000000,0.000000}%
\pgfsetstrokecolor{currentstroke}%
\pgfsetdash{}{0pt}%
\pgfsys@defobject{currentmarker}{\pgfqpoint{-0.048611in}{0.000000in}}{\pgfqpoint{0.000000in}{0.000000in}}{%
\pgfpathmoveto{\pgfqpoint{0.000000in}{0.000000in}}%
\pgfpathlineto{\pgfqpoint{-0.048611in}{0.000000in}}%
\pgfusepath{stroke,fill}%
}%
\begin{pgfscope}%
\pgfsys@transformshift{0.725000in}{4.506755in}%
\pgfsys@useobject{currentmarker}{}%
\end{pgfscope}%
\end{pgfscope}%
\begin{pgfscope}%
\definecolor{textcolor}{rgb}{0.000000,0.000000,0.000000}%
\pgfsetstrokecolor{textcolor}%
\pgfsetfillcolor{textcolor}%
\pgftext[x=0.326222in,y=4.468199in,left,base]{\color{textcolor}\rmfamily\fontsize{8.000000}{9.600000}\selectfont −0.05}%
\end{pgfscope}%
\begin{pgfscope}%
\pgfsetbuttcap%
\pgfsetroundjoin%
\definecolor{currentfill}{rgb}{0.000000,0.000000,0.000000}%
\pgfsetfillcolor{currentfill}%
\pgfsetlinewidth{0.803000pt}%
\definecolor{currentstroke}{rgb}{0.000000,0.000000,0.000000}%
\pgfsetstrokecolor{currentstroke}%
\pgfsetdash{}{0pt}%
\pgfsys@defobject{currentmarker}{\pgfqpoint{-0.048611in}{0.000000in}}{\pgfqpoint{0.000000in}{0.000000in}}{%
\pgfpathmoveto{\pgfqpoint{0.000000in}{0.000000in}}%
\pgfpathlineto{\pgfqpoint{-0.048611in}{0.000000in}}%
\pgfusepath{stroke,fill}%
}%
\begin{pgfscope}%
\pgfsys@transformshift{0.725000in}{4.766278in}%
\pgfsys@useobject{currentmarker}{}%
\end{pgfscope}%
\end{pgfscope}%
\begin{pgfscope}%
\definecolor{textcolor}{rgb}{0.000000,0.000000,0.000000}%
\pgfsetstrokecolor{textcolor}%
\pgfsetfillcolor{textcolor}%
\pgftext[x=0.418000in,y=4.727722in,left,base]{\color{textcolor}\rmfamily\fontsize{8.000000}{9.600000}\selectfont 0.00}%
\end{pgfscope}%
\begin{pgfscope}%
\pgfsetbuttcap%
\pgfsetroundjoin%
\definecolor{currentfill}{rgb}{0.000000,0.000000,0.000000}%
\pgfsetfillcolor{currentfill}%
\pgfsetlinewidth{0.803000pt}%
\definecolor{currentstroke}{rgb}{0.000000,0.000000,0.000000}%
\pgfsetstrokecolor{currentstroke}%
\pgfsetdash{}{0pt}%
\pgfsys@defobject{currentmarker}{\pgfqpoint{-0.048611in}{0.000000in}}{\pgfqpoint{0.000000in}{0.000000in}}{%
\pgfpathmoveto{\pgfqpoint{0.000000in}{0.000000in}}%
\pgfpathlineto{\pgfqpoint{-0.048611in}{0.000000in}}%
\pgfusepath{stroke,fill}%
}%
\begin{pgfscope}%
\pgfsys@transformshift{0.725000in}{5.025801in}%
\pgfsys@useobject{currentmarker}{}%
\end{pgfscope}%
\end{pgfscope}%
\begin{pgfscope}%
\definecolor{textcolor}{rgb}{0.000000,0.000000,0.000000}%
\pgfsetstrokecolor{textcolor}%
\pgfsetfillcolor{textcolor}%
\pgftext[x=0.418000in,y=4.987245in,left,base]{\color{textcolor}\rmfamily\fontsize{8.000000}{9.600000}\selectfont 0.05}%
\end{pgfscope}%
\begin{pgfscope}%
\pgfsetbuttcap%
\pgfsetroundjoin%
\definecolor{currentfill}{rgb}{0.000000,0.000000,0.000000}%
\pgfsetfillcolor{currentfill}%
\pgfsetlinewidth{0.803000pt}%
\definecolor{currentstroke}{rgb}{0.000000,0.000000,0.000000}%
\pgfsetstrokecolor{currentstroke}%
\pgfsetdash{}{0pt}%
\pgfsys@defobject{currentmarker}{\pgfqpoint{-0.048611in}{0.000000in}}{\pgfqpoint{0.000000in}{0.000000in}}{%
\pgfpathmoveto{\pgfqpoint{0.000000in}{0.000000in}}%
\pgfpathlineto{\pgfqpoint{-0.048611in}{0.000000in}}%
\pgfusepath{stroke,fill}%
}%
\begin{pgfscope}%
\pgfsys@transformshift{0.725000in}{5.285324in}%
\pgfsys@useobject{currentmarker}{}%
\end{pgfscope}%
\end{pgfscope}%
\begin{pgfscope}%
\definecolor{textcolor}{rgb}{0.000000,0.000000,0.000000}%
\pgfsetstrokecolor{textcolor}%
\pgfsetfillcolor{textcolor}%
\pgftext[x=0.418000in,y=5.246768in,left,base]{\color{textcolor}\rmfamily\fontsize{8.000000}{9.600000}\selectfont 0.10}%
\end{pgfscope}%
\begin{pgfscope}%
\pgfpathrectangle{\pgfqpoint{0.725000in}{4.295217in}}{\pgfqpoint{4.495000in}{0.984783in}}%
\pgfusepath{clip}%
\pgfsetrectcap%
\pgfsetroundjoin%
\pgfsetlinewidth{1.505625pt}%
\definecolor{currentstroke}{rgb}{1.000000,0.498039,0.054902}%
\pgfsetstrokecolor{currentstroke}%
\pgfsetdash{}{0pt}%
\pgfpathmoveto{\pgfqpoint{0.929318in}{4.926811in}}%
\pgfpathlineto{\pgfqpoint{0.937394in}{4.926811in}}%
\pgfpathlineto{\pgfqpoint{0.937394in}{4.755188in}}%
\pgfpathlineto{\pgfqpoint{0.953546in}{4.755188in}}%
\pgfpathlineto{\pgfqpoint{0.953546in}{4.684607in}}%
\pgfpathlineto{\pgfqpoint{0.969697in}{4.684607in}}%
\pgfpathlineto{\pgfqpoint{0.969697in}{4.709591in}}%
\pgfpathlineto{\pgfqpoint{0.985849in}{4.709591in}}%
\pgfpathlineto{\pgfqpoint{0.985849in}{4.903863in}}%
\pgfpathlineto{\pgfqpoint{1.002001in}{4.903863in}}%
\pgfpathlineto{\pgfqpoint{1.002001in}{4.838744in}}%
\pgfpathlineto{\pgfqpoint{1.018152in}{4.838744in}}%
\pgfpathlineto{\pgfqpoint{1.018152in}{4.724658in}}%
\pgfpathlineto{\pgfqpoint{1.034304in}{4.724658in}}%
\pgfpathlineto{\pgfqpoint{1.034304in}{4.649840in}}%
\pgfpathlineto{\pgfqpoint{1.050455in}{4.649840in}}%
\pgfpathlineto{\pgfqpoint{1.050455in}{4.799650in}}%
\pgfpathlineto{\pgfqpoint{1.066607in}{4.799650in}}%
\pgfpathlineto{\pgfqpoint{1.066607in}{4.776849in}}%
\pgfpathlineto{\pgfqpoint{1.082759in}{4.776849in}}%
\pgfpathlineto{\pgfqpoint{1.082759in}{4.754318in}}%
\pgfpathlineto{\pgfqpoint{1.098910in}{4.754318in}}%
\pgfpathlineto{\pgfqpoint{1.098910in}{4.671229in}}%
\pgfpathlineto{\pgfqpoint{1.115062in}{4.671229in}}%
\pgfpathlineto{\pgfqpoint{1.115062in}{5.050143in}}%
\pgfpathlineto{\pgfqpoint{1.131214in}{5.050143in}}%
\pgfpathlineto{\pgfqpoint{1.131214in}{4.526353in}}%
\pgfpathlineto{\pgfqpoint{1.147365in}{4.526353in}}%
\pgfpathlineto{\pgfqpoint{1.147365in}{4.803608in}}%
\pgfpathlineto{\pgfqpoint{1.163517in}{4.803608in}}%
\pgfpathlineto{\pgfqpoint{1.163517in}{4.915372in}}%
\pgfpathlineto{\pgfqpoint{1.179669in}{4.915372in}}%
\pgfpathlineto{\pgfqpoint{1.179669in}{4.559540in}}%
\pgfpathlineto{\pgfqpoint{1.195820in}{4.559540in}}%
\pgfpathlineto{\pgfqpoint{1.195820in}{4.903793in}}%
\pgfpathlineto{\pgfqpoint{1.211972in}{4.903793in}}%
\pgfpathlineto{\pgfqpoint{1.211972in}{4.799434in}}%
\pgfpathlineto{\pgfqpoint{1.228123in}{4.799434in}}%
\pgfpathlineto{\pgfqpoint{1.228123in}{4.570440in}}%
\pgfpathlineto{\pgfqpoint{1.244275in}{4.570440in}}%
\pgfpathlineto{\pgfqpoint{1.244275in}{4.815336in}}%
\pgfpathlineto{\pgfqpoint{1.260427in}{4.815336in}}%
\pgfpathlineto{\pgfqpoint{1.260427in}{4.858981in}}%
\pgfpathlineto{\pgfqpoint{1.276578in}{4.858981in}}%
\pgfpathlineto{\pgfqpoint{1.276578in}{4.800513in}}%
\pgfpathlineto{\pgfqpoint{1.292730in}{4.800513in}}%
\pgfpathlineto{\pgfqpoint{1.292730in}{4.820586in}}%
\pgfpathlineto{\pgfqpoint{1.308882in}{4.820586in}}%
\pgfpathlineto{\pgfqpoint{1.308882in}{4.434374in}}%
\pgfpathlineto{\pgfqpoint{1.325033in}{4.434374in}}%
\pgfpathlineto{\pgfqpoint{1.325033in}{5.163469in}}%
\pgfpathlineto{\pgfqpoint{1.341185in}{5.163469in}}%
\pgfpathlineto{\pgfqpoint{1.341185in}{4.469893in}}%
\pgfpathlineto{\pgfqpoint{1.357337in}{4.469893in}}%
\pgfpathlineto{\pgfqpoint{1.357337in}{4.963435in}}%
\pgfpathlineto{\pgfqpoint{1.373488in}{4.963435in}}%
\pgfpathlineto{\pgfqpoint{1.373488in}{4.734621in}}%
\pgfpathlineto{\pgfqpoint{1.389640in}{4.734621in}}%
\pgfpathlineto{\pgfqpoint{1.389640in}{4.696828in}}%
\pgfpathlineto{\pgfqpoint{1.405791in}{4.696828in}}%
\pgfpathlineto{\pgfqpoint{1.405791in}{4.733060in}}%
\pgfpathlineto{\pgfqpoint{1.421943in}{4.733060in}}%
\pgfpathlineto{\pgfqpoint{1.421943in}{4.872482in}}%
\pgfpathlineto{\pgfqpoint{1.438095in}{4.872482in}}%
\pgfpathlineto{\pgfqpoint{1.438095in}{4.670400in}}%
\pgfpathlineto{\pgfqpoint{1.454246in}{4.670400in}}%
\pgfpathlineto{\pgfqpoint{1.454246in}{4.732843in}}%
\pgfpathlineto{\pgfqpoint{1.470398in}{4.732843in}}%
\pgfpathlineto{\pgfqpoint{1.470398in}{4.925430in}}%
\pgfpathlineto{\pgfqpoint{1.486550in}{4.925430in}}%
\pgfpathlineto{\pgfqpoint{1.486550in}{4.641453in}}%
\pgfpathlineto{\pgfqpoint{1.502701in}{4.641453in}}%
\pgfpathlineto{\pgfqpoint{1.502701in}{4.998542in}}%
\pgfpathlineto{\pgfqpoint{1.518853in}{4.998542in}}%
\pgfpathlineto{\pgfqpoint{1.518853in}{4.532158in}}%
\pgfpathlineto{\pgfqpoint{1.535004in}{4.532158in}}%
\pgfpathlineto{\pgfqpoint{1.535004in}{4.644918in}}%
\pgfpathlineto{\pgfqpoint{1.551156in}{4.644918in}}%
\pgfpathlineto{\pgfqpoint{1.551156in}{5.008158in}}%
\pgfpathlineto{\pgfqpoint{1.567308in}{5.008158in}}%
\pgfpathlineto{\pgfqpoint{1.567308in}{4.660421in}}%
\pgfpathlineto{\pgfqpoint{1.583459in}{4.660421in}}%
\pgfpathlineto{\pgfqpoint{1.583459in}{5.004616in}}%
\pgfpathlineto{\pgfqpoint{1.599611in}{5.004616in}}%
\pgfpathlineto{\pgfqpoint{1.599611in}{4.481166in}}%
\pgfpathlineto{\pgfqpoint{1.615763in}{4.481166in}}%
\pgfpathlineto{\pgfqpoint{1.615763in}{4.916304in}}%
\pgfpathlineto{\pgfqpoint{1.631914in}{4.916304in}}%
\pgfpathlineto{\pgfqpoint{1.631914in}{4.605169in}}%
\pgfpathlineto{\pgfqpoint{1.648066in}{4.605169in}}%
\pgfpathlineto{\pgfqpoint{1.648066in}{4.924009in}}%
\pgfpathlineto{\pgfqpoint{1.664218in}{4.924009in}}%
\pgfpathlineto{\pgfqpoint{1.664218in}{4.732442in}}%
\pgfpathlineto{\pgfqpoint{1.680369in}{4.732442in}}%
\pgfpathlineto{\pgfqpoint{1.680369in}{4.614295in}}%
\pgfpathlineto{\pgfqpoint{1.696521in}{4.614295in}}%
\pgfpathlineto{\pgfqpoint{1.696521in}{5.105864in}}%
\pgfpathlineto{\pgfqpoint{1.712672in}{5.105864in}}%
\pgfpathlineto{\pgfqpoint{1.712672in}{4.358032in}}%
\pgfpathlineto{\pgfqpoint{1.728824in}{4.358032in}}%
\pgfpathlineto{\pgfqpoint{1.728824in}{5.132159in}}%
\pgfpathlineto{\pgfqpoint{1.744976in}{5.132159in}}%
\pgfpathlineto{\pgfqpoint{1.744976in}{4.395938in}}%
\pgfpathlineto{\pgfqpoint{1.761127in}{4.395938in}}%
\pgfpathlineto{\pgfqpoint{1.761127in}{5.217372in}}%
\pgfpathlineto{\pgfqpoint{1.777279in}{5.217372in}}%
\pgfpathlineto{\pgfqpoint{1.777279in}{4.517591in}}%
\pgfpathlineto{\pgfqpoint{1.793431in}{4.517591in}}%
\pgfpathlineto{\pgfqpoint{1.793431in}{4.690939in}}%
\pgfpathlineto{\pgfqpoint{1.809582in}{4.690939in}}%
\pgfpathlineto{\pgfqpoint{1.809582in}{4.877002in}}%
\pgfpathlineto{\pgfqpoint{1.825734in}{4.877002in}}%
\pgfpathlineto{\pgfqpoint{1.825734in}{4.559649in}}%
\pgfpathlineto{\pgfqpoint{1.841886in}{4.559649in}}%
\pgfpathlineto{\pgfqpoint{1.841886in}{5.027588in}}%
\pgfpathlineto{\pgfqpoint{1.858037in}{5.027588in}}%
\pgfpathlineto{\pgfqpoint{1.858037in}{4.703903in}}%
\pgfpathlineto{\pgfqpoint{1.874189in}{4.703903in}}%
\pgfpathlineto{\pgfqpoint{1.874189in}{4.811082in}}%
\pgfpathlineto{\pgfqpoint{1.890340in}{4.811082in}}%
\pgfpathlineto{\pgfqpoint{1.890340in}{4.690188in}}%
\pgfpathlineto{\pgfqpoint{1.906492in}{4.690188in}}%
\pgfpathlineto{\pgfqpoint{1.906492in}{4.790854in}}%
\pgfpathlineto{\pgfqpoint{1.922644in}{4.790854in}}%
\pgfpathlineto{\pgfqpoint{1.922644in}{4.665397in}}%
\pgfpathlineto{\pgfqpoint{1.938795in}{4.665397in}}%
\pgfpathlineto{\pgfqpoint{1.938795in}{5.018360in}}%
\pgfpathlineto{\pgfqpoint{1.954947in}{5.018360in}}%
\pgfpathlineto{\pgfqpoint{1.954947in}{4.552063in}}%
\pgfpathlineto{\pgfqpoint{1.971099in}{4.552063in}}%
\pgfpathlineto{\pgfqpoint{1.971099in}{4.900024in}}%
\pgfpathlineto{\pgfqpoint{1.987250in}{4.900024in}}%
\pgfpathlineto{\pgfqpoint{1.987250in}{4.546795in}}%
\pgfpathlineto{\pgfqpoint{2.003402in}{4.546795in}}%
\pgfpathlineto{\pgfqpoint{2.003402in}{5.071773in}}%
\pgfpathlineto{\pgfqpoint{2.019554in}{5.071773in}}%
\pgfpathlineto{\pgfqpoint{2.019554in}{4.586538in}}%
\pgfpathlineto{\pgfqpoint{2.035705in}{4.586538in}}%
\pgfpathlineto{\pgfqpoint{2.035705in}{4.812088in}}%
\pgfpathlineto{\pgfqpoint{2.051857in}{4.812088in}}%
\pgfpathlineto{\pgfqpoint{2.051857in}{4.576904in}}%
\pgfpathlineto{\pgfqpoint{2.068008in}{4.576904in}}%
\pgfpathlineto{\pgfqpoint{2.068008in}{4.853167in}}%
\pgfpathlineto{\pgfqpoint{2.084160in}{4.853167in}}%
\pgfpathlineto{\pgfqpoint{2.084160in}{5.037601in}}%
\pgfpathlineto{\pgfqpoint{2.100312in}{5.037601in}}%
\pgfpathlineto{\pgfqpoint{2.100312in}{4.390752in}}%
\pgfpathlineto{\pgfqpoint{2.116463in}{4.390752in}}%
\pgfpathlineto{\pgfqpoint{2.116463in}{5.148946in}}%
\pgfpathlineto{\pgfqpoint{2.132615in}{5.148946in}}%
\pgfpathlineto{\pgfqpoint{2.132615in}{4.430347in}}%
\pgfpathlineto{\pgfqpoint{2.148767in}{4.430347in}}%
\pgfpathlineto{\pgfqpoint{2.148767in}{4.947456in}}%
\pgfpathlineto{\pgfqpoint{2.164918in}{4.947456in}}%
\pgfpathlineto{\pgfqpoint{2.164918in}{4.718768in}}%
\pgfpathlineto{\pgfqpoint{2.181070in}{4.718768in}}%
\pgfpathlineto{\pgfqpoint{2.181070in}{4.802216in}}%
\pgfpathlineto{\pgfqpoint{2.197222in}{4.802216in}}%
\pgfpathlineto{\pgfqpoint{2.197222in}{4.770482in}}%
\pgfpathlineto{\pgfqpoint{2.213373in}{4.770482in}}%
\pgfpathlineto{\pgfqpoint{2.213373in}{4.778862in}}%
\pgfpathlineto{\pgfqpoint{2.229525in}{4.778862in}}%
\pgfpathlineto{\pgfqpoint{2.229525in}{4.812036in}}%
\pgfpathlineto{\pgfqpoint{2.245676in}{4.812036in}}%
\pgfpathlineto{\pgfqpoint{2.245676in}{4.679516in}}%
\pgfpathlineto{\pgfqpoint{2.261828in}{4.679516in}}%
\pgfpathlineto{\pgfqpoint{2.261828in}{4.607710in}}%
\pgfpathlineto{\pgfqpoint{2.277980in}{4.607710in}}%
\pgfpathlineto{\pgfqpoint{2.277980in}{5.087264in}}%
\pgfpathlineto{\pgfqpoint{2.294131in}{5.087264in}}%
\pgfpathlineto{\pgfqpoint{2.294131in}{4.450651in}}%
\pgfpathlineto{\pgfqpoint{2.310283in}{4.450651in}}%
\pgfpathlineto{\pgfqpoint{2.310283in}{5.086096in}}%
\pgfpathlineto{\pgfqpoint{2.326435in}{5.086096in}}%
\pgfpathlineto{\pgfqpoint{2.326435in}{4.667564in}}%
\pgfpathlineto{\pgfqpoint{2.342586in}{4.667564in}}%
\pgfpathlineto{\pgfqpoint{2.342586in}{4.570652in}}%
\pgfpathlineto{\pgfqpoint{2.358738in}{4.570652in}}%
\pgfpathlineto{\pgfqpoint{2.358738in}{4.803596in}}%
\pgfpathlineto{\pgfqpoint{2.374890in}{4.803596in}}%
\pgfpathlineto{\pgfqpoint{2.374890in}{4.900655in}}%
\pgfpathlineto{\pgfqpoint{2.391041in}{4.900655in}}%
\pgfpathlineto{\pgfqpoint{2.391041in}{4.772374in}}%
\pgfpathlineto{\pgfqpoint{2.407193in}{4.772374in}}%
\pgfpathlineto{\pgfqpoint{2.407193in}{4.818936in}}%
\pgfpathlineto{\pgfqpoint{2.423344in}{4.818936in}}%
\pgfpathlineto{\pgfqpoint{2.423344in}{4.741220in}}%
\pgfpathlineto{\pgfqpoint{2.439496in}{4.741220in}}%
\pgfpathlineto{\pgfqpoint{2.439496in}{4.607764in}}%
\pgfpathlineto{\pgfqpoint{2.455648in}{4.607764in}}%
\pgfpathlineto{\pgfqpoint{2.455648in}{5.005428in}}%
\pgfpathlineto{\pgfqpoint{2.471799in}{5.005428in}}%
\pgfpathlineto{\pgfqpoint{2.471799in}{4.504479in}}%
\pgfpathlineto{\pgfqpoint{2.487951in}{4.504479in}}%
\pgfpathlineto{\pgfqpoint{2.487951in}{4.955112in}}%
\pgfpathlineto{\pgfqpoint{2.504103in}{4.955112in}}%
\pgfpathlineto{\pgfqpoint{2.504103in}{4.794541in}}%
\pgfpathlineto{\pgfqpoint{2.520254in}{4.794541in}}%
\pgfpathlineto{\pgfqpoint{2.520254in}{4.687633in}}%
\pgfpathlineto{\pgfqpoint{2.536406in}{4.687633in}}%
\pgfpathlineto{\pgfqpoint{2.536406in}{4.696047in}}%
\pgfpathlineto{\pgfqpoint{2.552557in}{4.696047in}}%
\pgfpathlineto{\pgfqpoint{2.552557in}{4.750000in}}%
\pgfpathlineto{\pgfqpoint{2.568709in}{4.750000in}}%
\pgfpathlineto{\pgfqpoint{2.568709in}{4.859783in}}%
\pgfpathlineto{\pgfqpoint{2.584861in}{4.859783in}}%
\pgfpathlineto{\pgfqpoint{2.584861in}{4.899697in}}%
\pgfpathlineto{\pgfqpoint{2.601012in}{4.899697in}}%
\pgfpathlineto{\pgfqpoint{2.601012in}{4.550127in}}%
\pgfpathlineto{\pgfqpoint{2.617164in}{4.550127in}}%
\pgfpathlineto{\pgfqpoint{2.617164in}{4.849655in}}%
\pgfpathlineto{\pgfqpoint{2.633316in}{4.849655in}}%
\pgfpathlineto{\pgfqpoint{2.633316in}{4.688078in}}%
\pgfpathlineto{\pgfqpoint{2.649467in}{4.688078in}}%
\pgfpathlineto{\pgfqpoint{2.649467in}{4.785531in}}%
\pgfpathlineto{\pgfqpoint{2.665619in}{4.785531in}}%
\pgfpathlineto{\pgfqpoint{2.665619in}{5.042951in}}%
\pgfpathlineto{\pgfqpoint{2.681771in}{5.042951in}}%
\pgfpathlineto{\pgfqpoint{2.681771in}{4.480933in}}%
\pgfpathlineto{\pgfqpoint{2.697922in}{4.480933in}}%
\pgfpathlineto{\pgfqpoint{2.697922in}{4.925046in}}%
\pgfpathlineto{\pgfqpoint{2.714074in}{4.925046in}}%
\pgfpathlineto{\pgfqpoint{2.714074in}{4.678017in}}%
\pgfpathlineto{\pgfqpoint{2.730225in}{4.678017in}}%
\pgfpathlineto{\pgfqpoint{2.730225in}{4.587603in}}%
\pgfpathlineto{\pgfqpoint{2.746377in}{4.587603in}}%
\pgfpathlineto{\pgfqpoint{2.746377in}{5.073178in}}%
\pgfpathlineto{\pgfqpoint{2.762529in}{5.073178in}}%
\pgfpathlineto{\pgfqpoint{2.762529in}{4.643869in}}%
\pgfpathlineto{\pgfqpoint{2.778680in}{4.643869in}}%
\pgfpathlineto{\pgfqpoint{2.778680in}{4.697250in}}%
\pgfpathlineto{\pgfqpoint{2.794832in}{4.697250in}}%
\pgfpathlineto{\pgfqpoint{2.794832in}{4.739333in}}%
\pgfpathlineto{\pgfqpoint{2.810984in}{4.739333in}}%
\pgfpathlineto{\pgfqpoint{2.810984in}{4.958257in}}%
\pgfpathlineto{\pgfqpoint{2.827135in}{4.958257in}}%
\pgfpathlineto{\pgfqpoint{2.827135in}{4.628398in}}%
\pgfpathlineto{\pgfqpoint{2.843287in}{4.628398in}}%
\pgfpathlineto{\pgfqpoint{2.843287in}{4.907663in}}%
\pgfpathlineto{\pgfqpoint{2.859439in}{4.907663in}}%
\pgfpathlineto{\pgfqpoint{2.859439in}{4.661288in}}%
\pgfpathlineto{\pgfqpoint{2.875590in}{4.661288in}}%
\pgfpathlineto{\pgfqpoint{2.875590in}{4.707608in}}%
\pgfpathlineto{\pgfqpoint{2.891742in}{4.707608in}}%
\pgfpathlineto{\pgfqpoint{2.891742in}{4.920006in}}%
\pgfpathlineto{\pgfqpoint{2.907893in}{4.920006in}}%
\pgfpathlineto{\pgfqpoint{2.907893in}{4.835260in}}%
\pgfpathlineto{\pgfqpoint{2.924045in}{4.835260in}}%
\pgfpathlineto{\pgfqpoint{2.924045in}{4.562806in}}%
\pgfpathlineto{\pgfqpoint{2.940197in}{4.562806in}}%
\pgfpathlineto{\pgfqpoint{2.940197in}{4.871484in}}%
\pgfpathlineto{\pgfqpoint{2.956348in}{4.871484in}}%
\pgfpathlineto{\pgfqpoint{2.956348in}{4.639780in}}%
\pgfpathlineto{\pgfqpoint{2.972500in}{4.639780in}}%
\pgfpathlineto{\pgfqpoint{2.972500in}{4.989389in}}%
\pgfpathlineto{\pgfqpoint{2.988652in}{4.989389in}}%
\pgfpathlineto{\pgfqpoint{2.988652in}{4.486479in}}%
\pgfpathlineto{\pgfqpoint{3.004803in}{4.486479in}}%
\pgfpathlineto{\pgfqpoint{3.004803in}{5.013837in}}%
\pgfpathlineto{\pgfqpoint{3.020955in}{5.013837in}}%
\pgfpathlineto{\pgfqpoint{3.020955in}{4.546170in}}%
\pgfpathlineto{\pgfqpoint{3.037107in}{4.546170in}}%
\pgfpathlineto{\pgfqpoint{3.037107in}{4.868452in}}%
\pgfpathlineto{\pgfqpoint{3.053258in}{4.868452in}}%
\pgfpathlineto{\pgfqpoint{3.053258in}{4.967636in}}%
\pgfpathlineto{\pgfqpoint{3.069410in}{4.967636in}}%
\pgfpathlineto{\pgfqpoint{3.069410in}{4.446829in}}%
\pgfpathlineto{\pgfqpoint{3.085561in}{4.446829in}}%
\pgfpathlineto{\pgfqpoint{3.085561in}{4.954467in}}%
\pgfpathlineto{\pgfqpoint{3.101713in}{4.954467in}}%
\pgfpathlineto{\pgfqpoint{3.101713in}{4.785420in}}%
\pgfpathlineto{\pgfqpoint{3.117865in}{4.785420in}}%
\pgfpathlineto{\pgfqpoint{3.117865in}{4.630408in}}%
\pgfpathlineto{\pgfqpoint{3.134016in}{4.630408in}}%
\pgfpathlineto{\pgfqpoint{3.134016in}{5.007187in}}%
\pgfpathlineto{\pgfqpoint{3.150168in}{5.007187in}}%
\pgfpathlineto{\pgfqpoint{3.150168in}{4.530980in}}%
\pgfpathlineto{\pgfqpoint{3.166320in}{4.530980in}}%
\pgfpathlineto{\pgfqpoint{3.166320in}{4.918615in}}%
\pgfpathlineto{\pgfqpoint{3.182471in}{4.918615in}}%
\pgfpathlineto{\pgfqpoint{3.182471in}{4.547956in}}%
\pgfpathlineto{\pgfqpoint{3.198623in}{4.547956in}}%
\pgfpathlineto{\pgfqpoint{3.198623in}{4.955692in}}%
\pgfpathlineto{\pgfqpoint{3.214775in}{4.955692in}}%
\pgfpathlineto{\pgfqpoint{3.214775in}{4.785333in}}%
\pgfpathlineto{\pgfqpoint{3.230926in}{4.785333in}}%
\pgfpathlineto{\pgfqpoint{3.230926in}{4.625697in}}%
\pgfpathlineto{\pgfqpoint{3.247078in}{4.625697in}}%
\pgfpathlineto{\pgfqpoint{3.247078in}{4.873146in}}%
\pgfpathlineto{\pgfqpoint{3.263229in}{4.873146in}}%
\pgfpathlineto{\pgfqpoint{3.263229in}{4.750410in}}%
\pgfpathlineto{\pgfqpoint{3.295533in}{4.749547in}}%
\pgfpathlineto{\pgfqpoint{3.295533in}{5.000199in}}%
\pgfpathlineto{\pgfqpoint{3.311684in}{5.000199in}}%
\pgfpathlineto{\pgfqpoint{3.311684in}{4.517283in}}%
\pgfpathlineto{\pgfqpoint{3.327836in}{4.517283in}}%
\pgfpathlineto{\pgfqpoint{3.327836in}{4.783274in}}%
\pgfpathlineto{\pgfqpoint{3.343988in}{4.783274in}}%
\pgfpathlineto{\pgfqpoint{3.343988in}{4.762807in}}%
\pgfpathlineto{\pgfqpoint{3.360139in}{4.762807in}}%
\pgfpathlineto{\pgfqpoint{3.360139in}{4.856605in}}%
\pgfpathlineto{\pgfqpoint{3.376291in}{4.856605in}}%
\pgfpathlineto{\pgfqpoint{3.376291in}{4.750518in}}%
\pgfpathlineto{\pgfqpoint{3.392443in}{4.750518in}}%
\pgfpathlineto{\pgfqpoint{3.392443in}{4.705817in}}%
\pgfpathlineto{\pgfqpoint{3.424746in}{4.704323in}}%
\pgfpathlineto{\pgfqpoint{3.424746in}{4.835067in}}%
\pgfpathlineto{\pgfqpoint{3.440897in}{4.835067in}}%
\pgfpathlineto{\pgfqpoint{3.440897in}{4.939429in}}%
\pgfpathlineto{\pgfqpoint{3.457049in}{4.939429in}}%
\pgfpathlineto{\pgfqpoint{3.457049in}{4.478195in}}%
\pgfpathlineto{\pgfqpoint{3.473201in}{4.478195in}}%
\pgfpathlineto{\pgfqpoint{3.473201in}{5.036562in}}%
\pgfpathlineto{\pgfqpoint{3.489352in}{5.036562in}}%
\pgfpathlineto{\pgfqpoint{3.489352in}{4.716775in}}%
\pgfpathlineto{\pgfqpoint{3.505504in}{4.716775in}}%
\pgfpathlineto{\pgfqpoint{3.505504in}{4.534631in}}%
\pgfpathlineto{\pgfqpoint{3.521656in}{4.534631in}}%
\pgfpathlineto{\pgfqpoint{3.521656in}{5.060828in}}%
\pgfpathlineto{\pgfqpoint{3.537807in}{5.060828in}}%
\pgfpathlineto{\pgfqpoint{3.537807in}{4.707102in}}%
\pgfpathlineto{\pgfqpoint{3.553959in}{4.707102in}}%
\pgfpathlineto{\pgfqpoint{3.553959in}{4.597085in}}%
\pgfpathlineto{\pgfqpoint{3.570110in}{4.597085in}}%
\pgfpathlineto{\pgfqpoint{3.570110in}{4.819047in}}%
\pgfpathlineto{\pgfqpoint{3.586262in}{4.819047in}}%
\pgfpathlineto{\pgfqpoint{3.586262in}{4.775393in}}%
\pgfpathlineto{\pgfqpoint{3.602414in}{4.775393in}}%
\pgfpathlineto{\pgfqpoint{3.602414in}{4.925237in}}%
\pgfpathlineto{\pgfqpoint{3.618565in}{4.925237in}}%
\pgfpathlineto{\pgfqpoint{3.618565in}{4.599243in}}%
\pgfpathlineto{\pgfqpoint{3.634717in}{4.599243in}}%
\pgfpathlineto{\pgfqpoint{3.634717in}{4.956154in}}%
\pgfpathlineto{\pgfqpoint{3.650869in}{4.956154in}}%
\pgfpathlineto{\pgfqpoint{3.650869in}{4.467672in}}%
\pgfpathlineto{\pgfqpoint{3.667020in}{4.467672in}}%
\pgfpathlineto{\pgfqpoint{3.667020in}{5.005822in}}%
\pgfpathlineto{\pgfqpoint{3.683172in}{5.005822in}}%
\pgfpathlineto{\pgfqpoint{3.683172in}{4.791707in}}%
\pgfpathlineto{\pgfqpoint{3.699324in}{4.791707in}}%
\pgfpathlineto{\pgfqpoint{3.699324in}{4.555312in}}%
\pgfpathlineto{\pgfqpoint{3.715475in}{4.555312in}}%
\pgfpathlineto{\pgfqpoint{3.715475in}{5.084255in}}%
\pgfpathlineto{\pgfqpoint{3.731627in}{5.084255in}}%
\pgfpathlineto{\pgfqpoint{3.731627in}{4.347683in}}%
\pgfpathlineto{\pgfqpoint{3.747778in}{4.347683in}}%
\pgfpathlineto{\pgfqpoint{3.747778in}{5.137815in}}%
\pgfpathlineto{\pgfqpoint{3.763930in}{5.137815in}}%
\pgfpathlineto{\pgfqpoint{3.763930in}{4.596049in}}%
\pgfpathlineto{\pgfqpoint{3.780082in}{4.596049in}}%
\pgfpathlineto{\pgfqpoint{3.780082in}{4.910224in}}%
\pgfpathlineto{\pgfqpoint{3.796233in}{4.910224in}}%
\pgfpathlineto{\pgfqpoint{3.796233in}{4.663873in}}%
\pgfpathlineto{\pgfqpoint{3.812385in}{4.663873in}}%
\pgfpathlineto{\pgfqpoint{3.812385in}{4.643213in}}%
\pgfpathlineto{\pgfqpoint{3.828537in}{4.643213in}}%
\pgfpathlineto{\pgfqpoint{3.828537in}{5.038795in}}%
\pgfpathlineto{\pgfqpoint{3.844688in}{5.038795in}}%
\pgfpathlineto{\pgfqpoint{3.844688in}{4.436734in}}%
\pgfpathlineto{\pgfqpoint{3.860840in}{4.436734in}}%
\pgfpathlineto{\pgfqpoint{3.860840in}{5.113821in}}%
\pgfpathlineto{\pgfqpoint{3.876992in}{5.113821in}}%
\pgfpathlineto{\pgfqpoint{3.876992in}{4.671103in}}%
\pgfpathlineto{\pgfqpoint{3.893143in}{4.671103in}}%
\pgfpathlineto{\pgfqpoint{3.893143in}{4.653349in}}%
\pgfpathlineto{\pgfqpoint{3.909295in}{4.653349in}}%
\pgfpathlineto{\pgfqpoint{3.909295in}{4.781258in}}%
\pgfpathlineto{\pgfqpoint{3.925446in}{4.781258in}}%
\pgfpathlineto{\pgfqpoint{3.925446in}{4.817554in}}%
\pgfpathlineto{\pgfqpoint{3.941598in}{4.817554in}}%
\pgfpathlineto{\pgfqpoint{3.941598in}{4.675278in}}%
\pgfpathlineto{\pgfqpoint{3.957750in}{4.675278in}}%
\pgfpathlineto{\pgfqpoint{3.957750in}{4.919297in}}%
\pgfpathlineto{\pgfqpoint{3.973901in}{4.919297in}}%
\pgfpathlineto{\pgfqpoint{3.973901in}{4.683941in}}%
\pgfpathlineto{\pgfqpoint{3.990053in}{4.683941in}}%
\pgfpathlineto{\pgfqpoint{3.990053in}{4.963080in}}%
\pgfpathlineto{\pgfqpoint{4.006205in}{4.963080in}}%
\pgfpathlineto{\pgfqpoint{4.006205in}{4.407157in}}%
\pgfpathlineto{\pgfqpoint{4.022356in}{4.407157in}}%
\pgfpathlineto{\pgfqpoint{4.022356in}{5.038086in}}%
\pgfpathlineto{\pgfqpoint{4.038508in}{5.038086in}}%
\pgfpathlineto{\pgfqpoint{4.038508in}{4.691393in}}%
\pgfpathlineto{\pgfqpoint{4.054660in}{4.691393in}}%
\pgfpathlineto{\pgfqpoint{4.054660in}{4.843335in}}%
\pgfpathlineto{\pgfqpoint{4.070811in}{4.843335in}}%
\pgfpathlineto{\pgfqpoint{4.070811in}{4.681760in}}%
\pgfpathlineto{\pgfqpoint{4.086963in}{4.681760in}}%
\pgfpathlineto{\pgfqpoint{4.086963in}{4.827834in}}%
\pgfpathlineto{\pgfqpoint{4.103114in}{4.827834in}}%
\pgfpathlineto{\pgfqpoint{4.103114in}{4.615817in}}%
\pgfpathlineto{\pgfqpoint{4.119266in}{4.615817in}}%
\pgfpathlineto{\pgfqpoint{4.119266in}{5.030287in}}%
\pgfpathlineto{\pgfqpoint{4.135418in}{5.030287in}}%
\pgfpathlineto{\pgfqpoint{4.135418in}{4.491629in}}%
\pgfpathlineto{\pgfqpoint{4.151569in}{4.491629in}}%
\pgfpathlineto{\pgfqpoint{4.151569in}{5.027327in}}%
\pgfpathlineto{\pgfqpoint{4.167721in}{5.027327in}}%
\pgfpathlineto{\pgfqpoint{4.167721in}{4.407214in}}%
\pgfpathlineto{\pgfqpoint{4.183873in}{4.407214in}}%
\pgfpathlineto{\pgfqpoint{4.183873in}{5.235237in}}%
\pgfpathlineto{\pgfqpoint{4.200024in}{5.235237in}}%
\pgfpathlineto{\pgfqpoint{4.200024in}{4.451932in}}%
\pgfpathlineto{\pgfqpoint{4.216176in}{4.451932in}}%
\pgfpathlineto{\pgfqpoint{4.216176in}{4.812860in}}%
\pgfpathlineto{\pgfqpoint{4.232328in}{4.812860in}}%
\pgfpathlineto{\pgfqpoint{4.232328in}{4.712957in}}%
\pgfpathlineto{\pgfqpoint{4.248479in}{4.712957in}}%
\pgfpathlineto{\pgfqpoint{4.248479in}{4.876072in}}%
\pgfpathlineto{\pgfqpoint{4.264631in}{4.876072in}}%
\pgfpathlineto{\pgfqpoint{4.264631in}{4.839324in}}%
\pgfpathlineto{\pgfqpoint{4.280782in}{4.839324in}}%
\pgfpathlineto{\pgfqpoint{4.280782in}{4.671703in}}%
\pgfpathlineto{\pgfqpoint{4.296934in}{4.671703in}}%
\pgfpathlineto{\pgfqpoint{4.296934in}{4.912062in}}%
\pgfpathlineto{\pgfqpoint{4.313086in}{4.912062in}}%
\pgfpathlineto{\pgfqpoint{4.313086in}{4.549779in}}%
\pgfpathlineto{\pgfqpoint{4.329237in}{4.549779in}}%
\pgfpathlineto{\pgfqpoint{4.329237in}{4.778864in}}%
\pgfpathlineto{\pgfqpoint{4.345389in}{4.778864in}}%
\pgfpathlineto{\pgfqpoint{4.345389in}{4.919388in}}%
\pgfpathlineto{\pgfqpoint{4.361541in}{4.919388in}}%
\pgfpathlineto{\pgfqpoint{4.361541in}{4.830784in}}%
\pgfpathlineto{\pgfqpoint{4.377692in}{4.830784in}}%
\pgfpathlineto{\pgfqpoint{4.377692in}{4.675328in}}%
\pgfpathlineto{\pgfqpoint{4.393844in}{4.675328in}}%
\pgfpathlineto{\pgfqpoint{4.393844in}{4.566958in}}%
\pgfpathlineto{\pgfqpoint{4.409996in}{4.566958in}}%
\pgfpathlineto{\pgfqpoint{4.409996in}{4.988595in}}%
\pgfpathlineto{\pgfqpoint{4.426147in}{4.988595in}}%
\pgfpathlineto{\pgfqpoint{4.426147in}{4.673878in}}%
\pgfpathlineto{\pgfqpoint{4.442299in}{4.673878in}}%
\pgfpathlineto{\pgfqpoint{4.442299in}{4.773389in}}%
\pgfpathlineto{\pgfqpoint{4.458450in}{4.773389in}}%
\pgfpathlineto{\pgfqpoint{4.458450in}{4.987184in}}%
\pgfpathlineto{\pgfqpoint{4.474602in}{4.987184in}}%
\pgfpathlineto{\pgfqpoint{4.474602in}{4.552371in}}%
\pgfpathlineto{\pgfqpoint{4.490754in}{4.552371in}}%
\pgfpathlineto{\pgfqpoint{4.490754in}{4.742467in}}%
\pgfpathlineto{\pgfqpoint{4.506905in}{4.742467in}}%
\pgfpathlineto{\pgfqpoint{4.506905in}{4.862620in}}%
\pgfpathlineto{\pgfqpoint{4.523057in}{4.862620in}}%
\pgfpathlineto{\pgfqpoint{4.523057in}{4.742455in}}%
\pgfpathlineto{\pgfqpoint{4.539209in}{4.742455in}}%
\pgfpathlineto{\pgfqpoint{4.539209in}{4.730169in}}%
\pgfpathlineto{\pgfqpoint{4.555360in}{4.730169in}}%
\pgfpathlineto{\pgfqpoint{4.555360in}{4.824280in}}%
\pgfpathlineto{\pgfqpoint{4.571512in}{4.824280in}}%
\pgfpathlineto{\pgfqpoint{4.571512in}{4.968275in}}%
\pgfpathlineto{\pgfqpoint{4.587663in}{4.968275in}}%
\pgfpathlineto{\pgfqpoint{4.587663in}{4.356604in}}%
\pgfpathlineto{\pgfqpoint{4.603815in}{4.356604in}}%
\pgfpathlineto{\pgfqpoint{4.603815in}{4.932747in}}%
\pgfpathlineto{\pgfqpoint{4.619967in}{4.932747in}}%
\pgfpathlineto{\pgfqpoint{4.619967in}{5.003390in}}%
\pgfpathlineto{\pgfqpoint{4.636118in}{5.003390in}}%
\pgfpathlineto{\pgfqpoint{4.636118in}{4.391481in}}%
\pgfpathlineto{\pgfqpoint{4.652270in}{4.391481in}}%
\pgfpathlineto{\pgfqpoint{4.652270in}{5.155251in}}%
\pgfpathlineto{\pgfqpoint{4.668422in}{5.155251in}}%
\pgfpathlineto{\pgfqpoint{4.668422in}{4.461846in}}%
\pgfpathlineto{\pgfqpoint{4.684573in}{4.461846in}}%
\pgfpathlineto{\pgfqpoint{4.684573in}{5.015798in}}%
\pgfpathlineto{\pgfqpoint{4.700725in}{5.015798in}}%
\pgfpathlineto{\pgfqpoint{4.700725in}{4.448993in}}%
\pgfpathlineto{\pgfqpoint{4.716877in}{4.448993in}}%
\pgfpathlineto{\pgfqpoint{4.716877in}{5.109148in}}%
\pgfpathlineto{\pgfqpoint{4.733028in}{5.109148in}}%
\pgfpathlineto{\pgfqpoint{4.733028in}{4.584136in}}%
\pgfpathlineto{\pgfqpoint{4.749180in}{4.584136in}}%
\pgfpathlineto{\pgfqpoint{4.749180in}{4.871327in}}%
\pgfpathlineto{\pgfqpoint{4.765331in}{4.871327in}}%
\pgfpathlineto{\pgfqpoint{4.765331in}{4.684410in}}%
\pgfpathlineto{\pgfqpoint{4.781483in}{4.684410in}}%
\pgfpathlineto{\pgfqpoint{4.781483in}{4.652067in}}%
\pgfpathlineto{\pgfqpoint{4.797635in}{4.652067in}}%
\pgfpathlineto{\pgfqpoint{4.797635in}{4.808726in}}%
\pgfpathlineto{\pgfqpoint{4.813786in}{4.808726in}}%
\pgfpathlineto{\pgfqpoint{4.813786in}{5.092235in}}%
\pgfpathlineto{\pgfqpoint{4.829938in}{5.092235in}}%
\pgfpathlineto{\pgfqpoint{4.829938in}{4.339980in}}%
\pgfpathlineto{\pgfqpoint{4.846090in}{4.339980in}}%
\pgfpathlineto{\pgfqpoint{4.846090in}{5.008507in}}%
\pgfpathlineto{\pgfqpoint{4.862241in}{5.008507in}}%
\pgfpathlineto{\pgfqpoint{4.862241in}{4.742558in}}%
\pgfpathlineto{\pgfqpoint{4.878393in}{4.742558in}}%
\pgfpathlineto{\pgfqpoint{4.878393in}{4.925279in}}%
\pgfpathlineto{\pgfqpoint{4.894545in}{4.925279in}}%
\pgfpathlineto{\pgfqpoint{4.894545in}{4.497683in}}%
\pgfpathlineto{\pgfqpoint{4.910696in}{4.497683in}}%
\pgfpathlineto{\pgfqpoint{4.910696in}{4.751631in}}%
\pgfpathlineto{\pgfqpoint{4.926848in}{4.751631in}}%
\pgfpathlineto{\pgfqpoint{4.926848in}{4.979393in}}%
\pgfpathlineto{\pgfqpoint{4.942999in}{4.979393in}}%
\pgfpathlineto{\pgfqpoint{4.942999in}{4.654608in}}%
\pgfpathlineto{\pgfqpoint{4.959151in}{4.654608in}}%
\pgfpathlineto{\pgfqpoint{4.959151in}{4.952973in}}%
\pgfpathlineto{\pgfqpoint{4.975303in}{4.952973in}}%
\pgfpathlineto{\pgfqpoint{4.975303in}{4.636374in}}%
\pgfpathlineto{\pgfqpoint{4.991454in}{4.636374in}}%
\pgfpathlineto{\pgfqpoint{4.991454in}{4.614686in}}%
\pgfpathlineto{\pgfqpoint{5.007606in}{4.614686in}}%
\pgfpathlineto{\pgfqpoint{5.007606in}{4.822922in}}%
\pgfpathlineto{\pgfqpoint{5.015682in}{4.822922in}}%
\pgfpathlineto{\pgfqpoint{5.015682in}{4.822922in}}%
\pgfusepath{stroke}%
\end{pgfscope}%
\begin{pgfscope}%
\pgfpathrectangle{\pgfqpoint{0.725000in}{4.295217in}}{\pgfqpoint{4.495000in}{0.984783in}}%
\pgfusepath{clip}%
\pgfsetbuttcap%
\pgfsetroundjoin%
\definecolor{currentfill}{rgb}{1.000000,0.498039,0.054902}%
\pgfsetfillcolor{currentfill}%
\pgfsetlinewidth{1.003750pt}%
\definecolor{currentstroke}{rgb}{1.000000,0.498039,0.054902}%
\pgfsetstrokecolor{currentstroke}%
\pgfsetdash{}{0pt}%
\pgfsys@defobject{currentmarker}{\pgfqpoint{-0.041667in}{-0.041667in}}{\pgfqpoint{0.041667in}{0.041667in}}{%
\pgfpathmoveto{\pgfqpoint{0.000000in}{-0.041667in}}%
\pgfpathcurveto{\pgfqpoint{0.011050in}{-0.041667in}}{\pgfqpoint{0.021649in}{-0.037276in}}{\pgfqpoint{0.029463in}{-0.029463in}}%
\pgfpathcurveto{\pgfqpoint{0.037276in}{-0.021649in}}{\pgfqpoint{0.041667in}{-0.011050in}}{\pgfqpoint{0.041667in}{0.000000in}}%
\pgfpathcurveto{\pgfqpoint{0.041667in}{0.011050in}}{\pgfqpoint{0.037276in}{0.021649in}}{\pgfqpoint{0.029463in}{0.029463in}}%
\pgfpathcurveto{\pgfqpoint{0.021649in}{0.037276in}}{\pgfqpoint{0.011050in}{0.041667in}}{\pgfqpoint{0.000000in}{0.041667in}}%
\pgfpathcurveto{\pgfqpoint{-0.011050in}{0.041667in}}{\pgfqpoint{-0.021649in}{0.037276in}}{\pgfqpoint{-0.029463in}{0.029463in}}%
\pgfpathcurveto{\pgfqpoint{-0.037276in}{0.021649in}}{\pgfqpoint{-0.041667in}{0.011050in}}{\pgfqpoint{-0.041667in}{0.000000in}}%
\pgfpathcurveto{\pgfqpoint{-0.041667in}{-0.011050in}}{\pgfqpoint{-0.037276in}{-0.021649in}}{\pgfqpoint{-0.029463in}{-0.029463in}}%
\pgfpathcurveto{\pgfqpoint{-0.021649in}{-0.037276in}}{\pgfqpoint{-0.011050in}{-0.041667in}}{\pgfqpoint{0.000000in}{-0.041667in}}%
\pgfpathclose%
\pgfusepath{stroke,fill}%
}%
\begin{pgfscope}%
\pgfsys@transformshift{0.929318in}{4.926811in}%
\pgfsys@useobject{currentmarker}{}%
\end{pgfscope}%
\begin{pgfscope}%
\pgfsys@transformshift{0.945470in}{4.755188in}%
\pgfsys@useobject{currentmarker}{}%
\end{pgfscope}%
\begin{pgfscope}%
\pgfsys@transformshift{0.961621in}{4.684607in}%
\pgfsys@useobject{currentmarker}{}%
\end{pgfscope}%
\begin{pgfscope}%
\pgfsys@transformshift{0.977773in}{4.709591in}%
\pgfsys@useobject{currentmarker}{}%
\end{pgfscope}%
\begin{pgfscope}%
\pgfsys@transformshift{0.993925in}{4.903863in}%
\pgfsys@useobject{currentmarker}{}%
\end{pgfscope}%
\begin{pgfscope}%
\pgfsys@transformshift{1.010076in}{4.838744in}%
\pgfsys@useobject{currentmarker}{}%
\end{pgfscope}%
\begin{pgfscope}%
\pgfsys@transformshift{1.026228in}{4.724658in}%
\pgfsys@useobject{currentmarker}{}%
\end{pgfscope}%
\begin{pgfscope}%
\pgfsys@transformshift{1.042380in}{4.649840in}%
\pgfsys@useobject{currentmarker}{}%
\end{pgfscope}%
\begin{pgfscope}%
\pgfsys@transformshift{1.058531in}{4.799650in}%
\pgfsys@useobject{currentmarker}{}%
\end{pgfscope}%
\begin{pgfscope}%
\pgfsys@transformshift{1.074683in}{4.776849in}%
\pgfsys@useobject{currentmarker}{}%
\end{pgfscope}%
\begin{pgfscope}%
\pgfsys@transformshift{1.090835in}{4.754318in}%
\pgfsys@useobject{currentmarker}{}%
\end{pgfscope}%
\begin{pgfscope}%
\pgfsys@transformshift{1.106986in}{4.671229in}%
\pgfsys@useobject{currentmarker}{}%
\end{pgfscope}%
\begin{pgfscope}%
\pgfsys@transformshift{1.123138in}{5.050143in}%
\pgfsys@useobject{currentmarker}{}%
\end{pgfscope}%
\begin{pgfscope}%
\pgfsys@transformshift{1.139289in}{4.526353in}%
\pgfsys@useobject{currentmarker}{}%
\end{pgfscope}%
\begin{pgfscope}%
\pgfsys@transformshift{1.155441in}{4.803608in}%
\pgfsys@useobject{currentmarker}{}%
\end{pgfscope}%
\begin{pgfscope}%
\pgfsys@transformshift{1.171593in}{4.915372in}%
\pgfsys@useobject{currentmarker}{}%
\end{pgfscope}%
\begin{pgfscope}%
\pgfsys@transformshift{1.187744in}{4.559540in}%
\pgfsys@useobject{currentmarker}{}%
\end{pgfscope}%
\begin{pgfscope}%
\pgfsys@transformshift{1.203896in}{4.903793in}%
\pgfsys@useobject{currentmarker}{}%
\end{pgfscope}%
\begin{pgfscope}%
\pgfsys@transformshift{1.220048in}{4.799434in}%
\pgfsys@useobject{currentmarker}{}%
\end{pgfscope}%
\begin{pgfscope}%
\pgfsys@transformshift{1.236199in}{4.570440in}%
\pgfsys@useobject{currentmarker}{}%
\end{pgfscope}%
\begin{pgfscope}%
\pgfsys@transformshift{1.252351in}{4.815336in}%
\pgfsys@useobject{currentmarker}{}%
\end{pgfscope}%
\begin{pgfscope}%
\pgfsys@transformshift{1.268503in}{4.858981in}%
\pgfsys@useobject{currentmarker}{}%
\end{pgfscope}%
\begin{pgfscope}%
\pgfsys@transformshift{1.284654in}{4.800513in}%
\pgfsys@useobject{currentmarker}{}%
\end{pgfscope}%
\begin{pgfscope}%
\pgfsys@transformshift{1.300806in}{4.820586in}%
\pgfsys@useobject{currentmarker}{}%
\end{pgfscope}%
\begin{pgfscope}%
\pgfsys@transformshift{1.316957in}{4.434374in}%
\pgfsys@useobject{currentmarker}{}%
\end{pgfscope}%
\begin{pgfscope}%
\pgfsys@transformshift{1.333109in}{5.163469in}%
\pgfsys@useobject{currentmarker}{}%
\end{pgfscope}%
\begin{pgfscope}%
\pgfsys@transformshift{1.349261in}{4.469893in}%
\pgfsys@useobject{currentmarker}{}%
\end{pgfscope}%
\begin{pgfscope}%
\pgfsys@transformshift{1.365412in}{4.963435in}%
\pgfsys@useobject{currentmarker}{}%
\end{pgfscope}%
\begin{pgfscope}%
\pgfsys@transformshift{1.381564in}{4.734621in}%
\pgfsys@useobject{currentmarker}{}%
\end{pgfscope}%
\begin{pgfscope}%
\pgfsys@transformshift{1.397716in}{4.696828in}%
\pgfsys@useobject{currentmarker}{}%
\end{pgfscope}%
\begin{pgfscope}%
\pgfsys@transformshift{1.413867in}{4.733060in}%
\pgfsys@useobject{currentmarker}{}%
\end{pgfscope}%
\begin{pgfscope}%
\pgfsys@transformshift{1.430019in}{4.872482in}%
\pgfsys@useobject{currentmarker}{}%
\end{pgfscope}%
\begin{pgfscope}%
\pgfsys@transformshift{1.446170in}{4.670400in}%
\pgfsys@useobject{currentmarker}{}%
\end{pgfscope}%
\begin{pgfscope}%
\pgfsys@transformshift{1.462322in}{4.732843in}%
\pgfsys@useobject{currentmarker}{}%
\end{pgfscope}%
\begin{pgfscope}%
\pgfsys@transformshift{1.478474in}{4.925430in}%
\pgfsys@useobject{currentmarker}{}%
\end{pgfscope}%
\begin{pgfscope}%
\pgfsys@transformshift{1.494625in}{4.641453in}%
\pgfsys@useobject{currentmarker}{}%
\end{pgfscope}%
\begin{pgfscope}%
\pgfsys@transformshift{1.510777in}{4.998542in}%
\pgfsys@useobject{currentmarker}{}%
\end{pgfscope}%
\begin{pgfscope}%
\pgfsys@transformshift{1.526929in}{4.532158in}%
\pgfsys@useobject{currentmarker}{}%
\end{pgfscope}%
\begin{pgfscope}%
\pgfsys@transformshift{1.543080in}{4.644918in}%
\pgfsys@useobject{currentmarker}{}%
\end{pgfscope}%
\begin{pgfscope}%
\pgfsys@transformshift{1.559232in}{5.008158in}%
\pgfsys@useobject{currentmarker}{}%
\end{pgfscope}%
\begin{pgfscope}%
\pgfsys@transformshift{1.575384in}{4.660421in}%
\pgfsys@useobject{currentmarker}{}%
\end{pgfscope}%
\begin{pgfscope}%
\pgfsys@transformshift{1.591535in}{5.004616in}%
\pgfsys@useobject{currentmarker}{}%
\end{pgfscope}%
\begin{pgfscope}%
\pgfsys@transformshift{1.607687in}{4.481166in}%
\pgfsys@useobject{currentmarker}{}%
\end{pgfscope}%
\begin{pgfscope}%
\pgfsys@transformshift{1.623838in}{4.916304in}%
\pgfsys@useobject{currentmarker}{}%
\end{pgfscope}%
\begin{pgfscope}%
\pgfsys@transformshift{1.639990in}{4.605169in}%
\pgfsys@useobject{currentmarker}{}%
\end{pgfscope}%
\begin{pgfscope}%
\pgfsys@transformshift{1.656142in}{4.924009in}%
\pgfsys@useobject{currentmarker}{}%
\end{pgfscope}%
\begin{pgfscope}%
\pgfsys@transformshift{1.672293in}{4.732442in}%
\pgfsys@useobject{currentmarker}{}%
\end{pgfscope}%
\begin{pgfscope}%
\pgfsys@transformshift{1.688445in}{4.614295in}%
\pgfsys@useobject{currentmarker}{}%
\end{pgfscope}%
\begin{pgfscope}%
\pgfsys@transformshift{1.704597in}{5.105864in}%
\pgfsys@useobject{currentmarker}{}%
\end{pgfscope}%
\begin{pgfscope}%
\pgfsys@transformshift{1.720748in}{4.358032in}%
\pgfsys@useobject{currentmarker}{}%
\end{pgfscope}%
\begin{pgfscope}%
\pgfsys@transformshift{1.736900in}{5.132159in}%
\pgfsys@useobject{currentmarker}{}%
\end{pgfscope}%
\begin{pgfscope}%
\pgfsys@transformshift{1.753052in}{4.395938in}%
\pgfsys@useobject{currentmarker}{}%
\end{pgfscope}%
\begin{pgfscope}%
\pgfsys@transformshift{1.769203in}{5.217372in}%
\pgfsys@useobject{currentmarker}{}%
\end{pgfscope}%
\begin{pgfscope}%
\pgfsys@transformshift{1.785355in}{4.517591in}%
\pgfsys@useobject{currentmarker}{}%
\end{pgfscope}%
\begin{pgfscope}%
\pgfsys@transformshift{1.801506in}{4.690939in}%
\pgfsys@useobject{currentmarker}{}%
\end{pgfscope}%
\begin{pgfscope}%
\pgfsys@transformshift{1.817658in}{4.877002in}%
\pgfsys@useobject{currentmarker}{}%
\end{pgfscope}%
\begin{pgfscope}%
\pgfsys@transformshift{1.833810in}{4.559649in}%
\pgfsys@useobject{currentmarker}{}%
\end{pgfscope}%
\begin{pgfscope}%
\pgfsys@transformshift{1.849961in}{5.027588in}%
\pgfsys@useobject{currentmarker}{}%
\end{pgfscope}%
\begin{pgfscope}%
\pgfsys@transformshift{1.866113in}{4.703903in}%
\pgfsys@useobject{currentmarker}{}%
\end{pgfscope}%
\begin{pgfscope}%
\pgfsys@transformshift{1.882265in}{4.811082in}%
\pgfsys@useobject{currentmarker}{}%
\end{pgfscope}%
\begin{pgfscope}%
\pgfsys@transformshift{1.898416in}{4.690188in}%
\pgfsys@useobject{currentmarker}{}%
\end{pgfscope}%
\begin{pgfscope}%
\pgfsys@transformshift{1.914568in}{4.790854in}%
\pgfsys@useobject{currentmarker}{}%
\end{pgfscope}%
\begin{pgfscope}%
\pgfsys@transformshift{1.930720in}{4.665397in}%
\pgfsys@useobject{currentmarker}{}%
\end{pgfscope}%
\begin{pgfscope}%
\pgfsys@transformshift{1.946871in}{5.018360in}%
\pgfsys@useobject{currentmarker}{}%
\end{pgfscope}%
\begin{pgfscope}%
\pgfsys@transformshift{1.963023in}{4.552063in}%
\pgfsys@useobject{currentmarker}{}%
\end{pgfscope}%
\begin{pgfscope}%
\pgfsys@transformshift{1.979174in}{4.900024in}%
\pgfsys@useobject{currentmarker}{}%
\end{pgfscope}%
\begin{pgfscope}%
\pgfsys@transformshift{1.995326in}{4.546795in}%
\pgfsys@useobject{currentmarker}{}%
\end{pgfscope}%
\begin{pgfscope}%
\pgfsys@transformshift{2.011478in}{5.071773in}%
\pgfsys@useobject{currentmarker}{}%
\end{pgfscope}%
\begin{pgfscope}%
\pgfsys@transformshift{2.027629in}{4.586538in}%
\pgfsys@useobject{currentmarker}{}%
\end{pgfscope}%
\begin{pgfscope}%
\pgfsys@transformshift{2.043781in}{4.812088in}%
\pgfsys@useobject{currentmarker}{}%
\end{pgfscope}%
\begin{pgfscope}%
\pgfsys@transformshift{2.059933in}{4.576904in}%
\pgfsys@useobject{currentmarker}{}%
\end{pgfscope}%
\begin{pgfscope}%
\pgfsys@transformshift{2.076084in}{4.853167in}%
\pgfsys@useobject{currentmarker}{}%
\end{pgfscope}%
\begin{pgfscope}%
\pgfsys@transformshift{2.092236in}{5.037601in}%
\pgfsys@useobject{currentmarker}{}%
\end{pgfscope}%
\begin{pgfscope}%
\pgfsys@transformshift{2.108388in}{4.390752in}%
\pgfsys@useobject{currentmarker}{}%
\end{pgfscope}%
\begin{pgfscope}%
\pgfsys@transformshift{2.124539in}{5.148946in}%
\pgfsys@useobject{currentmarker}{}%
\end{pgfscope}%
\begin{pgfscope}%
\pgfsys@transformshift{2.140691in}{4.430347in}%
\pgfsys@useobject{currentmarker}{}%
\end{pgfscope}%
\begin{pgfscope}%
\pgfsys@transformshift{2.156842in}{4.947456in}%
\pgfsys@useobject{currentmarker}{}%
\end{pgfscope}%
\begin{pgfscope}%
\pgfsys@transformshift{2.172994in}{4.718768in}%
\pgfsys@useobject{currentmarker}{}%
\end{pgfscope}%
\begin{pgfscope}%
\pgfsys@transformshift{2.189146in}{4.802216in}%
\pgfsys@useobject{currentmarker}{}%
\end{pgfscope}%
\begin{pgfscope}%
\pgfsys@transformshift{2.205297in}{4.770482in}%
\pgfsys@useobject{currentmarker}{}%
\end{pgfscope}%
\begin{pgfscope}%
\pgfsys@transformshift{2.221449in}{4.778862in}%
\pgfsys@useobject{currentmarker}{}%
\end{pgfscope}%
\begin{pgfscope}%
\pgfsys@transformshift{2.237601in}{4.812036in}%
\pgfsys@useobject{currentmarker}{}%
\end{pgfscope}%
\begin{pgfscope}%
\pgfsys@transformshift{2.253752in}{4.679516in}%
\pgfsys@useobject{currentmarker}{}%
\end{pgfscope}%
\begin{pgfscope}%
\pgfsys@transformshift{2.269904in}{4.607710in}%
\pgfsys@useobject{currentmarker}{}%
\end{pgfscope}%
\begin{pgfscope}%
\pgfsys@transformshift{2.286056in}{5.087264in}%
\pgfsys@useobject{currentmarker}{}%
\end{pgfscope}%
\begin{pgfscope}%
\pgfsys@transformshift{2.302207in}{4.450651in}%
\pgfsys@useobject{currentmarker}{}%
\end{pgfscope}%
\begin{pgfscope}%
\pgfsys@transformshift{2.318359in}{5.086096in}%
\pgfsys@useobject{currentmarker}{}%
\end{pgfscope}%
\begin{pgfscope}%
\pgfsys@transformshift{2.334510in}{4.667564in}%
\pgfsys@useobject{currentmarker}{}%
\end{pgfscope}%
\begin{pgfscope}%
\pgfsys@transformshift{2.350662in}{4.570652in}%
\pgfsys@useobject{currentmarker}{}%
\end{pgfscope}%
\begin{pgfscope}%
\pgfsys@transformshift{2.366814in}{4.803596in}%
\pgfsys@useobject{currentmarker}{}%
\end{pgfscope}%
\begin{pgfscope}%
\pgfsys@transformshift{2.382965in}{4.900655in}%
\pgfsys@useobject{currentmarker}{}%
\end{pgfscope}%
\begin{pgfscope}%
\pgfsys@transformshift{2.399117in}{4.772374in}%
\pgfsys@useobject{currentmarker}{}%
\end{pgfscope}%
\begin{pgfscope}%
\pgfsys@transformshift{2.415269in}{4.818936in}%
\pgfsys@useobject{currentmarker}{}%
\end{pgfscope}%
\begin{pgfscope}%
\pgfsys@transformshift{2.431420in}{4.741220in}%
\pgfsys@useobject{currentmarker}{}%
\end{pgfscope}%
\begin{pgfscope}%
\pgfsys@transformshift{2.447572in}{4.607764in}%
\pgfsys@useobject{currentmarker}{}%
\end{pgfscope}%
\begin{pgfscope}%
\pgfsys@transformshift{2.463723in}{5.005428in}%
\pgfsys@useobject{currentmarker}{}%
\end{pgfscope}%
\begin{pgfscope}%
\pgfsys@transformshift{2.479875in}{4.504479in}%
\pgfsys@useobject{currentmarker}{}%
\end{pgfscope}%
\begin{pgfscope}%
\pgfsys@transformshift{2.496027in}{4.955112in}%
\pgfsys@useobject{currentmarker}{}%
\end{pgfscope}%
\begin{pgfscope}%
\pgfsys@transformshift{2.512178in}{4.794541in}%
\pgfsys@useobject{currentmarker}{}%
\end{pgfscope}%
\begin{pgfscope}%
\pgfsys@transformshift{2.528330in}{4.687633in}%
\pgfsys@useobject{currentmarker}{}%
\end{pgfscope}%
\begin{pgfscope}%
\pgfsys@transformshift{2.544482in}{4.696047in}%
\pgfsys@useobject{currentmarker}{}%
\end{pgfscope}%
\begin{pgfscope}%
\pgfsys@transformshift{2.560633in}{4.750000in}%
\pgfsys@useobject{currentmarker}{}%
\end{pgfscope}%
\begin{pgfscope}%
\pgfsys@transformshift{2.576785in}{4.859783in}%
\pgfsys@useobject{currentmarker}{}%
\end{pgfscope}%
\begin{pgfscope}%
\pgfsys@transformshift{2.592937in}{4.899697in}%
\pgfsys@useobject{currentmarker}{}%
\end{pgfscope}%
\begin{pgfscope}%
\pgfsys@transformshift{2.609088in}{4.550127in}%
\pgfsys@useobject{currentmarker}{}%
\end{pgfscope}%
\begin{pgfscope}%
\pgfsys@transformshift{2.625240in}{4.849655in}%
\pgfsys@useobject{currentmarker}{}%
\end{pgfscope}%
\begin{pgfscope}%
\pgfsys@transformshift{2.641391in}{4.688078in}%
\pgfsys@useobject{currentmarker}{}%
\end{pgfscope}%
\begin{pgfscope}%
\pgfsys@transformshift{2.657543in}{4.785531in}%
\pgfsys@useobject{currentmarker}{}%
\end{pgfscope}%
\begin{pgfscope}%
\pgfsys@transformshift{2.673695in}{5.042951in}%
\pgfsys@useobject{currentmarker}{}%
\end{pgfscope}%
\begin{pgfscope}%
\pgfsys@transformshift{2.689846in}{4.480933in}%
\pgfsys@useobject{currentmarker}{}%
\end{pgfscope}%
\begin{pgfscope}%
\pgfsys@transformshift{2.705998in}{4.925046in}%
\pgfsys@useobject{currentmarker}{}%
\end{pgfscope}%
\begin{pgfscope}%
\pgfsys@transformshift{2.722150in}{4.678017in}%
\pgfsys@useobject{currentmarker}{}%
\end{pgfscope}%
\begin{pgfscope}%
\pgfsys@transformshift{2.738301in}{4.587603in}%
\pgfsys@useobject{currentmarker}{}%
\end{pgfscope}%
\begin{pgfscope}%
\pgfsys@transformshift{2.754453in}{5.073178in}%
\pgfsys@useobject{currentmarker}{}%
\end{pgfscope}%
\begin{pgfscope}%
\pgfsys@transformshift{2.770605in}{4.643869in}%
\pgfsys@useobject{currentmarker}{}%
\end{pgfscope}%
\begin{pgfscope}%
\pgfsys@transformshift{2.786756in}{4.697250in}%
\pgfsys@useobject{currentmarker}{}%
\end{pgfscope}%
\begin{pgfscope}%
\pgfsys@transformshift{2.802908in}{4.739333in}%
\pgfsys@useobject{currentmarker}{}%
\end{pgfscope}%
\begin{pgfscope}%
\pgfsys@transformshift{2.819059in}{4.958257in}%
\pgfsys@useobject{currentmarker}{}%
\end{pgfscope}%
\begin{pgfscope}%
\pgfsys@transformshift{2.835211in}{4.628398in}%
\pgfsys@useobject{currentmarker}{}%
\end{pgfscope}%
\begin{pgfscope}%
\pgfsys@transformshift{2.851363in}{4.907663in}%
\pgfsys@useobject{currentmarker}{}%
\end{pgfscope}%
\begin{pgfscope}%
\pgfsys@transformshift{2.867514in}{4.661288in}%
\pgfsys@useobject{currentmarker}{}%
\end{pgfscope}%
\begin{pgfscope}%
\pgfsys@transformshift{2.883666in}{4.707608in}%
\pgfsys@useobject{currentmarker}{}%
\end{pgfscope}%
\begin{pgfscope}%
\pgfsys@transformshift{2.899818in}{4.920006in}%
\pgfsys@useobject{currentmarker}{}%
\end{pgfscope}%
\begin{pgfscope}%
\pgfsys@transformshift{2.915969in}{4.835260in}%
\pgfsys@useobject{currentmarker}{}%
\end{pgfscope}%
\begin{pgfscope}%
\pgfsys@transformshift{2.932121in}{4.562806in}%
\pgfsys@useobject{currentmarker}{}%
\end{pgfscope}%
\begin{pgfscope}%
\pgfsys@transformshift{2.948273in}{4.871484in}%
\pgfsys@useobject{currentmarker}{}%
\end{pgfscope}%
\begin{pgfscope}%
\pgfsys@transformshift{2.964424in}{4.639780in}%
\pgfsys@useobject{currentmarker}{}%
\end{pgfscope}%
\begin{pgfscope}%
\pgfsys@transformshift{2.980576in}{4.989389in}%
\pgfsys@useobject{currentmarker}{}%
\end{pgfscope}%
\begin{pgfscope}%
\pgfsys@transformshift{2.996727in}{4.486479in}%
\pgfsys@useobject{currentmarker}{}%
\end{pgfscope}%
\begin{pgfscope}%
\pgfsys@transformshift{3.012879in}{5.013837in}%
\pgfsys@useobject{currentmarker}{}%
\end{pgfscope}%
\begin{pgfscope}%
\pgfsys@transformshift{3.029031in}{4.546170in}%
\pgfsys@useobject{currentmarker}{}%
\end{pgfscope}%
\begin{pgfscope}%
\pgfsys@transformshift{3.045182in}{4.868452in}%
\pgfsys@useobject{currentmarker}{}%
\end{pgfscope}%
\begin{pgfscope}%
\pgfsys@transformshift{3.061334in}{4.967636in}%
\pgfsys@useobject{currentmarker}{}%
\end{pgfscope}%
\begin{pgfscope}%
\pgfsys@transformshift{3.077486in}{4.446829in}%
\pgfsys@useobject{currentmarker}{}%
\end{pgfscope}%
\begin{pgfscope}%
\pgfsys@transformshift{3.093637in}{4.954467in}%
\pgfsys@useobject{currentmarker}{}%
\end{pgfscope}%
\begin{pgfscope}%
\pgfsys@transformshift{3.109789in}{4.785420in}%
\pgfsys@useobject{currentmarker}{}%
\end{pgfscope}%
\begin{pgfscope}%
\pgfsys@transformshift{3.125941in}{4.630408in}%
\pgfsys@useobject{currentmarker}{}%
\end{pgfscope}%
\begin{pgfscope}%
\pgfsys@transformshift{3.142092in}{5.007187in}%
\pgfsys@useobject{currentmarker}{}%
\end{pgfscope}%
\begin{pgfscope}%
\pgfsys@transformshift{3.158244in}{4.530980in}%
\pgfsys@useobject{currentmarker}{}%
\end{pgfscope}%
\begin{pgfscope}%
\pgfsys@transformshift{3.174395in}{4.918615in}%
\pgfsys@useobject{currentmarker}{}%
\end{pgfscope}%
\begin{pgfscope}%
\pgfsys@transformshift{3.190547in}{4.547956in}%
\pgfsys@useobject{currentmarker}{}%
\end{pgfscope}%
\begin{pgfscope}%
\pgfsys@transformshift{3.206699in}{4.955692in}%
\pgfsys@useobject{currentmarker}{}%
\end{pgfscope}%
\begin{pgfscope}%
\pgfsys@transformshift{3.222850in}{4.785333in}%
\pgfsys@useobject{currentmarker}{}%
\end{pgfscope}%
\begin{pgfscope}%
\pgfsys@transformshift{3.239002in}{4.625697in}%
\pgfsys@useobject{currentmarker}{}%
\end{pgfscope}%
\begin{pgfscope}%
\pgfsys@transformshift{3.255154in}{4.873146in}%
\pgfsys@useobject{currentmarker}{}%
\end{pgfscope}%
\begin{pgfscope}%
\pgfsys@transformshift{3.271305in}{4.750410in}%
\pgfsys@useobject{currentmarker}{}%
\end{pgfscope}%
\begin{pgfscope}%
\pgfsys@transformshift{3.287457in}{4.749547in}%
\pgfsys@useobject{currentmarker}{}%
\end{pgfscope}%
\begin{pgfscope}%
\pgfsys@transformshift{3.303609in}{5.000199in}%
\pgfsys@useobject{currentmarker}{}%
\end{pgfscope}%
\begin{pgfscope}%
\pgfsys@transformshift{3.319760in}{4.517283in}%
\pgfsys@useobject{currentmarker}{}%
\end{pgfscope}%
\begin{pgfscope}%
\pgfsys@transformshift{3.335912in}{4.783274in}%
\pgfsys@useobject{currentmarker}{}%
\end{pgfscope}%
\begin{pgfscope}%
\pgfsys@transformshift{3.352063in}{4.762807in}%
\pgfsys@useobject{currentmarker}{}%
\end{pgfscope}%
\begin{pgfscope}%
\pgfsys@transformshift{3.368215in}{4.856605in}%
\pgfsys@useobject{currentmarker}{}%
\end{pgfscope}%
\begin{pgfscope}%
\pgfsys@transformshift{3.384367in}{4.750518in}%
\pgfsys@useobject{currentmarker}{}%
\end{pgfscope}%
\begin{pgfscope}%
\pgfsys@transformshift{3.400518in}{4.705817in}%
\pgfsys@useobject{currentmarker}{}%
\end{pgfscope}%
\begin{pgfscope}%
\pgfsys@transformshift{3.416670in}{4.704323in}%
\pgfsys@useobject{currentmarker}{}%
\end{pgfscope}%
\begin{pgfscope}%
\pgfsys@transformshift{3.432822in}{4.835067in}%
\pgfsys@useobject{currentmarker}{}%
\end{pgfscope}%
\begin{pgfscope}%
\pgfsys@transformshift{3.448973in}{4.939429in}%
\pgfsys@useobject{currentmarker}{}%
\end{pgfscope}%
\begin{pgfscope}%
\pgfsys@transformshift{3.465125in}{4.478195in}%
\pgfsys@useobject{currentmarker}{}%
\end{pgfscope}%
\begin{pgfscope}%
\pgfsys@transformshift{3.481277in}{5.036562in}%
\pgfsys@useobject{currentmarker}{}%
\end{pgfscope}%
\begin{pgfscope}%
\pgfsys@transformshift{3.497428in}{4.716775in}%
\pgfsys@useobject{currentmarker}{}%
\end{pgfscope}%
\begin{pgfscope}%
\pgfsys@transformshift{3.513580in}{4.534631in}%
\pgfsys@useobject{currentmarker}{}%
\end{pgfscope}%
\begin{pgfscope}%
\pgfsys@transformshift{3.529731in}{5.060828in}%
\pgfsys@useobject{currentmarker}{}%
\end{pgfscope}%
\begin{pgfscope}%
\pgfsys@transformshift{3.545883in}{4.707102in}%
\pgfsys@useobject{currentmarker}{}%
\end{pgfscope}%
\begin{pgfscope}%
\pgfsys@transformshift{3.562035in}{4.597085in}%
\pgfsys@useobject{currentmarker}{}%
\end{pgfscope}%
\begin{pgfscope}%
\pgfsys@transformshift{3.578186in}{4.819047in}%
\pgfsys@useobject{currentmarker}{}%
\end{pgfscope}%
\begin{pgfscope}%
\pgfsys@transformshift{3.594338in}{4.775393in}%
\pgfsys@useobject{currentmarker}{}%
\end{pgfscope}%
\begin{pgfscope}%
\pgfsys@transformshift{3.610490in}{4.925237in}%
\pgfsys@useobject{currentmarker}{}%
\end{pgfscope}%
\begin{pgfscope}%
\pgfsys@transformshift{3.626641in}{4.599243in}%
\pgfsys@useobject{currentmarker}{}%
\end{pgfscope}%
\begin{pgfscope}%
\pgfsys@transformshift{3.642793in}{4.956154in}%
\pgfsys@useobject{currentmarker}{}%
\end{pgfscope}%
\begin{pgfscope}%
\pgfsys@transformshift{3.658944in}{4.467672in}%
\pgfsys@useobject{currentmarker}{}%
\end{pgfscope}%
\begin{pgfscope}%
\pgfsys@transformshift{3.675096in}{5.005822in}%
\pgfsys@useobject{currentmarker}{}%
\end{pgfscope}%
\begin{pgfscope}%
\pgfsys@transformshift{3.691248in}{4.791707in}%
\pgfsys@useobject{currentmarker}{}%
\end{pgfscope}%
\begin{pgfscope}%
\pgfsys@transformshift{3.707399in}{4.555312in}%
\pgfsys@useobject{currentmarker}{}%
\end{pgfscope}%
\begin{pgfscope}%
\pgfsys@transformshift{3.723551in}{5.084255in}%
\pgfsys@useobject{currentmarker}{}%
\end{pgfscope}%
\begin{pgfscope}%
\pgfsys@transformshift{3.739703in}{4.347683in}%
\pgfsys@useobject{currentmarker}{}%
\end{pgfscope}%
\begin{pgfscope}%
\pgfsys@transformshift{3.755854in}{5.137815in}%
\pgfsys@useobject{currentmarker}{}%
\end{pgfscope}%
\begin{pgfscope}%
\pgfsys@transformshift{3.772006in}{4.596049in}%
\pgfsys@useobject{currentmarker}{}%
\end{pgfscope}%
\begin{pgfscope}%
\pgfsys@transformshift{3.788158in}{4.910224in}%
\pgfsys@useobject{currentmarker}{}%
\end{pgfscope}%
\begin{pgfscope}%
\pgfsys@transformshift{3.804309in}{4.663873in}%
\pgfsys@useobject{currentmarker}{}%
\end{pgfscope}%
\begin{pgfscope}%
\pgfsys@transformshift{3.820461in}{4.643213in}%
\pgfsys@useobject{currentmarker}{}%
\end{pgfscope}%
\begin{pgfscope}%
\pgfsys@transformshift{3.836612in}{5.038795in}%
\pgfsys@useobject{currentmarker}{}%
\end{pgfscope}%
\begin{pgfscope}%
\pgfsys@transformshift{3.852764in}{4.436734in}%
\pgfsys@useobject{currentmarker}{}%
\end{pgfscope}%
\begin{pgfscope}%
\pgfsys@transformshift{3.868916in}{5.113821in}%
\pgfsys@useobject{currentmarker}{}%
\end{pgfscope}%
\begin{pgfscope}%
\pgfsys@transformshift{3.885067in}{4.671103in}%
\pgfsys@useobject{currentmarker}{}%
\end{pgfscope}%
\begin{pgfscope}%
\pgfsys@transformshift{3.901219in}{4.653349in}%
\pgfsys@useobject{currentmarker}{}%
\end{pgfscope}%
\begin{pgfscope}%
\pgfsys@transformshift{3.917371in}{4.781258in}%
\pgfsys@useobject{currentmarker}{}%
\end{pgfscope}%
\begin{pgfscope}%
\pgfsys@transformshift{3.933522in}{4.817554in}%
\pgfsys@useobject{currentmarker}{}%
\end{pgfscope}%
\begin{pgfscope}%
\pgfsys@transformshift{3.949674in}{4.675278in}%
\pgfsys@useobject{currentmarker}{}%
\end{pgfscope}%
\begin{pgfscope}%
\pgfsys@transformshift{3.965826in}{4.919297in}%
\pgfsys@useobject{currentmarker}{}%
\end{pgfscope}%
\begin{pgfscope}%
\pgfsys@transformshift{3.981977in}{4.683941in}%
\pgfsys@useobject{currentmarker}{}%
\end{pgfscope}%
\begin{pgfscope}%
\pgfsys@transformshift{3.998129in}{4.963080in}%
\pgfsys@useobject{currentmarker}{}%
\end{pgfscope}%
\begin{pgfscope}%
\pgfsys@transformshift{4.014280in}{4.407157in}%
\pgfsys@useobject{currentmarker}{}%
\end{pgfscope}%
\begin{pgfscope}%
\pgfsys@transformshift{4.030432in}{5.038086in}%
\pgfsys@useobject{currentmarker}{}%
\end{pgfscope}%
\begin{pgfscope}%
\pgfsys@transformshift{4.046584in}{4.691393in}%
\pgfsys@useobject{currentmarker}{}%
\end{pgfscope}%
\begin{pgfscope}%
\pgfsys@transformshift{4.062735in}{4.843335in}%
\pgfsys@useobject{currentmarker}{}%
\end{pgfscope}%
\begin{pgfscope}%
\pgfsys@transformshift{4.078887in}{4.681760in}%
\pgfsys@useobject{currentmarker}{}%
\end{pgfscope}%
\begin{pgfscope}%
\pgfsys@transformshift{4.095039in}{4.827834in}%
\pgfsys@useobject{currentmarker}{}%
\end{pgfscope}%
\begin{pgfscope}%
\pgfsys@transformshift{4.111190in}{4.615817in}%
\pgfsys@useobject{currentmarker}{}%
\end{pgfscope}%
\begin{pgfscope}%
\pgfsys@transformshift{4.127342in}{5.030287in}%
\pgfsys@useobject{currentmarker}{}%
\end{pgfscope}%
\begin{pgfscope}%
\pgfsys@transformshift{4.143494in}{4.491629in}%
\pgfsys@useobject{currentmarker}{}%
\end{pgfscope}%
\begin{pgfscope}%
\pgfsys@transformshift{4.159645in}{5.027327in}%
\pgfsys@useobject{currentmarker}{}%
\end{pgfscope}%
\begin{pgfscope}%
\pgfsys@transformshift{4.175797in}{4.407214in}%
\pgfsys@useobject{currentmarker}{}%
\end{pgfscope}%
\begin{pgfscope}%
\pgfsys@transformshift{4.191948in}{5.235237in}%
\pgfsys@useobject{currentmarker}{}%
\end{pgfscope}%
\begin{pgfscope}%
\pgfsys@transformshift{4.208100in}{4.451932in}%
\pgfsys@useobject{currentmarker}{}%
\end{pgfscope}%
\begin{pgfscope}%
\pgfsys@transformshift{4.224252in}{4.812860in}%
\pgfsys@useobject{currentmarker}{}%
\end{pgfscope}%
\begin{pgfscope}%
\pgfsys@transformshift{4.240403in}{4.712957in}%
\pgfsys@useobject{currentmarker}{}%
\end{pgfscope}%
\begin{pgfscope}%
\pgfsys@transformshift{4.256555in}{4.876072in}%
\pgfsys@useobject{currentmarker}{}%
\end{pgfscope}%
\begin{pgfscope}%
\pgfsys@transformshift{4.272707in}{4.839324in}%
\pgfsys@useobject{currentmarker}{}%
\end{pgfscope}%
\begin{pgfscope}%
\pgfsys@transformshift{4.288858in}{4.671703in}%
\pgfsys@useobject{currentmarker}{}%
\end{pgfscope}%
\begin{pgfscope}%
\pgfsys@transformshift{4.305010in}{4.912062in}%
\pgfsys@useobject{currentmarker}{}%
\end{pgfscope}%
\begin{pgfscope}%
\pgfsys@transformshift{4.321162in}{4.549779in}%
\pgfsys@useobject{currentmarker}{}%
\end{pgfscope}%
\begin{pgfscope}%
\pgfsys@transformshift{4.337313in}{4.778864in}%
\pgfsys@useobject{currentmarker}{}%
\end{pgfscope}%
\begin{pgfscope}%
\pgfsys@transformshift{4.353465in}{4.919388in}%
\pgfsys@useobject{currentmarker}{}%
\end{pgfscope}%
\begin{pgfscope}%
\pgfsys@transformshift{4.369616in}{4.830784in}%
\pgfsys@useobject{currentmarker}{}%
\end{pgfscope}%
\begin{pgfscope}%
\pgfsys@transformshift{4.385768in}{4.675328in}%
\pgfsys@useobject{currentmarker}{}%
\end{pgfscope}%
\begin{pgfscope}%
\pgfsys@transformshift{4.401920in}{4.566958in}%
\pgfsys@useobject{currentmarker}{}%
\end{pgfscope}%
\begin{pgfscope}%
\pgfsys@transformshift{4.418071in}{4.988595in}%
\pgfsys@useobject{currentmarker}{}%
\end{pgfscope}%
\begin{pgfscope}%
\pgfsys@transformshift{4.434223in}{4.673878in}%
\pgfsys@useobject{currentmarker}{}%
\end{pgfscope}%
\begin{pgfscope}%
\pgfsys@transformshift{4.450375in}{4.773389in}%
\pgfsys@useobject{currentmarker}{}%
\end{pgfscope}%
\begin{pgfscope}%
\pgfsys@transformshift{4.466526in}{4.987184in}%
\pgfsys@useobject{currentmarker}{}%
\end{pgfscope}%
\begin{pgfscope}%
\pgfsys@transformshift{4.482678in}{4.552371in}%
\pgfsys@useobject{currentmarker}{}%
\end{pgfscope}%
\begin{pgfscope}%
\pgfsys@transformshift{4.498830in}{4.742467in}%
\pgfsys@useobject{currentmarker}{}%
\end{pgfscope}%
\begin{pgfscope}%
\pgfsys@transformshift{4.514981in}{4.862620in}%
\pgfsys@useobject{currentmarker}{}%
\end{pgfscope}%
\begin{pgfscope}%
\pgfsys@transformshift{4.531133in}{4.742455in}%
\pgfsys@useobject{currentmarker}{}%
\end{pgfscope}%
\begin{pgfscope}%
\pgfsys@transformshift{4.547284in}{4.730169in}%
\pgfsys@useobject{currentmarker}{}%
\end{pgfscope}%
\begin{pgfscope}%
\pgfsys@transformshift{4.563436in}{4.824280in}%
\pgfsys@useobject{currentmarker}{}%
\end{pgfscope}%
\begin{pgfscope}%
\pgfsys@transformshift{4.579588in}{4.968275in}%
\pgfsys@useobject{currentmarker}{}%
\end{pgfscope}%
\begin{pgfscope}%
\pgfsys@transformshift{4.595739in}{4.356604in}%
\pgfsys@useobject{currentmarker}{}%
\end{pgfscope}%
\begin{pgfscope}%
\pgfsys@transformshift{4.611891in}{4.932747in}%
\pgfsys@useobject{currentmarker}{}%
\end{pgfscope}%
\begin{pgfscope}%
\pgfsys@transformshift{4.628043in}{5.003390in}%
\pgfsys@useobject{currentmarker}{}%
\end{pgfscope}%
\begin{pgfscope}%
\pgfsys@transformshift{4.644194in}{4.391481in}%
\pgfsys@useobject{currentmarker}{}%
\end{pgfscope}%
\begin{pgfscope}%
\pgfsys@transformshift{4.660346in}{5.155251in}%
\pgfsys@useobject{currentmarker}{}%
\end{pgfscope}%
\begin{pgfscope}%
\pgfsys@transformshift{4.676497in}{4.461846in}%
\pgfsys@useobject{currentmarker}{}%
\end{pgfscope}%
\begin{pgfscope}%
\pgfsys@transformshift{4.692649in}{5.015798in}%
\pgfsys@useobject{currentmarker}{}%
\end{pgfscope}%
\begin{pgfscope}%
\pgfsys@transformshift{4.708801in}{4.448993in}%
\pgfsys@useobject{currentmarker}{}%
\end{pgfscope}%
\begin{pgfscope}%
\pgfsys@transformshift{4.724952in}{5.109148in}%
\pgfsys@useobject{currentmarker}{}%
\end{pgfscope}%
\begin{pgfscope}%
\pgfsys@transformshift{4.741104in}{4.584136in}%
\pgfsys@useobject{currentmarker}{}%
\end{pgfscope}%
\begin{pgfscope}%
\pgfsys@transformshift{4.757256in}{4.871327in}%
\pgfsys@useobject{currentmarker}{}%
\end{pgfscope}%
\begin{pgfscope}%
\pgfsys@transformshift{4.773407in}{4.684410in}%
\pgfsys@useobject{currentmarker}{}%
\end{pgfscope}%
\begin{pgfscope}%
\pgfsys@transformshift{4.789559in}{4.652067in}%
\pgfsys@useobject{currentmarker}{}%
\end{pgfscope}%
\begin{pgfscope}%
\pgfsys@transformshift{4.805711in}{4.808726in}%
\pgfsys@useobject{currentmarker}{}%
\end{pgfscope}%
\begin{pgfscope}%
\pgfsys@transformshift{4.821862in}{5.092235in}%
\pgfsys@useobject{currentmarker}{}%
\end{pgfscope}%
\begin{pgfscope}%
\pgfsys@transformshift{4.838014in}{4.339980in}%
\pgfsys@useobject{currentmarker}{}%
\end{pgfscope}%
\begin{pgfscope}%
\pgfsys@transformshift{4.854165in}{5.008507in}%
\pgfsys@useobject{currentmarker}{}%
\end{pgfscope}%
\begin{pgfscope}%
\pgfsys@transformshift{4.870317in}{4.742558in}%
\pgfsys@useobject{currentmarker}{}%
\end{pgfscope}%
\begin{pgfscope}%
\pgfsys@transformshift{4.886469in}{4.925279in}%
\pgfsys@useobject{currentmarker}{}%
\end{pgfscope}%
\begin{pgfscope}%
\pgfsys@transformshift{4.902620in}{4.497683in}%
\pgfsys@useobject{currentmarker}{}%
\end{pgfscope}%
\begin{pgfscope}%
\pgfsys@transformshift{4.918772in}{4.751631in}%
\pgfsys@useobject{currentmarker}{}%
\end{pgfscope}%
\begin{pgfscope}%
\pgfsys@transformshift{4.934924in}{4.979393in}%
\pgfsys@useobject{currentmarker}{}%
\end{pgfscope}%
\begin{pgfscope}%
\pgfsys@transformshift{4.951075in}{4.654608in}%
\pgfsys@useobject{currentmarker}{}%
\end{pgfscope}%
\begin{pgfscope}%
\pgfsys@transformshift{4.967227in}{4.952973in}%
\pgfsys@useobject{currentmarker}{}%
\end{pgfscope}%
\begin{pgfscope}%
\pgfsys@transformshift{4.983379in}{4.636374in}%
\pgfsys@useobject{currentmarker}{}%
\end{pgfscope}%
\begin{pgfscope}%
\pgfsys@transformshift{4.999530in}{4.614686in}%
\pgfsys@useobject{currentmarker}{}%
\end{pgfscope}%
\begin{pgfscope}%
\pgfsys@transformshift{5.015682in}{4.822922in}%
\pgfsys@useobject{currentmarker}{}%
\end{pgfscope}%
\end{pgfscope}%
\begin{pgfscope}%
\pgfsetrectcap%
\pgfsetmiterjoin%
\pgfsetlinewidth{0.803000pt}%
\definecolor{currentstroke}{rgb}{0.000000,0.000000,0.000000}%
\pgfsetstrokecolor{currentstroke}%
\pgfsetdash{}{0pt}%
\pgfpathmoveto{\pgfqpoint{0.725000in}{4.295217in}}%
\pgfpathlineto{\pgfqpoint{0.725000in}{5.280000in}}%
\pgfusepath{stroke}%
\end{pgfscope}%
\begin{pgfscope}%
\pgfsetrectcap%
\pgfsetmiterjoin%
\pgfsetlinewidth{0.803000pt}%
\definecolor{currentstroke}{rgb}{0.000000,0.000000,0.000000}%
\pgfsetstrokecolor{currentstroke}%
\pgfsetdash{}{0pt}%
\pgfpathmoveto{\pgfqpoint{5.220000in}{4.295217in}}%
\pgfpathlineto{\pgfqpoint{5.220000in}{5.280000in}}%
\pgfusepath{stroke}%
\end{pgfscope}%
\begin{pgfscope}%
\pgfsetrectcap%
\pgfsetmiterjoin%
\pgfsetlinewidth{0.803000pt}%
\definecolor{currentstroke}{rgb}{0.000000,0.000000,0.000000}%
\pgfsetstrokecolor{currentstroke}%
\pgfsetdash{}{0pt}%
\pgfpathmoveto{\pgfqpoint{0.725000in}{4.295217in}}%
\pgfpathlineto{\pgfqpoint{5.220000in}{4.295217in}}%
\pgfusepath{stroke}%
\end{pgfscope}%
\begin{pgfscope}%
\pgfsetrectcap%
\pgfsetmiterjoin%
\pgfsetlinewidth{0.803000pt}%
\definecolor{currentstroke}{rgb}{0.000000,0.000000,0.000000}%
\pgfsetstrokecolor{currentstroke}%
\pgfsetdash{}{0pt}%
\pgfpathmoveto{\pgfqpoint{0.725000in}{5.280000in}}%
\pgfpathlineto{\pgfqpoint{5.220000in}{5.280000in}}%
\pgfusepath{stroke}%
\end{pgfscope}%
\begin{pgfscope}%
\pgfsetbuttcap%
\pgfsetmiterjoin%
\definecolor{currentfill}{rgb}{1.000000,1.000000,1.000000}%
\pgfsetfillcolor{currentfill}%
\pgfsetlinewidth{0.000000pt}%
\definecolor{currentstroke}{rgb}{0.000000,0.000000,0.000000}%
\pgfsetstrokecolor{currentstroke}%
\pgfsetstrokeopacity{0.000000}%
\pgfsetdash{}{0pt}%
\pgfpathmoveto{\pgfqpoint{0.725000in}{3.113478in}}%
\pgfpathlineto{\pgfqpoint{5.220000in}{3.113478in}}%
\pgfpathlineto{\pgfqpoint{5.220000in}{4.098261in}}%
\pgfpathlineto{\pgfqpoint{0.725000in}{4.098261in}}%
\pgfpathclose%
\pgfusepath{fill}%
\end{pgfscope}%
\begin{pgfscope}%
\pgfsetbuttcap%
\pgfsetroundjoin%
\definecolor{currentfill}{rgb}{0.000000,0.000000,0.000000}%
\pgfsetfillcolor{currentfill}%
\pgfsetlinewidth{0.803000pt}%
\definecolor{currentstroke}{rgb}{0.000000,0.000000,0.000000}%
\pgfsetstrokecolor{currentstroke}%
\pgfsetdash{}{0pt}%
\pgfsys@defobject{currentmarker}{\pgfqpoint{0.000000in}{-0.048611in}}{\pgfqpoint{0.000000in}{0.000000in}}{%
\pgfpathmoveto{\pgfqpoint{0.000000in}{0.000000in}}%
\pgfpathlineto{\pgfqpoint{0.000000in}{-0.048611in}}%
\pgfusepath{stroke,fill}%
}%
\begin{pgfscope}%
\pgfsys@transformshift{0.913167in}{3.113478in}%
\pgfsys@useobject{currentmarker}{}%
\end{pgfscope}%
\end{pgfscope}%
\begin{pgfscope}%
\pgfsetbuttcap%
\pgfsetroundjoin%
\definecolor{currentfill}{rgb}{0.000000,0.000000,0.000000}%
\pgfsetfillcolor{currentfill}%
\pgfsetlinewidth{0.803000pt}%
\definecolor{currentstroke}{rgb}{0.000000,0.000000,0.000000}%
\pgfsetstrokecolor{currentstroke}%
\pgfsetdash{}{0pt}%
\pgfsys@defobject{currentmarker}{\pgfqpoint{0.000000in}{-0.048611in}}{\pgfqpoint{0.000000in}{0.000000in}}{%
\pgfpathmoveto{\pgfqpoint{0.000000in}{0.000000in}}%
\pgfpathlineto{\pgfqpoint{0.000000in}{-0.048611in}}%
\pgfusepath{stroke,fill}%
}%
\begin{pgfscope}%
\pgfsys@transformshift{1.720748in}{3.113478in}%
\pgfsys@useobject{currentmarker}{}%
\end{pgfscope}%
\end{pgfscope}%
\begin{pgfscope}%
\pgfsetbuttcap%
\pgfsetroundjoin%
\definecolor{currentfill}{rgb}{0.000000,0.000000,0.000000}%
\pgfsetfillcolor{currentfill}%
\pgfsetlinewidth{0.803000pt}%
\definecolor{currentstroke}{rgb}{0.000000,0.000000,0.000000}%
\pgfsetstrokecolor{currentstroke}%
\pgfsetdash{}{0pt}%
\pgfsys@defobject{currentmarker}{\pgfqpoint{0.000000in}{-0.048611in}}{\pgfqpoint{0.000000in}{0.000000in}}{%
\pgfpathmoveto{\pgfqpoint{0.000000in}{0.000000in}}%
\pgfpathlineto{\pgfqpoint{0.000000in}{-0.048611in}}%
\pgfusepath{stroke,fill}%
}%
\begin{pgfscope}%
\pgfsys@transformshift{2.528330in}{3.113478in}%
\pgfsys@useobject{currentmarker}{}%
\end{pgfscope}%
\end{pgfscope}%
\begin{pgfscope}%
\pgfsetbuttcap%
\pgfsetroundjoin%
\definecolor{currentfill}{rgb}{0.000000,0.000000,0.000000}%
\pgfsetfillcolor{currentfill}%
\pgfsetlinewidth{0.803000pt}%
\definecolor{currentstroke}{rgb}{0.000000,0.000000,0.000000}%
\pgfsetstrokecolor{currentstroke}%
\pgfsetdash{}{0pt}%
\pgfsys@defobject{currentmarker}{\pgfqpoint{0.000000in}{-0.048611in}}{\pgfqpoint{0.000000in}{0.000000in}}{%
\pgfpathmoveto{\pgfqpoint{0.000000in}{0.000000in}}%
\pgfpathlineto{\pgfqpoint{0.000000in}{-0.048611in}}%
\pgfusepath{stroke,fill}%
}%
\begin{pgfscope}%
\pgfsys@transformshift{3.335912in}{3.113478in}%
\pgfsys@useobject{currentmarker}{}%
\end{pgfscope}%
\end{pgfscope}%
\begin{pgfscope}%
\pgfsetbuttcap%
\pgfsetroundjoin%
\definecolor{currentfill}{rgb}{0.000000,0.000000,0.000000}%
\pgfsetfillcolor{currentfill}%
\pgfsetlinewidth{0.803000pt}%
\definecolor{currentstroke}{rgb}{0.000000,0.000000,0.000000}%
\pgfsetstrokecolor{currentstroke}%
\pgfsetdash{}{0pt}%
\pgfsys@defobject{currentmarker}{\pgfqpoint{0.000000in}{-0.048611in}}{\pgfqpoint{0.000000in}{0.000000in}}{%
\pgfpathmoveto{\pgfqpoint{0.000000in}{0.000000in}}%
\pgfpathlineto{\pgfqpoint{0.000000in}{-0.048611in}}%
\pgfusepath{stroke,fill}%
}%
\begin{pgfscope}%
\pgfsys@transformshift{4.143494in}{3.113478in}%
\pgfsys@useobject{currentmarker}{}%
\end{pgfscope}%
\end{pgfscope}%
\begin{pgfscope}%
\pgfsetbuttcap%
\pgfsetroundjoin%
\definecolor{currentfill}{rgb}{0.000000,0.000000,0.000000}%
\pgfsetfillcolor{currentfill}%
\pgfsetlinewidth{0.803000pt}%
\definecolor{currentstroke}{rgb}{0.000000,0.000000,0.000000}%
\pgfsetstrokecolor{currentstroke}%
\pgfsetdash{}{0pt}%
\pgfsys@defobject{currentmarker}{\pgfqpoint{0.000000in}{-0.048611in}}{\pgfqpoint{0.000000in}{0.000000in}}{%
\pgfpathmoveto{\pgfqpoint{0.000000in}{0.000000in}}%
\pgfpathlineto{\pgfqpoint{0.000000in}{-0.048611in}}%
\pgfusepath{stroke,fill}%
}%
\begin{pgfscope}%
\pgfsys@transformshift{4.951075in}{3.113478in}%
\pgfsys@useobject{currentmarker}{}%
\end{pgfscope}%
\end{pgfscope}%
\begin{pgfscope}%
\pgfsetbuttcap%
\pgfsetroundjoin%
\definecolor{currentfill}{rgb}{0.000000,0.000000,0.000000}%
\pgfsetfillcolor{currentfill}%
\pgfsetlinewidth{0.803000pt}%
\definecolor{currentstroke}{rgb}{0.000000,0.000000,0.000000}%
\pgfsetstrokecolor{currentstroke}%
\pgfsetdash{}{0pt}%
\pgfsys@defobject{currentmarker}{\pgfqpoint{-0.048611in}{0.000000in}}{\pgfqpoint{0.000000in}{0.000000in}}{%
\pgfpathmoveto{\pgfqpoint{0.000000in}{0.000000in}}%
\pgfpathlineto{\pgfqpoint{-0.048611in}{0.000000in}}%
\pgfusepath{stroke,fill}%
}%
\begin{pgfscope}%
\pgfsys@transformshift{0.725000in}{3.223506in}%
\pgfsys@useobject{currentmarker}{}%
\end{pgfscope}%
\end{pgfscope}%
\begin{pgfscope}%
\definecolor{textcolor}{rgb}{0.000000,0.000000,0.000000}%
\pgfsetstrokecolor{textcolor}%
\pgfsetfillcolor{textcolor}%
\pgftext[x=0.326222in,y=3.184951in,left,base]{\color{textcolor}\rmfamily\fontsize{8.000000}{9.600000}\selectfont −0.05}%
\end{pgfscope}%
\begin{pgfscope}%
\pgfsetbuttcap%
\pgfsetroundjoin%
\definecolor{currentfill}{rgb}{0.000000,0.000000,0.000000}%
\pgfsetfillcolor{currentfill}%
\pgfsetlinewidth{0.803000pt}%
\definecolor{currentstroke}{rgb}{0.000000,0.000000,0.000000}%
\pgfsetstrokecolor{currentstroke}%
\pgfsetdash{}{0pt}%
\pgfsys@defobject{currentmarker}{\pgfqpoint{-0.048611in}{0.000000in}}{\pgfqpoint{0.000000in}{0.000000in}}{%
\pgfpathmoveto{\pgfqpoint{0.000000in}{0.000000in}}%
\pgfpathlineto{\pgfqpoint{-0.048611in}{0.000000in}}%
\pgfusepath{stroke,fill}%
}%
\begin{pgfscope}%
\pgfsys@transformshift{0.725000in}{3.636117in}%
\pgfsys@useobject{currentmarker}{}%
\end{pgfscope}%
\end{pgfscope}%
\begin{pgfscope}%
\definecolor{textcolor}{rgb}{0.000000,0.000000,0.000000}%
\pgfsetstrokecolor{textcolor}%
\pgfsetfillcolor{textcolor}%
\pgftext[x=0.418000in,y=3.597561in,left,base]{\color{textcolor}\rmfamily\fontsize{8.000000}{9.600000}\selectfont 0.00}%
\end{pgfscope}%
\begin{pgfscope}%
\pgfsetbuttcap%
\pgfsetroundjoin%
\definecolor{currentfill}{rgb}{0.000000,0.000000,0.000000}%
\pgfsetfillcolor{currentfill}%
\pgfsetlinewidth{0.803000pt}%
\definecolor{currentstroke}{rgb}{0.000000,0.000000,0.000000}%
\pgfsetstrokecolor{currentstroke}%
\pgfsetdash{}{0pt}%
\pgfsys@defobject{currentmarker}{\pgfqpoint{-0.048611in}{0.000000in}}{\pgfqpoint{0.000000in}{0.000000in}}{%
\pgfpathmoveto{\pgfqpoint{0.000000in}{0.000000in}}%
\pgfpathlineto{\pgfqpoint{-0.048611in}{0.000000in}}%
\pgfusepath{stroke,fill}%
}%
\begin{pgfscope}%
\pgfsys@transformshift{0.725000in}{4.048727in}%
\pgfsys@useobject{currentmarker}{}%
\end{pgfscope}%
\end{pgfscope}%
\begin{pgfscope}%
\definecolor{textcolor}{rgb}{0.000000,0.000000,0.000000}%
\pgfsetstrokecolor{textcolor}%
\pgfsetfillcolor{textcolor}%
\pgftext[x=0.418000in,y=4.010171in,left,base]{\color{textcolor}\rmfamily\fontsize{8.000000}{9.600000}\selectfont 0.05}%
\end{pgfscope}%
\begin{pgfscope}%
\pgfpathrectangle{\pgfqpoint{0.725000in}{3.113478in}}{\pgfqpoint{4.495000in}{0.984783in}}%
\pgfusepath{clip}%
\pgfsetrectcap%
\pgfsetroundjoin%
\pgfsetlinewidth{1.505625pt}%
\definecolor{currentstroke}{rgb}{1.000000,0.498039,0.054902}%
\pgfsetstrokecolor{currentstroke}%
\pgfsetdash{}{0pt}%
\pgfpathmoveto{\pgfqpoint{0.953546in}{3.497451in}}%
\pgfpathlineto{\pgfqpoint{0.969697in}{3.497451in}}%
\pgfpathlineto{\pgfqpoint{0.969697in}{3.821726in}}%
\pgfpathlineto{\pgfqpoint{1.002001in}{3.821726in}}%
\pgfpathlineto{\pgfqpoint{1.002001in}{3.726488in}}%
\pgfpathlineto{\pgfqpoint{1.034304in}{3.726488in}}%
\pgfpathlineto{\pgfqpoint{1.034304in}{3.413339in}}%
\pgfpathlineto{\pgfqpoint{1.066607in}{3.413339in}}%
\pgfpathlineto{\pgfqpoint{1.066607in}{3.583776in}}%
\pgfpathlineto{\pgfqpoint{1.098910in}{3.583776in}}%
\pgfpathlineto{\pgfqpoint{1.098910in}{3.886176in}}%
\pgfpathlineto{\pgfqpoint{1.131214in}{3.886176in}}%
\pgfpathlineto{\pgfqpoint{1.131214in}{3.497142in}}%
\pgfpathlineto{\pgfqpoint{1.163517in}{3.497142in}}%
\pgfpathlineto{\pgfqpoint{1.163517in}{3.637638in}}%
\pgfpathlineto{\pgfqpoint{1.195820in}{3.637638in}}%
\pgfpathlineto{\pgfqpoint{1.195820in}{3.723113in}}%
\pgfpathlineto{\pgfqpoint{1.228123in}{3.723113in}}%
\pgfpathlineto{\pgfqpoint{1.228123in}{3.386125in}}%
\pgfpathlineto{\pgfqpoint{1.260427in}{3.386125in}}%
\pgfpathlineto{\pgfqpoint{1.260427in}{4.021011in}}%
\pgfpathlineto{\pgfqpoint{1.292730in}{4.021011in}}%
\pgfpathlineto{\pgfqpoint{1.292730in}{3.317060in}}%
\pgfpathlineto{\pgfqpoint{1.325033in}{3.317060in}}%
\pgfpathlineto{\pgfqpoint{1.325033in}{3.769406in}}%
\pgfpathlineto{\pgfqpoint{1.357337in}{3.769406in}}%
\pgfpathlineto{\pgfqpoint{1.357337in}{3.739988in}}%
\pgfpathlineto{\pgfqpoint{1.389640in}{3.739988in}}%
\pgfpathlineto{\pgfqpoint{1.389640in}{3.450534in}}%
\pgfpathlineto{\pgfqpoint{1.421943in}{3.450534in}}%
\pgfpathlineto{\pgfqpoint{1.421943in}{3.607760in}}%
\pgfpathlineto{\pgfqpoint{1.454246in}{3.607760in}}%
\pgfpathlineto{\pgfqpoint{1.454246in}{3.760485in}}%
\pgfpathlineto{\pgfqpoint{1.486550in}{3.760485in}}%
\pgfpathlineto{\pgfqpoint{1.486550in}{3.832746in}}%
\pgfpathlineto{\pgfqpoint{1.518853in}{3.832746in}}%
\pgfpathlineto{\pgfqpoint{1.518853in}{3.165277in}}%
\pgfpathlineto{\pgfqpoint{1.551156in}{3.165277in}}%
\pgfpathlineto{\pgfqpoint{1.551156in}{4.053498in}}%
\pgfpathlineto{\pgfqpoint{1.583459in}{4.053498in}}%
\pgfpathlineto{\pgfqpoint{1.583459in}{3.559683in}}%
\pgfpathlineto{\pgfqpoint{1.615763in}{3.559683in}}%
\pgfpathlineto{\pgfqpoint{1.615763in}{3.508427in}}%
\pgfpathlineto{\pgfqpoint{1.648066in}{3.508427in}}%
\pgfpathlineto{\pgfqpoint{1.648066in}{3.682695in}}%
\pgfpathlineto{\pgfqpoint{1.680369in}{3.682695in}}%
\pgfpathlineto{\pgfqpoint{1.680369in}{3.732089in}}%
\pgfpathlineto{\pgfqpoint{1.712672in}{3.732089in}}%
\pgfpathlineto{\pgfqpoint{1.712672in}{3.510638in}}%
\pgfpathlineto{\pgfqpoint{1.744976in}{3.510638in}}%
\pgfpathlineto{\pgfqpoint{1.744976in}{3.921865in}}%
\pgfpathlineto{\pgfqpoint{1.777279in}{3.921865in}}%
\pgfpathlineto{\pgfqpoint{1.777279in}{3.309988in}}%
\pgfpathlineto{\pgfqpoint{1.809582in}{3.309988in}}%
\pgfpathlineto{\pgfqpoint{1.809582in}{3.555237in}}%
\pgfpathlineto{\pgfqpoint{1.841886in}{3.555237in}}%
\pgfpathlineto{\pgfqpoint{1.841886in}{3.981901in}}%
\pgfpathlineto{\pgfqpoint{1.874189in}{3.981901in}}%
\pgfpathlineto{\pgfqpoint{1.874189in}{3.531459in}}%
\pgfpathlineto{\pgfqpoint{1.906492in}{3.531459in}}%
\pgfpathlineto{\pgfqpoint{1.906492in}{3.594048in}}%
\pgfpathlineto{\pgfqpoint{1.938795in}{3.594048in}}%
\pgfpathlineto{\pgfqpoint{1.938795in}{3.752549in}}%
\pgfpathlineto{\pgfqpoint{1.971099in}{3.752549in}}%
\pgfpathlineto{\pgfqpoint{1.971099in}{3.504211in}}%
\pgfpathlineto{\pgfqpoint{2.003402in}{3.504211in}}%
\pgfpathlineto{\pgfqpoint{2.003402in}{3.797959in}}%
\pgfpathlineto{\pgfqpoint{2.035705in}{3.797959in}}%
\pgfpathlineto{\pgfqpoint{2.035705in}{3.219931in}}%
\pgfpathlineto{\pgfqpoint{2.068008in}{3.219931in}}%
\pgfpathlineto{\pgfqpoint{2.068008in}{4.041327in}}%
\pgfpathlineto{\pgfqpoint{2.100312in}{4.041327in}}%
\pgfpathlineto{\pgfqpoint{2.100312in}{3.601791in}}%
\pgfpathlineto{\pgfqpoint{2.132615in}{3.601791in}}%
\pgfpathlineto{\pgfqpoint{2.132615in}{3.533489in}}%
\pgfpathlineto{\pgfqpoint{2.164918in}{3.533489in}}%
\pgfpathlineto{\pgfqpoint{2.164918in}{3.755890in}}%
\pgfpathlineto{\pgfqpoint{2.197222in}{3.755890in}}%
\pgfpathlineto{\pgfqpoint{2.197222in}{3.741098in}}%
\pgfpathlineto{\pgfqpoint{2.229525in}{3.741098in}}%
\pgfpathlineto{\pgfqpoint{2.229525in}{3.422281in}}%
\pgfpathlineto{\pgfqpoint{2.261828in}{3.422281in}}%
\pgfpathlineto{\pgfqpoint{2.261828in}{3.703579in}}%
\pgfpathlineto{\pgfqpoint{2.294131in}{3.703579in}}%
\pgfpathlineto{\pgfqpoint{2.294131in}{3.822806in}}%
\pgfpathlineto{\pgfqpoint{2.326435in}{3.822806in}}%
\pgfpathlineto{\pgfqpoint{2.326435in}{3.218071in}}%
\pgfpathlineto{\pgfqpoint{2.358738in}{3.218071in}}%
\pgfpathlineto{\pgfqpoint{2.358738in}{3.894914in}}%
\pgfpathlineto{\pgfqpoint{2.391041in}{3.894914in}}%
\pgfpathlineto{\pgfqpoint{2.391041in}{3.863138in}}%
\pgfpathlineto{\pgfqpoint{2.423344in}{3.863138in}}%
\pgfpathlineto{\pgfqpoint{2.423344in}{3.430302in}}%
\pgfpathlineto{\pgfqpoint{2.455648in}{3.430302in}}%
\pgfpathlineto{\pgfqpoint{2.455648in}{3.606206in}}%
\pgfpathlineto{\pgfqpoint{2.487951in}{3.606206in}}%
\pgfpathlineto{\pgfqpoint{2.487951in}{3.883222in}}%
\pgfpathlineto{\pgfqpoint{2.520254in}{3.883222in}}%
\pgfpathlineto{\pgfqpoint{2.520254in}{3.290601in}}%
\pgfpathlineto{\pgfqpoint{2.552557in}{3.290601in}}%
\pgfpathlineto{\pgfqpoint{2.552557in}{3.870520in}}%
\pgfpathlineto{\pgfqpoint{2.584861in}{3.870520in}}%
\pgfpathlineto{\pgfqpoint{2.584861in}{3.579428in}}%
\pgfpathlineto{\pgfqpoint{2.617164in}{3.579428in}}%
\pgfpathlineto{\pgfqpoint{2.617164in}{3.491942in}}%
\pgfpathlineto{\pgfqpoint{2.649467in}{3.491942in}}%
\pgfpathlineto{\pgfqpoint{2.649467in}{4.052851in}}%
\pgfpathlineto{\pgfqpoint{2.681771in}{4.052851in}}%
\pgfpathlineto{\pgfqpoint{2.681771in}{3.484031in}}%
\pgfpathlineto{\pgfqpoint{2.714074in}{3.484031in}}%
\pgfpathlineto{\pgfqpoint{2.714074in}{3.369702in}}%
\pgfpathlineto{\pgfqpoint{2.746377in}{3.369702in}}%
\pgfpathlineto{\pgfqpoint{2.746377in}{3.879187in}}%
\pgfpathlineto{\pgfqpoint{2.778680in}{3.879187in}}%
\pgfpathlineto{\pgfqpoint{2.778680in}{3.462543in}}%
\pgfpathlineto{\pgfqpoint{2.810984in}{3.462543in}}%
\pgfpathlineto{\pgfqpoint{2.810984in}{3.856106in}}%
\pgfpathlineto{\pgfqpoint{2.843287in}{3.856106in}}%
\pgfpathlineto{\pgfqpoint{2.843287in}{3.566668in}}%
\pgfpathlineto{\pgfqpoint{2.875590in}{3.566668in}}%
\pgfpathlineto{\pgfqpoint{2.875590in}{3.834191in}}%
\pgfpathlineto{\pgfqpoint{2.907893in}{3.834191in}}%
\pgfpathlineto{\pgfqpoint{2.907893in}{3.521221in}}%
\pgfpathlineto{\pgfqpoint{2.940197in}{3.521221in}}%
\pgfpathlineto{\pgfqpoint{2.940197in}{3.600953in}}%
\pgfpathlineto{\pgfqpoint{2.972500in}{3.600953in}}%
\pgfpathlineto{\pgfqpoint{2.972500in}{3.597163in}}%
\pgfpathlineto{\pgfqpoint{3.004803in}{3.597163in}}%
\pgfpathlineto{\pgfqpoint{3.004803in}{3.560384in}}%
\pgfpathlineto{\pgfqpoint{3.037107in}{3.560384in}}%
\pgfpathlineto{\pgfqpoint{3.037107in}{3.931072in}}%
\pgfpathlineto{\pgfqpoint{3.069410in}{3.931072in}}%
\pgfpathlineto{\pgfqpoint{3.069410in}{3.498370in}}%
\pgfpathlineto{\pgfqpoint{3.101713in}{3.498370in}}%
\pgfpathlineto{\pgfqpoint{3.101713in}{3.698850in}}%
\pgfpathlineto{\pgfqpoint{3.134016in}{3.698850in}}%
\pgfpathlineto{\pgfqpoint{3.134016in}{3.662590in}}%
\pgfpathlineto{\pgfqpoint{3.166320in}{3.662590in}}%
\pgfpathlineto{\pgfqpoint{3.166320in}{3.442281in}}%
\pgfpathlineto{\pgfqpoint{3.198623in}{3.442281in}}%
\pgfpathlineto{\pgfqpoint{3.198623in}{3.847972in}}%
\pgfpathlineto{\pgfqpoint{3.230926in}{3.847972in}}%
\pgfpathlineto{\pgfqpoint{3.230926in}{3.558251in}}%
\pgfpathlineto{\pgfqpoint{3.263229in}{3.558251in}}%
\pgfpathlineto{\pgfqpoint{3.263229in}{3.829281in}}%
\pgfpathlineto{\pgfqpoint{3.295533in}{3.829281in}}%
\pgfpathlineto{\pgfqpoint{3.295533in}{3.600378in}}%
\pgfpathlineto{\pgfqpoint{3.327836in}{3.600378in}}%
\pgfpathlineto{\pgfqpoint{3.327836in}{3.542240in}}%
\pgfpathlineto{\pgfqpoint{3.360139in}{3.542240in}}%
\pgfpathlineto{\pgfqpoint{3.360139in}{3.763125in}}%
\pgfpathlineto{\pgfqpoint{3.392443in}{3.763125in}}%
\pgfpathlineto{\pgfqpoint{3.392443in}{3.386331in}}%
\pgfpathlineto{\pgfqpoint{3.424746in}{3.386331in}}%
\pgfpathlineto{\pgfqpoint{3.424746in}{3.934841in}}%
\pgfpathlineto{\pgfqpoint{3.457049in}{3.934841in}}%
\pgfpathlineto{\pgfqpoint{3.457049in}{3.691964in}}%
\pgfpathlineto{\pgfqpoint{3.489352in}{3.691964in}}%
\pgfpathlineto{\pgfqpoint{3.489352in}{3.414636in}}%
\pgfpathlineto{\pgfqpoint{3.521656in}{3.414636in}}%
\pgfpathlineto{\pgfqpoint{3.521656in}{3.878800in}}%
\pgfpathlineto{\pgfqpoint{3.553959in}{3.878800in}}%
\pgfpathlineto{\pgfqpoint{3.553959in}{3.318674in}}%
\pgfpathlineto{\pgfqpoint{3.586262in}{3.318674in}}%
\pgfpathlineto{\pgfqpoint{3.586262in}{3.946112in}}%
\pgfpathlineto{\pgfqpoint{3.618565in}{3.946112in}}%
\pgfpathlineto{\pgfqpoint{3.618565in}{3.579579in}}%
\pgfpathlineto{\pgfqpoint{3.650869in}{3.579579in}}%
\pgfpathlineto{\pgfqpoint{3.650869in}{3.666419in}}%
\pgfpathlineto{\pgfqpoint{3.683172in}{3.666419in}}%
\pgfpathlineto{\pgfqpoint{3.683172in}{3.636842in}}%
\pgfpathlineto{\pgfqpoint{3.715475in}{3.636842in}}%
\pgfpathlineto{\pgfqpoint{3.715475in}{3.523806in}}%
\pgfpathlineto{\pgfqpoint{3.747778in}{3.523806in}}%
\pgfpathlineto{\pgfqpoint{3.747778in}{3.897870in}}%
\pgfpathlineto{\pgfqpoint{3.780082in}{3.897870in}}%
\pgfpathlineto{\pgfqpoint{3.780082in}{3.502035in}}%
\pgfpathlineto{\pgfqpoint{3.812385in}{3.502035in}}%
\pgfpathlineto{\pgfqpoint{3.812385in}{3.649159in}}%
\pgfpathlineto{\pgfqpoint{3.844688in}{3.649159in}}%
\pgfpathlineto{\pgfqpoint{3.844688in}{3.820018in}}%
\pgfpathlineto{\pgfqpoint{3.876992in}{3.820018in}}%
\pgfpathlineto{\pgfqpoint{3.876992in}{3.428012in}}%
\pgfpathlineto{\pgfqpoint{3.909295in}{3.428012in}}%
\pgfpathlineto{\pgfqpoint{3.909295in}{3.632017in}}%
\pgfpathlineto{\pgfqpoint{3.941598in}{3.632017in}}%
\pgfpathlineto{\pgfqpoint{3.941598in}{3.759332in}}%
\pgfpathlineto{\pgfqpoint{3.973901in}{3.759332in}}%
\pgfpathlineto{\pgfqpoint{3.973901in}{3.745256in}}%
\pgfpathlineto{\pgfqpoint{4.006205in}{3.745256in}}%
\pgfpathlineto{\pgfqpoint{4.006205in}{3.524806in}}%
\pgfpathlineto{\pgfqpoint{4.038508in}{3.524806in}}%
\pgfpathlineto{\pgfqpoint{4.038508in}{3.790180in}}%
\pgfpathlineto{\pgfqpoint{4.070811in}{3.790180in}}%
\pgfpathlineto{\pgfqpoint{4.070811in}{3.523005in}}%
\pgfpathlineto{\pgfqpoint{4.103114in}{3.523005in}}%
\pgfpathlineto{\pgfqpoint{4.103114in}{3.737512in}}%
\pgfpathlineto{\pgfqpoint{4.135418in}{3.737512in}}%
\pgfpathlineto{\pgfqpoint{4.135418in}{3.528121in}}%
\pgfpathlineto{\pgfqpoint{4.167721in}{3.528121in}}%
\pgfpathlineto{\pgfqpoint{4.167721in}{3.855831in}}%
\pgfpathlineto{\pgfqpoint{4.200024in}{3.855831in}}%
\pgfpathlineto{\pgfqpoint{4.200024in}{3.327957in}}%
\pgfpathlineto{\pgfqpoint{4.232328in}{3.327957in}}%
\pgfpathlineto{\pgfqpoint{4.232328in}{3.865893in}}%
\pgfpathlineto{\pgfqpoint{4.264631in}{3.865893in}}%
\pgfpathlineto{\pgfqpoint{4.264631in}{3.787944in}}%
\pgfpathlineto{\pgfqpoint{4.296934in}{3.787944in}}%
\pgfpathlineto{\pgfqpoint{4.296934in}{3.402300in}}%
\pgfpathlineto{\pgfqpoint{4.329237in}{3.402300in}}%
\pgfpathlineto{\pgfqpoint{4.329237in}{3.910449in}}%
\pgfpathlineto{\pgfqpoint{4.361541in}{3.910449in}}%
\pgfpathlineto{\pgfqpoint{4.361541in}{3.536320in}}%
\pgfpathlineto{\pgfqpoint{4.393844in}{3.536320in}}%
\pgfpathlineto{\pgfqpoint{4.393844in}{3.545209in}}%
\pgfpathlineto{\pgfqpoint{4.426147in}{3.545209in}}%
\pgfpathlineto{\pgfqpoint{4.426147in}{3.785053in}}%
\pgfpathlineto{\pgfqpoint{4.458450in}{3.785053in}}%
\pgfpathlineto{\pgfqpoint{4.458450in}{3.639536in}}%
\pgfpathlineto{\pgfqpoint{4.490754in}{3.639536in}}%
\pgfpathlineto{\pgfqpoint{4.490754in}{3.620111in}}%
\pgfpathlineto{\pgfqpoint{4.523057in}{3.620111in}}%
\pgfpathlineto{\pgfqpoint{4.523057in}{3.615885in}}%
\pgfpathlineto{\pgfqpoint{4.555360in}{3.615885in}}%
\pgfpathlineto{\pgfqpoint{4.555360in}{3.901797in}}%
\pgfpathlineto{\pgfqpoint{4.587663in}{3.901797in}}%
\pgfpathlineto{\pgfqpoint{4.587663in}{3.405181in}}%
\pgfpathlineto{\pgfqpoint{4.619967in}{3.405181in}}%
\pgfpathlineto{\pgfqpoint{4.619967in}{3.749307in}}%
\pgfpathlineto{\pgfqpoint{4.652270in}{3.749307in}}%
\pgfpathlineto{\pgfqpoint{4.652270in}{3.738144in}}%
\pgfpathlineto{\pgfqpoint{4.684573in}{3.738144in}}%
\pgfpathlineto{\pgfqpoint{4.684573in}{3.505066in}}%
\pgfpathlineto{\pgfqpoint{4.716877in}{3.505066in}}%
\pgfpathlineto{\pgfqpoint{4.716877in}{3.850710in}}%
\pgfpathlineto{\pgfqpoint{4.749180in}{3.850710in}}%
\pgfpathlineto{\pgfqpoint{4.749180in}{3.455818in}}%
\pgfpathlineto{\pgfqpoint{4.781483in}{3.455818in}}%
\pgfpathlineto{\pgfqpoint{4.781483in}{3.659013in}}%
\pgfpathlineto{\pgfqpoint{4.813786in}{3.659013in}}%
\pgfpathlineto{\pgfqpoint{4.813786in}{3.623123in}}%
\pgfpathlineto{\pgfqpoint{4.846090in}{3.623123in}}%
\pgfpathlineto{\pgfqpoint{4.846090in}{3.944738in}}%
\pgfpathlineto{\pgfqpoint{4.878393in}{3.944738in}}%
\pgfpathlineto{\pgfqpoint{4.878393in}{3.344256in}}%
\pgfpathlineto{\pgfqpoint{4.910696in}{3.344256in}}%
\pgfpathlineto{\pgfqpoint{4.910696in}{3.807140in}}%
\pgfpathlineto{\pgfqpoint{4.942999in}{3.807140in}}%
\pgfpathlineto{\pgfqpoint{4.942999in}{3.881187in}}%
\pgfpathlineto{\pgfqpoint{4.975303in}{3.881187in}}%
\pgfpathlineto{\pgfqpoint{4.975303in}{3.158241in}}%
\pgfpathlineto{\pgfqpoint{4.991454in}{3.158241in}}%
\pgfpathlineto{\pgfqpoint{4.991454in}{3.158241in}}%
\pgfusepath{stroke}%
\end{pgfscope}%
\begin{pgfscope}%
\pgfpathrectangle{\pgfqpoint{0.725000in}{3.113478in}}{\pgfqpoint{4.495000in}{0.984783in}}%
\pgfusepath{clip}%
\pgfsetbuttcap%
\pgfsetroundjoin%
\definecolor{currentfill}{rgb}{1.000000,0.498039,0.054902}%
\pgfsetfillcolor{currentfill}%
\pgfsetlinewidth{1.003750pt}%
\definecolor{currentstroke}{rgb}{1.000000,0.498039,0.054902}%
\pgfsetstrokecolor{currentstroke}%
\pgfsetdash{}{0pt}%
\pgfsys@defobject{currentmarker}{\pgfqpoint{-0.041667in}{-0.041667in}}{\pgfqpoint{0.041667in}{0.041667in}}{%
\pgfpathmoveto{\pgfqpoint{0.000000in}{-0.041667in}}%
\pgfpathcurveto{\pgfqpoint{0.011050in}{-0.041667in}}{\pgfqpoint{0.021649in}{-0.037276in}}{\pgfqpoint{0.029463in}{-0.029463in}}%
\pgfpathcurveto{\pgfqpoint{0.037276in}{-0.021649in}}{\pgfqpoint{0.041667in}{-0.011050in}}{\pgfqpoint{0.041667in}{0.000000in}}%
\pgfpathcurveto{\pgfqpoint{0.041667in}{0.011050in}}{\pgfqpoint{0.037276in}{0.021649in}}{\pgfqpoint{0.029463in}{0.029463in}}%
\pgfpathcurveto{\pgfqpoint{0.021649in}{0.037276in}}{\pgfqpoint{0.011050in}{0.041667in}}{\pgfqpoint{0.000000in}{0.041667in}}%
\pgfpathcurveto{\pgfqpoint{-0.011050in}{0.041667in}}{\pgfqpoint{-0.021649in}{0.037276in}}{\pgfqpoint{-0.029463in}{0.029463in}}%
\pgfpathcurveto{\pgfqpoint{-0.037276in}{0.021649in}}{\pgfqpoint{-0.041667in}{0.011050in}}{\pgfqpoint{-0.041667in}{0.000000in}}%
\pgfpathcurveto{\pgfqpoint{-0.041667in}{-0.011050in}}{\pgfqpoint{-0.037276in}{-0.021649in}}{\pgfqpoint{-0.029463in}{-0.029463in}}%
\pgfpathcurveto{\pgfqpoint{-0.021649in}{-0.037276in}}{\pgfqpoint{-0.011050in}{-0.041667in}}{\pgfqpoint{0.000000in}{-0.041667in}}%
\pgfpathclose%
\pgfusepath{stroke,fill}%
}%
\begin{pgfscope}%
\pgfsys@transformshift{0.953546in}{3.497451in}%
\pgfsys@useobject{currentmarker}{}%
\end{pgfscope}%
\begin{pgfscope}%
\pgfsys@transformshift{0.985849in}{3.821726in}%
\pgfsys@useobject{currentmarker}{}%
\end{pgfscope}%
\begin{pgfscope}%
\pgfsys@transformshift{1.018152in}{3.726488in}%
\pgfsys@useobject{currentmarker}{}%
\end{pgfscope}%
\begin{pgfscope}%
\pgfsys@transformshift{1.050455in}{3.413339in}%
\pgfsys@useobject{currentmarker}{}%
\end{pgfscope}%
\begin{pgfscope}%
\pgfsys@transformshift{1.082759in}{3.583776in}%
\pgfsys@useobject{currentmarker}{}%
\end{pgfscope}%
\begin{pgfscope}%
\pgfsys@transformshift{1.115062in}{3.886176in}%
\pgfsys@useobject{currentmarker}{}%
\end{pgfscope}%
\begin{pgfscope}%
\pgfsys@transformshift{1.147365in}{3.497142in}%
\pgfsys@useobject{currentmarker}{}%
\end{pgfscope}%
\begin{pgfscope}%
\pgfsys@transformshift{1.179669in}{3.637638in}%
\pgfsys@useobject{currentmarker}{}%
\end{pgfscope}%
\begin{pgfscope}%
\pgfsys@transformshift{1.211972in}{3.723113in}%
\pgfsys@useobject{currentmarker}{}%
\end{pgfscope}%
\begin{pgfscope}%
\pgfsys@transformshift{1.244275in}{3.386125in}%
\pgfsys@useobject{currentmarker}{}%
\end{pgfscope}%
\begin{pgfscope}%
\pgfsys@transformshift{1.276578in}{4.021011in}%
\pgfsys@useobject{currentmarker}{}%
\end{pgfscope}%
\begin{pgfscope}%
\pgfsys@transformshift{1.308882in}{3.317060in}%
\pgfsys@useobject{currentmarker}{}%
\end{pgfscope}%
\begin{pgfscope}%
\pgfsys@transformshift{1.341185in}{3.769406in}%
\pgfsys@useobject{currentmarker}{}%
\end{pgfscope}%
\begin{pgfscope}%
\pgfsys@transformshift{1.373488in}{3.739988in}%
\pgfsys@useobject{currentmarker}{}%
\end{pgfscope}%
\begin{pgfscope}%
\pgfsys@transformshift{1.405791in}{3.450534in}%
\pgfsys@useobject{currentmarker}{}%
\end{pgfscope}%
\begin{pgfscope}%
\pgfsys@transformshift{1.438095in}{3.607760in}%
\pgfsys@useobject{currentmarker}{}%
\end{pgfscope}%
\begin{pgfscope}%
\pgfsys@transformshift{1.470398in}{3.760485in}%
\pgfsys@useobject{currentmarker}{}%
\end{pgfscope}%
\begin{pgfscope}%
\pgfsys@transformshift{1.502701in}{3.832746in}%
\pgfsys@useobject{currentmarker}{}%
\end{pgfscope}%
\begin{pgfscope}%
\pgfsys@transformshift{1.535004in}{3.165277in}%
\pgfsys@useobject{currentmarker}{}%
\end{pgfscope}%
\begin{pgfscope}%
\pgfsys@transformshift{1.567308in}{4.053498in}%
\pgfsys@useobject{currentmarker}{}%
\end{pgfscope}%
\begin{pgfscope}%
\pgfsys@transformshift{1.599611in}{3.559683in}%
\pgfsys@useobject{currentmarker}{}%
\end{pgfscope}%
\begin{pgfscope}%
\pgfsys@transformshift{1.631914in}{3.508427in}%
\pgfsys@useobject{currentmarker}{}%
\end{pgfscope}%
\begin{pgfscope}%
\pgfsys@transformshift{1.664218in}{3.682695in}%
\pgfsys@useobject{currentmarker}{}%
\end{pgfscope}%
\begin{pgfscope}%
\pgfsys@transformshift{1.696521in}{3.732089in}%
\pgfsys@useobject{currentmarker}{}%
\end{pgfscope}%
\begin{pgfscope}%
\pgfsys@transformshift{1.728824in}{3.510638in}%
\pgfsys@useobject{currentmarker}{}%
\end{pgfscope}%
\begin{pgfscope}%
\pgfsys@transformshift{1.761127in}{3.921865in}%
\pgfsys@useobject{currentmarker}{}%
\end{pgfscope}%
\begin{pgfscope}%
\pgfsys@transformshift{1.793431in}{3.309988in}%
\pgfsys@useobject{currentmarker}{}%
\end{pgfscope}%
\begin{pgfscope}%
\pgfsys@transformshift{1.825734in}{3.555237in}%
\pgfsys@useobject{currentmarker}{}%
\end{pgfscope}%
\begin{pgfscope}%
\pgfsys@transformshift{1.858037in}{3.981901in}%
\pgfsys@useobject{currentmarker}{}%
\end{pgfscope}%
\begin{pgfscope}%
\pgfsys@transformshift{1.890340in}{3.531459in}%
\pgfsys@useobject{currentmarker}{}%
\end{pgfscope}%
\begin{pgfscope}%
\pgfsys@transformshift{1.922644in}{3.594048in}%
\pgfsys@useobject{currentmarker}{}%
\end{pgfscope}%
\begin{pgfscope}%
\pgfsys@transformshift{1.954947in}{3.752549in}%
\pgfsys@useobject{currentmarker}{}%
\end{pgfscope}%
\begin{pgfscope}%
\pgfsys@transformshift{1.987250in}{3.504211in}%
\pgfsys@useobject{currentmarker}{}%
\end{pgfscope}%
\begin{pgfscope}%
\pgfsys@transformshift{2.019554in}{3.797959in}%
\pgfsys@useobject{currentmarker}{}%
\end{pgfscope}%
\begin{pgfscope}%
\pgfsys@transformshift{2.051857in}{3.219931in}%
\pgfsys@useobject{currentmarker}{}%
\end{pgfscope}%
\begin{pgfscope}%
\pgfsys@transformshift{2.084160in}{4.041327in}%
\pgfsys@useobject{currentmarker}{}%
\end{pgfscope}%
\begin{pgfscope}%
\pgfsys@transformshift{2.116463in}{3.601791in}%
\pgfsys@useobject{currentmarker}{}%
\end{pgfscope}%
\begin{pgfscope}%
\pgfsys@transformshift{2.148767in}{3.533489in}%
\pgfsys@useobject{currentmarker}{}%
\end{pgfscope}%
\begin{pgfscope}%
\pgfsys@transformshift{2.181070in}{3.755890in}%
\pgfsys@useobject{currentmarker}{}%
\end{pgfscope}%
\begin{pgfscope}%
\pgfsys@transformshift{2.213373in}{3.741098in}%
\pgfsys@useobject{currentmarker}{}%
\end{pgfscope}%
\begin{pgfscope}%
\pgfsys@transformshift{2.245676in}{3.422281in}%
\pgfsys@useobject{currentmarker}{}%
\end{pgfscope}%
\begin{pgfscope}%
\pgfsys@transformshift{2.277980in}{3.703579in}%
\pgfsys@useobject{currentmarker}{}%
\end{pgfscope}%
\begin{pgfscope}%
\pgfsys@transformshift{2.310283in}{3.822806in}%
\pgfsys@useobject{currentmarker}{}%
\end{pgfscope}%
\begin{pgfscope}%
\pgfsys@transformshift{2.342586in}{3.218071in}%
\pgfsys@useobject{currentmarker}{}%
\end{pgfscope}%
\begin{pgfscope}%
\pgfsys@transformshift{2.374890in}{3.894914in}%
\pgfsys@useobject{currentmarker}{}%
\end{pgfscope}%
\begin{pgfscope}%
\pgfsys@transformshift{2.407193in}{3.863138in}%
\pgfsys@useobject{currentmarker}{}%
\end{pgfscope}%
\begin{pgfscope}%
\pgfsys@transformshift{2.439496in}{3.430302in}%
\pgfsys@useobject{currentmarker}{}%
\end{pgfscope}%
\begin{pgfscope}%
\pgfsys@transformshift{2.471799in}{3.606206in}%
\pgfsys@useobject{currentmarker}{}%
\end{pgfscope}%
\begin{pgfscope}%
\pgfsys@transformshift{2.504103in}{3.883222in}%
\pgfsys@useobject{currentmarker}{}%
\end{pgfscope}%
\begin{pgfscope}%
\pgfsys@transformshift{2.536406in}{3.290601in}%
\pgfsys@useobject{currentmarker}{}%
\end{pgfscope}%
\begin{pgfscope}%
\pgfsys@transformshift{2.568709in}{3.870520in}%
\pgfsys@useobject{currentmarker}{}%
\end{pgfscope}%
\begin{pgfscope}%
\pgfsys@transformshift{2.601012in}{3.579428in}%
\pgfsys@useobject{currentmarker}{}%
\end{pgfscope}%
\begin{pgfscope}%
\pgfsys@transformshift{2.633316in}{3.491942in}%
\pgfsys@useobject{currentmarker}{}%
\end{pgfscope}%
\begin{pgfscope}%
\pgfsys@transformshift{2.665619in}{4.052851in}%
\pgfsys@useobject{currentmarker}{}%
\end{pgfscope}%
\begin{pgfscope}%
\pgfsys@transformshift{2.697922in}{3.484031in}%
\pgfsys@useobject{currentmarker}{}%
\end{pgfscope}%
\begin{pgfscope}%
\pgfsys@transformshift{2.730225in}{3.369702in}%
\pgfsys@useobject{currentmarker}{}%
\end{pgfscope}%
\begin{pgfscope}%
\pgfsys@transformshift{2.762529in}{3.879187in}%
\pgfsys@useobject{currentmarker}{}%
\end{pgfscope}%
\begin{pgfscope}%
\pgfsys@transformshift{2.794832in}{3.462543in}%
\pgfsys@useobject{currentmarker}{}%
\end{pgfscope}%
\begin{pgfscope}%
\pgfsys@transformshift{2.827135in}{3.856106in}%
\pgfsys@useobject{currentmarker}{}%
\end{pgfscope}%
\begin{pgfscope}%
\pgfsys@transformshift{2.859439in}{3.566668in}%
\pgfsys@useobject{currentmarker}{}%
\end{pgfscope}%
\begin{pgfscope}%
\pgfsys@transformshift{2.891742in}{3.834191in}%
\pgfsys@useobject{currentmarker}{}%
\end{pgfscope}%
\begin{pgfscope}%
\pgfsys@transformshift{2.924045in}{3.521221in}%
\pgfsys@useobject{currentmarker}{}%
\end{pgfscope}%
\begin{pgfscope}%
\pgfsys@transformshift{2.956348in}{3.600953in}%
\pgfsys@useobject{currentmarker}{}%
\end{pgfscope}%
\begin{pgfscope}%
\pgfsys@transformshift{2.988652in}{3.597163in}%
\pgfsys@useobject{currentmarker}{}%
\end{pgfscope}%
\begin{pgfscope}%
\pgfsys@transformshift{3.020955in}{3.560384in}%
\pgfsys@useobject{currentmarker}{}%
\end{pgfscope}%
\begin{pgfscope}%
\pgfsys@transformshift{3.053258in}{3.931072in}%
\pgfsys@useobject{currentmarker}{}%
\end{pgfscope}%
\begin{pgfscope}%
\pgfsys@transformshift{3.085561in}{3.498370in}%
\pgfsys@useobject{currentmarker}{}%
\end{pgfscope}%
\begin{pgfscope}%
\pgfsys@transformshift{3.117865in}{3.698850in}%
\pgfsys@useobject{currentmarker}{}%
\end{pgfscope}%
\begin{pgfscope}%
\pgfsys@transformshift{3.150168in}{3.662590in}%
\pgfsys@useobject{currentmarker}{}%
\end{pgfscope}%
\begin{pgfscope}%
\pgfsys@transformshift{3.182471in}{3.442281in}%
\pgfsys@useobject{currentmarker}{}%
\end{pgfscope}%
\begin{pgfscope}%
\pgfsys@transformshift{3.214775in}{3.847972in}%
\pgfsys@useobject{currentmarker}{}%
\end{pgfscope}%
\begin{pgfscope}%
\pgfsys@transformshift{3.247078in}{3.558251in}%
\pgfsys@useobject{currentmarker}{}%
\end{pgfscope}%
\begin{pgfscope}%
\pgfsys@transformshift{3.279381in}{3.829281in}%
\pgfsys@useobject{currentmarker}{}%
\end{pgfscope}%
\begin{pgfscope}%
\pgfsys@transformshift{3.311684in}{3.600378in}%
\pgfsys@useobject{currentmarker}{}%
\end{pgfscope}%
\begin{pgfscope}%
\pgfsys@transformshift{3.343988in}{3.542240in}%
\pgfsys@useobject{currentmarker}{}%
\end{pgfscope}%
\begin{pgfscope}%
\pgfsys@transformshift{3.376291in}{3.763125in}%
\pgfsys@useobject{currentmarker}{}%
\end{pgfscope}%
\begin{pgfscope}%
\pgfsys@transformshift{3.408594in}{3.386331in}%
\pgfsys@useobject{currentmarker}{}%
\end{pgfscope}%
\begin{pgfscope}%
\pgfsys@transformshift{3.440897in}{3.934841in}%
\pgfsys@useobject{currentmarker}{}%
\end{pgfscope}%
\begin{pgfscope}%
\pgfsys@transformshift{3.473201in}{3.691964in}%
\pgfsys@useobject{currentmarker}{}%
\end{pgfscope}%
\begin{pgfscope}%
\pgfsys@transformshift{3.505504in}{3.414636in}%
\pgfsys@useobject{currentmarker}{}%
\end{pgfscope}%
\begin{pgfscope}%
\pgfsys@transformshift{3.537807in}{3.878800in}%
\pgfsys@useobject{currentmarker}{}%
\end{pgfscope}%
\begin{pgfscope}%
\pgfsys@transformshift{3.570110in}{3.318674in}%
\pgfsys@useobject{currentmarker}{}%
\end{pgfscope}%
\begin{pgfscope}%
\pgfsys@transformshift{3.602414in}{3.946112in}%
\pgfsys@useobject{currentmarker}{}%
\end{pgfscope}%
\begin{pgfscope}%
\pgfsys@transformshift{3.634717in}{3.579579in}%
\pgfsys@useobject{currentmarker}{}%
\end{pgfscope}%
\begin{pgfscope}%
\pgfsys@transformshift{3.667020in}{3.666419in}%
\pgfsys@useobject{currentmarker}{}%
\end{pgfscope}%
\begin{pgfscope}%
\pgfsys@transformshift{3.699324in}{3.636842in}%
\pgfsys@useobject{currentmarker}{}%
\end{pgfscope}%
\begin{pgfscope}%
\pgfsys@transformshift{3.731627in}{3.523806in}%
\pgfsys@useobject{currentmarker}{}%
\end{pgfscope}%
\begin{pgfscope}%
\pgfsys@transformshift{3.763930in}{3.897870in}%
\pgfsys@useobject{currentmarker}{}%
\end{pgfscope}%
\begin{pgfscope}%
\pgfsys@transformshift{3.796233in}{3.502035in}%
\pgfsys@useobject{currentmarker}{}%
\end{pgfscope}%
\begin{pgfscope}%
\pgfsys@transformshift{3.828537in}{3.649159in}%
\pgfsys@useobject{currentmarker}{}%
\end{pgfscope}%
\begin{pgfscope}%
\pgfsys@transformshift{3.860840in}{3.820018in}%
\pgfsys@useobject{currentmarker}{}%
\end{pgfscope}%
\begin{pgfscope}%
\pgfsys@transformshift{3.893143in}{3.428012in}%
\pgfsys@useobject{currentmarker}{}%
\end{pgfscope}%
\begin{pgfscope}%
\pgfsys@transformshift{3.925446in}{3.632017in}%
\pgfsys@useobject{currentmarker}{}%
\end{pgfscope}%
\begin{pgfscope}%
\pgfsys@transformshift{3.957750in}{3.759332in}%
\pgfsys@useobject{currentmarker}{}%
\end{pgfscope}%
\begin{pgfscope}%
\pgfsys@transformshift{3.990053in}{3.745256in}%
\pgfsys@useobject{currentmarker}{}%
\end{pgfscope}%
\begin{pgfscope}%
\pgfsys@transformshift{4.022356in}{3.524806in}%
\pgfsys@useobject{currentmarker}{}%
\end{pgfscope}%
\begin{pgfscope}%
\pgfsys@transformshift{4.054660in}{3.790180in}%
\pgfsys@useobject{currentmarker}{}%
\end{pgfscope}%
\begin{pgfscope}%
\pgfsys@transformshift{4.086963in}{3.523005in}%
\pgfsys@useobject{currentmarker}{}%
\end{pgfscope}%
\begin{pgfscope}%
\pgfsys@transformshift{4.119266in}{3.737512in}%
\pgfsys@useobject{currentmarker}{}%
\end{pgfscope}%
\begin{pgfscope}%
\pgfsys@transformshift{4.151569in}{3.528121in}%
\pgfsys@useobject{currentmarker}{}%
\end{pgfscope}%
\begin{pgfscope}%
\pgfsys@transformshift{4.183873in}{3.855831in}%
\pgfsys@useobject{currentmarker}{}%
\end{pgfscope}%
\begin{pgfscope}%
\pgfsys@transformshift{4.216176in}{3.327957in}%
\pgfsys@useobject{currentmarker}{}%
\end{pgfscope}%
\begin{pgfscope}%
\pgfsys@transformshift{4.248479in}{3.865893in}%
\pgfsys@useobject{currentmarker}{}%
\end{pgfscope}%
\begin{pgfscope}%
\pgfsys@transformshift{4.280782in}{3.787944in}%
\pgfsys@useobject{currentmarker}{}%
\end{pgfscope}%
\begin{pgfscope}%
\pgfsys@transformshift{4.313086in}{3.402300in}%
\pgfsys@useobject{currentmarker}{}%
\end{pgfscope}%
\begin{pgfscope}%
\pgfsys@transformshift{4.345389in}{3.910449in}%
\pgfsys@useobject{currentmarker}{}%
\end{pgfscope}%
\begin{pgfscope}%
\pgfsys@transformshift{4.377692in}{3.536320in}%
\pgfsys@useobject{currentmarker}{}%
\end{pgfscope}%
\begin{pgfscope}%
\pgfsys@transformshift{4.409996in}{3.545209in}%
\pgfsys@useobject{currentmarker}{}%
\end{pgfscope}%
\begin{pgfscope}%
\pgfsys@transformshift{4.442299in}{3.785053in}%
\pgfsys@useobject{currentmarker}{}%
\end{pgfscope}%
\begin{pgfscope}%
\pgfsys@transformshift{4.474602in}{3.639536in}%
\pgfsys@useobject{currentmarker}{}%
\end{pgfscope}%
\begin{pgfscope}%
\pgfsys@transformshift{4.506905in}{3.620111in}%
\pgfsys@useobject{currentmarker}{}%
\end{pgfscope}%
\begin{pgfscope}%
\pgfsys@transformshift{4.539209in}{3.615885in}%
\pgfsys@useobject{currentmarker}{}%
\end{pgfscope}%
\begin{pgfscope}%
\pgfsys@transformshift{4.571512in}{3.901797in}%
\pgfsys@useobject{currentmarker}{}%
\end{pgfscope}%
\begin{pgfscope}%
\pgfsys@transformshift{4.603815in}{3.405181in}%
\pgfsys@useobject{currentmarker}{}%
\end{pgfscope}%
\begin{pgfscope}%
\pgfsys@transformshift{4.636118in}{3.749307in}%
\pgfsys@useobject{currentmarker}{}%
\end{pgfscope}%
\begin{pgfscope}%
\pgfsys@transformshift{4.668422in}{3.738144in}%
\pgfsys@useobject{currentmarker}{}%
\end{pgfscope}%
\begin{pgfscope}%
\pgfsys@transformshift{4.700725in}{3.505066in}%
\pgfsys@useobject{currentmarker}{}%
\end{pgfscope}%
\begin{pgfscope}%
\pgfsys@transformshift{4.733028in}{3.850710in}%
\pgfsys@useobject{currentmarker}{}%
\end{pgfscope}%
\begin{pgfscope}%
\pgfsys@transformshift{4.765331in}{3.455818in}%
\pgfsys@useobject{currentmarker}{}%
\end{pgfscope}%
\begin{pgfscope}%
\pgfsys@transformshift{4.797635in}{3.659013in}%
\pgfsys@useobject{currentmarker}{}%
\end{pgfscope}%
\begin{pgfscope}%
\pgfsys@transformshift{4.829938in}{3.623123in}%
\pgfsys@useobject{currentmarker}{}%
\end{pgfscope}%
\begin{pgfscope}%
\pgfsys@transformshift{4.862241in}{3.944738in}%
\pgfsys@useobject{currentmarker}{}%
\end{pgfscope}%
\begin{pgfscope}%
\pgfsys@transformshift{4.894545in}{3.344256in}%
\pgfsys@useobject{currentmarker}{}%
\end{pgfscope}%
\begin{pgfscope}%
\pgfsys@transformshift{4.926848in}{3.807140in}%
\pgfsys@useobject{currentmarker}{}%
\end{pgfscope}%
\begin{pgfscope}%
\pgfsys@transformshift{4.959151in}{3.881187in}%
\pgfsys@useobject{currentmarker}{}%
\end{pgfscope}%
\begin{pgfscope}%
\pgfsys@transformshift{4.991454in}{3.158241in}%
\pgfsys@useobject{currentmarker}{}%
\end{pgfscope}%
\end{pgfscope}%
\begin{pgfscope}%
\pgfsetrectcap%
\pgfsetmiterjoin%
\pgfsetlinewidth{0.803000pt}%
\definecolor{currentstroke}{rgb}{0.000000,0.000000,0.000000}%
\pgfsetstrokecolor{currentstroke}%
\pgfsetdash{}{0pt}%
\pgfpathmoveto{\pgfqpoint{0.725000in}{3.113478in}}%
\pgfpathlineto{\pgfqpoint{0.725000in}{4.098261in}}%
\pgfusepath{stroke}%
\end{pgfscope}%
\begin{pgfscope}%
\pgfsetrectcap%
\pgfsetmiterjoin%
\pgfsetlinewidth{0.803000pt}%
\definecolor{currentstroke}{rgb}{0.000000,0.000000,0.000000}%
\pgfsetstrokecolor{currentstroke}%
\pgfsetdash{}{0pt}%
\pgfpathmoveto{\pgfqpoint{5.220000in}{3.113478in}}%
\pgfpathlineto{\pgfqpoint{5.220000in}{4.098261in}}%
\pgfusepath{stroke}%
\end{pgfscope}%
\begin{pgfscope}%
\pgfsetrectcap%
\pgfsetmiterjoin%
\pgfsetlinewidth{0.803000pt}%
\definecolor{currentstroke}{rgb}{0.000000,0.000000,0.000000}%
\pgfsetstrokecolor{currentstroke}%
\pgfsetdash{}{0pt}%
\pgfpathmoveto{\pgfqpoint{0.725000in}{3.113478in}}%
\pgfpathlineto{\pgfqpoint{5.220000in}{3.113478in}}%
\pgfusepath{stroke}%
\end{pgfscope}%
\begin{pgfscope}%
\pgfsetrectcap%
\pgfsetmiterjoin%
\pgfsetlinewidth{0.803000pt}%
\definecolor{currentstroke}{rgb}{0.000000,0.000000,0.000000}%
\pgfsetstrokecolor{currentstroke}%
\pgfsetdash{}{0pt}%
\pgfpathmoveto{\pgfqpoint{0.725000in}{4.098261in}}%
\pgfpathlineto{\pgfqpoint{5.220000in}{4.098261in}}%
\pgfusepath{stroke}%
\end{pgfscope}%
\begin{pgfscope}%
\pgfsetbuttcap%
\pgfsetmiterjoin%
\definecolor{currentfill}{rgb}{1.000000,1.000000,1.000000}%
\pgfsetfillcolor{currentfill}%
\pgfsetlinewidth{0.000000pt}%
\definecolor{currentstroke}{rgb}{0.000000,0.000000,0.000000}%
\pgfsetstrokecolor{currentstroke}%
\pgfsetstrokeopacity{0.000000}%
\pgfsetdash{}{0pt}%
\pgfpathmoveto{\pgfqpoint{0.725000in}{1.931739in}}%
\pgfpathlineto{\pgfqpoint{5.220000in}{1.931739in}}%
\pgfpathlineto{\pgfqpoint{5.220000in}{2.916522in}}%
\pgfpathlineto{\pgfqpoint{0.725000in}{2.916522in}}%
\pgfpathclose%
\pgfusepath{fill}%
\end{pgfscope}%
\begin{pgfscope}%
\pgfsetbuttcap%
\pgfsetroundjoin%
\definecolor{currentfill}{rgb}{0.000000,0.000000,0.000000}%
\pgfsetfillcolor{currentfill}%
\pgfsetlinewidth{0.803000pt}%
\definecolor{currentstroke}{rgb}{0.000000,0.000000,0.000000}%
\pgfsetstrokecolor{currentstroke}%
\pgfsetdash{}{0pt}%
\pgfsys@defobject{currentmarker}{\pgfqpoint{0.000000in}{-0.048611in}}{\pgfqpoint{0.000000in}{0.000000in}}{%
\pgfpathmoveto{\pgfqpoint{0.000000in}{0.000000in}}%
\pgfpathlineto{\pgfqpoint{0.000000in}{-0.048611in}}%
\pgfusepath{stroke,fill}%
}%
\begin{pgfscope}%
\pgfsys@transformshift{0.913167in}{1.931739in}%
\pgfsys@useobject{currentmarker}{}%
\end{pgfscope}%
\end{pgfscope}%
\begin{pgfscope}%
\pgfsetbuttcap%
\pgfsetroundjoin%
\definecolor{currentfill}{rgb}{0.000000,0.000000,0.000000}%
\pgfsetfillcolor{currentfill}%
\pgfsetlinewidth{0.803000pt}%
\definecolor{currentstroke}{rgb}{0.000000,0.000000,0.000000}%
\pgfsetstrokecolor{currentstroke}%
\pgfsetdash{}{0pt}%
\pgfsys@defobject{currentmarker}{\pgfqpoint{0.000000in}{-0.048611in}}{\pgfqpoint{0.000000in}{0.000000in}}{%
\pgfpathmoveto{\pgfqpoint{0.000000in}{0.000000in}}%
\pgfpathlineto{\pgfqpoint{0.000000in}{-0.048611in}}%
\pgfusepath{stroke,fill}%
}%
\begin{pgfscope}%
\pgfsys@transformshift{1.720748in}{1.931739in}%
\pgfsys@useobject{currentmarker}{}%
\end{pgfscope}%
\end{pgfscope}%
\begin{pgfscope}%
\pgfsetbuttcap%
\pgfsetroundjoin%
\definecolor{currentfill}{rgb}{0.000000,0.000000,0.000000}%
\pgfsetfillcolor{currentfill}%
\pgfsetlinewidth{0.803000pt}%
\definecolor{currentstroke}{rgb}{0.000000,0.000000,0.000000}%
\pgfsetstrokecolor{currentstroke}%
\pgfsetdash{}{0pt}%
\pgfsys@defobject{currentmarker}{\pgfqpoint{0.000000in}{-0.048611in}}{\pgfqpoint{0.000000in}{0.000000in}}{%
\pgfpathmoveto{\pgfqpoint{0.000000in}{0.000000in}}%
\pgfpathlineto{\pgfqpoint{0.000000in}{-0.048611in}}%
\pgfusepath{stroke,fill}%
}%
\begin{pgfscope}%
\pgfsys@transformshift{2.528330in}{1.931739in}%
\pgfsys@useobject{currentmarker}{}%
\end{pgfscope}%
\end{pgfscope}%
\begin{pgfscope}%
\pgfsetbuttcap%
\pgfsetroundjoin%
\definecolor{currentfill}{rgb}{0.000000,0.000000,0.000000}%
\pgfsetfillcolor{currentfill}%
\pgfsetlinewidth{0.803000pt}%
\definecolor{currentstroke}{rgb}{0.000000,0.000000,0.000000}%
\pgfsetstrokecolor{currentstroke}%
\pgfsetdash{}{0pt}%
\pgfsys@defobject{currentmarker}{\pgfqpoint{0.000000in}{-0.048611in}}{\pgfqpoint{0.000000in}{0.000000in}}{%
\pgfpathmoveto{\pgfqpoint{0.000000in}{0.000000in}}%
\pgfpathlineto{\pgfqpoint{0.000000in}{-0.048611in}}%
\pgfusepath{stroke,fill}%
}%
\begin{pgfscope}%
\pgfsys@transformshift{3.335912in}{1.931739in}%
\pgfsys@useobject{currentmarker}{}%
\end{pgfscope}%
\end{pgfscope}%
\begin{pgfscope}%
\pgfsetbuttcap%
\pgfsetroundjoin%
\definecolor{currentfill}{rgb}{0.000000,0.000000,0.000000}%
\pgfsetfillcolor{currentfill}%
\pgfsetlinewidth{0.803000pt}%
\definecolor{currentstroke}{rgb}{0.000000,0.000000,0.000000}%
\pgfsetstrokecolor{currentstroke}%
\pgfsetdash{}{0pt}%
\pgfsys@defobject{currentmarker}{\pgfqpoint{0.000000in}{-0.048611in}}{\pgfqpoint{0.000000in}{0.000000in}}{%
\pgfpathmoveto{\pgfqpoint{0.000000in}{0.000000in}}%
\pgfpathlineto{\pgfqpoint{0.000000in}{-0.048611in}}%
\pgfusepath{stroke,fill}%
}%
\begin{pgfscope}%
\pgfsys@transformshift{4.143494in}{1.931739in}%
\pgfsys@useobject{currentmarker}{}%
\end{pgfscope}%
\end{pgfscope}%
\begin{pgfscope}%
\pgfsetbuttcap%
\pgfsetroundjoin%
\definecolor{currentfill}{rgb}{0.000000,0.000000,0.000000}%
\pgfsetfillcolor{currentfill}%
\pgfsetlinewidth{0.803000pt}%
\definecolor{currentstroke}{rgb}{0.000000,0.000000,0.000000}%
\pgfsetstrokecolor{currentstroke}%
\pgfsetdash{}{0pt}%
\pgfsys@defobject{currentmarker}{\pgfqpoint{0.000000in}{-0.048611in}}{\pgfqpoint{0.000000in}{0.000000in}}{%
\pgfpathmoveto{\pgfqpoint{0.000000in}{0.000000in}}%
\pgfpathlineto{\pgfqpoint{0.000000in}{-0.048611in}}%
\pgfusepath{stroke,fill}%
}%
\begin{pgfscope}%
\pgfsys@transformshift{4.951075in}{1.931739in}%
\pgfsys@useobject{currentmarker}{}%
\end{pgfscope}%
\end{pgfscope}%
\begin{pgfscope}%
\pgfsetbuttcap%
\pgfsetroundjoin%
\definecolor{currentfill}{rgb}{0.000000,0.000000,0.000000}%
\pgfsetfillcolor{currentfill}%
\pgfsetlinewidth{0.803000pt}%
\definecolor{currentstroke}{rgb}{0.000000,0.000000,0.000000}%
\pgfsetstrokecolor{currentstroke}%
\pgfsetdash{}{0pt}%
\pgfsys@defobject{currentmarker}{\pgfqpoint{-0.048611in}{0.000000in}}{\pgfqpoint{0.000000in}{0.000000in}}{%
\pgfpathmoveto{\pgfqpoint{0.000000in}{0.000000in}}%
\pgfpathlineto{\pgfqpoint{-0.048611in}{0.000000in}}%
\pgfusepath{stroke,fill}%
}%
\begin{pgfscope}%
\pgfsys@transformshift{0.725000in}{2.108396in}%
\pgfsys@useobject{currentmarker}{}%
\end{pgfscope}%
\end{pgfscope}%
\begin{pgfscope}%
\definecolor{textcolor}{rgb}{0.000000,0.000000,0.000000}%
\pgfsetstrokecolor{textcolor}%
\pgfsetfillcolor{textcolor}%
\pgftext[x=0.326222in,y=2.069840in,left,base]{\color{textcolor}\rmfamily\fontsize{8.000000}{9.600000}\selectfont −0.02}%
\end{pgfscope}%
\begin{pgfscope}%
\pgfsetbuttcap%
\pgfsetroundjoin%
\definecolor{currentfill}{rgb}{0.000000,0.000000,0.000000}%
\pgfsetfillcolor{currentfill}%
\pgfsetlinewidth{0.803000pt}%
\definecolor{currentstroke}{rgb}{0.000000,0.000000,0.000000}%
\pgfsetstrokecolor{currentstroke}%
\pgfsetdash{}{0pt}%
\pgfsys@defobject{currentmarker}{\pgfqpoint{-0.048611in}{0.000000in}}{\pgfqpoint{0.000000in}{0.000000in}}{%
\pgfpathmoveto{\pgfqpoint{0.000000in}{0.000000in}}%
\pgfpathlineto{\pgfqpoint{-0.048611in}{0.000000in}}%
\pgfusepath{stroke,fill}%
}%
\begin{pgfscope}%
\pgfsys@transformshift{0.725000in}{2.381119in}%
\pgfsys@useobject{currentmarker}{}%
\end{pgfscope}%
\end{pgfscope}%
\begin{pgfscope}%
\definecolor{textcolor}{rgb}{0.000000,0.000000,0.000000}%
\pgfsetstrokecolor{textcolor}%
\pgfsetfillcolor{textcolor}%
\pgftext[x=0.418000in,y=2.342563in,left,base]{\color{textcolor}\rmfamily\fontsize{8.000000}{9.600000}\selectfont 0.00}%
\end{pgfscope}%
\begin{pgfscope}%
\pgfsetbuttcap%
\pgfsetroundjoin%
\definecolor{currentfill}{rgb}{0.000000,0.000000,0.000000}%
\pgfsetfillcolor{currentfill}%
\pgfsetlinewidth{0.803000pt}%
\definecolor{currentstroke}{rgb}{0.000000,0.000000,0.000000}%
\pgfsetstrokecolor{currentstroke}%
\pgfsetdash{}{0pt}%
\pgfsys@defobject{currentmarker}{\pgfqpoint{-0.048611in}{0.000000in}}{\pgfqpoint{0.000000in}{0.000000in}}{%
\pgfpathmoveto{\pgfqpoint{0.000000in}{0.000000in}}%
\pgfpathlineto{\pgfqpoint{-0.048611in}{0.000000in}}%
\pgfusepath{stroke,fill}%
}%
\begin{pgfscope}%
\pgfsys@transformshift{0.725000in}{2.653841in}%
\pgfsys@useobject{currentmarker}{}%
\end{pgfscope}%
\end{pgfscope}%
\begin{pgfscope}%
\definecolor{textcolor}{rgb}{0.000000,0.000000,0.000000}%
\pgfsetstrokecolor{textcolor}%
\pgfsetfillcolor{textcolor}%
\pgftext[x=0.418000in,y=2.615286in,left,base]{\color{textcolor}\rmfamily\fontsize{8.000000}{9.600000}\selectfont 0.02}%
\end{pgfscope}%
\begin{pgfscope}%
\pgfpathrectangle{\pgfqpoint{0.725000in}{1.931739in}}{\pgfqpoint{4.495000in}{0.984783in}}%
\pgfusepath{clip}%
\pgfsetrectcap%
\pgfsetroundjoin%
\pgfsetlinewidth{1.505625pt}%
\definecolor{currentstroke}{rgb}{1.000000,0.498039,0.054902}%
\pgfsetstrokecolor{currentstroke}%
\pgfsetdash{}{0pt}%
\pgfpathmoveto{\pgfqpoint{1.002001in}{2.766546in}}%
\pgfpathlineto{\pgfqpoint{1.034304in}{2.766546in}}%
\pgfpathlineto{\pgfqpoint{1.034304in}{1.980471in}}%
\pgfpathlineto{\pgfqpoint{1.098910in}{1.980471in}}%
\pgfpathlineto{\pgfqpoint{1.098910in}{2.614470in}}%
\pgfpathlineto{\pgfqpoint{1.163517in}{2.614470in}}%
\pgfpathlineto{\pgfqpoint{1.163517in}{2.279156in}}%
\pgfpathlineto{\pgfqpoint{1.228123in}{2.279156in}}%
\pgfpathlineto{\pgfqpoint{1.228123in}{2.523764in}}%
\pgfpathlineto{\pgfqpoint{1.292730in}{2.523764in}}%
\pgfpathlineto{\pgfqpoint{1.292730in}{2.324494in}}%
\pgfpathlineto{\pgfqpoint{1.357337in}{2.324494in}}%
\pgfpathlineto{\pgfqpoint{1.357337in}{2.265286in}}%
\pgfpathlineto{\pgfqpoint{1.421943in}{2.265286in}}%
\pgfpathlineto{\pgfqpoint{1.421943in}{2.628226in}}%
\pgfpathlineto{\pgfqpoint{1.486550in}{2.628226in}}%
\pgfpathlineto{\pgfqpoint{1.486550in}{2.149052in}}%
\pgfpathlineto{\pgfqpoint{1.551156in}{2.149052in}}%
\pgfpathlineto{\pgfqpoint{1.551156in}{2.731691in}}%
\pgfpathlineto{\pgfqpoint{1.615763in}{2.731691in}}%
\pgfpathlineto{\pgfqpoint{1.615763in}{2.196217in}}%
\pgfpathlineto{\pgfqpoint{1.680369in}{2.196217in}}%
\pgfpathlineto{\pgfqpoint{1.680369in}{2.582556in}}%
\pgfpathlineto{\pgfqpoint{1.744976in}{2.582556in}}%
\pgfpathlineto{\pgfqpoint{1.744976in}{2.110536in}}%
\pgfpathlineto{\pgfqpoint{1.809582in}{2.110536in}}%
\pgfpathlineto{\pgfqpoint{1.809582in}{2.681801in}}%
\pgfpathlineto{\pgfqpoint{1.874189in}{2.681801in}}%
\pgfpathlineto{\pgfqpoint{1.874189in}{2.399326in}}%
\pgfpathlineto{\pgfqpoint{1.938795in}{2.399326in}}%
\pgfpathlineto{\pgfqpoint{1.938795in}{2.441723in}}%
\pgfpathlineto{\pgfqpoint{2.003402in}{2.441723in}}%
\pgfpathlineto{\pgfqpoint{2.003402in}{1.976502in}}%
\pgfpathlineto{\pgfqpoint{2.068008in}{1.976502in}}%
\pgfpathlineto{\pgfqpoint{2.068008in}{2.871759in}}%
\pgfpathlineto{\pgfqpoint{2.132615in}{2.871759in}}%
\pgfpathlineto{\pgfqpoint{2.132615in}{2.481993in}}%
\pgfpathlineto{\pgfqpoint{2.197222in}{2.481993in}}%
\pgfpathlineto{\pgfqpoint{2.197222in}{2.266006in}}%
\pgfpathlineto{\pgfqpoint{2.261828in}{2.266006in}}%
\pgfpathlineto{\pgfqpoint{2.261828in}{2.489000in}}%
\pgfpathlineto{\pgfqpoint{2.326435in}{2.489000in}}%
\pgfpathlineto{\pgfqpoint{2.326435in}{2.328215in}}%
\pgfpathlineto{\pgfqpoint{2.391041in}{2.328215in}}%
\pgfpathlineto{\pgfqpoint{2.391041in}{2.622793in}}%
\pgfpathlineto{\pgfqpoint{2.455648in}{2.622793in}}%
\pgfpathlineto{\pgfqpoint{2.455648in}{2.463950in}}%
\pgfpathlineto{\pgfqpoint{2.520254in}{2.463950in}}%
\pgfpathlineto{\pgfqpoint{2.520254in}{2.263039in}}%
\pgfpathlineto{\pgfqpoint{2.584861in}{2.263039in}}%
\pgfpathlineto{\pgfqpoint{2.584861in}{2.421229in}}%
\pgfpathlineto{\pgfqpoint{2.649467in}{2.421229in}}%
\pgfpathlineto{\pgfqpoint{2.649467in}{2.697854in}}%
\pgfpathlineto{\pgfqpoint{2.714074in}{2.697854in}}%
\pgfpathlineto{\pgfqpoint{2.714074in}{2.054196in}}%
\pgfpathlineto{\pgfqpoint{2.778680in}{2.054196in}}%
\pgfpathlineto{\pgfqpoint{2.778680in}{2.639615in}}%
\pgfpathlineto{\pgfqpoint{2.843287in}{2.639615in}}%
\pgfpathlineto{\pgfqpoint{2.843287in}{2.786765in}}%
\pgfpathlineto{\pgfqpoint{2.907893in}{2.786765in}}%
\pgfpathlineto{\pgfqpoint{2.907893in}{2.140644in}}%
\pgfpathlineto{\pgfqpoint{2.972500in}{2.140644in}}%
\pgfpathlineto{\pgfqpoint{2.972500in}{2.311497in}}%
\pgfpathlineto{\pgfqpoint{3.037107in}{2.311497in}}%
\pgfpathlineto{\pgfqpoint{3.037107in}{2.760043in}}%
\pgfpathlineto{\pgfqpoint{3.101713in}{2.760043in}}%
\pgfpathlineto{\pgfqpoint{3.101713in}{2.328275in}}%
\pgfpathlineto{\pgfqpoint{3.166320in}{2.328275in}}%
\pgfpathlineto{\pgfqpoint{3.166320in}{2.383322in}}%
\pgfpathlineto{\pgfqpoint{3.230926in}{2.383322in}}%
\pgfpathlineto{\pgfqpoint{3.230926in}{2.812412in}}%
\pgfpathlineto{\pgfqpoint{3.295533in}{2.812412in}}%
\pgfpathlineto{\pgfqpoint{3.295533in}{2.324382in}}%
\pgfpathlineto{\pgfqpoint{3.360139in}{2.324382in}}%
\pgfpathlineto{\pgfqpoint{3.360139in}{2.246046in}}%
\pgfpathlineto{\pgfqpoint{3.424746in}{2.246046in}}%
\pgfpathlineto{\pgfqpoint{3.424746in}{2.870608in}}%
\pgfpathlineto{\pgfqpoint{3.489352in}{2.870608in}}%
\pgfpathlineto{\pgfqpoint{3.489352in}{2.217540in}}%
\pgfpathlineto{\pgfqpoint{3.553959in}{2.217540in}}%
\pgfpathlineto{\pgfqpoint{3.553959in}{2.516456in}}%
\pgfpathlineto{\pgfqpoint{3.618565in}{2.516456in}}%
\pgfpathlineto{\pgfqpoint{3.618565in}{2.572813in}}%
\pgfpathlineto{\pgfqpoint{3.683172in}{2.572813in}}%
\pgfpathlineto{\pgfqpoint{3.683172in}{2.345840in}}%
\pgfpathlineto{\pgfqpoint{3.747778in}{2.345840in}}%
\pgfpathlineto{\pgfqpoint{3.747778in}{2.615555in}}%
\pgfpathlineto{\pgfqpoint{3.812385in}{2.615555in}}%
\pgfpathlineto{\pgfqpoint{3.812385in}{2.586554in}}%
\pgfpathlineto{\pgfqpoint{3.876992in}{2.586554in}}%
\pgfpathlineto{\pgfqpoint{3.876992in}{2.108886in}}%
\pgfpathlineto{\pgfqpoint{3.941598in}{2.108886in}}%
\pgfpathlineto{\pgfqpoint{3.941598in}{2.861688in}}%
\pgfpathlineto{\pgfqpoint{4.006205in}{2.861688in}}%
\pgfpathlineto{\pgfqpoint{4.006205in}{2.483805in}}%
\pgfpathlineto{\pgfqpoint{4.070811in}{2.483805in}}%
\pgfpathlineto{\pgfqpoint{4.070811in}{2.390139in}}%
\pgfpathlineto{\pgfqpoint{4.135418in}{2.390139in}}%
\pgfpathlineto{\pgfqpoint{4.135418in}{2.487197in}}%
\pgfpathlineto{\pgfqpoint{4.200024in}{2.487197in}}%
\pgfpathlineto{\pgfqpoint{4.200024in}{2.493807in}}%
\pgfpathlineto{\pgfqpoint{4.264631in}{2.493807in}}%
\pgfpathlineto{\pgfqpoint{4.264631in}{2.594396in}}%
\pgfpathlineto{\pgfqpoint{4.329237in}{2.594396in}}%
\pgfpathlineto{\pgfqpoint{4.329237in}{2.545438in}}%
\pgfpathlineto{\pgfqpoint{4.393844in}{2.545438in}}%
\pgfpathlineto{\pgfqpoint{4.393844in}{2.445321in}}%
\pgfpathlineto{\pgfqpoint{4.458450in}{2.445321in}}%
\pgfpathlineto{\pgfqpoint{4.458450in}{2.456259in}}%
\pgfpathlineto{\pgfqpoint{4.523057in}{2.456259in}}%
\pgfpathlineto{\pgfqpoint{4.523057in}{2.785472in}}%
\pgfpathlineto{\pgfqpoint{4.587663in}{2.785472in}}%
\pgfpathlineto{\pgfqpoint{4.587663in}{2.393077in}}%
\pgfpathlineto{\pgfqpoint{4.652270in}{2.393077in}}%
\pgfpathlineto{\pgfqpoint{4.652270in}{2.579999in}}%
\pgfpathlineto{\pgfqpoint{4.716877in}{2.579999in}}%
\pgfpathlineto{\pgfqpoint{4.716877in}{2.376765in}}%
\pgfpathlineto{\pgfqpoint{4.781483in}{2.376765in}}%
\pgfpathlineto{\pgfqpoint{4.781483in}{2.511686in}}%
\pgfpathlineto{\pgfqpoint{4.846090in}{2.511686in}}%
\pgfpathlineto{\pgfqpoint{4.846090in}{2.553229in}}%
\pgfpathlineto{\pgfqpoint{4.910696in}{2.553229in}}%
\pgfpathlineto{\pgfqpoint{4.910696in}{2.776500in}}%
\pgfpathlineto{\pgfqpoint{4.942999in}{2.776500in}}%
\pgfusepath{stroke}%
\end{pgfscope}%
\begin{pgfscope}%
\pgfpathrectangle{\pgfqpoint{0.725000in}{1.931739in}}{\pgfqpoint{4.495000in}{0.984783in}}%
\pgfusepath{clip}%
\pgfsetbuttcap%
\pgfsetroundjoin%
\definecolor{currentfill}{rgb}{1.000000,0.498039,0.054902}%
\pgfsetfillcolor{currentfill}%
\pgfsetlinewidth{1.003750pt}%
\definecolor{currentstroke}{rgb}{1.000000,0.498039,0.054902}%
\pgfsetstrokecolor{currentstroke}%
\pgfsetdash{}{0pt}%
\pgfsys@defobject{currentmarker}{\pgfqpoint{-0.041667in}{-0.041667in}}{\pgfqpoint{0.041667in}{0.041667in}}{%
\pgfpathmoveto{\pgfqpoint{0.000000in}{-0.041667in}}%
\pgfpathcurveto{\pgfqpoint{0.011050in}{-0.041667in}}{\pgfqpoint{0.021649in}{-0.037276in}}{\pgfqpoint{0.029463in}{-0.029463in}}%
\pgfpathcurveto{\pgfqpoint{0.037276in}{-0.021649in}}{\pgfqpoint{0.041667in}{-0.011050in}}{\pgfqpoint{0.041667in}{0.000000in}}%
\pgfpathcurveto{\pgfqpoint{0.041667in}{0.011050in}}{\pgfqpoint{0.037276in}{0.021649in}}{\pgfqpoint{0.029463in}{0.029463in}}%
\pgfpathcurveto{\pgfqpoint{0.021649in}{0.037276in}}{\pgfqpoint{0.011050in}{0.041667in}}{\pgfqpoint{0.000000in}{0.041667in}}%
\pgfpathcurveto{\pgfqpoint{-0.011050in}{0.041667in}}{\pgfqpoint{-0.021649in}{0.037276in}}{\pgfqpoint{-0.029463in}{0.029463in}}%
\pgfpathcurveto{\pgfqpoint{-0.037276in}{0.021649in}}{\pgfqpoint{-0.041667in}{0.011050in}}{\pgfqpoint{-0.041667in}{0.000000in}}%
\pgfpathcurveto{\pgfqpoint{-0.041667in}{-0.011050in}}{\pgfqpoint{-0.037276in}{-0.021649in}}{\pgfqpoint{-0.029463in}{-0.029463in}}%
\pgfpathcurveto{\pgfqpoint{-0.021649in}{-0.037276in}}{\pgfqpoint{-0.011050in}{-0.041667in}}{\pgfqpoint{0.000000in}{-0.041667in}}%
\pgfpathclose%
\pgfusepath{stroke,fill}%
}%
\begin{pgfscope}%
\pgfsys@transformshift{1.002001in}{2.766546in}%
\pgfsys@useobject{currentmarker}{}%
\end{pgfscope}%
\begin{pgfscope}%
\pgfsys@transformshift{1.066607in}{1.980471in}%
\pgfsys@useobject{currentmarker}{}%
\end{pgfscope}%
\begin{pgfscope}%
\pgfsys@transformshift{1.131214in}{2.614470in}%
\pgfsys@useobject{currentmarker}{}%
\end{pgfscope}%
\begin{pgfscope}%
\pgfsys@transformshift{1.195820in}{2.279156in}%
\pgfsys@useobject{currentmarker}{}%
\end{pgfscope}%
\begin{pgfscope}%
\pgfsys@transformshift{1.260427in}{2.523764in}%
\pgfsys@useobject{currentmarker}{}%
\end{pgfscope}%
\begin{pgfscope}%
\pgfsys@transformshift{1.325033in}{2.324494in}%
\pgfsys@useobject{currentmarker}{}%
\end{pgfscope}%
\begin{pgfscope}%
\pgfsys@transformshift{1.389640in}{2.265286in}%
\pgfsys@useobject{currentmarker}{}%
\end{pgfscope}%
\begin{pgfscope}%
\pgfsys@transformshift{1.454246in}{2.628226in}%
\pgfsys@useobject{currentmarker}{}%
\end{pgfscope}%
\begin{pgfscope}%
\pgfsys@transformshift{1.518853in}{2.149052in}%
\pgfsys@useobject{currentmarker}{}%
\end{pgfscope}%
\begin{pgfscope}%
\pgfsys@transformshift{1.583459in}{2.731691in}%
\pgfsys@useobject{currentmarker}{}%
\end{pgfscope}%
\begin{pgfscope}%
\pgfsys@transformshift{1.648066in}{2.196217in}%
\pgfsys@useobject{currentmarker}{}%
\end{pgfscope}%
\begin{pgfscope}%
\pgfsys@transformshift{1.712672in}{2.582556in}%
\pgfsys@useobject{currentmarker}{}%
\end{pgfscope}%
\begin{pgfscope}%
\pgfsys@transformshift{1.777279in}{2.110536in}%
\pgfsys@useobject{currentmarker}{}%
\end{pgfscope}%
\begin{pgfscope}%
\pgfsys@transformshift{1.841886in}{2.681801in}%
\pgfsys@useobject{currentmarker}{}%
\end{pgfscope}%
\begin{pgfscope}%
\pgfsys@transformshift{1.906492in}{2.399326in}%
\pgfsys@useobject{currentmarker}{}%
\end{pgfscope}%
\begin{pgfscope}%
\pgfsys@transformshift{1.971099in}{2.441723in}%
\pgfsys@useobject{currentmarker}{}%
\end{pgfscope}%
\begin{pgfscope}%
\pgfsys@transformshift{2.035705in}{1.976502in}%
\pgfsys@useobject{currentmarker}{}%
\end{pgfscope}%
\begin{pgfscope}%
\pgfsys@transformshift{2.100312in}{2.871759in}%
\pgfsys@useobject{currentmarker}{}%
\end{pgfscope}%
\begin{pgfscope}%
\pgfsys@transformshift{2.164918in}{2.481993in}%
\pgfsys@useobject{currentmarker}{}%
\end{pgfscope}%
\begin{pgfscope}%
\pgfsys@transformshift{2.229525in}{2.266006in}%
\pgfsys@useobject{currentmarker}{}%
\end{pgfscope}%
\begin{pgfscope}%
\pgfsys@transformshift{2.294131in}{2.489000in}%
\pgfsys@useobject{currentmarker}{}%
\end{pgfscope}%
\begin{pgfscope}%
\pgfsys@transformshift{2.358738in}{2.328215in}%
\pgfsys@useobject{currentmarker}{}%
\end{pgfscope}%
\begin{pgfscope}%
\pgfsys@transformshift{2.423344in}{2.622793in}%
\pgfsys@useobject{currentmarker}{}%
\end{pgfscope}%
\begin{pgfscope}%
\pgfsys@transformshift{2.487951in}{2.463950in}%
\pgfsys@useobject{currentmarker}{}%
\end{pgfscope}%
\begin{pgfscope}%
\pgfsys@transformshift{2.552557in}{2.263039in}%
\pgfsys@useobject{currentmarker}{}%
\end{pgfscope}%
\begin{pgfscope}%
\pgfsys@transformshift{2.617164in}{2.421229in}%
\pgfsys@useobject{currentmarker}{}%
\end{pgfscope}%
\begin{pgfscope}%
\pgfsys@transformshift{2.681771in}{2.697854in}%
\pgfsys@useobject{currentmarker}{}%
\end{pgfscope}%
\begin{pgfscope}%
\pgfsys@transformshift{2.746377in}{2.054196in}%
\pgfsys@useobject{currentmarker}{}%
\end{pgfscope}%
\begin{pgfscope}%
\pgfsys@transformshift{2.810984in}{2.639615in}%
\pgfsys@useobject{currentmarker}{}%
\end{pgfscope}%
\begin{pgfscope}%
\pgfsys@transformshift{2.875590in}{2.786765in}%
\pgfsys@useobject{currentmarker}{}%
\end{pgfscope}%
\begin{pgfscope}%
\pgfsys@transformshift{2.940197in}{2.140644in}%
\pgfsys@useobject{currentmarker}{}%
\end{pgfscope}%
\begin{pgfscope}%
\pgfsys@transformshift{3.004803in}{2.311497in}%
\pgfsys@useobject{currentmarker}{}%
\end{pgfscope}%
\begin{pgfscope}%
\pgfsys@transformshift{3.069410in}{2.760043in}%
\pgfsys@useobject{currentmarker}{}%
\end{pgfscope}%
\begin{pgfscope}%
\pgfsys@transformshift{3.134016in}{2.328275in}%
\pgfsys@useobject{currentmarker}{}%
\end{pgfscope}%
\begin{pgfscope}%
\pgfsys@transformshift{3.198623in}{2.383322in}%
\pgfsys@useobject{currentmarker}{}%
\end{pgfscope}%
\begin{pgfscope}%
\pgfsys@transformshift{3.263229in}{2.812412in}%
\pgfsys@useobject{currentmarker}{}%
\end{pgfscope}%
\begin{pgfscope}%
\pgfsys@transformshift{3.327836in}{2.324382in}%
\pgfsys@useobject{currentmarker}{}%
\end{pgfscope}%
\begin{pgfscope}%
\pgfsys@transformshift{3.392443in}{2.246046in}%
\pgfsys@useobject{currentmarker}{}%
\end{pgfscope}%
\begin{pgfscope}%
\pgfsys@transformshift{3.457049in}{2.870608in}%
\pgfsys@useobject{currentmarker}{}%
\end{pgfscope}%
\begin{pgfscope}%
\pgfsys@transformshift{3.521656in}{2.217540in}%
\pgfsys@useobject{currentmarker}{}%
\end{pgfscope}%
\begin{pgfscope}%
\pgfsys@transformshift{3.586262in}{2.516456in}%
\pgfsys@useobject{currentmarker}{}%
\end{pgfscope}%
\begin{pgfscope}%
\pgfsys@transformshift{3.650869in}{2.572813in}%
\pgfsys@useobject{currentmarker}{}%
\end{pgfscope}%
\begin{pgfscope}%
\pgfsys@transformshift{3.715475in}{2.345840in}%
\pgfsys@useobject{currentmarker}{}%
\end{pgfscope}%
\begin{pgfscope}%
\pgfsys@transformshift{3.780082in}{2.615555in}%
\pgfsys@useobject{currentmarker}{}%
\end{pgfscope}%
\begin{pgfscope}%
\pgfsys@transformshift{3.844688in}{2.586554in}%
\pgfsys@useobject{currentmarker}{}%
\end{pgfscope}%
\begin{pgfscope}%
\pgfsys@transformshift{3.909295in}{2.108886in}%
\pgfsys@useobject{currentmarker}{}%
\end{pgfscope}%
\begin{pgfscope}%
\pgfsys@transformshift{3.973901in}{2.861688in}%
\pgfsys@useobject{currentmarker}{}%
\end{pgfscope}%
\begin{pgfscope}%
\pgfsys@transformshift{4.038508in}{2.483805in}%
\pgfsys@useobject{currentmarker}{}%
\end{pgfscope}%
\begin{pgfscope}%
\pgfsys@transformshift{4.103114in}{2.390139in}%
\pgfsys@useobject{currentmarker}{}%
\end{pgfscope}%
\begin{pgfscope}%
\pgfsys@transformshift{4.167721in}{2.487197in}%
\pgfsys@useobject{currentmarker}{}%
\end{pgfscope}%
\begin{pgfscope}%
\pgfsys@transformshift{4.232328in}{2.493807in}%
\pgfsys@useobject{currentmarker}{}%
\end{pgfscope}%
\begin{pgfscope}%
\pgfsys@transformshift{4.296934in}{2.594396in}%
\pgfsys@useobject{currentmarker}{}%
\end{pgfscope}%
\begin{pgfscope}%
\pgfsys@transformshift{4.361541in}{2.545438in}%
\pgfsys@useobject{currentmarker}{}%
\end{pgfscope}%
\begin{pgfscope}%
\pgfsys@transformshift{4.426147in}{2.445321in}%
\pgfsys@useobject{currentmarker}{}%
\end{pgfscope}%
\begin{pgfscope}%
\pgfsys@transformshift{4.490754in}{2.456259in}%
\pgfsys@useobject{currentmarker}{}%
\end{pgfscope}%
\begin{pgfscope}%
\pgfsys@transformshift{4.555360in}{2.785472in}%
\pgfsys@useobject{currentmarker}{}%
\end{pgfscope}%
\begin{pgfscope}%
\pgfsys@transformshift{4.619967in}{2.393077in}%
\pgfsys@useobject{currentmarker}{}%
\end{pgfscope}%
\begin{pgfscope}%
\pgfsys@transformshift{4.684573in}{2.579999in}%
\pgfsys@useobject{currentmarker}{}%
\end{pgfscope}%
\begin{pgfscope}%
\pgfsys@transformshift{4.749180in}{2.376765in}%
\pgfsys@useobject{currentmarker}{}%
\end{pgfscope}%
\begin{pgfscope}%
\pgfsys@transformshift{4.813786in}{2.511686in}%
\pgfsys@useobject{currentmarker}{}%
\end{pgfscope}%
\begin{pgfscope}%
\pgfsys@transformshift{4.878393in}{2.553229in}%
\pgfsys@useobject{currentmarker}{}%
\end{pgfscope}%
\begin{pgfscope}%
\pgfsys@transformshift{4.942999in}{2.776500in}%
\pgfsys@useobject{currentmarker}{}%
\end{pgfscope}%
\end{pgfscope}%
\begin{pgfscope}%
\pgfsetrectcap%
\pgfsetmiterjoin%
\pgfsetlinewidth{0.803000pt}%
\definecolor{currentstroke}{rgb}{0.000000,0.000000,0.000000}%
\pgfsetstrokecolor{currentstroke}%
\pgfsetdash{}{0pt}%
\pgfpathmoveto{\pgfqpoint{0.725000in}{1.931739in}}%
\pgfpathlineto{\pgfqpoint{0.725000in}{2.916522in}}%
\pgfusepath{stroke}%
\end{pgfscope}%
\begin{pgfscope}%
\pgfsetrectcap%
\pgfsetmiterjoin%
\pgfsetlinewidth{0.803000pt}%
\definecolor{currentstroke}{rgb}{0.000000,0.000000,0.000000}%
\pgfsetstrokecolor{currentstroke}%
\pgfsetdash{}{0pt}%
\pgfpathmoveto{\pgfqpoint{5.220000in}{1.931739in}}%
\pgfpathlineto{\pgfqpoint{5.220000in}{2.916522in}}%
\pgfusepath{stroke}%
\end{pgfscope}%
\begin{pgfscope}%
\pgfsetrectcap%
\pgfsetmiterjoin%
\pgfsetlinewidth{0.803000pt}%
\definecolor{currentstroke}{rgb}{0.000000,0.000000,0.000000}%
\pgfsetstrokecolor{currentstroke}%
\pgfsetdash{}{0pt}%
\pgfpathmoveto{\pgfqpoint{0.725000in}{1.931739in}}%
\pgfpathlineto{\pgfqpoint{5.220000in}{1.931739in}}%
\pgfusepath{stroke}%
\end{pgfscope}%
\begin{pgfscope}%
\pgfsetrectcap%
\pgfsetmiterjoin%
\pgfsetlinewidth{0.803000pt}%
\definecolor{currentstroke}{rgb}{0.000000,0.000000,0.000000}%
\pgfsetstrokecolor{currentstroke}%
\pgfsetdash{}{0pt}%
\pgfpathmoveto{\pgfqpoint{0.725000in}{2.916522in}}%
\pgfpathlineto{\pgfqpoint{5.220000in}{2.916522in}}%
\pgfusepath{stroke}%
\end{pgfscope}%
\begin{pgfscope}%
\pgfsetbuttcap%
\pgfsetmiterjoin%
\definecolor{currentfill}{rgb}{1.000000,1.000000,1.000000}%
\pgfsetfillcolor{currentfill}%
\pgfsetlinewidth{0.000000pt}%
\definecolor{currentstroke}{rgb}{0.000000,0.000000,0.000000}%
\pgfsetstrokecolor{currentstroke}%
\pgfsetstrokeopacity{0.000000}%
\pgfsetdash{}{0pt}%
\pgfpathmoveto{\pgfqpoint{0.725000in}{0.750000in}}%
\pgfpathlineto{\pgfqpoint{5.220000in}{0.750000in}}%
\pgfpathlineto{\pgfqpoint{5.220000in}{1.734783in}}%
\pgfpathlineto{\pgfqpoint{0.725000in}{1.734783in}}%
\pgfpathclose%
\pgfusepath{fill}%
\end{pgfscope}%
\begin{pgfscope}%
\pgfsetbuttcap%
\pgfsetroundjoin%
\definecolor{currentfill}{rgb}{0.000000,0.000000,0.000000}%
\pgfsetfillcolor{currentfill}%
\pgfsetlinewidth{0.803000pt}%
\definecolor{currentstroke}{rgb}{0.000000,0.000000,0.000000}%
\pgfsetstrokecolor{currentstroke}%
\pgfsetdash{}{0pt}%
\pgfsys@defobject{currentmarker}{\pgfqpoint{0.000000in}{-0.048611in}}{\pgfqpoint{0.000000in}{0.000000in}}{%
\pgfpathmoveto{\pgfqpoint{0.000000in}{0.000000in}}%
\pgfpathlineto{\pgfqpoint{0.000000in}{-0.048611in}}%
\pgfusepath{stroke,fill}%
}%
\begin{pgfscope}%
\pgfsys@transformshift{0.913167in}{0.750000in}%
\pgfsys@useobject{currentmarker}{}%
\end{pgfscope}%
\end{pgfscope}%
\begin{pgfscope}%
\definecolor{textcolor}{rgb}{0.000000,0.000000,0.000000}%
\pgfsetstrokecolor{textcolor}%
\pgfsetfillcolor{textcolor}%
\pgftext[x=0.913167in,y=0.652778in,,top]{\color{textcolor}\rmfamily\fontsize{8.000000}{9.600000}\selectfont 0}%
\end{pgfscope}%
\begin{pgfscope}%
\pgfsetbuttcap%
\pgfsetroundjoin%
\definecolor{currentfill}{rgb}{0.000000,0.000000,0.000000}%
\pgfsetfillcolor{currentfill}%
\pgfsetlinewidth{0.803000pt}%
\definecolor{currentstroke}{rgb}{0.000000,0.000000,0.000000}%
\pgfsetstrokecolor{currentstroke}%
\pgfsetdash{}{0pt}%
\pgfsys@defobject{currentmarker}{\pgfqpoint{0.000000in}{-0.048611in}}{\pgfqpoint{0.000000in}{0.000000in}}{%
\pgfpathmoveto{\pgfqpoint{0.000000in}{0.000000in}}%
\pgfpathlineto{\pgfqpoint{0.000000in}{-0.048611in}}%
\pgfusepath{stroke,fill}%
}%
\begin{pgfscope}%
\pgfsys@transformshift{1.720748in}{0.750000in}%
\pgfsys@useobject{currentmarker}{}%
\end{pgfscope}%
\end{pgfscope}%
\begin{pgfscope}%
\definecolor{textcolor}{rgb}{0.000000,0.000000,0.000000}%
\pgfsetstrokecolor{textcolor}%
\pgfsetfillcolor{textcolor}%
\pgftext[x=1.720748in,y=0.652778in,,top]{\color{textcolor}\rmfamily\fontsize{8.000000}{9.600000}\selectfont 50}%
\end{pgfscope}%
\begin{pgfscope}%
\pgfsetbuttcap%
\pgfsetroundjoin%
\definecolor{currentfill}{rgb}{0.000000,0.000000,0.000000}%
\pgfsetfillcolor{currentfill}%
\pgfsetlinewidth{0.803000pt}%
\definecolor{currentstroke}{rgb}{0.000000,0.000000,0.000000}%
\pgfsetstrokecolor{currentstroke}%
\pgfsetdash{}{0pt}%
\pgfsys@defobject{currentmarker}{\pgfqpoint{0.000000in}{-0.048611in}}{\pgfqpoint{0.000000in}{0.000000in}}{%
\pgfpathmoveto{\pgfqpoint{0.000000in}{0.000000in}}%
\pgfpathlineto{\pgfqpoint{0.000000in}{-0.048611in}}%
\pgfusepath{stroke,fill}%
}%
\begin{pgfscope}%
\pgfsys@transformshift{2.528330in}{0.750000in}%
\pgfsys@useobject{currentmarker}{}%
\end{pgfscope}%
\end{pgfscope}%
\begin{pgfscope}%
\definecolor{textcolor}{rgb}{0.000000,0.000000,0.000000}%
\pgfsetstrokecolor{textcolor}%
\pgfsetfillcolor{textcolor}%
\pgftext[x=2.528330in,y=0.652778in,,top]{\color{textcolor}\rmfamily\fontsize{8.000000}{9.600000}\selectfont 100}%
\end{pgfscope}%
\begin{pgfscope}%
\pgfsetbuttcap%
\pgfsetroundjoin%
\definecolor{currentfill}{rgb}{0.000000,0.000000,0.000000}%
\pgfsetfillcolor{currentfill}%
\pgfsetlinewidth{0.803000pt}%
\definecolor{currentstroke}{rgb}{0.000000,0.000000,0.000000}%
\pgfsetstrokecolor{currentstroke}%
\pgfsetdash{}{0pt}%
\pgfsys@defobject{currentmarker}{\pgfqpoint{0.000000in}{-0.048611in}}{\pgfqpoint{0.000000in}{0.000000in}}{%
\pgfpathmoveto{\pgfqpoint{0.000000in}{0.000000in}}%
\pgfpathlineto{\pgfqpoint{0.000000in}{-0.048611in}}%
\pgfusepath{stroke,fill}%
}%
\begin{pgfscope}%
\pgfsys@transformshift{3.335912in}{0.750000in}%
\pgfsys@useobject{currentmarker}{}%
\end{pgfscope}%
\end{pgfscope}%
\begin{pgfscope}%
\definecolor{textcolor}{rgb}{0.000000,0.000000,0.000000}%
\pgfsetstrokecolor{textcolor}%
\pgfsetfillcolor{textcolor}%
\pgftext[x=3.335912in,y=0.652778in,,top]{\color{textcolor}\rmfamily\fontsize{8.000000}{9.600000}\selectfont 150}%
\end{pgfscope}%
\begin{pgfscope}%
\pgfsetbuttcap%
\pgfsetroundjoin%
\definecolor{currentfill}{rgb}{0.000000,0.000000,0.000000}%
\pgfsetfillcolor{currentfill}%
\pgfsetlinewidth{0.803000pt}%
\definecolor{currentstroke}{rgb}{0.000000,0.000000,0.000000}%
\pgfsetstrokecolor{currentstroke}%
\pgfsetdash{}{0pt}%
\pgfsys@defobject{currentmarker}{\pgfqpoint{0.000000in}{-0.048611in}}{\pgfqpoint{0.000000in}{0.000000in}}{%
\pgfpathmoveto{\pgfqpoint{0.000000in}{0.000000in}}%
\pgfpathlineto{\pgfqpoint{0.000000in}{-0.048611in}}%
\pgfusepath{stroke,fill}%
}%
\begin{pgfscope}%
\pgfsys@transformshift{4.143494in}{0.750000in}%
\pgfsys@useobject{currentmarker}{}%
\end{pgfscope}%
\end{pgfscope}%
\begin{pgfscope}%
\definecolor{textcolor}{rgb}{0.000000,0.000000,0.000000}%
\pgfsetstrokecolor{textcolor}%
\pgfsetfillcolor{textcolor}%
\pgftext[x=4.143494in,y=0.652778in,,top]{\color{textcolor}\rmfamily\fontsize{8.000000}{9.600000}\selectfont 200}%
\end{pgfscope}%
\begin{pgfscope}%
\pgfsetbuttcap%
\pgfsetroundjoin%
\definecolor{currentfill}{rgb}{0.000000,0.000000,0.000000}%
\pgfsetfillcolor{currentfill}%
\pgfsetlinewidth{0.803000pt}%
\definecolor{currentstroke}{rgb}{0.000000,0.000000,0.000000}%
\pgfsetstrokecolor{currentstroke}%
\pgfsetdash{}{0pt}%
\pgfsys@defobject{currentmarker}{\pgfqpoint{0.000000in}{-0.048611in}}{\pgfqpoint{0.000000in}{0.000000in}}{%
\pgfpathmoveto{\pgfqpoint{0.000000in}{0.000000in}}%
\pgfpathlineto{\pgfqpoint{0.000000in}{-0.048611in}}%
\pgfusepath{stroke,fill}%
}%
\begin{pgfscope}%
\pgfsys@transformshift{4.951075in}{0.750000in}%
\pgfsys@useobject{currentmarker}{}%
\end{pgfscope}%
\end{pgfscope}%
\begin{pgfscope}%
\definecolor{textcolor}{rgb}{0.000000,0.000000,0.000000}%
\pgfsetstrokecolor{textcolor}%
\pgfsetfillcolor{textcolor}%
\pgftext[x=4.951075in,y=0.652778in,,top]{\color{textcolor}\rmfamily\fontsize{8.000000}{9.600000}\selectfont 250}%
\end{pgfscope}%
\begin{pgfscope}%
\pgfsetbuttcap%
\pgfsetroundjoin%
\definecolor{currentfill}{rgb}{0.000000,0.000000,0.000000}%
\pgfsetfillcolor{currentfill}%
\pgfsetlinewidth{0.803000pt}%
\definecolor{currentstroke}{rgb}{0.000000,0.000000,0.000000}%
\pgfsetstrokecolor{currentstroke}%
\pgfsetdash{}{0pt}%
\pgfsys@defobject{currentmarker}{\pgfqpoint{-0.048611in}{0.000000in}}{\pgfqpoint{0.000000in}{0.000000in}}{%
\pgfpathmoveto{\pgfqpoint{0.000000in}{0.000000in}}%
\pgfpathlineto{\pgfqpoint{-0.048611in}{0.000000in}}%
\pgfusepath{stroke,fill}%
}%
\begin{pgfscope}%
\pgfsys@transformshift{0.725000in}{1.023676in}%
\pgfsys@useobject{currentmarker}{}%
\end{pgfscope}%
\end{pgfscope}%
\begin{pgfscope}%
\definecolor{textcolor}{rgb}{0.000000,0.000000,0.000000}%
\pgfsetstrokecolor{textcolor}%
\pgfsetfillcolor{textcolor}%
\pgftext[x=0.359000in,y=0.985120in,left,base]{\color{textcolor}\rmfamily\fontsize{8.000000}{9.600000}\selectfont 0.000}%
\end{pgfscope}%
\begin{pgfscope}%
\pgfsetbuttcap%
\pgfsetroundjoin%
\definecolor{currentfill}{rgb}{0.000000,0.000000,0.000000}%
\pgfsetfillcolor{currentfill}%
\pgfsetlinewidth{0.803000pt}%
\definecolor{currentstroke}{rgb}{0.000000,0.000000,0.000000}%
\pgfsetstrokecolor{currentstroke}%
\pgfsetdash{}{0pt}%
\pgfsys@defobject{currentmarker}{\pgfqpoint{-0.048611in}{0.000000in}}{\pgfqpoint{0.000000in}{0.000000in}}{%
\pgfpathmoveto{\pgfqpoint{0.000000in}{0.000000in}}%
\pgfpathlineto{\pgfqpoint{-0.048611in}{0.000000in}}%
\pgfusepath{stroke,fill}%
}%
\begin{pgfscope}%
\pgfsys@transformshift{0.725000in}{1.321989in}%
\pgfsys@useobject{currentmarker}{}%
\end{pgfscope}%
\end{pgfscope}%
\begin{pgfscope}%
\definecolor{textcolor}{rgb}{0.000000,0.000000,0.000000}%
\pgfsetstrokecolor{textcolor}%
\pgfsetfillcolor{textcolor}%
\pgftext[x=0.359000in,y=1.283434in,left,base]{\color{textcolor}\rmfamily\fontsize{8.000000}{9.600000}\selectfont 0.025}%
\end{pgfscope}%
\begin{pgfscope}%
\pgfsetbuttcap%
\pgfsetroundjoin%
\definecolor{currentfill}{rgb}{0.000000,0.000000,0.000000}%
\pgfsetfillcolor{currentfill}%
\pgfsetlinewidth{0.803000pt}%
\definecolor{currentstroke}{rgb}{0.000000,0.000000,0.000000}%
\pgfsetstrokecolor{currentstroke}%
\pgfsetdash{}{0pt}%
\pgfsys@defobject{currentmarker}{\pgfqpoint{-0.048611in}{0.000000in}}{\pgfqpoint{0.000000in}{0.000000in}}{%
\pgfpathmoveto{\pgfqpoint{0.000000in}{0.000000in}}%
\pgfpathlineto{\pgfqpoint{-0.048611in}{0.000000in}}%
\pgfusepath{stroke,fill}%
}%
\begin{pgfscope}%
\pgfsys@transformshift{0.725000in}{1.620303in}%
\pgfsys@useobject{currentmarker}{}%
\end{pgfscope}%
\end{pgfscope}%
\begin{pgfscope}%
\definecolor{textcolor}{rgb}{0.000000,0.000000,0.000000}%
\pgfsetstrokecolor{textcolor}%
\pgfsetfillcolor{textcolor}%
\pgftext[x=0.359000in,y=1.581747in,left,base]{\color{textcolor}\rmfamily\fontsize{8.000000}{9.600000}\selectfont 0.050}%
\end{pgfscope}%
\begin{pgfscope}%
\pgfpathrectangle{\pgfqpoint{0.725000in}{0.750000in}}{\pgfqpoint{4.495000in}{0.984783in}}%
\pgfusepath{clip}%
\pgfsetrectcap%
\pgfsetroundjoin%
\pgfsetlinewidth{1.505625pt}%
\definecolor{currentstroke}{rgb}{1.000000,0.498039,0.054902}%
\pgfsetstrokecolor{currentstroke}%
\pgfsetdash{}{0pt}%
\pgfpathmoveto{\pgfqpoint{1.098910in}{0.928108in}}%
\pgfpathlineto{\pgfqpoint{1.163517in}{0.928108in}}%
\pgfpathlineto{\pgfqpoint{1.163517in}{1.154399in}}%
\pgfpathlineto{\pgfqpoint{1.292730in}{1.154399in}}%
\pgfpathlineto{\pgfqpoint{1.292730in}{0.967838in}}%
\pgfpathlineto{\pgfqpoint{1.421943in}{0.967838in}}%
\pgfpathlineto{\pgfqpoint{1.421943in}{1.146124in}}%
\pgfpathlineto{\pgfqpoint{1.551156in}{1.146124in}}%
\pgfpathlineto{\pgfqpoint{1.551156in}{1.227733in}}%
\pgfpathlineto{\pgfqpoint{1.680369in}{1.227733in}}%
\pgfpathlineto{\pgfqpoint{1.680369in}{0.983573in}}%
\pgfpathlineto{\pgfqpoint{1.809582in}{0.983573in}}%
\pgfpathlineto{\pgfqpoint{1.809582in}{1.350378in}}%
\pgfpathlineto{\pgfqpoint{1.938795in}{1.350378in}}%
\pgfpathlineto{\pgfqpoint{1.938795in}{0.794763in}}%
\pgfpathlineto{\pgfqpoint{2.068008in}{0.794763in}}%
\pgfpathlineto{\pgfqpoint{2.068008in}{1.572699in}}%
\pgfpathlineto{\pgfqpoint{2.197222in}{1.572699in}}%
\pgfpathlineto{\pgfqpoint{2.197222in}{1.035173in}}%
\pgfpathlineto{\pgfqpoint{2.326435in}{1.035173in}}%
\pgfpathlineto{\pgfqpoint{2.326435in}{1.354900in}}%
\pgfpathlineto{\pgfqpoint{2.455648in}{1.354900in}}%
\pgfpathlineto{\pgfqpoint{2.455648in}{1.100699in}}%
\pgfpathlineto{\pgfqpoint{2.584861in}{1.100699in}}%
\pgfpathlineto{\pgfqpoint{2.584861in}{1.297367in}}%
\pgfpathlineto{\pgfqpoint{2.714074in}{1.297367in}}%
\pgfpathlineto{\pgfqpoint{2.714074in}{1.249926in}}%
\pgfpathlineto{\pgfqpoint{2.843287in}{1.249926in}}%
\pgfpathlineto{\pgfqpoint{2.843287in}{1.323121in}}%
\pgfpathlineto{\pgfqpoint{2.972500in}{1.323121in}}%
\pgfpathlineto{\pgfqpoint{2.972500in}{1.301331in}}%
\pgfpathlineto{\pgfqpoint{3.101713in}{1.301331in}}%
\pgfpathlineto{\pgfqpoint{3.101713in}{1.311702in}}%
\pgfpathlineto{\pgfqpoint{3.230926in}{1.311702in}}%
\pgfpathlineto{\pgfqpoint{3.230926in}{1.457184in}}%
\pgfpathlineto{\pgfqpoint{3.360139in}{1.457184in}}%
\pgfpathlineto{\pgfqpoint{3.360139in}{1.392488in}}%
\pgfpathlineto{\pgfqpoint{3.489352in}{1.392488in}}%
\pgfpathlineto{\pgfqpoint{3.489352in}{1.284646in}}%
\pgfpathlineto{\pgfqpoint{3.618565in}{1.284646in}}%
\pgfpathlineto{\pgfqpoint{3.618565in}{1.390775in}}%
\pgfpathlineto{\pgfqpoint{3.747778in}{1.390775in}}%
\pgfpathlineto{\pgfqpoint{3.747778in}{1.466506in}}%
\pgfpathlineto{\pgfqpoint{3.876992in}{1.466506in}}%
\pgfpathlineto{\pgfqpoint{3.876992in}{1.431952in}}%
\pgfpathlineto{\pgfqpoint{4.006205in}{1.431952in}}%
\pgfpathlineto{\pgfqpoint{4.006205in}{1.426981in}}%
\pgfpathlineto{\pgfqpoint{4.135418in}{1.426981in}}%
\pgfpathlineto{\pgfqpoint{4.135418in}{1.408093in}}%
\pgfpathlineto{\pgfqpoint{4.264631in}{1.408093in}}%
\pgfpathlineto{\pgfqpoint{4.264631in}{1.596707in}}%
\pgfpathlineto{\pgfqpoint{4.393844in}{1.596707in}}%
\pgfpathlineto{\pgfqpoint{4.393844in}{1.455391in}}%
\pgfpathlineto{\pgfqpoint{4.523057in}{1.455391in}}%
\pgfpathlineto{\pgfqpoint{4.523057in}{1.690020in}}%
\pgfpathlineto{\pgfqpoint{4.652270in}{1.690020in}}%
\pgfpathlineto{\pgfqpoint{4.652270in}{1.341372in}}%
\pgfpathlineto{\pgfqpoint{4.781483in}{1.341372in}}%
\pgfpathlineto{\pgfqpoint{4.781483in}{1.592059in}}%
\pgfpathlineto{\pgfqpoint{4.846090in}{1.592059in}}%
\pgfusepath{stroke}%
\end{pgfscope}%
\begin{pgfscope}%
\pgfpathrectangle{\pgfqpoint{0.725000in}{0.750000in}}{\pgfqpoint{4.495000in}{0.984783in}}%
\pgfusepath{clip}%
\pgfsetbuttcap%
\pgfsetroundjoin%
\definecolor{currentfill}{rgb}{1.000000,0.498039,0.054902}%
\pgfsetfillcolor{currentfill}%
\pgfsetlinewidth{1.003750pt}%
\definecolor{currentstroke}{rgb}{1.000000,0.498039,0.054902}%
\pgfsetstrokecolor{currentstroke}%
\pgfsetdash{}{0pt}%
\pgfsys@defobject{currentmarker}{\pgfqpoint{-0.041667in}{-0.041667in}}{\pgfqpoint{0.041667in}{0.041667in}}{%
\pgfpathmoveto{\pgfqpoint{0.000000in}{-0.041667in}}%
\pgfpathcurveto{\pgfqpoint{0.011050in}{-0.041667in}}{\pgfqpoint{0.021649in}{-0.037276in}}{\pgfqpoint{0.029463in}{-0.029463in}}%
\pgfpathcurveto{\pgfqpoint{0.037276in}{-0.021649in}}{\pgfqpoint{0.041667in}{-0.011050in}}{\pgfqpoint{0.041667in}{0.000000in}}%
\pgfpathcurveto{\pgfqpoint{0.041667in}{0.011050in}}{\pgfqpoint{0.037276in}{0.021649in}}{\pgfqpoint{0.029463in}{0.029463in}}%
\pgfpathcurveto{\pgfqpoint{0.021649in}{0.037276in}}{\pgfqpoint{0.011050in}{0.041667in}}{\pgfqpoint{0.000000in}{0.041667in}}%
\pgfpathcurveto{\pgfqpoint{-0.011050in}{0.041667in}}{\pgfqpoint{-0.021649in}{0.037276in}}{\pgfqpoint{-0.029463in}{0.029463in}}%
\pgfpathcurveto{\pgfqpoint{-0.037276in}{0.021649in}}{\pgfqpoint{-0.041667in}{0.011050in}}{\pgfqpoint{-0.041667in}{0.000000in}}%
\pgfpathcurveto{\pgfqpoint{-0.041667in}{-0.011050in}}{\pgfqpoint{-0.037276in}{-0.021649in}}{\pgfqpoint{-0.029463in}{-0.029463in}}%
\pgfpathcurveto{\pgfqpoint{-0.021649in}{-0.037276in}}{\pgfqpoint{-0.011050in}{-0.041667in}}{\pgfqpoint{0.000000in}{-0.041667in}}%
\pgfpathclose%
\pgfusepath{stroke,fill}%
}%
\begin{pgfscope}%
\pgfsys@transformshift{1.098910in}{0.928108in}%
\pgfsys@useobject{currentmarker}{}%
\end{pgfscope}%
\begin{pgfscope}%
\pgfsys@transformshift{1.228123in}{1.154399in}%
\pgfsys@useobject{currentmarker}{}%
\end{pgfscope}%
\begin{pgfscope}%
\pgfsys@transformshift{1.357337in}{0.967838in}%
\pgfsys@useobject{currentmarker}{}%
\end{pgfscope}%
\begin{pgfscope}%
\pgfsys@transformshift{1.486550in}{1.146124in}%
\pgfsys@useobject{currentmarker}{}%
\end{pgfscope}%
\begin{pgfscope}%
\pgfsys@transformshift{1.615763in}{1.227733in}%
\pgfsys@useobject{currentmarker}{}%
\end{pgfscope}%
\begin{pgfscope}%
\pgfsys@transformshift{1.744976in}{0.983573in}%
\pgfsys@useobject{currentmarker}{}%
\end{pgfscope}%
\begin{pgfscope}%
\pgfsys@transformshift{1.874189in}{1.350378in}%
\pgfsys@useobject{currentmarker}{}%
\end{pgfscope}%
\begin{pgfscope}%
\pgfsys@transformshift{2.003402in}{0.794763in}%
\pgfsys@useobject{currentmarker}{}%
\end{pgfscope}%
\begin{pgfscope}%
\pgfsys@transformshift{2.132615in}{1.572699in}%
\pgfsys@useobject{currentmarker}{}%
\end{pgfscope}%
\begin{pgfscope}%
\pgfsys@transformshift{2.261828in}{1.035173in}%
\pgfsys@useobject{currentmarker}{}%
\end{pgfscope}%
\begin{pgfscope}%
\pgfsys@transformshift{2.391041in}{1.354900in}%
\pgfsys@useobject{currentmarker}{}%
\end{pgfscope}%
\begin{pgfscope}%
\pgfsys@transformshift{2.520254in}{1.100699in}%
\pgfsys@useobject{currentmarker}{}%
\end{pgfscope}%
\begin{pgfscope}%
\pgfsys@transformshift{2.649467in}{1.297367in}%
\pgfsys@useobject{currentmarker}{}%
\end{pgfscope}%
\begin{pgfscope}%
\pgfsys@transformshift{2.778680in}{1.249926in}%
\pgfsys@useobject{currentmarker}{}%
\end{pgfscope}%
\begin{pgfscope}%
\pgfsys@transformshift{2.907893in}{1.323121in}%
\pgfsys@useobject{currentmarker}{}%
\end{pgfscope}%
\begin{pgfscope}%
\pgfsys@transformshift{3.037107in}{1.301331in}%
\pgfsys@useobject{currentmarker}{}%
\end{pgfscope}%
\begin{pgfscope}%
\pgfsys@transformshift{3.166320in}{1.311702in}%
\pgfsys@useobject{currentmarker}{}%
\end{pgfscope}%
\begin{pgfscope}%
\pgfsys@transformshift{3.295533in}{1.457184in}%
\pgfsys@useobject{currentmarker}{}%
\end{pgfscope}%
\begin{pgfscope}%
\pgfsys@transformshift{3.424746in}{1.392488in}%
\pgfsys@useobject{currentmarker}{}%
\end{pgfscope}%
\begin{pgfscope}%
\pgfsys@transformshift{3.553959in}{1.284646in}%
\pgfsys@useobject{currentmarker}{}%
\end{pgfscope}%
\begin{pgfscope}%
\pgfsys@transformshift{3.683172in}{1.390775in}%
\pgfsys@useobject{currentmarker}{}%
\end{pgfscope}%
\begin{pgfscope}%
\pgfsys@transformshift{3.812385in}{1.466506in}%
\pgfsys@useobject{currentmarker}{}%
\end{pgfscope}%
\begin{pgfscope}%
\pgfsys@transformshift{3.941598in}{1.431952in}%
\pgfsys@useobject{currentmarker}{}%
\end{pgfscope}%
\begin{pgfscope}%
\pgfsys@transformshift{4.070811in}{1.426981in}%
\pgfsys@useobject{currentmarker}{}%
\end{pgfscope}%
\begin{pgfscope}%
\pgfsys@transformshift{4.200024in}{1.408093in}%
\pgfsys@useobject{currentmarker}{}%
\end{pgfscope}%
\begin{pgfscope}%
\pgfsys@transformshift{4.329237in}{1.596707in}%
\pgfsys@useobject{currentmarker}{}%
\end{pgfscope}%
\begin{pgfscope}%
\pgfsys@transformshift{4.458450in}{1.455391in}%
\pgfsys@useobject{currentmarker}{}%
\end{pgfscope}%
\begin{pgfscope}%
\pgfsys@transformshift{4.587663in}{1.690020in}%
\pgfsys@useobject{currentmarker}{}%
\end{pgfscope}%
\begin{pgfscope}%
\pgfsys@transformshift{4.716877in}{1.341372in}%
\pgfsys@useobject{currentmarker}{}%
\end{pgfscope}%
\begin{pgfscope}%
\pgfsys@transformshift{4.846090in}{1.592059in}%
\pgfsys@useobject{currentmarker}{}%
\end{pgfscope}%
\end{pgfscope}%
\begin{pgfscope}%
\pgfsetrectcap%
\pgfsetmiterjoin%
\pgfsetlinewidth{0.803000pt}%
\definecolor{currentstroke}{rgb}{0.000000,0.000000,0.000000}%
\pgfsetstrokecolor{currentstroke}%
\pgfsetdash{}{0pt}%
\pgfpathmoveto{\pgfqpoint{0.725000in}{0.750000in}}%
\pgfpathlineto{\pgfqpoint{0.725000in}{1.734783in}}%
\pgfusepath{stroke}%
\end{pgfscope}%
\begin{pgfscope}%
\pgfsetrectcap%
\pgfsetmiterjoin%
\pgfsetlinewidth{0.803000pt}%
\definecolor{currentstroke}{rgb}{0.000000,0.000000,0.000000}%
\pgfsetstrokecolor{currentstroke}%
\pgfsetdash{}{0pt}%
\pgfpathmoveto{\pgfqpoint{5.220000in}{0.750000in}}%
\pgfpathlineto{\pgfqpoint{5.220000in}{1.734783in}}%
\pgfusepath{stroke}%
\end{pgfscope}%
\begin{pgfscope}%
\pgfsetrectcap%
\pgfsetmiterjoin%
\pgfsetlinewidth{0.803000pt}%
\definecolor{currentstroke}{rgb}{0.000000,0.000000,0.000000}%
\pgfsetstrokecolor{currentstroke}%
\pgfsetdash{}{0pt}%
\pgfpathmoveto{\pgfqpoint{0.725000in}{0.750000in}}%
\pgfpathlineto{\pgfqpoint{5.220000in}{0.750000in}}%
\pgfusepath{stroke}%
\end{pgfscope}%
\begin{pgfscope}%
\pgfsetrectcap%
\pgfsetmiterjoin%
\pgfsetlinewidth{0.803000pt}%
\definecolor{currentstroke}{rgb}{0.000000,0.000000,0.000000}%
\pgfsetstrokecolor{currentstroke}%
\pgfsetdash{}{0pt}%
\pgfpathmoveto{\pgfqpoint{0.725000in}{1.734783in}}%
\pgfpathlineto{\pgfqpoint{5.220000in}{1.734783in}}%
\pgfusepath{stroke}%
\end{pgfscope}%
\end{pgfpicture}%
\makeatother%
\endgroup%

    \caption{Zweite Ableitung nach Mittelwertbildung\label{polynomials:noise:average2nd}}
\end{figure}

Daubechies Wavelets bieten also eine komfortable Methode um verrauschte Signale
abzuleiten. Die Multiskalenanalyse kann dabei einen Kompromiss zwischen
Empfindlichkeit gegenüber Rauschen und zeitlicher Auflösung bieten.

Die Anwendung zur Ableitung kann auf der Binder Plattform im
\texttt{Ableitungen.ipynb} Notebook%
\footnote{\url{https://mybinder.org/v2/gh/rnestler/mathsem-FS2019/paper?filepath=Ableitungen.ipynb}}
nachvollzogen werden.

\section{Hochfrequente Anteile in Polynomen}
\rhead{Hochfrequente Anteile in Polynomen}

\section{Schlussfolgerung}
\rhead{Schlussfolgerung}

\printbibliography[heading=subbibliography]
\end{refsection}
