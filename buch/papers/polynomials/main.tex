%
% main.tex -- Paper zum Thema Polynome
%
% (c) 2019 Hochschule Rapperswil
%
\chapter{Wavelets und polynomiale Signale\label{chapter:thema}}
\lhead{Wavelets und polynomiale Signale}
\begin{refsection}
\chapterauthor{Raphael Nestler}

In der Literatur zu Wavelets findet man die folgende Aussage zu Daubechie
Wavelets:
\begin{displayquote}[\cite{wikipedia:daubechies}]
For example, $D2$, with one vanishing moment, easily encodes polynomials of one
coefficient, or constant signal components. $D4$ encodes polynomials with two
coefficients, i.e.\ constant and linear signal components; and $D6$ encodes
3-polynomials, i.e.\ constant, linear and quadratic signal components.
\end{displayquote}
Ein Daubechie Wavelet mit $A$ verschwindender Momente und Filterlänge $N=2A$
soll also ein Polynom der Ordnung $A-1$ einfach darstellen können. Wir wollen
erörtern, was das nun in der Praxis genau bedeutet und welche Anwendungen es
ermöglicht.

\section{Abschnitt}
\rhead{Abschnitt}

\section{Schlussfolgerung}
\rhead{Schlussfolgerung}

\printbibliography[heading=subbibliography]
\end{refsection}
