%
% main.tex -- Paper zum Thema Polynome
%
% (c) 2019 Hochschule Rapperswil
%
\chapter{Wavelets und polynomiale Signale\label{chapter:thema}}
\lhead{Wavelets und polynomiale Signale}
\begin{refsection}
\chapterauthor{Raphael Nestler}

In der Literatur zu Wavelets findet man die folgende Aussage zu Daubechies
Wavelets:
\begin{displayquote}[\cite{wikipedia:daubechies}]
For example, $D2$, with one vanishing moment, easily encodes polynomials of one
coefficient, or constant signal components. $D4$ encodes polynomials with two
coefficients, i.e.\ constant and linear signal components; and $D6$ encodes
3-polynomials, i.e.\ constant, linear and quadratic signal components.
\end{displayquote}
Ein Daubechies Wavelet mit $A$ verschwindender Momente und Filterlänge $N=2A$
soll also ein Polynom der Ordnung $A-1$ einfach darstellen können. Wir wollen
erörtern, was das nun in der Praxis genau bedeutet und welche Anwendungen es
ermöglicht.

Die Analysen und Simulationen wurden jeweils mittels Python\cite{python} und im
speziellem dem PyWavelets\cite{gregory_r_lee_2019_2634243} Paket durchgeführt.

\section{Analyse von polynomialen Signalen}
\rhead{Polynomiale Signale}

Wir werden als erstes die Signale in~\ref{polynomials:signals} mittels des db1
(Haar) Wavelets analysieren.

\begin{figure}
	\centering
	%% Creator: Matplotlib, PGF backend
%%
%% To include the figure in your LaTeX document, write
%%   \input{<filename>.pgf}
%%
%% Make sure the required packages are loaded in your preamble
%%   \usepackage{pgf}
%%
%% Figures using additional raster images can only be included by \input if
%% they are in the same directory as the main LaTeX file. For loading figures
%% from other directories you can use the `import` package
%%   \usepackage{import}
%% and then include the figures with
%%   \import{<path to file>}{<filename>.pgf}
%%
%% Matplotlib used the following preamble
%%   \usepackage{fontspec}
%%
\begingroup%
\makeatletter%
\begin{pgfpicture}%
\pgfpathrectangle{\pgfpointorigin}{\pgfqpoint{2.900000in}{3.000000in}}%
\pgfusepath{use as bounding box, clip}%
\begin{pgfscope}%
\pgfsetbuttcap%
\pgfsetmiterjoin%
\definecolor{currentfill}{rgb}{1.000000,1.000000,1.000000}%
\pgfsetfillcolor{currentfill}%
\pgfsetlinewidth{0.000000pt}%
\definecolor{currentstroke}{rgb}{1.000000,1.000000,1.000000}%
\pgfsetstrokecolor{currentstroke}%
\pgfsetdash{}{0pt}%
\pgfpathmoveto{\pgfqpoint{0.000000in}{0.000000in}}%
\pgfpathlineto{\pgfqpoint{2.900000in}{0.000000in}}%
\pgfpathlineto{\pgfqpoint{2.900000in}{3.000000in}}%
\pgfpathlineto{\pgfqpoint{0.000000in}{3.000000in}}%
\pgfpathclose%
\pgfusepath{fill}%
\end{pgfscope}%
\begin{pgfscope}%
\pgfsetbuttcap%
\pgfsetmiterjoin%
\definecolor{currentfill}{rgb}{1.000000,1.000000,1.000000}%
\pgfsetfillcolor{currentfill}%
\pgfsetlinewidth{0.000000pt}%
\definecolor{currentstroke}{rgb}{0.000000,0.000000,0.000000}%
\pgfsetstrokecolor{currentstroke}%
\pgfsetstrokeopacity{0.000000}%
\pgfsetdash{}{0pt}%
\pgfpathmoveto{\pgfqpoint{0.362500in}{0.375000in}}%
\pgfpathlineto{\pgfqpoint{2.610000in}{0.375000in}}%
\pgfpathlineto{\pgfqpoint{2.610000in}{2.640000in}}%
\pgfpathlineto{\pgfqpoint{0.362500in}{2.640000in}}%
\pgfpathclose%
\pgfusepath{fill}%
\end{pgfscope}%
\begin{pgfscope}%
\pgfsetbuttcap%
\pgfsetroundjoin%
\definecolor{currentfill}{rgb}{0.000000,0.000000,0.000000}%
\pgfsetfillcolor{currentfill}%
\pgfsetlinewidth{0.803000pt}%
\definecolor{currentstroke}{rgb}{0.000000,0.000000,0.000000}%
\pgfsetstrokecolor{currentstroke}%
\pgfsetdash{}{0pt}%
\pgfsys@defobject{currentmarker}{\pgfqpoint{0.000000in}{-0.048611in}}{\pgfqpoint{0.000000in}{0.000000in}}{%
\pgfpathmoveto{\pgfqpoint{0.000000in}{0.000000in}}%
\pgfpathlineto{\pgfqpoint{0.000000in}{-0.048611in}}%
\pgfusepath{stroke,fill}%
}%
\begin{pgfscope}%
\pgfsys@transformshift{0.464659in}{0.375000in}%
\pgfsys@useobject{currentmarker}{}%
\end{pgfscope}%
\end{pgfscope}%
\begin{pgfscope}%
\definecolor{textcolor}{rgb}{0.000000,0.000000,0.000000}%
\pgfsetstrokecolor{textcolor}%
\pgfsetfillcolor{textcolor}%
\pgftext[x=0.464659in,y=0.277778in,,top]{\color{textcolor}\rmfamily\fontsize{8.000000}{9.600000}\selectfont 0.0}%
\end{pgfscope}%
\begin{pgfscope}%
\pgfsetbuttcap%
\pgfsetroundjoin%
\definecolor{currentfill}{rgb}{0.000000,0.000000,0.000000}%
\pgfsetfillcolor{currentfill}%
\pgfsetlinewidth{0.803000pt}%
\definecolor{currentstroke}{rgb}{0.000000,0.000000,0.000000}%
\pgfsetstrokecolor{currentstroke}%
\pgfsetdash{}{0pt}%
\pgfsys@defobject{currentmarker}{\pgfqpoint{0.000000in}{-0.048611in}}{\pgfqpoint{0.000000in}{0.000000in}}{%
\pgfpathmoveto{\pgfqpoint{0.000000in}{0.000000in}}%
\pgfpathlineto{\pgfqpoint{0.000000in}{-0.048611in}}%
\pgfusepath{stroke,fill}%
}%
\begin{pgfscope}%
\pgfsys@transformshift{0.975455in}{0.375000in}%
\pgfsys@useobject{currentmarker}{}%
\end{pgfscope}%
\end{pgfscope}%
\begin{pgfscope}%
\definecolor{textcolor}{rgb}{0.000000,0.000000,0.000000}%
\pgfsetstrokecolor{textcolor}%
\pgfsetfillcolor{textcolor}%
\pgftext[x=0.975455in,y=0.277778in,,top]{\color{textcolor}\rmfamily\fontsize{8.000000}{9.600000}\selectfont 0.5}%
\end{pgfscope}%
\begin{pgfscope}%
\pgfsetbuttcap%
\pgfsetroundjoin%
\definecolor{currentfill}{rgb}{0.000000,0.000000,0.000000}%
\pgfsetfillcolor{currentfill}%
\pgfsetlinewidth{0.803000pt}%
\definecolor{currentstroke}{rgb}{0.000000,0.000000,0.000000}%
\pgfsetstrokecolor{currentstroke}%
\pgfsetdash{}{0pt}%
\pgfsys@defobject{currentmarker}{\pgfqpoint{0.000000in}{-0.048611in}}{\pgfqpoint{0.000000in}{0.000000in}}{%
\pgfpathmoveto{\pgfqpoint{0.000000in}{0.000000in}}%
\pgfpathlineto{\pgfqpoint{0.000000in}{-0.048611in}}%
\pgfusepath{stroke,fill}%
}%
\begin{pgfscope}%
\pgfsys@transformshift{1.486250in}{0.375000in}%
\pgfsys@useobject{currentmarker}{}%
\end{pgfscope}%
\end{pgfscope}%
\begin{pgfscope}%
\definecolor{textcolor}{rgb}{0.000000,0.000000,0.000000}%
\pgfsetstrokecolor{textcolor}%
\pgfsetfillcolor{textcolor}%
\pgftext[x=1.486250in,y=0.277778in,,top]{\color{textcolor}\rmfamily\fontsize{8.000000}{9.600000}\selectfont 1.0}%
\end{pgfscope}%
\begin{pgfscope}%
\pgfsetbuttcap%
\pgfsetroundjoin%
\definecolor{currentfill}{rgb}{0.000000,0.000000,0.000000}%
\pgfsetfillcolor{currentfill}%
\pgfsetlinewidth{0.803000pt}%
\definecolor{currentstroke}{rgb}{0.000000,0.000000,0.000000}%
\pgfsetstrokecolor{currentstroke}%
\pgfsetdash{}{0pt}%
\pgfsys@defobject{currentmarker}{\pgfqpoint{0.000000in}{-0.048611in}}{\pgfqpoint{0.000000in}{0.000000in}}{%
\pgfpathmoveto{\pgfqpoint{0.000000in}{0.000000in}}%
\pgfpathlineto{\pgfqpoint{0.000000in}{-0.048611in}}%
\pgfusepath{stroke,fill}%
}%
\begin{pgfscope}%
\pgfsys@transformshift{1.997045in}{0.375000in}%
\pgfsys@useobject{currentmarker}{}%
\end{pgfscope}%
\end{pgfscope}%
\begin{pgfscope}%
\definecolor{textcolor}{rgb}{0.000000,0.000000,0.000000}%
\pgfsetstrokecolor{textcolor}%
\pgfsetfillcolor{textcolor}%
\pgftext[x=1.997045in,y=0.277778in,,top]{\color{textcolor}\rmfamily\fontsize{8.000000}{9.600000}\selectfont 1.5}%
\end{pgfscope}%
\begin{pgfscope}%
\pgfsetbuttcap%
\pgfsetroundjoin%
\definecolor{currentfill}{rgb}{0.000000,0.000000,0.000000}%
\pgfsetfillcolor{currentfill}%
\pgfsetlinewidth{0.803000pt}%
\definecolor{currentstroke}{rgb}{0.000000,0.000000,0.000000}%
\pgfsetstrokecolor{currentstroke}%
\pgfsetdash{}{0pt}%
\pgfsys@defobject{currentmarker}{\pgfqpoint{0.000000in}{-0.048611in}}{\pgfqpoint{0.000000in}{0.000000in}}{%
\pgfpathmoveto{\pgfqpoint{0.000000in}{0.000000in}}%
\pgfpathlineto{\pgfqpoint{0.000000in}{-0.048611in}}%
\pgfusepath{stroke,fill}%
}%
\begin{pgfscope}%
\pgfsys@transformshift{2.507841in}{0.375000in}%
\pgfsys@useobject{currentmarker}{}%
\end{pgfscope}%
\end{pgfscope}%
\begin{pgfscope}%
\definecolor{textcolor}{rgb}{0.000000,0.000000,0.000000}%
\pgfsetstrokecolor{textcolor}%
\pgfsetfillcolor{textcolor}%
\pgftext[x=2.507841in,y=0.277778in,,top]{\color{textcolor}\rmfamily\fontsize{8.000000}{9.600000}\selectfont 2.0}%
\end{pgfscope}%
\begin{pgfscope}%
\pgfsetbuttcap%
\pgfsetroundjoin%
\definecolor{currentfill}{rgb}{0.000000,0.000000,0.000000}%
\pgfsetfillcolor{currentfill}%
\pgfsetlinewidth{0.803000pt}%
\definecolor{currentstroke}{rgb}{0.000000,0.000000,0.000000}%
\pgfsetstrokecolor{currentstroke}%
\pgfsetdash{}{0pt}%
\pgfsys@defobject{currentmarker}{\pgfqpoint{-0.048611in}{0.000000in}}{\pgfqpoint{0.000000in}{0.000000in}}{%
\pgfpathmoveto{\pgfqpoint{0.000000in}{0.000000in}}%
\pgfpathlineto{\pgfqpoint{-0.048611in}{0.000000in}}%
\pgfusepath{stroke,fill}%
}%
\begin{pgfscope}%
\pgfsys@transformshift{0.362500in}{0.477955in}%
\pgfsys@useobject{currentmarker}{}%
\end{pgfscope}%
\end{pgfscope}%
\begin{pgfscope}%
\definecolor{textcolor}{rgb}{0.000000,0.000000,0.000000}%
\pgfsetstrokecolor{textcolor}%
\pgfsetfillcolor{textcolor}%
\pgftext[x=0.206278in,y=0.439399in,left,base]{\color{textcolor}\rmfamily\fontsize{8.000000}{9.600000}\selectfont 0}%
\end{pgfscope}%
\begin{pgfscope}%
\pgfsetbuttcap%
\pgfsetroundjoin%
\definecolor{currentfill}{rgb}{0.000000,0.000000,0.000000}%
\pgfsetfillcolor{currentfill}%
\pgfsetlinewidth{0.803000pt}%
\definecolor{currentstroke}{rgb}{0.000000,0.000000,0.000000}%
\pgfsetstrokecolor{currentstroke}%
\pgfsetdash{}{0pt}%
\pgfsys@defobject{currentmarker}{\pgfqpoint{-0.048611in}{0.000000in}}{\pgfqpoint{0.000000in}{0.000000in}}{%
\pgfpathmoveto{\pgfqpoint{0.000000in}{0.000000in}}%
\pgfpathlineto{\pgfqpoint{-0.048611in}{0.000000in}}%
\pgfusepath{stroke,fill}%
}%
\begin{pgfscope}%
\pgfsys@transformshift{0.362500in}{0.735341in}%
\pgfsys@useobject{currentmarker}{}%
\end{pgfscope}%
\end{pgfscope}%
\begin{pgfscope}%
\definecolor{textcolor}{rgb}{0.000000,0.000000,0.000000}%
\pgfsetstrokecolor{textcolor}%
\pgfsetfillcolor{textcolor}%
\pgftext[x=0.206278in,y=0.696785in,left,base]{\color{textcolor}\rmfamily\fontsize{8.000000}{9.600000}\selectfont 1}%
\end{pgfscope}%
\begin{pgfscope}%
\pgfsetbuttcap%
\pgfsetroundjoin%
\definecolor{currentfill}{rgb}{0.000000,0.000000,0.000000}%
\pgfsetfillcolor{currentfill}%
\pgfsetlinewidth{0.803000pt}%
\definecolor{currentstroke}{rgb}{0.000000,0.000000,0.000000}%
\pgfsetstrokecolor{currentstroke}%
\pgfsetdash{}{0pt}%
\pgfsys@defobject{currentmarker}{\pgfqpoint{-0.048611in}{0.000000in}}{\pgfqpoint{0.000000in}{0.000000in}}{%
\pgfpathmoveto{\pgfqpoint{0.000000in}{0.000000in}}%
\pgfpathlineto{\pgfqpoint{-0.048611in}{0.000000in}}%
\pgfusepath{stroke,fill}%
}%
\begin{pgfscope}%
\pgfsys@transformshift{0.362500in}{0.992727in}%
\pgfsys@useobject{currentmarker}{}%
\end{pgfscope}%
\end{pgfscope}%
\begin{pgfscope}%
\definecolor{textcolor}{rgb}{0.000000,0.000000,0.000000}%
\pgfsetstrokecolor{textcolor}%
\pgfsetfillcolor{textcolor}%
\pgftext[x=0.206278in,y=0.954172in,left,base]{\color{textcolor}\rmfamily\fontsize{8.000000}{9.600000}\selectfont 2}%
\end{pgfscope}%
\begin{pgfscope}%
\pgfsetbuttcap%
\pgfsetroundjoin%
\definecolor{currentfill}{rgb}{0.000000,0.000000,0.000000}%
\pgfsetfillcolor{currentfill}%
\pgfsetlinewidth{0.803000pt}%
\definecolor{currentstroke}{rgb}{0.000000,0.000000,0.000000}%
\pgfsetstrokecolor{currentstroke}%
\pgfsetdash{}{0pt}%
\pgfsys@defobject{currentmarker}{\pgfqpoint{-0.048611in}{0.000000in}}{\pgfqpoint{0.000000in}{0.000000in}}{%
\pgfpathmoveto{\pgfqpoint{0.000000in}{0.000000in}}%
\pgfpathlineto{\pgfqpoint{-0.048611in}{0.000000in}}%
\pgfusepath{stroke,fill}%
}%
\begin{pgfscope}%
\pgfsys@transformshift{0.362500in}{1.250114in}%
\pgfsys@useobject{currentmarker}{}%
\end{pgfscope}%
\end{pgfscope}%
\begin{pgfscope}%
\definecolor{textcolor}{rgb}{0.000000,0.000000,0.000000}%
\pgfsetstrokecolor{textcolor}%
\pgfsetfillcolor{textcolor}%
\pgftext[x=0.206278in,y=1.211558in,left,base]{\color{textcolor}\rmfamily\fontsize{8.000000}{9.600000}\selectfont 3}%
\end{pgfscope}%
\begin{pgfscope}%
\pgfsetbuttcap%
\pgfsetroundjoin%
\definecolor{currentfill}{rgb}{0.000000,0.000000,0.000000}%
\pgfsetfillcolor{currentfill}%
\pgfsetlinewidth{0.803000pt}%
\definecolor{currentstroke}{rgb}{0.000000,0.000000,0.000000}%
\pgfsetstrokecolor{currentstroke}%
\pgfsetdash{}{0pt}%
\pgfsys@defobject{currentmarker}{\pgfqpoint{-0.048611in}{0.000000in}}{\pgfqpoint{0.000000in}{0.000000in}}{%
\pgfpathmoveto{\pgfqpoint{0.000000in}{0.000000in}}%
\pgfpathlineto{\pgfqpoint{-0.048611in}{0.000000in}}%
\pgfusepath{stroke,fill}%
}%
\begin{pgfscope}%
\pgfsys@transformshift{0.362500in}{1.507500in}%
\pgfsys@useobject{currentmarker}{}%
\end{pgfscope}%
\end{pgfscope}%
\begin{pgfscope}%
\definecolor{textcolor}{rgb}{0.000000,0.000000,0.000000}%
\pgfsetstrokecolor{textcolor}%
\pgfsetfillcolor{textcolor}%
\pgftext[x=0.206278in,y=1.468944in,left,base]{\color{textcolor}\rmfamily\fontsize{8.000000}{9.600000}\selectfont 4}%
\end{pgfscope}%
\begin{pgfscope}%
\pgfsetbuttcap%
\pgfsetroundjoin%
\definecolor{currentfill}{rgb}{0.000000,0.000000,0.000000}%
\pgfsetfillcolor{currentfill}%
\pgfsetlinewidth{0.803000pt}%
\definecolor{currentstroke}{rgb}{0.000000,0.000000,0.000000}%
\pgfsetstrokecolor{currentstroke}%
\pgfsetdash{}{0pt}%
\pgfsys@defobject{currentmarker}{\pgfqpoint{-0.048611in}{0.000000in}}{\pgfqpoint{0.000000in}{0.000000in}}{%
\pgfpathmoveto{\pgfqpoint{0.000000in}{0.000000in}}%
\pgfpathlineto{\pgfqpoint{-0.048611in}{0.000000in}}%
\pgfusepath{stroke,fill}%
}%
\begin{pgfscope}%
\pgfsys@transformshift{0.362500in}{1.764886in}%
\pgfsys@useobject{currentmarker}{}%
\end{pgfscope}%
\end{pgfscope}%
\begin{pgfscope}%
\definecolor{textcolor}{rgb}{0.000000,0.000000,0.000000}%
\pgfsetstrokecolor{textcolor}%
\pgfsetfillcolor{textcolor}%
\pgftext[x=0.206278in,y=1.726331in,left,base]{\color{textcolor}\rmfamily\fontsize{8.000000}{9.600000}\selectfont 5}%
\end{pgfscope}%
\begin{pgfscope}%
\pgfsetbuttcap%
\pgfsetroundjoin%
\definecolor{currentfill}{rgb}{0.000000,0.000000,0.000000}%
\pgfsetfillcolor{currentfill}%
\pgfsetlinewidth{0.803000pt}%
\definecolor{currentstroke}{rgb}{0.000000,0.000000,0.000000}%
\pgfsetstrokecolor{currentstroke}%
\pgfsetdash{}{0pt}%
\pgfsys@defobject{currentmarker}{\pgfqpoint{-0.048611in}{0.000000in}}{\pgfqpoint{0.000000in}{0.000000in}}{%
\pgfpathmoveto{\pgfqpoint{0.000000in}{0.000000in}}%
\pgfpathlineto{\pgfqpoint{-0.048611in}{0.000000in}}%
\pgfusepath{stroke,fill}%
}%
\begin{pgfscope}%
\pgfsys@transformshift{0.362500in}{2.022273in}%
\pgfsys@useobject{currentmarker}{}%
\end{pgfscope}%
\end{pgfscope}%
\begin{pgfscope}%
\definecolor{textcolor}{rgb}{0.000000,0.000000,0.000000}%
\pgfsetstrokecolor{textcolor}%
\pgfsetfillcolor{textcolor}%
\pgftext[x=0.206278in,y=1.983717in,left,base]{\color{textcolor}\rmfamily\fontsize{8.000000}{9.600000}\selectfont 6}%
\end{pgfscope}%
\begin{pgfscope}%
\pgfsetbuttcap%
\pgfsetroundjoin%
\definecolor{currentfill}{rgb}{0.000000,0.000000,0.000000}%
\pgfsetfillcolor{currentfill}%
\pgfsetlinewidth{0.803000pt}%
\definecolor{currentstroke}{rgb}{0.000000,0.000000,0.000000}%
\pgfsetstrokecolor{currentstroke}%
\pgfsetdash{}{0pt}%
\pgfsys@defobject{currentmarker}{\pgfqpoint{-0.048611in}{0.000000in}}{\pgfqpoint{0.000000in}{0.000000in}}{%
\pgfpathmoveto{\pgfqpoint{0.000000in}{0.000000in}}%
\pgfpathlineto{\pgfqpoint{-0.048611in}{0.000000in}}%
\pgfusepath{stroke,fill}%
}%
\begin{pgfscope}%
\pgfsys@transformshift{0.362500in}{2.279659in}%
\pgfsys@useobject{currentmarker}{}%
\end{pgfscope}%
\end{pgfscope}%
\begin{pgfscope}%
\definecolor{textcolor}{rgb}{0.000000,0.000000,0.000000}%
\pgfsetstrokecolor{textcolor}%
\pgfsetfillcolor{textcolor}%
\pgftext[x=0.206278in,y=2.241104in,left,base]{\color{textcolor}\rmfamily\fontsize{8.000000}{9.600000}\selectfont 7}%
\end{pgfscope}%
\begin{pgfscope}%
\pgfsetbuttcap%
\pgfsetroundjoin%
\definecolor{currentfill}{rgb}{0.000000,0.000000,0.000000}%
\pgfsetfillcolor{currentfill}%
\pgfsetlinewidth{0.803000pt}%
\definecolor{currentstroke}{rgb}{0.000000,0.000000,0.000000}%
\pgfsetstrokecolor{currentstroke}%
\pgfsetdash{}{0pt}%
\pgfsys@defobject{currentmarker}{\pgfqpoint{-0.048611in}{0.000000in}}{\pgfqpoint{0.000000in}{0.000000in}}{%
\pgfpathmoveto{\pgfqpoint{0.000000in}{0.000000in}}%
\pgfpathlineto{\pgfqpoint{-0.048611in}{0.000000in}}%
\pgfusepath{stroke,fill}%
}%
\begin{pgfscope}%
\pgfsys@transformshift{0.362500in}{2.537045in}%
\pgfsys@useobject{currentmarker}{}%
\end{pgfscope}%
\end{pgfscope}%
\begin{pgfscope}%
\definecolor{textcolor}{rgb}{0.000000,0.000000,0.000000}%
\pgfsetstrokecolor{textcolor}%
\pgfsetfillcolor{textcolor}%
\pgftext[x=0.206278in,y=2.498490in,left,base]{\color{textcolor}\rmfamily\fontsize{8.000000}{9.600000}\selectfont 8}%
\end{pgfscope}%
\begin{pgfscope}%
\pgfpathrectangle{\pgfqpoint{0.362500in}{0.375000in}}{\pgfqpoint{2.247500in}{2.265000in}}%
\pgfusepath{clip}%
\pgfsetrectcap%
\pgfsetroundjoin%
\pgfsetlinewidth{1.505625pt}%
\definecolor{currentstroke}{rgb}{0.121569,0.466667,0.705882}%
\pgfsetstrokecolor{currentstroke}%
\pgfsetdash{}{0pt}%
\pgfpathmoveto{\pgfqpoint{0.464659in}{0.477955in}}%
\pgfpathlineto{\pgfqpoint{0.472672in}{0.500749in}}%
\pgfpathlineto{\pgfqpoint{0.488697in}{0.517436in}}%
\pgfpathlineto{\pgfqpoint{0.504721in}{0.528925in}}%
\pgfpathlineto{\pgfqpoint{0.528759in}{0.542427in}}%
\pgfpathlineto{\pgfqpoint{0.560809in}{0.556917in}}%
\pgfpathlineto{\pgfqpoint{0.608884in}{0.574664in}}%
\pgfpathlineto{\pgfqpoint{0.664971in}{0.591927in}}%
\pgfpathlineto{\pgfqpoint{0.737083in}{0.610868in}}%
\pgfpathlineto{\pgfqpoint{0.825221in}{0.630865in}}%
\pgfpathlineto{\pgfqpoint{0.937395in}{0.653043in}}%
\pgfpathlineto{\pgfqpoint{1.073607in}{0.676673in}}%
\pgfpathlineto{\pgfqpoint{1.233857in}{0.701294in}}%
\pgfpathlineto{\pgfqpoint{1.426156in}{0.727656in}}%
\pgfpathlineto{\pgfqpoint{1.650506in}{0.755262in}}%
\pgfpathlineto{\pgfqpoint{1.914918in}{0.784623in}}%
\pgfpathlineto{\pgfqpoint{2.219392in}{0.815283in}}%
\pgfpathlineto{\pgfqpoint{2.507841in}{0.841954in}}%
\pgfpathlineto{\pgfqpoint{2.507841in}{0.841954in}}%
\pgfusepath{stroke}%
\end{pgfscope}%
\begin{pgfscope}%
\pgfpathrectangle{\pgfqpoint{0.362500in}{0.375000in}}{\pgfqpoint{2.247500in}{2.265000in}}%
\pgfusepath{clip}%
\pgfsetrectcap%
\pgfsetroundjoin%
\pgfsetlinewidth{1.505625pt}%
\definecolor{currentstroke}{rgb}{1.000000,0.498039,0.054902}%
\pgfsetstrokecolor{currentstroke}%
\pgfsetdash{}{0pt}%
\pgfpathmoveto{\pgfqpoint{0.464659in}{0.735341in}}%
\pgfpathlineto{\pgfqpoint{2.507841in}{0.735341in}}%
\pgfpathlineto{\pgfqpoint{2.507841in}{0.735341in}}%
\pgfusepath{stroke}%
\end{pgfscope}%
\begin{pgfscope}%
\pgfpathrectangle{\pgfqpoint{0.362500in}{0.375000in}}{\pgfqpoint{2.247500in}{2.265000in}}%
\pgfusepath{clip}%
\pgfsetrectcap%
\pgfsetroundjoin%
\pgfsetlinewidth{1.505625pt}%
\definecolor{currentstroke}{rgb}{0.172549,0.627451,0.172549}%
\pgfsetstrokecolor{currentstroke}%
\pgfsetdash{}{0pt}%
\pgfpathmoveto{\pgfqpoint{0.464659in}{0.477955in}}%
\pgfpathlineto{\pgfqpoint{2.507841in}{0.992727in}}%
\pgfpathlineto{\pgfqpoint{2.507841in}{0.992727in}}%
\pgfusepath{stroke}%
\end{pgfscope}%
\begin{pgfscope}%
\pgfpathrectangle{\pgfqpoint{0.362500in}{0.375000in}}{\pgfqpoint{2.247500in}{2.265000in}}%
\pgfusepath{clip}%
\pgfsetrectcap%
\pgfsetroundjoin%
\pgfsetlinewidth{1.505625pt}%
\definecolor{currentstroke}{rgb}{0.839216,0.152941,0.156863}%
\pgfsetstrokecolor{currentstroke}%
\pgfsetdash{}{0pt}%
\pgfpathmoveto{\pgfqpoint{0.464659in}{0.477955in}}%
\pgfpathlineto{\pgfqpoint{0.544784in}{0.479538in}}%
\pgfpathlineto{\pgfqpoint{0.624909in}{0.484288in}}%
\pgfpathlineto{\pgfqpoint{0.705033in}{0.492204in}}%
\pgfpathlineto{\pgfqpoint{0.785158in}{0.503287in}}%
\pgfpathlineto{\pgfqpoint{0.865283in}{0.517537in}}%
\pgfpathlineto{\pgfqpoint{0.945408in}{0.534954in}}%
\pgfpathlineto{\pgfqpoint{1.025533in}{0.555537in}}%
\pgfpathlineto{\pgfqpoint{1.105657in}{0.579286in}}%
\pgfpathlineto{\pgfqpoint{1.185782in}{0.606202in}}%
\pgfpathlineto{\pgfqpoint{1.265907in}{0.636285in}}%
\pgfpathlineto{\pgfqpoint{1.346032in}{0.669535in}}%
\pgfpathlineto{\pgfqpoint{1.426156in}{0.705951in}}%
\pgfpathlineto{\pgfqpoint{1.506281in}{0.745533in}}%
\pgfpathlineto{\pgfqpoint{1.586406in}{0.788283in}}%
\pgfpathlineto{\pgfqpoint{1.674543in}{0.838964in}}%
\pgfpathlineto{\pgfqpoint{1.762680in}{0.893478in}}%
\pgfpathlineto{\pgfqpoint{1.850818in}{0.951823in}}%
\pgfpathlineto{\pgfqpoint{1.938955in}{1.013999in}}%
\pgfpathlineto{\pgfqpoint{2.027092in}{1.080007in}}%
\pgfpathlineto{\pgfqpoint{2.115230in}{1.149847in}}%
\pgfpathlineto{\pgfqpoint{2.203367in}{1.223518in}}%
\pgfpathlineto{\pgfqpoint{2.291504in}{1.301021in}}%
\pgfpathlineto{\pgfqpoint{2.379641in}{1.382355in}}%
\pgfpathlineto{\pgfqpoint{2.475791in}{1.475454in}}%
\pgfpathlineto{\pgfqpoint{2.507841in}{1.507500in}}%
\pgfpathlineto{\pgfqpoint{2.507841in}{1.507500in}}%
\pgfusepath{stroke}%
\end{pgfscope}%
\begin{pgfscope}%
\pgfpathrectangle{\pgfqpoint{0.362500in}{0.375000in}}{\pgfqpoint{2.247500in}{2.265000in}}%
\pgfusepath{clip}%
\pgfsetrectcap%
\pgfsetroundjoin%
\pgfsetlinewidth{1.505625pt}%
\definecolor{currentstroke}{rgb}{0.580392,0.403922,0.741176}%
\pgfsetstrokecolor{currentstroke}%
\pgfsetdash{}{0pt}%
\pgfpathmoveto{\pgfqpoint{0.464659in}{0.477955in}}%
\pgfpathlineto{\pgfqpoint{0.648946in}{0.479465in}}%
\pgfpathlineto{\pgfqpoint{0.745096in}{0.483279in}}%
\pgfpathlineto{\pgfqpoint{0.825221in}{0.489271in}}%
\pgfpathlineto{\pgfqpoint{0.897333in}{0.497509in}}%
\pgfpathlineto{\pgfqpoint{0.969445in}{0.509006in}}%
\pgfpathlineto{\pgfqpoint{1.033545in}{0.522400in}}%
\pgfpathlineto{\pgfqpoint{1.097645in}{0.539181in}}%
\pgfpathlineto{\pgfqpoint{1.153732in}{0.556941in}}%
\pgfpathlineto{\pgfqpoint{1.209820in}{0.577840in}}%
\pgfpathlineto{\pgfqpoint{1.265907in}{0.602135in}}%
\pgfpathlineto{\pgfqpoint{1.321994in}{0.630082in}}%
\pgfpathlineto{\pgfqpoint{1.378082in}{0.661934in}}%
\pgfpathlineto{\pgfqpoint{1.426156in}{0.692539in}}%
\pgfpathlineto{\pgfqpoint{1.474231in}{0.726363in}}%
\pgfpathlineto{\pgfqpoint{1.522306in}{0.763567in}}%
\pgfpathlineto{\pgfqpoint{1.570381in}{0.804311in}}%
\pgfpathlineto{\pgfqpoint{1.618456in}{0.848757in}}%
\pgfpathlineto{\pgfqpoint{1.666531in}{0.897065in}}%
\pgfpathlineto{\pgfqpoint{1.714606in}{0.949397in}}%
\pgfpathlineto{\pgfqpoint{1.762680in}{1.005913in}}%
\pgfpathlineto{\pgfqpoint{1.810755in}{1.066775in}}%
\pgfpathlineto{\pgfqpoint{1.858830in}{1.132143in}}%
\pgfpathlineto{\pgfqpoint{1.906905in}{1.202178in}}%
\pgfpathlineto{\pgfqpoint{1.954980in}{1.277041in}}%
\pgfpathlineto{\pgfqpoint{2.003055in}{1.356893in}}%
\pgfpathlineto{\pgfqpoint{2.059142in}{1.456575in}}%
\pgfpathlineto{\pgfqpoint{2.115230in}{1.563522in}}%
\pgfpathlineto{\pgfqpoint{2.171317in}{1.677989in}}%
\pgfpathlineto{\pgfqpoint{2.227404in}{1.800233in}}%
\pgfpathlineto{\pgfqpoint{2.283492in}{1.930509in}}%
\pgfpathlineto{\pgfqpoint{2.339579in}{2.069073in}}%
\pgfpathlineto{\pgfqpoint{2.395666in}{2.216180in}}%
\pgfpathlineto{\pgfqpoint{2.451754in}{2.372086in}}%
\pgfpathlineto{\pgfqpoint{2.507841in}{2.537045in}}%
\pgfpathlineto{\pgfqpoint{2.507841in}{2.537045in}}%
\pgfusepath{stroke}%
\end{pgfscope}%
\begin{pgfscope}%
\pgfsetrectcap%
\pgfsetmiterjoin%
\pgfsetlinewidth{0.803000pt}%
\definecolor{currentstroke}{rgb}{0.000000,0.000000,0.000000}%
\pgfsetstrokecolor{currentstroke}%
\pgfsetdash{}{0pt}%
\pgfpathmoveto{\pgfqpoint{0.362500in}{0.375000in}}%
\pgfpathlineto{\pgfqpoint{0.362500in}{2.640000in}}%
\pgfusepath{stroke}%
\end{pgfscope}%
\begin{pgfscope}%
\pgfsetrectcap%
\pgfsetmiterjoin%
\pgfsetlinewidth{0.803000pt}%
\definecolor{currentstroke}{rgb}{0.000000,0.000000,0.000000}%
\pgfsetstrokecolor{currentstroke}%
\pgfsetdash{}{0pt}%
\pgfpathmoveto{\pgfqpoint{2.610000in}{0.375000in}}%
\pgfpathlineto{\pgfqpoint{2.610000in}{2.640000in}}%
\pgfusepath{stroke}%
\end{pgfscope}%
\begin{pgfscope}%
\pgfsetrectcap%
\pgfsetmiterjoin%
\pgfsetlinewidth{0.803000pt}%
\definecolor{currentstroke}{rgb}{0.000000,0.000000,0.000000}%
\pgfsetstrokecolor{currentstroke}%
\pgfsetdash{}{0pt}%
\pgfpathmoveto{\pgfqpoint{0.362500in}{0.375000in}}%
\pgfpathlineto{\pgfqpoint{2.610000in}{0.375000in}}%
\pgfusepath{stroke}%
\end{pgfscope}%
\begin{pgfscope}%
\pgfsetrectcap%
\pgfsetmiterjoin%
\pgfsetlinewidth{0.803000pt}%
\definecolor{currentstroke}{rgb}{0.000000,0.000000,0.000000}%
\pgfsetstrokecolor{currentstroke}%
\pgfsetdash{}{0pt}%
\pgfpathmoveto{\pgfqpoint{0.362500in}{2.640000in}}%
\pgfpathlineto{\pgfqpoint{2.610000in}{2.640000in}}%
\pgfusepath{stroke}%
\end{pgfscope}%
\begin{pgfscope}%
\pgfsetbuttcap%
\pgfsetmiterjoin%
\definecolor{currentfill}{rgb}{1.000000,1.000000,1.000000}%
\pgfsetfillcolor{currentfill}%
\pgfsetfillopacity{0.800000}%
\pgfsetlinewidth{1.003750pt}%
\definecolor{currentstroke}{rgb}{0.800000,0.800000,0.800000}%
\pgfsetstrokecolor{currentstroke}%
\pgfsetstrokeopacity{0.800000}%
\pgfsetdash{}{0pt}%
\pgfpathmoveto{\pgfqpoint{0.440278in}{1.774028in}}%
\pgfpathlineto{\pgfqpoint{1.002719in}{1.774028in}}%
\pgfpathquadraticcurveto{\pgfqpoint{1.024941in}{1.774028in}}{\pgfqpoint{1.024941in}{1.796251in}}%
\pgfpathlineto{\pgfqpoint{1.024941in}{2.562222in}}%
\pgfpathquadraticcurveto{\pgfqpoint{1.024941in}{2.584444in}}{\pgfqpoint{1.002719in}{2.584444in}}%
\pgfpathlineto{\pgfqpoint{0.440278in}{2.584444in}}%
\pgfpathquadraticcurveto{\pgfqpoint{0.418056in}{2.584444in}}{\pgfqpoint{0.418056in}{2.562222in}}%
\pgfpathlineto{\pgfqpoint{0.418056in}{1.796251in}}%
\pgfpathquadraticcurveto{\pgfqpoint{0.418056in}{1.774028in}}{\pgfqpoint{0.440278in}{1.774028in}}%
\pgfpathclose%
\pgfusepath{stroke,fill}%
\end{pgfscope}%
\begin{pgfscope}%
\pgfsetrectcap%
\pgfsetroundjoin%
\pgfsetlinewidth{1.505625pt}%
\definecolor{currentstroke}{rgb}{0.121569,0.466667,0.705882}%
\pgfsetstrokecolor{currentstroke}%
\pgfsetdash{}{0pt}%
\pgfpathmoveto{\pgfqpoint{0.462500in}{2.500583in}}%
\pgfpathlineto{\pgfqpoint{0.684722in}{2.500583in}}%
\pgfusepath{stroke}%
\end{pgfscope}%
\begin{pgfscope}%
\definecolor{textcolor}{rgb}{0.000000,0.000000,0.000000}%
\pgfsetstrokecolor{textcolor}%
\pgfsetfillcolor{textcolor}%
\pgftext[x=0.773611in,y=2.461694in,left,base]{\color{textcolor}\rmfamily\fontsize{8.000000}{9.600000}\selectfont \(\displaystyle x^{0.5}\)}%
\end{pgfscope}%
\begin{pgfscope}%
\pgfsetrectcap%
\pgfsetroundjoin%
\pgfsetlinewidth{1.505625pt}%
\definecolor{currentstroke}{rgb}{1.000000,0.498039,0.054902}%
\pgfsetstrokecolor{currentstroke}%
\pgfsetdash{}{0pt}%
\pgfpathmoveto{\pgfqpoint{0.462500in}{2.345167in}}%
\pgfpathlineto{\pgfqpoint{0.684722in}{2.345167in}}%
\pgfusepath{stroke}%
\end{pgfscope}%
\begin{pgfscope}%
\definecolor{textcolor}{rgb}{0.000000,0.000000,0.000000}%
\pgfsetstrokecolor{textcolor}%
\pgfsetfillcolor{textcolor}%
\pgftext[x=0.773611in,y=2.306278in,left,base]{\color{textcolor}\rmfamily\fontsize{8.000000}{9.600000}\selectfont \(\displaystyle x^{0}\)}%
\end{pgfscope}%
\begin{pgfscope}%
\pgfsetrectcap%
\pgfsetroundjoin%
\pgfsetlinewidth{1.505625pt}%
\definecolor{currentstroke}{rgb}{0.172549,0.627451,0.172549}%
\pgfsetstrokecolor{currentstroke}%
\pgfsetdash{}{0pt}%
\pgfpathmoveto{\pgfqpoint{0.462500in}{2.189750in}}%
\pgfpathlineto{\pgfqpoint{0.684722in}{2.189750in}}%
\pgfusepath{stroke}%
\end{pgfscope}%
\begin{pgfscope}%
\definecolor{textcolor}{rgb}{0.000000,0.000000,0.000000}%
\pgfsetstrokecolor{textcolor}%
\pgfsetfillcolor{textcolor}%
\pgftext[x=0.773611in,y=2.150861in,left,base]{\color{textcolor}\rmfamily\fontsize{8.000000}{9.600000}\selectfont \(\displaystyle x^{1}\)}%
\end{pgfscope}%
\begin{pgfscope}%
\pgfsetrectcap%
\pgfsetroundjoin%
\pgfsetlinewidth{1.505625pt}%
\definecolor{currentstroke}{rgb}{0.839216,0.152941,0.156863}%
\pgfsetstrokecolor{currentstroke}%
\pgfsetdash{}{0pt}%
\pgfpathmoveto{\pgfqpoint{0.462500in}{2.034334in}}%
\pgfpathlineto{\pgfqpoint{0.684722in}{2.034334in}}%
\pgfusepath{stroke}%
\end{pgfscope}%
\begin{pgfscope}%
\definecolor{textcolor}{rgb}{0.000000,0.000000,0.000000}%
\pgfsetstrokecolor{textcolor}%
\pgfsetfillcolor{textcolor}%
\pgftext[x=0.773611in,y=1.995445in,left,base]{\color{textcolor}\rmfamily\fontsize{8.000000}{9.600000}\selectfont \(\displaystyle x^{2}\)}%
\end{pgfscope}%
\begin{pgfscope}%
\pgfsetrectcap%
\pgfsetroundjoin%
\pgfsetlinewidth{1.505625pt}%
\definecolor{currentstroke}{rgb}{0.580392,0.403922,0.741176}%
\pgfsetstrokecolor{currentstroke}%
\pgfsetdash{}{0pt}%
\pgfpathmoveto{\pgfqpoint{0.462500in}{1.878917in}}%
\pgfpathlineto{\pgfqpoint{0.684722in}{1.878917in}}%
\pgfusepath{stroke}%
\end{pgfscope}%
\begin{pgfscope}%
\definecolor{textcolor}{rgb}{0.000000,0.000000,0.000000}%
\pgfsetstrokecolor{textcolor}%
\pgfsetfillcolor{textcolor}%
\pgftext[x=0.773611in,y=1.840028in,left,base]{\color{textcolor}\rmfamily\fontsize{8.000000}{9.600000}\selectfont \(\displaystyle x^{3}\)}%
\end{pgfscope}%
\end{pgfpicture}%
\makeatother%
\endgroup%

	\caption{Die verschiedenen zu analysierenden polynomialen Signale\label{polynomials:signals}}
\end{figure}

\section{Anwendung zur rauscharmen Ableitung}
\rhead{Rauscharme Ableitung}

\section{Hochfrequente Anteile in Polynomen}
\rhead{Hochfrequente Anteile in Polynomen}

\section{Schlussfolgerung}
\rhead{Schlussfolgerung}

\printbibliography[heading=subbibliography]
\end{refsection}
