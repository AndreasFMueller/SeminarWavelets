%
% main.tex -- Paper zum Thema wwt
%
% (c) 2019 Michael Schmid, Hochschule Rapperswil
%
\chapter{Wetter-Wavelet-Transformation\label{chapter:wwt}}
\lhead{Wetter-Wavelet-Transformation}
\begin{refsection}
\chapterauthor{Michael Schmid}

\section{Einführung}
\rhead{Einführung}


Seit langen konsultiere ich meine aktuellen Wetterdaten von eher unüblichen Wetter Internetseite.
Dabei handelt es sich um eine Privat geführte Wetterstation welche die gemessenen Daten, kostenlos und sehr rudimentär im Internet grafisch darstellt und auch tabellarisch zur Verfügung stellt.
Das Feature welches is bis anhin regelmässig nutze, war die grafische Darstellung der aktuellen Wetterdaten über den Zeitraum der letzten 24 Stunden.
Bei speziellen Ereignissen des Wetters vielen mir besondere und regelmässige Charakteristiken auf.
\\
\\
Nach der Einführung in die Theorie der Wavelets kam mir die Idee solche Wetterphänomene mittels einer geeigneten Wavelet Transformation zu detektieren.
In diesem Paper wird einerseits auf die theoretische Grundlage der Angewendeten Methoden sowie den besprochenen meteorologischen Phänomene zurückgegriffen und kurz erläutert. 
Weiter wird auf den Prozess der eigentlichen Inhaltes des Papers vertieft eingegangen. Ein besonderes Augenmerk wird auf die allgemeine Vorgehensweise sowie deren Schwierigkeiten gelegt.
\\
\\

\section{Wetterstation Seegräben}
\rhead{Wetterstation Seegräben}

Die Wetterstation Seegräben ist eine

\section{Datenaufarbeitung}
\rhead{Datenaufarbeitung}

\section{Stetige Wavelet Transformation}
\rhead{Stetige Wavelet Transformation}

\section{Analyse von Wettereignissen}
\rhead{Analyse von Wetterereignissen}

\subsection{Sturmtief}
\rhead{Sturmtief}


\section{Schlussfolgerung}
\rhead{Schlussfolgerung}

\printbibliography[heading=subbibliography]
\end{refsection}
