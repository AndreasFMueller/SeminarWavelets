%
% main.tex -- Paper zum Thema wwt
%
% (c) 2019 Michael Schmid, Hochschule Rapperswil
%
\chapter{Wetter-Wavelet-Transformation\label{chapter:wwt}}
\lhead{Wetter-Wavelet-Transformation}
\begin{refsection}
\chapterauthor{Michael Schmid}

\section{Abschnitt}
\rhead{Abschnitt}
\subsection{Einführung}

Seit langen konsultiere ich meine aktuellen Wetterdaten von eher unüblichen Wetter Internetseite.
Dabei handelt es sich um eine Privat geführte Wetterstation welche die gemessenen Daten, kostenlos und sehr rudimentär im Internet darstellt und zur Verfügung stellt.
Das Feature welches is bis anhin regelmässig nutze, war das Diagramm der aktuellen Wetterdaten über einen Zeitraum von 24 Stunden.
Bei speziellen Ereignissen wie Gewitter oder Winterstürme vielen mir besondere Charakteristiken immer wieder auf.
Beispielsweise traten bei den meisten Sturmtiefen rapide wechsel des Luftdruckes statt.
\\
\\
Nach der Einführung in die Theorie der Wavelets kam mir die Idee solche Wetterphänomene mittels der Wavelet-Transformation zu detektieren.



sdfdsfsf
sdfdsf

sdfsdf

\section{Schlussfolgerung}
\rhead{Schlussfolgerung}

\printbibliography[heading=subbibliography]
\end{refsection}
