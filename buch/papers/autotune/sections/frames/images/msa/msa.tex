%
% msa.tex -- visual representation of MSA
%
% (c) 2019 Prof Dr Andreas Müller, Hochschule Rapperswil
%
\documentclass[tikz]{standalone}
\usepackage{amsmath}
\usepackage{times}
\usepackage{txfonts}
\usepackage{pgfplots}
\usepackage{csvsimple}
\usetikzlibrary{arrows,intersections,math,fadings,shadings}
\begin{document}
\begin{tikzpicture}



\draw (-0.5,-1.5) node[rotate=90]{Level 0};
\draw (-0.5,-4.5) node[rotate=90]{Level 1};
\draw (-0.5,-7.5) node[rotate=90]{Level 2};
\draw (-0.5,-10.5) node[rotate=90]{Level 3};



\draw (15.5,0.2) node{Basen};

\foreach \x in {0,-3,...,-11}{
	
	\fill [blue!100!white] (14.8,\x-0.5) rectangle (16.5,\x);
	
	\fill [blue!80!white] (14.8,\x-1) rectangle (16.5,\x-0.5);
	
	\fill [blue!60!white] (14.8,\x-1.5) rectangle (16.5,\x-1);
	
	\fill [blue!40!white] (14.8,\x-2) rectangle (16.5,\x-1.5);
	
	\fill [blue!20!white] (14.8,\x-2.5) rectangle (16.5,\x-2);
	
	\fill [blue!10!white] (14.8,\x-3) rectangle (16.5,\x-2.5);
	
}
	\draw (15.5,-0.25) node{$fs\cdot2^{\frac{5}{6}}$};
	\draw (15.5,-0.75) node{$fs\cdot2^{\frac{4}{6}}$};
	\draw (15.5,-1.25) node{$fs\cdot2^{\frac{3}{6}}$};
	\draw (15.5,-1.75) node{$fs\cdot2^{\frac{2}{6}}$};
	\draw (15.5,-2.25) node{$fs\cdot2^{\frac{1}{6}}$};
	\draw (15.5,-2.75) node{$fs\cdot2^{\frac{0}{6}}$};
\draw[ultra thick] (0,0)--(0,-12)--(15,-12)--(15,0);

\foreach \x in {0,-3,...,-12}
\draw[ultra thick] (0,\x)--(15,\x);



\foreach \x in {0,-0.5,...,-12}
\draw[thick] (0,\x)--(15,\x);

\foreach \k / \y in {98/-0.25, 88/-0.75, 78/-1.25, 70/-1.75, 62/-2.25, 55/-2.75}{
	\foreach \x in {0.5,(0.5+14/\k),...,14.5}
	\fill({\x},\y) circle[radius=0.08];
}


\foreach \k / \y in {49/-3.25, 44/-3.75, 39/-4.25, 35/-4.75, 31/-5.25, 28/-5.75}{
	\foreach \x in {0.5,(0.5+14/\k),...,14.5}
	\fill({\x},\y) circle[radius=0.08];
}

\foreach \k / \y in {25/-6.25, 22/-6.75, 20/-7.25, 18/-7.75, 16/-8.25, 14/-8.75}{
	\foreach \x in {0.5,(0.5+14/\k),...,14.5}
	\fill({\x},\y) circle[radius=0.08];
}

\foreach \k / \y in {13/-9.25, 11/-9.75, 10/-10.25, 9/-10.75, 8/-11.25, 7/-11.75}{
	\foreach \x in {0.5,(0.5+14/\k),...,14.5}
	\fill({\x},\y) circle[radius=0.08];
}

\end{tikzpicture}
\end{document}

