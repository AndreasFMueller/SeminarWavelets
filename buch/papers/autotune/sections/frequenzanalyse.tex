
\subsection{Stetige Wavelet-Transformation}
Die Theoretischen Grundlagen rund um die Stetige Wavelet-Transformation wurde im Kapitel \ref{chapter:cwt} genaustens erläutert. 
In diesem Abschnitt der Seminararbeit wird öfters auf die Theorie des angesprochene Kapitel \ref{chapter:cwt} referenziert ohne diese weiter zu erläutern. 

\begin{equation}
\mathcal{W}f (a,b)
=
\langle f,\psi_{a,b}\rangle
=
\frac{1}{\sqrt{|a|}}\int_{-\infty}^\infty f(t)\,\overline{
	\psi\biggl(\frac{t-b}{a}\biggr)}\,dt
\label{eq:cwt}
\end{equation}

Mit der Stetigen Wavelet-Transformation können die Frequenzen einen 
\subsection{Diskrete Wavelet-Transformation}

\newpage
Folgend der essenzielle Ausschnitt aus dem Python Code:
\begin{figure}[h]
	\centering
	\lstinputlisting[language=Python,firstline=1,lastline=49,numbers=left,style = mystyle]{papers/autotune/code/mdwt.py}
	\caption{Python Codeausschnitt}
	\label{fig:python-code}
\end{figure}
\newpage