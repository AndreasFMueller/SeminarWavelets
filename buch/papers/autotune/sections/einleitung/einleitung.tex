Autotune ist in der heutigen Musikbranche nicht mehr wegzudenken. Man singt in ein Mikrofon, die gesungenen Töne werden analysiert und nach Belieben angepasst. Diese Form von Signalanalyse findet natürlich nicht nur in der Musikbranche Verwendung, sondern überall wo Frequenzen gemessen, eingeordnet und manipuliert werden sollen. In den meisten Fällen wird dies mit einer gefensterten Fouriertransformation gemacht. Dieses Kapitel untersucht inwiefern dies durch Verwendung von Wavelets ersetzt werden kann.\\

Um meinen Ansatz besser zu verstehen, werden zuerst die gängigen Methoden der gefensterten Fouriertransformation erklärt. Danach wird aufgezeigt welche gemeinsamkeiten diese mit den Wavelets teilen. Dann wird die Diskrete Wavelet-Transformation und ihre Eigenschaften zur frequenzanalyse betrachtet. Der Hauptteil besteht aus dem Framing, ein Begriff der noch eingeführt wird, und welche Resultate damit erreicht werden konnten. 