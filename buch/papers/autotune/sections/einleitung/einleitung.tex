Autotune ist in der heutigen Musikbranche nicht mehr wegzudenken. Nachdem man in ein Mikrofon singt, werden die gesungenen Töne analysiert und nach Belieben angepasst. Diese Form von Signalanalyse findet nicht nur in der Musikbranche Verwendung, sondern überall wo Frequenzen gemessen, eingeordnet und manipuliert werden sollen. In den meisten Fällen wird dies mit einer gefensterten Fouriertransformation gemacht. Dieses Kapitel untersucht inwiefern dieses Vorgehen der Tonhöheanalyse durch Verwendung von Wavelets ersetzt werden kann.\\

Um den im Paper verfolgten Ansatz besser zu verstehen, werden zuerst die gängigen Methoden der gefensterten Fouriertransformation erklärt. Danach wird aufgezeigt welche Gemeinsamkeiten diese mit den Wavelets teilen. Dann wird die diskrete Wavelet-Transformation und ihre Eigenschaften zur Frequenzanalyse betrachtet. Der Hauptteil dieses Papers besteht aus dem Framing, ein Begriff der noch eingeführt wird, und welche Resultate mit den Frames erreicht werden konnten. 