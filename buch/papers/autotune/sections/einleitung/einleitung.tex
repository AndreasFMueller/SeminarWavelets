Autotune ist in der heutigen Musikbranche nicht mehr wegzudenken. Man singt in ein Microphon, die gesungenen Töne werden analysiert und nach Belieben angepasst. Diese Form von Signalanalyse findet natürlich nicht nur in der Musikbranche Verwendung sondern überall wo Frequenzen gemessen, eingeordnet und manipuliert werden sollen. In den meisten Fällen wird dies mit einer gefensterten Fouriertransformation gemacht. Dieses Kapitel untersucht in wie fern dies durch verwendung von Wavelets ersetzt werden kann.\\
Um meinen Ansatz besser zu verstehen werden die gängigen Methoden der gefensterten Fouriertransformation erklärt und welche gemeinsamkeiten dies mit den Wavelets teilen. Danach wird die Diskrete Wavelet-Transformation und ihre Eigenschaften zur frequenzanalyse betrachtet. Der Hauptteil besteht aus dem Framing, ein Begriff der noch genauer eingeführt wird, und welche Eigenschaften erreicht werden konnten. 