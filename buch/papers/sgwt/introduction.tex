% !TeX spellcheck = de_CH_frami

F\"ur viele Probleme lassen sich nur schwer kontinuierliche Funktionen finden, 
die den vorhanden Sachverhalt korrekt wiedergeben. Meistens liegen Daten 
nur in diskreter Form vor und m\"ussen so weiterverarbeitet werden. Besonders 
die Fourier- und Wavelettheorie hat sich in der Praxis als zur 
Weiterverarbeitung geeignet herausgestellt. Daf\"ur werden aber jeweils 
passende Basisfunktionen ben\"otigt. Diese zu finden, ist aber meist keine 
triviale Angelegenheit.

F\"ur eindimensionale Funktionen $f: \mathbb{R} \rightarrow \mathbb{R}$ stellen 
die Exponentialfunktion $e^{i2\pi\omega x}$ eine m\"ogliche Basis dar, diese 
wird meist in der klassischen Fouriertheorie verwendet. Im 
zweidimensionalen Fall $f: \mathbb{R}^2 \rightarrow \mathbb{R}$ wie 
beispielsweise einem Bild, wird oft einfach jeweils eine eindimensionale 
Transformation f\"ur jeweils eine Dimension vorgenommen, man erh\"alt dann 
Basisfunktionen der Form $e^{i2\pi(ux+uy)}$. In drei Dimensionen $f: 
\mathbb{R}^3 \rightarrow \mathbb{R}$ wie zum Beispiel Kugeloberfl\"achen, gibt 
es die Kugelfunktionen $Y^m_l$, welche eine Basis darstellen.

Was ist nun aber mit Funktionen die nicht auf einfachen K\"orperoberfl\"achen 
liegen, zum Beispiel Funktionen auf Osterhasen, oder bei solchen f\"ur die der 
darunter liegende K\"orper keine wesentliche Rolle spielt, wie bei einem 
Routernetzerwerk, einem Strassennetz oder generell einem Graphen? In diesem 
Abschnitt wollen wir deshalb eine Theorie entwickeln, wie wir die bereits 
bekannte Fourier- und Wavelettheorie auch auf Graphen anwenden k\"onnen, bei 
dem die Funktion auf seine Knoten abgebildet wird. Diese Theorie ist bekannt 
als Spectral Graph Wavelet Transform oder kurz SGWT - nicht zu verwechseln mit 
der Second Generation Wavelet Transform~\cite{noauthor_second-generation_2018} 
- die erstmals 2009 in ``Wavelets on Graphs via Spectral Graph 
Theory''~\cite{hammond_wavelets_2009} 
vorgestellt wurde.
