% !TeX spellcheck = de_CH_frami

\section{Graph Spektralanalyse\label{sec:sgwt:spectralanalysis}}
\rhead{Graph Spektralanalyse}

Nachdem wir nun die Grundlagen von Graphen und den Laplace Operator sowie die 
Laplace Matrix angeschaut haben, wollen wir uns nun der Spektralanalyse eines 
Graphen widmen.

\subsection{Laplace Matrix Eigenwerte}

Wenn man vom Spektrum eines Graphen spricht, meint man damit die Eigenwerte 
$\lambda$ seiner Laplace Matrix. Da diese Matrix bei einem ungerichteten 
Graphen immer symmetrisch ist, werden die daraus folgenden Eigenwerte und 
Eigenfunktionen auch immer reell und gr\"osser gleich $0$ sein. Zudem wollen wir 
uns hier auf zusammenh\"angende Graphen beschr\"anken, also Graphen, bei denen jeder 
Knoten von jedem anderen Knoten aus mittels eines Sets von Kanten erreicht 
werden kann. Dies stellt sicher, dass genau ein Eigenwert $0$ 
ist. Es gilt somit
\begin{equation}
0 = \lambda_0 < \lambda_1 \leq \lambda_2 \leq \cdots \leq \lambda_{N-1}.
\label{eq:sgwt:lambda:series}
\end{equation}
Die berechneten Eigenwerte eines Ringgraphen mit $1000$ Knoten 
in~\cref{fig:sgwt:lambda:line} zeigt beispielhaft, wie diese aussehen k\"onnen.
\begin{figure}
    \centering
    \includegraphics[
    angle=-90,
    origin=c,
    scale=0.5
    ]{papers/sgwt/images/ring-lambda.pdf}
    \vspace{-45pt}
    \caption{Eigenwerte $\lambda_0$ bis $\lambda_{999}$ eines ungewichteten Ringraphen.
        \label{fig:sgwt:lambda:line}}
\end{figure}

\subsection{Laplace Matrix Eigenvektoren}

Eine weitere wichtige Rolle spielen die Eigenvektoren $\chi$ der Laplace 
Matrix. Diese sind n\"amlich approximierte Eigenfunktionen des Laplace 
Operators $\Delta$. So ist die komplexe Exponentialfunktion $e^{i\omega x}$ die 
Eigenfunktion des eindimensionalen Laplace Operators 
$\frac{d^2}{dx^2}$~\cite{chung_spectral_nodate}.

Auch die Kugelfunktionen $Y^m_l$, lassen mit den Hilfe der 
Eigenfunktionen eines Kugelgraphen approximieren, wie man 
in~\cref{fig:sgwt:chi:sphere} sehen kann. Dabei wird aber auch deutlich, dass 
diese Approximation nicht \"uberall gleich gut ist. Insbesondere dadurch, dass 
der Grad an den Polen stark von den anderen Knoten abweicht, findet dort eine 
Verzerrung statt.
\begin{figure}
    \centering
    \includegraphics[
    angle=-90,
    origin=c,
    scale=0.7
    ]{papers/sgwt/images/graph-100-100-chi-150.pdf}
    \vspace{-80pt}
    \caption{Kugelgraph mit 10002 Knoten. Darstellung des Eigenvektors 
    $\chi_{150}$.
    \label{fig:sgwt:chi:sphere}}
\end{figure}

Wie schon der Eigenwert $\lambda_0$, der gem\"ass unserer Definition immer $0$ 
ist, ist auch der dazugeh\"orende Eigenvektor $\chi_0$ etwas spezieller. Bei 
$\chi_0$ handelt es sich n\"amlich um die konstante Funktion. Die insbesondere 
f\"ur die Rekonstruktion der Funktion eine wichtige Rolle spielt.

\subsubsection{Eigenvektoren: Ringgraph}
Die Eigenfunktionen des Ringgraphen, zu sehen 
in~\cref{fig:sgwt:chi:ring0,fig:sgwt:chi:ring1,fig:sgwt:chi:ring2,fig:sgwt:chi:ring3,fig:sgwt:chi:ring4,fig:sgwt:chi:ring5},
treten jeweils paarweise auf. Die Funktionen sind dabei jeweils um 
$\frac{\pi}{2}$ phasenverschoben. Es handelt sich also um eine 
Approximation der auch in der Fouriertheorie verwendeten periodischen Sinus und 
Kosinus Funktionen.
\begin{figure}
    \centering
    \begin{minipage}[t]{0.49\textwidth}
        \includegraphics[
        angle=-90,
        origin=c,
        width=\textwidth]{papers/sgwt/images/ring-chi-1.pdf}
        \vspace{-45pt}
        \caption{$\chi_0$ eines ungewichteten Ringraphen mit 1000 Knoten.}
        \label{fig:sgwt:chi:ring0}
    \end{minipage}
    ~
    \begin{minipage}[t]{0.49\textwidth}
        \includegraphics[
        angle=-90,
        origin=c,
        width=\textwidth]{papers/sgwt/images/ring-chi-2.pdf}
        \vspace{-45pt}
        \caption{$\chi_1$ eines ungewichteten Ringraphen mit 1000 Knoten.}
        \label{fig:sgwt:chi:ring1}
    \end{minipage}
    ~
    \begin{minipage}[t]{0.49\textwidth}
        \includegraphics[
        angle=-90,
        origin=c,
        width=\textwidth]{papers/sgwt/images/ring-chi-3.pdf}
        \vspace{-45pt}
        \caption{$\chi_2$ eines ungewichteten Ringraphen mit 1000 Knoten.}
        \label{fig:sgwt:chi:ring2}
    \end{minipage}
    ~
    \begin{minipage}[t]{0.49\textwidth}
        \includegraphics[
        angle=-90,
        origin=c,
        width=\textwidth]{papers/sgwt/images/ring-chi-4.pdf}
        \vspace{-45pt}
        \caption{$\chi_3$ eines ungewichteten Ringraphen mit 1000 Knoten.}
        \label{fig:sgwt:chi:ring3}
    \end{minipage}
    ~
    \begin{minipage}[t]{0.49\textwidth}
        \includegraphics[
        angle=-90,
        origin=c,
        width=\textwidth]{papers/sgwt/images/ring-chi-5.pdf}
        \vspace{-45pt}
        \caption{$\chi_4$ eines ungewichteten Ringraphen mit 1000 Knoten.}
        \label{fig:sgwt:chi:ring4}
    \end{minipage}
    ~
    \begin{minipage}[t]{0.49\textwidth}
        \includegraphics[
        angle=-90,
        origin=c,
        width=\textwidth]{papers/sgwt/images/ring-chi-6.pdf}
        \vspace{-45pt}
        \caption{$\chi_5$ eines ungewichteten Ringraphen mit 1000 Knoten.}
        \label{fig:sgwt:chi:ring5}
    \end{minipage}
\end{figure}

\subsubsection{Eigenvektoren: Streckengraph}
Der Streckengraph ist beinahe identisch mit dem Ringgraph. Der Unterschied ist 
lediglich die fehlende Kante zwischen zweier Knoten. Dies hat allerdings 
gravierende Auswirkungen, wie wir die Eigenvektoren, zu sehen 
in~\cref{fig:sgwt:chi:line0,fig:sgwt:chi:line1,fig:sgwt:chi:line2,fig:sgwt:chi:line3,fig:sgwt:chi:line4,fig:sgwt:chi:line5},
betrachten. Es zeigt sich, dass die Funktionen nicht mehr paarweise auftreten, 
stattdessen finden sich hier Funktionen mit halber Frequenz. Weiter gelten 
f\"ur die Funktionen des Streckengraphen jeweils homogene 
Neumann-Randbedingungen, das heisst, die Ableitung an an den 
jeweiligen Funktionsenden ist $0$.
\begin{figure}
    \centering
    \begin{minipage}[t]{0.49\textwidth}
        \includegraphics[
        angle=-90,
        origin=c,
        width=\textwidth]{papers/sgwt/images/line-chi-1.pdf}
        \vspace{-45pt}
        \caption{$\chi_0$ eines ungewichteten Streckengraphen mit 1000 
        Knoten.}
        \label{fig:sgwt:chi:line0}
    \end{minipage}
    ~
    \begin{minipage}[t]{0.49\textwidth}
        \includegraphics[
        angle=-90,
        origin=c,
        width=\textwidth]{papers/sgwt/images/line-chi-2.pdf}
        \vspace{-45pt}
        \caption{$\chi_1$ eines ungewichteten Streckengraphen mit 1000 
        Knoten.}
        \label{fig:sgwt:chi:line1}
    \end{minipage}
    ~
    \begin{minipage}[t]{0.49\textwidth}
        \includegraphics[
        angle=-90,
        origin=c,
        width=\textwidth]{papers/sgwt/images/line-chi-3.pdf}
        \vspace{-45pt}
        \caption{$\chi_2$ eines ungewichteten Streckengraphen mit 1000 
        Knoten.}
        \label{fig:sgwt:chi:line2}
    \end{minipage}
    ~
    \begin{minipage}[t]{0.49\textwidth}
        \includegraphics[
        angle=-90,
        origin=c,
        width=\textwidth]{papers/sgwt/images/line-chi-4.pdf}
        \vspace{-45pt}
        \caption{$\chi_3$ eines ungewichteten Streckengraphen mit 1000 
        Knoten.}
        \label{fig:sgwt:chi:line3}
    \end{minipage}
    ~
    \begin{minipage}[t]{0.49\textwidth}
        \includegraphics[
        angle=-90,
        origin=c,
        width=\textwidth]{papers/sgwt/images/line-chi-5.pdf}
        \vspace{-45pt}
        \caption{$\chi_4$ eines ungewichteten Streckengraphen mit 1000 
        Knoten.}
        \label{fig:sgwt:chi:line4}
    \end{minipage}
    ~
    \begin{minipage}[t]{0.49\textwidth}
        \includegraphics[
        angle=-90,
        origin=c,
        width=\textwidth]{papers/sgwt/images/line-chi-6.pdf}
        \vspace{-45pt}
        \caption{$\chi_5$ eines ungewichteten Streckengraphen mit 1000 
        Knoten.}
        \label{fig:sgwt:chi:line5}
    \end{minipage}
\end{figure}
