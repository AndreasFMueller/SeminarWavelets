% !TeX spellcheck = de_CH_frami

\section{Zusammenfassung\label{sec:sgwt:summary}}
\rhead{Zusammenfassung}

Mit der Spektral Graph Wavelet Transformation ist es uns gelungen, die bekannte 
Wavelettheorie auch auf Graphen anzuwenden. Damit k\"onnen wir uns von der 
Geometrie der Oberfl\"ache einer Funktion l\"osen und betrachten nur noch, was 
in der Nachbarschaft der einzelnen Knoten geschieht.

Um nun eine Funktion $f(v)$ auf einem Graphen $G$ zu analysieren, k\"onnen wir 
also wie folgt vorgehen
\begin{itemize}
    \item[1.] Generierung der \laplaceL{} Matrix aus dem Graphen $G$.
    \item[2.] Berechnung der Eigenwerte $\lambda$ und Eigenvektoren $\chi$.
    \item[3.] Berechnung der $\psi_j$ und $\phi$ Wavelets 
    mit~\cref{eq:sgwt:psi,eq:sgwt:phi}.
    \item[4.] Berechnung der Wavelet Koeffizienten $\hat{f}$ 
    mit~\cref{eq:sgwt:hatf}.
\end{itemize}
Danach k\"onnen wir die Funktion $f$ mittels~\cref{eq:sgwt:pseudof} wieder 
synthetisieren, indem wir das Pseudoinverse der $T$-Matrix verwenden.

Auch wenn die SGWT, im Vergleich zu zur Wavelet- oder gar Fouriertheorie noch 
ziemlich neu ist, scheinen die M\"oglichkeiten \"ausserst weitl\"aufig zu sein, 
wie Beispiele in der in der Bildbearbeitung~\cite{shuman_emerging_2013} oder 
bei der Verwendung von neuronalen Netzwerken~\cite{xu_graph_2019} zeigen.

