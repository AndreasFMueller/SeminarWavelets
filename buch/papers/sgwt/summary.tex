% !TeX spellcheck = de_CH_frami

\section{Zusammenfassung\label{sec:sgwt:summary}}
\rhead{Zusammenfassung}

Mit der Spektral Graph Wavelet Transformation ist es uns gelungen, die bekannte 
Wavelettheorie auch auf Graphen anzuwenden. Damit k\"onnen wir uns von der 
Geometrie der Oberfl\"ache einer Funktion l\"osen und betrachten nur noch, was 
in der Nachbarschaft der einzelnen Knoten geschieht. Auch wenn die SGWT, im 
Vergleich zu zur Wavelet- oder gar Fouriertheorie noch ziemlich neu ist, 
scheinen die M\"oglichkeiten \"ausserst weitl\"aufig zu sein, wie Beispiele in 
der in der Bildbearbeitung~\cite{shuman_emerging_2013} oder bei der Verwendung 
von neuronalen Netzwerken~\cite{xu_graph_2019} zeigen.

