%
% main.tex -- Paper zum Thema SGWT
%
% (c) 2019 Hochschule Rapperswil
%
\chapter{Spectral Graph Wavelet Transform\label{chapter:sgwt}}
\lhead{SGWT}
\begin{refsection}
\chapterauthor{Hansruedi Patzen}

\newcommand*{\laplaceL}{\ensuremath{\mathcal{L}}}
\newcommand*{\laplaceLnorm}{\ensuremath{\mathcal{L}^{\text{norm}}}}
\newcommand*{\degreeM}{\ensuremath{D}}
\newcommand*{\adjacencyM}{\ensuremath{A}}
\newcommand*{\diag}[1]{\ensuremath{\text{diag}(#1)}}


\section{Einf\"urung}
\rhead{Einf\"urung}

Bisher haben wir uns mit Wavelets auf kontinuierlichen Funktionen 
besch\"aftigt. Mit der Spectral Graph Wavelet Transform oder kurz SGWT - 
nicht zu verwechseln mit der Second Generation Wavelet 
Transform~\cite{noauthor_second-generation_2018} - wollen wir uns nun 
von der Geometrie, auf der sich unsere Funktion $f$ befindet, l\"osen. Konkret 
bedeutet das, dass wir die Geometrie nehmen und ein Netz dar\"uber spannen. 
Dieses Netz ist dann unser Graph den wir analysieren, eventuell manipulieren 
und wieder synthetisieren wollen.

\section{Grundlagen Graphentheorie\label{sec:sgwt:graphs}}

Die Laplace Matrix 
\laplaceL~\cite{noauthor_laplace-matrix_2017}, welche wir aus der 
Gradmatrix \degreeM~\cite{noauthor_degree_2018} und der Adjazenzmatrix 
\adjacencyM~\cite{noauthor_adjacency_2019} berechnen k\"onnen, spielt bei der 
SGWT eine entscheidende Rolle. Dabei gibt es 
die Unterscheidung zwischen der Laplace Matrix aus \cref{eq:sgwt:laplace} und 
\cref{eq:sgwt:laplace:norm}, bei der \laplaceLnorm{} Matrix handelt es sich um 
die normalisierte \laplaceL{} Matrix.

% TODO Example Degree matrix
% TODO Example Adjacency matrix
% TODO Example Laplace
% TODO Example normalized Laplace

\begin{equation}
\laplaceL = \degreeM - \adjacencyM
\label{eq:sgwt:laplace}
\end{equation}

\begin{equation}
\laplaceLnorm
= \degreeM^{-1/2}\laplaceL\degreeM^{-1/2}
= I - \degreeM^{-1/2}\adjacencyM\degreeM^{-1/2}
\label{eq:sgwt:laplace:norm}
\end{equation}



\section{Eigenfunktionen von Graphen\label{sec:sgwt:eigenfunction}}
\rhead{Eigenfunktionen von Graphen}

\section{Kugelgraphen\label{sec:sgwt:spheregraph}}
\rhead{Kugelgraphen}

\section{Schlussfolgerung}
\rhead{Schlussfolgerung}

\begin{equation}
\chi = 
\left[
\begin{pmatrix}\\\chi_0\\\\\end{pmatrix}
\begin{pmatrix}\\\chi_1\\\\\end{pmatrix}
\begin{pmatrix}\\\chi_2\\\\\end{pmatrix}
\cdots
\begin{pmatrix}\\\chi_{N-1}\\\\\end{pmatrix}
\right]
\end{equation}

\begin{equation}
0 = \lambda_0 < \lambda_1 \le \lambda_2 \le \cdots \le \lambda_{N-1}
\end{equation}

\begin{equation}
    F_{i,j}(f) = \begin{cases}
    f, & \text{if } i == j \\
    0, & \text{sonst}
    \end{cases}
\end{equation}

\begin{equation}
\psi_j = \chi F(g(\lambda)) \chi'
\end{equation}

Spektral Graph Wavelets Transformationen 


% TODO: Remove in final version. replaces \nocite{*}.
\cite{hammond_wavelets_2011}
\cite{hammond_image_nodate}
\cite{shuman_emerging_2013}
\cite{xu_graph_2019}
\cite{chung_spectral_nodate}
\cite{spielman_spectral_nodate}
\cite{nica_brief_2018}
\cite{marsden_eigenvalues_nodate}


\printbibliography[heading=subbibliography]
\end{refsection}
