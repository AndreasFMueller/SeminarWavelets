%
% main.tex -- Paper zum Thema SGWT
%
% (c) 2019 Hochschule Rapperswil
%
\chapter{Spectral Graph Wavelet Transform\label{chapter:sgwt}}
\lhead{SGWT}
\begin{refsection}
\chapterauthor{Hansruedi Patzen}

\newcommand*{\laplaceL}{\ensuremath{\mathcal{L}}}
\newcommand*{\laplaceLnorm}{\ensuremath{\mathcal{L}^{\text{norm}}}}
\newcommand*{\degreeM}{\ensuremath{D}}
\newcommand*{\adjacencyM}{\ensuremath{A}}
\newcommand*{\diag}[1]{\ensuremath{\text{diag}(#1)}}


\section{Einf\"urung}
\rhead{Einf\"urung}

%TODO: Rework
Mit der Spectral Graph Wavelet Transform oder kurz SGWT - 
nicht zu verwechseln mit der Second Generation Wavelet 
Transform~\cite{noauthor_second-generation_2018} - wollen wir die bekannte 
Wavelet Theorie erweitern, sodass deren Konzepte auf gewichtet und ungerichtete 
Graphen anwendbar ist.

Daf\"ur starten in \cref{sec:sgwt:graphs} wir mit einer kleinen \"Ubersicht, 
was genau unter einem 
solchen Graphen verstanden wird. F\"ur die Graphen Wavelets spielt die 
Laplacematrix \laplaceL{} eine bedeutende Rolle und wird in 
\cref{subsec:sgwt:laplaceop} behandelt.

\section{Graphen\label{sec:sgwt:graphs}}
\rhead{Graphen}

Ein Graph $G$ setzt sich aus Kanten $E$ und Knoten $V$ zusammen.

\begin{equation}
G = \{V, E\}
\end{equation}

Eine Kannte verbindet dabei jeweils zwei Knoten miteinander.

\begin{equation}
E = (V_1, V_2)
\end{equation}


\section{Laplace\label{sec:sgwt:laplace}}
\rhead{Laplace}

\subsection{Laplace Operator\label{subsec:sgwt:laplaceop}}
\rhead{Laplace Operator}

Der Laplace operator $\Delta$ in $\mathbb{R}^n$ beschreibt die zweite Ableitung.

\begin{equation}
\Delta = \frac{\partial^2}{\partial x_1^2}
+ \frac{\partial^2}{\partial x_2^2}
+ \dots
+ \frac{\partial^2}{\partial x_n^2}
\end{equation}

\subsection{Laplace Matrix\label{subsec:sgwt:laplacm}}
\rhead{Laplace Matrix}

%TODO: Rework
Die Laplace Matrix 
\laplaceL~\cite{noauthor_laplace-matrix_2017}, welche wir aus der 
Gradmatrix \degreeM~\cite{noauthor_degree_2018} und der Adjazenzmatrix 
\adjacencyM~\cite{noauthor_adjacency_2019} berechnen k\"onnen, spielt bei der 
SGWT eine entscheidende Rolle. Dabei gibt es 
die Unterscheidung zwischen der Laplace Matrix aus \cref{eq:sgwt:laplace} und 
\cref{eq:sgwt:laplace:norm}, bei der \laplaceLnorm{} Matrix handelt es sich um 
die normalisierte \laplaceL{} Matrix.

\begin{equation}
\laplaceL = \degreeM - \adjacencyM
\label{eq:sgwt:laplace}
\end{equation}

\begin{equation}
\laplaceLnorm
= \degreeM^{-1/2}\laplaceL\degreeM^{-1/2}
= I - \degreeM^{-1/2}\adjacencyM\degreeM^{-1/2}
\label{eq:sgwt:laplace:norm}
\end{equation}

\begin{equation}
\chi = 
\left[
\begin{pmatrix}\\\chi_0\\\\\end{pmatrix}
\begin{pmatrix}\\\chi_1\\\\\end{pmatrix}
\begin{pmatrix}\\\chi_2\\\\\end{pmatrix}
\cdots
\begin{pmatrix}\\\chi_{N-1}\\\\\end{pmatrix}
\right]
\end{equation}

\begin{equation}
0 = \lambda_0 < \lambda_1 \le \lambda_2 \le \cdots \le \lambda_{N-1}
\end{equation}

\section{Spektralanalyse}
\rhead{Spektralanalyse}

\subsection{Graph Fourier Transformation\label{subsec:sgwt:gft}}

\subsection{Graph Wavelet Transformation\label{subsec:sgwt:gwt}}

\begin{equation}
\psi_j = \chi \diag{g(\lambda)} \chi'
\end{equation}

\subsection{Kernel-Functionen}

\subsubsection{Bandpassfilter \texorpdfstring{$g$}{g}}

\subsubsection{Tiefpassfilter \texorpdfstring{$h$}{h}}

\subsection{Frames}

\subsection{Analyse}

\subsection{Synthese}

\section{Zusammenfassung\label{sec:sgwt:summary}}


% TODO: Remove in final version. replaces \nocite{*}.
Literatur:
\cite{noauthor_second-generation_2018}
\cite{noauthor_laplace-matrix_2017}
\cite{noauthor_degree_2018}
\cite{noauthor_adjacency_2019}
\cite{hammond_wavelets_2011}
\cite{hammond_image_nodate}
\cite{shuman_emerging_2013}
\cite{xu_graph_2019}
\cite{chung_spectral_nodate}
\cite{spielman_spectral_nodate}
\cite{nica_brief_2018}
\cite{marsden_eigenvalues_nodate}


\printbibliography[heading=subbibliography]
\end{refsection}
