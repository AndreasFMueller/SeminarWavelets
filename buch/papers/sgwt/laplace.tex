% !TeX spellcheck = de_CH_frami

\section{Laplace\label{sec:sgwt:laplace}}
\rhead{Laplace}

\subsection{Laplace Operator\label{subsec:sgwt:laplaceop}}
\rhead{Laplace Operator}

Der Laplace operator $\Delta$ in $\mathbb{R}^n$ beschreibt die zweite Ableitung.

\begin{equation}
\Delta = \frac{\partial^2}{\partial x_1^2}
+ \frac{\partial^2}{\partial x_2^2}
+ \dots
+ \frac{\partial^2}{\partial x_n^2}
\end{equation}

\subsection{Laplace Matrix\label{subsec:sgwt:laplacm}}
\rhead{Laplace Matrix}

%TODO: Why could norm be usefull?

%TODO: Rework
Die Laplace Matrix 
\laplaceL~\cite{noauthor_laplace-matrix_2017}, welche wir aus der 
Gradmatrix \degreeM~\cite{noauthor_degree_2018} und der Adjazenzmatrix 
\adjacencyM~\cite{noauthor_adjacency_2019} berechnen k\"onnen, spielt bei der 
SGWT eine entscheidende Rolle. Dabei gibt es 
die Unterscheidung zwischen der Laplace Matrix aus \cref{eq:sgwt:laplace} und 
\cref{eq:sgwt:laplace:norm}, bei der \laplaceLnorm{} Matrix handelt es sich um 
die normalisierte \laplaceL{} Matrix.

\begin{equation}
\laplaceL = \degreeM - \adjacencyM
\label{eq:sgwt:laplace}
\end{equation}

\begin{equation}
\laplaceLnorm
= \degreeM^{-1/2}\laplaceL\degreeM^{-1/2}
= I - \degreeM^{-1/2}\adjacencyM\degreeM^{-1/2}
\label{eq:sgwt:laplace:norm}
\end{equation}

\begin{equation}
\chi = 
\left[
\begin{pmatrix}\\\chi_0\\\\\end{pmatrix}
\begin{pmatrix}\\\chi_1\\\\\end{pmatrix}
\begin{pmatrix}\\\chi_2\\\\\end{pmatrix}
\cdots
\begin{pmatrix}\\\chi_{N-1}\\\\\end{pmatrix}
\right]
\end{equation}

\begin{equation}
0 = \lambda_0 < \lambda_1 \le \lambda_2 \le \cdots \le \lambda_{N-1}
\end{equation}

\subsection{Beispiele}

Nehmen wir nun den Graphen aus \cref{fig:sgwt:sphere:graph}.

\begin{equation}
\laplaceL =
\begin{bmatrix}
3 & -1 & -1 & -1 & 0 & 0 & 0 & 0 \\
-1 &  4 & -1 & -1 & -1 & 0 & 0 & 0 \\
-1 & -1 &  4 & -1 & 0 & -1 & 0 & 0 \\
-1 & -1 & -1 &  4 & 0 & 0 & -1 & 0 \\
0 & -1 & 0 & 0 &  4 & -1 & -1 & -1 \\
0 & 0 & -1 & 0 & -1 &  4 & -1 & -1 \\
0 & 0 & 0 & -1 & -1 & -1 &  4 & -1 \\
0 & 0 & 0 & 0 & -1 & -1 & -1 &  3
\end{bmatrix}
\end{equation}

\begin{equation}
\degreeM =
\begin{bmatrix}
3 & 0 & 0 & 0 & 0 & 0 & 0 & 0 \\
0 & 4 & 0 & 0 & 0 & 0 & 0 & 0 \\
0 & 0 & 4 & 0 & 0 & 0 & 0 & 0 \\
0 & 0 & 0 & 4 & 0 & 0 & 0 & 0 \\
0 & 0 & 0 & 0 & 4 & 0 & 0 & 0 \\
0 & 0 & 0 & 0 & 0 & 4 & 0 & 0 \\
0 & 0 & 0 & 0 & 0 & 0 & 4 & 0 \\
0 & 0 & 0 & 0 & 0 & 0 & 0 & 3
\end{bmatrix}
\end{equation}

\begin{equation}
\adjacencyM =
\begin{bmatrix}
0 & 1 & 1 & 1 & 0 & 0 & 0 & 0 \\
1 & 0 & 1 & 1 & 1 & 0 & 0 & 0 \\
1 & 1 & 0 & 1 & 0 & 1 & 0 & 0 \\
1 & 1 & 1 & 0 & 0 & 0 & 1 & 0 \\
0 & 1 & 0 & 0 & 0 & 1 & 1 & 1 \\
0 & 0 & 1 & 0 & 1 & 0 & 1 & 1 \\
0 & 0 & 0 & 1 & 1 & 1 & 0 & 1 \\
0 & 0 & 0 & 0 & 1 & 1 & 1 & 0
\end{bmatrix}
\end{equation}

\begin{align}
\laplaceL&^{norm} = \nonumber\\
%&\qquad \laplaceLnorm =\nonumber\\
&\begin{bmatrix}
1 & -0.28868 & -0.28868 & -0.28868 & 0 & 0 & 0 & 0 \\
-0.28868 &  1 & -0.25000 & -0.25000 & -0.25000 & 0 & 0 & 0 \\
-0.28868 & -0.25000 &  1 & -0.25000 & 0 & -0.25000 & 0 & 0 \\
-0.28868 & -0.25000 & -0.25000 &  1 & 0 & 0 & -0.25000 & 0 \\
0 & -0.25000 & 0 & 0 &  1 & -0.25000 & -0.25000 & -0.28868 \\
0 & 0 & -0.25000 & 0 & -0.25000 &  1 & -0.25000 & -0.28868 \\
0 & 0 & 0 & -0.25000 & -0.25000 & -0.25000 &  1 & -0.28868 \\
0 & 0 & 0 & 0 & -0.28868 & -0.28868 & -0.28868 &  1
\end{bmatrix}
\end{align}
