% !TeX spellcheck = de_CH_frami

\section{Laplace\label{sec:sgwt:laplace}}
\rhead{Laplace}

F\"ur die Analyse und Synthese von Graphen und deren Spektren spielen der 
Laplace Operator und besonders die Laplace Matrix eine wichtige Rolle, wie 
bereits Graphen in~\cref{sec:sgwt:graphs}, wollen wir daher zuerst 
die Grundlagen des Laplace Operators in~\cref{subsec:sgwt:laplaceop} und der 
Laplace Matrix \laplaceL{} in~\cref{subsec:sgwt:laplacem} eingehen, bevor wir 
dann in~\cref{sec:sgwt:spectralanalysis} mit der Analyse des Spektrums und 
Konstruktion der SGWT beginnen.

\subsection{Laplace Operator\label{subsec:sgwt:laplaceop}}
\rhead{Laplace Operator}

Der Laplace Operator $\Delta$ in $\mathbb{R}^n$ wird im kartesischen 
Koordinatensystem durch die Summe der $n$ zweiten Ableitungen
\begin{equation*}
\Delta = 
\sum_{i = 1}^{n}\frac{\partial^2}{\partial x_i^2}
=
\frac{\partial^2}{\partial x_1^2}
+ \frac{\partial^2}{\partial x_2^2}
+ \dots
+ \frac{\partial^2}{\partial x_n^2}
\end{equation*}
beschrieben. H\"aufig trifft man erstmals auf den Laplace Operator wenn man 
versucht die Poisson-Gleichung $-\Delta u = f$ zu l\"osen, die im Falle von $f 
= 0$ zur homogenen Laplace-Gleichung $\Delta \phi = 0$ wird.

Wie Anfangs bereits geschrieben, sind unsere Daten meist nur in Diskreter Form 
vorhanden. Daher w\"are se gut auch eine diskretisierte oder zumindest 
approximierte Form des Laplace Operators zu haben. Eine M\"oglichkeit ist dabei 
die Finite-Differenzen-Methode.

\subsection{Finite-Differenzen-Methode}
\rhead{Finite-Differenzen-Methode}

Bei der Finite-Differenzen-Methode oder kurz FDM ist eine diskrete 
Approximation des Laplace Operators deren Idee es ist es, eine 
Differentialgleichung mittels Differenzenquotienten zu approximieren.

Als Ausgangslage nehmen wir die Funktion $u(x)$ mit dem eindimensionalen 
Definitionsbereich $u: \mathbb{R} \rightarrow \mathbb{R}$. Beginnen wir zuerst 
mit der Approximation der ersten Ableitung an der Stelle $x_i$, mit dem Abstand 
$h = \Delta x_i$,
\begin{equation*}
\frac{\partial u}{\partial x_i}
= \frac{u(x_i+h)-u(x_i)}{h}.
\end{equation*}
Indem wir diese Gleichung nochmals Ableiten erhalten wir die f\"ur den Laplace 
Operator relevante zweite Ableitung
\begin{equation*}
\frac{\partial}{\partial x_i}\frac{\partial}{\partial x_i}u
= \frac{\partial u}{\partial x_i^2}
= \frac{\frac{u(x_i+h)-u(x_i)}{h}-\frac{u(x_i)-u(x_i+h)}{h}}{h}
= \frac{u(x_i+h)-2u(x_i)+u(x_i-h)}{h^2}.
\end{equation*}

Wenn wir nun Funktionen $u(x, y)$ des zweidimensionalen Definitionsbereichs $u: 
\mathbb{R}^2 \rightarrow \mathbb{R}$ in einem kartesischen Koordinatensystem 
zusammen mit dem Laplace Operator $\Delta$ anschauen, erhalten wir
\begin{equation*}
\Delta u(x, y) = u_{xx}(x, y) + u_{yy}(x, y).
\end{equation*}
Dies k\"onnen wir wiederum mittels Finiten-Differenzen approximieren, mit $h = 
\Delta x = \Delta y$ liefert uns das
\begin{align}
\Delta u(x,y)
&=
\frac{u(x+h,y)-2u(x, y)+u(x-h,y)}{h^2}
+\frac{u(x,y+h)-2u(x, y)+u(x,y-h)}{h^2} \nonumber \\
&=
\frac{u(x+h,y)+u(x-h,y)+u(x,y+h)+u(x,y-h)-4u(x, y)}{h^2}.
\label{eq:sgwt:fivepointstencil}
\end{align}
Dieser Operator ist auch bekannt als der F\"unfpunkte-Stern-Operator 
oder five-point stencil im Englischen, siehe~\cref{fig:sgwt:graph:star}.
\begin{figure}
\centering
\includegraphics[
angle=-90,
origin=c,
scale=0.6
]{papers/sgwt/images/star.pdf}
\vspace{0pt}
\caption{F\"unfpunkte-Stern-Operator mit $\text{P} = u(x,y), \text{P}_1 = 
u(x-h,y), \text{P}_2 = u(x,y+h), \text{P}_3 = u(x+h,y) \text{und P}_4 = 
u(x,y-h)$.
\label{fig:sgwt:graph:star}}
\end{figure}

Es folgt daraus, dass bei der zweiten Ableitung die ``aufeinanderfolgenden 
Differenzen'' immer den Knoten in der MItte gemeinsam haben. Daraus k\"onnen 
wir nun wieder zur\"uck auf unsere Graphen 
in~\cref{fig:sgwt:graph:simple} aus aus~\cref{sec:sgwt:graphs} kommen. Wenn wir 
diesen n\"amlich mit dem den F\"unfpunkte-Stern-Operator 
aus~\cref{fig:sgwt:graph:star} vergleichen wird klar, dass der Graph, obwohl er 
auf den ersten Blick anders aussieht, der Gleiche ist. Wenn wir also den 
Laplace Operator auf den jeweiligen Graph anwenden, werden wir das gleiche 
Resultat erhalten.

Wir k\"onnen somit noch einen Schritt weitergehen und uns von den Achsen 
l\"osen. Nehmen wir als Beispiel den Graphen aus~\cref{fig:sgwt:laplace:nstar}. 
F\"ur die zweite Ableitung beim Punkt P brauchen wir also nach dem Schema 
aus~\cref{eq:sgwt:fivepointstencil} die Summe aller Knoten P1 bis P8 
abz\"uglich Anzahl Nachbarn von P, also dem Grad $d(\text{P})$, multipliziert 
mit dem 
Knoten P, was uns folgende Gleichung
\begin{equation*}
\Delta u = \frac{1}{h^2}\left(\sum_{i = 1}^{8}\text{P}_i - 8P\right)
\end{equation*}
liefert. Generell erhalten wir f\"ur den Laplace Operator an einem Knoten P mit 
$n$ Nachbarn die folgende~\cref{eq:sgwt:generallaplace}.
\begin{equation}
\Delta u = \frac{1}{h^2}\left(\sum_{i = 1}^{n}\text{P}_i - nP\right)
\label{eq:sgwt:generallaplace}
\end{equation}

\begin{figure}
\centering
\includegraphics[
angle=-90,
origin=c,
scale=0.7
]{papers/sgwt/images/nstar.pdf}
\vspace{0pt}
\caption{Ein Graph mit neun Knoten und acht Kanten. Der Knote P mit Grad 
$d(\text{P}) = 8$ stich hier klar gegen\"uber den anderen Knoten mit jeweils 
Grad $d(\text{P}_i) = 1$ hervor.
    \label{fig:sgwt:laplace:nstar}}
\end{figure}

\subsection{Laplace Matrix\label{subsec:sgwt:laplacem}}
\rhead{Laplace Matrix}

Die Laplace Matrix \laplaceL{}~\cite{noauthor_laplace-matrix_2017} beschreibt 
einen Graphen anhand seiner Grad- \degreeM{}~\cite{noauthor_degree_2018} 
und Adjazenzmatrix \adjacencyM{}~\cite{noauthor_adjacency_2019} wie folgt
\begin{equation}
\laplaceL = \degreeM - \adjacencyM.
\label{eq:sgwt:laplace}
\end{equation}
Die Gradmatrix \degreeM{} ist dabei eine Diagonalmatrix, die den Grad der 
einzelnen Knoten beinhaltet. Die Adjazenzmatrix hingegen zeigt die Kanten und 
deren Gewichte des Graphen auf. Als Beispiel dient uns 
wieder der Graph aus~\cref{fig:sgwt:graph:simple}, dessen Grad- und 
Adjazenzmatrix wie folgt
\begin{equation*}
\degreeM =
\begin{bmatrix}
1 & 0 & 0 & 0 & 0 \\
0 & 4 & 0 & 0 & 0 \\
0 & 0 & 1 & 0 & 0 \\
0 & 0 & 0 & 1 & 0 \\
0 & 0 & 0 & 0 & 1
\end{bmatrix}
\end{equation*}
sowie
\begin{equation*}
\adjacencyM =
\begin{bmatrix}
0 & 1 & 0 & 0 & 0 \\
1 & 0 & 1 & 1 & 1 \\
0 & 1 & 0 & 0 & 0 \\
0 & 1 & 0 & 0 & 0 \\
0 & 1 & 0 & 0 & 0
\end{bmatrix}
\end{equation*}
aussehen. Die Differenz der beiden ergibt dann die Laplace Matrix
\begin{equation*}
\laplaceL =
\begin{bmatrix}
1 & -1 & 0 & 0 & 0 \\
-1 & 4 & -1 & -1 & -1 \\
0 & -1 & 1 & 0 & 0 \\
0 & -1 & 0 & 1 & 0 \\
0 & -1 & 0 & 0 & 1
\end{bmatrix}.
\end{equation*}
Wenden wir nun die Funktion $u$ auf diese Matrix an, erhalten wir
\begin{equation*}
\laplaceL u =
\begin{bmatrix}
1 & -1 & 0 & 0 & 0 \\
-1 & 4 & -1 & -1 & -1 \\
0 & -1 & 1 & 0 & 0 \\
0 & -1 & 0 & 1 & 0 \\
0 & -1 & 0 & 0 & 1
\end{bmatrix}
\begin{pmatrix}
u_0 \\
u_1 \\
u_2 \\
u_3 \\
u_4
\end{pmatrix}
=
\begin{pmatrix}
u_0 - u_1 \\
4 u_1 - u_0 - u_2 - u_3 - u_4 \\
u_2 - u_1 \\
u_3 - u_1 \\
u_4 - u_1
\end{pmatrix}
\end{equation*}
oder generell ausgedr\"uckt, mit $h = 1$
\begin{equation*}
\laplaceL u = \frac{1}{h^2}n\left(\frac{1}{n}\sum_{i\text{ Nachbarn}}^{n}u_i - 
u_0\right) = \Delta u.
\end{equation*}
