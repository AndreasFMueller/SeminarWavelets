Innerhalb dieser Seminararbeit konnte keine fertige Funktion zur Bildschärfung erarbeitet werden.
Der bestehende Ansatz mit der Beziehung \eqref{deconvolve:funktion}, welche zwei Freiheitsgrade beinhaltet, könnte aber noch verfeinert werden.
Die beiden Parameter $m$ und $\alpha$ wurden hier mithilfe einer eindimensionalen Funktion $g(x)$ (Abbildung \ref{deconvolve:1d}) festgelegt.
Es wäre daher vielleicht möglich, durch geschickteres wählen dieser Parameter auf ein besseres Ergebnis zu kommen.
Auch ist die Anwendung auf das Bild noch nicht ausgereift.

Allerdings bleiben folgende zwei Schwierigkeiten unabhängig von dem verwendeten Ansatz bestehen:
\begin{itemize}
	\item Höhere, relevante Level liefern zu wenige Koeffizienten, um einen geeigneten Zusammenhang zwischen einer unscharfen und einer scharfen Kurve zu erkennen.
	\item Das Bild ist mit $9600\times9600$ Pixel eher gross. Kleinere Bilder liefern bei der Wavelet-Transformation noch weniger Daten, wobei man dann auf das selbe Problem wie beim vorangehenden Punkt stösst.
\end{itemize}

Es wurden auch nur das Haar- bzw. db1-Wavelet verwendet.
Ein ähnlicher Ansatz, z.B. mit höheren Debauchies-Wavelets wäre sicher auch ein Versuch Wert.

Die Hauptursache für das unbefriedigende Resultat liegt im allgemeinen aber woanders.
Es wurde nur versucht, die Koeffizienten auf den einzelnen Level zu verändern, bzw jede Zeile in Abbildung \ref{deconvolve:y1_cwt} wurde unabhängig von der benachbarten verstärkt oder geschwächt.
Eine bessere Methode würde berücksichtigen, dass die Wavelet-Transformation ja zwei Dimensionen hat (Dilatation $a$ und Translation $b$) und somit \glqq Masse \grqq{} in der Ebene, also zwischen Level und nicht nur auf einer Zeile nach links oder rechts verschoben werden muss.

