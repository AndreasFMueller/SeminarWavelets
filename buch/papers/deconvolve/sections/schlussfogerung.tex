Innerhalb dieser Seminararbeit konnte keine fertige Funktion zur Bildschärfung erarbeitet werden.
Der bestehende Ansatz mit der Beziehung \eqref{deconvolve:funtion} welche die zwei Freiheitsgrade beinhaltet könnte aber noch verfeinert werden.
Die beiden Parameter $m$ und $\alpha$ wurden hier mithilfe einer eindimensionalen Funktion $g(x)$ (Abbildung \ref{deconvolve:1d}) festgelegt.
Es wäre daher vielleicht möglich, durch geschickteres wählen dieser Parameter auf ein besseres Ergebnis zu kommen.
Auch ist die Anwendung auf das Bild noch nicht ausgereift.
Würde man dort noch mehr Zeit aufwenden könnte man dort sicher auch noch etwas herausholen.

Allerdings bleiben folgende zwei Schwierigkeiten unabhängig von dem verwendeten Ansatz bestehen:
\begin{itemize}
	\item Höhere, relevante Level liefern zu wenige Koeffizienten, um einen geeigneten Zusammenhang zwischen einer unscharfen und einer scharfen zu erkennen.
	\item Das Bild ist mit 9600x9600 Pixel eher gross. Kleinere Bilder liefern bei der Wavelet-Transformation noch weniger Daten, wobei man dann auf das selbe Problem wie beim vorangehenden Punkt stösst.
\end{itemize}
