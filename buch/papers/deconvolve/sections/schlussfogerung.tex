Innerhalb dieser Seminararbeit konnte keine fertige Funktion zur Bildschärfung erarbeitet werden.
Der bestehende Ansatz mit der Beziehung \eqref{deconvolve:funktion}, welche zwei Freiheitsgrade beinhaltet, könnte aber noch verfeinert werden.
Die beiden Parameter $m$ und $\alpha$ wurden hier mithilfe einer eindimensionalen Funktion $g(x)$ (Abbildung \ref{deconvolve:1d}) festgelegt.
Es wäre daher vielleicht möglich, durch geschickteres wählen dieser Parameter auf ein besseres Ergebnis zu kommen.
Ausserdem ist die Anwendung auf das Bild noch nicht ausgereift.
Es wurden auch nur das Haar- bzw. db1-Wavelet verwendet.
Ein ähnliches Vorgehen, z.B. mit höheren Debauchies-Wavelets wäre sicher ein Versuch Wert.

Folgende Punkte scheinen aber die Hauptgründe zu sein, den bestehenden Ansatz zu verwerfen:
\begin{itemize}
	\item Es stellte sich als problematisch heraus, die Koeffizienten auf den einzelnen Level zu verändern, bzw jede Zeile in Abbildung \ref{deconvolve:y1_cwt} unabhängig von den anderen zu verstärken oder abzuschwächen.
	
	Die Wavelet-Transformation hat ja zwei Dimensionen (Dilatation $a$ und Translation $b$) und somit muss \glqq Masse \grqq{} in der Ebene, also zwischen Level und nicht nur auf einer Zeile nach links oder rechts verschoben werden.
	Ändert man auf einem Level etwas, so muss die Auswirkung auf die anderen Level unbedingt berücksichtigt werden.
	\item Wie beim eindimensionalen Versuch beobachtet, darf nicht davon ausgegangen werden, dass eine lokale Änderung auf einem Level nur diese Stelle beeinflusst.
	Folglich könnte sich eine Manipulation an einem anderen Ort, der sich weiter weg von der zu schärfenden Stelle befindet, umgekehrt eben auf diese Stelle auswirken. 
\end{itemize}

Diese zwei Erkenntnisse sollte man immer im Hinterkopf behalten, wenn man die Wavelet-Transformation vergleichbar einsetzten will.



