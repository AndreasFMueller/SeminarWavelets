%
% main.tex -- lifting steps
%
% (c) 2019 Prof Dr Andreas Müller, Hochschule Rapperswil
%
\chapter{Faktorisierung der Polyphasenmatrix und Lifting
\label{chapter:lifting}}
\lhead{Lifting}
\rhead{}
\begin{refsection}
\chapterauthor{Andreas Müller}

{\parindent 0pt
In Kapitel~\ref{chapter:fpga} wurde gezeigt, wie man die
Wavelet-Transformation als Folge sogenannter Lifting-Steps implementieren
kann, die sich durch eine besonders einfache Polyphasenmatrix auszeichnen.}
Die Multiplikation mit diesen einfach Polyphasenmatrizen ist mit nur
einer Multiplikation und einer Addition möglich.
Das Kapitel hat jedoch nicht erklärt, wie diese Faktorisierung
überhaupt gefunden werden kann.
Dies soll hier mindestens ansatzweise nachgeholt werden.

%
% euklid.tex
%
% (c) 2019 Prof Dr Andreas Müller, Hochschule Rapperswil
%
\section{Der euklidische Algorithmus}
Der euklidische Algorithmus bestimmt zu zwei gegebenen Zahlen $a$ und $b$
den grössten gemeinsamen Teiler $g$.
Zusätzlich findet er ganze Zahlen $s$ und $t$ derart, dass
\[
sa + tb = g.
\]
In diesem Abschnitt soll der Algorithmus zunächst für ganze Zahlen
vorgestellt werden, bevor er auf Polynome verallgemeinert und dann
in Matrixform niedergeschrieben wird.

\subsection{Ganze Zahlen}
Gegeben sind zwei ganze Zahlen $a$ und $b$ und wir dürfen annehmen,
dass $a\ge b$.
Gesucht ist der grösste gemeinsame Teiler $g$ von $a$ und $b$.
Ist $b|a$, dann ist offenbar $b$ der grösste gemeinsame Teiler.
Im Allgemeinen wird der grösste Teiler aber kleiner sein.
Wir teilen daher $a$ durch $b$, was nur mit Rest möglich ist.
Es gibt daher Zahlen $t$ und $r$ derart, dass
\[
a = tb+ r
\qquad \Rightarrow \qquad
r = a - tb.
\]
Es ist sicher $r < b$.
Da der grösste gemeinsame Teiler sowohl $a$ als auch $b$ teilt, muss er
auch $r$ teilen.
Somit haben wir das Problem, den grössten gemeinsamen Teiler von $a$ und
$b$ zu finden auf das ``kleinere'' Problem zurückgeführt, den grössten
gemeinsamen Teiler von $b$ und $r$ zu finden.

Um den eben beschriebenen Schritt zu wiederholen, wählen wir die folgende
Notation.
Wir schreiben $a_0=a$ und $b_0=b$.
Im ersten Schritt finden wird $q_0$ und $r_0$ derart,
dass $a_0-q_0b_0 = r_0$.
Dann setzen wir $a_1=b_0$ und $b_1=r_0$.
Mit $a_1$ und $b_1$ wiederholen wir den Divisionsschritt, der Zahlen
$q_1$ und $r_1$ liefert mit $a_1-q_1b_1=r_1$.
So entstehen vier Folgen von Zahlen $a_k$, $b_k$, $q_k$ und $r_k$ derart,
dass in jedem Schritt gilt
\begin{align*}
a_k - q_kb_k &= r_k & g&|a_k & g&|b_k & a_k &= b_{k-1} & b_k = r_{k-1}
\end{align*}
Der Algorithmus bricht im Schritt $n$ ab, wenn $r_{n+1}=0$.
Der letzte nicht verschwindende Rest $r_n$ muss daher der grösste gemeinsame
Teiler sein: $g=r_n$.

\begin{beispiel}
Wir bestimmen den grössten gemeinsamen Teiler von $76415$ und $23205$
mit Hilfe des eben beschriebenen Algorithmus.
Wir schreiben die gefundenen Zahlen in eine Tabelle:
\begin{center}
\renewcommand{\arraystretch}{1.1}
\begin{tabular}{|>{$}r<{$}|>{$}r<{$}|>{$}r<{$}|>{$}r<{$}|>{$}r<{$}|}
\hline
k&  a_k&  b_k&   q_k&  r_k\\
\hline
0&76415&23205&     3&6800\\
1&23205& 6800&     3&2805\\
2& 6800& 2805&     2&1190\\
3& 2805& 1190&     2& 425\\
4& 1190&  425&     2& 340\\
5&  425&  340&     1&  85\\
6&  340&   85&     4&   0\\
\hline
\end{tabular}
\end{center}
Der Algorithmus bricht also mit dem letzten Rest $r_n=85$ ab, dies
ist der grösste gemeinsame Teiler.
\end{beispiel}

Die oben protokollierten Werte von $q_k$ werden für die Bestimmung
des grössten gemeinsamen Teilers nicht benötigt.
Wir können sie aber verwenden, um die Zahlen $s$ und $t$ zu bestimmen.

\begin{beispiel}
Wir drücken die Reste im obigen Beispiel durch die Zahlen $a_k$, $b_k$ und
$q_k$ aus und setzen sie in den Ausdruck $g=a_5-q_5b_5$ ein, bis wir
einen Ausdruck in $a_0$ und $b_0$ für $g$ finden:
\begin{align*}
r_5&=a_5-q_5 b_5=a_5-1\cdot b_5& g &= a_5 - 1 \cdot b_5 = b_4 - 1 \cdot r_4
\\
r_4&=a_4-q_4 b_4=a_4-2\cdot b_4&   &= b_4 - (a_4 -2b_4) 
                                    = -a_4 +3b_4 = -b_3 + 3r_3
\\
r_3&=a_3-q_3 b_3=a_3-2\cdot b_3&   &= -b_3 + 3(a_3-2b_3)
                                    = 3a_3 - 7b_3 = 3b_2 -7r_2
\\
r_2&=a_2-q_2 b_2=a_2-2\cdot b_2&   &= 3b_2 -7(a_2-2b_2)
                                    = -7a_2 + 17b_2 = -7b_1 + 17r_1
\\
r_1&=a_1-q_1 b_1=a_1-3\cdot b_1&   &= -7b_1 + 17(a_1-3b_1)
                                    = 17a_1 - 58b_1 = 17 b_0 - 58 r_0
\\
r_0&=a_0-q_0 b_0=a_0-3\cdot b_0&   &= 17b_0 - 58(a_0t-3b_0)
                                    = -58a_0+191b_0
\end{align*}
Tatsächlich gilt
\[
-58\cdot 76415 + 191 \cdot 23205 = 85,
\]
die Zahlen $t=-58$ und $s=191$ sind also genau die eingangs versprochenen
Faktoren.
\end{beispiel}

\subsection{Matrixschreibweise}
Die Berechnung der Zahlen $s$ und $t$ lässt sich besser verstehen, wenn
wir die Koeffizienten von $a_k$ und $b_k$ als Vektoren betrachen und
die Berechnung in Matrixschreibweise durchführen.
In jedem Schritt finden wir so einen Vektor mit Koordinaten $s_k$ und $t_k$
derart, dass das Skalarprodukt
\[
\begin{pmatrix}s_k\\t_k \end{pmatrix}
\cdot
\begin{pmatrix}a_k\\b_k \end{pmatrix}
=
g
\]
ist.

In jedem Schritt verwenden wir, dass
\[
\begin{pmatrix}
a_{k}\\b_{k}
\end{pmatrix}
=
\begin{pmatrix}
b_{k-1}\\r_{k-1}
\end{pmatrix}
=
\begin{pmatrix}
b_{k-1}\\
a_{k-1}-q_{k-1}b_{k-1}
\end{pmatrix}
=
\begin{pmatrix}
0&1\\
1&-q_{k-1}
\end{pmatrix}
\begin{pmatrix}
a_{k-1}\\b_{k-1}
\end{pmatrix}.
\]
Durch Iteration findet man daher
\[
\begin{pmatrix}
a_{n}\\b_{n}
\end{pmatrix}
=
\begin{pmatrix}
0&1\\
1&-q_{n-1}
\end{pmatrix}
\begin{pmatrix}
a_{n-1}\\b_{n-1}
\end{pmatrix}
=
\begin{pmatrix}
0&1\\
1&-q_{n-1}
\end{pmatrix}
\begin{pmatrix}
0&1\\
1&-q_{n-2}
\end{pmatrix}
\cdots
\begin{pmatrix}
0&1\\
1&-q_{0}
\end{pmatrix}
\begin{pmatrix}
a_{0}\\b_{0}
\end{pmatrix}.
\]
Der grösste gemeinsame Teiler $g$  erfüllt die Bedingung $g=a_n-q_nb_n$,
dies ist die zweite Zeile von
\[
\begin{pmatrix}
0&1\\
1&-q_n
\end{pmatrix}
\begin{pmatrix}a_n\\b_n\end{pmatrix}
=
\begin{pmatrix}
b_n\\
g
\end{pmatrix}
\]
Insbesondere kann man $g$ erhalten als Skalarprodukt des Vektors
$\begin{pmatrix}0&1\end{pmatrix}^t$ mit dem Produkt
\[
\underbrace{
\begin{pmatrix} 0&1 \\ 1&-q_n\end{pmatrix}
\begin{pmatrix} 0&1 \\ 1&-q_{n-1}\end{pmatrix}
\cdots
\begin{pmatrix} 0&1 \\ 1&-q_{0}\end{pmatrix}
}_{\displaystyle = Q_n}
\begin{pmatrix}a_0\\b_0\end{pmatrix}
=
Q \begin{pmatrix}a_0\\b_0\end{pmatrix}
\]
Das Skalarprodukt mit
$\begin{pmatrix}0&1\end{pmatrix}^t$ ist dasselbe wie die Multiplikation
von links mit dem Zeilenvektor $\begin{pmatrix}0&1\end{pmatrix}$.
Dafür ist nur die zweite Zeile von $Q$ nötig:
\[
\begin{pmatrix}0&1\end{pmatrix}
Q
=
\begin{pmatrix}0&1\end{pmatrix}
\begin{pmatrix}
*&*\\
s&t
\end{pmatrix}
\qquad\Rightarrow\qquad
g
=
\begin{pmatrix}0&1\end{pmatrix}
\begin{pmatrix}
*&*\\
s&t
\end{pmatrix}
\begin{pmatrix}a\\b\end{pmatrix}
=
ta+sb
\]
Das Matrizenprodukt enthält also in der zweiten Zeile genau die
gesuchten Zahlen $t$ und $s$.

\begin{beispiel}
Wir verifizieren die Behauptung 
\begin{align*}
Q
&=
\begin{pmatrix} 0&1 \\ 1&-q_n\end{pmatrix}
\begin{pmatrix} 0&1 \\ 1&-q_{n-1}\end{pmatrix}
\cdots
\begin{pmatrix} 0&1 \\ 1&-q_{0}\end{pmatrix}
\\
&=
\underbrace{
\begin{pmatrix} 0&1 \\ 1& -1 \end{pmatrix}
\begin{pmatrix} 0&1 \\ 1& -2 \end{pmatrix}
}_{}
\underbrace{
\begin{pmatrix} 0&1 \\ 1& -2 \end{pmatrix}
\begin{pmatrix} 0&1 \\ 1& -2 \end{pmatrix}
}_{}
\underbrace{
\begin{pmatrix} 0&1 \\ 1& -3 \end{pmatrix}
\begin{pmatrix} 0&1 \\ 1& -3 \end{pmatrix}
}_{}
\\
&=
\underbrace{
\begin{pmatrix} 1 & -2 \\ -1 &  3 \end{pmatrix}
\begin{pmatrix} 1 & -2 \\ -2 &  5 \end{pmatrix}
}_{}
\begin{pmatrix} 1 & -3 \\ -3 & 10 \end{pmatrix}
\\
&=
\begin{pmatrix} 5 & -12 \\ -7 & 17 \end{pmatrix}
\begin{pmatrix} 1 &  -3 \\ -3 & 10 \end{pmatrix}
=
\begin{pmatrix} 41 & -135 \\ -58 & 191 \end{pmatrix}.
\end{align*}
In der zweiten Zeile finden wir tatsächlich die beiden Zahlen $t=-58$ und 
$s=191$, für die wir schon früher erkannt haben, dass sie die beiden
Zahlen $a$ und $b$ linear zum grössten gemeinsamen Teiler kombinieren.

Wir können das Matrizenprodukt auch noch einen Schritt weiter treiben:
\[
\begin{pmatrix} 0 & 1 \\ 1 & -4 \end{pmatrix}
\begin{pmatrix} 41 & -135 \\ -58 & 191 \end{pmatrix}
=
\begin{pmatrix} -58 & 191 \\ 273 & -899 \end{pmatrix}
\]
In der zweiten Zeile findet man Zahlen, die $a$ und $b$ zu 0 kombinieren:
\[
273 \cdot 76415 - 899 \cdot 23205 = 0,
\]
denn dies ist die Bedingung $r_{n+1} = 0$.
\end{beispiel}

Wir fassen das Gefundene zusammen in folgendem Satz:

\begin{satz}
Seien $q_k$ die Reste, die während der Durchführung des euklidischen
Algorithmus auf den Zahlen $a$ und $b$ durchgeführten Quotienten.
Der euklidische Algorithmus berechnet dann
\[
\begin{pmatrix} 0 & 1 \\ 1 & -q_k \end{pmatrix}
\begin{pmatrix} a_k \\ b_k \end{pmatrix}
=
\begin{pmatrix} b_k \\ r_k \end{pmatrix}
=
\begin{pmatrix} r_{k-1} \\ r_k \end{pmatrix}
\]
Das Produkt dieser $2\times 2$-Matrizen erfüllt
\[
\begin{pmatrix}g\\0\end{pmatrix}
=
\begin{pmatrix}r_n\\r_{n+1}\end{pmatrix}
=
\begin{pmatrix} 0&1 \\ 1&-q_n\end{pmatrix}
\begin{pmatrix} 0&1 \\ 1&-q_{n-1}\end{pmatrix}
\cdots
\begin{pmatrix} 0&1 \\ 1&-q_{0}\end{pmatrix}
\begin{pmatrix}a\\b\end{pmatrix}
=
Q_n
\begin{pmatrix}a\\b\end{pmatrix}
\]
\end{satz}

Die Wirkung der Matrix
\[
Q(q) = \begin{pmatrix} 0 & 1 \\ 1 & -q \end{pmatrix}
\]
lässt sich mit genau einer Multiplikation und einer Addition
berechnen.
Dies ist die Art von Matrix, die wir für die Implementation der
Wavelet-Transformation anstreben.

\subsection{Polynome}
Der Ring $\mathbb{Q}[X]$ der Polynome in der Variablen $X$ mit rationalen
Koeffizienten verhält
sich bezüglich Teilbarkeit ganz genau gleich wie die ganzen Zahlen.
Insbesondere ist der euklidische Algorithmus genauso wie die
Matrixschreibweise auch für Polynome durchführbar.

\begin{beispiel}
Wir berechnen als Beispiel den grössten gemeinsamen Teiler 
der Polynome
\[
a = X^4 - 2X^3 -7 X^2 + 8X + 12,
\qquad
b = X^4 + X^3 -7X^2 -X + 6.
\]
Wir erstellen wieder die Tabelle der Reste
\begin{center}
\renewcommand{\arraystretch}{1.4}
\begin{tabular}{|>{$}r<{$}|>{$}r<{$}|>{$}r<{$}|>{$}r<{$}|>{$}r<{$}|}
\hline
k&  a_k&  b_k&   q_k&  r_k\\
\hline
0& X^4 - 2X^3 -7 X^2 + 8X + 12& X^4 + X^3 -7X^2 -X + 6&       1&-3X^3+9X+6\\
1&X^4+X^3-7X^2-X+6            &-3X^3+9X+6             &-\frac13X-\frac13&-4X^2+4X+8\\
2&-3X^3+9X+6      &-4X^2+4X+8& \frac34 X + \frac34& 0\\
\hline
\end{tabular}
\end{center}
Daraus kann man ablesen, dass $-4x^2+4x+8$ grösster gemeinsamer Teiler ist.
Normiert auf einen führenden Koeffizienten $1$ ist dies das Polynom
$x^2-x+2=(x+2)(x-1)$.

Wir berechnen auch noch die Polynome $s$ und $t$.
Dazu müssen wir die Matrizen $Q(q_k)$ miteinander multiplizieren:
\begin{align*}
Q
&=Q(q_2) Q(q_1) Q(q_0)
\\
&=
\begin{pmatrix} 0 & 1 \\ 1 & -\frac34(X+1) \end{pmatrix}
\begin{pmatrix} 0 & 1 \\ 1 & \frac13(X+1) \end{pmatrix}
\begin{pmatrix} 0 & 1 \\ 1 & -1 \end{pmatrix}
\\
&=
%                        [     x   1         2   x    ]
%                        [     - + -         - - -    ]
%                        [     3   3         3   3    ]
%(%o22)                  [                            ]
%                        [     2            2         ]
%                        [    x     x   3  x    x   3 ]
%                        [ (- --) - - + -  -- - - - - ]
%                        [    4     2   4  4    4   2 ]
\begin{pmatrix}
\frac13(X+1)&-\frac13(X-2)\\
-\frac14(X^2+2X-3)&\frac14(X^2-X-6)
\end{pmatrix}.
\end{align*}
In der ersten Zeile finden wir die Polynome $t(X)$ und $s(X)$, mit denen
\begin{align*}
ta+sb
&=
\frac13(X+1)
(X^4-2X^3-7X^2+8X+12)
-\frac13(X-2)
(X^4+X^3-7X^2-X+6)
\\
&=
-4X^2+4X+8
\end{align*}
und dies ist tatsächlich der gefundene grösste gemeinsame Teiler.
Die zweite Zeile von $Q$ gibt uns die Polynomfaktoren, mit denen
$a$ und $b$ gleich werden:
\begin{align*}
q_{21}a+q_{22}b
&=
-\frac14(X^2+2X-3)
(X^4-2X^3-7X^2+8X+12)
+\frac14(X^2-X-6)
(X^4+X^3-7X^2-X+6)
\\
&=0.
\qedhere
\end{align*}
Man kann natürlich den grössten gemeinsamen Teiler auch mit Hilfe einer
Faktorisierung der Polynome $a$ und $b$ finden:
\begin{align*}
a  &=         (X-3) (X-2)\phantom{(X-1)}(X+1)         (X+2) \phantom{(X+3)}\\
b  &=\phantom{(X-3)}(X-2)         (X-1) (X+1)\phantom{(X+2)}         (X+3) \\
g  &=\phantom{(X-3)}(X-2)\phantom{(X-1)}(X+1)\phantom{(X+2)}\phantom{(X+3)}
    = X^2 -X + 2\\
v=a/g&=         (X-3)\phantom{(X-2)(X-1)(X+1)} (X+2) \phantom{(X+3)}
    = X^2-X-6 \\
u=b/g&=\phantom{(X-3)(X-2)} (X-1)\phantom{(X+1)(X+2)}(X+3)
    = X^2+2X-3
\end{align*}
Aus den letzten zwei Zeilen folgt
$ua-vb = ab/g - ab/g = 0$, wie erwartet.
\end{beispiel}



%
% phase.tex
%
\section{Faktorisierung der Phasenmatrix}
In diesem Abschnitt wenden wir den euklidischen Algorithmus auf das
Problem an, die Phasenmatrix in einfache Schritte zu zerlegen.



\printbibliography[heading=subbibliography]
\end{refsection}
