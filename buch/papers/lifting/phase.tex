%
% phase.tex
%
\section{Faktorisierung der Polyphasenmatrix}
\rhead{Lifting Steps}
In diesem Abschnitt wenden wir den euklidischen Algorithmus auf das
Problem an, die Polyphasenmatrix in einfache Schritte zu zerlegen.
Um möglichst direkt zur Faktorisierung~\eqref{lifting:faktorisierung}
zu kommen, werden wir mit einzelnen Details etwas salop umgehen, dies wird
im letzten Abschnitt~\ref{lifting:verschwiegen} teilweise korrigiert werden.

\subsection{Die Polyphasen-Matrix}
In \cite{fpga:Daubechies1998} wird erklärt, wie man die beiden
Filter einer Multiskalen-Analyse in Form einer $2\times 2$-Matrix
schreiben kann.
Die Filter mit Koeffizientenvektoren $h_k$ und $g_k$ können durch 
ihre Z-Transformation beschrieben werden, also durch die Polynome
\[
h(z) = \sum_{k} h_kz^{-k}
\qquad\text{und}\qquad
g(z) = \sum_{k} g_kz^{-k}.
\]
Im schnellen Algorithmus (Kapitel~\ref{chapter:algo}), der zu einer
Multiskalenanalyse (Kapitel~\ref{chapter:msa}) gehört, werden die
geraden und die ungeraden Samples leicht unterschiedlich behandelt.
Daher liegt es nahe, auch die Polynome $h(z)$ und $g(z)$ in die
geraden und ungeraden Terme aufzuteilen:
\[
h_e(z) = \sum_k h_{2k}z^{-k},
\quad
h_o(z) = \sum_k h_{2k+1}z^{-k},
\quad
g_e(z) = \sum_k g_{2k}z^{-k}
\quad\text{und}\quad
g_o(z) = \sum_k g_{2k+1}z^{-k}.
\]
Die Polyphasenmatrix ist dann definiert als
\[
P(z)
=
\begin{pmatrix}
h_e(z)&g_e(z)\\
h_o(z)&g_o(z)\\
\end{pmatrix}.
\]
Die ursprünglichen Polynome sind
\[
\begin{aligned}
h(z) &= h_e(z^2) + z^{-1}h_o(z^2)\\
g(z) &= g_e(z^2) + z^{-1}g_o(z^2)
\end{aligned}
\qquad\Rightarrow\qquad
\begin{pmatrix}
h(z)\\
g(z)
\end{pmatrix}
=
P(z^2)^t
\begin{pmatrix} 1\\ z^{-1} \end{pmatrix}.
\]
Die gesamte Information über die Filter $h$ und $g$ ist in der Matrix
$P(z)$ enthalten.

Die Wirkung des Filters auf einem Signal wird durch die Multiplikation
von $h(z)$ mit der Z-Transformation des Signals durchgeführt.
Um dies mit der Polyphasenmatrix durchzuführen, müssen auch die Samples
des Signals in gerade und ungerade numeriert Samples aufgeteilt werden.
Statt Signals $x_k$ verwendet man auch
\[
\left.
\begin{aligned}
x_e(z) &= \sum_k x_{2k}z^{-k} \\
x_o(z) &= \sum_k x_{2k+1}z^{-k}
\end{aligned}
\quad\right\}
\qquad
\Rightarrow
\qquad
x(z) = x_e(z^2) + z^{-1} x_o(z^2).
\]
Die Wirkung der beiden Filter auf dem Signal kann durch das Produkt
\[
P(z)^t \begin{pmatrix} x_e(z)\\ x_o(z) \end{pmatrix}
\]
beschrieben werden.

Wir versuchen, die Anzahl der Operationen abzuschätzen, die zur
Berechnung nötig sind.
Dazu nehmen wir an, dass die Polynome $h(z)$ und $g(z)$ höchstens $n$
von $0$ verschiedene Koeffizienten hat.
Die Multiplikation mit $h(z)$ und $g(z)$ erfordert bei naiver 
Implementation je $n$ Multiplikationen und Additionen, also
insgesamt $2n$.

\subsection{Faktorisierung eines Filters}
Wir wenden den euklidischen Algorithmus auf die beiden Polynome $h_e(z)$
und $h_o(z)$ an.
Die Quotienten bezeichnen wir mit $q_0(z),\dots,q_n(z)$ und erhalten dann
\begin{align*}
\begin{pmatrix} K(z) \\ 0 \end{pmatrix}
&=
\begin{pmatrix} 0&1\\ 1&-q_n(z) \end{pmatrix}
\begin{pmatrix} 0&1\\ 1&-q_{n-1}(z) \end{pmatrix}
\cdots
\begin{pmatrix} 0&1\\ 1&-q_0(z) \end{pmatrix}
\begin{pmatrix} h_e(z)\\h_o(z)\end{pmatrix}
\\
&=
Q(q_n(z))
Q(q_{n-1}(z))
\cdots
Q(q_0(z))
\begin{pmatrix} h_e(z)\\h_o(z)\end{pmatrix}.
\end{align*}
Durch Invertieren aller Matrizen $Q(q_k(z))$ können wir nach $h_e(z)$
und $h_o(z)$ auflösen und erhalten
\[
\begin{pmatrix} h_e(z)\\h_o(z)\end{pmatrix}
=
Q(q_0(z))^{-1}
\cdots
Q(q_{n-1}(z))^{-1}
Q(q_n(z))^{-1}
\begin{pmatrix}K(z)\\0\end{pmatrix}.
\]
Die inverse Matrix von $Q(q(z))$ ist
\[
Q(q(z))^{-1}
= 
\begin{pmatrix}
0&1\\
1&-q(z)
\end{pmatrix}^{-1}
=
\begin{pmatrix}
q(z)&1\\
1&0
\end{pmatrix},
\]
so dass für $h_e(z)$ und $h_o(z)$ folgt
\[
\begin{pmatrix} h_e(z)\\h_o(z)\end{pmatrix}
=
\begin{pmatrix} q_0(z)&1\\1&0 \end{pmatrix}
\begin{pmatrix} q_1(z)&1\\1&0 \end{pmatrix}
\cdots
\begin{pmatrix} q_n(z)&1\\1&0 \end{pmatrix}
\begin{pmatrix} K(z) \\ 0 \end{pmatrix}.
\]

\subsection{Lifting Steps}
Ein Lifting-Step ist eine besonders einfache Operation, die sich in
Hardware mit nur einer Multiplikation und einer Addition durchführen lässt.
Aus zwei Input-Samples $x$ und $y$ wird ein neuer Wert $x$ 
in der Form $x_{\text{neu}}=x+sy$ oder für $y$ in der Form
$y_{\text{neu}}=tx+y$ berechnet, der andere Wert wird unverändert
gelassen.
In Matrixform lauten diese Operationen
\[
\begin{pmatrix} x_{\text{neu}}\\y_{\text{neu}} \end{pmatrix}
=
\begin{pmatrix} 1&s \\ 0&1\end{pmatrix}
\begin{pmatrix}x\\y\end{pmatrix}
\qquad\text{bzw.}\qquad
\begin{pmatrix} x_{\text{neu}}\\y_{\text{neu}} \end{pmatrix}
=
\begin{pmatrix} 1&0 \\ t&1\end{pmatrix}
\begin{pmatrix}x\\y\end{pmatrix}.
\]
Für die Anwendung auf die Wavelettransformation müssen wir zulassen,
dass in einem Lifting Step ein ``verzögertes'' Sample verwendet wird.
Dies wird dadurch ausgedrückt, dass $s$ nicht nur eine Konstante,
sondern möglicherweise ein Polynom in $z$ ist.

\begin{definition}
Ein Lifting Step ist die Operation auf zwei Samplewerten $x$ und $y$,
die durch eine der Matrizen
\[
R(s)
=
\begin{pmatrix}
1&s\\
0&1
\end{pmatrix}
\qquad\text{bzw.}\qquad
L(t)
=
\begin{pmatrix}
1&0\\
t&1
\end{pmatrix}
\]
beschrieben werden.
\end{definition}

\subsection{Zerlegung in Lifting Steps}
Die mit dem euklidischen Algorithmus gefundenen Matrizen $Q(q)$ sind
keine Lifting-Steps.
Sie lassen sich aber leicht in diese Form bringen, wenn man die
Vertauschungsmatrix 
\[
V
=
\begin{pmatrix}0&1\\1&0\end{pmatrix}
\qquad\text{mit}\qquad
V^{-1} = V, \qquad V^2=E
\]
hinzuzieht.
Es gelten nämlich die Relationen
\begin{align}
Q(q(z))V 
&=
\begin{pmatrix} 0 & 1 \\ 1 & -q(z) \end{pmatrix}
\begin{pmatrix}0&1\\1&0\end{pmatrix}
=
\begin{pmatrix} 1 & 0 \\ -q(z) & 1 \end{pmatrix}
=
L(-q(z)),
\notag
\\
VQ(q(z))
&=
\begin{pmatrix}0&1\\1&0\end{pmatrix}
\begin{pmatrix} 0 & 1 \\ 1 & -q(z) \end{pmatrix}
=
\begin{pmatrix} 1 & -q(z) \\ 0 & 1 \end{pmatrix}
=
R(-q(z)),
\notag
\\
Q(q(z))^{-1}V
&=
\begin{pmatrix} q(z) & 1 \\ 1 & 0 \end{pmatrix}
\begin{pmatrix}0&1\\1&0\end{pmatrix}
=
\begin{pmatrix} 1 & q(z) \\ 0 & 1 \end{pmatrix}
=
R(q(z)),
\label{lifting:step:qvr}
\\
VQ(q(z))^{-1}
&=
\begin{pmatrix}0&1\\1&0\end{pmatrix}
\begin{pmatrix} q(z) & 1 \\ 1 & 0 \end{pmatrix}
=
\begin{pmatrix} 1 & 0 \\ q(z) & 1 \end{pmatrix}
=
L(q(z)).
\label{lifting:step:vql}
\end{align}

Zwei aufeinanderfolgende $Q$-Matrizen können wir jetzt durch Zwischenschieben
der Einheitsmatrix $E=V^2$ in zwei Lifting Steps verwandeln.
Wir rechnen
\begin{equation}
Q(t(z))^{-1}Q(s(z))^{-1}
=
Q(t(z))^{-1}V^2Q(s(z))^{-1}
=
(Q(t(z))^{-1}V)(VQ(-s(z))^{-1})
=
R(t(z)) L(s(z)),
\label{lifting:split}
\end{equation}
wobei wir im letzten Schritt die zwei Regeln~\eqref{lifting:step:qvr}
und \eqref{lifting:step:vql} angewendet haben.

Für eine gerade Anzahl von Schritten im euklidischen Algorithmus 
kann man \eqref{lifting:split} für jedes Paar von $Q$-Matrizen anwenden.
So können die beiden Polynome $h_e(z)$ und $h_o(z)$ durch die Folge
\begin{align}
\begin{pmatrix} h_e(z)\\h_o(z)\end{pmatrix}
&=
Q(q_0(z))^{-1}
Q(q_1(z))^{-1}
\cdots
Q(q_n(z))^{-1}
\begin{pmatrix}K(z)\\ 0 \end{pmatrix}
\notag
\\
&=
R(q_0(z)) L(q_1(z)) \cdots R(q_{2k}(z)) L(q_{2k+1}(z)) \cdots 
\begin{pmatrix}K(z)\\ 0 \end{pmatrix}
\label{lifting:faktorisierung}
\end{align}
von Lifting Steps konstruiert werden.
Falls $n$ ungerade ist, bleibt am Ende die Matrix $Q(q_n(z))^{-1}$
übrig.

\subsection{Rechenaufwand für die faktorisierte Polyphasenmatrix}
Die Faktorisierung der Polyphasenmatrix liefert $n/2$ Faktoren,
die je mit einer Multiplikation und einer Addition berechnet werden
können.
Die Berechnung ist daher mit $n$ Multiplikationen und Additionen
möglich.
Die Faktorisierung hat also eine Beschleunigung um den Faktor $2$ 
gebraucht.

\subsection{Was wir bis jetzt verschwiegen haben
\label{lifting:verschwiegen}}
In der obigen Darstellung der Zerlegung der Polyphasen-Matrix 
haben wir im Interesse einer flüssigen Präsentation einige
wichtige Punkte einfach übergangen.
Dieser Abschnitt soll auf diese Schwierigkeiten aufmerksam machen
und mindestens andeuten, wie man mit ihnen umgehen kann.
Für vollständige Lösungen sei auf \cite{fpga:Daubechies1998} verwiesen.

\subsubsection{Laurent-Polynome}
Bei der Durchführung der Faktorisierung sind wir davon ausgegangen,
dass die $h(z)$ und $g(z)$ Polynome sind.
Dies ist jedoch nicht korrekt, denn es können auch negative
Exponenten vorkommen.
Man nennt Polynome in $z$ und $z^{-1}$ auch Laurent-Polynome.
\index{Laurent-Polynom}
Dies ist jedoch kein wesentliches Problem, denn der euklidische
Algorithmus funktioniert mit etwas Vorsicht auch für Laurent-Polynome.

\subsubsection{Was ist $K(z)$?}
Der grösste gemeinsame Teiler von $h_e(z)$ und $h_o(z)$ ist $K(z)$,
also ein Laurent-Polynom.
Dies ist der Teil, der nicht in Lifting-Steps faktorisiert werden kann.
Es ist daher nicht viel gewonnen, wenn dies ein Polynom hohen Grades ist.
Allerdings kann man mit Hilfe von hier nicht weiter diskutierten
Eigenschaften der beiden Filter einer Multiskalenanalyse zeigen,
dass die Polyphasenmatrix Determinante $1$ hat.
Da die Lifting-Steps alle Determinante $1$ haben, kann man daraus
ableiten, dass $K(z)=K$ nur eine Konstante ist.

\subsubsection{Was ist der Grad von $q_k(z)$?}
Die einzelnen Quotienten $q_k(z)$ könnten Polynome hohen Grades sein.
Dies würde bedeuten, dass ein einzelner Lifting Step doch wieder
mehrere Additionen und Multiplikationen verlangen würde.
Ebenfalls aus Eigenschaften der beiden Filter $h$ und $g$ lässt
sich ableiten, dass die Polynome $q_k(z)$ genau einen
von $0$ verschiedenen Koeffizienten haben.
Sie sind daher immer mit einer einzigen Multiplikation und einer
Addition und schlimmstenfalls einer Verzögerung realisierbar.



