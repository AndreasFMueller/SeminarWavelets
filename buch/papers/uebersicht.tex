%
% uebersicht.tex -- Uebersicht ueber die Seminar-Arbeiten
%
% (c) 2018 Prof Dr Andreas Mueller, Hochschule Rapperswil
%
\chapter*{Übersicht}
\lhead{Übersicht}
\rhead{}
\label{buch:uebersicht}
Im zweiten Teil kommen die Teilnehmer des Seminars selbst zu Wort.
Die im ersten Teil dargelegten mathematischen Methoden und
grundlegenden Modelle werden dabei verfeinert, verallgemeinert
und auch numerisch überprüft.

Die Fourier-Theorie ist die Basis vieler technisch wichtiger Algorithmen.
Die Wavelet-Theorie bietet sich als alternatives Werkzeug an.
Dabei werden andere Schwierigkeiten auftreten, die in verschiedenen
Arbeiten untersucht werden.
In einem Autotune-System muss man die nächstliegende Frequenz der
Tonleiter aus einem Signal extrahieren.
{\em Cédric Renda} untersucht, wie man dies mit Wavelets machen könnte.
Deconvolution lässt sich mit Fourier-Theorie dank der schnellen
Fourier-Transformation (FFT) effizient durchführen.
Der Versuch von {\em Manuel Tischhauser} deckt auf, dass Deconvolution
mit Wavelets zwar auch möglich ist, dass die Lokalisierung der
Wavelet-Transformation aber neue Schwierigkeiten bringt.
Das MP3-Audioformat verwendet Fourier-Theorie und Psychoakustische
Gesetzmässigkeiten zur Reduktion der Datenmenge.
{\em Julian Bärtschi} skizziert verschiedene Ideen, wie auch Wavelets
zur Datenreduktion verwendet werden können.

Die Wavelet-Transformation ist nur dann nützlich, wenn sie sich auch
effizient implementieren lässt.
Wavelets mit kompaktem Träger und Multiskalen-Analysen sind der Weg dazu.
{\em Nicolas Tobler} und {\em Jonas Gründler} beschreiben, wie eine
solche Wavelettransformation in einem FPGA implementiert werden kann.

Es ist schon länger bekannt, dass der visuelle Kortex eine Vorverarbeitung
der von der Netzhaut kommenden Signale durchführt, die der Analyse mit
einem Gabor-Wavelet ähnelt.
{\em Raphael Unterer} untersucht, ob sich die Leistung eines neuronalen
Netzwerks bei der Klassifizierung von Bildern verbessern lässt, indem
man dem Netzwerk einen Layer vorschaltet, der eine Wavelettransformation
durchführt.

Viele Beispiele von interessanten Wavelets sind reell, dies führt jedoch
auch dazu, dass die stetige Wavelet-Transformation eines periodischen Signals
in eine Folge von Punkten hoher Intensität zerfällt.
Nur bei komplexen Wavelets wie dem komplexen Morlet-Wavelet zeigt sich 
ein periodisches Signal als langezogene Zone hohen Absolutwertes.
{\em Roy Seitz} geht diesem Phänomen auf den Grund und untersucht,
wie man mit Hilfe der Hilbert-Transformation zu einem reellen Wavelet
einen ``Imaginärteil'' finden kann.

Die Daubechies-Wavelet haben besondere Eigenschaften, wenn man sie
auf polynomiale Signale anwendet.
{\em Raphael Nestler} untersucht, was dies genau bedeutet und was
sich aus der Wavelet-Transformation etwas über hochfrequente, aber sehr
schwache Störungen und Ableitungen sagen lässt, Dinge, die in der
Fourier-Transformation schwierig zu analysieren sind.

In der Praxis kann man oft mit der Wavelettransformation die
Klassifizierung von Signalen vereinfachen.
Der Ingenieur hat dabei oft mit Vorurteilen zu kämpfen, die ihm die
``Indoktrination'' mit den Prinzipien der Fourier-Theorie eingetrichtert hat.
Dinge wie die Bestimmung einer Frequenz, die bei der Fourier-Transformation
naheligend und natürlich sind, funktionieren mit der Wavelet-Transformation
plötzlich etwas anders.
{\em Kris Wyss} versucht dies bei Signalen, die bei Fehlern in gasisolierten
Schaltanlagen erzeugt werden.
{\em Dominic Hüppi} analysiert Radar-Echos von Meteoren und Flugzeugen
und findet neue Anforderung an die Sampling-Frequenz, die stringenter sind
als bei der Fourier-Theorie.
Wetter-Daten zeichnen sich durch wenige Periodizitäten und häufige
Transienten aus.
Die Analyse der Wetterereignisse muss diese Transienten detailliert
untersuchen, was mit der Fourier-Theorie nicht funktioniert.
{\em Michael Schmid} zeigt, wie man mit einer Wavelet-Analyse zum Ziel
kommen kann.

Die Wavelet-Transformation ist nur anwendbar auf Funktionen auf
Definitionsgebieten, in denen Translation und Dilatation wohldefinierte
Operationen sind.
In jüngster Zeit hat sich aber herausgestellt, dass die Dilatationseigenschaft
aleine auch schon ein nützliches Hilfsmittel bei der Funktionsanalyse sein 
kann.
Dies ermöglicht, den Definitionsbereich zu erweitern.
{\em Hansruedi Patzen} zeigt, wie die spektrale Graphentheorie sich dazu
verwenden lässt, eine Familie von Wavelets auf einem Graphen zu definieren.
Diese ``Spectral Graph Wavelet Transform'' scheint grosses Potential in
Anwendungen zu haben.





