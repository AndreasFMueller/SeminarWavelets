%
% main.tex -- Paper zum Thema fpga
%
% (c) 2019 Hochschule Rapperswil
%
\chapter{FPGA Implementation der schnellen Wavelet-Transformation\label{chapter:fpga}}
\lhead{FPGA Implementation der schnellen Wavelet-Transformation}
\begin{refsection}
\chapterauthor{Jonas Gründler und Nicolas Tobler}

\section{Introduction}
\rhead{Introduction}

Hardware implementation of a Wavelet transform and inverse transform using VHDL


\section{Lifting Scheme}
\rhead{Lifting Scheme}

The Lifting Scheme is an widely used algorythm which is tho times more efficient in termes of multiplication uses. 

%https://en.wikipedia.org/wiki/Lifting_scheme
%http://wavelets.org/schemes-lifting.php

\subsection{Polyphase Decomposition}

%https://www.dsprelated.com/freebooks/sasp/Polyphase_Decomposition.html

Polyphase definition is
\begin{equation}
H(z)=E_{0}(z^2)+z^{−1} E_1(z^2)
\end{equation}

Polyphase components are
\begin{equation}
E_0(z) = \sum_{n=-\inf}^\inf h(2n)z^{-n}
\end{equation}
\begin{equation}
E_1(z) = \sum_{n=-\inf}^\inf h(2n+1)z^{-n}
\end{equation}

Haar Wavelet in z notation is
$g(n)$ are the Father Wavelet coefficients whereas $h(n)$ are the corresponding mother coefficinents. 
\begin{equation}
{\mathcal {Z}} \{g(n)\} = 1 + z^-1
\end{equation}
\begin{equation}
{\mathcal {Z}} \{(n)\} = 1 - z^-1
\end{equation}


\section{Implementation}


\section{Conclusions}
\rhead{Schlussfolgerung}

\printbibliography[heading=subbibliography]
\end{refsection}
