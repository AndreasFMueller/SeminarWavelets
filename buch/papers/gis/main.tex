%
% main.tex -- Paper zum Thema gis
%
% (c) 2019 Hochschule Rapperswil
%
\chapter{Signalanalyse von UHF Teilentladungssignale im Zeitbereich \label{chapter:gis}}
\lhead{Signalanalyse von UHF Teilentladungssignale im Zeitbereich}
\begin{refsection}
\chapterauthor{Kris Wyss}

\section{Vorwort}
\rhead{Vorwort}

Gasisolierte Schaltanlagen (GIS) dienen der elektrischen Energieverteilung, dem Schutz der Komponenten und ermöglichen eine sichere Wartung des Energieübertragungsnetzes.
Als Isoliergas wird Schwefelhaxafluorid (SF6) eingesetzt. SF6 hat bei Atmosphärendruck eine etwa drei mal so grosse dielekrische Festigkeit wie Luft. 
Es besitzt auch um den Faktor 3-4 mal bessere Lichtbogenlöscheingeschaft als Luft. 
Diese Eigenschaften führen zu einer erheblichen Platzeinsparung gegenüber einer luftisolierten Schaltanlage (AIS). \cite{buch:ABB}
Deshalb und wegen einigen anderen Gründen, aufgezeigt in \cite{buch:GIS/AIS}, hat sich der Einsatz von GIS überall dort durchgesetzt wo Boden teuer ist. Zum Beispiel in urbanen Gebieten. Aber auch bei luftisolierten Schaltanlagen ab 72.5 kV sind die Leistungsschalter mit SF6 Isoliert. 
Bei Leistungsschaltern dient das SF6 als Löschgas des Lichtbogens, welcher bis zu mehren tausend Ampere trägt, wenn ein Teil des Übertragungsnetzes von der Last freigeschaltet wird. \cite{buch:ABB} 

Bei der Quantifizierung der Qualität einer GIS spielt der Begriff Teilentladung (TE) eine wichtige Rolle. 
Wenn das E-Feld örtlich über die dielektrische Festigkeit steigt entstehen elektrische Entladungen welche nur ein Teil der Isolationsstrecke überbrücken und nicht zu einem kompletten Durchschlag führen, daher die Namensgebung Teilentladung. \cite{buch:Kuchler}
Teilentladung im Isolierstoff lässt das Isoliermedium  schneller altern. Dies kann zum dielektrischen Durchschlag führen welcher einen Totalausfall der Anlage zur folge hat.
Aufgrund dessen ist TE ein wichtiges Qualitätsmerkmal bei Hochspannungskomponenten. 

Im Falle einer Teilentladung werden verschiedene  physikalische Prozesse gestartet. 
Die Entladung hat einen Strompuls zur folge.
Durch den Lichtbogen wird eine ultrahochfrequente (UHF) elektromagnetische Welle emittiert.
Die schnellste gemessene Anstiegszeit des Strompulses in SF6 liegt bei 24ps.
Dies entspricht einem Frequenzband hoch bis zu 14 GHz. \cite{skript:Judd24ps} 
Aufgrund der schnellen Erhitzung des Gases rund um den Lichtbogen wird danach eine Schallwelle abgestrahlt. 
Ebenfalls durch die grosse Erhitzung wird das Isoliergas SF6 zersetzt. 
Somit kann TE chemisch, akustisch oder elektromagnetisch gemessen und lokalisiert werden.\cite{skript:StatusReviewPDMeasurement}
In dieser Arbeit wird auf die Analyse der UHF TE-Signale eingegangen.
Es wird versucht, mittels dem Signalanalyseverfahren kontinuierliche Wavelet Transformation, fehlerspezifische Charakteristiken im Zeitsignal der elektromagnetischen Welle auszumachen.


\section{Einleitung}
\rhead{Abschnitt}

TE wird in zwei Grundkategorien unterteilt \cite{buch:Kuchler}, die Innere- und Äussere-TE. 
Die Innere-TE ist dadurch charakterisiert das sich die TE-Quelle im Innern eines festen oder flüssigen Isolierstoffes befindet. 
Spezifiziert auf GIS gehören zu Kategorie der Inneren-TE Lunker, Risse/Spalten in Isolierstoffe und Delamination zwischen Isolierschichten.
 
Entladungen an äusseren Leiterstrukturen wird als Äussere-TE oder Korona bezeichnet. 
In GIS kommen folgende Fehler aus dieser Kategorie vor, Oberflächenentladung und Koronaentladung. 
Oberflächenentladungen treten bei nicht einhalten der Herstellungstoleranzen, zwischen leitendem und isolierendem Material, auf.
Eine weitere Quelle sind potenzialfrei leitende Partikel auf festen Isolierstoffen.
Brauen an spannungsführenden- und an leitfähigen-Teilen auf Erdpotenzial führen zu Koronaentladung. \cite{buch:Kuchler}\cite{skript:AeussreTE}\cite{skript:InnereTE}
Hier werden UHF TE-Signale von Oberflächenentladung und Hohlraumentladungen analysiert. 
Im weiteren Abschnitt werfen wir einen genauer physikalischen Blick auf die zwei Endladungsarten und die herkömmliche Fehleranalyse.

\subsection{Oberflächenentladung}

Diese Endladungsart entsteht wenn an Oberflächen von Festisolierstoffen hohe elektrische Feldstärken auftreten. 
Dies kann Konstruktionsbedingt oder Aufgrund von unsauberer Montage entstehen.
Wenn sich ein Braue oberhalb eines Festisolierstoffes befindet oder ein loser metallischer Partikel auf einem Isolierstoff, dann entladen sich die Stromimpulse über dem Isoliermaterial.
%Bild..... 

Das Ersatzschaltbild besteht aus einer Kapazität C parallel zu einer Funkenstrecke F und einem Widerstand R in Serie. 
Wenn das inhomogene elektrische Feld über der Kapazität C einen kritischen Wert überschreitet entlädt sich die gespeicherte Energie über die Funkenstrecke F gegen Erde.

\begin{figure}[H]
	\centering
\begin{circuitikz} \draw
(0,1)node[vee]{Spitze} (0,1)--(0,0)
to[C=C] (0,-3)
to[R=R]  (0,-5)
node[rground]{}

(0, 0) -- (1.5, 0.0) -- (1.5, -1) 
to[american gas filled surge arrester=F] (1.5, -2) -- (1.5, -3) -- (0,-3)
	;
\end{circuitikz}
\caption{ESB: Koronaentladung} \label{fig:M1}
\end{figure}


\subsection{Hohlraumentladung}

%Fehlerarten -> ESB Fehlstellen -> Messverfahren (TE repetitiv 50Hz)für PRPD Pattern UHF FFT, etc.....-> 

\section{Datenbearbeitung}
\rhead{Abschnitt}

\section{Auswertung}
\rhead{Abschnitt}

\section{Schlussfolgerung}
\rhead{Schlussfolgerung}

\printbibliography[heading=subbibliography]
\end{refsection}
