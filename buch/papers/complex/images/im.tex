%
% sweep.tex -- template for standalon tikz images
%
% (c) 2019 Prof Dr Andreas Müller, Hochschule Rapperswil
% (c) 2019 Roy Seitz, Hochschule Rapperswil
%
\documentclass[tikz]{standalone}
\usepackage{amsmath}
\usepackage{times}
\usepackage{txfonts}
\usepackage{pgfplots}
\usepackage{csvsimple}
\usetikzlibrary{arrows,intersections,math}
\begin{document}
\begin{tikzpicture}[>=latex]

% Can be overwritten by compiling with: pdflatex '\newcommand{\imname}{im.jpg} %
% sweep.tex -- template for standalon tikz images
%
% (c) 2019 Prof Dr Andreas Müller, Hochschule Rapperswil
% (c) 2019 Roy Seitz, Hochschule Rapperswil
%
\documentclass[tikz]{standalone}
\usepackage{amsmath}
\usepackage{times}
\usepackage{txfonts}
\usepackage{pgfplots}
\usepackage{csvsimple}
\usetikzlibrary{arrows,intersections,math}
\begin{document}
\begin{tikzpicture}[>=latex]

\pgfmathparse{(12/15*9/16)}
\xdef\s{\pgfmathresult} % scale to picture width

\pgfmathparse{(12/4)} % scale to tend
\xdef\a{\pgfmathresult} 


\node at (6,{6*9/16}) {\includegraphics[width=12cm]{im.jpg}};

\draw[->,line width=0.7pt] (-0.1,0)--(12.3,0) coordinate[label={$b$}];
\draw[->,line width=0.7pt] (0,-0.1)--(0,{12*9/16+0.3})
	coordinate[label={right:$1/a$}];

\foreach \x in {1, 2, 4, ..., 16}{
	\draw[line width=0.7pt] (-0.1,{(\x-1)*\s})--(0.1,{(\x-1)*\s});
	\node at (-0.1,{(\x-1)*\s}) [left] {$\x$};
}

\foreach \x in {1, ..., 4}{
	\draw[line width=0.7pt] ({\x*\a}, -0.1) -- ({\x*\a}, 0.1);
	\node at ({\x*\a}, -0.1) [below] {$\x$};
}

%\begin{scope}[yshift=-1.8cm]
%	\draw[color=red,line width=1pt]
%		plot[domain=0:4,samples=4000]
%			({\x*\a},{sin(360*\x*(2 + 6/\a * \x))});
%	\draw[->,line width=0.7pt] (-0.1,0)--(12.3,0) coordinate[label={$t$}];
%	\draw[->,line width=0.7pt] (0,-1.1)--(0,1.3) coordinate[label={right:$x(t)$}];
%	
%	\foreach \x in {0,...,4}{
%		\draw[line width=0.7pt]
%			({\x*\a},-0.1)--({\x*\a},0.1);
%	}
%\end{scope}

%\begin{scope}[yshift=-1.8cm]
%	\draw[color=red,line width=1pt]
%		plot[domain=0:4,samples=4000]
%			({\x*\a},{sin(360*\x*(6 + 2 * sign(sin(180*\x)) ))});
%	\draw[->,line width=0.7pt] (-0.1,0)--(12.3,0) coordinate[label={$t$}];
%	\draw[->,line width=0.7pt] (0,-1.1)--(0,1.3) coordinate[label={right:$x(t)$}];
%	
%	\foreach \x in {0,...,4}{
%		\draw[line width=0.7pt]
%			({\x*\a},-0.1)--({\x*\a},0.1);
%	}
%\end{scope}


\end{tikzpicture}
\end{document}

'
\providecommand{\imname}{im.jpg} 
\providecommand{\amax}{16}

\pgfmathparse{(12/(\amax-1)*9/16)}
\xdef\s{\pgfmathresult} % scale to picture width

\pgfmathparse{(12/4)} % scale to tend
\xdef\a{\pgfmathresult} 

\node at (6,{6*9/16}) {\includegraphics[width=12cm]{\imname}};

\draw[->,line width=0.75pt] (-0.1,0)--(12.3,0) coordinate[label={$b$}];
\draw[->,line width=0.75pt] (0,-0.1)--(0,{12*9/16+0.3})
	coordinate[label={right:$1/a$}];

\foreach \x in {1, 2, 4, ..., \amax}{
	\draw[line width=0.75pt] (-0.1,{(\x-1)*\s})--(0,{(\x-1)*\s});
	\node at (-0.1,{(\x-1)*\s}) [left] {$\x$};
}

\foreach \x in {1, ..., 4}{
	\draw[line width=0.75pt] ({\x*\a}, -0.1) -- ({\x*\a}, 0);
	\node at ({\x*\a}, -0.1) [below] {$\x$};
}

%\begin{scope}[yshift=-1.8cm]
%	\draw[color=red,line width=1pt]
%		plot[domain=0:4,samples=4000]
%			({\x*\a},{sin(360*\x*(2 + 6/\a * \x))});
%	\draw[->,line width=0.7pt] (-0.1,0)--(12.3,0) coordinate[label={$t$}];
%	\draw[->,line width=0.7pt] (0,-1.1)--(0,1.3) coordinate[label={right:$x(t)$}];
%	
%	\foreach \x in {0,...,4}{
%		\draw[line width=0.7pt]
%			({\x*\a},-0.1)--({\x*\a},0.1);
%	}
%\end{scope}

%\begin{scope}[yshift=-1.8cm]
%	\draw[color=red,line width=1pt]
%		plot[domain=0:4,samples=4000]
%			({\x*\a},{sin(360*\x*(6 + 2 * sign(sin(180*\x)) ))});
%	\draw[->,line width=0.7pt] (-0.1,0)--(12.3,0) coordinate[label={$t$}];
%	\draw[->,line width=0.7pt] (0,-1.1)--(0,1.3) coordinate[label={right:$x(t)$}];
%	
%	\foreach \x in {0,...,4}{
%		\draw[line width=0.7pt]
%			({\x*\a},-0.1)--({\x*\a},0.1);
%	}
%\end{scope}


\end{tikzpicture}
\end{document}

