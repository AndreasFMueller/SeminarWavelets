\subsection{Analytische Signale und die Hilbert-Transformation}
\rhead{Hilbert-Transformation}
Im ersten Kapitel haben wir aus dem reellen Gabor-Wavelet und dem Wunsch nach Separation von Amplitude und Phase das komplexe Morlet-Wavelet gefunden.
Hierzu haben wir die negativen Frequenzen aus dem Spektrum des Wavelets entfernt.
Dieses Verfahren lässt isch mit der Hilbert-Transformation
\[
	\mathcal{H}f(t) =
	\frac{1}{\pi}\int_{-\infty}^{\infty}\frac{f(x)}{t-x} dx
\]
elegant verallgemeinern.
Hierzu benötigen wir den Begriff des analytischen Signals.
\begin{definition}
	Sei $f(t) \in \mathbb{R}$ ein reelles Zeitsignal, dann heisst
	\begin{align*}
		f^\ast(t) 
		&= f(t) + i\,\mathcal{H}f(t)
	\end{align*}
	das zu $f(t)$ \emph{analytische Signal}\footnote{Der Begriff `analytisch' ist in diesem Kapitel immer im Sinne dieser Definition aus der Signaltheorie zu verstehen.
	Er ist nicht zu verwechseln mit der Eigenschaft analytischer Funktionen in der Analysis.
%	Die Analysis kennt den Begriff der analytischen Funktion. 
%	Diese Eigenschaft ist in der komplexen Analysis jedoch äquivalent zur Holomorphie.
%	Wo benötigt wird folglich von holomorphen Funktionen gesprochen, um Verwechslungen zu vermeiden.
	}.
\end{definition}

\begin{satz}
	Die Fourier-Transformierte eines analytischen Signals verschwindet für alle negativen Frequenzen.
	Es gilt
	\[
		\hat{f}^\ast_{a,b}(\omega) = \left\lbrace\begin{matrix*}[r]	
			2\hat{f}_{a,b} & 0 \le \omega \\ 0 & \text{sonst}\end{matrix*} \right..
	\]
\end{satz}

\begin{proof}[Beweis]
	Die Hilbert-Transformation besitzt die Form eines Faltungs-Integrals.
	Es gilt
	\begin{align*}
		\mathcal{H} f(t) &= f(t) * \frac{1}{\pi t}.
	\end{align*}
	Diese Faltung wird im Frequenzbereich zur Multiplikation.
	Mit der Signumsfunktion $\sgn(\omega)$ und der Identität
	\begin{align*}
		\mathcal{F} \left\lbrace t^{-1}\right\rbrace  &= -i\pi\sgn(\omega)
	\end{align*}
	wird die Hilberttransformation im Frequenzbereich zu
	\begin{align*}
		\mathcal{F}\left\lbrace \mathcal{H}f(t)\right\rbrace 
		&= -i\sgn(\omega) \hat{f}.
	\end{align*}
	Für die Fourier-Transformierte von $f^\ast_{a,b}$ gilt folglich 
	\begin{align*}
		\hat{f}^\ast_{a,b} 
		&= \hat{f}_{a,b} - i^2\sgn(\omega)\hat{f}_{a,b}\\
		&=\left\lbrace\begin{matrix}
			2\hat{f}_{a,b} & \omega > 0\\
			0 & \omega \le 0
		\end{matrix} \right.
	\end{align*}
\end{proof}

Mittels Hilbert-Transformation kann man also aus jedem beliebigen, reellen Zeitsignal ein analytisches Signal konstruieren.
Spektral ist dies gleichbedeutend damit, die negativen Frequenzen zu entfernen und die positiven zu verdoppeln, wie wir das im ersten Kapitel auf dem Weg vom Gabor- zum Morlet-Wavelet auch getan haben.
Tatsächlich ist das Morlet-Wavelet das zum Gabor-Wavelet analytische Wavelet.
Die Norm erhöht sich hierbei jedoch um den Faktor $\sqrt{2}$.
Beim Gabor- und Morlet-Wavelet versteckt sich dieser Faktor im $c_\sigma$.
