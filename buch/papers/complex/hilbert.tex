\section{Analytische Wavelets}
Im vorherigen Abschnitt haben wir negative Frequenzen als Problem identifiziert.
Am Beispiel des Gabor-Wavelets suchten wir eine Lösung und fanden so das Morlet-Wavelet.
Dazu mussten alle negativen Frequenzen aus dem Spektrum des Wavelets entfernt werden.
Wie lässt sich dieses Verfahren veralgemeinern? 

Die Signaltheorie kennt genau dieses Verfahren in der Einseitenband-Modulation.
Wir entlehnen uns von dort den Begriff des \emph{analytischen Signals}\footnote{
	Der Begriff `analytisch' ist in diesem Kapitel immer im Sinne der Signaltheorie zu verstehen, also $\forall \omega < 0 \colon \hat f (\omega) = 0 $.
	Er ist nicht zu verwechseln mit der Eigenschaft analytischer Funktionen in der Analysis.
}
und defineiren analog dazu \emph{analytische Wavelets}.

\subsection{Analytische Signale und die Hilbert-Transformation}
\rhead{Hilbert-Transformation}
Bevor wir den Begriff des analytischen Signals einführen können, benötigen wir die Hilbert-Transformation
\[
\Hilb f(t) =
\frac{1}{\pi} \CH\int_{-\infty}^{\infty}\frac{f(x)}{t-x} \mathrm{d}x.
\]
Hierbei bezeichnet $\CH\int_{-\infty}^{\infty} \dots \mathrm{d}x$ den cauchysche Hauptwert des divergenten Integrals.
Diese Integral-Transformation wird uns erlauben, zu einem reellen Signal einen Imaginärteil zu finden, so dass gerade alle negativen Frequenzen im verscheinden.
Damit sind wir bereit für die Definition des analytischen Signals.

\begin{satz}
	\label{complex:analytic-signal}
	Sei $f(t) \in \mathbb{R}$ ein reelles Signal, dann heisst
	\begin{align*}
		f^\ast(t) 
		&= (1 + i\Hilb\,)f(t)
	\end{align*}
	das $f(t)$ zugeordnete \emph{analytische Signal}.
	Die Fourier-Transformierte eines analytischen Signals verschwindet für alle negativen Frequenzen.
	\[
		\forall \omega < 0 \colon \hat{f}^\ast(\omega) = 0
	\]
\end{satz}

\begin{proof}[Beweis]
	
	Die Hilbert-Transformation besitzt die Form eines Faltungs-Integrals.
	Im Frequenzbereich wird daraus eine Multiplikation.
	Es gilt mit der Signumsfunktion $\sgn(\omega)$ und der Identität
	\begin{align*}
		\Four \frac{1}{\pi t}  &= -i\sgn(\omega), \\
		\Hilb f(t) &= \frac{1}{\pi t} * f(t)\\
		\Four\Hilb f(t) &= -i\sgn(\omega) \hat{f}(\omega).
	\end{align*}
	Für die Fourier-Transformierte $\hat f^\ast(\omega)$ gilt folglich 
	\begin{align*}
		\hat{f}^\ast(\omega) 
		&= \hat{f}(\omega) - i^2\sgn(\omega)\hat{f}(\omega)\\
		&= (1 +\sgn(\omega))\hat{f}(\omega)\qedhere
	\end{align*}
\end{proof}

\begin{lemma}\label{complex:re-f-ast}
	Die Realteile von $f(t)$ und $f^\ast(t)$ sind identisch.
\end{lemma}

\begin{proof}
	Die Hilbert-Transformation ist eine reelle Integraltransformation
	\[\Hilb\,\colon\mathbb{R}\to\mathbb{R}.\]
	Folglich gilt nach Definition
	\[\Re f^\ast(t) = f(t) \quad \text{und}\quad \Im f^\ast(t) = \Hilb f(t)\qedhere\]
\end{proof}
Lemma~\ref{complex:re-f-ast} hält fest, dass wir uns durch die Hilbert-Transformation nicht zu weit vom Originalsignal entfernen.