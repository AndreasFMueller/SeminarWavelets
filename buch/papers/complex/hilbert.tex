\subsection{Analytische Signale und die Hilbert-Transformation}
\rhead{Hilbert-Transformation}
Im vorherigen Abschnitt haben wir aus dem reellen Gabor-Wavelet und dem Wunsch nach Separation von Amplitude und Phase das komplexe Morlet-Wavelet gefunden.
Wir fanden heraus, dass dazu alle negativen Frequenzen aus dem Spektrum des Wavelets entfernt werden müssen.
Dieses Verfahren lässt sich mit der Hilbert-Transformation
\[
	\mathcal{H}f(t) =
	\frac{1}{\pi} \CH\int_{-\infty}^{\infty}\frac{f(x)}{t-x} dx
\]
elegant verallgemeinern.
Hierbei bezeichnet $\CH\int_{-\infty}^{\infty} \dots \mathrm{d}x$ den cauchysche Hauptwert des divergenten Integrals.
Hierzu benötigen wir den Begriff des analytischen Signals.
\begin{definition}
	Sei $f(t) \in \mathbb{R}$ ein reelles Zeitsignal, dann heisst
	\begin{align*}
		f^\ast(t) 
		&= f(t) + i\,\mathcal{H}f(t)
	\end{align*}
	das zu $f(t)$ zugeordnete \emph{analytische Signal}\footnote{Der Begriff `analytisch' ist in diesem Kapitel immer im Sinne dieser Definition aus der Signaltheorie zu verstehen.
	Er ist nicht zu verwechseln mit der Eigenschaft analytischer Funktionen in der Analysis.
	}.
\end{definition}

\begin{satz}
	Die Fourier-Transformierte eines analytischen Signals verschwindet für alle negativen Frequenzen.
	Es gilt
	\[
		\hat{f}^\ast(\omega) = \left\lbrace\begin{matrix*}[r]	
			2\hat{f} & 0 < \omega \\ \hat{f} & \omega= 0 \\ 0 & \omega > 0\end{matrix*} \right..
	\]
\end{satz}

\begin{proof}[Beweis]
	Die Hilbert-Transformation besitzt die Form eines Faltungs-Integrals.
	Im Frequenzbereich wird daraus eine Multiplikation.
	Es gilt mit der Signumsfunktion $\sgn(\omega)$ und der Identität
	\begin{align*}
		\mathcal{F} \left\lbrace t^{-1}\right\rbrace  &= -i\pi\sgn(\omega) \\
		\mathcal{H} f(t) &= f(t) * \frac{1}{\pi t}. \\
		\mathcal{F}\left\lbrace \mathcal{H}f(t)\right\rbrace &= -i\sgn(\omega) \hat{f}.
	\end{align*}
	Für die Fourier-Transformierte $\hat f^\ast$ gilt folglich 
	\begin{align*}
		\hat{f}^\ast 
		&= \hat{f} - i^2\sgn(\omega)\hat{f}\\
		&= \hat{f}(1 +\sgn(\omega))\\
		&=\left\lbrace\begin{matrix}
			2\hat{f} & \omega > 0\\
			\hat{f} & \omega = 0\\
			0 & \omega < 0
		\end{matrix} \right.\qedhere
	\end{align*}
\end{proof}

Ohne Beweis gilt des Weiteren folgender Satz. %TODO Beweis!
\begin{satz}
	Sei $f \in C^\infty(\mathbb{R})$ ein glattes, reelles Signal, dann ist 
	\[
		f^\ast = f + i\Hilb f \in C^\infty(\mathbb{C})
	\]
	holomorph.
\end{satz}
Im Komplexen sind die Begriffe \emph{holomorph}, \emph{analytisch} und \emph{komplex differenzierbar} äquivalent.
Dies erklärt die Herkunft der Bezeichnung \emph{analytisches Signal}.