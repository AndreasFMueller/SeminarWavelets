\subsection{Analytische Signale und die Hilbert-Transformation}
\rhead{Hilbert-Transformation}
Im vorherigen Abschnitt haben wir aus dem reellen Gabor-Wavelet und dem Wunsch nach Separation von Amplitude und Phase das komplexe Morlet-Wavelet gefunden.
Wir fanden heraus, dass dazu alle negativen Frequenzen aus dem Spektrum des Wavelets entfernt werden müssen.

Die Signaltheorie kennt genau dieses Verfahren in der Einseitenband-Modulation.
Wir entlehnen uns von dort den Begriff des \emph{analytischen Signals}\footnote{
Der Begriff `analytisch' ist in diesem Kapitel immer im Sinne der Signaltheorie zu verstehen, also $\hat f (\omega) = 0 \forall \omega < 0$.
Er ist nicht zu verwechseln mit der Eigenschaft analytischer Funktionen in der Analysis.
}.
Dazu benötigen wir die Hilbert-Transformation
\[
\mathcal{H}f(t) =
\frac{1}{\pi} \CH\int_{-\infty}^{\infty}\frac{f(x)}{t-x} \mathrm{d}x,
\]
wobei $\CH\int_{-\infty}^{\infty} \dots \mathrm{d}x$ den cauchysche Hauptwert des divergenten Integrals bezeichnet.
Damit sind wir bereit für die Definition des analytischen Signals.
\pagebreak
\begin{satz}
	\label{complex:analytic-signal}
	Sei $f(t) \in \mathbb{R}$ ein reelles Zeitsignal, dann heisst
	\begin{align*}
		f^\ast(t) 
		&= f(t) + i\,\mathcal{H}f(t)
	\end{align*}
	das zu $f(t)$ zugeordnete \emph{analytische Signal}.
	Die Fourier-Transformierte eines analytischen Signals verschwindet für alle negativen Frequenzen.
	\[
		\hat{f}^\ast(\omega) = \left\lbrace\begin{matrix*}[r]	
			2\hat{f} & 0 > \omega \\ \hat{f} & \omega= 0 \\ 0 & \omega < 0\end{matrix*} \right..
	\]
\end{satz}

\begin{proof}[Beweis]
	
	Die Hilbert-Transformation besitzt die Form eines Faltungs-Integrals.
	Im Frequenzbereich wird daraus eine Multiplikation.
	Es gilt mit der Signumsfunktion $\sgn(\omega)$ und der Identität
	\begin{align*}
		\mathcal{F} \left\lbrace t^{-1}\right\rbrace  &= -i\pi\sgn(\omega), \\
		\mathcal{H} f(t) &= f(t) * \frac{1}{\pi t} \\
		\mathcal{F}\left\lbrace \mathcal{H}f(t)\right\rbrace &= -i\sgn(\omega) \hat{f}.
	\end{align*}
	Für die Fourier-Transformierte $\hat f^\ast$ gilt folglich 
	\begin{align*}
		\hat{f}^\ast 
		&= \hat{f} - i^2\sgn(\omega)\hat{f}\\
		&= \hat{f}(1 +\sgn(\omega))\qedhere
	\end{align*}
\end{proof}