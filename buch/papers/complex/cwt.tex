\section{Effiziente Berechnung der CWT}
\rhead{Berechnung der CWT}

Betrachten wir zuerst die kontinuierliche Wavelet-Transformation aus Gleichung~\eqref{cwt:definition:eq},
\begin{equation}
\mathcal{W}f (a,b)
=
\langle f,\psi_{a,b}\rangle
=
\frac{1}{\sqrt{|a|}}\int_{-\infty}^\infty f(t)\,
	\overline{\psi\left(\frac{t-b}{a}\right)}\,\mathrm{d}t.\label{complex:CWT}
\end{equation}
Dieses Integral entspricht der Faltung zwischen $f(t)$ und 
\begin{equation} 
    g(t) 
    = \frac{1}{\sqrt{|a|}} \overline{\psi\left(\frac{-t}{a}\right)} \,.
\end{equation}
Der Standard-Trick zur effizienten Berechnung der Faltung ist die Multiplikation im Frequenzbereich.
\begin{equation} 
\mathcal{W}f (a,b) = (f*g)(t) = \mathcal{F}^{-1}\left\lbrace\hat f(\omega) \cdot \hat g (\omega) \right\rbrace
\end{equation}

Dafür benötigen wir die Fouriertransformierte $\hat g (\omega)$:
\begin{align*}
	\hat g (\omega) = 
    \mathcal{F}\left\lbrace \frac{1}{\sqrt{|a|}} \overline{\psi\left(\frac{-t}{a}\right)}\right\rbrace 
	&= \frac{1}{\sqrt{|a|}} \int_{-\infty}^{\infty}\overline{\psi\left(\frac{-t}{a}\right)}\,e^{-i\omega t}\,\mathrm{d}t\\
	&= \frac{1}{\sqrt{|a|}} \overline{\int_{-\infty}^{\infty}\psi\left(\frac{-t}{a}\right)e^{i\omega t}\,\mathrm{d}t}  
    & \left(\text{Subst. } t' = \frac{-t}{a}\right)\\
	&= \frac{1}{\sqrt{|a|}} \overline{\int_{-\infty}^{\infty}\psi\left(t'\right)e^{-ia\omega t'} |a|\,\mathrm{d}t'}\\
	&= \sqrt{|a|} \, \overline{\hat{\psi}\left(a\omega\right)}.
\end{align*}
Gleichung~\eqref{complex:CWT} lässt sich also schreiben als
\begin{equation}
\mathcal{W}f(a,b)
= \mathcal{F}^{-1}\left\lbrace\hat{f}(\omega)\cdot\! \sqrt{|a|}\, \overline{\hat{\psi}\left(a\omega\right)}\right\rbrace. \label{complex:fcwt}
\end{equation}

Mittels Fourier-Transformation lässt sich die Wavelet-Transformation also besonders elegant berechnen.
Kontinuierliche Funktionen sind für numerische Systeme jedoch ungeeignet.
Die CWT muss in $a$ und $b$ diskretisiert werden.
Die Diskretisierung von $b$ entspricht vorteilhaft gerade der jener des Signals selbst.
Dann lässt sich die Fourier-Transformation mittels FFT effizient berechnen und Gleichung~\eqref{complex:fcwt} wird zu
\begin{equation}
	\mathcal{W}f(a,b) = \text{IFFT}\left(\text{FFT}(f) \cdot \overline{\hat{\psi}\left(a\omega\right)}\right)\footnote{Der Faktor $\sqrt{|a|}$ wurde hierbei weggelassen.
		Hierdurch werden die hohen Frequenzen stärker gewichtet und $|\mathcal{W}f(a,b)|$ ist gerade proportional zur Amplitude der analysierten Signalkomponente.
        Zudem erzielen wir im Diskreten nicht exakt die Faltung, sondern die zirkuläre Version davon. 
        Mehr dazu im Abschnitt~\ref{complex:circ-conv-padding}.
	}. \label{complex:ffcwt}
\end{equation}

Diese Gleichung muss für jedes $a$ einzeln gelöst werden.
Diese Gleichung wird besonders interessant, wenn das Wavelet im Frequenzbereich eine geschlossene, analytische Form besitzt.
Dann benötigt man nur eine FFT für das Signal, so wie für jedes $a$ eine inverse FFT und eine punktweise Multiplikation zwischen Signal und Wavelet.

\subsection{Beispiel -- CWT mit Haar-Wavelet}
Für ein Beispiel benötigen wir zwei Dinge: ein Wavelet und ein Beispielsignal.
Nehmen wir für den Anfang das einfachste aller Wavelets, das Haar-Wavelet.
Als Signal nehmen wir zwei Cosinus-Schwingungen, einmal mit exponentiell ansteigender Frequenz, und einmal mit stückweise konstanter Frequenz.
\begin{align}
    x_1(t) &= \cos\left( \int_{0}^{t} 2\pi f_1(t')\,\mathrm{d}t'\right) & f_1(t) &= 10^{2 \cdot t/10} \\
    x_2(t) &= \cos\left( \int_{0}^{t} 2\pi x_2(t')\,\mathrm{d}t'\right) & f_2(t) &= \left\lbrace \begin{matrix}
    2, & &t& < 2.5\\
    8, & 2.5 \le &t& < 5.0\\
    2, & 5.0 \le &t& < 7.5\\
    8, & 7.5 \le &t&\\
    \end{matrix}\right.
\end{align}
\begin{figure}
    \centering
	\includegraphics{papers/complex/images/signals.pdf}
    \caption{Die beiden Beispielsignale $f_1(t)$ und $f_2(t)$}
\end{figure}
Zudem benötigen wir das Haar-Wavelet und seine Fourier-Transformierte.
In diesem Abschnitt sei das Wavelet zentriert um $t=0$.
Dies daraus resultierende Symmetrie wird sich in der Berechnung der Fourier-Transformation als hilfreich erweisen.

\begin{definition}
	\label{complex:def-haar-wavelet}
	Das zentrierte, reelle Haar-Wavelet besitzt folgende Gestalt:
	\[
	\psi_{\text{Haar}} = \left\lbrace\begin{matrix*}[r]
	1 & -\frac{1}{2} \le t < 0  \\
	-1 & 0 \le t < \frac{1}{2} \\
	0 & \text{sonst}
	\end{matrix*} \right.\label{complex:def-haar}
	\]
\end{definition}
Die Fourier-Transformierte von $\psi_{\text{Haar}}$ berechnet sich wie folgt:
\begin{align}
	\mathcal F \psi_\text{Haar}  
	&= \frac{1}{\sqrt{2\pi}}\int_{-\infty}^{\infty} \psi_\text{Haar} e^{-i\omega t} \,\mathrm{d}t\nonumber\\
	&= \frac{1}{\sqrt{2\pi}}\left( \int_{-1/2}^{0} e^{-i\omega t} \,\mathrm{d}t - \int_{0}^{1/2} e^{-i\omega t}\,\mathrm{d}t \right) \nonumber\\
	&= \frac{1i}{\sqrt{2\pi}\omega}\left( \left[ e^{-i\omega t}\right]_{-1/2}^0  - \left[ e^{-i\omega t}\right]_{0}^{1/2} \right)\nonumber\\
	&= \frac{1i}{\sqrt{2\pi}}\left( \frac{1-\cos(\omega/2)}{\omega/2}\right)\label{complex:f-psi-haar}
\end{align}
Das Haar-Wavelet ist also nicht nur im Zeitbereich besonder einfach, sondern auch im Frequenzbereich.
Insbesondere lässt sich die mit $a$ skalierte Version des Wavelets durch Satz~\ref{four-int:trans-dial} direkt im Frequenzbereich berechnen.

Somit haben wir für unser Beispiel alles zusammen.
Nach einer Diskretisierung der Variablen $t$, $\omega$ und $a$ überlassen wir die Arbeit dem Computer.
Das verwendete Skript findet sich auf der CD, welche diesem Buch beiliegt (buch/papers/complex/fcwt/test\_matlab.m, siehe Abschnitt~\ref{complex:circ-conv-padding})

% TODO: Weitermachen hier!
\section{Old Stuff -- überarbeiten!}

Sei \verb|psi_hat| eine Matrix, in deren Spalten die Fourier-Transformierten des Wavelets für jedes $a$ stehen,
und sei \verb|f| das zu analysierende Signal $f(t)$ als Spaltenvektor,
dann lässt sich die Wavelet-Transformation in Matlab oder Octave mit folgendem Code einfach berechnen.

\begin{lstlisting}[style=Matlab-editor]
% Example for Morlet-Wavelet
function psi_hat = myMorlet(w, a)
  sigma = 2 * pi; % Sigma for time vs freq. localisation
  c_sigma = pi^(-1/4) / ... Scaling factor for norm = 1
            sqrt(1 + exp(-sigma^2) - 2 * exp(-3/4 * sigma^2));
  k = exp(-sigma^2 / 2 ); % Kappa for DC-free wavelet
  psi_hat = c_sigma .* (exp(-(sigma - w).^2 ./ (2*a).^2) - k);
end

% Column freqency vector
w = linspace(0, fs, length(f))' * 2 * pi; 
% Interesting a values as row vector
a = 1:10; 
psi_hat = myMorlet(w, a); % Wavelet in frequency domain
yab = ifft(FFT(f) .* psi_hat); % Wavelet-Transform
\end{lstlisting}

%% Wavelet-TraFo als Faltung zwischen Signal und Wavelet
%% Morlet-Wavelet in Frequenzbereich geschlossen berechenbar
%%  -> Faltungs-Matrix mit verschieden skalierten Morlet-Wavelets in den Spalten
%%   - Matrix-Multiplikation mit Signal
%%   - ifft über die Spalten
