\subsection{Analytische Wavelets und deren Norm}
\rhead{Analytische WAvelets}

Die Norm eines Wavelets muss $1$ sein.
Folglich muss die Norm und jene, des zum Original-Wavelet analytischen Wavelets gleich sein.
Bei analytischen Signalen ist dies jedoch nicht gegeben.
Dies führt zur
\begin{definition}
	Sei $\psi$ ein beliebiges, reelles Wavelet. Dann ist
	\begin{equation}
		\psi^\ast = \frac{1}{\sqrt{2}}\left(\psi + i\,\mathcal{H}\psi\right) \label{complex:anawave}
	\end{equation}
	das zu $\psi$ \emph{analytische Wavelet}.
\end{definition}

Für die Norm eines analytischen Wavelets gilt
\begin{satz}
	\label{complex:norm}
	Die Norm eines analytischen Wavelets ist gleich der Norm des Original-Wavelets.
	\[\left\|\psi\right\| = \left\|\psi^\ast\right\|\].
\end{satz}

\begin{proof}
	Wir rechnen nach:
	\begin{align*}
		\left\|\psi\right\| = \left\|\hat{\psi}\right\|
		&= \int_{-\infty}^{\infty}\left|\hat{\psi}(\omega)\right|^2 d\omega \\
		&= \int_{-\infty}^{0}\left|\hat{\psi}(\omega)\right|^2 d\omega +  \int_{0}^{\infty}\left|\hat{\psi}(\omega)\right|^2 d\omega \\
		&=  2\int_{0}^{\infty}\left|\hat{\psi}(\omega)\right|^2 d\omega \\
		&=  \int_{0}^{\infty}\left|\frac{2}{\sqrt{2}}\hat{\psi}(\omega)\right|^2 d\omega \\
		&=  \int_{-\infty}^{\infty}\left|\hat{\psi}^\ast(\omega)\right|^2 d\omega 
		= \left\|\hat{\psi}^\ast\right\| = \left\|\psi^\ast\right\|.
	\end{align*}
	Hierbei haben wir zu begin und am Ende die Placherel-Formel verwendet.
	In Zeile drei nutzten wir die hermitesche Symmetrie der Fourier-Transformierten eines reellen Wavelets, nach welcher
	\[\left|\hat{f}(\omega)\right| = \left|\hat{f}(-\omega)\right|\]
	gilt.
	Somit ist Satz~\ref{complex:norm} bewiesen.
\end{proof}
