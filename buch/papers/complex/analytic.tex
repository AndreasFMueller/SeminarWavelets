\section{Analytische Wavelets}
Der Operator $\Ana\,$ hat den ersten Test also bestanden.
Nun möchten wir ihn noch etwas besser verstehen.
Was genau passiert eigentlich?
Ist es Zufall, dass das Morlet- und Gabor-Wavelet so ähnlich sind?
Dass der Realteil beim Morlet-Wavelet -- bis auf Skalierung -- genau dem Gabor-Wavelet entspricht?
Eine Anforderung an den Operator $\Ana\,$ war ja, dass er das Wavelet nur so wenig wie möglich ändern soll.
Ist diese Eigenschaft im Allgemeinen erfüllt?
Diese Fragen möchten wir in diesem Abschnitt beantworten.

Im vorherigen Abschnitt haben wir negative Frequenzen als Problem identifiziert.
Wir definierten den Operator $\Ana\,$ so, dass er genau die negativen Frequenzen entfernt und dabei die Norm erhällt.
Dazu wechselten wir in den Frequenzbereich, nahmen ein paar Änderungen vor und wechselten wieder zurück.
Wie lässt sich dieses Verfahren veralgemeinern? 
Ist dieses Hin- und Herwechseln tatsächlich notwendig?
Und was ist denn der Effekt im Zeitbereich?

Wir machen einen kurzen Ausflug in die Signaltheorie.


\subsection{Analytische Signale und die Hilbert-Transformation}
\rhead{Hilbert-Transformation}
In der Nachrichtentechnik ist Bandbreite ein beschränktes und dadurch rares, wertvolles Gut.
Reelle Signale haben aber immer hermitesch-symmetrische Spektra.
Es reicht also, die Hälfte des Spektrums eines Signals zu übertragen, etwa nur die positiven Frequenzen.
In der Signaltheorie ist dieses Verfahren unter dem Namen Einseitenband-Modulation bekannt.

Ein Signal, bei welchem die negativen Frequenzen entfernt wurden, nennt man \emph{analytisches Signals}\footnote{
	Der Begriff `analytisch' ist in diesem Kapitel immer im Sinne der Signaltheorie zu verstehen, also $\forall \omega < 0 \colon \hat f (\omega) = 0 $.
	Er ist nicht zu verwechseln mit der Eigenschaft analytischer Funktionen in der Analysis.
}.
Wir werden analog dazu \emph{analytische Wavelets} definieren und zeigen, dass der Auslöschungsoperator genau dies tut.
Dadurch übertragen sich wesentliche Eigneschaften alanytischer Signale auf unsere neuen Wavelets.

Bevor wir den Begriff des analytischen Signals einführen können, benötigen wir aber die Hilbert-Transformation.
\begin{definition}
	Die Operator
 	\[
 	\Hilb\,\colon L^2(\mathbb R) \to L^2(\mathbb R)
 	~\quad~
 	f(t) \mapsto \Hilb f(t)
 	= \frac{1}{\pi} \CH\int_{-\infty}^{\infty}\frac{f(x)}{t-x} \mathrm{d}x
 	\]
 	heist \emph{Hilbert-Transformation}.
 	Hierbei bezeichnet $\CH\int_{-\infty}^{\infty} \dots \mathrm{d}x$ den cauchysche Hauptwert des divergenten Integrals.
\end{definition}

Die Hilberttransformation ist, wie die Wavelet- oder Fouriertransformation, eine Integraltransformation.
Ein wesentlicher Unterschied besteht jedoch darin, dass sie den Raum nicht wechselt.
Es ist eine Transformation aus der Zeit, in die Zeit.

Nun sind wir bereit für die Definition eines analytischen Signals.
\begin{definition}
	\label{complex:analytic-signal}
	Sei $f \in L^2(\mathbb R)$ ein reelles Signal.
	Dann heisst
	\[f^\ast = (1 + i\Hilb\,)f \]
	das $f$ zugeordnete \emph{analytische Signal}.
\end{definition}
\begin{satz}
	Ein analytisches Signal wird durch den Auslöschungsoperator erzeugt.
	\[1 + i\Hilb\, \equiv \sqrt 2 \Ana\, \Rightarrow f^\ast \equiv \sqrt 2 \Ana f\]
\end{satz}

\begin{proof}
	Der Auslöschungsoperator wechselt in den Frequenzbereich.
	Dort berechnet er die punktweise Multiplikation mit der Signumsfunktion und wechselt wieder zurück in den Zeitbereich.
	\[\Ana\, = \mathcal{F}^{-1}\frac{1+\sgn(\omega)}{\sqrt 2}\Four\]
	
	Dies ist folglich analog zur Faltung mit der inversen Fouriertransformierten der Signumsfunktion direkt im Zeitbereich.
	\[ \Ana f(t) = f(t) * \mathcal{F}^{-1}\frac{1 + \sgn(\omega)}{\sqrt 2} \]
	
	Mit der Identität
	\[\Four\frac{1}{\pi t} = -i\sgn(\omega)\]
	so wie der Linearität der Fouriertransformation und der Faltung folg schliesslich
	\begin{align*}
		\sqrt 2 \Ana f(t) 
		&= f(t) * \biggl(\delta(t) + \frac{i}{\pi t} \biggr)\\
		&= f(t) + \frac{i}{\pi} \CH\int_{-\infty}^{\infty} \frac{f(x)}{t - x} \,\mathrm{d}x\\
		&= (1 + i\Hilb\,) f(t)\qedhere
	\end{align*}
\end{proof}

Jetzt haben wir endlich alle szusammen, was wir für den letzten Beweis brauchen.
Wir wollten ja noch zeigen, dass der Auslöschungsoperator das Signal nur gerade so stark verändert, wie notwendig.
Beim Morlet-Wavelet haben wir gesehen, dass er lediglich einen passenden Imaginärteil hinzufügt und das ganze so skaliert, dass die Norm erhalten bleibt.
Damit landen wir beim letzten Satz für diesen Abschnitt.

\begin{satz}
	Sei $\psi \in L^2(\mathbb R)$ ein reelles Wavelet. Dann sind die Realteile von $\psi$ und $\Ana\psi$ bis auf Skalierung identisch.
	\[ \sqrt 2 \psi = \Re\Ana\psi\]
\end{satz}

\begin{proof}
	Die Hilbert-Transformation ist eine reelle Integraltransformation
	\[\Hilb\,\colon\mathbb{R}\to\mathbb{R}.\]
	Folglich gilt nach Definition
	\[\Ana\psi = \frac{1 + i}{\sqrt 2}\psi\]
	Daraus folgt direkt
	\[\Re \Ana\psi = \frac{1}{\sqrt 2}\psi \quad \text{und}\quad \Im \Ana\psi = \frac{\Hilb }{\sqrt 2}\psi\qedhere\]
\end{proof}

Mittels Hilbert-Transformation kann also zu einem reellen Wavelet ein passender Imaginärteil gefunden werden, so dass Amplitude und Phase separiert werden können.
Wir defionieren non noch das eigentliche Objekt unserer Begierde.

\begin{satz}[Analytisches Wavelet]
	\label{complex:analytic-wavelet}
	Sei $\psi(t)$ ein reelles Wavelet. Dann ist
	\begin{equation}
	\psi^\ast = \Ana\psi
	\end{equation}
	das $\psi$ zugeordnete \emph{analytische Wavelet}\footnote{Auch hierbei ist ``analytisch'' wieder im Sinne der Signaltheorie zu verstehen, also $\forall \omega < 0 \colon \hat\psi^\ast(\omega) = 0$.}.
\end{satz}

Die analytischen Wavelets erben nun natürlich all die schönen Eigenschaften, welche durch den Auslöschungsoperator erzeugt werden.
Sie bleiben möglichst nahe am Original, lassen aber eine Separierung von Phase und Amplitude zu.
Analytische Wavelets eignen sich dadurch besonders gut, um periodische Anteile in einem Signal zu finden, da die Grösse des Skalarproduktes zwischen Wavelet und Signal unabhängig ist von der Phase.
