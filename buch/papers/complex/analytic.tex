\subsection{Analytische Wavelets und deren Norm}
\rhead{Analytische Wavelets}

Mittels Hilbert-Transformation kann also jedem glatten, reellen Zeitsignal ein analytisches Signal zugeordnet werden.
Der Realteile des zugeordneten analytischen Signals bleibt dabei gleich dem Original.
Spektral erhällt man das analytische Signal, indem man die negativen Frequenzen entfernt und die positiven verdoppelt, wie wir das im Abschnitt~\ref{complex:gabor-to-morlet} auf dem Weg vom Gabor- zum Morlet-Wavelet auch getan haben.
Tatsächlich ist das Morlet-Wavelet das dem Gabor-Wavelet zugeordnete Wavelet.

Die Norm eines Wavelets muss $1$ sein.
Folglich muss die Norm und jene, des zum Original-Wavelet zugeordnete analytischen Wavelets gleich sein.
Bei analytischen Signalen ist dies jedoch nicht gegeben.
Dies führt zur folgenden Definition.
\begin{definition}
	Sei $\psi$ ein beliebiges, reelles Wavelet. Dann ist
	\begin{equation}
		\psi^\ast = \frac{1}{\sqrt{2}}\left(\psi + i\,\mathcal{H}\psi\right) \label{complex:anawave}
	\end{equation}
	das $\psi$ \emph{zugeordnete analytische Wavelet}.
\end{definition}

Durch den Skalierungsfaktor in obiger Definition gilt für die Norm eines analytischen Wavelets der folgende Satz.
\begin{satz}
	\label{complex:norm}
	Die Norm eines analytischen Wavelets ist gleich der Norm des Original-Wavelets, also $1$.
	\[1 = \left\|\psi\right\| = \left\|\psi^\ast\right\|\]
\end{satz}

\begin{proof}
	Wir rechnen nach:
	\begin{align}
		1 = \left\|\psi\right\| = \left\|\hat{\psi}\right\| 
		&= \int_{-\infty}^{\infty}\left|\hat{\psi}(\omega)\right|^2 d\omega \label{complex:norm-proof-p1}\\
		&= \int_{-\infty}^{0}\left|\hat{\psi}(\omega)\right|^2 d\omega +  \int_{0}^{\infty}\left|\hat{\psi}(\omega)\right|^2 d\omega \notag\\
		&=  2\int_{0}^{\infty}\left|\hat{\psi}(\omega)\right|^2 d\omega \label{complex:norm-proof}\\
		&=  \int_{0}^{\infty}\left|\frac{2}{\sqrt{2}}\hat{\psi}(\omega)\right|^2 d\omega \notag\\
		&=  \int_{-\infty}^{\infty}\left|\hat{\psi}^\ast(\omega)\right|^2 d\omega 
		= \left\|\hat{\psi}^\ast\right\| = \left\|\psi^\ast\right\|.\label{complex:norm-proof-p2}
	\end{align}
	Hierbei haben wir in Gleichung~\eqref{complex:norm-proof-p1} und \eqref{complex:norm-proof-p2} die Placherel-Formel verwendet.
	In Gleichung~\eqref{complex:norm-proof} nutzten wir die hermitesche Symmetrie der Fourier-Transformierten eines reellen Wavelets, nach welcher
	\[\left|\hat{f}(\omega)\right| = \left|\hat{f}(-\omega)\right|\]
	gilt.
	Somit ist Satz~\ref{complex:norm} bewiesen.
\end{proof}
