\subsection{Analytische Wavelets}
\rhead{Analytische Wavelets}

Mittels Hilbert-Transformation kann also zu einem reellen Zeitsignal ein zugeordnetes analytisches Signal gefunden werden.
Der Realteile des zugeordneten analytischen Signals bleibt dabei gleich dem Original.
Spektral erhällt man das analytische Signal, indem man die negativen Frequenzen entfernt und die positiven verdoppelt, wie wir das im Abschnitt~\ref{complex:gabor-to-morlet} auf dem Weg vom Gabor- zum Morlet-Wavelet auch getan haben.
Tatsächlich ist das Morlet-Wavelet das dem Gabor-Wavelet zugeordnete analytische Wavelet.

\begin{satz}
	\label{complex:analytic-wavelet}
	Sei $\psi$ ein reelles Wavelet. Dann ist
	\begin{equation}
		\psi^\ast = \frac{1}{\sqrt{2}}\left(\psi + i\,\mathcal{H}\psi\right)
	\end{equation}
	das zu $\psi$ zugeordnete \emph{analytische Wavelet}\footnote{Auch hierbei ist ``analytisch'' wieder im Sinne der Signaltheorie zu verstehen, also $\hat\psi^\ast = 0 \,\forall\,\omega < 0$.}.
	Die Fouriertransformierte $\hat\psi^\ast$ verschwindet für alle negativen Frequenzen.
	\[\hat\psi^\ast(\omega) = \,\forall\,\omega<0\]
\end{satz}

\begin{proof}
	Dass $\hat\psi^\ast = 0$ für $\omega < 0$ gilt, folgt trivial aus Satz~\ref{complex:analytic-signal}.

	Um sicherzustellen, dass $\psi^\ast$ wirklich ein Wavelet ist, müssen wir die Zulässigkeitsbedingung aus Gleichung~\eqref{cwt:zulaessig} prüfen.
	Nach Voraussetzung ist 
	\[
	C_{\psi}
	=
	2\pi
	\int_{-\infty}^\infty \frac{|\hat{\psi}(\omega)|^2}{|\omega|}\,\mathrm{d}\omega < \infty.
	\]
	Daraus folgt
	\[
	C_{\psi^\ast}
	=
	2\pi
	\int_{0}^\infty \frac{|\hat{\psi}^\ast(\omega)|^2}{|\omega|}\,\mathrm{d}\omega < \infty,
	\]
	da $\hat\psi^\ast$ für $\omega > 0$ lediglich mit einem Faktor $\sqrt 2$ skaliert wurde.
	Nun überprüfen wir noch die Norm $\|\psi^\ast\|$. 
	Sie muss $1$ sein.
	\begin{align}
		1 = \left\|\psi\right\| = \left\|\hat{\psi}\right\| 
		&= \int_{-\infty}^{\infty}\left|\hat{\psi}(\omega)\right|^2 \mathrm{d}\omega \label{complex:norm-proof-p1}\\
		&= \int_{-\infty}^{0}\left|\hat{\psi}(\omega)\right|^2 \mathrm{d}\omega +  \int_{0}^{\infty}\left|\hat{\psi}(\omega)\right|^2 \mathrm{d}\omega \notag\\
		&=  2\int_{0}^{\infty}\left|\hat{\psi}(\omega)\right|^2 \mathrm{d}\omega \label{complex:norm-proof}\\
		&=  \int_{0}^{\infty}\left|\frac{2}{\sqrt{2}}\hat{\psi}(\omega)\right|^2 \mathrm{d}\omega \notag\\
		&=  \int_{-\infty}^{\infty}\left|\hat{\psi}^\ast(\omega)\right|^2 \mathrm{d}\omega 
		= \left\|\hat{\psi}^\ast\right\| = \left\|\psi^\ast\right\|.\label{complex:norm-proof-p2}
	\end{align}
	In den Gleichungen~\eqref{complex:norm-proof-p1} und \eqref{complex:norm-proof-p2} kam die Placherel-Formel zur Anwendung.
	In Gleichung~\eqref{complex:norm-proof} nutzten wir die hermitesche Symmetrie der Fourier-Transformierten eines reellen Wavelets, nach welcher
	\[\left|\hat{f}(\omega)\right| = \left|\hat{f}(-\omega)\right|.\]


	Bei $\psi^\ast$ handelt es sich also tatsächlich um ein Wavelet.
\end{proof}
