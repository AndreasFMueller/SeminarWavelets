\section{Analytische Wavelets}
Der Operator $\Ana\,$ hat den ersten Test also bestanden.
Nun möchten wir ihn noch etwas besser verstehen.
Was genau passiert eigentlich?
Ist es Zufall, dass das Morlet- und Gabor-Wavelet so ähnlich sind?
Dass der Realteil beim Morlet-Wavelet -- bis auf Skalierung -- genau dem Gabor-Wavelet entspricht?
Eine Anforderung an den Operator $\Ana\,$ war ja, dass er das Wavelet nur so wenig wie möglich ändern soll.
Beim Gabor-Wavelet hat das funktioniert.
Aber ist diese Eigenschaft im Allgemeinen erfüllt?
Diese Fragen möchten wir in diesem Abschnitt beantworten.

Im vorherigen Abschnitt haben wir negative Frequenzen als Problem identifiziert.
Wir definierten den Operator $\Ana\,$ so, dass er genau die negativen Frequenzen entfernt und dabei die Norm erhält.
Dazu wechselten wir in den Frequenzbereich, entfernten, was wir nicht wollten, und wechselten wieder zurück.

Wie lässt sich dieses Verfahren verallgemeinern? 
Ist dieses Hin- und Herwechseln tatsächlich notwendig?
Und was ist denn der Effekt im Zeitbereich?
Hierzu machen wir einen kurzen Ausflug in die Signaltheorie.
Die Theorie der analytischen Signale liefert uns die gesuchten Antworten.


\subsection{Analytische Signale und Hilbert-Transformation}
\rhead{Hilbert-Transformation}
In der Nachrichtentechnik ist Bandbreite ein beschränktes und dadurch wertvolles Gut.
Die Spektra reeller Signale weisen aber immer eine hermitesche Symmetrie auf.
Es reicht also, die Hälfte des Spektrums eines Signals zu übertragen, etwa nur die positiven Frequenzen.
In der Signaltheorie ist dieses Verfahren unter dem Namen Einseitenband-Modulation bekannt.
Ein Nebeneffekt hierbei ist, dass der Betrag des Signals gerade der Umhüllenden entspricht.
Eine Schwingung verliert damit ihre Nullstellen, ohne aber Information zu verlieren.
Genau das möchten wir von unseren Wavelets.

Ein Signal, bei welchem die negativen Frequenzen entfernt wurden, nennt man \emph{analytisches Signal}\footnote{
	Der Begriff `analytisch' ist in diesem Kapitel immer im Sinne der Signaltheorie zu verstehen, also $\forall \omega < 0 \colon \hat f (\omega) = 0 $.
	Er ist nicht zu verwechseln mit der Eigenschaft analytischer Funktionen in der Analysis.
}.
Wir werden analog dazu \emph{analytische Wavelets} definieren und zeigen, dass der Auslöschungsoperator eben solche erzeugt.
Dadurch übertragen sich die wesentlichen Eigenschaften analytischer Signale auf unsere neuen Wavelets.

Bevor wir den Begriff des analytischen Signals einführen können, benötigen wir aber die Hilbert-Transformation.
\begin{definition}
	Der Operator
 	\[
 	\Hilb\,\colon L^2(\mathbb R) \to L^2(\mathbb R)
 	~\quad~
 	f(t) \mapsto \Hilb f(t)
 	= \frac{1}{\pi} \CH\int_{-\infty}^{\infty}\frac{f(x)}{t-x} \mathrm{d}x
 	\]
 	heisst \emph{Hilbert-Transformation}.
 	Hierbei bezeichnet $\CH\int_{-\infty}^{\infty} \dots \mathrm{d}x$ den cauchyschen Hauptwert des divergenten Integrals.
\index{Hauptwert}%
\index{Cauchy-Hauptwert}%
\end{definition}

Die Hilberttransformation ist, wie die Wavelet- oder Fouriertransformation, eine Integraltransformation.
Ein wesentlicher Unterschied besteht jedoch darin, dass sie den Raum nicht wechselt.
Es ist eine Transformation aus der Zeit in die Zeit.

Nun sind wir bereit für die Definition eines analytischen Signals.
\begin{definition}
	\label{complex:analytic-signal}
	Sei $f \in L^2(\mathbb R)$ ein reelles Signal.
	Dann heisst
	\[f^\ast = (1 + i\Hilb\,)f \]
	das $f$ zugeordnete \emph{analytische Signal}.
\end{definition}
\index{analytisches Signal}
\index{Signal, analytisch}
\begin{satz}
	Ein analytisches Signal wird durch den Auslöschungsoperator erzeugt.
	\[1 + i\Hilb\, \equiv \sqrt 2 \Ana\, \Rightarrow f^\ast \equiv \sqrt 2 \Ana f\]
\end{satz}

\begin{proof}
	Der Auslöschungsoperator wechselt in den Frequenzbereich.
	Dort berechnet er die punktweise Multiplikation mit der Signumsfunktion und wechselt wieder zurück in den Zeitbereich.
	\[\Ana\, = \mathcal{F}^{-1}\frac{1+\sgn(\omega)}{\sqrt 2}\Four\]
	
	Dies ist folglich analog zur Faltung mit der inversen Fouriertransformierten der Signumsfunktion direkt im Zeitbereich,
	\[ \Ana f(t) = f(t) * \mathcal{F}^{-1}\frac{1 + \sgn(\omega)}{\sqrt 2}. \]
	
	Mit der Identität
	\[\Four\frac{1}{\pi t} = -i\sgn(\omega)\]
	so wie der Linearität der Fouriertransformation und der Faltung folgt schliesslich
	\begin{align*}
		\sqrt 2 \Ana f(t) 
		&= f(t) * \biggl(\delta(t) + \frac{i}{\pi t} \biggr)\\
		&= f(t) + \frac{i}{\pi} \CH\int_{-\infty}^{\infty} \frac{f(x)}{t - x} \,\mathrm{d}x\\
		&= (1 + i\Hilb\,) f(t)\qedhere
	\end{align*}
\end{proof}

Jetzt haben wir endlich alles zusammen, was wir für den Satz zur Ähnlichkeit brauchen.
Wir wollten ja zeigen, dass der Auslöschungsoperator das Signal nur gerade so stark verändert, wie notwendig.
Beim Morlet-Wavelet haben wir gesehen, dass er lediglich einen passenden Imaginärteil hinzufügt und das ganze so skaliert, dass die Norm erhalten bleibt.
Das stimmt für alle Wavelets.

\begin{satz}
	Sei $\psi \in L^2(\mathbb R)$ ein reelles Wavelet. Dann sind die Realteile von $\psi$ und $\Ana\psi$ bis auf Skalierung
	\[ \psi = \sqrt 2 \Re\Ana\psi\]
identisch.
\end{satz}

\begin{proof}
	Die Hilbert-Transformation ist eine reelle Integraltransformation
	\[\Hilb\,\colon L^2(\mathbb{R}) \to L^2(\mathbb{R}).\]
	Nach Definition gilt
	\[\Ana\psi = \frac{1 + i\Hilb\,}{\sqrt 2}\psi,\]
	woraus sich nun Real- und Imaginärteil direkt ablesen lassen:
	\[\Re \Ana\psi = \frac{1}{\sqrt 2}\psi \quad \text{und}\quad \Im \Ana\psi = \frac{\Hilb }{\sqrt 2}\psi.\qedhere\]
\end{proof}

Mittels Hilbert-Transformation kann also zu einem reellen Wavelet ein passender Imaginärteil gefunden werden, so dass Amplitude und Phase separiert werden können.
Wir definieren nun noch das eigentliche Objekt unserer Begierde.

\begin{satz}
	\label{complex:analytic-wavelet}
	Sei $\psi(t)$ ein reelles Wavelet. Dann ist
	\begin{equation}
	\psi^\ast = \Ana\psi
	\end{equation}
	das $\psi$ zugeordnete \emph{analytische Wavelet}.
\index{analytisches Wavelet}%
\index{Wavelet, analytisch}%
%	\footnote{Auch hierbei ist ``analytisch'' wieder im Sinne der Signaltheorie zu verstehen, also $\forall \omega < 0 \colon \hat\psi^\ast(\omega) = 0$.}
\end{satz}

Die analytischen Wavelets erben nun all die Eigenschaften analytischer Signale, welche durch den Auslöschungsoperator erzeugt werden.
Sie erhalten einen Imaginärteil, so dass sich Phase und Amplitude separieren lassen.
Analytische Wavelets eignen sich dadurch besonders gut, um periodische Anteile in einem Signal zu finden, da die Grösse des Skalarproduktes zwischen Wavelet und Signal unabhängig ist von der Phase.
