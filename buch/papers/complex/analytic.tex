\subsection{Analytische Wavelets}
\rhead{Analytische Wavelets}

Mittels Hilbert-Transformation kann also einem reellen Signal ein Imaginärteil gefunden werden, so dass alle negativen Frequenzen verschwinden.
Der Realteil bleibt dabei unverändert, und auch das Spektrum der positiven Frequenzen bleibt bis auf Skalierung unverändert.
Das ist auch im Abschnitt~\ref{complex:gabor-to-morlet} auf dem Weg vom Gabor- zum Morlet-Wavelet passiert.
Tatsächlich ist das Morlet-Wavelet das dem Gabor-Wavelet zugeordnete analytische Wavelet.

\begin{satz}
	\label{complex:analytic-wavelet}
	Sei $\psi(t)$ ein reelles Wavelet. Dann ist
	\begin{equation}
		\psi^\ast(t) = \frac{1+i\Hilb}{\sqrt{2}}\psi(t)
	\end{equation}
	das $\psi(t)$ zugeordnete \emph{analytische Wavelet}\footnote{Auch hierbei ist ``analytisch'' wieder im Sinne der Signaltheorie zu verstehen, also $\forall \omega < 0 \colon \hat\psi^\ast(\omega) = 0$.}.
	Die Fouriertransformierte $\hat\psi^\ast$ verschwindet für alle negativen Frequenzen.
	\[\forall \omega < 0 \colon \hat\psi^\ast(\omega) = 0\]
\end{satz}

\begin{proof}
	Für analytische Wavelets findet man analog zum Beweis von Satz~\ref{complex:analytic-signal}, dass
	\[\hat\psi^\ast(\omega) = \frac{1}{\sqrt{2}}\hat\psi(1 +\sgn(\omega))\]
	
	Um sicherzustellen, dass $\psi^\ast$ wirklich ein Wavelet ist, müssen wir die Zulässigkeitsbedingung aus Gleichung~\eqref{cwt:zulaessig} prüfen.
	Nach Voraussetzung ist $\psi$ ein Wavelet, also gilt
	\[
	C_{\psi}
	=
	2\pi
	\int_{-\infty}^\infty \frac{|\hat{\psi}(\omega)|^2}{|\omega|}\,\mathrm{d}\omega < \infty.
	\]
	Daraus folgt nun
	\begin{align}
	C_{\psi^\ast}
	&= 2\pi	\int_{0}^\infty \frac{|\hat{\psi}^\ast(\omega)|^2}{|\omega|}\,\mathrm{d}\omega \\
	&= 2\pi \int_{0}^\infty \frac{|\sqrt{2}\hat{\psi}(\omega)|^2}{|\omega|}\,\mathrm{d}\omega 
	< \infty,
	\end{align}
	da $\hat\psi^\ast(\omega)$ für $\omega > 0$ lediglich mit einem Faktor $\sqrt 2$ skaliert wurde.
	Somit bleibt nur noch die Norm zu prüfen.
	Sie muss $1$ sein.
	\begin{align}
		1 = \|\psi\| = \|\hat{\psi}\| 
		&= \int_{-\infty}^{\infty}|\hat{\psi}(\omega)|^2 \mathrm{d}\omega \label{complex:norm-proof-p1}\\
		&= \int_{-\infty}^{0}|\hat{\psi}(\omega)|^2 \mathrm{d}\omega +  \int_{0}^{\infty}|\hat{\psi}(\omega)|^2 \mathrm{d}\omega \notag\\
		&=  2\int_{0}^{\infty} |\hat{\psi}(\omega)|^2 \mathrm{d}\omega \label{complex:norm-proof}\\
		&=  \int_{0}^{\infty}\biggl|\frac{2}{\sqrt{2}}\hat{\psi}(\omega)\biggr|^2 \mathrm{d}\omega \notag\\
		&=  \int_{-\infty}^{\infty}|\hat{\psi}^\ast(\omega)|^2 \mathrm{d}\omega 
		= \|\hat{\psi}^\ast\| = \|\psi^\ast\|.\label{complex:norm-proof-p2}
	\end{align}
	In den Gleichungen~\eqref{complex:norm-proof-p1} und \eqref{complex:norm-proof-p2} kam die Placherel-Formel zur Anwendung.
	In Gleichung~\eqref{complex:norm-proof} nutzten wir die hermitesche Symmetrie der Fourier-Transformierten eines reellen Wavelets, nach welcher
	\[|\hat{f}(\omega)| = |\hat{f}(-\omega)|.\]
	Bei $\psi^\ast(\omega)$ handelt es sich also tatsächlich um ein Wavelet.
\end{proof}
