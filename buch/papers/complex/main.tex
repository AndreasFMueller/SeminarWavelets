%
% main.tex -- Paper zum Thema komplexe Morlet Wavelets und CWT
%
% (c) 2019 Hochschule Rapperswil
%
\chapter{Komplexe Morlet Wavelets und CWT\label{chapter:thema}}
\lhead{Komplexe Wavelets und CWT}
\begin{refsection}
\chapterauthor{Roy Seitz}

Die schnelle Wavelettransformation (Fast Wavelet Transform, FWT) mit reellen Wavelets besitzt eine Vielzahl nützlicher Eigenschaften.
Je nach Anwendung besitzt sie jedoch auch zwei wesentliche Nachteile.

Die Frequenz in der FWT ist nur in Zweierpotenzen der Abtastfrequenz berechnet.
Aus mathematischer Sicht ist das zwar ausreichend, für manche Anwendungen ist diese Auflösung jedoch zu grob.

Aus der Fouriertheorie ist man sich zudem gewohnt, Phasen- und Amplitudeninformation getrennt betrachten zu können.
Dies ist jedoch nur durch die Wahl komplexer Basisfunktionen möglich.

Die Antworten hierzu fanden wir in Kapitel~\ref{chapter:cwt}, die kontinuierliche Wavelettransformation mit komplexen Wavelets (Continuous Wavelet Transform, CCWT).

Im ersten Teil dieses Kapitels benutzen wir für die CCWT ein analytisches Wavelet, das Morlet-Wavelet.
Dadurch können wir ähnlich wie in der Fouriertransformation Amplitude und Phase separat betrachten.
Wir werden auch ansehen, wie man ein beliebiges Wavelet zu einem komplexen Wavelet erweitern kann.

Als nächstes finden wir eine Methode, die CCWT mittels Fouriertransformation effizient zu berechnen.
Nach diskretisierung der Variablen $a$ und $b$ lässt sich die CCWT dann mittels FFT berechnen.
Zudem erlaubt dies eine anschauliche Interpretation dessen, was die CCWT eigentlich tut.

Die Berechnung mittels FFT eignet sich jedoch nicht für alle Wavelets.
Welche Eigneschaften das Wavelet erfüllen muss, damit die Berechnung effizient durchfürbar ist, sehen wir im dritten Teil.

Die Multiplikation im Frequenz-Bereich entspricht bekanntlich der Faltung im Zeit-Bereich.
Die Multiplikation mittels FFT führt allerdings zu einer zyklischen Faltung und dadurch zu unerwünschten Randeffekten.
Im vierten Teil dieses Kapitels betrachten wir die Konsequenzen der zyklischen Faltung und wie man sie vermeiden kann (Signal Padding).

Als letztes betrachten wir eine Implementation in Matlab und analysieren, welche Operationen am meisten Zeit benötigen.
Wir werden sehen, dass das Padding die Performanz deutlich beeinträchtigt.

% 1. Morlet zur analyse unabh. von Phase
% 2. CCWT als Faltung -> FFT
% 3. Einschränkungen Wavelets
% 4. Zyklische Faltung, Padding des Signals und Performanz

\section{Abschnitt}
\rhead{Abschnitt}

\section{Schlussfolgerung}
\rhead{Schlussfolgerung}

\section{Die kontinuierliche komplexe Wavelettransformation}
\rhead{Die CCWT}
Die kontinuierliche Wavelettransformation löst das Problem der Frequenzauflösung.
In der Anwendung kann die Frequenz, respektive die hierzu äquivalente Variable $a$ in entsprechender Auflösung diskretisiert werden.

Durch die Wahl reeller Wavelets erscheint eine reelle harmonische Schwingung in der kontinuierlichen Wavelettransformation allerdings wieder als Schwingung.
Dies ist eine Konsequenz aus der Orthogonalität von Sinus und Cosinus: Zeitverschiebungen $b = (2k+1)\omega / 4, k \in\mathbb Z$ führen zu einer Auslöschung

Sie ist gegeben in Gleichung~\eqref{cwt:definition:eq}:
\[
\mathcal{W}f (a,b)
=
\langle f,\psi_{a,b}\rangle
=
\frac{1}{\sqrt{|a|}}\int_{-\infty}^\infty f(t)\,\overline{
	\psi\biggl(\frac{t-b}{a}\biggr)}\,dt
\]
und besagt, dass die kontinuierliche Wavelettransformation $\mathcal{W}f(a,b)$ gegeben ist durch das Skalarprodukt mit den Funktionen $\psi_{a,b}$.


\printbibliography[heading=subbibliography]
\end{refsection}
