%
% main.tex -- Paper zum Thema komplexe Morlet Wavelets und CWT
%
% (c) 2019 Hochschule Rapperswil
%
\chapter{Komplexe Morlet Wavelets und CWT\label{chapter:thema}}
\lhead{Komplexe Wavelets und CWT}
\begin{refsection}
\chapterauthor{Roy Seitz}

Die schnelle Wavelettransformation (Fast Wavelet Transform, FWT) mit reellen Wavelets besitzt eine Vielzahl nützlicher Eigenschaften.
Je nach Anwendung besitzt sie jedoch auch zwei wesentliche Nachteile.

Erstens sind die Frequenzen in der FWT nur in Zweierpotenzen der Abtastfrequenz berechnet.
Aus mathematischer Sicht ist das zwar ausreichend, für manche Anwendungen ist diese Auflösung jedoch zu grob.

Zweitens ist man sich aus der Fouriertheorie gewohnt, Phasen- und Amplitudeninformation getrennt betrachten zu können.
Diese Eigenschaft folg direkt aus der Wahl komplexer Basisfunktionen und ist mit reellen Wavelets folglich nicht möglich.

Die Antworten hierzu fanden wir in Kapitel~\ref{chapter:cwt}, die kontinuierliche Wavelettransformation mit komplexen Wavelets (Continuous Wavelet Transform, CCWT).

Als erstes möchten wir folglich betrachten, wie wir mit komplexen Wavelets Amplituden- und Phasen-Information getrennt auswerten können, und wie wir geeignete Wavelets finden.

Als nächstes möchten wir die kontinuierliche Wavelettransformation mit komplexen Wavelets (Continuous Complex Wavelet Transform, CCWT) effizient berechnen können.
Wir werden sehen, wie dies mittels Fourier-Transformation elegant erledigt werden kann.

Diese Berechnung liefert eine neue Interpretation der Wavelettransformation und führt zugleich zu einer Einschränkung der nützlichen Wavelets.
Als drittes möchten wir deshalb betrachten, wie eine schnelle, komplexe Wavelettransformation verwendet werden kann.

Abschliessend betrachten wir noch die Effekte der zyklischen Faltung, welche durch die FFT erzielt wird.
Dies führt zu -- möglicherweise störenden -- Randeffekten.
Deren Vermeidung ist durch padding des Sinal vermeidbar, was allerdings zusätzliche Rechenleistung erfordert.
Im letzten Teil dieses Kapitels betrachten wir folglich die Performance-Unterschiede, welche durch das PAdding entstehen.


%%%%%%%%%%%%%%%%%%%%%%%%%%%%%%%%%%%%%%%%%%%%%%%

\section{Komplexe Wavelets}
\rhead{Komplexe Wavelets}

\section{Schnelle Berechnung der kontinuierlichen komplexen Wavelettransformation}
\rhead{CCWT}

\section{Abschnitt}
\rhead{Abschnitt}

\section{Schlussfolgerung}
\rhead{Schlussfolgerung}

\section{Die kontinuierliche komplexe Wavelettransformation}
\rhead{Die CCWT}
Die kontinuierliche Wavelettransformation löst das Problem der Frequenzauflösung.
In der Anwendung kann die Frequenz, respektive die hierzu äquivalente Variable $a$ in entsprechender Auflösung diskretisiert werden.

Durch die Wahl reeller Wavelets erscheint eine reelle harmonische Schwingung in der kontinuierlichen Wavelettransformation allerdings wieder als Schwingung.
Dies ist eine Konsequenz aus der Orthogonalität von Sinus und Cosinus: Zeitverschiebungen $b = (2k+1)\omega / 4, k \in\mathbb Z$ führen zu einer Auslöschung

Sie ist gegeben in Gleichung~\eqref{cwt:definition:eq}:
\[
\mathcal{W}f (a,b)
=
\langle f,\psi_{a,b}\rangle
=
\frac{1}{\sqrt{|a|}}\int_{-\infty}^\infty f(t)\,\overline{
	\psi\biggl(\frac{t-b}{a}\biggr)}\,dt
\]
und besagt, dass die kontinuierliche Wavelettransformation $\mathcal{W}f(a,b)$ gegeben ist durch das Skalarprodukt mit den Funktionen $\psi_{a,b}$.


\printbibliography[heading=subbibliography]
\end{refsection}
