%
% main.tex -- Paper zum Thema komplexe Morlet Wavelets und CWT
%
% (c) 2019 Hochschule Rapperswil
%

\chapter{Komplexe Morlet Wavelets und CWT\label{chapter:thema}}
\lhead{Komplexe Wavelets und CWT}
\begin{refsection}
\chapterauthor{Roy Seitz}

Die schnelle Wavelettransformation (Fast Wavelet Transform, FWT) mit reellen Wavelets besitzt eine Vielzahl nützlicher Eigenschaften.
Je nach Anwendung besitzt sie jedoch auch zwei wesentliche Nachteile.

Erstens ist man sich aus der Fouriertheorie gewohnt, Phasen- und Amplitudeninformation getrennt betrachten zu können.
Diese Eigenschaft folg direkt aus der Wahl komplexer Basisfunktionen (oder Sinus und Cosinus als REal- und Imaginärteil der komplexen Schwingung) und ist mit reellen Wavelets folglich nicht möglich.

Als zweites sind die Frequenzen in der FWT nur in Zweierpotenzen der Abtastfrequenz berechnet, ähnlich der Fourier-Reihen, wo nur ganzzahlige Vielfache der Grundfrequenz berechnet werden.
Aus mathematischer Sicht ist das zwar ausreichend, für manche Anwendungen ist diese Auflösung jedoch zu grob.

Die Antworten hierzu fanden wir in Kapitel~\ref{chapter:cwt}, die kontinuierliche Wavelettransformation mit komplexen Wavelets (Continuous Wavelet Transform, CCWT).
Als erstes möchten wir folglich betrachten, wie wir mit komplexen Wavelets Amplituden- und Phasen-Information getrennt auswerten können und als zweites, wie wir geeignete Wavelets finden.

Als drittes möchten wir die CCWT effizient berechnen können.
Wir werden sehen, wie dies mittels Fourier-Transformation elegant erledigt werden kann.
Diese Berechnung liefert eine neue Interpretation der Wavelettransformation und führt zugleich zu einer Bedingung für nützliche Wavelets.

Abschliessend betrachten wir noch die Effekte der zyklischen Faltung, welche durch die FFT erzielt wird.
Dies führt zu -- möglicherweise störenden -- Randeffekten.
Deren Vermeidung ist durch padding des Sinal vermeidbar, was allerdings zusätzliche Rechenleistung erfordert.
Im letzten Teil dieses Kapitels betrachten wir folglich die Performance-Unterschiede, welche durch das PAdding entstehen.


%%%%%%%%%%%%%%%%%%%%%%%%%%%%%%%%%%%%%%%%%%%%%%%
\documentclass{standalone}
%\usepackage{tikz}
\usepackage{amsmath}
\usepackage{pgfplots}
%\usepackage{tikz-3dplot}

\pgfplotsset{scale=0.95}

\usetikzlibrary{arrows,intersections,math}

\begin{document}
	\begin{tikzpicture}[>=latex]
		\begin{axis}[width=15cm, height=7.5cm, xmin=-3.5, xmax=3.5, ymin=-0.9, ymax=0.9, axis lines=middle, ytick={-1, -0.5, ..., 1}, xtick={-3, -2, ..., 3}]
		\addplot[fill=gray, opacity=0.1]  table[x=t,y=envp] {morlet.data};
		\addplot[fill=gray, opacity=0.1]  table[x=t,y=envn] {morlet.data};
		
		\addplot[smooth, mark=none, color=blue] table[x=t,y=re] {morlet.data};
		\addplot[smooth, mark=none, color=red]  table[x=t,y=im] {morlet.data};
		\end{axis}
	\end{tikzpicture}
\end{document}


\section{Analytische Wavelets und Hilbert-Transformation}
\subsection{Analytische Signale und die Hilbert-Transformation}
\rhead{Hilbert-Transformation}
Im ersten Kapitel haben wir aus dem reellen Gabor-Wavelet und dem Wunsch nach Separation von Amplitude und Phase das komplexe Morlet-Wavelet gefunden.
Hierzu haben wir die negativen Frequenzen aus dem Spektrum des Wavelets entfernt.
Dieses Verfahren lässt isch mit der Hilbert-Transformation
\[
	\mathcal{H}f(t) =
	\frac{1}{\pi}\int_{-\infty}^{\infty}\frac{f(x)}{t-x} dx
\]
elegant verallgemeinern.
Hierzu benötigen wir den Begriff des analytischen Signals.
\begin{definition}
	Sei $f(t) \in \mathbb{R}$ ein reelles Zeitsignal, dann heisst
	\begin{align*}
		f^\ast(t) 
		&= f(t) + i\,\mathcal{H}f(t)
	\end{align*}
	das zu $f(t)$ \emph{analytische Signal}\footnote{Der Begriff `analytisch' ist in diesem Kapitel immer im Sinne dieser Definition aus der Signaltheorie zu verstehen.
	Er ist nicht zu verwechseln mit der Eigenschaft analytischer Funktionen in der Analysis.
%	Die Analysis kennt den Begriff der analytischen Funktion. 
%	Diese Eigenschaft ist in der komplexen Analysis jedoch äquivalent zur Holomorphie.
%	Wo benötigt wird folglich von holomorphen Funktionen gesprochen, um Verwechslungen zu vermeiden.
	}.
\end{definition}

\begin{satz}
	Die Fourier-Transformierte eines analytischen Signals verschwindet für alle negativen Frequenzen.
	Es gilt
	\[
		\hat{f}^\ast_{a,b}(\omega) = \left\lbrace\begin{matrix*}[r]	
			2\hat{f}_{a,b} & 0 \le \omega \\ 0 & \text{sonst}\end{matrix*} \right..
	\]
\end{satz}

\begin{proof}[Beweis]
	Die Hilbert-Transformation besitzt die Form eines Faltungs-Integrals.
	Es gilt
	\begin{align*}
		\mathcal{H} f(t) &= f(t) * \frac{1}{\pi t}.
	\end{align*}
	Diese Faltung wird im Frequenzbereich zur Multiplikation.
	Mit der Signumsfunktion $\sgn(\omega)$ und der Identität
	\begin{align*}
		\mathcal{F} \left\lbrace t^{-1}\right\rbrace  &= -i\pi\sgn(\omega)
	\end{align*}
	wird die Hilberttransformation im Frequenzbereich zu
	\begin{align*}
		\mathcal{F}\left\lbrace \mathcal{H}f(t)\right\rbrace 
		&= -i\sgn(\omega) \hat{f}.
	\end{align*}
	Für die Fourier-Transformierte von $f^\ast_{a,b}$ gilt folglich 
	\begin{align*}
		\hat{f}^\ast_{a,b} 
		&= \hat{f}_{a,b} - i^2\sgn(\omega)\hat{f}_{a,b}\\
		&=\left\lbrace\begin{matrix}
			2\hat{f}_{a,b} & \omega > 0\\
			0 & \omega \le 0
		\end{matrix} \right.
	\end{align*}
\end{proof}

Mittels Hilbert-Transformation kann man also aus jedem beliebigen, reellen Zeitsignal ein analytisches Signal konstruieren.
Spektral ist dies gleichbedeutend damit, die negativen Frequenzen zu entfernen und die positiven zu verdoppeln, wie wir das im ersten Kapitel auf dem Weg vom Gabor- zum Morlet-Wavelet auch getan haben.
Tatsächlich ist das Morlet-Wavelet das zum Gabor-Wavelet analytische Wavelet.
Die Norm erhöht sich hierbei jedoch um den Faktor $\sqrt{2}$.
Beim Gabor- und Morlet-Wavelet versteckt sich dieser Faktor im $c_\sigma$.

\subsection{Analytische Wavelets}
\rhead{Analytische Wavelets}

Mittels Hilbert-Transformation kann also zu einem reellen Zeitsignal ein zugeordnetes analytisches Signal gefunden werden.
Der Realteile des zugeordneten analytischen Signals bleibt dabei gleich dem Original.
Spektral erhällt man das analytische Signal, indem man die negativen Frequenzen entfernt und die positiven verdoppelt, wie wir das im Abschnitt~\ref{complex:gabor-to-morlet} auf dem Weg vom Gabor- zum Morlet-Wavelet auch getan haben.
Tatsächlich ist das Morlet-Wavelet das dem Gabor-Wavelet zugeordnete analytische Wavelet.

\begin{satz}
	\label{complex:analytic-wavelet}
	Sei $\psi$ ein reelles Wavelet. Dann ist
	\begin{equation}
		\psi^\ast = \frac{1}{\sqrt{2}}\left(\psi + i\,\mathcal{H}\psi\right)
	\end{equation}
	das zu $\psi$ zugeordnete \emph{analytische Wavelet}\footnote{Auch hierbei ist ``analytisch'' wieder im Sinne der Signaltheorie zu verstehen, also $\hat\psi^\ast = 0 \,\forall\,\omega < 0$.}.
	Die Fouriertransformierte $\hat\psi^\ast$ verschwindet für alle negativen Frequenzen.
	\[\hat\psi^\ast(\omega) = \,\forall\,\omega<0\]
\end{satz}

\begin{proof}
	Dass $\hat\psi^\ast = 0$ für $\omega < 0$ gilt, folgt trivial aus Satz~\ref{complex:analytic-signal}.

	Um sicherzustellen, dass $\psi^\ast$ wirklich ein Wavelet ist, müssen wir die Zulässigkeitsbedingung aus Gleichung~\eqref{cwt:zulaessig} prüfen.
	Nach Voraussetzung ist 
	\[
	C_{\psi}
	=
	2\pi
	\int_{-\infty}^\infty \frac{|\hat{\psi}(\omega)|^2}{|\omega|}\,\mathrm{d}\omega < \infty.
	\]
	Daraus folgt
	\[
	C_{\psi^\ast}
	=
	2\pi
	\int_{0}^\infty \frac{|\hat{\psi}^\ast(\omega)|^2}{|\omega|}\,\mathrm{d}\omega < \infty,
	\]
	da $\hat\psi^\ast$ für $\omega > 0$ lediglich mit einem Faktor $\sqrt 2$ skaliert wurde.
	Nun überprüfen wir noch die Norm $\|\psi^\ast\|$. 
	Sie muss $1$ sein.
	\begin{align}
		1 = \left\|\psi\right\| = \left\|\hat{\psi}\right\| 
		&= \int_{-\infty}^{\infty}\left|\hat{\psi}(\omega)\right|^2 \mathrm{d}\omega \label{complex:norm-proof-p1}\\
		&= \int_{-\infty}^{0}\left|\hat{\psi}(\omega)\right|^2 \mathrm{d}\omega +  \int_{0}^{\infty}\left|\hat{\psi}(\omega)\right|^2 \mathrm{d}\omega \notag\\
		&=  2\int_{0}^{\infty}\left|\hat{\psi}(\omega)\right|^2 \mathrm{d}\omega \label{complex:norm-proof}\\
		&=  \int_{0}^{\infty}\left|\frac{2}{\sqrt{2}}\hat{\psi}(\omega)\right|^2 \mathrm{d}\omega \notag\\
		&=  \int_{-\infty}^{\infty}\left|\hat{\psi}^\ast(\omega)\right|^2 \mathrm{d}\omega 
		= \left\|\hat{\psi}^\ast\right\| = \left\|\psi^\ast\right\|.\label{complex:norm-proof-p2}
	\end{align}
	In den Gleichungen~\eqref{complex:norm-proof-p1} und \eqref{complex:norm-proof-p2} kam die Placherel-Formel zur Anwendung.
	In Gleichung~\eqref{complex:norm-proof} nutzten wir die hermitesche Symmetrie der Fourier-Transformierten eines reellen Wavelets, nach welcher
	\[\left|\hat{f}(\omega)\right| = \left|\hat{f}(-\omega)\right|.\]


	Bei $\psi^\ast$ handelt es sich also tatsächlich um ein Wavelet.
\end{proof}

\documentclass[tikz]{standalone}
\usepackage{amsmath}
\usepackage{pgfplots}
\usepackage{csvsimple}

\usetikzlibrary{arrows,intersections,math}

\begin{document}
\begin{tikzpicture}[>=latex,scale=0.95]

\newcommand{\A}{0.70711}
\begin{axis}[xmin=-0.8, xmax=1.8, ymin=-1.2, ymax=1.2,
		axis lines=middle, ytick={-1, -0.5, ..., 1}, xtick={-1, -0.5, ..., 1.5}]
	\addplot[mark=none, color=blue] coordinates {(-1, 0 ) (0, 0) (0, \A) (0.5, \A) (0.5, -\A) (1, -\A) (1, 0) (2, 0)};
	\addplot[smooth, mark=none, color=red] table[x=t,y=im] {haar.data};

	\addplot[fill=gray, opacity=0.1]  table[x=t,y=envp] {haar.data};
	\addplot[fill=gray, opacity=0.1]  table[x=t,y=envn] {haar.data};
	
\end{axis}

\end{tikzpicture}
\end{document}


\section{Schnelle Berechnung der CWT}
\rhead{Berechnung der CWT}

Betrachten wir nun die kontinuierliche Wavelet-Transformation aus Definition~\ref{cwt:definition}, Gleichung~\eqref{cwt:definition:eq},
\[
\mathcal{W}f (a,b)
=
\langle f,\psi_{a,b}\rangle
=
\frac{1}{\sqrt{|a|}}\int_{-\infty}^\infty f(t)\,
	\overline{\psi}\left(\frac{t-b}{a}\right)\,dt.
\]

Dieses Integralt entspricht der Faltung zwischen $f(t)$ und $\frac{1}{\sqrt{|a|}} \overline{\psi}\left(\frac{-t}{a}\right)$.
Die Faltung wird im Fre\-quenz\-bereich zur Multiplikation. 
Die Fouriertransformierte von $\frac{1}{\sqrt{|a|}} \overline{\psi}\left(\frac{-t}{a}\right)$ ist
\begin{align*}
	\mathcal{F}\left\lbrace \frac{1}{\sqrt{|a|}} \overline{\psi}\left(\frac{-t}{a}\right)\right\rbrace 
	&= \frac{1}{\sqrt{|a|}} \int_{-\infty}^{\infty}\overline{\psi}\left(\frac{-t}{a}\right)e^{-i\omega t}\, dt\\
	&= \frac{1}{\sqrt{|a|}} \overline{\int_{-\infty}^{\infty}\psi\left(\frac{-t}{a}\right)e^{i\omega t}\, dt}\\
	&= \frac{1}{\sqrt{|a|}} \overline{\int_{-\infty}^{\infty}\psi\left(t'\right)e^{-ia\omega t'} |a|\, dt'} \qquad \left(t' = \frac{-t}{a}\right)\\
	&= \sqrt{|a|} \, \overline{\hat{\psi}}\left(a\omega\right),
\end{align*}
 also gilt

\begin{equation}
\mathcal{W}f(a,b)
= \mathcal{F}^{-1}\left\lbrace\hat{f}(\omega)\cdot \sqrt{|a|}\, \overline{\hat{\psi}}\left(a\omega\right)\right\rbrace. \label{complex:fcwt}
\end{equation}

Mittels Fourier-Transformation lässt sich die Wavelet-Transformation also besonders elegant berechnen.
Kontinuierliche Funktionen sind für nummerische Systeme jedoch ungeeignet.
Die CWT muss also in $a$ und $b$ diskretisiert werden.
Die Diskretisierung von $b$ entspricht vorteilhaft gerade der Diskretisierung des Signals selbst.
Dann lässt sich die Fourier-Transformation mittels FFT berechnen und Gleichung~\eqref{complex:fcwt} wird zu
\begin{equation}
	\mathcal{W}f(a,b) = \text{IFFT}(\text{FFT}(f) \cdot \overline{\hat{\psi}}\left(a\omega\right))\footnote{Der Faktor $\sqrt{|a|}$ wurde hierbei weggelassen.
		Hierdurch werden die hohen Frequenzen stärker gewichtet und $|\mathcal{W}f(a,b)|$ ist gerade proportional zur Amplitude der analysierten Signalkomponente.
	}. \label{complex:ffcwt}
\end{equation}
Diese Gleichung muss für jedes $a$ einzeln gelöst werden.
Falls das Wavelet im Frequenzbereich eine geschlossene Form besitzt, wird diese Gleichung besonders interessant.
Dann benötigt man eine FFT für das Signal, und eine inverse FFT sowie eine punktweise Multiplikation mit dem Wavelet für jedes $a$.
Sei \verb|Psi_ab| eine Matrix, in deren Spalten die Fourier-Transformierten für jedes $a$ stehen, und sei \verb|f| das zu analysierende Signal $f(t)$, dann lässt sich die Wavelettransformation in Matlab oder Octave mit folgendem Code einfach berechnen
\begin{verbatim}
	function Psi_ab = myMorlet(w, a)
	  sigma = 2 * pi;
	  c_sigma = pi^(-1/4) / sqrt(1 + exp(-sigma^2) - 2 * exp(-3/4 * sigma^2));
	  w = w ./ a;
	  Psi_ab = c_sigma .* exp(-(sigma - w).^2 / 2);
	end
	
	w = linspace(0, fs, length(f))' * 2 * pi; % Column freq. vector
	a = ... ; % Interesting a values as row vector
	Psi_ab = myMorlet(w, a);
	F   = fft(f);
	yab = ifft(F .* Psi_ab);
\end{verbatim}

%% Wavelet-TraFo als Faltung zwischen Signal und Wavelet
%% Morlet-Wavelet in Frequenzbereich geschlossen berechenbar
%%  -> Faltungs-Matrix mit verschieden skalierten Morlet-Wavelets in den Spalten
%%   - Matrix-Multiplikation mit Signal
%%   - ifft über die Spalten

\section{Zyklische Faltung, Signal-Padding und Performanz}
\rhead{Zyklische Faltung}
%% Effekt der zyklischen Faltung
%% Lösen durch padden des Signals
%%  - zero padding
%%  - padden durch spiegeln des Signals
%% Performanz-Vergleich anhand Matlab-Skript

\section{Schlussfolgerung}
\rhead{Schlussfolgerung}
%% Jedes Wavelet kann in ein analytisches transformiert werden
%% Die Wavelets aus der FWT sind allerdings für die CWT ungeeignet, 
%% da sie nich in geschlossener Form im Frequenzbereich berechnet werden können
%% (Aufstellen der Faltungs-Matrix)
%% Allerdings ist das Morlet-WAvelet auch zu bevorzugen, da es die optimale 
%% Schärfe zwischen Frequenz- und Zeitauflösung beitet (Gaus\ldots)

\printbibliography[heading=subbibliography]
\end{refsection}
