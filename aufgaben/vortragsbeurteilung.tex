%
% vortragsbeurteilung.tex -- Vortragsbeurteilungsformular fuer MathSem
%
% (c) 2009 Prof. Dr. Andreas Mueller, HSR
%
\documentclass[a4paper,12pt]{article}
\usepackage[utf8]{inputenc}
\usepackage{german}
\usepackage{times}
\usepackage{alltt}
\usepackage{verbatim}
\usepackage{fancyhdr}
\usepackage{amsmath}
\usepackage{amssymb}
\usepackage{amsfonts}
\usepackage{amsthm}
\usepackage{textcomp}
\usepackage{graphicx}
\usepackage{ifthen}
\usepackage{multirow}
\usepackage{txfonts}
\usepackage{german}
\usepackage{tabularx}
\usepackage{booktabs}
\usepackage{paralist}
\usepackage{fontenc}
\usepackage{color}
\usepackage{environ}
\usepackage{geometry}
\geometry{papersize={210mm,297mm},total={165mm,240mm}}

\begin{document}
\pagestyle{fancy}
\pagenumbering{gobble}

\lhead{Mathematisches Seminar 2019}
\rhead{Vortragsbeurteilung}
\vspace{1cm}

\begin{center}
{\LARGE Vortragsbeurteilung}
\end{center}

\vspace{0.8cm}

{\parindent0pt\hbox to\hsize{%
{\large Referenten:} \dotfill}}


\begin{center}
\begin{tabular}{|c|l|c|c|c|c|c|}
\hline
\raisebox{0pt}[13pt][6pt]{A}&Vortragstechnik&sehr gut&gut&gen"ugend&ungen"ugend&kinb\\
\hline
\raisebox{0pt}[13pt][6pt]{1}&
\begin{minipage}{6.6cm}\raggedright\strut
Die Ziele des Vortrags waren klar, auf die Erreichung
von Zwischenzielen wurde hingewiesen
\strut\end{minipage}&
\raisebox{0pt}[24pt][15pt]{\phantom{XX}}&
\raisebox{0pt}[24pt][15pt]{\phantom{XX}}&
\raisebox{0pt}[24pt][15pt]{\phantom{XX}}&
\raisebox{0pt}[24pt][15pt]{\phantom{XX}}&
\raisebox{0pt}[24pt][15pt]{\phantom{XX}}\\
\hline
\raisebox{0pt}[13pt][6pt]{2}&
\begin{minipage}{6.6cm}\raggedright\strut
Der Vortrag war strukturiert, der ``rote Faden''
war jederzeit erkennbar
\strut\end{minipage}&
\raisebox{0pt}[24pt][15pt]{\phantom{XX}}&
\raisebox{0pt}[24pt][15pt]{\phantom{XX}}&
\raisebox{0pt}[24pt][15pt]{\phantom{XX}}&
\raisebox{0pt}[24pt][15pt]{\phantom{XX}}&
\raisebox{0pt}[24pt][15pt]{\phantom{XX}}\\
\hline
\raisebox{0pt}[13pt][6pt]{3}&
\begin{minipage}{6.6cm}\raggedright\strut
Die Zeiteinteilung war zweckm"assig
\strut\end{minipage}&
\raisebox{0pt}[24pt][15pt]{\phantom{XX}}&
\raisebox{0pt}[24pt][15pt]{\phantom{XX}}&
\raisebox{0pt}[24pt][15pt]{\phantom{XX}}&
\raisebox{0pt}[24pt][15pt]{\phantom{XX}}&
\raisebox{0pt}[24pt][15pt]{\phantom{XX}}\\
\hline
\raisebox{0pt}[13pt][6pt]{4}&
\begin{minipage}{6.6cm}\raggedright\strut
Die gew"ahlten Pr"asentations\-mittel waren der Problemstellung
angemessen
\strut\end{minipage}&
\raisebox{0pt}[24pt][15pt]{\phantom{XX}}&
\raisebox{0pt}[24pt][15pt]{\phantom{XX}}&
\raisebox{0pt}[24pt][15pt]{\phantom{XX}}&
\raisebox{0pt}[24pt][15pt]{\phantom{XX}}&
\raisebox{0pt}[24pt][15pt]{\phantom{XX}}\\
\hline
\raisebox{0pt}[13pt][6pt]{5}&
\begin{minipage}{6.6cm}\raggedright\strut
Die Vortragenden waren engagiert
\strut\end{minipage}&
\raisebox{0pt}[24pt][15pt]{\phantom{XX}}&
\raisebox{0pt}[24pt][15pt]{\phantom{XX}}&
\raisebox{0pt}[24pt][15pt]{\phantom{XX}}&
\raisebox{0pt}[24pt][15pt]{\phantom{XX}}&
\raisebox{0pt}[24pt][15pt]{\phantom{XX}}\\
\hline
\hline
\raisebox{0pt}[13pt][6pt]{B}&Inhaltliche Beurteilung&sehr gut&gut&gen"ugend&ungen"ugend&kinb\\
\hline
\raisebox{0pt}[13pt][6pt]{1}&
\begin{minipage}{6.6cm}\raggedright\strut
Die mathematische Problemstellung wurde erkl"art
\strut\end{minipage}&
\raisebox{0pt}[24pt][15pt]{\phantom{XX}}&
\raisebox{0pt}[24pt][15pt]{\phantom{XX}}&
\raisebox{0pt}[24pt][15pt]{\phantom{XX}}&
\raisebox{0pt}[24pt][15pt]{\phantom{XX}}&
\raisebox{0pt}[24pt][15pt]{\phantom{XX}}\\
\hline
\raisebox{0pt}[13pt][6pt]{2}&
\begin{minipage}{6.6cm}\raggedright\strut
Der L"osungsansatz war verst"andlich
\strut\end{minipage}&
\raisebox{0pt}[24pt][15pt]{\phantom{XX}}&
\raisebox{0pt}[24pt][15pt]{\phantom{XX}}&
\raisebox{0pt}[24pt][15pt]{\phantom{XX}}&
\raisebox{0pt}[24pt][15pt]{\phantom{XX}}&
\raisebox{0pt}[24pt][15pt]{\phantom{XX}}\\
\hline
\raisebox{0pt}[13pt][6pt]{3}&
\begin{minipage}{6.6cm}\raggedright\strut
Der Detailierungsgrad war ausreichend, um die
Lösung zu verstehen
\strut\end{minipage}&
\raisebox{0pt}[24pt][15pt]{\phantom{XX}}&
\raisebox{0pt}[24pt][15pt]{\phantom{XX}}&
\raisebox{0pt}[24pt][15pt]{\phantom{XX}}&
\raisebox{0pt}[24pt][15pt]{\phantom{XX}}&
\raisebox{0pt}[24pt][15pt]{\phantom{XX}}\\
\hline
\raisebox{0pt}[13pt][6pt]{4}&
\begin{minipage}{6.6cm}\raggedright\strut
Verwendete Notation und graphischen Darstellungen waren
zweckm"assig und verst"andlich
\strut\end{minipage}&
\raisebox{0pt}[24pt][15pt]{\phantom{XX}}&
\raisebox{0pt}[24pt][15pt]{\phantom{XX}}&
\raisebox{0pt}[24pt][15pt]{\phantom{XX}}&
\raisebox{0pt}[24pt][15pt]{\phantom{XX}}&
\raisebox{0pt}[24pt][15pt]{\phantom{XX}}\\
\hline
\raisebox{0pt}[13pt][6pt]{5}&
\begin{minipage}{6.6cm}\raggedright\strut
Der Vortrag war interessant
\strut\end{minipage}&
\raisebox{0pt}[24pt][15pt]{\phantom{XX}}&
\raisebox{0pt}[24pt][15pt]{\phantom{XX}}&
\raisebox{0pt}[24pt][15pt]{\phantom{XX}}&
\raisebox{0pt}[24pt][15pt]{\phantom{XX}}&
\raisebox{0pt}[24pt][15pt]{\phantom{XX}}\\
\hline
%\hline
%\raisebox{0pt}[13pt][6pt]{C}&
%\begin{minipage}{6.6cm}\raggedright\strut
%Auswertung
%\strut\end{minipage}&
%\raisebox{0pt}[24pt][15pt]{\phantom{XX}}&
%\raisebox{0pt}[24pt][15pt]{\phantom{XX}}&
%\raisebox{0pt}[24pt][15pt]{\phantom{XX}}&
%\raisebox{0pt}[24pt][15pt]{\phantom{XX}}&
%\raisebox{0pt}[24pt][15pt]{\phantom{XX}}\\
%\hline
\end{tabular}
\end{center}

\vspace{0.8mm}
{\parindent0pt\hbox to\hsize{%
{\large Bemerkungen:}}}


\end{document}
