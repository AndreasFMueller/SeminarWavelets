%
% skalarprodukt.tex
%
% (c) 2019 Prof Dr Andreas Müller, Hochschule Rapperswil
%
\theoremstyle{definition}
\newtheorem{cauchyschwarz}{Chauchy-Schwarz-Ungleichung}

\begin{frame}
\frametitle{Skalarprodukt}
Abgetastete Signale:
\[
\langle x,y\rangle
=
\sum_{k\in\mathbb Z} x_k\bar{y}_k
\]
\uncover<2->{%
$f$, $g$ zeitabhängige Signale $t\mapsto f(t)$ und $t\mapsto g(t)$
\begin{equation}
\langle f,g\rangle
=
\int_{-\infty}^\infty f(t)\bar{g}(t)\,dt
\label{skalarprodukt}
\end{equation}%
}
\uncover<3->{%
\begin{definition}
Die quadratintegrierbaren Funktionen $L^2=\{f\colon\mathbb R\to\mathbb C\,|
\,\int_{\mathbb R}|f(t)|^2\,dt<\infty\}$ bilden einen Vektorraum mit dem Skalarprodukt
\eqref{skalarprodukt}.
\end{definition}
}
\uncover<4->{%
\begin{cauchyschwarz}
Für $f,g\in L^2(\mathbb R)$ gilt
\[
\langle f,g\rangle \le
\|f\|\cdot \|g\|
\]
mit Gleichheit genau dann wenn $f$ und $g$ linear abhängig sind.
\end{cauchyschwarz}
}
\end{frame}


