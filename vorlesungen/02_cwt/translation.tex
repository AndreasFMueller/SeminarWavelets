%
% translation.tex
%
% (c) 2019 Prof Dr Andreas Müller, Hochschule Rapperswil
%
\begin{frame}
\frametitle{Translation}

\begin{center}
\begin{tikzpicture}[>=latex]

\draw[line width=1pt,color=blue!40] (-6.5,-3)--(6.5,-3);

\ifthenelse{\boolean{presentation}}{
\node at (0,2) [left] {$T_{b}\psi(t)=\mathstrut$};


\foreach \b in {1,...,10}{
	\only<\b>{
		\pgfmathparse{0.5*(\b-11)}
		\xdef\B{\pgfmathresult}
		\pgfmathparse{-\B}
		\xdef\minusB{\pgfmathresult}
		\node at (0,2) [right] {$T_{\B}\psi(t)=\psi(t+\minusB)$};
		\flaeche{1}{\B}{1}
		\achsen{1.5}
		\curve{1}{\B}{1}
		\cwt{1}{\B}{-2.0}
	}
}
\only<11>{
	\node at (0,2) [right] {$T_0\psi(t)=\psi(t)$};
	\flaeche{1}{0}{1}
	\achsen{1.5}
	\curve{1}{0}{1}
	\cwt{1}{0}{-2.0}
}
\foreach \b in {12,...,21}{
	\only<\b>{
		\pgfmathparse{0.5*(\b-11)}
		\xdef\B{\pgfmathresult}
		\pgfmathparse{-\B}
		\xdef\minusB{\pgfmathresult}
		\node at (0,2) [right] {$T_{\B}\psi(t)=\psi(t-\B)$};
		\flaeche{1}{\B}{1}
		\achsen{1.5}
		\curve{1}{\B}{1}
		\cwt{1}{\B}{-2.0}
	}
}
}{
	\node at (0,2) {$T_{b}\psi(t)= T_{3}\psi(t)=\psi(t-3)$};
	\flaeche{1}{3}{1}
	\achsen{1.5}
	\curve{1}{3}{1}
	\cwt{1}{3}{-2.0}
}

\end{tikzpicture}
\end{center}

\end{frame}

