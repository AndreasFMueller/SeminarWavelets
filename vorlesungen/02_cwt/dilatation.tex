%
% dilatation.tex
%
% (c) 2019 Prof Dr Andreas Müller, Hochschule Rapperswil
%
\begin{frame}
\frametitle{Dilatation}
\begin{center}
\begin{tikzpicture}[>=latex]

\draw[line width=1pt,color=blue!40] (2.5,-4)--(2.5,-1.3);

\ifthenelse{\boolean{presentation}}{

\node at (0,2) [left] {$T_{2.5}D_{a}\psi(t)=\mathstrut$};

\foreach \a in {1,...,5}{
	\only<\a>{
		\pgfmathparse{exp(0.2*(\a-6))}
		\xdef\A{\pgfmathresult}
		\node at (0,2) [right]
			{$T_{2.5}D_{\A}\psi(t)=\psi((t-2.5)/\A)$};
		\flaeche{\A}{2.5}{1}
		\achsen{1.5}
		\curve{\A}{2.5}{1}
		\cwt{\A}{2.5}{-2.0}
	}
}
\only<6>{
	\node at (0,2) [right] {$T_{2.5}D_1\psi(t)=\psi(t-2.5)$};
	\flaeche{1}{2.5}{1}
	\achsen{1.5}
	\curve{1}{2.5}{1}
	\cwt{1}{2.5}{-2.0}
}
\foreach \a in {7,...,11}{
	\only<\a>{
		\pgfmathparse{exp(0.2*(\a-6))}
		\xdef\A{\pgfmathresult}
		\node at (0,2) [right]
			{$T_{2.5}D_{\A}\psi(t)=\psi((t-2.5)/\A)$};
		\flaeche{\A}{2.5}{1}
		\achsen{1.5}
		\curve{\A}{2.5}{1}
		\cwt{\A}{2.5}{-2.0}
	}
}
}{
	\node at (0,2) {$T_bD_a\psi(t)=T_{2.5}D_a\psi(t)=\psi((t-2.5)/a)=\psi((5-2.5)/1.4)$};
	\flaeche{1.4}{2.5}{1}
	\achsen{1.5}
	\curve{1.4}{2.5}{1}
	\cwt{1.4}{2.5}{-2.0}
}

\end{tikzpicture}
\end{center}

\end{frame}

