%
% resultate.tex -- Resultate
%
% (c) 2019 Prof Dr Andreas Müller, Hochschule Rapperswil
%
\begin{frame}
\frametitle{Resultate}
\begin{zulaessig}
Eine Funktion $\psi\in L^2(\mathbb R)$ mit $\|\psi\|=1$ ist zulässig,
wenn
\[
C_{\psi}
=
2\pi
\int_{\mathbb R^*} \frac{|\hat{\psi}(a\omega)|^2}{|a|}\,da < \infty
\uncover<6->{%
\quad\Rightarrow\quad
\hat{\psi}(0)=0}
\uncover<7->{%
\quad\Rightarrow\quad
\int_{-\infty}^\infty \psi(t)\,dt = 0}
\]
\uncover<2->{%
Gilt zum Beispiel für differenzierbare Wavelets mit kompaktem Träger.
}
\end{zulaessig}

\uncover<3->{%
\begin{umkehrformelW}
Für $f\in L^2(\mathbb R)$ gilt
\[
\mathring{f}(t)
=
\frac{1}{C_\psi}
\int_{\mathbb R^*}\int_{-\infty}^\infty \mathcal{W}f(a,b)\,\psi_{a,b}(t)\,db\,\frac{da}{|a|^2}
\uncover<4->{
\quad\Rightarrow\quad
\langle\mathring{f},g\rangle = \langle f,g\rangle
\quad\forall g\in L^2(\mathbb R)
}
\]
\uncover<5->{Unter zusätzlichen ``Regularitätsbedingungen'':
$\mathring{f}(t)=f(t)$}
\end{umkehrformelW}}

\end{frame}
