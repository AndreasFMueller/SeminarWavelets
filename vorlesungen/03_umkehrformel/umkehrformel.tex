%
% umkehrformel.tex
%
% (c) 2019 Prof Dr Andreas Müller, Hochschule Rapperswil
%
\begin{frame}
\frametitle{Umkehrformel für $\mathbb R^n$}
Gegeben: $\mathcal{B}=\{\vec{b}_1,\dots,\vec{b}_n\}\subset\mathbb R^n$
orthonormierte Basis, d.~h.
\[
\langle \vec{b}_j,\vec{b}_i\rangle = \delta_{ji}
\]
\uncover<2->{%
Koordinaten eines Vektors $\vec{v}$ in der Basis
\[
	v_i = \langle \vec{v},\vec{b}_i\rangle
\]
}
\uncover<3->{%
Rekonstruktion:
\[
\vec{w}
=
\sum_{j=1}^{n} v_j\vec{b}_j
\uncover<4->{%
=
\sum_{j=1}^n \langle \vec{v},\vec{b}_j\rangle \vec{b}_j}
\]
}
\uncover<5->{%
Verifikation: $\vec{w}$ hat die gleichen Koordinaten wie $\vec{v}$
\begin{align*}
\uncover<6->{\langle \vec{w},\vec{b}_i\rangle}
&\uncover<7->{=\biggl\langle
\sum_{j=1}^n \langle \vec{v},\vec{b}_j\rangle\vec{b}_j, \vec{b}_i
\biggr\rangle}
\uncover<8->{=
\sum_{j=1}^n \langle \vec{v},\vec{b}_j\rangle \langle \vec{b}_j,\vec{b}_i\rangle}
\uncover<9->{=
\sum_{j=1}^n \langle \vec{v},\vec{b}_j\rangle \delta_{ji}}
\uncover<10->{=
\langle \vec{v},\vec{b}_i\rangle}
\uncover<11->{\quad{\color{red}\checkmark}}
\end{align*}
}

\end{frame}

%
% Vermutung für Umkehrformel für CWT
%
\begin{frame}
\frametitle{Umkehrformel für CWT}
\begin{cwt}
Wavelet-Transformierte von $f\in L^2(\mathbb R)$:
\[
\mathcal{W}f(a,b)
=
\frac{1}{\sqrt{|a|}}
\int_{-\infty}^\infty f(t) \overline{\psi\biggl(\frac{t-b}{a}\biggr)} \,dt
\uncover<2->{=
\langle f,\psi_{a,b}\rangle}
\]
\end{cwt}
\uncover<3->{%
Analyse mit
\begin{itemize}
\item<4->
$L^2$-Skalarprodukt
\item<5->
``Basis'': $T_bD_a \psi = \psi_{a,b}$
\end{itemize}
}
\uncover<6->{%
\begin{umkehrformel}
\ifthenelse{\boolean{presentation}}{
\only<6>{%
Für $\vec{v}\in \mathbb R^n$ mit Basis
$\mathcal{B}=\{\vec{b}_1,\dots,\vec{b}_n\}$
}
\only<7>{%
Für $f\in L^2$ mit Basis
$\mathcal{B}=\{\vec{b}_1,\dots,\vec{b}_n\}$
}
\only<8-13>{%
Für $f\in L^2$ mit ``Basis''
$\psi_{a,b}$:
}
\only<14->{für $\mathcal{W}f(a,b)$:\phantom{$L^2$ mit ``Basis''}}
\[
\only<6>{
\vec{v}=\sum_{j=1}^\infty \langle \vec{v},\vec{b}_j\rangle\,\vec{b}_j
}
\only<7>{
f=\sum_{j=1}^\infty \langle f,\vec{b}_j\rangle\, \vec{b}_j
}
\only<8>{
f=\sum_{a,b}^{\phantom{\infty}} \langle f,\psi_{a,b}\rangle\, \psi_{a,b}
}
\only<9>{
f=\sum_{a,b}^{\phantom{\infty}} \mathcal{W}f(a,b)\,\psi_{a,b}
}
\only<10>{
f(t)=\sum_{a,b}^{\phantom{\infty}} \mathcal{W}f(a,b)\,\psi_{a,b}(t)
}
\only<11>{
f(t)=\sum_{a}^{\phantom{\infty}} \int_{-\infty}^\infty \mathcal{W}f(a,b)\,\psi_{a,b}(t)\, db
}
\only<12>{
f(t)=\int_{\mathbb R^+} \int_{-\infty}^\infty \mathcal{W}f(a,b)\psi_{a,b}(t)\, db \,da
}
\only<13>{
f(t)=\int_{\mathbb R^+} \int_{-\infty}^\infty \mathcal{W}f(a,b)\psi_{a,b}(t)\, db \,\frac{da}{|a|^2}
}
\only<14>{
f(t)
\stackrel{{\color{red}?}}{=}
\int_{\mathbb R^+} \int_{-\infty}^\infty \mathcal{W}f(a,b)\psi_{a,b}(t)\, db \,\frac{da}{|a|^2}
}
\]}{
\[
f(t)
\stackrel{{\color{red}?}}{=}
\int_{\mathbb R^+} \int_{-\infty}^\infty \mathcal{W}f(a,b)\psi_{a,b}(t)\, db \,\frac{da}{|a|^2}
\]
}
\end{umkehrformel}
}
\end{frame}

