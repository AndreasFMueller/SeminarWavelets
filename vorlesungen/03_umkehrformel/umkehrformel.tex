%
% umkehrformel.tex
%
% (c) 2019 Prof Dr Andreas Müller, Hochschule Rapperswil
%
\begin{frame}
\frametitle{Umkehrformel für $\mathbb R^n$}
Gegeben: $\mathcal{B}=\{\vec{b}_1,\dots,\vec{b}_n\}\subset\mathbb R^n$
orthonormierte Basis, d.~h.
\[
\langle \vec{b}_j,\vec{b}_i\rangle = \delta_{ji}
\]
\uncover<2->{%
Koordinaten eines Vektors $\vec{v}$ in der Basis
\[
	v_i = \langle \vec{v},\vec{b}_i\rangle
\]
}
\uncover<3->{%
Rekonstruktion:
\[
\vec{w}
=
\sum_{j=1}^{n} v_j\vec{b}_j
\uncover<4->{%
=
\sum_{j=1}^n \langle \vec{v},\vec{b}_j\rangle \vec{b}_j}
\]
}
\uncover<5->{%
Verifikation: $\vec{w}$ hat die gleichen Koordinaten wie $\vec{v}$
\begin{align*}
\uncover<6->{\langle \vec{w},\vec{b}_i\rangle}
&\uncover<7->{=\biggl\langle
\sum_{j=1}^n \langle \vec{v},\vec{b}_j\rangle\vec{b}_j, \vec{b}_i
\biggr\rangle}
\uncover<8->{=
\sum_{j=1}^n \langle \vec{v},\vec{b}_j\rangle \langle \vec{b}_j,\vec{b}_i\rangle}
\uncover<9->{=
\sum_{j=1}^n \langle \vec{v},\vec{b}_j\rangle \delta_{ji}}
\uncover<10->{=
\langle \vec{v},\vec{b}_i\rangle}
\uncover<11->{\quad{\color{red}\checkmark}}
\end{align*}
}

\end{frame}

%
% Vermutung für Umkehrformel für CWT
%
\begin{frame}
\frametitle{Umkehrformel für CWT}
\begin{cwt}
Wavelet-Transformierte von $f\in L^2(\mathbb R)$:
\[
\mathcal{W}f(a,b)
=
\frac{1}{\sqrt{|a|}}
\int_{-\infty}^\infty f(t) \overline{\psi\biggl(\frac{t-b}{a}\biggr)} \,dt
\uncover<2->{=
\langle f,\psi_{a,b}\rangle}
\]
\end{cwt}
\uncover<3->{%
Analyse mit
\begin{itemize}
\item<4->
$L^2$-Skalarprodukt
\item<5->
``Basis'': $T_bD_a \psi = \psi_{a,b}$
\end{itemize}
}
\uncover<6->{%
\begin{umkehrformel}
\ifthenelse{\boolean{presentation}}{
\only<6>{%
Für $\vec{v}\in \mathbb R^n$ mit Basis
$\mathcal{B}=\{\vec{b}_1,\dots,\vec{b}_n\}$
}
\only<7>{%
Für $f\in L^2$ mit Basis
$\mathcal{B}=\{\vec{b}_1,\dots,\vec{b}_n\}$
}
\only<8-13>{%
Für $f\in L^2$ mit ``Basis''
$\psi_{a,b}$:
}
\only<14->{für $\mathcal{W}f(a,b)$:\phantom{$L^2$ mit ``Basis''}}
\[
\only<6>{
\vec{v}=\sum_{j=1}^\infty \langle \vec{v},\vec{b}_j\rangle\,\vec{b}_j
}
\only<7>{
f=\sum_{j=1}^\infty \langle f,\vec{b}_j\rangle\, \vec{b}_j
}
\only<8>{
f=\sum_{a,b}^{\phantom{\infty}} \langle f,\psi_{a,b}\rangle\, \psi_{a,b}
}
\only<9>{
f=\sum_{a,b}^{\phantom{\infty}} \mathcal{W}f(a,b)\,\psi_{a,b}
}
\only<10>{
f(t)=\sum_{a,b}^{\phantom{\infty}} \mathcal{W}f(a,b)\,\psi_{a,b}(t)
}
\only<11>{
f(t)=\sum_{a}^{\phantom{\infty}} \int_{-\infty}^\infty \mathcal{W}f(a,b)\,\psi_{a,b}(t)\, db
}
\only<12>{
f(t)=\int_{\mathbb R^+} \int_{-\infty}^\infty \mathcal{W}f(a,b)\psi_{a,b}(t)\, db \,da
}
\only<13>{
f(t)=\int_{\mathbb R^+} \int_{-\infty}^\infty \mathcal{W}f(a,b)\psi_{a,b}(t)\, db \,\frac{da}{|a|^2}
}
\only<14>{
f(t)
\stackrel{{\color{red}?}}{=}
\int_{\mathbb R^+} \int_{-\infty}^\infty \mathcal{W}f(a,b)\psi_{a,b}(t)\, db \,\frac{da}{|a|^2}
}
\]}{
\[
f(t)
\stackrel{{\color{red}?}}{=}
\int_{\mathbb R^+} \int_{-\infty}^\infty \mathcal{W}f(a,b)\psi_{a,b}(t)\, db \,\frac{da}{|a|^2}
\]
}
\end{umkehrformel}
}
\end{frame}

%
% Verifikation der Umkehrformel
%
\begin{frame}
\frametitle{Vergleichsprinzip}

\begin{vergleich}
Für zwei Funktionen $f_1$ und $f_2$ in $L^2(\mathbb R)$ gilt
\[
\left.
\begin{aligned}
f_1&=f_2
&&\Leftrightarrow&
\langle f_1,g\rangle &= \langle f_2,g\rangle
\\
f_1-f_2&=0
&&\Leftrightarrow&
\langle f_1-f_2,g\rangle &= 0
\end{aligned}
\quad\right\}
\quad
\text{für alle $g\in L^2(\mathbb R)$}
\]
\end{vergleich}
\vspace*{-10pt}

\begin{proof}[Beweis]
Richtung $\boxed{\Rightarrow}$\;: Verwende Cauchy-Schwarz-Ungleichung
\begin{align*}
f_1&=f_2
&&\Rightarrow&
f_1-f_2&=0
&&\Rightarrow&
|\langle f_1-f_2,g\rangle|
&\le
\underbrace{\|f_1-f_2\|}_{\displaystyle=0}\cdot \|g\|
\end{align*}
\vspace{-20pt}

Richtung $\boxed{\Leftarrow}$\;: Verwende $g=f_1-f_2$
\begin{align*}
\langle f_1-f_2,g\rangle&=0
&&\Rightarrow&
\|f_1-f_2\|
&=
\langle f_1-f_2,f_1-f_2\rangle
=
0
&&\Rightarrow&
f_1-f_2&=0
\end{align*}
\end{proof}

\end{frame}

%
% Verifikation der Umkehrformel
%
\begin{frame}
\frametitle{Verifikation der Umkehrformel $\mathbb R^n$}
Umkehrformel muss gleiche Skalarprodukte mit $\vec{w}$ haben wie $\vec{v}$:
\[
\langle \vec{v},\vec{w}\rangle
\stackrel{?}{=}
\langle \text{Umkehrformel für $\vec{v}$},\vec{w}\rangle
\qquad\text{für alle $\vec{w}\in\mathbb R^n$}
\]
Nachrechnen
\begin{align*}
&\hbox to8cm{\hfill}\\[-20pt]
\langle\vec{v},\vec{w}\rangle
&\stackrel{?}{=}
\ifthenelse{\boolean{presentation}}{
\only<1>{
\biggl\langle
\sum_{j=1}^n \langle \vec{v},\vec{b}_j\rangle\,\vec{b}_j,\vec{w}
\biggr\rangle
}
\only<2>{
\sum_{j=1}^n \langle \vec{v},\vec{b}_j\rangle\,\langle\vec{b}_j,\vec{w}\rangle
}
\only<3>{
\sum_{j=1}^n \langle \vec{v},\vec{b}_j\rangle\,
\overline{\langle\vec{w},\vec{b}_j\rangle}
}
\only<4->{
\sum_{j=1}^n v_j\bar{w}_j
}
\only<5->{
\qquad\text{Parseval-/Plancherel-Formel}
}}{
\sum_{j=1}^n v_j\bar{w}_j
\qquad\text{Parseval-/Plancherel-Formel}
}
\end{align*}
\uncover<6->{%
\begin{parseval}
$f,g$ $2\pi$-periodische Formeln mit Fourierkoeffizienten 
$a_k^f,a_k^g$ und $b_k^f,b_k^g$, dann gilt
\vspace{-10pt}

\[
\langle f,g\rangle
=
\frac12
a_0^f \overline{a_0^g}
+
\sum_{k=1}^\infty
(
a_k^f \overline{a_k^g}
+
b_k^f \overline{b_k^g}
)
\]
\end{parseval}}
\vspace*{-10pt}
\uncover<7->{%
\begin{plancherel}
Für $f,g\in L^2(\mathbb R)$ gilt
\[
\langle f, g\rangle
=
\langle \mathcal{F}f,\mathcal{F}g\rangle
=
\langle \hat{f},\hat{g}\rangle
\]
\end{plancherel}}

\end{frame}

%
% Verifikation der Umkehrformel
%
\begin{frame}
\frametitle{Verifikation der Umkehrformel $\mathcal{W}f$}

Umkehrformel muss gleiche Skalarprodukte mit $g$ haben wie $f$:
\[
\langle f,g\rangle
\stackrel{?}{=}
\langle \text{Umkehrformel für $f$},g\rangle
\qquad\text{für alle $g\in L^2$}
\]
Nachrechnen
\begin{align*}
&\hbox to8cm{\hfill}\\[-20pt]
\langle f, g\rangle
&
\stackrel{?}{=}
\ifthenelse{\boolean{presentation}}{
\only<1>{
\biggl\langle
\int_{\mathbb R^+}\int_{-\infty}^\infty
\mathcal{W}f(a,b)\,\psi_{a,b}
\, db\,\frac{da}{|a|^2},
g
\biggr\rangle}
\only<2>{
\int_{-\infty}^\infty
\int_{\mathbb R^+}\int_{-\infty}^\infty
\mathcal{W}f(a,b)\,\psi_{a,b}(t)
\, db\,\frac{da}{|a|^2}
\,
\overline{g(t)}
\,dt
}
\only<3>{
\int_{\mathbb R^+}\int_{-\infty}^\infty
\mathcal{W}f(a,b)
\int_{-\infty}^\infty
\psi_{a,b}(t)
\,
\overline{g(t)}
\,dt
\, db\,\frac{da}{|a|^2}
}
\only<4>{
\int_{\mathbb R^+}\int_{-\infty}^\infty
\mathcal{W}f(a,b)
\langle
\psi_{a,b},
g
\rangle
\, db\,\frac{da}{|a|^2}
}
\only<5>{
\int_{\mathbb R^+}\int_{-\infty}^\infty
\mathcal{W}f(a,b)
\overline{
\langle
g,
\psi_{a,b}
\rangle}
\, db\,\frac{da}{|a|^2}
}
\only<6-8>{
\int_{\mathbb R^+}\int_{-\infty}^\infty
\mathcal{W}f(a,b)
\overline{\mathcal{W}g(a,b)}
\, db\,\frac{da}{|a|^2}}
\only<9->{
\langle 
\mathcal{W}f(a,b),
\mathcal{W}g(a,b)
\rangle_H}
\only<10->{\qquad\text{Plancherel-Formel}}}{
\langle 
\mathcal{W}f(a,b),
\mathcal{W}g(a,b)
\rangle_H
\qquad\text{Plancherel-Formel}
}
\end{align*}

\uncover<7->{
\begin{definition}
Die Menge $H=\{(a,b)\,|\, a\in\mathbb R^*\wedge b\in\mathbb R\}$
heisst die {\em Heisenberg-Gruppe}.
\uncover<8->{%
Für Funktionen auf $H$ gilt das Skalarprodukt
\[
\langle u,v\rangle_H
=
\int_{\mathbb R^*}\int_{-\infty}^\infty
u(a,b)\overline{v(a,b)}
\,db\,\frac{da}{|a|^2}
\]
}
\end{definition}
}

\end{frame}

%
% Plan für den Beweis der Plancherel Formel
%
\begin{frame}
\frametitle{Plan für den Beweis der Plancherel-Formel}
\begin{enumerate}
\item<2-> $a$ als konstant betrachten:
\[
\int_{-\infty}^\infty
\mathcal{W}f(a,b)\overline{\mathcal{W}g(a,b)}
\,db
=
\langle \mathcal{W}f(a,\,\cdot\,), \mathcal{W}g(a,\,\cdot\,)\rangle
\]
\uncover<3->{%
Plancherel-Formel für Fourier-Transformation ist anwendbar.
}
\item<4-> Plancherel-Formel für die Integration über $a\in\mathbb R^*$ 
\end{enumerate}
\end{frame}

%
% Partielle Funktion $b\mapsto \mathcal{W}f(a,b)$
%
\begin{frame}
\frametitle{Partielle Funktion $b\mapsto\mathcal{W}f(a,b)$}
\begin{align*}
&\hbox to11cm{\hfill}\\[-20pt]
b \mapsto
\mathcal{W}f(a,b)&=
\langle f,\psi_{a,b}\rangle
=
\int_{-\infty}^\infty 
f(t) \overline{\psi_{a,b}(t)}\,dt
\intertext{\uncover<2->{Umformung mit Plancherel-Formel}}
\uncover<3->{
\langle f,\psi_{a,b}\rangle}
&\uncover<4->{=
\langle \hat{f}, \widehat{\psi_{a,b}}\rangle}
\uncover<4->{=
\int_{-\infty}^\infty
\hat{f}(\omega) \overline{\widehat{\psi_{a,b}}(\omega)}
\,d\omega}
\\
&\uncover<5->{=
\int_{-\infty}^\infty
\hat{f}(\omega) \overline{
\frac{1}{\sqrt{2\pi}}
\int_{-\infty}^\infty
\psi_{a,b}(t) e^{-i\omega t}
\,dt
}
\,d\omega
}
\\
&\only<6>{=
\int_{-\infty}^\infty
\hat{f}(\omega) \overline{
\frac{1}{\sqrt{2\pi}}
\int_{-\infty}^\infty
\psi_{a,0}(t') e^{-i\omega (t'+b)}
\,dt'
}
\,d\omega
}
\only<7>{=
\int_{-\infty}^\infty
\hat{f}(\omega)
e^{i\omega b}
\overline{
\frac{1}{\sqrt{2\pi}}
\int_{-\infty}^\infty
\psi_{a,0}(t') e^{-i\omega t'}
\,dt'
}
\,d\omega
}
\only<8>{=
\int_{-\infty}^\infty
\hat{f}(\omega)
e^{i\omega b}
\sqrt{|a|}
\,
\overline{
\widehat{\psi}(a\omega)
}
\,d\omega
}
\only<9>{=
\int_{-\infty}^\infty
\hat{f}(\omega)
\sqrt{|a|}
\,
\overline{
\widehat{\psi}(a\omega)
}
e^{i\omega b}
\,d\omega
}
\only<10>{=
\frac{1}{\sqrt{2\pi}}
\int_{-\infty}^\infty
\sqrt{2\pi}
\hat{f}(\omega)
\sqrt{|a|}
\,
\overline{
\widehat{\psi}(a\omega)
}
e^{i\omega b}
\,d\omega
}
\only<11->{=
\frac{1}{\sqrt{2\pi}}
\int_{-\infty}^\infty
\underbrace{
\sqrt{2\pi}
\hat{f}(\omega)
\sqrt{|a|}
\,
\overline{
\widehat{\psi}(a\omega)
}}_{\displaystyle=F_a(\omega)}
e^{i\omega b}
\,d\omega
}
%\only<12->{=
%\frac{1}{\sqrt{2\pi}}
%\int_{-\infty}^\infty
%\sqrt{2\pi}
%\hat{f}(\omega)
%\sqrt{|a|}
%\,
%\overline{
%\widehat{\psi}(a\omega)
%}
%e^{i\omega b}
%\,d\omega
%}
\\
\uncover<13->{\mathcal{W}f(a,b)}
&\uncover<12->{=\check{F}_a(b)}
\end{align*}
\end{frame}


%
% Plancherel Formel für \mathcal{W}
%
\begin{frame}
\frametitle{Plancherel-Formel für $\mathcal{W}$}
\begin{plancherelW}
\[
\langle \mathcal{W}f,\mathcal{W}g\rangle_H
=
C_{\psi} \langle f,g\rangle
\]
\end{plancherelW}

\uncover<2->{
\begin{proof}[Beweis]
Durch Nachrechnen
\begin{align*}
&\hbox to11cm{\hfill}\\[-20pt]
\uncover<3->{\langle \mathcal{W}f,\mathcal{W}g\rangle_H}
&\uncover<4->{=
\int_{\mathbb R^*}\int_{-\infty}^\infty
\mathcal{W}f(a,b)
\overline{\mathcal{W}g(a,b)}
\,db\,\frac{da}{|a|^2}}
\\
&\only<5>{=
\int_{\mathbb R^*} \int_{-\infty}^\infty
\check{F}_a(b) \overline{\check{G}_a(b)} \,db\,\frac{da}{|a|^2}}
\only<6>{=
\int_{\mathbb R^*}
\langle \check{F}_a,\check{G}_a\rangle
\,\frac{da}{|a|^2}}
\only<7>{=
\int_{\mathbb R^*}
\langle F_a,G_a\rangle
\,\frac{da}{|a|^2}}
\only<8>{=
\int_{\mathbb R^*}
\int_{-\infty}^\infty F_a(\omega)\overline{G_a(\omega)}\,d\omega
\,\frac{da}{|a|^2}}
\only<9>{=
\int_{\mathbb R^*}
\int_{-\infty}^\infty
\sqrt{2\pi}\sqrt{|a|}\,\hat{f}(\omega) \,\overline{\hat{\psi}(a\omega)}
\,\overline{\sqrt{2\pi}\,\sqrt{|a|}\,\hat{g}(\omega)\, \overline{\hat{\psi}(a\omega)}}
\,d\omega
\,\frac{da}{|a|^2}}
\only<10->{=
\int_{\mathbb R^*} \int_{-\infty}^\infty
2\pi |a|\,\hat{f}(\omega)\, \overline{\hat{\psi}(a\omega)}\,
\overline{\hat{g}(\omega)}\,  \hat{\psi}(a\omega)
\,d\omega
\,\frac{da}{|a|^2}
}
\\
&
\only<11>{=
2\pi
\int_{-\infty}^\infty
\hat{f}(\omega) 
\overline{\hat{g}(\omega)}
\int_{\mathbb R^*}
|\hat{\psi}(a\omega)|^2
\,\frac{da}{|a|}
\,d\omega
}
\only<12>{=
\int_{-\infty}^\infty
\hat{f}(\omega) 
\overline{\hat{g}(\omega)}
\underbrace{
2\pi
\int_{\mathbb R^*}
|\hat{\psi}(a\omega)|^2
\,\frac{da}{|a|}}_{\displaystyle = C_\psi}
\,d\omega
}
\qedhere
\end{align*}
\end{proof}
}
\end{frame}

%
% Resultate
%
\begin{frame}
\frametitle{Resultate}
\begin{zulaessig}
Eine Funktion $\psi\in L^2(\mathbb R)$ mit $\|\psi\|=1$ ist zulässig,
wenn
\[
C_{\psi}
=
2\pi
\int_{\mathbb R^*} \frac{|\hat{\psi}(a\omega)|^2}{|a|}\,da < \infty
\uncover<4->{%
\quad\Rightarrow\quad
\hat{\psi}(0)=0
\quad\Rightarrow\quad
\int_{-\infty}^\infty \psi(t)\,dt = 0
}
\]
\uncover<2->{%
Gilt zum Beispiel für differenzierbare Wavelets mit kompaktem Träger.
}
\end{zulaessig}

\uncover<3->{%
\begin{umkehrformelW}
Für $f\in L^2(\mathbb R)$ gilt
\[
\mathring{f}(t)
=
\frac{1}{C_\psi}
\int_{\mathbb R^*}\int_{-\infty}^\infty \mathcal{W}f(a,b)\,\psi_{a,b}(t)\,db\,\frac{da}{|a|^2}
\quad\Rightarrow\quad
\langle\mathring{f},g\rangle = \langle f,g\rangle\quad\forall g\in L^2(\mathbb R)
\]
Unter zusätzlichen ``Regularitätsbedingungen'': $\mathring{f}(t)=f(t)$
\end{umkehrformelW}}

\end{frame}
