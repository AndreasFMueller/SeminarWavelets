%
% common.tex -- gemeinsame Definitionen
%
% (c) 2019 Prof Dr Andreas Müller, Hochschule Rapperswil
%
\usepackage[utf8]{inputenc}
\usepackage[T1]{fontenc}
\usepackage{epic}
\usepackage{color}
\usepackage{array}
\usepackage{ifthen}
\usepackage{amsmath}
\usepackage{lmodern}
\usepackage{tikz}
\usetikzlibrary{shapes.geometric}
\mode<beamer>{%
\usetheme[hideothersubsections,hidetitle]{Hannover}
}
\beamertemplatenavigationsymbolsempty
\title[Umkehrformel]{Umkehrformel für die CWT}
\author[A.~Müller]{Prof.~Dr.~Andreas Müller}
\date[]{11. März 2019}
\newboolean{presentation}

\theoremstyle{definition}
\newtheorem{umkehrformel}{Umkehrformel}
\newtheorem{cwt}{Stetige Wavelet-Transformation}
\newtheorem{vergleich}{Vergleichsprinzip}

\newtheorem{parseval}{Parseval-Formel für Fourier-Reihen}
\newtheorem{plancherel}{Plancherel-Formel für Fourier-Transformation}
\newtheorem{plancherelW}{Plancherel-Formel für $\mathcal{W}$}
\newtheorem{zulaessig}{Zulässigkeitsbedingung}
\newtheorem{umkehrformelW}{Umkehrformel für $\mathcal{W}$}
\newtheorem{aufgabe}{Aufgaben}

