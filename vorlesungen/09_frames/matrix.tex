%
% matrix.tex -- Matrixberechnung für den Gram-Operator
%
% (c) 2019 Prof Dr Andreas Müller, Hochschule Rapperswil
%
\begin{frame}
\frametitle{Gram-Operator in $\mathbb R^n$}
\begin{columns}[T,totalwidth=\textwidth]
\begin{column}{0.48\linewidth}
\begin{block}{Frame in Matrixform}
Frame
$\mathcal{B}=\{b_1,\dots,b_m\}$ in Matrixform:
\begin{align*}
\mathcal{B}
&=
\begin{pmatrix}
b_1&b_2&\dots&b_m
\end{pmatrix}
\\
&=
\begin{pmatrix}
b_{11}&\dots &b_{1m}\\
\vdots&\ddots&\vdots\\
b_{n1}&\dots &b_{nm}
\end{pmatrix}
\end{align*}
\end{block}
\end{column}

\begin{column}{0.48\linewidth}
\uncover<2->{%
\begin{block}{Gram-Operator}
\vspace{-15pt}
\begin{align*}
Gx
&\uncover<3->{=
\sum_{j=1}^m b_j\langle x,b_j\rangle}
\uncover<4->{=
\sum_{j=1}^m b_j \underbrace{b_{ij}}_{\displaystyle B^t} x_i}
\\
&\uncover<5->{=
BB^t x}
\end{align*}
\uncover<6->{$G$ symmetrisch $\Rightarrow$ diagonalisierbar:}
\vspace{-10pt}
\[
\uncover<7->{A=\lambda_1<\dots<\lambda_n=B}
\]
\vspace{-10pt}
\end{block}}
\end{column}
\end{columns}
\vspace{-15pt}
\uncover<8->{%
\begin{block}{Straffe Frames}
\[
\text{Frame $\mathcal{B}$ ist straff}
\uncover<9->{\qquad\Leftrightarrow\qquad
A=B}
\uncover<10->{\qquad\Leftrightarrow\qquad
G=A\cdot E}
\]
\end{block}}

\end{frame}
