%
% comparison.tex
%
% (c) 2019 Prof Dr Andreas Müller, Hochschule Rapperswil
%

\begin{frame}
\frametitle{Zwei Extreme: CWT und DWT}

\begin{block}{Funktionen}

\vspace{-20pt}

\begin{columns}[T,totalwidth=\linewidth]
\begin{column}{0.48\textwidth}
\uncover<2->{
\vspace{-10pt}
\[
D_bT_a \psi(t)
=
\psi_{a,b}(t)
=
\frac{1}{\sqrt{a}}
\psi\biggl(\frac{t-b}{a}\biggr)
\]}
\end{column}
\begin{column}{0.48\textwidth}
\vspace{-15pt}
\uncover<3->{
\begin{align*}
\psi_{j,k} &= 2^{j/2}\psi(2^jt-k)
\\
\varphi_{j,k} &= 2^{j/2}\varphi(2^jt-k)
\end{align*}}
\end{column}
\end{columns}
\end{block}

\vspace{-15pt}

\uncover<4->{
\begin{block}{Analyse}
\vspace{-15pt}

\begin{columns}[T,totalwidth=\linewidth]
\begin{column}{0.48\textwidth}
\vspace{-5pt}
\uncover<5->{
\[
\mathcal{W}f(a,b)
=
\frac{1}{\sqrt{a}}
\int_{\mathbb R} f(t)
\bar{\psi}\biggl(\frac{t-b}{a}\biggr)
\,dt
\]}
\end{column}
\begin{column}{0.48\textwidth}
\uncover<6->{
\vspace{-15pt}
\begin{align*}
b_{j,k}
&=
\langle f,\psi_{j,k}\rangle
=
2^{j/2}
\int_{\mathbb R} f(t)\overline{\psi(2^jt-k)}\,dt
\\
a_{j,k}
&=
\langle f,\varphi_{j,k}\rangle
\end{align*}}
\end{column}
\end{columns}
\end{block}}

\vspace{-10pt}

\uncover<7->{
\begin{block}{Umkehrformel}
\begin{columns}[T,totalwidth=\linewidth]
\begin{column}{0.48\textwidth}
\uncover<8->{
\[
f(t)
=
\frac{1}{C_\psi}
\int_{\mathbb R^*}^{\mathstrut}
\int_{\mathbb R}
\mathcal{W}f(a,b)\psi_{a,b}(t)
\,db\,\frac{da}{|a|^2}
\]}
\end{column}
\begin{column}{0.48\textwidth}
\uncover<9->{
\[
f(t)
=
\sum_{j=1}^{N}
\sum_{k\in\mathbb Z} b_{j,k} \psi_{j,k}(t)
+
\sum_{k\in\mathbb Z} a_{0,k} \varphi_{0,k}(t)
\]}
\end{column}
\end{columns}
\end{block}}

%\begin{columns}[T,totalwidth=\linewidth]
%\begin{column}{0.48\textwidth}
%\end{column}
%\begin{column}{0.48\textwidth}
%\end{column}
%\end{columns}


%Plancherel-Formel:
%\[
%\langle f,g\rangle
%=
%\frac{1}{C_\psi}
%\int_{\mathbb R}
%\int_{\mathbb R^*}
%\mathcal{W}f(a,b)
%\overline{\mathcal{W}g(a,b)}
%\,db
%\,\frac{da}{|a|^2}
%\]

\end{frame}
