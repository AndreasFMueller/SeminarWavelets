%
% common.tex -- gemeinsame Definitionen
%
% (c) 2019 Prof Dr Andreas Müller, Hochschule Rapperswil
%
\usepackage[utf8]{inputenc}
\usepackage[T1]{fontenc}
\usepackage{epic}
\usepackage{color}
\usepackage{array}
\usepackage{ifthen}
\usepackage{amsmath}
\usepackage{lmodern}
\usepackage{tikz}
\usetikzlibrary{shapes.geometric}
\mode<beamer>{%
\usetheme[hideothersubsections,hidetitle]{Hannover}
}
\beamertemplatenavigationsymbolsempty
\title[MSA]{Multiskalenanalyse}
\author[A.~Müller]{Prof.~Dr.~Andreas Müller}
\date[]{18.~März 2019}
\newboolean{presentation}

\theoremstyle{definition}
\newtheorem{beispiel}{Beispiel}
\newtheorem{folgerungen}{Folgerungen}
\newtheorem{frage}{Frage}
\newtheorem{raumW}{Orthogonalkomplement $W_j$}
\newtheorem{osumme}{Orthogonale Summe}
\newtheorem{haarfall}{Spezialfall Haar-Wavelet}

% linsys
\newcolumntype{\linsysR}{>{$}r<{$}}
\newcolumntype{\linsysL}{>{$}l<{$}}
\newcolumntype{\linsysC}{>{$}c<{$}}
\newenvironment{linsys}[1]{%
\begin{tabular}{*{#1}{\linsysR@{\;}\linsysC}@{\;}\linsysR}}%
{\end{tabular}}

\definecolor{darkgreen}{rgb}{0,0.6,0}
