%
% schnell.tex -- Schnelle Algorithmen
%
% (c) 2019 Prof Dr Andreas Müller, Hochschule Rapperswil
%
\begin{frame}
\frametitle{Schneller Algorithmus}
\begin{align*}
\varphi
&=
\frac1{\sqrt{2}}(D_{\frac12}\varphi + D_{\frac12}T_1\varphi)
&&\uncover<2->{\Rightarrow}&
\uncover<2->{\langle f,\varphi\rangle}
&\uncover<2->{=
\frac1{\sqrt{2}} \langle f,D_{\frac12}\varphi\rangle
+
\frac1{\sqrt{2}} \langle f,D_{\frac12}T_1\varphi\rangle}
\\
&&&\uncover<8->{\Rightarrow}&
\uncover<9->{a_{j,k}}&\uncover<9->{=\frac1{\sqrt{2}} (a_{j+1,2k} + a_{j+1,2k+1})}
\\
\uncover<3->{\psi}
&\uncover<3->{=
\frac1{\sqrt{2}}(D_{\frac12}\varphi - D_{\frac12}T_1\varphi)}
&&\uncover<4->{\Rightarrow}&
\uncover<4->{\langle f,\psi\rangle}
&\uncover<4->{=
\frac1{\sqrt{2}} \langle f,D_{\frac12}\varphi\rangle
-
\frac1{\sqrt{2}} \langle f,D_{\frac12}T_1\varphi\rangle}
\\
&&&\uncover<10->{\Rightarrow}&
\uncover<10->{b_{j,k}}&\uncover<10->{=\frac1{\sqrt{2}} (a_{j+1,2k} - a_{j+1,2k+1})}
\end{align*}
\uncover<5->{
Für Sampling-Interval $2^{-N}$:
\begin{itemize}
\item<6-> $a_{N,k}$ sind Sample-Werte
\item<7-> $a_{j,k}$ mit $j<N$ sind gemittelte Samples für Auflösung $2^{-j}$
\end{itemize}
}

\end{frame}

%
% allgemeiner schneller Algorithmus
%
\begin{frame}
\frametitle{Schneller Algorithmus (Verallgemeinerung)}
\begin{align*}
\varphi
&=
\frac1{\sqrt{2}}
\sum_{l\in\mathbb Z} h_l D_{\frac12}T_l\varphi
&&\uncover<3->{\Rightarrow}&
\uncover<3->{\langle f,\varphi\rangle}
&\uncover<3->{=
\frac{1}{\sqrt{2}} \sum_{k\in\mathbb Z} \bar{h}_k\langle f,D_{\frac12}T_k\varphi\rangle}
\\
&&&\uncover<7->{\Rightarrow}&
\uncover<7->{a_{j,k}}
&\uncover<7->{=
\frac{1}{\sqrt{2}} \sum_{l\in\mathbb Z}\bar{h}_l a_{j+1,2k+l}}
\\
\uncover<4->{\psi}
&\uncover<4->{=
\frac1{\sqrt{2}}
\sum_{l\in\mathbb Z} g_l D_{\frac12}T_l\varphi}
&&\uncover<6->{\Rightarrow}&
\uncover<6->{\langle f,\varphi\rangle}
&\uncover<6->{=
\frac{1}{\sqrt{2}} \sum_{l\in\mathbb Z} \bar{g}_l\langle f,D_{\frac12}T_l\varphi\rangle}
\\
&&&\uncover<8->{\Rightarrow}&
\uncover<8->{b_{j,k}}
&\uncover<8->{=
\frac{1}{\sqrt{2}} 
\sum_{l\in\mathbb Z} \bar{g}_l a_{j+1,2k+l}}
\end{align*}

\uncover<2->{
\begin{haarfall}
\[
\left.
\begin{aligned}
\varphi&=\text{Vater-Haar-Wavelet}\\
\uncover<5->{\psi}&\uncover<5->{=\text{Haar-Wavelet}}
\end{aligned}
\quad
\right\}
\qquad
\Rightarrow
\qquad
\left\{
\quad
\begin{aligned}
h_0&=1,&h_1&= 1\\
\uncover<5->{g_0}&\uncover<5->{=1,}&\uncover<5->{g_1}&\uncover<5->{=-1}
\end{aligned}
\right.
\]
\end{haarfall}
}
\end{frame}
