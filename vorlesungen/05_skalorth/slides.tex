%
% slides.tex
%
% (c) 2019 Prof Dr Andreas Müller, Hochschule Rapperswil
%

\ifthenelse{\boolean{presentation}}{
\begin{frame}
\titlepage
\end{frame}
}{}

\begin{frame}
\ifthenelse{\boolean{presentation}}{
\frametitle{Waveleteigenschaften $\rightarrow$ Frequenzbereich}
}{
\frametitle{Übersetzung der Waveleteigenschaften in den Frequenzbereich}
}
\begin{columns}[T]
\begin{column}{0.45\hsize}
\begin{block}{Periodisierungstrick}
\vspace{-10pt}
\[
\mathcal{P}f(\omega) = \sum_{k\in\mathbb Z} f(\omega + 2\pi k)
\]
\end{block}
\vspace{-10pt}
\begin{block}{Orthogonalität}
$T_b\varphi$ sind alle orthonormiert genau dann, wenn
\[
\mathcal{P}|\hat{\varphi}|^2(\omega)
=
\sum_{k\in\mathbb Z} |\hat{\varphi}(\omega+2\pi k)|^2 = \frac{1}{2\pi}
\]
\end{block}
\vspace{-20pt}
\begin{block}{Allgemein}
$T_b \psi\perp \varphi\forall b\in\mathbb Z$:
\[
\mathcal{P}\hat{\varphi}\bar{\hat{\psi}}(\omega)
=
\sum_{k\in\mathbb Z} \hat{\varphi}(\omega + 2\pi k)\bar{\hat{\psi}}(\omega+2\pi k) = 0
\]
\end{block}
\end{column}
\begin{column}{0.45\hsize}
\begin{block}{Erzeugende Funktion}
\vspace{-15pt}
\begin{gather*}
\varphi(t) = \sqrt{2}\sum_{k\in\mathbb Z} h_k\varphi(2t-k)\\
\Rightarrow
H(\omega) = \frac1{\sqrt{2}} \sum_{b\in\mathbb Z}h_be^{-ib\omega}
\end{gather*}
\end{block}
\vspace{-10pt}
\begin{block}{Skalierungsrelation für $\varphi$}
\vspace{-10pt}
\[
\hat{\varphi}(\omega)
=
H\biggl(\frac{\omega}2\biggr)
\hat{\varphi}\biggl(\frac{\omega}2\biggr)
\]
\end{block}
\begin{block}{Orthogonalität der $T_b\varphi$}
\[
|H(\omega)|^2
+
|H(\omega+\pi)|^2
=
1
\]
\end{block}
\end{column}
\end{columns}
\end{frame}

