%
% phipsi.tex
%
% (c) 2019 Prof Dr Andreas Müller, Hochschule Rapperswil
%
\begin{frame}
\frametitle{Orthogonalität von $\varphi$ und $\psi$}
\begin{block}{$\psi \perp T_b\varphi$}
\uncover<2->{
Periodisierungstrick anwenden:
\begin{align*}
0
=
\langle \psi, T_b\varphi\rangle
&\uncover<3->{=
\langle \hat\psi, \widehat{T_b\varphi}\rangle}
\uncover<4->{=
\int_{-\infty}^\infty
\hat{\psi}(\omega)
e^{ib\omega}
\overline{\hat{\varphi}(\omega)}
\,d\omega}
\uncover<5->{=
\int_{-\infty}^\infty
\hat{\psi}(\omega)
\overline{\hat{\varphi}(\omega)}
e^{ib\omega}
\,d\omega}
\\
&\uncover<6->{=
\int_0^{2\pi}
(P\hat\psi\overline{\hat{\varphi}})(\omega)
e^{ib\omega}
\,d\omega}
\end{align*}
}
\end{block}
\uncover<7->{
\begin{block}{Fourier-Bedingung für $\psi \perp T_b\varphi$}
Alle ganzzahlige Translate von $\varphi$ sind orthognal auf $\psi$ genau
dann, wenn
\[
\uncover<8->{
(\mathcal{P}\hat{\psi}\overline{\hat{\varphi}})(\omega)}
\uncover<9->{=
\sum_{k\in\mathbb Z}
\hat{\psi}(\omega+2\pi k)\overline{\hat{\varphi}(\omega+2\pi k)}}
\uncover<8->{=
0}
\]
\end{block}
}
\end{frame}
