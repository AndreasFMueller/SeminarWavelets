%
% periodisierung.tex
%
% (c) 2019 Prof Dr Andreas Müller, Hochschule Rapperswil
%

%
% Periodisierungstrick
%
\begin{frame}
\frametitle{Periodisierungstrick}
\begin{block}{Aufgabe}
Integrale der Form
$\int_{-\infty}^\infty f(t) e^{ibt}\,dt$ mit $b\in\mathbb Z$ vereinfachen
\end{block}
\uncover<2->{
\begin{block}{Idee}
Integral in Integrale über Intervalle der Länge $2\pi$ aufteilen:
\begin{align*}
\int_{-\infty}^\infty f(t) e^{ibt}\,dt
&\uncover<3->{=
\sum_{k\in\mathbb Z}
\int_{2\pi k}^{2\pi(k+1)} f(t) e^{ibt}\,dt}
\uncover<4->{=
\sum_{k\in\mathbb Z}
\int_0^{2\pi}
f(2\pi k+t) e^{ib(2\pi k+t)}\,dt}
\\
&\uncover<5->{=
\sum_{k\in\mathbb Z}
\int_0^{2\pi}
f(2\pi k+t)
e^{ibt}
\,dt}
\uncover<6->{=
\int_0^{2\pi}
e^{ibt}
\underbrace{
\sum_{k\in\mathbb Z}
f(2\pi k+t)
}_{\displaystyle=\mathcal{P}f(t)}
\,dt}
\end{align*}
\uncover<7->{$2\pi \times$Fourier-$b$-Koeffizient der $2\pi$-periodischen
Funktion
$\mathcal{P}f(t)$}
\end{block}
}
\end{frame}

%
% Anwendungen des Periodisierungstricks
%
\begin{frame}
\frametitle{Anwendungen des Periodisierungstricks}
In Orthonormalisierungsbedingungen treten vor allem Integralwerte $0$ oder $1$
auf:
\uncover<2->{
\begin{block}{$\int_{-\infty}^{\infty}f(t)e^{ibt}\,dt=0$ für alle $b\in\mathbb Z$}
\vspace{-20pt}
\uncover<3->{
\begin{align*}
\uncover<5->{
2\pi\cdot c_b
=
}
\int_{-\infty}^\infty f(t) e^{ibt}\,dt
&\uncover<4->{=
\int_{0}^{2\pi} \mathcal Pf(t) e^{ibt} \,dt
=
0}
\end{align*}
}
\uncover<6->{%
Alle Fourier-Koeffizienten von $\mathcal{P}f$ verschwinden:}
\uncover<7->{$\Rightarrow\; \mathcal{P}f(t)=0$.}
\end{block}
}

\uncover<8->{
\begin{block}{$\int_{-\infty}^{\infty}f(t)e^{ibt}\,dt=\delta_{b0}$ für $b\in\mathbb Z$}
\vspace{-20pt}
\uncover<9->{
\begin{align*}
\uncover<10->{2\pi\cdot c_b
=}
\int_{-\infty}^\infty f(t) e^{ibt}\,dt
&=
\int_{0}^{2\pi} \mathcal Pf(t) e^{ibt} \,dt
=
\delta_{b0}
\end{align*}
}
\uncover<11->{
Fourier-Reihe von $\mathcal{P}f$ summieren:}
\uncover<12->{$\mathcal Pf(t) = c_0 = \displaystyle\frac1{2\pi}$}
\end{block}
}

\end{frame}
