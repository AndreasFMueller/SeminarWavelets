%
% H.tex -- Skalierungsrelation
%
% (c) 2019 Prof Dr Andreas Müller, Hochschule Rapperswil
%

%
% Erzeugende Funktion
%
\begin{frame}
\frametitle{Erzeugende Funktion einer MSA}
\begin{block}{Skalierungsrelation für $\varphi$}
\vspace{-20pt}
\begin{align*}
\varphi(t)
&=
\sqrt{2}
\sum_{k\in\mathbb Z} h_k\varphi(2t-k)
\\[-10pt]
\intertext{\vspace{-10pt} \uncover<2->{Fourier-Transformierte:}}
\uncover<3->{
\hat{\varphi}(\omega)}
&\uncover<3->{=
\sqrt{2}
\sum_{k\in\mathbb Z} h_k e^{-ik\omega/2}
\frac12\hat{\varphi}\biggl(\frac{\omega}2\biggr)}
\uncover<4->{=
\biggl(
\frac1{\sqrt{2}}
\sum_{k\in\mathbb Z} h_k e^{-ik\omega/2}
\biggr)
\hat{\varphi}\biggl(\frac{\omega}2\biggr)}
\end{align*}
\end{block}
\vspace{-10pt}

\begin{columns}[T,totalwidth=\textwidth]
\begin{column}{0.62\linewidth}
\uncover<5->{
\begin{block}{Erzeugende Funktion}
Die $2\pi$-periodische Funktion
\[
H(s)
=
\frac1{\sqrt{2}}
\sum_{k\in\mathbb Z} h_k e^{-iks}
\]
heisst {\em erzeugende Funktion} der Multiskalenanalyse
\end{block}
}
\end{column}
\begin{column}{0.35\linewidth}
\uncover<6->{
\begin{block}{Funktionalgleichung für $\varphi$}
Für fast alle $\omega$ gilt
\[
\hat{\varphi}(\omega)
=
H\biggl(\frac{\omega}{2}\biggr)
\hat{\varphi}\biggl(\frac{\omega}2\biggr)
\]
\end{block}
}
\end{column}
\end{columns}
\end{frame}

%
% Orthonormalitätsbedingung für H
%
\begin{frame}
\frametitle{Orthogonalitätsbedingung für $H(s)$}
\begin{block}{Orthogonalität für $\varphi$ auf Funktionalgleichung anwenden}
\vspace{-20pt}

\begin{align*}
\frac{1}{2\pi}
&=
\sum_{k\in\mathbb Z}
|\hat{\varphi}(\omega+2\pi k)|^2
=
\sum_{k\in\mathbb Z}
\biggl|
H\biggl(\frac{\omega + 2\pi k}2\biggr)
\hat{\varphi}\biggl(\frac{\omega + 2\pi k}2\biggr)
\biggr|^2
\\
&\uncover<2->{=
\sum_{l\in\mathbb Z}
\biggl|
H\biggl(\frac{\omega + 4\pi l}2\biggr)
\hat{\varphi}\biggl(\frac{\omega + 4\pi l}2\biggr)
\biggr|^2
+
\sum_{l\in\mathbb Z}
\biggl|
H\biggl(\frac{\omega + 2\pi(2l+1)}2\biggr)
\hat{\varphi}\biggl(\frac{\omega + 2\pi(2l+1)}2\biggr)
\biggr|^2}
\\
&
\ifthenelse{\boolean{presentation}}{
\only<3>{=
\sum_{l\in\mathbb Z}
\biggl|
H\biggl(\frac{\omega}2\biggr)
\biggr|^2
\biggl|
\hat{\varphi}\biggl(\frac{\omega}2 + 2\pi l\biggr)
\biggr|^2
+
\sum_{l\in\mathbb Z}
\biggl|
H\biggl(\frac{\omega}2+\pi\biggr)
\biggr|^2
\biggl|
\hat{\varphi}\biggl(\frac{\omega}2 + \pi+ 2\pi l\biggr)
\biggr|^2
}
\only<4>{=
\biggl|
H\biggl(\frac{\omega}2\biggr)
\biggr|^2
\sum_{l\in\mathbb Z}
\biggl|
\hat{\varphi}\biggl(\frac{\omega}2 + 2\pi l\biggr)
\biggr|^2
+
\biggl|
H\biggl(\frac{\omega}2+\pi\biggr)
\biggr|^2
\sum_{l\in\mathbb Z}
\biggl|
\hat{\varphi}\biggl(\frac{\omega}2 + \pi+ 2\pi l\biggr)
\biggr|^2
}
\only<5>{=
\biggl|
H\biggl(\frac{\omega}2\biggr)
\biggr|^2
\mathcal{P}|\hat{\varphi}|^2
\biggl(\frac{\omega}2\biggr)
+
\biggl|
H\biggl(\frac{\omega}2+\pi\biggr)
\biggr|^2
\mathcal{P}|\hat{\varphi}|^2
\biggl(\frac{\omega}2 + \pi\biggr)
}
}{=\dots}
\only<6->{=
\biggl|
H\biggl(\frac{\omega}2\biggr)
\biggr|^2
\cdot
\frac{1}{2\pi}
+
\biggl|
H\biggl(\frac{\omega}2+\pi\biggr)
\biggr|^2
\cdot
\frac{1}{2\pi}
}
\end{align*}
\vspace{-15pt}
\end{block}
\uncover<7->{
\begin{block}{Orthogonalitätsbedingung für $H(s)$}
Die ganzzahligen Translate von $\varphi$ sind genau dann orthonormiert, wenn
\[
|H(\omega)|^2 + |H(\omega+\pi)|^2 = 1
\]
für fast alle $\omega$.
\end{block}
}

\end{frame}
