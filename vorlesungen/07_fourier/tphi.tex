%
% tphi.tex -- translsate von phi sind orthogonal
%
% (c) 2019 Prof Dr Andreas Müller, Hochschule Rapperswil
%

\begin{frame}
\frametitle{Translate von $\varphi$}
\begin{block}{Translate von $\varphi$ sind orthonormiert}
\[
\langle \varphi,T_b\varphi\rangle = \delta_{0b}
\]
\end{block}
\uncover<2->{
Periodisierungstrick anwenden:
\begin{align*}
\delta_{0b}
=
\langle \varphi,T_b\varphi\rangle
&\uncover<3->{=
\langle \hat{\varphi}, \widehat{T_b\varphi}\rangle}
\uncover<4->{=
\int_{-\infty}^\infty
\hat{\varphi}(\omega) e^{i\omega b}
\overline{\hat{\varphi}(\omega)}
\,d\omega}
\uncover<5->{=
\int_{-\infty}^{\infty}
|\hat{\varphi}(\omega)|^2 e^{ib\omega}
\,d\omega}
\\
&\uncover<6->{= \int_0^{2\pi}
\mathcal{P}
|\hat{\varphi}(\omega)|^2
e^{ib\omega}
\,d\omega}
\end{align*}
}

\uncover<7->{
\begin{block}{Fourier-Bedingung für orthonormierte Translate}
Ganzzahlige Translate von $\varphi$ sind orthonormiert genau dann wenn
\[
\mathcal{P}|\hat{\varphi}|^2 (\omega)
\uncover<9->{=
\sum_{k\in\mathbb Z} |\hat{\varphi}(\omega+2\pi k)|^2
}
\uncover<8->{=
\frac1{2\pi}
}
\]
\end{block}
}
\end{frame}
