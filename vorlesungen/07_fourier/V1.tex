%
% V1.tex
%
% (c) 2019 Prof Dr Andreas Müller, Hochschule Rapperswil
%

\begin{frame}
\frametitle{Darstellung von Funktionen in $V_1$}

\begin{block}{$f\in V_1$ lässt sich aus $\varphi_{1,b}$ linear kombinieren}
\vspace{-20pt}

\begin{align*}
f(t)
&=
\sum_{b\in\mathbb Z} f_k \varphi_{1,b}(t)
\\[-15pt]
\intertext{\vspace{-10pt} Fourier-Transformierte:}
\hat{f}(\omega)
&=
\sum_{b\in\mathbb Z} f_b\frac{1}{\sqrt{2}} e^{ib\omega/2}
\hat{\varphi}\biggl(\frac{\omega}2\biggr)
\end{align*}
\end{block}

\vspace{-15pt}

\begin{definition}
\vspace{-10pt}
\[
m_f(s)
=
\frac{1}{\sqrt{2}}
\sum_{b\in\mathbb Z} f_k e^{-ibs}
\]
\end{definition}

\vspace{-15pt}

\begin{block}{Allgemeine Skalierungsrelation}
\vspace{-10pt}
\[
\hat{f}(\omega)
=
m_f\biggl(\frac{\omega}2\biggr)
\hat{\varphi}\biggl(\frac{\omega}2\biggr)
\]
Spezialfall: $H=m_\varphi$, gleiche Rechnung
\end{block}

\end{frame}

%
%
%
\begin{frame}
\frametitle{Orthogonalitätsrelationen}
\begin{align*}
\{T_b\varphi\,|\,b\in\mathbb Z\}&\;\text{orthonormiert}
&&\Rightarrow&
1&=|H(\omega+\pi)|^2 + |H(\omega)|^2
\\
\{T_b\psi\,|\,b\in\mathbb Z\}&\;\text{orthonormiert}
&&\Rightarrow&
1&=|m_\psi(\omega+\pi)|^2 + |m_\psi(\omega)|^2
\\
\psi&\perp T_b\varphi
\;\forall b\in\mathbb Z
&&\Rightarrow&
0&=m_\psi(\omega+\pi) \overline{H(\omega+\pi)} + m_\psi(\omega)\overline{H(\omega)}
\end{align*}
Vektoriell:
\begin{align*}
\left|
\begin{pmatrix}
H(\omega)\\
H(\omega+\pi)
\end{pmatrix}
\right|^2
=
\left|
\begin{pmatrix}
m_\psi(\omega)\\
m_\psi(\omega+\pi)
\end{pmatrix}
\right|^2
&=
1
&&\text{und}
&
\begin{pmatrix}
m_\psi(\omega)\\
m_\psi(\omega+\pi)
\end{pmatrix}
\cdot
\begin{pmatrix}
H(\omega)\\
H(\omega+\pi)
\end{pmatrix}
&=
0
\end{align*}
\[
\begin{pmatrix}
m_\psi(\omega)\\
m_\psi(\omega+\pi)
\end{pmatrix}
=
\lambda(\omega)
\begin{pmatrix}
\overline{H(\omega+\pi)}\\
-\overline{H(\omega)}
\end{pmatrix},
\quad \lambda(\omega) \in\mathbb C, |\lambda(\omega)|=1,
\lambda(\omega)=-\lambda(\omega+\pi)
\]
\end{frame}


\begin{frame}
\frametitle{$\psi$ bestimmt durch $H$}
\[
\hat{\psi}(\omega)
=
m_\psi\biggl(\frac{\omega}2\biggr)
\hat{\varphi}\biggl(\frac{\omega}2\biggr)
=
\lambda\biggl(\frac{\omega}2\biggr)
\overline{H\biggl(\frac{\omega}2+\pi\biggr)}
\hat{\varphi}\biggl(\frac{\omega}2\biggr)
=
e^{i\omega/2}
p\biggl(\frac{\omega}2\biggr)
\overline{H\biggl(\frac{\omega}2+\pi\biggr)}
\hat{\varphi}\biggl(\frac{\omega}2\biggr)
\]


\end{frame}



