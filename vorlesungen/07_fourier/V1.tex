%
% V1.tex
%
% (c) 2019 Prof Dr Andreas Müller, Hochschule Rapperswil
%

\begin{frame}
\frametitle{Darstellung von Funktionen in $V_1$}

\begin{block}{$f\in V_1$ lässt sich aus $\varphi_{1,b}$ linear kombinieren}
\vspace{-20pt}

\begin{align*}
f(t)
&=
\sqrt{2}
\sum_{b\in\mathbb Z} f_k \varphi_{1,b}(t)
\\[-15pt]
\intertext{\vspace{-10pt}%
\uncover<2->{Fourier-Transformierte:}}
\uncover<3->{\hat{f}(\omega)}
&\uncover<3->{=
\sum_{b\in\mathbb Z} f_b\frac{1}{\sqrt{2}} e^{ib\omega/2}
\hat{\varphi}\biggl(\frac{\omega}2\biggr)}
\end{align*}
\end{block}

\vspace{-25pt}

\uncover<4->{
\begin{definition}
\vspace{-15pt}
\begin{align*}
m_f(s)
&=
\frac{1}{\sqrt{2}}
\sum_{b\in\mathbb Z} f_k e^{-ibs}
&
\uncover<5->{m_\psi(s)}
&\uncover<5->{=
\frac{1}{\sqrt{2}}
\sum_{b\in\mathbb Z} g_k e^{-ibs}}
\end{align*}
\end{definition}
}

\vspace{-15pt}

\uncover<6->{
\begin{block}{Allgemeine Skalierungsrelation}
\vspace{-15pt}
\begin{align*}
\hat{f}(\omega)
&=
m_f\biggl(\frac{\omega}2\biggr)
\hat{\varphi}\biggl(\frac{\omega}2\biggr)
&
\uncover<7->{\hat{\psi}(\omega)}
&\uncover<7->{=
m_\psi\biggl(\frac{\omega}2\biggr)
\hat{\varphi}\biggl(\frac{\omega}2\biggr)}
\end{align*}
Spezialfall: $H=m_\varphi$, gleiche Rechnung
\end{block}
}

\end{frame}

%
%
%
\begin{frame}
\frametitle{Orthogonalitätsrelationen}
\begin{align*}
\{T_b\varphi\,|\,b\in\mathbb Z\}&\;\text{orthonormiert}
&&\Rightarrow&
1&=|H(\omega+\pi)|^2 + |H(\omega)|^2
\\
\{T_b\psi\,|\,b\in\mathbb Z\}&\;\text{orthonormiert}
&&\Rightarrow&
1&=|m_\psi(\omega+\pi)|^2 + |m_\psi(\omega)|^2
\\
\psi&\perp T_b\varphi
\;\forall b\in\mathbb Z
&&\Rightarrow&
0&=m_\psi(\omega+\pi) \overline{H(\omega+\pi)} + m_\psi(\omega)\overline{H(\omega)}
\end{align*}
\uncover<2->{
Vektoriell:
\begin{align*}
\left|
\begin{pmatrix}
H(\omega)\\
H(\omega+\pi)
\end{pmatrix}
\right|^2
=
\left|
\begin{pmatrix}
m_\psi(\omega)\\
m_\psi(\omega+\pi)
\end{pmatrix}
\right|^2
&=
1
&&\text{und}
&
\begin{pmatrix}
m_\psi(\omega)\\
m_\psi(\omega+\pi)
\end{pmatrix}
\cdot
\begin{pmatrix}
H(\omega)\\
H(\omega+\pi)
\end{pmatrix}
&=
0
\end{align*}
}
\uncover<3->{
\[
\begin{pmatrix}
m_\psi(\omega)\\
m_\psi(\omega+\pi)
\end{pmatrix}
=
\lambda(\omega)
\begin{pmatrix}
\overline{H(\omega+\pi)}\\
-\overline{H(\omega)}
\end{pmatrix},
\quad \lambda(\omega) \in\mathbb C, |\lambda(\omega)|=1,
\lambda(\omega)=-\lambda(\omega+\pi)
\]
}
\end{frame}

%
% psi bestimmt durch H
%
\begin{frame}
\frametitle{$\psi$ bestimmt durch $H$}
\[
\hat{\psi}(\omega)
=
m_\psi\biggl(\frac{\omega}2\biggr)
\hat{\varphi}\biggl(\frac{\omega}2\biggr)
\uncover<2->{=
\lambda\biggl(\frac{\omega}2\biggr)
\overline{H\biggl(\frac{\omega}2+\pi\biggr)}
\hat{\varphi}\biggl(\frac{\omega}2\biggr)
}
\uncover<3->{=
e^{i\omega/2}
p\biggl(\frac{\omega}2\biggr)
\overline{H\biggl(\frac{\omega}2+\pi\biggr)}
\hat{\varphi}\biggl(\frac{\omega}2\biggr)
}
\]
\uncover<4->{%
D.~h.~es gibt verschiedene Lösungen für $m_{\psi}$. 
}
\uncover<5->{%
Naheliegend: $p(\omega)=1$:
}
\begin{align*}
\uncover<6->{m_\psi\biggl(\frac{\omega}2\biggr)}
&\uncover<6->{=
e^{i\omega/2}
\overline{H\biggl(\frac{\omega}2+\pi\biggr)}
}
\ifthenelse{\boolean{presentation}}{
\only<7>{=
\frac1{\sqrt{2}}
\sum_{k\in\mathbb Z}
\bar{h}_k e^{ik(\frac{\omega}2 + \pi)}
e^{i\omega/2}
}
\only<8>{=
\frac{1}{\sqrt{2}}
\sum_{k\in\mathbb Z}
\bar{h}_k (-1)^k e^{i(k+1)\omega/2}
}
\only<9>{=
\frac{1}{\sqrt{2}}
\sum_{k\in\mathbb Z}
\bar{h}_k (-1)^k e^{-i(-k-1)\omega/2}
\qquad \text{setze } -k'=k+1
}
}{}
\only<10->{=
\frac{1}{\sqrt{2}}
\sum_{k'\in\mathbb Z}
\bar{h}_{-k'-1}
(-1)^{k'+1}
e^{-ik'\omega/2}
\quad\text{wegen $k=-k'-1$}
}
\\
&\uncover<11->{\mathstrut=
\phantom{e^{i\omega/2}
\overline{H\biggl(\frac{\omega}2+\pi\biggr)}=\mathstrut}
\frac{1}{\sqrt{2}} \sum_{k'\in\mathbb Z} g_{k'}e^{-ik'\omega/2}
}
\end{align*}
\uncover<12->{%
$\Rightarrow$
Darstellung von $\psi$ in der Basis $\varphi_{1,b}$ ist durch $h_k$
vollständig bestimmt%
}%
\uncover<13->{:
\[
g_k = (-1)^{k+1}\bar{h}_{-k-1}
\]
}
\end{frame}



