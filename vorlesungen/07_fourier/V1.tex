%
% V1.tex
%
% (c) 2019 Prof Dr Andreas Müller, Hochschule Rapperswil
%

\begin{frame}
\frametitle{Darstellung von Funktionen in $V_1$}

\begin{block}{$f\in V_1$ lässt sich aus $\varphi_{1,b}$ linear kombinieren}
\vspace{-20pt}

\begin{align*}
f(t)
&=
\sum_{b\in\mathbb Z} f_k \varphi_{1,b}(t)
\\[-15pt]
\intertext{\vspace{-10pt} Fourier-Transformierte:}
\hat{f}(\omega)
&=
\sum_{b\in\mathbb Z} f_b\frac{1}{\sqrt{2}} e^{ib\omega/2}
\hat{\varphi}\biggl(\frac{\omega}2\biggr)
\end{align*}
\end{block}

\vspace{-15pt}

\begin{definition}
\vspace{-10pt}
\[
m_f(s)
=
\frac{1}{\sqrt{2}}
\sum_{b\in\mathbb Z} f_k e^{-ibs}
\]
\end{definition}

\vspace{-15pt}

\begin{block}{Allgemeine Skalierungsrelation}
\vspace{-10pt}
\[
\hat{f}(\omega)
=
m_f\biggl(\frac{\omega}2\biggr)
\hat{\varphi}\biggl(\frac{\omega}2\biggr)
\]
Spezialfall: $H=m_\varphi$, gleiche Rechnung
\end{block}

\end{frame}
