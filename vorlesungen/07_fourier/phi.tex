%
% phi.tex
%
% (c) 2019 Prof Dr Andreas Müller, Hochschule Rapperswil
%
\begin{frame}
\frametitle{$\varphi$ bestimmt durch $H$}
Ausgangspunkt:
\[
\hat{\varphi}(\omega)
=
H\biggl(\frac{\omega}2\biggr)
\hat{\varphi}\biggl(\frac{\omega}2\biggr)
\]
Iteration:
\[
\hat{\varphi}(\omega)
=
\only<2->{H\biggl(\frac{\omega}2\biggr)}
\only<3->{H\biggl(\frac{\omega}4\biggr)}
\only<4->{H\biggl(\frac{\omega}8\biggr)}
\only<5->{H\biggl(\frac{\omega}{16}\biggr)}
\only<6->{H\biggl(\frac{\omega}{32}\biggr)}
\ifthenelse{\boolean{presentation}}{
\only<2>{\hat{\varphi}\biggl(\frac{\omega}2\biggr)}
\only<3>{\hat{\varphi}\biggl(\frac{\omega}4\biggr)}
\only<4>{\hat{\varphi}\biggl(\frac{\omega}8\biggr)}
\only<5>{\hat{\varphi}\biggl(\frac{\omega}{16}\biggr)}
\only<6>{\hat{\varphi}\biggl(\frac{\omega}{32}\biggr)}
}{}
\only<7->{\dots}
\]
\uncover<7->{Grenzwert }
\uncover<9->{existiert für stetiges $\hat{\varphi}$ mit
$\hat{\varphi}=1$}
\[
\uncover<7->{\hat{\varphi}(\omega)=}
\uncover<8->{
\prod_{k=1}^\infty
H\biggl(\frac{\omega}{2^k}\biggr)
}
\ifthenelse{\boolean{presentation}}{
\uncover<8-9>{\cdot \lim_{k\to\infty}\hat{\varphi}\biggl(\frac{\omega}{2^k}\biggr)}
}{}
\]
\uncover<10->{$\Rightarrow$ $\hat{\varphi}$ vollständig durch $H$ bestimmt.}
\end{frame}
